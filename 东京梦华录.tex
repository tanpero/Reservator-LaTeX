\PassOptionsToPackage{unicode=true}{hyperref} % options for packages loaded elsewhere
\PassOptionsToPackage{hyphens}{url}
%
\documentclass[]{article}
\usepackage{lmodern}
\usepackage{amssymb,amsmath}
\usepackage{ifxetex,ifluatex}
\usepackage{fixltx2e} % provides \textsubscript
\ifnum 0\ifxetex 1\fi\ifluatex 1\fi=0 % if pdftex
  \usepackage[T1]{fontenc}
  \usepackage[utf8]{inputenc}
  \usepackage{textcomp} % provides euro and other symbols
\else % if luatex or xelatex
  \usepackage{unicode-math}
  \defaultfontfeatures{Ligatures=TeX,Scale=MatchLowercase}
\fi
% use upquote if available, for straight quotes in verbatim environments
\IfFileExists{upquote.sty}{\usepackage{upquote}}{}
% use microtype if available
\IfFileExists{microtype.sty}{%
\usepackage[]{microtype}
\UseMicrotypeSet[protrusion]{basicmath} % disable protrusion for tt fonts
}{}
\IfFileExists{parskip.sty}{%
\usepackage{parskip}
}{% else
\setlength{\parindent}{0pt}
\setlength{\parskip}{6pt plus 2pt minus 1pt}
}
\usepackage{hyperref}
\hypersetup{
            pdfborder={0 0 0},
            breaklinks=true}
\urlstyle{same}  % don't use monospace font for urls
\setlength{\emergencystretch}{3em}  % prevent overfull lines
\providecommand{\tightlist}{%
  \setlength{\itemsep}{0pt}\setlength{\parskip}{0pt}}
\setcounter{secnumdepth}{0}
% Redefines (sub)paragraphs to behave more like sections
\ifx\paragraph\undefined\else
\let\oldparagraph\paragraph
\renewcommand{\paragraph}[1]{\oldparagraph{#1}\mbox{}}
\fi
\ifx\subparagraph\undefined\else
\let\oldsubparagraph\subparagraph
\renewcommand{\subparagraph}[1]{\oldsubparagraph{#1}\mbox{}}
\fi

% set default figure placement to htbp
\makeatletter
\def\fps@figure{htbp}
\makeatother


\date{}

\begin{document}

\hypertarget{header-n0}{%
\section{东京梦华录}\label{header-n0}}

\begin{center}\rule{0.5\linewidth}{\linethickness}\end{center}

\tableofcontents

\begin{center}\rule{0.5\linewidth}{\linethickness}\end{center}

\hypertarget{header-n7}{%
\subsection{序}\label{header-n7}}

仆从先人宦游南北,崇宁癸未到京师,卜居于州西金梁桥西夹道之南。渐次长立,正当辇毂之下。太平日久,人物繁阜。垂髫之童,但习鼓舞;班白之老,不识干戈。时节相次,各有观赏。灯宵月夕,雪际花时,乞巧登高,教池游苑。举目则青楼画阁,绣户珠帘。雕车竞驻于天街,宝马争驰于御路,金翠耀目,罗绮飘香。新声巧笑于柳陌花衢,按管调弦于茶坊酒肆。八荒争凑,万国咸通。集四海之珍奇,皆归市易;会寰区之异味,悉在庖厨。花光满路,何限春游;箫鼓喧空,几家夜宴。伎巧则惊人耳目,侈奢则长人精神。瞻天表则元夕教池,拜郊孟享。频观公主下降,皇子纳妃。修造则创建明堂,冶铸则立成鼎鼐。仆数十年烂赏叠游,莫知厌足。\\
一旦兵火,靖康丙午之明年,出京南来,避地江左,情绪牢落,渐入桑榆。暗想当年,节物风流,人情和美,但成怅恨。近与亲戚会面,谈及曩昔,后生往往妄生不然。仆恐浸久,论其风俗者,失于事实,诚为可惜。谨省记编次成集,庶几开卷得睹当时之盛。古人有梦游华胥之国,其乐无涯者,仆今追念,回首怅然,岂非华胥之梦觉哉?目之曰《梦华录》。\\
然以京师之浩穰,及有未尝经从处,得之于人,不无遗阙。倘遇乡党宿德,补缀周备,不胜幸甚。此录语言鄙俚,不以文饰者,盖欲上下通晓尔,观者幸详焉。绍兴丁卯岁除日,幽兰居士孟元老序。

\hypertarget{header-n10}{%
\subsection{卷一}\label{header-n10}}

东都外城

东都外城.方圆四十余里.城壕曰护龙河.阔十余丈.濠之内外.皆植杨柳.粉墙朱戸.禁人往来.城门皆瓮城三层.屈曲开门.唯南薫门.新郑门.新宋门.封丘门皆直门两重.盖此系四正门.皆留御路故也.新城南壁.其门有三.正南门曰南薫门.城南一边.东南则陈州门.傍有蔡河水门.西南则戴楼门.傍亦有蔡河水门.蔡河正名惠民河.为通蔡州故也.东城一边.其门有四.东南曰东水门.乃汴河下流水门也.其门跨河.有铁裹窗门.遇夜如闸垂下水面.两岸各有门通人行路.出拐子城.夹岸百余丈.次则曰新宋门.次曰新曹门.又次曰东北水门.乃五丈河之水门也.西城一边.其门有四.从南曰新郑门.次曰西水门.汴河上水门也.次曰万胜门.又次曰固子门.又次曰西北水门.乃金水河水门也.北城一边.其门有四.从东曰陈桥门.〔乃大辽人使驿路.〕次曰封丘门.〔北郊御路.〕次曰新酸枣门.次曰卫州门.〔诸门名皆俗呼.其正名如西水门曰利泽.郑门本顺天门.固子门本金耀门.〕新城毎百歩设马面.战棚.密置女头.旦暮修整.望之耸然.城里牙道.各植楡柳成阴.毎二百歩置一防城库.贮守御之器.有广固兵士二十指挥.毎日修造泥饰.专有京城所提总其事.

旧京城

旧京城方圆约二十里许.南壁其门有三.正南曰朱雀门.左曰保康门.右曰新门.东壁其门有三.从南汴河南岸角门子.河北岸曰旧宋门.次曰旧曹门.西壁其门有三.从南曰旧郑门.次汴河北岸角门子.次曰梁门.北壁其门有三.从东曰旧封丘门.次曰景龙门.〔乃大内城角宝箓宫前也.〕次曰金水门.

河道

穿城河道有四.南壁曰蔡河.自陈蔡由西南戴楼门入京城.辽绕自东南陈州门出.河上有桥十一.自陈州门里曰观桥.〔在五岳观后门.〕从北.次曰宣泰桥.次曰云骑桥.次曰横桥子.〔在彭婆婆宅前.〕次曰高桥.次曰西保康门桥.次曰龙津桥.〔正对内前.〕次曰新桥.次曰太平桥.〔高殿前宅前.〕次曰粜麦桥.次曰第一座桥.次曰宜男桥.出戴楼门外曰四里桥.中曰汴河.自西京洛口分水入京城.东去至泗州.入淮.运东南之粮.凡东南方物.自此入京城.公私仰给焉.自东水门外七里至西水门外.河上有桥十三.从东水门外七里曰虹桥.其桥无柱.皆以巨木虚架.饰以丹艧.宛如飞虹.其上下土桥亦如之.次曰顺成仓桥.入水门里曰便桥.次曰下土桥.次曰上土桥.投西角子门曰相国寺桥.次曰州桥.〔正名天汉桥.〕正对于大内御街.其桥与相国寺桥皆低平不通舟船.唯西河平船可过.其柱皆青石为之.石梁石笋楯栏.近桥两岸.皆石壁.雕镌海马水兽飞云之状.桥下密排石柱.盖车驾御路也.州桥之北岸御路.东西两阙.楼观对耸.桥之西有方浅船二只.头置巨干铁鎗数条.岸上有铁索三条.遇夜绞上水面.盖防遗火舟船矣.西去曰浚仪桥.次曰兴国寺桥.〔亦名马军衙桥.〕次曰太师府桥.〔蔡相宅前.〕次曰金梁桥.次曰西浮桥.〔旧以船为之桥.今皆用木石造矣.〕次曰西水门便桥.门外曰横桥.东北曰五丈河.来自济郓.般挽京东路粮斛入京城.自新曹门北入京.河上有桥五.东去曰小横桥.次曰广备桥.次曰蔡市桥.次曰青晖桥.染院桥.西北曰金水河.自京城西南分京.索河水筑堤.从汴河上用木槽架过.从西北水门入京城.夹墙遮拥.入大内灌后苑池蒲矣.河上有桥三.曰白虎桥.横桥.五王宫桥之类.又曹门小河子桥曰念佛桥.盖内诸司辇官亲事官之类.军营皆在曹门.侵晨上直.有瞽者在桥上念经求化.得其名矣.

大内

大内正门宣徳楼列五门.门皆金钉朱漆.壁皆砖石间甃.镌镂龙凤飞云之状.莫非雕甍画楝.峻桷层榱.覆以琉璃瓦.曲尺朶楼.朱栏彩槛.下列两阙亭相对.悉用朱红杈子.入宣徳楼正门.乃大庆殿.庭设两楼.如寺院钟楼.上有太史局保章正.测验刻漏.逐时刻执牙牌奏.毎遇大礼车驾斋宿及正朔朝会于此殿.殿外左右横门曰左右长庆门.内城南壁有门三座.孙大朝会趋朝路.宣徳楼左曰左掖门.右曰右掖门.左掖门里乃明堂.右掖门里西去乃天章.宝文等阁.宫城至北廊的百余丈.入门东去街北廊乃枢密院.次中书省.次都堂.〔宰相朝退治事于此.〕次门下省.次大庆殿.外廊横门北去百余歩.又一横门.毎日宰执趋朝.此处下马.饯侍从台諌于第一横门下马.行至文徳殿.入第二横门.东廊大庆殿东偏门.西廊中书.门下后省.次修国史院.次南向小角门.正对文徳殿.〔常朝殿也.〕殿前东西大街.东出东华门.西出西华门.近里又两门相对.左右嘉肃门也.南去左右银台门.自东华门里皇太子宫入嘉肃门.街南大庆殿后门.东西上合门.街北宣佑门.南北大街西廊.面东曰凝晖殿.乃通会通门.入禁中矣.殿相对东廊门楼.乃殿中省六尚局御厨.殿上常列禁卫两重.时刻提警.出入甚严.近里皆近侍中贵.殿之外皆知省.御药幕次.快行.亲从官.辇官.车子院.黄院子.内诸司兵士.祗候宣唤.及官禁买卖进贡.皆由此入.唯此浩穰.诸司人自卖饮食珍奇之物.市井之间未有也.毎遇早晩进膳.自殿中省对凝晖殿.禁卫成列.约栏不得过往.省门上有一人呼喝.谓之『拨食家』.次有紫衣.裹脚子向后曲折幞头者.谓之『院子家』.托一合.用黄繍龙合衣笼罩.左手携一红罗繍手巾.进入于此.约十余合.继托金瓜合二十余面进入.非时取唤.谓之『泛索』.宣佑门外.西去紫宸殿.〔正朔受朝于此.〕次曰文徳殿.〔常朝所御.〕次曰垂拱殿.次曰皇仪殿.次曰集英殿.〔御宴及试举人于此.〕后殿曰崇政殿.保和殿.内书阁曰睿思殿.后门曰拱辰门.东华门外.市井最盛.盖禁中买卖在此.凡饮食.时新花果.鱼虾鳖蟹.鹑兔脯腊.金玉珍玩衣着.无非天下之奇.其品味若数十分.客要一二十味下酒.随索目下便有之.其歳时果瓜蔬茹新上市.并茄瓠之类新出.毎对可直三五十千.诸合分争以贵价取之.

内诸司

内诸司皆在禁中.如学士院.皇城司.四方馆.客省.东西上合门.通进司.内弓剑鎗甲军器等库.翰林司.〔茶酒局也.〕内侍省.入内内侍省.内藏库.奉宸库.景福殿库.延福宫.殿中省六尚局.〔尚药.尚食.尚辇.尚酝.尚舎.尚衣.〕诸合分.内香药库.后苑作.翰林书艺局.医官局.天章等阁.明堂颁朔布政府.

外诸司

外诸司.左右金吾街仗司.法酒库.内酒坊.牛羊司.乳酪院.仪鸾司.〔帐设局也.〕车辂院.供奉库.杂物库.杂卖务.东西作坊.万全.〔造军器所.〕修内司.文思院.上下界绫锦院.交繍院.军器所.上下竹木务.箔场.车营.致远务.骡务.駞坊.象院.作坊.物料库.东西窑务.内外物库.油醋库.京城守具所.鞍辔库.养马曰左右骐骥院.天驷十监.河南北十炭场.四熟药局.内外柴炭库.军头引见司.架子营.〔楼店务.店宅务.〕榷货务.都茶场.大宗正司.左藏大观元丰宣和等库.编估局.打套所.诸米麦等.自州东虹桥元丰仓.顺成仓.东水门里广济.里河折中.外河折中.富国.广盈.万盈.永丰.济远等仓.陈州门里麦仓子.州北夷门山.五丈河诸仓.约共有五十余所.日有支纳.下卸即有下卸指军兵士.支遣即有袋家.毎人肩两石布袋.遇有支遣.仓前成市.近新城有草汤二十余所.毎遇冬月诸郷纳粟秆草.牛车阗塞道路.车尾相衔.数千万量不絶.场内堆积如山.诸军打请.营在州北.即往州南仓.不许雇人般担.并要亲自肩来.祖宗之法也.

\hypertarget{header-n25}{%
\subsection{卷二}\label{header-n25}}

御街

坊巷御街.自宣徳楼一直南去.约阔二百余歩.两边刀御廊.旧许市人买卖于其间.自政和间官司禁止.各安立黒漆杈子.路心又安朱漆杈子两行.中心御道.不得人马行往.行人皆在廊下朱杈子之外.杈子里有砖石甃砌御沟水两道.宣和间尽植莲荷.近岸植桃李梨杏.杂花相间.春夏之间.望之如繍.

宣徳楼前省府宫宇

宣徳楼前.左南庙对左掖门.为明堂颁朔布政府.秘书省右廊南对右掖门.近东则两府八位.西则尚书省.御街大内前南去.左前景灵东宫.右则西宫.近南大晟府.次曰太常寺.州桥曲转.大街面南.曰左藏库.近东郑太宰宅.青鱼市内行.景灵东宫.南门大街以东.南则唐家金银铺.温州漆器什物铺.大相国寺.直至十三间楼.旧宋门.自大内西廊南去.即景灵西宫.南曲对即报慈寺街.都进奏院.百钟圆药铺.至浚仪桥大街.西宫南皆御廊杈子.至州桥投西大街.乃果子行.街北都亭驿〔大辽人使驿也.〕.相对梁家珠子铺.余皆卖时行纸画花果铺席.至浚仪桥之西.即开封府.御街一直南去.过州桥.两边皆居民.街东车家炭.张家酒店.次则王楼山洞梅花包子.李家香铺.曹婆婆肉饼.李四分茶.至朱雀门街西过桥.即投西大街.谓之(麦曲)院街.街南遇仙正店.前有楼子.后有台.都人谓之『台上』.此一店最是酒店上戸.银缾酒七十二文一角.羊羔酒八十一文一角.街北薛家分茶.羊饭.熟羊内铺.向西去皆妓女馆舎.都人谓之『院街』.御廊西即鹿家包子.余皆羹店.分茶.酒店.香药铺.居民.

朱雀门外街巷

出朱雀门东壁.亦人家.东去大街.麦秸巷.状元楼.余皆妓馆.至保康门街.其御街东朱雀门外.西通新门瓦子以南杀猪巷.亦妓馆.以南东西两教坊.余皆居民或茶坊.街心市井.至夜尤盛.过龙津桥南去.路心又设朱漆杈子.如内前.东刘廉访宅.以南太学.国子监.过太学.又有横街.乃太学南门.街南熟药惠民南局.以南五里许.皆民居.又东去横大街.乃五岳观后门.大街约半里许.乃看街亭.寻常车驾行幸.登亭观马骑于此.东至贡院什物库.礼部贡院.车营务草场.街南葆眞宫.直至蔡河云骑桥.御街至南薫门里街西五岳观.最为雄壮.自西门东去观桥.宣泰桥.柳阴牙道.约五里许.内有中太一宫.佑神观.街南明丽殿.奉灵园.九成宫内安顿九鼎.近东即迎祥池.夹岸垂杨.菰蒲莲荷.凫鴈游泳其间.桥亭台榭.棊布相峙.唯毎歳清明日放万姓烧香游观一日.龙津桥南西壁邓枢密宅.以南武学巷内曲子张宅.武成王庙.以南张家油饼.明节皇后宅.西去大街.曰大巷口.又西曰清风楼酒店.都人夏月多乘凉于此.以西老鸦巷口军器所.直接第一座桥.自大巷口南去延眞观.延接四方道民于此.以南西去小巷口三学院.西去直抵宜男桥小巷.南去即南薫门.其门寻常士庶殡葬车舆.皆不得经由此门而出.谓正与大内相对.唯民间所宰猪.须从比入京.毎日至晩.毎羣万数.止十数人驱逐.无有乱行者.

州桥夜市

出朱雀门.直至龙津桥.自州桥南去.当街水饭.爊肉.干脯.王楼前貛儿.野狐.肉脯.鶏.梅家鹿家鹅鸭鶏兔肚肺鳝鱼包子.鶏皮.腰肾.鶏碎.毎个不过十五文.曹家从食.至朱雀门.旋煎羊.白肠.鲊脯.(火赞)冻鱼头.姜豉(枼刂)子.抹臓.红丝.批切羊头.辣脚子.姜辣萝卜.夏月麻腐鶏皮.麻饮细粉.素签沙糖.冰雪冷元子.水晶皂儿.生淹水木瓜.药不瓜.鶏头穰沙糖.菉豆.甘草冰雪凉水.荔枝膏.广芥瓜儿.醎菜.杏片.梅子姜.莴苣笋.芥辣瓜儿.细料馉饳儿.香糖果子.间道糖荔枝.越梅.(金屈)刀紫苏膏.金丝党梅.香枨元.皆用梅红匣儿盛贮.冬月盘兔.旋炙猪皮肉.野鸭肉.滴酥水晶鲙.煎夹子.猪脏之类.直至龙津桥须脑子肉止.谓之杂嚼.直至三更.

东角楼街巷

自宣徳东去东角楼.乃皇城东南角也.十字街南去姜行.高头街北去.从纱行至东华门街.晨晖门.宝箓宫.直至旧酸枣门.最是铺席要闹.宣和间展夹城牙道矣.东去乃潘楼街.街南曰『鹰店』.只下贩鹰鹘客.余皆眞珠疋帛香药铺席.南通一巷.谓之『界身』.并是金银彩帛交易之所.屋宇雄壮.门前广阔.望之森然.毎一交易.动即千万.骇人闻见.以东街北曰潘楼酒店.其下毎日自五更市合.买卖衣物书画珍玩犀玉.至平明.羊头.肚肺.赤白腰子.奶房.肚胘.鹑兔.鸠鸽.野味.螃蟹.蛤蜊之类讫.方有诸手作人上市买卖零碎作料.饭后饮食上市.如酥蜜食.枣(飠固).(氵蹬)砂团子.香糖果子.蜜煎雕花之类.向晩卖河娄头面.冠梳额抹.珍玩动使之类.东去则徐家瓠羹店.街南桑家瓦子.近北则中瓦.次里瓦.其中大小勾栏五十余座.内中瓦子莲花棚.牡丹棚.里瓦子夜叉棚.象棚最大.可容数千人.自丁先现.王团子.张七圣辈.后来可有人于此作场.瓦中多有货药.卖卦.喝故衣.探搏.饮食.剃剪.纸画.令曲之类.终日居此.不觉抵暮.

潘楼东街巷

潘楼东去十字街.谓之土市子.又谓之竹竿市.又东十字大街.曰从行裹角.茶坊毎五更点灯.博易买卖衣物图画花环领抹之类.至晓即散.谓之『鬼市子』.以东街北赵十万宅街.南中山正店.东楡林巷.西楡林巷.北郑皇后宅.东曲首向北墙畔单将军庙.乃单雄信墓也.上有枣树.也传乃枣槊发芽生长成树.又谓之枣冢子巷.又投东.则旧曹门街.北山子茶坊.内有仙洞.仙桥.仕女往往夜游.吃茶于彼.又李生菜小儿药铺.仇防御药铺.出旧曹门.朱家桥瓦子.下桥.南斜街.北斜街.内有泰山庙.两街有妓馆.桥头人烟市井.不下州南.以东牛行街.下马刘家药铺.看牛楼酒店.亦有妓馆.一直抵新城.自土市于南去铁屑楼酒店.皇建院街.得胜桥郑家油饼店.动二十余炉.直南抵太庙街.高阳正店.夜市尤盛.土市北去.乃马行街也.人烟浩闹.先至十字街.曰鹩儿市.向东曰东鶏儿巷.向西曰西鶏儿巷.皆妓馆所居.近北街曰杨楼街.东曰庄楼.今改作和乐楼.楼下乃卖马市也.近北曰任店.今改作欣乐楼.对门马铛家羹店.

酒楼

凡京师酒店.门首皆缚彩楼欢门.唯任店入其门.一直主廊约百余歩.南北天井两廊皆小合子.向晩灯烛荧煌.上下相照.浓妆妓女数百.聚于主廊槏面上.以待酒客呼唤.望之宛若神仙.北去杨楼.以北穿马行街.东西两巷.谓之大小货行.皆工作伎巧所居.小货行通鶏儿巷妓馆.大货行通笺纸店白矾楼.后改为丰乐楼.宣和间.更修三层相高.五楼相向.各有飞桥栏槛.明暗相通.珠帘绣额.灯烛晃耀.明开数日.毎先到者赏金旗.过一两夜.则已元夜.则金一瓦陇中皆置莲灯一盏.内西楼后来禁人登眺.以第一层下视禁中.大抵诸酒肆瓦市.不以风雨寒暑.白昼通夜.骈阗如此.州东宋门外仁和店.姜店.州西宜城楼.药张四店.班楼.金粱桥下刘楼.曹门蛮王家.乳酪张家.州北八仙楼.戴楼门张八家园宅正店.郑门河王家.李七家正店.景灵宫东墙长庆楼.在京正店七十二戸.此外不能遍数.其余皆谓之『脚店』.卖贵细下酒.迎接中贵饮食.则第一白厨.州西安州巷张秀.以次保康门李庆家.东鶏儿巷郭厨.郑皇后宅后宋厨.曹门砖筒李家.寺东骰子李家.黄胖家.九桥门街市酒店.彩楼相对.繍旆相招.掩翳天日.政和后来.景灵宫东墙下长庆楼尤盛.

饮食果子

凡店内卖下酒厨子.谓之『茶饭量酒博士』.至店中小儿子.皆通谓之『大伯』.更有街坊妇人.腰繋青花布手巾.绾危髻.为酒客换汤斟酒.俗谓之『焌糟』.更有百姓入酒肆.见子弟少年辈饮食.近前小心供过.使令买物命妓.取送钱物之类.谓之『闲汉』.又有向前换汤斟酒歌唱.或献菓子香乐之类.客散得钱.谓之『厮波』.又有下等妓女.不呼自来.筵前歌唱.临时以些小钱物赠之而去.谓之『剳客』.亦谓之『打酒坐』.又有卖药或果实萝卜之类.不问酒客买与不买.散与坐客.然后得钱.谓之『撒暂』.如比处处有之.唯州桥炭张家.乳酪张家.不放前项人入店.亦不卖下酒.唯以好淹藏菜蔬.卖一色好洒.所谓茶饭者.乃百味羹.头羹.新法鹌子羹.三脆羹.二色腰子.虾蕈.鶏蕈.浑炮等羹.旋索粉.玉碁子.羣仙羹.假河鲀.白渫(上艹下韲).货鳜鱼.假元鱼.决明兜子.决明汤(上艹下韲).肉醋托胎衬肠沙鱼.两熟紫苏鱼.假蛤蜊.白肉.夹面子茸割肉.胡饼.汤骨头.乳炊羊(肫右下灬).羊闹厅.羊角.(上夕下肉)腰子.鹅鸭.排蒸荔枝腰子.还元腰子.烧臆子.入炉细项.莲花鸭.签酒炙肚胘.虚汁垂丝羊头.入炉羊羊头.签鹅鸭.签鶏.签盘兔.炒兔.葱溌兔.假野狐.金丝肚羹.石肚羹.假炙獐.煎鹌子.生炒肺.炒蛤蜊.炒蟹.煠蟹.洗手蟹之类.遂时施行索唤.不许一味有阙.或别呼索变造下酒.亦实时供应.又有外来托卖炙鶏.燠鸭.羊脚子.点羊头.脆筋巴子.姜虾.酒蟹.獐巴.鹿脯.从食蒸作.海鲜时菓.旋切莴苣生菜.西京笋.又有小儿子.着白虔布衫.青花手巾.挟白磁缸子.卖辣菜.又有托小盘卖干菓子.乃旋炒银杏.栗子.河北鹅梨.梨条.梨干.梨肉.胶枣.枣圏.梨圏.桃圏.核桃.肉牙枣.海红嘉庆子.林檎旋乌李.李子旋樱桃.煎西京雪梨.夫梨.甘棠梨.凤栖梨.鎭府浊梨.河阴石榴.河阳査子.査条.沙苑榲桲.回马孛萄.西川乳糖.狮子糖.霜蜂儿.橄榄.温柑.绵枨金橘.龙眼.荔枝.召白藕.甘蔗.漉梨.林檎干.枝头干.芭蕉干.人面子.巴览子.榛子.榧子.虾具之类.诸般蜜煎香药.菓子罐子.党梅.柿膏儿.香药.小元儿.小臈茶.鹏沙元之类.更外卖软羊诸色包子.猪羊荷包.烧肉干脯.玉板鲊(犭巴).鲊片醤之类.其余小酒店.亦卖下酒.如煎鱼.鸭子.炒鶏兔.煎燠肉.梅汁.血羹.粉羹之类.毎分不遇十五钱.诸酒店必有厅院.廊庑掩映.排列小合子.吊窗花竹.各垂帘幙.命妓歌笑.各得稳便.

\hypertarget{header-n44}{%
\subsection{卷三}\label{header-n44}}

马行街北诸医铺

马行北去.乃小货行.时楼大骨传药铺.直抵正系旧封丘门.两行金紫医官药铺.如社金钩家.曹家.独胜元.山水李家.口齿咽喉药.石鱼儿.班防御.银孩儿.栢郎中家.医小儿.大鞋任家.产科.其余香药铺席.官员宅合.不欲遍记.夜市北州桥又盛百倍.车马阗拥.不可驻足.都人谓之『里颜』.

大内西右掖门外街巷

大内西去右掖门.祅庙.直南浚仪桥街.西尚书省东门.至省前横街南.即御史台.西即郊社.省南门正对开封府后墙.名西门谓之西车子曲.史家瓠羹.万家馒头.在京第一.次曰呉起庙.出巷乃大内西角楼大街.西去踊路街.南太平兴国寺后门.北对启圣院街.以西殿前司相对清风楼.无比客店.张戴花洗面药.国太丞张老儿金龟儿.丑婆婆药铺.唐家酒店.直至梁门.正名阖阊.出梁门西去.街北建隆观.观内东廊于道士卖齿药.都人用之.街南蔡太师宅.西去州西瓦子.南自汴河岸.北抵梁门大街亚其里瓦.约一里有余.过街北即旧宜城楼.近西去金梁桥街.西大街.荆筐儿药铺.枣王家金银铺.近北巷口熟药惠氏西局.西去瓮市子.乃开封府刑入之所也.西去盖防御药铺.大佛寺.都亭西驿.相对京城守具所.自瓮市子北去大街.班楼酒店.以北大三桥子.至白虎桥.直北即卫州门.

大内前州桥东街巷

大内前州桥之东.临汴河大街.曰相国寺.有桥平正.如州桥.与保康门相对.桥西贾家瓠羹.孙好手馒头.近南则保康门潘家黄耆圆.延宁宫禁.女道士观.人罕得入.街西保康门瓦子.东去沿城皆客店.南方官员商贾兵级.皆于此安泊.近东四圣观.袜袎巷.以东城角定力院.内有朱梁高祖御容.出保康门外.新建三尸庙.徳安公庙.南至横街.西去通御街.曰麦稍巷口.以南太学东门.水柜街余家染店.以南街东法云寺.又西去横街.张驸马宅.寺南佑神观后门.

相国寺内万姓交易

相国寺毎月五次开放万姓交易.大三门上皆是飞禽猫犬之类.珍禽奇兽.无所不有.第二三门皆动用什物.庭中设彩幙露屋义铺.卖蒲合.簟席.屏帏.洗漱.鞍辔.弓剑.时果.腊脯之类.近佛殿.孟家道院王道人蜜煎.赵文秀笔.及潘各墨.占定两廊.皆诸寺师姑卖繍作.领抹.花朶.珠翠头面.生色销金花样幞头帽子.特髻冠子.绦线之类.殿后资圣门前.皆书籍玩好图画及诸路罢任官员土物香药之类.后廊皆日者货术传神之类.寺三门阁上并资圣门.各有金银铸罗汉五百尊.佛牙等.凡有斋供.皆取旨方开三门.左右有两缾琉璃塔.寺内有智海.惠林.宝梵.河沙东西塔院.乃出角院舎.各有住持僧官.毎遇斋会.凡饮食茶果.动使器皿.虽三五百分.莫不咄嗟而辧.大殿两廊.皆国朝名公笔迹.左壁画炽盛光佛降九矅鬼百戏.右壁佛降鬼子母掲孟.殿庭供献乐部马队之类.大殿朶庙.皆壁隐楼殿人物.莫非精妙.

寺东门街巷

寺东门大街.皆是幞头.腰带.书籍.冠朶铺席.丁家素茶.寺南即録事巷妓馆.繍巷皆师姑繍作居住.北即小甜水巷.巷内南食店甚盛.妓馆亦多.向北李庆糟姜铺.直北出景灵宫东门前.又向北曲东税务街.高头街.姜行后巷.乃脂皮画曲妓馆.南北讲堂巷.孙殿丞药铺.靴店.出界身北巷.巷口宋家生药铺.铺中两壁.皆李成所画山水.自景灵宫东门大街向东.街北旧干明寺.沿火改作五寺三监.以东向南曰第三条甜水巷.以东熙熙楼客店.都下着数.以东街南高阳正店.向北入马行.向东.街北曰车辂院.南曰第二甜水巷.以东审计院.以东桐树子韩家.直抵太庙前门.南往观音除.乃第一条甜水巷也.太庙北入楡林巷.通曹门大街.不能遍数也.

上清宫

上清宫.在新宋门里街北.以西茆山下院.醴泉观.在东水门里.观音隙.在旧宋门后太庙南门.景徳寺.在上清宫背.寺前有桃花洞.皆妓馆.开宝寺.在旧封丘门外斜街子.内有二十四院.惟仁王院最盛.天清寺.在州北清晖桥.兴徳院.在金水门外.长生宫.在鹿家巷.显宁寺.在炭场巷.北婆台寺.在陈州门里.兜率寺.在红门道地.踊佛寺.在州西草场巷街.南十方静因院.在川西油醋巷.浴室院.在第三条甜水巷.福田院.在旧曹门外.报恩寺.在卸盐巷.太和宫女道士.在州西洪桥子大街.洞元观女道士.在班楼北.瑶华宫.在金水门外.万寿观.在旧酸枣门外十王宫前.

马行街铺席

马行北去旧封丘门外祅庙斜街州北瓦子.新封丘门大街两边民戸铺席外.余诸班直军营相对.至门约十里余.其余坊巷院落.纵横万数.莫知纪极.处处拥门.各有茶坊酒店.勾肆饮食.市井经纪之家.往往只于市店旋买饮食.不置家蔬.北食则矾楼前李四家.段家爊物.石逢巴子.南食则寺桥金家.九曲子周家.最为屈指.夜市直至三更尽.纔五更又复开张.如要闹去处.通晓不絶.寻常四梢远静去处.夜市亦有燋酸豏.猪胰.胡饼.和菜饼.貛儿.野狐肉.果不翘羹.灌肠.香糖果子之类.冬月虽大风雪阴雨.亦有夜市.(枼刂)子姜豉.抹脏.红丝水晶脍.煎肝臓.蛤蜊.螃蟹.胡桃.泽州饧.奇豆.鹅梨.石榴.査子.榲桲.糍糕.团子.盐豉汤之类.至三更方有提瓶卖茶者.盖都人公私荣干.夜深方归也.

船载杂卖

东京般载车.大者曰『太平』.上有箱无盖.箱如构栏而平.板壁前出两木.长二三尺许.驾车人在中间.两手扶捉鞭(纟安)驾之.前列骡或驴二十余.前后作两行.或牛五七头拽之.车两轮与箱齐.后有两斜木脚拖夜.中间悬一铁铃.行则有声.使远来者车相避.仍于车后繋驴骡二头.遇下峻险桥路.以鞭諕之.使倒坐缍车.令缓行也.可载数十石.官中车惟用驴差小耳.其次有『平头车』.亦如『太平车』而小.两轮前出长不作辕木.梢横一木.以独牛在辕内.项负横木.人在一边.以手牵牛鼻绳驾之.酒正店多以此载酒梢桶矣.梢桶如长水桶.面安靥口.毎梢三斗许.一贯五百文.又有宅眷坐车子.与『平头车』大抵相似.但椶作盖.及前后有构栏门.垂帘.又有独轮车.前后二人把驾.两旁两人扶拐.前有驴拽.谓之『串车』.以不用耳子转轮也.般载竹木瓦石.但无前辕.止一人或两人推之.此车往往卖糕及餻麋之类人用.不中载物也.平盘两轮.谓之『浪子车』.唯别人拽.又有载巨石大木.只有短梯盘而无轮.谓之『痴车』.皆省人力也.又有駞骡驴(马犬)子.或皮或竹为之.如方匾竹(上竹下差).两搭背上.斛(豆斗)则用布袋駞之.

都市饯陌

都市饯陌.官用七十七.街市通用七十五.鱼肉菜七十二陌.金银七十四.珠珍.雇婢妮.买(上丿下虫)蚁六十八.文字五十六陌.行市各有长短使用.

雇觅人力

凡雇觅人力.干当人.酒.食作匠之类.各有行老供雇.觅女使即有引至牙人.

防火

毎坊巷三百歩许.有军巡铺屋一所.铺兵五人.夜间巡警収领公事.又于高处砖砌望火楼.楼上有人卓望.下有官屋数间.屯驻军兵官余人.及有救火家事.谓如大小桶.洒子.麻搭.斧锯.梯子.火叉.大索.铁猫儿之类.毎遇有遗火去处.则有马军奔报军厢主.马歩军.殿前三衙.开封府各领军级扑灭.不劳百姓.

天晓诸人入市

毎日交五更.诸寺院行者打铁牌子或木无循门报晓.亦各分地分.日间求化.诸趍朝入市之人.闻此而起.诸门桥市井已开.如瓠羹店门首坐一小儿.叫饶骨头.间有灌肺及炒肺.酒店多点灯烛沽卖.毎分不过二十文.并粥饭点心.亦间或有卖洗面水.煎点汤茶药者.道至天明.其杀猪羊作坊.毎人檐猪羊及车子上市.动即百数.如果木亦集于朱雀门外及州桥之西.谓之菓子行.纸画儿亦在彼处.行贩不絶.其卖麦麺.毎秤作一布袋.谓之『一宛』.或三五秤作一宛.用太平车或驴马(马犬)之.从城外守门入城货卖.至天明不絶.更有御街州桥至南内前趂朝卖药及饮食者.吟叫百端.

诸色杂卖

若养马.则有两人日供切草.养犬则供饧糟.养猫则供猫食并小鱼.其锢路.钉铰.(上竹下陋)桶.修整动使.掌鞋.刷腰带.修幞头帽子.补角冠.日供打香印者.则管定铺席人家牌额.时节即印施佛像等.其供人家打水者.各有地分坊巷.及有使漆.打钗环.荷大斧斫柴.换扇子柄.供香饼子.炭团.夏月则有洗毡淘井老.举意皆在目前.或军营放停.乐人动皷乐于空闲.就坊巷引小儿妇女观看.散糖果子之类.谓之『卖梅子』.又谓之『把街』.毎日如宅舎宫院前.则有就门卖羊肉.头肚.腰子.白肠.鹑兔鱼虾.退毛鶏鸭.蛤蜊.螃蟹.杂燠.香药果子.博卖冠梳领抹.头面衣着.动使铜铁器.衣箱.磁器之类.亦有扑上件物事者.谓之『勘宅』.其后街或闲空处团转盖屋.向背聚居.谓之『院子』.皆小民居止.毎日卖蒸梨枣.黄糕麋.宿蒸饼.发牙豆之类.毎遇春时.官中差人夫监淘在城渠.别开坑盛淘出者(泥,"匕"作"工").谓之『(泥,"匕"作"工")盆』.候官差人来捡视了方盖覆.夜间出入.月黒宜照管也.

\hypertarget{header-n73}{%
\subsection{卷四}\label{header-n73}}

军头司

军头司毎旬休按阅内等子
相扑手.剑棒手格闘.诸军营.殿前指挥使直.在禁中有左右班内殿直.散员.散都头.散直.散指挥.御龙左右直系打御从物.御龙骨朶子直.弓箭直.弩直.习驭直.骑御马钧容直.招箭班.金鎗班.银鎗班.殿侍诸军东西五班常入祇候.毎日教阅野战.毎遇诸路解到武艺人.对御格闘.天武.捧日.龙卫.神卫.各二十指挥.谓之上四军.不出戌.骁骑.云骑.拱圣.龙猛.龙骑.各十指挥.殿前司.歩军司有虎翼各二十指挥.虎翼水军.宣武各十五指挥.神勇.广勇各十指挥.飞山.床子弩.雄武.广固等指挥.诸司则宣效六军.武肃.武和.街道司诸司诸军指挥.动以百数.诸宫观宅院各有清卫厢军禁军剰员十指挥.其余工匠.修内司.八作司.广固作坊.后苑作坊.书艺局.绫锦院.文繍院.内酒坊.法酒库.牛羊司.油醋库.仪鸾司.翰林司.喝探.武严.辇官.车子院.皇城官.亲从官.亲事官.上下宫.皇城.黄皁院子.涤除.各有指挥.记省不尽.

皇太子纳妃

皇太子纳妃.卤部仪仗.宴乐仪卫.妃乘厌翟车.车上设紫色团盖.四柱维幕.四垂大带.四马驾之.

公主出降

公主出降.亦设仪杖.行幕.歩障.水路.凡亲王公主出则有之.皆系街道司兵级数十人.各执扫具.镀金银水桶.前导洒之.名曰『水路』.用檐床数百.铺设房卧.并紫衫巻脚幞头天武官抬舁.又有宫嫔数十.皆眞珠钗插吊朶玲珑簇罗头面.红罗销金袍帔.乘马双控双搭青盖前导.谓之『短镫』.前后用红罗销金掌扇遮簇.乘金铜裙檐子.覆以剪椶.朱红梁脊.上列渗金银铸云凤花朶檐子.约高五尺许.深八尺.阔四尺许.内容六人.四维垂繍额珠帘.白藤间花.匡箱之外.两壁出栏槛皆缕金花.装雕木人物神仙.出队两竿十二人.竿前后皆设緑丝绦金鱼勾子勾定.

皇后出乘舆

皇太后.皇后出乘者.谓之『舆』.比檐子稍増广.花样皆龙.前后檐皆剪椶.仪仗与驾出相似而少.仍无驾头.警跸耳.士庶家与贵家婚嫁.亦乘檐子.只无脊上铜凤花朶.左右两军.自有假赁所在.以至从人衫帽衣服从物倶可赁.不须借徣.余命妇王宫士庶通乘坐车子.如檐子样制.亦可客六人.前后有小勾栏.底下轴贯两挟朱轮.前出长辕约七八尺.独牛驾之.亦可假赁.

杂赁

若(上亠下凶)事出殡.自上而下.(上亠下凶)肆各有体例.如方相.车轝.结络.彩帛.皆有定价.不须劳力.寻常出街市干事.稍似路远倦行.逐坊巻桥市.自有假赁鞍马者.不过百钱.

修整杂货及斋僧请道

傥欲修整屋宇.泥补墙壁.生辰忌日.欲设斋僧尼道士.即早辰桥市街巷口皆有木竹匠人.谓之杂货工匠.必至杂作人夫.道士僧人.罗立会聚.候人请唤.谓之『罗斋』.竹不作料.亦有铺席.砖瓦泥匠.随手即就.

筵官假赁

凡民间吉(上亠下凶)筵会.椅桌陈设.器皿合盘.酒檐动使之类.自有茶酒司管赁.吃食下酒.自有厨司.以至托盘.下请书.安排坐次.尊前执事歌说劝酒.谓之『白席人』.总谓之『四司人』.欲就园馆亭榭寺院游赏命客之类.举意便辧.亦各有地分.承揽排备.自有则例.亦不敢过越取钱.虽百十分.厅馆整肃.主人只出钱而已.不用费力.

会仙酒楼

如州东仁和店.新门里会仙楼正店.常有百十分.厅馆动使各各足备.不尚少阙一件.大抵都人风俗奢侈.度量稍寛.凡酒店中不问何人.止两人对坐饮酒.亦须用注碗一副.盘盏两副.菓菜楪各五片.水菜椀三五只.即银近百两矣.虽一人独饮.盌遂亦用银盂之类.其菓子菜蔬.无非精洁.若别要下酒.即使人外买软羊.龟背.大小骨.诸色包子.玉板鲊.生削巴子.瓜姜之类.

食店

大凡食店.大者谓之『分茶』.则有头羹.石髓羹.白肉.胡饼.软羊.大小骨角.(上夕下肉)犒腰子.石肚羹.入炉羊罨.生软手麹.桐皮麺.姜溌刀.回刀.冷淘.棊子.寄炉麺饭之类.吃全茶.饶(上艹下韲)头羹.更有川饭店.即有插肉麺.大燠麺.大小抹肉淘.煎燠肉.杂煎事件.生熟烧饭.更有南食店.鱼兜子.桐皮熟脍麹.煎鱼饭.又有瓠羹店.门前以枋不及花样沓结缚如山棚.上挂成边猪羊.相间三二十边.近里门面窗戸.皆朱緑装余.谓之『驩门』.毎店各有厅院东西廊称呼坐次.客坐.则一人执筯纸.遍问坐客.都人侈纵.百端呼索.或热或常.或温或整.或絶冷.精浇.臕浇之类.人人索唤不同.行菜得之.近局次立.从头唱念.报与局内.当局者谓之『铛头』.又曰『着案』.讫.须臾.行菜者左手杈三椀.右臂自手至肩(马犬)叠约二十碗.散下尽合各人呼索.不容差错.一有差错.坐客白之主人.必加叱骂.或罚工价.甚者逐之.吾辈入店.则用一等琉璃浅棱椀.谓之『碧椀』.亦谓之『造羹』.菜蔬精细.谓之『造(上艹下韲)』.毎碗十文.麺与肉相停.谓之『合羹』.又有『单羹』.乃半个也.旧只用匙.今皆用筯矣.更有插肉.拨刀.炒羊.细物料.碁子.馄饨店.及有素分茶.如寺院斋食也.又有菜麺.胡蝶(上艹下韲)肐(月逹).及卖随饭.荷包.白饭.旋切细料馉饳儿.瓜(上艹下韲).萝卜之类.

肉行

坊巻桥市.皆有肉案.列三五人操刀.生熟肉从便索唤.阔切.片批.细抹.顿刀之类.至晩即有燠爆熟食上市.凡买物不上数钱得者是数.

饼店

凡饼店有油饼店.有胡饼店.若油饼店.即卖蒸饼.糖饼.装合.引盘之类.胡饼店则卖门油.菊花.寛焦.侧厚.油碢.髓饼.新样满麻.毎案用三五人捍剂卓花入炉.自五更卓案之声远近相闻.唯武成王庙前海州张家.皇建院前郑家最盛.毎家有五十余炉.

鱼行

卖生鱼则用浅抱桶.以柳叶间串清水中浸.或循街出卖.毎日早惟新郑门.西水门.万胜门.如此生鱼有数千檐入门.冬月即黄河诸远处客鱼来.谓之『车鱼』.毎斤不上一百文.

\hypertarget{header-n100}{%
\subsection{卷五}\label{header-n100}}

民俗

凡百所卖饮食之人.装鲜净盘合器皿.车檐动便.奇巧可爱食味和羹.不敢草略.其卖药卖卦.皆具冠带.至于乞丐者.亦有规格.稍似懈怠.众所不容.其土农工商诸行百戸衣装.各有本色.不敢越外.谓如香铺裹香人.即顶帽披背.质库掌事.即着皂衫角带不顶幅之类.街市行人.便认得是何色目.加之人情高谊.若见外方之入为都人凌欺.众必救护之.或见军铺収领到闘争公事.横身劝救.有陪酒食檐官方救之者.亦无惮也.或有从外新来.邻左居住.则相借徣动使.戏遗汤茶.指引买卖之类.更有提茶瓶之人.毎日邻里互相支茶.相问动静.凡百吉凶之家.人皆盈门.其正酒店戸.见脚店三两次打酒.便敢借与三五百两银器.以至贫下人家.就店呼酒.亦用银器供送.有连夜饮若.次日取之.诸妓馆只就店呼酒而已.银器供送.亦复如是.其阔略大量.天下无之也.以其人烟浩穰.添十数万众不加多.减之不觉少.所谓花阵酒池.香山药海.别有幽坊小巷.燕馆歌楼.举之万数.不欲繁碎.

京瓦伎艺

崇.观以来.在京瓦肆伎艺.张廷叟.孟子书.主张小唱.李师师.徐婆惜.封宜奴.孙三四等.诚其角者.嘌唱弟子.张七七.王京奴.左小四.安娘.毛团等.教坊减罢并温习.张翠盖.张成弟子.薛子大.薛子小.俏枝儿.杨总惜.周寿奴.称心等.般杂剧.杖头傀儡任小三.毎日五更头回小杂剧.差晩看不及矣.悬丝傀儡.张金线.李外宁.药发傀儡.张臻妙.温奴哥.眞个强.没勃脐.小掉刀.筋骨上索杂手伎.浑身眼.李宗正.张哥.球杖踢弄.孙寛.孙十五.曾无党.高恕.李孝详.讲史.李慥.杨中立.张十一.徐明.赵世亨.贾九.小说.王颜喜.盖中宝.刘名广.散乐.张眞奴.舞旋.杨望京.小儿相扑.杂剧.掉刀.蛮牌.董十五.赵七.曹保义.朱婆儿.没困驼.风僧哥.俎六姐.影戏.丁仪.痩吉等.弄乔影戏.刘百禽.弄(上丿下虫)蚁.孔三传.耍秀才.诸宫调.毛详.霍伯丑.商谜.呉八儿.合生.张山人.说诨话.刘乔.河北子.帛遂.呉牛儿.达眼五.重明乔.骆驼儿.李敦等.杂(口班).外入孙三神鬼.霍四究.说三分.尹常卖.五代史.文八娘.叫果子.其余不可胜数.不以风雨寒暑.诸棚看人.日日如是.教坊钧容直.毎遇旬休按乐.亦请人观看.毎遇内宴前一日.教坊内勾集弟子小儿.习队舞.作乐杂剧节次.

娶妇

凡娶媳妇.先起草砧子.两家允许.然后起细帖子.序三代名讳.议亲人有服亲田产官职之类.次檐许口酒.以络盛酒瓶.袋以大花八朶.罗绢生色或银胜八枚.又以花红缴檐上.谓之『缴檐红』.与女家.女家以淡水二瓶.活金三五个.筯一双.悉送在元酒瓶内.谓之『回鱼筯』.或下小定.大定.或相媳妇与不相.若相媳妇.即男家亲人或婆往女家看中.即以钗子插冠中.谓之『插钗子』.或不入意.即留一两端彩段.与之压惊.则此亲不谐矣.其媒人有数等.上等戴盖头.着紫背子.说官亲官院恩泽.中等戴冠子.黄包髻背子.或只繁裙手.把青凉伞儿.皆两人同行.下定了.即旦望媒人传语.遇节序.即以节物颜面羊酒之类追女家.随家丰俭.女家多回巧作之类.次下财礼.次报成结日子.次过大礼.先一日或是日早下催妆冠帔花粉.女家回公裳花幞头之类.前一日女家先来挂帐.铺设房卧.谓之『铺房』.女家亲人有茶酒利市之类.至迎娶日.儿家以车子或花檐子发迎客引至女家门.女家管待迎客.与之彩段.作乐催妆上车檐.从人未肯起.炒咬利市.谓之『起檐子』.与了然后行.迎客先回至儿家门.从人及儿家人乞觅利市钱物花红等.谓之『栏门』.新妇下车子.有阴阳人执斗.内盛谷豆钱菓草节等.呪祝望门而撤.小儿辈争拾之.谓之『撤谷豆』.俗云厌青羊等杀神也.新人下车檐.踏青布条或毡席.不得踏地.一人捧镜倒行.引新人跨鞍蓦草及秤上过.入门.于一室内当中悬帐.谓之『坐虚帐』.或只径入房中坐于床上.亦谓之『坐富贵』.其送女客.急三盏而退.谓之『走送』.众客就筵三杯之后.婿具公裳花胜簇面.于中堂升一榻.上置椅子.谓之『高坐』.先媒氏请.次姨氏或妗氏请.各斟一杯饮之.次丈母请.方下坐.新人门额.用彩一段.碎裂其下.横抹挂之.婿入房.即众争撦小片而去.谓之『利市缴门红』.婿于床前请新妇出.二家各出彩段.绾一同心.谓之『牵巾』.男挂于笏.女搭于手.男倒行出.面皆相向.至家庙前参拜毕.女复倒行.扶入房讲拜.男女各尹先后对拜毕.就床.女向左.男向右坐.妇女以金钱彩菓散掷.谓之『撒帐』.男左女右.留少头髪.二家出疋段.钗子.木梳.头须之类.谓之『合髻』.然后用两盏以彩结连之.互饮一盏.谓之『交杯酒』.饮讫掷盏.并花冠子于床下.盏一仰一合.俗云『大吉』.则众喜贺.然后掩帐讫.宫院中即亲随人抱女婿去.已下人家即行出房.参谢诸亲.复就坐饮酒.散后.次日五更.用一卓.盛镜台镜予于其上.望上展拜.谓之『新妇拜堂』.次拜尊长亲戚.各有彩段巧作鞋枕等为献.谓之『赏贺』.尊长则复换一疋回之.谓之『答贺』.婿往参妇家.谓之『拜门』.有力能趣辧.次日则往.谓之『复面拜门』.不然.三日七日皆可.赏贺亦如女家之礼.酒散.女家具皷吹从物.迎婿还家.三日.女家送彩段油蜜蒸饼.谓之『蜜和油蒸饼』.其女家来作会.谓之『暖女』.七日则取女归.盛送彩段头面与之.谓之『洗颜』.一日则大会相庆.谓之『满月』.自此以后.礼数简矣.

育子

凡孕妇入月.于初一日父母家以银盆.或錂或彩画盆.盛粟秆一束.上以锦繍或生色帕复盖之.上插花朶及通草.帖罗五男二女花样.用盘合装.送馒头.谓之『分痛」.并作眠羊.卧鹿羊.生菓实.取其眠卧之义.并牙儿衣物(衤朋)籍等.谓之『催生』.就蓐分娩讫.人争送粟米炭醋之类.三日落脐灸顖.七日谓之『一腊』.至满月则生色及(衤朋)繍銭.贵富家金银犀玉为之.并菓子.大展洗儿会.亲宾盛集.煎香汤于盆中.下菓子彩钱葱蒜等.用数丈彩绕之.名曰『围盆』.以钗子搅水.谓之『搅盆』.观者各撒钱于水中.谓之『添盆』.盆中枣子直立者.妇人争取食之.以为生男之征.浴儿毕.落胎髪.遍谢坐客.抱牙儿入他人房.谓之『移窠』.生子百日.置会.谓之『百晬』.至来歳生日.谓之『周晬』.罗列盘琖于地.盛菓木.饮食.官诰.笔研.筭秤等经巻针线应用之物.观其所先拈者.以为征兆.谓之『试晬』.此小儿之盛礼也.

\hypertarget{header-n111}{%
\subsection{卷六}\label{header-n111}}

正月

正月一日年节.开封府放关扑三日.士庶自早互相庆贺.坊巷以食物动使菓实柴炭之类.歌叫关扑.如马行.潘楼街.州东宋门外.州西梁门外踊路.州北封丘门外.及州南一带.皆结彩棚.铺陈冠梳.珠翠.头面.衣着.花朶.领抹.靴鞋.玩好之额.间列舞场歌馆.车马交驰.向晩.贵家妇女纵赏关赌.入场观看.入市店饮宴.惯习成风.不相笑鳞.至寒食冬至三日亦如此.小民虽贫者.亦须新洁衣服.把酒相酬尔.

元旦朝会

正旦大朝会.车驾坐大庆殿.有介冑长大人四人立于殿角.谓之『鎭殿将军』.诸国使人入贺.殿庭列法驾仪杖.百官皆冠冕朝服.诸路举人解首.亦士服立班.其服二量冠.白袍青縁.诸州进奏吏.各执方物入献.诸国使人.大辽大使顶金冠.后檐尖长.如大莲叶.服装窄袍.金蹀躞.副使展裹金带.如汉服.大使拜则立左足.跪右足.以两手着右肩为一拜.副使拜如汉仪.夏国使副.皆金冠.短小样制服.绯窄袍.金蹀躞.吊敦背.叉手展拜.高丽与南番交州使人.并如汉仪.回纥皆长髯高鼻.以疋帛缠头.散披其服.于阗皆小金花毡笠.金丝战袍.束带.并妻男同来.乘骆駞.毡兜铜铎入贡.三佛齐皆痩脊鞭头.绯衣.上织成佛面.又有南蛮五姓番.皆椎髻乌毡.并如僧人.礼拜入见.旋赐汉装锦袄之类.更有眞臈.大理.大石等国.有时来朝贡.其大辽使人.在都亭驿.夏国在都亭西驿.高丽在梁门外安州巷同文馆.回纥.于阗在礼宾院.诸番国在瞻云馆或懐远驿.唯大辽.高丽就馆赐宴.大辽使人朝见讫.翌日诣大相国寺烧香.次日诣南御苑射弓.朝廷旋选能射武臣伴射.就彼赐宴.三节人皆与焉.先列招箭班十余于垛子前.使人多用弩子射.一裹无脚小幞头子锦袄子辽人.踏开弩子.舞旋榙箭.过与使人.彼窥得端正.止令使人发牙.例本朝伴射用弓箭.中的则赐闹装.银鞍马.衣着.金银器物有差.伴射得捷.京师市井儿遮路争献口号.观者如堵.翌日人使朝辞.朝退.内前灯山已上彩.其速如神.

立春

立春前一日.开封府进春牛入禁中鞭春.开封.祥符两县.置春牛于府前.至日絶早.府僚打春.如方州仪.府前左右.百姓卖小春牛.往往花装栏坐.上列百戏人物.春幡雪柳.各相献遗.春日.宰执亲王百官.皆赐金银幡胜.入贺讫.戴归私第.

元宵

正月十五日元宵.大内前自歳前冬至后.开封府绞缚山棚.立不正对宣徳楼.游人已集御街两廊下.奇术异能.歌舞百戏.鳞鳞相切.乐声嘈杂十余里.撃丸蹴踘.踏索上竿.赵野人.倒吃冷淘.张九哥.呑铁剑.李外宁.药法傀儡.小健儿.吐五色水.旋烧泥丸子.大特落.灰药.榾柮儿.杂剧.温大头.小曹.嵇琴.党千.箫管.孙四.烧炼药方.王十二.作剧术.邹遇.田地广.杂扮.苏十.孟宣.筑球.尹常卖.五代史.刘百禽.(上丿下虫)蚁.杨文秀.皷笛.更有猴呈百戏.鱼跳刀门.使唤蜂蝶.追呼蝼蚁.其余卖药.卖卦.沙书地谜.奇巧百端.日新耳目.至正月七日.人使朝辞出门.灯山上彩.金碧相射.锦绣交辉.面北悉以彩结.山呇上皆画神仙故事.或坊市卖药卖卦之人.横列三门.各有彩结金书大牌.中曰『都门道』.左右曰『左右禁卫之门』.上有大牌曰『宣和与民同乐』.彩山左右.以彩结文殊.普贤.跨狮子白象.各于手指出水五道.其手摇动.用辘轳绞水上灯山尖高处.用木柜贮之.逐时放下.如瀑布状.又于左右门上.各以草把缚成戏龙之状.用青幕遮笼.草上密置灯烛数万盏.望之蜿蜒如双龙飞走.自灯山至宣徳门楼横大街.约百余丈.用棘刺围遶.谓之『棘盆』.内设两长竿.高数十丈.以缯彩结束.纸糊百戏人物.悬于竿上.风动宛若飞仙.内设乐棚.差衙前乐人作乐杂戏.并左右军百戏.在其中驾坐一时呈拽.宣徳楼上.皆垂黄縁.帘中一位.乃御座.用黄罗设一彩棚.御龙直执黄盖掌扇.列于帘外.两朶楼各挂灯球一枚.约方圆丈余.内燃椽烛.帘内亦作乐.宫嫔嬉笑之声.下闻于外.楼下用枋不垒成露台一所.彩结栏槛.两边皆禁卫排立.锦袍.幞头簮赐花.执骨朶子.面此乐棚.教坊钧容直.露台弟子.更互杂剧.近门亦有内等子班道排立.万姓皆在露台下观看.乐人时引万姓山呼.

十四日车驾幸五岳观

正月十四日.车驾幸五岳观迎祥池.有对御.〔谓赐群臣宴也.〕至晩还内围子.亲从官皆顶球头大帽.簪花.红锦团荅戏狮子衫.金镀天王腰带.数重骨朶.天武官皆顶双巻脚幞头.紫上大搭天鹅结带寛衫.殿前班顶两脚屈曲向后花装幞头.着绯青紫三色橪金线结带望仙苑袍.跨弓剑.乘马.一扎鞍辔.缨绋前导.御龙直一脚指天一脚圏曲幞头.着红方胜锦袄子.看带束带.执御从物.如金交椅.唾盂.水罐.菓垒.掌扇.缨绋之类.御椅子皆黄罗珠蹙背座.则亲从官执之.诸班直皆幞头锦袄束带.毎常驾出有红纱帖金烛笼二百对.元宵加以琉璃玉柱掌扇灯.快行家各执红纱珠络灯笼.驾将至.则围子数重.外有一人扑月样兀子锦.覆于马上.天武官十余人.簇拥扶策.喝曰.『看驾头.』次有吏部小使臣百余.皆公裳.执珠络球杖.乘马听唤.近侍余官皆服紫绯緑公服.三衙太尉.知合.御带罗列前导.两边皆内等子.选诸军膂力者.着锦袄顶帽.握拳顾望.有高声者捶之流血.教坊钩容直乐部前引.驾后诸班直马队作乐.驾后围子外左则宰执侍从.右则亲王.宗室南班官.驾近.则列横门十余人撃鞭.驾后有曲柄小红繍伞.亦殿侍执之于马上.驾入灯山.御辇院人员辇前喝『随竿媚来』.御辇团转一遭.倒行观灯山.谓之『鹁鸽旋』.又谓之『踏五花儿』.则辇官有喝赐矣.驾登宣徳楼.游人奔赴露台下.

十五日驾诣上清宫

十五日诣上清宫.亦有对御.至晩回内.

十六日

十六日车驾不出.自进早饍讫.登门乐作.巻帘.御座临轩.宣万姓.先到门下者.犹得瞻见天表.小帽红袍.独卓子.左右近侍.帘外伞扇执事之人.须臾下帘.则乐作.纵万姓游赏.两朶楼相对.左楼相对.郓王以次彩棚幕次.右楼相对.蔡太师以次执政戚里幕次.时复自楼上有金凤飞下诸幕次.宣赐不辍.诸幕次中.家妓竞奏新声.与山棚露台上下.乐声鼎沸.西条楼下.开封尹弹压幕次.罗列罪人满前.时复决遣.以警愚民.楼上时传口勑.特令放罪.于是华灯宝炬.月色花光.霏雾融融.动烛远近.至三皷.楼上以小红纱灯球縁索而至半空.都人皆知车驾还内矣.须臾闻楼外撃鞭之声.则山楼上下.灯烛数十万盏.一时灭矣.于是贵家车马.自内前鳞切.悉南去游相国寺.寺之大殿.前殿乐棚.诸军作乐.两廊有诗牌灯云.『天碧银河欲下来.月华如水照楼台.』并『火树银花合.星桥铁锁开』之诗.其灯以木牌为之.雕镂成字.以纱绢幂之于内.密燃其灯.相次排定.亦可爱赏.资圣阁前安顿佛牙.设以水灯.皆系宰执.戚里.贵近占设看位.最要闹.九子母殿及东西塔院.惠林.智海.宝梵.竞陈灯烛.光彩争华.直至达旦.其余宫观寺院.皆放万姓烧香.如开宝.景徳大佛守等处.皆有乐棚.作乐燃灯.惟禁宫观寺院.不设灯烛矣.次则葆眞宫有玉柱玉帘窗隔灯.诸坊巷.马行.诸香药铺唐.茶坊酒肆.灯烛各出新奇.就中莲华王家香铺灯火出羣.而又命僧道场打花钹.弄椎皷.游人无不驻足.诸门皆有官中乐棚.万街千巷.尽皆繁盛浩闹.毎一坊巷口.无乐棚去处.多设小影戏棚子.以防本坊游人小儿相失.以引聚之.殿前班在禁中右掖门里.则相对石掖门设一乐棚.放本班家口.登皇城观看.官中有宣赐茶酒妆粉钱之类.诸营班院于法不得夜游.各以竹竿出灯球于半空.远近高低.若飞星然.阡陌纵横.城闉不禁.别有深坊小巷.繍额珠帘.巧制新妆.竞夸华丽.春情荡扬.酒兴融怡.雅会幽欢.寸阴可惜.景色浩闹.不觉更阑.宝骑骎骎.香轮辘辘.五陵年少.满路行歌.万戸千门.笙簧未彻.市人卖玉海.夜蛾.蜂儿.雪柳.菩提叶.科头圆子.拍头焦(飠追).唯焦(飠追)以竹架子出青伞上.装缀梅红缕金小灯笼子.架子前后亦设灯笼.敲皷应拍.团团转走.谓之『打旋罗』.街巷处处有之.至十九日収灯.五夜城闉不禁.尝有旨展日.宣和年间.自十二月于政枣门〔二名景龙〕门上.如宣徳门元夜点照.门下亦置露台.南至宝箓宫.两边关扑买卖.晨晖门外设看位一所.前以荆棘围绕.周回的五七十歩.都下卖鹌鹑骨饳儿.圆子.(飠追)拍.白肠.水晶鲙.科头细粉.旋炒栗子.银杏.盐豉.汤鶏.段金橘.橄榄.龙眼.荔枝.诸般市合.团团密摆.准备御前索唤.以至尊有时在看位内.门司.御药.知省.太尉.悉在帘前.用三五人弟子祇应.籸盆照耀.有同白日.仕女观者.中贵邀住劝酒一金杯令退.直至上元.谓之『预赏』.惟周待诏瓠羹.贡余者一百二十文足一个.其精细果别如市店十文者.

収灯都人出城探春

収灯毕.都人争先出城探春.州南则玉津园外学方池亭榭.玉仙观.转龙弯西去一丈佛园子.王太尉园.奉圣寺前孟景初园.四里桥望牛冈剑客庙.自转龙弯东去陈州门外.园馆尤多.州东宋门外快活林.勃脐陂.独乐冈.砚台.蜘蛛楼.麦家园.虹桥王家园.曹.宋门之间东御苑.干明崇夏尼寺.州北李驸马园.州西新郑门大路.直过金明池西道者院.院前皆妓馆.以西宴宾楼有亭榭.曲折池塘秋千画舫.酒客税小舟.帐设游赏.相对祥祺观.直至板桥.有集贤楼.莲花楼.乃之官河东.陜西五略之别馆.寻常饯送.置酒于此.过板桥.有下松园.王太宰园.杏花冈.金明池角南去水虎翼巷水磨下蔡太师园.南洗马桥西巷内华严尼寺.王小姑酒店北金水河两浙尼寺巴娄寺.养种园.四时花木.繁盛可观.南去药梁园.童太师园.南去铁佛寺.鸿福寺.东西栢楡村.州北模天坡.角桥至仓王庙.十八寿壁尼寺.孟四翁酒店.州西北元有庶人园.有创台.流杯亭榭数处.放人春赏.大抵都城左近.皆是园圃.百里之内.并无閴地.次第春容满野.暖律暄晴.万花争出.粉墙细柳.斜笼绮陌.香轮暖辗.芳草如茵.骏骑骄嘶.杏花如繍.莺啼芳树.燕舞晴空.红妆按乐于宝榭层楼.白面行歌近画桥流水.举目则秋千巧笑.触处则蹴踘踈狂.寻芳选胜.花絮时坠.金樽折翠簪红.蜂蝶暗随归骑.于是相继清明节矣.

\hypertarget{header-n130}{%
\subsection{卷七}\label{header-n130}}

清明节

清明节.寻常京师以冬至后一百五日为大寒食.前一日谓之『炊熟』.用麺造枣(飠固)飞燕.柳条串之.插于门楣.谓之『子推燕』.予女及笄者.多以是日上头.寒食第三节.则清明日矣.凡新坟皆用此日拜扫.都城人出郊.禁中前半月发宫人车马朝陵.宗室南班近亲.亦分遣诣诸陵坟享祀.从人皆紫衫白绢三角子青行缠.皆系官给.节日亦禁中出车马.诣奉先寺道者院祀诸宫人坟.莫非金装绀幰.锦额珠帘.绣扇双遮.纱笼前导.士庶阗塞诸门.纸马铺皆于当街用纸衮叠成楼阁之状.四野如市.往往就芳树之下.或园囿之间.罗列杯盘.互相劝酬.都城之歌儿舞女.遍满园亭.抵暮而归.各携枣(飠固).炊饼.黄胖.掉刀.名花异果.山亭戏具.鸭卵鶏雏.谓之『门外土仪』.轿子即以杨柳杂花装簇顶上.四垂遮映.自此三日.皆出城上坟.但一百五日最盛.节日坊市卖稠饧.麦餻.乳酪.乳饼之类.缓入都门.斜阳御柳.醉归院落.明日梨花.诸军禁卫.各成队伍.跨马作乐四出.谓之『摔脚」.其旗旄鲜明.军容雄壮.人马精鋭.又别为一景也.

三日一日开金明池琼林苑

三月一日.州西顺天门外开金明池琼林苑.毎日教习军驾上池仪范.虽禁从士庶许纵赏.御史台有榜不得弹劾.池在顺天门外街北.周围约九里三十歩.池西直径七里许.入池门内南岸.西去百余歩.有面北临水殿.车驾临幸.观争标锡宴于此.往日旋以彩幄.政和间用土木工造成矣.又西去数百歩.乃仙桥.南北约数百歩.桥面三虹.朱漆阑楯.下排鴈柱.中央隆起.谓之『骆駞虹』.若飞虹之状.桥尽处.五殿正在池之中心.四岸石甃.向背大殿.中坐各设御幄.朱漆明金龙床.河间云水.戏龙屏风.不禁游人.殿上下回廊皆关扑钱物饮食伎艺人作场.勾肆罗列左右.桥上两边用瓦盆.内掷头钱.关扑钱物.衣服.动使.游人还往.荷盖相望.桥之南立棂星门.门里对立彩楼.毎争标作乐.列妓女于其上.门相对街南有砖石甃砌高台.上有楼观.广百丈许.曰宝津楼.前至池门.阔百余丈.下阚仙桥水殿.车驾临幸.观骑射百戏于此池之东岸.临水近墙皆垂杨.两边皆彩棚幕次.临水假赁.观看争标.街东皆酒食店舎.博易场戸.艺人勾肆.质库.不以几日解下.只至闭池.便典没出卖.北去直至池后门.乃汴河西水门也.其池之西岸.亦无屋宇.但垂杨蘸水.烟草铺堤.游人稀少.多垂钓之士.必于池苑所买牌子.方许捕鱼.游人得鱼.倍其价买之.临水砟脍.以荐芳樽.乃一时佳味也.习水教罢.繋小龙船于此.池岸正北对五殿.起大屋.盛大龙船.谓之『奥屋』.车驾临幸.往往取二十日.诸禁卫班直.簪花.披锦绣捻金线衫袍.金带勒帛之类结束.竞逞鲜新.出内府金鎗.宝装弓剑.龙凤绣旗.红缨锦辔.万骑争驰.铎声震地.

驾幸临水殿观争标锡宴

驾先幸池之临水殿锡燕羣臣.殿前出水棚.排立仪卫.近殿水中.横列四彩舟.上有诸军百戏.如大旗.狮豹.棹刀.蛮牌.神鬼.杂剧之类.又列两船.皆乐部.又有一小船.上结小彩楼.下有三小门.如傀儡棚.正对水中.乐船上参军色进致语.乐作.彩棚中门开.出小木偶人.小船子上有一白衣垂钓.后有小童举棹划船.辽遶数回.作语.乐作.钓出活小鱼一枚.又作乐.小船入棚.继有木偶筑球舞旋之类.亦各念致语.唱和.乐作而已.谓之『水傀儡』.又有两画船.上立秋千.船尾百戏人上竿.左右军院虞候监教皷苗相和.又一人上蹴秋千.将平架.筋斗掷身入水.谓之『水秋千』.水戏呈毕.百戏乐船.并各鸣锣皷.动乐舞旗.与水傀儡船分两壁退去.有小龙船二十只.上有绯衣军士各五十余人.各设旗皷铜锣.船头有一军校.舞旗招引.乃虎翼指挥兵级也.又有虎头船十只.上有一锦衣人.执小旗立船头上.余皆着青短衣.长顶头巾.齐舞棹.乃百姓卸在行人也.又有飞鱼船二只.彩画间金.最为精巧.上有杂彩戏衫五十余人.间列杂色小旗绯伞.左右招舞.鸣小锣皷铙铎之额.又有鳅鱼船二只.止容一人撑划.乃独木为之也.皆进花石朱缅所进.诸小船竞诣奥屋.牵拽大龙船出诣水殿.其小龙船争先团转翔舞.迎导于前.其虎头船以绳索引龙舟.大龙船约长三四十丈.阔三四丈.头尾鳞鬣.皆雕镂金饰.楻板皆退光.两边列十合子.充合分歇泊.中设御座龙水屏风.楻板到底深数尺.底上密排铁铸大银样.如卓面大者压重.庶不欹侧也.上有层楼台观.槛曲安设御座.龙头上人舞旗.左右水棚.排列六桨.宛若飞腾.至水殿.舣之一边.水殿前至仙桥.预以红旗插于水中.标识地分远近.所谓小龙船.列于水殿前.东西相向.虎头.飞鱼等船.布在其后.如两阵之势.须臾.水殿前水棚上一军校以红旗招之.龙船各鸣锣皷出阵.划棹旋转.共为圆阵.谓之『旋罗』.水殿前又以旗招之.其船分而为二.各圆阵.谓之『海眼』.又以旗招之.两队船相交互.谓之『交头』.又以旗招之.则诸船皆列五殿之东面.对水殿排成行列.则有小舟一军校执一竿.上挂以锦彩银盌之类.谓之『标竿』.插在近殿水中.又见旗招之.则两行舟鸣皷并进.捷者得标.则山呼拜舞.并虎头船之类.各三次争标而止.其小船复引大龙船入奥屋内矣.

驾幸琼林苑

驾方幸琼林苑.在顺天门大街.面北.与金晩池相对.大门牙道.皆古松怪柏.两傍有石榴园.樱桃园之类.各有亭榭.多是酒家所占.苑之东南隅.政和间创筑华觜冈.高数十丈.上有横观层楼.金碧相射.下有锦石缠道.宝砌池塘.柳锁虹桥.花萦凤舸.其花皆素馨.末莉.山丹.瑞香.含笑.射香等闽.广.二浙所进南花.有月池.梅亭牡丹之类.诸亭不可悉数.

驾幸宝津楼宴殿

宝律楼之南.有宴殿.驾临幸.嫔御车马在此.寻常亦禁人出入.有官监之.殿之西有射殿.殿之南有横街.牙道柳径.乃都人撃球之所.西去苑西门水虎翼巷横道之南.有古桐牙道.两傍亦有小园圃台榭.南过画桥.水心有大撮焦亭子.方池柳歩围绕.谓之虾(虫麻)亭.亦是酒家占.寻常驾未幸.习旱教于苑大门.御马立于门上.门之两壁.皆高设彩棚.许士庶观赏.呈引百戏.御马上池.则张黄盖撃鞭如仪.毎遇大龙船出.及御马上池.则游人増倍矣.

驾登宝津楼诸军呈百戏

鹫登宝津楼.诸军百戏.呈于楼下.先列皷子十数辈.一人摇双皷子.近前进致语.多唱『青春三月蓦山溪』也.唱讫.皷笛举一红巾者弄大旗.次狮豹入场.坐作进退.奋迅举止毕.次一红巾者.手执两白旗子.跳跃旋风而舞.谓之『扑旗子』.及上竿.打筋斗之类讫.乐部举动.琴家弄令.有花妆轻健军士百余.前列旗帜.各执雉尾.蛮牌.木刀.初成行列.拜舞互变开门夺桥等阵.然后列成偃月阵.乐部复动蛮牌令.数内两人出阵对舞.如撃刺之状.一人作奋撃之势.一人作僵仆.出场凡五七对.或以鎗对牌.剑对牌之类.忽作一声如霹雳.谓之『爆杖』.则蛮牌者引退.烟火大起.有假面披髪.口吐狼牙烟火.如鬼神状者上场.着青帖金花短后之衣.帖金皂袴.跣足.携大铜锣随身.歩舞而进退.谓之『抱锣』.遶场数遭.或就地放烟火之类.又一声爆杖.乐部动拜新月慢曲.有面涂青碌.戴面具金晴.饰以豹皮锦绣看带之类.谓之『硬鬼』.或执刀斧.或执杵棒之类.作脚歩蘸立.为驱捉视听之状.又爆仗一声.有假面长髯.展裹縁袍鞾简.如钟馗像者.傍一人以小锣相招和舞歩.谓之『舞判』.继有二三痩瘠.以粉涂身.金晴白面.如髑髅状.繋锦绣围肚看带.手执软仗.各作魁谐趋跄.举止若排戏.谓之『唖杂剧』.又爆仗响.有烟火就涌出.人面不相覩.烟中有七人.皆披髪文身.着青纱短后之衣.锦绣围肚看带.内一人金花小帽.执白旗.余皆头巾.执眞刀.互相格斗撃刺.作破面剖心之势.谓之『七圣刀』.忽有爆仗响.又复烟火.出散处以青幕围绕.列数十辈.皆假面异服.如祠庙中神鬼塑像.谓之『歇帐』.又爆仗响.巻退.次有一撃小铜锣.引百余人.或巾裹.或双髻.各着杂色半臂.围肚看带.以黄白粉涂其面.谓之『抹跄』.各执木棹刀一口.成行列.撃锣者指呼.各拜舞起居毕.喝喊变阵子数次.成一字阵.两两出阵格斗.作夺刀撃刺之态百端讫.一人弃刀在地.就地掷身.背着地有声.谓之『扳落』.如是数十对讫.复有一装田舎儿者入场.念诵言语讫.有一装村妇者入场.与村夫相値.各持捧杖互相撃触.如相驱态.其村夫者以杖背村妇出场毕.后部乐作.诸军缴队杂剧一段.继而露台弟子杂剧一段.是时弟子萧住儿.丁都赛.薛子大.薛子小.杨总惜.崔上寿之辈.后来者不足数.合曲舞旋讫.诸班直常入祗候子弟所呈马骑.先一人空手出马.谓之『引马』.次一人磨旗出马.谓之『开道旗』.次有马上抱红绣之球.撃以红锦索.掷下于地上.数骑追逐射之.左曰『仰手射』.右曰『合手射』.谓之『拖绣球』.又以柳枝插于地.数骑以刬子箭.或弓或弩射之.谓之『(礻昔)柳枝』.又有以十余小旗.遍装轮上而背之出马.谓之『旋风旗』.又有执旗挺立鞍上.谓之『立马』.或以身下马.以手攀鞍而复上.谓之『骗马』.或用手握定镫袴.以身从后鞭来往.谓之『跳马』.忽以身离鞍.屈右脚挂马鬃.左脚在镫.左手把鬃谓之『献鞍』.又曰『弃鬃背坐』.或以两手握镫袴.以肩着鞍桥.双脚直上.谓之『倒立』.忽掷脚着地.倒拖顺马而走.复跳上马.谓之『拖马』.或留左脚着镫.右脚出镫.离鞍横身.在鞍一边.右手捉鞍.左手把鬃存身.直一脚顺马而走.谓之『飞仙膊马』.又存身拳曲在鞍一边.谓之『镫里藏身』.或右臂挟鞍.足着地顺马而走.谓之『赶马』.或出一镫.坠身着秋.以手向下绰地.谓之『绰尘』.或放令马先走.以身追及.握马尾而上.谓之『豹子马』.或横身鞍上.或轮弄利刃.或重物大刀双刀百端讫.有黄衣老兵.谓之『黄院子』.数辈执小绣龙旗前导.宫监马骑百余.谓之『妙法院』.女童皆妙龄翘楚.结束如男子.短顶头巾.各着杂色锦绣捻金丝番段窄袍.红緑吊敦束带.莫非玉羁金勒.宝(革登)花鞯.艳色耀日.香风袭人.驰骤至楼前.团转数遭.轻帘皷声.马上亦有呈骁艺者.中贵人许畋押队.招呼成列.皷声一齐.掷身下马.一手执弓箭.揽缰子.就地如男子仪.拜舞山呼讫.复听皷声.骗马而上.大抵禁庭如男子装者.便随男子礼起居.复驰骤团旋分合阵子讫.分两阵.两两出阵.左右使马直背射弓.使番鎗或草棒.交马野战.呈骁骑讫.引退.又作乐.先设彩结小球门于殿前.有花装男子百余入.皆裹角子向后拳曲花幞头.半着红.半着青锦袄子.义襕束带.丝鞋.各跨雕鞍花(革毚)驴子.分为两队.各有朋头一名各执彩画球杖.谓之『小打』.一朋头用杖撃弄球子如缀球子方坠地.两朋争占.供与朋头.左朋撃球子过门入孟为胜.右明向前争占.不令入孟.互相追逐.得筹谢恩而退.续有黄院子引出宫监百余.亦如小打者.但加之珠翠装饰.玉带红靴.各跨小马.谓之『大打』.人人乘骑精熟.驰骤如神.雅态轻盈.姸姿绰约.人间但见其团画矣.呈讫.

驾幸射殿射弓

驾诣射殿射弓.垛子前列招箭班二十余人.皆长脚幞头.紫绣抹额紫寛衫.黄义襕.鴈翅排立.御箭去则齐声招舞.合而复开.箭中的矣.又一人口衔一银盌.两肩两手共五只.箭来皆能承之.射毕驾归宴殿.

池苑内纵人关扑游戏

池苑内除酒家艺人占外.多以彩幕缴络.铺设珍玉.奇玩.疋帛.动使.茶酒器物关扑.有以一笏扑三十笏者.以至车马.地宅.歌姫.舞女.皆约以价而扑之.出九和合有名者.任大头.快活三之类.余亦不数.池苑所进奉鱼藕果实.宣赐有差.后苑作进小龙船.雕牙缕翠.极尽精巧.随驾艺人池上作场者.宣.政间.张艺多.浑身眼.宋寿香.尹士安小乐器.李外宁水傀儡.其余莫知其数.池上饮食.水饭.凉水菉豆.螺蛳肉.饶梅花酒.査片.杏片.梅子.香药脆梅.旋切鱼脍.青鱼.盐鸭卵.杂和辣菜之类.池上水教罢.贵家以双缆黒漆平船.紫帷帐.设列家乐游池.宣.政间亦有假赁大小船子.许士庶游赏.其价有差.

驾回仪卫

驾回则御裹小帽.簪花乘马.前后从驾臣寮.百司仪卫.悉赐花.大观初.乘骢马至太和宫前.忽宣小鸟.其马至御前拒而不进.左右曰.『此愿封官.』勑赐龙骧将军.然后就辔.盖小鸟平日御爱之马也.莫非锦绣盈都.花光满目.御香拂路.广乐喧空.宝骑交驰.彩棚夹路.绮罗珠翠.戸戸神仙.画阁红楼.家家洞府.游人士庶.车马万数.妓女旧日多乘驴.宣.政间惟乘马.披凉衫.将盖头背繋冠子上.少年狎客.往往随后.亦跨马轻衫小帽.有三五文身恶少年控马.谓之『花褪马』.用短缰促马头.刺地而行.谓之『鞅缰』.呵喝驰骤.竞逞骏逸.游人往往以竹竿挑挂终日关扑所得之物而归.仍有贵家士女.小轿插花.不垂帘幙.自三月一日至四月八日闭池.虽风雨亦有游人.略无虚日矣.

是日季春.万花烂熳.牡丹芍药.棣棠木香.种种上市.卖花者以马头竹蓝铺排.歌叫之声.清奇可听.晴帘静院.晓幙高楼.宿病未醒.好梦初觉.闻之莫不新愁易感.幽恨悬生.最一时之佳况.诸军出郊.合教阵队.

\hypertarget{header-n152}{%
\subsection{卷九}\label{header-n152}}

四月八日

四月八日佛生日.十大禅院各有浴佛斋会.煎音药糖水相遗.名曰『浴佛水』.迤逦时光昼永.气序清和.榴花院落.时闻求友之莺.细柳亭轩.乍见引雏之燕.在京七十二戸诸正店.初卖煮酒.市井一新.唯州南清风楼最宜夏饮.初尝青杏.乍荐樱桃.时得佳宾.觥酬交作.是月茄瓠初出上市.东华门争先供进.一对可直三五十千者.时菓则御桃.李子.金杏.林檎之类.

端午

端午节物.百索艾花.银样皷儿花.花巧画扇.香糖果子.糉子.白团.紫苏.菖蒲.木瓜.并皆茸切.以香药相和.用梅红匣子盛裹.自五则一日及端午前一日.卖桃.柳.葵花.蒲叶.佛道艾.次日家家铺陈于门首.与糉子.五色水团.茶酒供养.又钉艾人于门上.士庶递相宴赏.

六月六日崔府君生日二十四日神保观神生日

六月六日州北崔府生日.多有戏送.无盛如此.二十四日州西灌口二郎生日.最为繁盛.庙在万胜门外一里许.勑赐神保观.二十三日御前献送后苑作与书艺局等处制造戏玩.如球杖.弹弓.弋射之具.鞍辔.衔勒.樊笼之类.悉皆精巧.作乐迎引至庙.于殿前露台上设乐棚.教坊钧容直作乐.更互杂剧舞旋.太官局供食.连夜二十四盏.各有节次.至二十四日.夜五更争烧头炉香.有在庙止宿.夜半起以争先者.天晓.诸司及诸行百姓献送甚多.其社火呈于露台之上.所献之物.动以万数.自早呈拽百戏.如上竿.趯弄.跳索.相扑.皷板.小唱.斗鶏.说诨话.杂扮.商谜.合笙.乔筋骨.乔相扑.浪子.杂剧.叫果子.学像生.倬刀.装鬼.砑皷.牌棒.道术之类.色色有之.至暮呈拽不尽.殿前两幡竿.高数十丈.左则京城所.右则修内司.搭材分占上竿呈艺解.或竿尖立横不列于其上.装神鬼.吐烟火.甚危险骇人.至夕而罢.

是月巷陌杂卖

是月时物.巷陌路口.桥门市井.皆卖大小米水饭.炙肉.干脯.莴苣笋.芥辣瓜儿.义塘甜瓜.卫州白桃.南京金桃.水鹅梨.金杏.小瑶李子.红菱.沙角儿.药木瓜.水木瓜.冰雪.凉水茘枝膏.皆用青布伞当街列床凳堆垛.冰雪惟旧宋门外两家最盛.悉用银器.沙糖菉豆.水晶皂儿.黄冷团子.鶏头穰.冰云.细料馉饳儿.麻饮鶏皮.细索凉粉.素签.成串熟林檎.脂麻团子.江豆碢儿.羊肉小馒头.龟儿沙馅之类.都人最重三伏.盖六月中别无时节.往往风亭水榭.峻宇高楼.云槛冰盘.浮瓜沈李.流杯曲沼.苞鲊新荷.远迩笙歌.通夕而罢.

七夕

七月七夕.潘楼街东宋门外瓦子.州西梁门外瓦子.北门外.南朱雀门外街及马行街内.皆卖磨喝乐.乃小塑土偶耳.悉以雕木彩装栏座.或用红纱碧笼.或饰以金珠牙翠.有一对直数千者.禁中及贵家与士庶为晴物追陪.又以黄(虫葛)铸为凫鴈.鸳鸯.鸂鶆.龟鱼之额.彩画金缕.谓之『水上浮』.又以小板上傅土.旋种粟令生苗.置小茅屋花木.作田舎家小人物.皆村落之态.语之『谷板』.又以瓜雕刻成花样.谓之「花瓜」.又以油麹糖蜜造为笑靥儿.谓之『果食花样』.奇巧百端.如捺香方胜之类.若买一斤数内有一对被介胄者.如门神之像.盖自来风梳.不知其从.谓之『果食将军』.又以菉豆.小豆.小麦.于磁器内以水浸之.生芽数寸.以红蓝彩缕束之.谓之『种生』.皆于街心彩幙帐设出络货卖.七夕前三五日.军马盈市.罗绮满街.旋折未开荷花.都人善假做双头莲.取玩一时.提携而归.路人往往嗟爱.又小儿须买新荷叶执之.盖効颦磨喝乐.儿童辈特地新妆.竞夸鲜丽.至初六日七日晩.贵家多结彩楼于庭.谓之『乞巧楼』.铺陈磨喝乐.花瓜.酒炙.笔砚.针线.或儿童裁诗.女郎呈巧.焚香列拜.谓之『乞巧』.妇女望月穿针.或以小物蜘蛛安合子内.次日看之.若网圆正.谓之『得巧』.里巻与妓馆.往往列之门首.争以侈靡相向.〔『磨喝乐』本佛经『摩睺罗』.今通俗而书之.〕

中元节

七月十五日中元节.先数日.市井卖冥器靴鞋.幞头帽子.金犀假带.五彩衣服.以纸糊架子盘游出卖.潘楼并州东西瓦子亦如七夕.要闹处亦卖果食种生花果之类.及印卖尊胜目连经.又以竹竿斫成三脚.高三五尺.上织灯窝之状.谓之孟兰盆.挂搭衣服冥饯在上焚之.杓肆乐人.自过七夕.便般『目连救母』杂剧.道至十五日止.观者増倍.中元前一日.则卖练叶.享祀时铺衬卓面.又卖麻谷窠儿.亦是繋在卓子脚上.乃告祖先秋成之意.又卖鶏冠花.谓之『洗手花』.十五日供养祖先素食.纔明即卖穄米饭.巡门叫卖.亦告成意也.又卖转明菜.花花油饼.馂豏.沙豏之类.城外有新坟者.即往拜扫.禁中亦出车马诣道者院谒坟.本院官给祠部十道.设大会.焚钱山.祭军阵亡殁.设孤魂之道场.

立秋

立秋日.满街卖楸叶.妇女儿童辈.皆剪成花样戴之.是月.瓜果梨枣方盛.京师枣有数品.灵枣.牙枣.青州枣.毫州枣.鶏头上市.则梁门里李和家最盛.中贵戚里.取索供卖.内中泛索.金合络绎.士庶买之.一裹十文.用小新荷叶包.糁以麝香.红小索儿繋之.卖者虽多.不及李和一色拣银皮子嫩者货之.

秋社

八月秋社.各以社糕.社酒相赍送贵戚.宫院以猪羊肉.腰子.妳房.肚肺.鸭饼.瓜姜之属.切作棊子片样.滋味调和.铺于饭上.谓之『社饭』.请客供养.人家妇女皆归外家.晩归.即外公姨舅皆以新葫芦儿.枣儿为遗.俗云宜良外甥.市学先生预敛诸生钱作社会.以致雇倩.祇应.白席.歌唱之人.归时各携花篮.果实.食物.社糕而散.春社.重午.重九.亦是如此.

中秋

中秋节前.诸店皆卖新酒.重新结络门面彩楼.花头画竿.醉仙锦旆.市人争饮.至午未间.家家无酒.拽下望子.是时螯蟹新出.石榴.榲勃.梨.枣.栗.孛萄.弄色枨橘.皆新上市.中秋夜.贵家结饰台榭.民间争占酒楼翫月.丝篁鼎沸.近内庭居民.夜深遥闻笙竿之声.宛若云外.闾里儿童.连宵嬉戏.夜市骈阗.至于通晓.

重阳

九月重赐.都下赏菊.有数种.其黄白色蘂若莲房.曰『万龄菊』.粉红色曰『桃花菊』.白而檀心曰.『木香菊』.黄色而圆者曰『金铃菊』.纯白而大者曰『喜容菊』.无处无之.酒家皆以菊花缚成洞戸.都人多出郊外登高.如仓王庙.四里桥.愁台.梁王城.砚台.毛驼冈.独乐冈等处宴聚.前一二日.各以粉(麦面)蒸餻遗送.上插剪彩小旗.掺饤果实.如石榴子.栗子黄.银杏.松子肉之类.又以粉作狮子蛮王之状.置于糕上.谓之『狮蛮』.诸禅寺各有斋会.惟开宝寺.仁王寺有狮子会.诸僧皆坐狮子上.作法事讲说.游人最盛.下旬即卖冥衣靴鞋席帽衣段.以十月朔日烧戏故也.

\hypertarget{header-n175}{%
\subsection{卷十}\label{header-n175}}

冬至

十一月冬至.京师最重此节.虽至贫者.一年之间.积累假借.至此日更易新衣.备辧饮食.享祀先祖.官放关扑.庆贺往来.一如年节.

大礼预教车象

遇大礼年.预于两日前教车象.自宣徳门至南薫门外.往来一遭.车五乘.以代五辂.轻重毎车上置旗二口.皷一面.驾以四马.挟车卫士.皆紫衫帽子.车前数人撃鞭.象七头.前列朱旗数十面.铜锣鼙皷十数面.先撃锣二下.皷急应三下.执旗人紫衫.帽子.毎一象则一人裹交脚幞头紫衫人跨其颈.手执短彬柄铜锣.尖其刃.象有不驯.撃之.象至宣徳楼楼前.团转行歩数遭成列.使之面北而拜.亦能唱喏.诸戚里.宗室.贵族之家.勾呼就私第观看.赠之银彩无虚日.御街游人嬉集.观者如织.卖扑土木粉揑小象儿.并纸画.看人携归.以为献遗.

车驾宿大庆殿

冬至前三日.驾宿大庆殿.殿庭广阔.可容数万人.尽列法驾仪仗于庭.不能周偏.有两楼对峙.谓之『钟皷楼』.上有太史局生.测験刻漏.毎时刻作鶏唱鸣皷一下.则一服緑者执牙牌而奏之.毎刻曰『某时几棒皷』.一时即曰『某时』.正宰执百官皆服法服.其头冠各有品从.宰执亲主加貂蝉笼巾九梁.从官七梁.余六梁至二梁有差.台諌増廌角也.所谓『梁』者.谓冠前额梁上排金铜叶也.皆绛袍皂縁.方心曲领.中单环佩.云头履鞋.随官品执笏.余执事人皆介帻绯袍.亦有等差.惟合门御史台加方心曲领尔.入殿祇应入给黄方号.余黄长号.绯方长号.各有所至去处.仪仗车辂.谓信幡龙旗相风鸟指南车.木辂.象辂.革辂.金辂.玉辂之类.自有三礼图可见.更不缕缕.排列殿门内外及御街.远近禁卫.全装铁骑.数万围绕大内.是夜内殿仪卫之外.又有裹锦縁小帽.锦络缝寛衫兵士.各执银裹头黒漆杖子.谓之『喝探』.兵士十余人作一队.聚首而立.凡十数队.各一名喝曰.『是与不是.』众曰.『是.』又曰.『是甚人.』众曰.『殿前都指挥使高俅.』更互喝叫不停.或如鶏叫.又置警场于宜徳门外.谓之『武严兵士』.画皷二百面.角称之.其角皆以彩帛如小旗脚装结其上.兵士皆小帽.黄繍抹额.黄繍寛衫.青窄衬衫.日晡时.三更时.各奏严也.毎奏先鸣角.角罢.一军校执一长软藤条.上繋朱拂子.擂皷者观拂子.随其高低.以皷声应其高下也.

驾行仪卫

次日五更.摄大宗伯执牌奏中严外辧.铁骑前导番衮.自三更时相续而行.象七头.各以文锦被其身.金莲花座安其背.金辔笼络其脑.锦衣人跨其颈.次第高旗大扇.画戟长矛.五色介胄.跨马之士.或小帽锦繍抹额者.或黒漆圆顶幞头者.或以皮如兜鍪者.或漆皮如戽斗而笼巾者.或衣红黄罨画锦繍之服者.或衣纯青纯皂以至鞋袴皆青黒者.或裹交脚幞头者.或以锦为绳如蛇而绕繋其身者.或数十人唱引持大旗面过者.或执大斧着.胯剑者.执鋭牌者.持镫棒者.或持竿上悬豹尾者.或持短杵者.其矛戟皆缀五色结带铜铎.其旗扇皆画以龙.或虎.或云彩.或山河.又有旗高五丈.谓之『次黄龙』.驾诣太庙青城.并先到.立斋宫前.叉竿舎索旗坐约百余人.或有交脚幞头.胯剑.足靴如四直使者千百数.不可名状.余诸司祇应人.皆锦袄.诸班直.亲从.亲事官.皆帽子.结带.红锦.或红罗上紫团答戏狮子.短后打甲背子.执御从物.御龙道皆眞珠结络.短顶头巾.紫上杂色小花繍衫.金束带.看带.丝鞋.天武官皆顶朱漆金装笠子.红上团花背子.三衙并带御器械官皆小帽.背子或紫繍战袍.跨马前导.千乘万骑.出宣徳门.由景灵宫太庙.

驾宿太庙奉神主出室

驾乘玉辂.冠服如图昼间星官之服.头冠皆北珠装结.顶通天冠.又谓之巻云冠.服络袍.执元圭.其玉辂顶皆镂金大莲叶攅簇.四性栏槛镂玉盘花龙凤.驾以四马.后出旗常.辂上御座.惟近侍二人.一从官傍立.谓之『执绥』.以备顾问.挟辂卫士.皆裹黒漆团顶无脚幞头.着黄生名寛衫.青窄衬衫.青袴.繋以锦绳.辂后四人.擎行马前.有朝服二人.执笏面辂倒行.是夜宿太庙.喝探警严如宿殿仪.至三更.车惊行事.执事皆宗室.宫架乐作.主上在殿上东南隅西面立.有一朱漆金字牌曰『皇帝位』.然后奉神主出室.亦奏中严外辨.逐室行礼毕.甲马仪仗车辂.番衮表出南薫门.

驾诣青城斋宫

驾御玉辂诣青城斋宫.所谓『青城』.旧来止以青布幕为之.画砌甃之文.旋结城阙殿宇.宣.政间悉用土木盖造矣.铁骑围斋宫外.诸军有紫巾绯衣素队约千余.罗布郊野.毎队军乐一火.行官巡检部领甲马来往巡逻.至夜严警喝探如前.

驾诣郊坛行礼

三更驾诣郊坛行礼.有三重壝墙.驾出青城.南行曲尺西去约一里许乃坛也.入外壝东门.至第二壝里面.南设一大幕次.谓之『大次』.更换祭服.平天冠.二十四旒.青衮龙服.中单朱鸟.纯玉佩.二中贵扶侍行至坛前.坛下又有一小幕殿.谓之『小次』.内有御座.坛高三层.七十二级.坛面方圆三丈许.有四踏道.正南曰午阶.东曰卯阶.西曰酉阶.北曰子阶.坛上设二黄褥.位北面南.曰『昊天上帝』.东南面曰『太祖皇帝』.惟两矮案上设礼料.有登歌道士十余人.列钟磬二架.余歌色及琴瑟之类.三五执事人而已.坛前设宫架乐.前列编钟玉磬.其架有如常乐.方响増其高大.编钟形稍褊上下两层.挂之架.两角缀以流苏.玉磬状如曲尺.繋其曲尖处.亦架之.上下两层挂之.次列数架大皷.或三或五.用不穿贯.立于架座上.又有大钟.曰景钟.曰节皷.有琴而长者.如筝而大者.截竹如箫管两头存节而横吹者.有土烧成如圆弹而开窍者.如笙而大者.如箫而増其管者.有歌者.其声清亮.非郑.卫之比.宫架前立两竿.乐工皆裹介帻如笼巾.绯寛杉.勒帛.二舞者.顶紫色冠.上有一横板.皂服朱裙履.乐作.初则文舞.皆手执一紫嚢.盛一笛管结带.武舞.一手轨短矟.一手执小牌.比文舞加数人.撃铜铙响环.又撃如铜灶突者.又两人共携一铜瓮就地撃者.舞者如撃刺.如乘云.如分手.皆舞容矣.乐作.先撃柷.以木为之.如方壷.画山水之状.毎奏乐撃之.内外共九下.乐止则撃敔.如伏虎.脊上如锯齿.一曲终.以彼竹刮之.礼直官奏请驾登坛.前导官皆躬身.侧引至坛止.惟大礼使登之.先正北一位拜.跪酒.殿中监东向一拜.进爵盏.再拜.兴.复诣正东一位.纔登坛而宫架声止.则坛上乐作.降坛则宫架乐复作.武舞上.复归小次.亚献终献上亦如前仪.当时燕越王为亚终献也.第二次登坛.乐作如初.跪酒毕.中书舎人读册.左右两人举册而跪读.降坛复归小次.亚终献如前.再登坛.进玉爵盏.皇帝饮福矣.亚终献毕.降坛.驾小次前立.明坛上礼料币帛玉册.由酉阶而下.南壝门外去坛百余歩.有燎炉.高丈许.诸物上台.一人点唱入炉焚之.坛三层.回踏道之间.有十二龛.祭十二宫神.内壝外祭百星.执事与陪祠官皆面北立班.宫架乐罢.皷吹未作.外内数十万众肃然.惟闻轻风环佩之声.一赞者喝曰.『赞一拜!』皆拜.礼毕.

郊毕驾回

驾自小次祭服还大次.惟近侍椽烛二百余条.列成围子.至大次更服衮冕.登大安辇.辇如玉辂而大.无轮.四垂大带.辇官服色.亦如侠路者.纔升辇.教坊在外壝东西排列.钧容直先奏乐.一甲士舞一曲破讫.教坊进口号.乐作.诸军队伍皷吹皆动.声震天地.回青城.天色未晓.百官常服入贺.赐茶酒毕.而法驾仪仗铁骑.皷吹入南薫门.御路数十里之间.起居幕次.贵家看棚.华彩鳞砌.略无空闲去处.

下赦

车驾登宣徳楼.楼前立大旗数口.内一口大者.与宣徳楼齐.谓之『盖天旗』.旗立御路中心不动.次一口稍小.随驾立.谓之『次黄龙』.青城.太庙.随逐立之.俗亦呼为盖天旗.亦设宫架.乐作.须臾.撃柝之声.旋立鶏竿.约高十数丈.竿尖有一大木盘.上有金鶏.口衔红幡子.书『皇帝万歳』字.盘底有彩索四条垂下.有四红巾者争先縁索而上.捷得金鶏红幡.则山呼谢恩讫.楼上以红绵索通门下一彩楼上.有金风衔赦而下.至彩楼上.而通事舎人得赦宣读.开封府大理寺排列罪人在楼前.罪人皆绯缝黄布衫.狱吏皆簪花鲜洁.闻皷声.疎枷放去.各山呼谢恩讫.楼下钧容直乐作.杂剧舞旋.御龙直装神鬼.斫眞刀倬刀.楼上百官赐茶酒.诸班直呈拽马队.六军归营.至日晡时礼毕.

驾还择日诣诸宫行谢

驾还内.择日诣景灵东西宫行恭谢之礼三日.第三日毕.即游幸别宫观或大臣私第.是月卖糍餻鹑兔方盛.

十二月

十二月.街市尽卖撒佛花.韭黄.生菜.兰芽.勃荷.胡桃.泽州饧.初八日.街巷中有僧尼三五人.作队念佛.以银铜沙罗或好盆器.坐一金铜或木佛像.浸以香水.杨枝洒浴.排门教化.诸大寺作浴佛会.并送七宝五味粥与门徒.谓之『腊八粥』.都人是日各家亦以果子杂料煮粥而食也.腊日.寺院送面油与门徒.却入疏教化上元灯油钱.闾巷家家互相遗送.是月景龙门预赏元夕于宝箓宫.一方灯火繁盛.二十四日交年.都人至夜请僧道看经.备酒果送神.烧合家替代钱纸.帖灶马于灶上.以酒糟涂抹灶门.谓之『醉司命』.夜于床底点灯.谓之『照虚耗』.此月虽无节序.而豪贵之家.遇雪即开筵.塑雪狮.装雪灯雪□.以会亲旧.近歳节.市井皆印卖门神.钟馗.桃板.桃符.及财门钝驴.回头鹿马.天行帖子.卖干茄瓠.马牙菜.胶牙饧之类.以备除夜之用.自入此日.即有贫者三数人为一火.装妇人神鬼.敲锣撃皷.巡门乞钱.俗呼为『打夜胡』.亦驱祟之道也.

除夕

至除日.禁中呈大傩仪.并用皇城亲事官.诸班直戴假面.繍画色衣.执金鎗龙旗.教坊使孟景初身品魁伟.贯全副金镀铜甲装将军.用鎭殿将军二人.亦介胄.装门神.教坊南河炭丑恶魁肥.装判官.又装钟馗.小妹.土地.灶神之类.共千余人.自禁中驱祟出南薫门外转龙弯.谓之『理祟』而罢.是夜禁中爆竹山呼.声闻于外.士庶之家.围炉团坐.达旦不寐.谓之『守歳』.

凡大礼与禁中节次.但尝见习按.又不知果为如何.不无脱略.或改而正之.则幸甚.

\end{document}
