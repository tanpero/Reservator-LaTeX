\PassOptionsToPackage{unicode=true}{hyperref} % options for packages loaded elsewhere
\PassOptionsToPackage{hyphens}{url}
%
\documentclass[]{article}
\usepackage{lmodern}
\usepackage{amssymb,amsmath}
\usepackage{ifxetex,ifluatex}
\usepackage{fixltx2e} % provides \textsubscript
\ifnum 0\ifxetex 1\fi\ifluatex 1\fi=0 % if pdftex
  \usepackage[T1]{fontenc}
  \usepackage[utf8]{inputenc}
  \usepackage{textcomp} % provides euro and other symbols
\else % if luatex or xelatex
  \usepackage{unicode-math}
  \defaultfontfeatures{Ligatures=TeX,Scale=MatchLowercase}
\fi
% use upquote if available, for straight quotes in verbatim environments
\IfFileExists{upquote.sty}{\usepackage{upquote}}{}
% use microtype if available
\IfFileExists{microtype.sty}{%
\usepackage[]{microtype}
\UseMicrotypeSet[protrusion]{basicmath} % disable protrusion for tt fonts
}{}
\IfFileExists{parskip.sty}{%
\usepackage{parskip}
}{% else
\setlength{\parindent}{0pt}
\setlength{\parskip}{6pt plus 2pt minus 1pt}
}
\usepackage{hyperref}
\hypersetup{
            pdfborder={0 0 0},
            breaklinks=true}
\urlstyle{same}  % don't use monospace font for urls
\setlength{\emergencystretch}{3em}  % prevent overfull lines
\providecommand{\tightlist}{%
  \setlength{\itemsep}{0pt}\setlength{\parskip}{0pt}}
\setcounter{secnumdepth}{0}
% Redefines (sub)paragraphs to behave more like sections
\ifx\paragraph\undefined\else
\let\oldparagraph\paragraph
\renewcommand{\paragraph}[1]{\oldparagraph{#1}\mbox{}}
\fi
\ifx\subparagraph\undefined\else
\let\oldsubparagraph\subparagraph
\renewcommand{\subparagraph}[1]{\oldsubparagraph{#1}\mbox{}}
\fi

% set default figure placement to htbp
\makeatletter
\def\fps@figure{htbp}
\makeatother


\date{}

\begin{document}

\hypertarget{header-n0}{%
\section{仪礼}\label{header-n0}}

\begin{center}\rule{0.5\linewidth}{\linethickness}\end{center}

\tableofcontents

\begin{center}\rule{0.5\linewidth}{\linethickness}\end{center}

\hypertarget{header-n10}{%
\subsection{士冠礼}\label{header-n10}}

士冠礼。筮于庙门。主人玄冠,朝服,缁带,素□,即位于门东,西面。有司如主人服,即位于西方,东面,北上。筮与席、所卦者,具馔于西塾。布席于门中,闑西阈外,西面。筮人执策,抽上韇,兼执之,进受命于主人。宰自右少退,赞命。筮人许诺,右还,即席坐,西面。卦者在左。卒筮,书卦,执以示主人。主人受□,反之。筮人还,东面,旅占,卒,进,告吉。若不吉,则筮远日,如初仪。彻筮席。宗人告事毕。

主人戒宾。宾礼辞,许。主人再拜,宾答拜。主人退,宾拜送。

前期三日,筮宾,如求日之仪。

乃宿宾。宾如主人服,出门左,西面再拜。主人东面答拜,乃宿宾。宾许,主人再拜,宾答拜。主人退,宾拜送。宿赞冠者一人,亦如之。

厥明夕,为期于庙门之外。主人立于门东,兄弟在其南,少退,西面,北上。有司皆如宿服,立于西方,东面,北上,摈者请期,宰告曰:「质明行事。」告兄弟及有司。告事毕。摈者告期于宾之家。

夙兴,设洗,直于东荣,南北以堂深,水在洗东。陈服于房中西墉下,东领,北上。爵弁服,纁裳,纯衣,缁带,韎韐。皮弁服:素积,缁带,素□。玄端,玄裳,黄裳、杂裳可也,缁带,爵□。缁布冠,缺项,青组缨,属于缺;缁纚,广终幅,长六尺。皮弁笄,爵弁笄,缁组紘,纁边,同箧。栉实于箪。蒲筵二,在南。侧尊一甒醴,在服北。有篚实勺、觯、角柶。脯醢,南上。爵弁、皮弁、缁布冠各一匴,执以待于西坫南,南面,东上。宾升则东面。

主人玄端爵□,立于阼阶下,直东序,西面。兄弟毕袗玄,立于洗东,西面,北上。摈者玄端,负东塾。将冠者采衣,紒,在房中,南面。宾如主人服,赞者玄端从之,立于外门之外。

摈者告。主人迎,出门左,西面,再拜。宾答拜。主人揖赞者,与宾揖,先入。每曲揖。至于庙门,揖入。三揖,至于阶,三让。主人升,立于序端,西面。宾西序,东面。赞者盥于洗西,升,立于房中,西面,南上。

主人之赞者筵于东序,少北,西面。将冠者出房,南面。赞者奠纚、笄、栉于筵南端。宾揖将冠者,将冠者即筵坐。赞者坐,栉,设纚。宾降,主人降。宾辞,主人对。宾盥,卒,壹揖,壹让,升。主人升,复初位。宾筵前坐,正纚,兴,降西阶一等。执冠者升一等,东面授宾。宾右手执项,左手执前,进容,乃祝,坐如初,乃冠,兴,复位。赞者卒。冠者兴,宾揖之。适房,服玄端爵□,出房,南面。

宾揖之,即筵坐。栉,设笄。宾盥、正纚如初,降二等,受皮弁,右执项,左执前,进、祝、加之如初,复位。赞者卒紘。兴,宾揖之。适房,服素积素□,容,出房,南面。

宾降三等,受爵弁,加之,服纁裳韎韐,其他如加皮弁之仪。

彻皮弁、冠、栉、筵入于房。筵于户西,南面。赞者洗于房中,侧酌醴;加柶,覆之,面叶。宾揖,冠者就筵,筵西,南面。宾授醴于户东,加柶,面枋,筵前北面。冠者筵西拜受觯,宾东面答拜。荐脯醢。冠者即筵坐,左执觯,右祭脯醢,以柶祭醴三,兴;筵末坐,啐醴,建柶,兴;降筵,坐奠觯,拜;执觯兴。宾答拜。

冠者奠觯于荐东,降筵;北面坐取脯;降自西阶,适东壁,北面见于母。母拜受,子拜送,母又拜。

宾降,直西序,东面。主人降,复初位。冠者立于西阶东,南面。宾字之,冠者对。

宾出主人送于庙门外。请醴宾,宾礼辞,许。宾就次。冠者见于兄弟,兄弟再拜,冠者答拜。见赞者,西面拜,亦如之。入见姑、姊,如见母。

乃易服,服玄冠、玄端、爵□,奠挚见于君。遂以挚见于乡大夫、乡先生。

乃醴宾,以一献之礼。主人酬宾,束帛、俪皮。赞者皆与。赞冠者为介。

宾出,主人送于外门外,再拜;归宾俎。

若不醴,则醮用酒。尊于房户之间,两甒,有禁,玄酒在西,加勺,南枋。洗,有篚在西,南顺。始加,醮用脯醢;宾降,取爵于篚,辞降如初;卒洗,升酌。冠者拜受,宾答拜如初。冠者升筵,坐;左执爵,右祭脯醢,祭酒,兴;筵末坐,啐酒;降筵,拜。宾答拜。冠者奠爵于荐东,立于筵西。彻荐、爵,筵尊不彻。加皮弁,如初仪;再醮,摄酒,其他皆如初。加爵弁,如初仪;三醮,有乾肉折俎,哜之,其他如初。北面取脯,见于母。若杀,则特豚,载合升,离肺实于鼎,设扃鼏。始醮,如初。再醮,两豆,葵菹、蠃醢;两笾,栗、脯。三醮,摄酒如再醮,加俎,哜之,皆如初,哜肺。卒醮,取笾脯以降,如初。

若孤子,则父兄戒、宿。冠之日,主人紒而迎宾,拜,揖,让,立于序端,皆如冠主;礼于阼。凡拜,北面于阼阶上,宾亦北面于西阶上答拜。若杀,则举鼎陈于门外,直东塾,北面。

若庶子,则冠于房外,南面,遂醮焉。

冠者母不在,则使人受脯于西阶下。

戒宾,曰:「某有子某。将加布于其首,愿吾子之教之也。」宾对曰:「某不敏,恐不能共事,以病吾子,敢辞。」主人曰:「某犹愿吾子之终教之也!」宾对曰:「吾子重有命,某敢不从?」宿,曰:「某将加布于某之首,吾子将莅之,敢宿。」宾对曰:「某敢不夙兴?」

始加,祝曰:「令月吉日,始加元服。弃尔幼志,顺尔成德。寿考惟祺,介尔景福。」再加,曰:「吉月令辰,乃申尔服。敬尔威仪,淑慎尔德。眉寿万年,永受胡福。」三加,曰:「以岁之正,以月之令,咸加尔服。兄弟具在,以成厥德。黄耇无疆,受天之庆。」

醴辞曰:「甘醴惟厚,嘉荐令芳。拜受祭之,以定尔祥。承天之休,寿考不忘。」

醮辞曰:「旨酒既清,嘉荐亶时。始加元服,兄弟具来。孝友时格,永乃保之。」再醮,曰:「旨酒既湑,嘉荐伊脯。乃申尔服,礼仪有序。祭此嘉爵,承天之祜。」三醮,曰:「旨酒令芳,笾豆有楚。咸加尔服,肴升折俎。承天之庆,受福无疆。」

字辞曰:「礼仪既备,令月吉日,昭告尔字。爰字孔嘉,髦士攸宜。宜之于假,永受保之,曰伯某甫。」仲、叔、委,唯其所当。

屦,夏用葛。玄端黑屦,青絇繶纯,纯博寸。素积白屦,以魁柎之,缁絇繶纯,纯博寸。爵弁纁屦,黑絇繶纯,纯博寸。冬,皮屦可也。不屦繐屦。

记。冠义:始冠,缁布之冠也。太古冠布,齐则缁之。其緌也,孔子曰:「吾未之闻也,冠而敝之可也。」适子冠于阼,以着代也。醮于客位,加有成也。三加弥尊,谕其志也。冠而字之,敬其名也。委貌,周道也。章甫,殷道也。毋追,夏后氏之道也。周弁。殷冔。夏收。三王共皮弁素积。无大夫冠礼,而有其昏礼。古者五十而后爵,何大夫冠礼之有?公侯之有冠礼也,夏之末造也。天子之元子,犹士也,天下无生而贵者也。继世以立诸侯,像贤也。以官爵人,德之杀也。死而谥,今也。古者生无爵,死无谥。

\hypertarget{header-n16}{%
\subsection{士昏礼}\label{header-n16}}

昏礼。下达。纳采,用雁。主人筵于户西,西上,右几。使者玄端至。摈者出请事,入告。主人如宾服,迎于门外,再拜,宾不答拜。揖入。至于庙门,揖入;三揖,至于阶,三让。主人以宾升,西面。宾升西阶。当阿,东面致命。主人阼阶上北面再拜;授于楹间,南面。宾降,出。主人降,授老雁。摈者出请。宾执雁,请问名,主人许。宾入,授,如初礼。摈者出请,宾告事皆。入告,出请醴宾。宾礼辞,许。主人彻几,改筵,东上。侧尊甒醴于房中。主人迎宾于庙门外,揖让如初,升。主人北面,再拜,宾西阶上北面答拜。主人拂几授校,拜送。宾以几辟,北面设于坐,左之,西阶上答拜。赞者酌醴,加角柶,面叶,出于房。主人受醴,面枋,筵前西北面。宾拜受醴,复位。主人阼阶上拜送。赞者荐脯醢。宾即筵坐,左执觯,祭脯醢,以柶祭醴三,西阶上北面坐,啐醴,建柶,兴,坐奠觯,遂拜。主人答拜。宾即筵,奠于荐左,降筵,北面坐取脯;主人辞。宾降,授人脯,出。主人送于门外,再拜。

纳吉用雁,如纳采礼。

纳征:玄纁束帛,俪皮。如纳吉礼。

请期,用雁。主人辞。宾许,告期,如纳征礼。

期,初昏,陈三鼎于寝门外东方,北面,北上。其实特豚,合升,去蹄。举肺脊二、祭肺二、鱼十有四、腊一肫。髀不升。皆饪。设扃鼏。设洗于阼阶东南。馔于房中:醯酱二豆,菹醢四豆,兼巾之:黍稷四敦,皆盖。大羹湆在爨。尊于室中北墉下,有禁,玄酒在西,綌幂,加勺,皆南枋。尊于房户之东,无玄酒,篚在南,实四爵合卺。

主人爵弁,纁裳缁袘。从者毕玄端。乘墨车,从车二乘,执烛前马。妇车亦如之,有示炎。至于门外。主人筵于户西,西上,右几。女次,纯衣纁袡,立于房中,南面。姆纚笄宵衣,在其右。女从者毕袗玄,纚笄,被纚黼,在其后。主人玄端迎于门外,西面再拜,宾东面答拜。主人揖入,宾执雁从。至于庙门,揖入。三揖,至于阶,三让。主人升,西面。宾升,北面,奠雁,再拜稽首,降,出。妇从,降自西阶。主人不降送。婿御妇车,授绥,姆辞不受。妇乘以几,姆加景,乃驱。御者代。婿乘其车先,俟于门外。

妇至,主人揖妇以入。乃寝门,揖入,升自西阶,媵布席于奥。夫入于室,即席,妇尊西,南面。媵御沃盥交。赞者彻尊幂。举者盥,出,除\{曰鼎\},举鼎入,陈于阼阶南,西面,北上。匕俎从设,北面载,执而俟。匕者逆退,复位于门东,北面,西上。赞者设酱于席前,菹醢在其北。俎入,设于豆东。鱼次。腊特于俎北。赞设黍于酱东,稷在其东。设湆于酱南。设对酱于东,菹醢在其南,北上。设黍于腊北,其西稷。设湆于酱北。御布对席,赞启会,却于敦南,对敦于北。赞告具。揖妇,即对筵,皆坐。皆祭,祭荐、黍、稷、肺。赞尔黍,授肺脊,皆食,以湆酱,皆祭举、食举也。三饭,卒食。赞洗爵,酌酳主人,主人拜受,赞户内北面答拜。酳妇亦如之。皆祭。赞以肝从,皆振祭。哜肝,皆实于菹豆。卒爵,皆拜。赞答拜,受爵,再酳如初,无从,三酳用卺,亦如之。赞洗爵,酌于户外尊,入户,西北面奠爵,拜。皆答拜。坐祭,卒爵,拜。皆答拜。兴。主人出,妇复位。乃彻于房中,如设于用室,尊否。主人说服于房,媵受;妇说服于室,御受。姆授巾。御衽于奥,媵衽良席在东,皆有枕,北止。主人入,亲说妇之缨。烛出。媵餕主人之馀,御餕妇余,赞酌外尊酳之。媵侍于户外,呼则闻。

夙兴,妇沐浴,纚笄、宵衣以俟见。质明,赞见妇于舅姑。席于阼,舅即席。席于房外,南面,姑即席。妇执□枣、栗,自门入,升自西阶,进拜,奠于席。舅坐抚之,兴,答拜。妇还,又拜,降阶,受□腶脩,升,进,北面拜,奠于席。姑坐举以兴,拜,授人。

赞醴妇。席于户牖间,侧尊甒醴于房中。妇疑立于席西。赞者酌醴,加柶,面枋,出房,席前北面。妇东面拜受。赞西阶上北面拜送。妇又拜。荐脯醢。妇升席,左执觯,右祭脯醢,以柶祭醴三,降席,东面坐,啐醴,建柶,兴,拜。赞答拜。妇又拜,奠于荐东,北面坐取脯;降,出,授人于门外。

舅姑入于室,妇盥馈。特豚,合升,侧载,无鱼腊,无稷。并南上。其他如取女礼。妇赞成祭,卒食,一酳,无从。席于北墉下。妇撤,设席前如初,西上。妇餕,舅辞,易酱。妇餕姑之馔,御赞祭豆、黍、肺、举肺、脊,乃食,卒。姑酳之,妇拜受,姑拜送。坐祭,卒爵,姑受,奠之。妇撤于房中,媵御餕,姑酳之,虽无娣,媵先。于是与始饭之错。

舅姑共飨妇以一献之礼。舅洗于南洗,姑洗于北洗,奠酬。舅姑先降自西阶,妇降自阼阶。归妇俎于妇氏人。

舅飨送者以一献之礼,酬以束锦。姑飨妇人送者,酬以束锦。若异邦,则赠丈夫送者以束锦。

若舅姑既没,则妇入三月,乃奠菜。席于庙奥,东面,右几。席于北方,南面。祝盥,妇盥于门外。妇执□菜,祝帅妇以入。祝告,称妇之姓,曰:「某氏来妇,敢奠嘉菜于皇舅某子。」妇拜扱地,坐奠菜于几东席上,还,又拜如初。妇降堂,取□菜,入,祝曰:「某氏来妇,敢告于皇姑某氏。」奠菜于席,如初礼。妇出,祝阖牖户。老醴妇于房中,南面,如舅姑醴妇之礼。婿飨妇送者丈夫、妇人,如舅姑飨礼。

记士昏礼,凡行事必用昏昕,受诸祢庙,辞无不腆,无辱。挚不用死,皮帛必可制。腊必用鲜,鱼用鲋,必殽全。女子许嫁,笄而醴之,称字。祖庙未毁,教于公宫,三月。若祖庙已毁,则教于宗室。问名。主人受雁,还,西面对。宾受命乃降。祭醴,始扱一祭,又扱再祭。宾右取脯,左奉之;乃归,执以反命。纳征:执皮,摄之,内文;兼执足,左首;随入,西上;参分庭一,在南。宾致命,释外足,见文。主人受币,士受皮者自东出于后,自左受,遂坐摄皮。逆退,适东壁。

父醴女而俟迎者,母南面于房外。女出于母左,父西面戒之,必有正焉。若衣,若笄,母戒诸西阶上,不降。妇乘以几,从者二人坐持几,相对。妇入寝门,赞者彻尊幂,酌玄酒,三属于尊,弃余水于堂下阶间,加勺。□,缁被纁里,加于桥。舅答拜,宰彻□。

妇席荐馔于房。飨妇,姑荐焉。妇洗在北堂,直室东隅;篚在东,北面盥。妇酢舅,更爵,自荐;不敢辞洗,舅降则辟于房;不敢拜洗。凡妇人相飨,无降。

妇入三月,然后祭行。

庶妇,则使人醮之。妇不馈。

昏辞曰:「吾子有惠,贶室某也。某有先人之礼,使某也请纳采。」对曰:「某之子舂愚,又弗能教。吾子命之,某不敢辞。」致命,曰:「敢纳采。」

问名,曰:「某既受命,将加诸卜,敢请女为谁氏?」对曰:「吾子有命,且以备数而择之,某不敢辞。」

醴,曰:「子为事故,至于某之室。某有先人之礼,请醴从者。」对曰:「某既得将事矣,敢辞。」「先人之礼,敢固以请。」「某辞不得命,敢不从也?」

纳吉,曰:「吾子有贶命,某加诸卜,占曰『吉』。使某也敢告。」对曰:「某之子不教,唯恐弗堪。子有吉,我与在。某不敢辞。」

纳征,曰:「吾子有嘉命,贶室某也。某有先人之礼,俪皮束帛,使某也请纳征。」致命,曰:「某敢纳征。」对曰:「吾子顺先典,贶某重礼,某不敢辞,敢不承命?」

请期,曰:「吾子有赐命,某既申受命矣。惟是三族之不虞,使某也请吉日。」对曰:「某既前受命矣,唯命是听。」曰:「某命某听命于吾子。」对曰:「某固唯命是听。」使者曰:「某使某受命,吾子不许,某敢不告期?」曰某日。对曰:「某敢不敬须?」

凡使者归,反命,曰:「某既得将事矣,敢以礼告。」主人曰:「闻命矣。」

父醮子,命之,曰:「往迎尔相,承我宗事。勖帅以敬,先妣之嗣。若则有常。」子曰:「诺。唯恐弗堪,不敢忘命。」

宾至摈者请,对曰:「吾子命某,以兹初昏,使某将,请承命。」对曰:「某固敬具以须。」

父送女,命之曰:「戒之敬之,夙夜毋违命!」母施衿结帨,曰:「勉之敬之,夙夜无违宫事!」庶母及门内,施鞶,申之以父母之命,命之曰:「敬恭听,宗尔父母之言。夙夜无愆,视诸衿鞶!」婿授绥,姆辞曰:「未教,不足与为礼也。」

宗子无父,母命之。亲皆没,己躬命之。支子,则称其宗。弟,则称其兄。

若不亲迎,则妇入三月,然后婿见,曰:「某以得为外昏姻,请觌。」主人对曰:「某以得为外昏姻之数,某之子未得濯溉于祭祀,是以未敢见。今吾子辱,请吾子之就宫,某将走见。」对曰:「某以非他故,不足以辱命,请终赐见。」对曰:「某得以为昏姻之故,不敢固辞,敢不从!」主人出门左,西面。婿入门,东面,奠挚,再拜,出。摈者以挚出,请受。婿礼辞,许,受挚,入。主人再拜受,婿再拜送,出。见主妇,主妇阖扉,立于其内。婿立于门外,东面。主妇一拜。婿答再拜,主妇又拜,婿出。主人请醴,及揖让入。醴以一献之礼。主妇荐,奠酬,无币。婿出,主人送,再拜。

\hypertarget{header-n20}{%
\subsection{士相见礼}\label{header-n20}}

士相见之礼。挚,冬用雉,夏用腒。左头奉之,曰:「某也愿见,无由达。某子以命命某见。」主人对曰:「某子命某见,吾子有辱。请吾子之就家也,某将走见。」宾对曰:「某不足以辱命,请终赐见。」主人曰:「某不敢为仪,固请吾子之就家也,某将走见。」宾对曰:「某不敢为仪,固以请。」主人对曰:「某也固辞,不得命,将走见。闻吾子称挚,敢辞挚。」宾对曰:「某不以挚,不敢见。」主人对曰:「某不足以习礼,敢固辞。」宾对曰:「某也不依于挚,不敢见,固以请。」主人对曰:「某也固辞,不得命,敢不敬从!」出迎于门外,再拜。客答再拜。主人揖,入门右。宾奉挚,入门左。主人再拜受,宾再拜送挚,出。主人请见,宾反见,退。主人送于门外,再拜。

主人复见之,以其挚,曰:「曏者吾子辱,使某见。请还挚于将命者。」主人对曰:「某也既得见矣,敢辞。」宾对曰:「某也非敢求见,请还挚于将命者。」主人对曰:「某也既得见矣,敢固辞。」宾对曰:「某不敢以闻,固以请于将命者。」主人对曰:「某也固辞,不得命,敢不从?」宾奉挚入,主人再拜受。宾再拜送挚,出。主人送于门外,再拜。

士见于大夫,终辞其挚。于其入也,一拜其辱也。宾退,送,再拜。

若尝为臣者,则礼辞其挚,曰:「某也辞,不得命,不敢固辞。」宾入,奠挚,再拜,主人答壹拜,宾出。使摈者还其挚于门外,曰:「某也使其还挚。」宾对曰:「某也既得见矣,敢辞。」摈者对曰:「某也命某:『某非敢为仪也。』敢以请。」宾对曰:「某也,夫子之贱私,不足以践礼,敢固辞!」摈者对曰:「某也使某,不敢为仪也,固以请!」宾对曰:「某固辞,不得命,敢不从?」再拜受。

下大夫相见以雁,饰之以布,维之以索,如执雉。上大夫相见以羔,饰之以布,四维之,结于面;左头,如麛执之。如士相见之礼。

始见于君执挚,至下,容弥蹙。庶人见于君,不为容,进退走。士大夫则奠挚,再拜稽首;君答壹拜。若他邦之人,则使摈者还其挚,曰:「寡君使某还挚。」宾对曰:「君不有其外臣,臣不敢辞。」再拜稽首,受。

凡燕见于君,必辩君之南面。若不得,则正方,不疑君。君在堂,升见无方阶,辩君所在。

凡言,非对也,妥而后传言。与君言,言使臣。与大人言,言事君。与老者言,言使弟子。与幼者言,言孝弟于父兄。与众言,言忠信慈祥。与居官者言,言忠信。凡与大人言,始视面,中视抱,卒视面,毋改。众皆若是。若父,则游目,毋上于面,毋下于带。若不言,立则视足,坐则视膝。

凡侍坐于君子,君子欠伸,问日之早晏,以食具告,改居,则请退可也。夜侍坐,问夜,膳荤,请退可也。

若君赐之食,则君祭先饭,遍尝膳,饮而俟,君命之食,然后食。若有将食者,则俟君之食,然后食。若君赐之爵,则下席,再拜稽首,受爵,升席祭,卒爵而俟,君卒爵,然后授虚爵。退,坐取屦,隐辟而后屦。君为之兴,则曰:「君无为兴,臣不敢辞。」君若降送之。则不敢顾辞,遂出。大夫则辞,退下,比及门三辞。

若先生异爵者请见之,则辞。辞不得命,则曰:「某无以见,辞不得命,将走见。」先见之。

非以君命使,则不称寡。大夫士,则曰寡君之老。凡执币者,不趋,容弥蹙以为仪。执玉者,则唯舒武,举前曳踵。凡自称于君,士大夫则曰下臣。宅者在邦,则曰市井之臣;在野,则曰草茅之臣,庶人则曰刺草之臣。他国之人则曰外臣。

\hypertarget{header-n24}{%
\subsection{乡饮酒礼}\label{header-n24}}

乡饮酒之礼。主人就先生而谋宾、介。主人戒宾,宾拜辱;主人答拜,乃请宾。宾礼辞,许。主人再拜,宾答拜。主人退,宾拜辱。介亦如之。

乃席宾、主人、介、众宾之席,皆不属焉。尊两壶于房户间,斯禁,有玄酒,在西。设篚于禁南,东肆,加二勺于两壶。设洗于阼阶东南,南北以堂深,东西当东荣。水在洗东,篚在洗西,南肆。

羹定,主人速宾,宾拜辱,主人答拜。还,宾拜辱。介亦如之。宾及众宾皆从之。主人一相迎于门外,再拜宾,宾答拜;拜介,介答拜;揖众宾。主人揖,先入。宾厌介,入门左;介厌众宾,入;众宾皆入门左;北上。主人与宾三揖,至于阶,三让。主人升,宾升。主人阼阶上当楣北面再拜。宾西阶上当楣北面答拜。

主人坐取爵于篚,降洗。宾降。主人坐奠爵于阶前,辞。宾对。主人坐取爵,兴,适洗,南面坐,奠爵于篚下,盥洗,宾进东,北面辞洗。主人坐奠爵于篚,兴对。宾复位,当西序,东面。主人坐取爵,沃洗者西北面。卒洗,主人壹揖,壹让。升。宾拜洗。主人坐奠爵,遂拜。降盥。宾降,主人辞;宾对,复位,当西序。卒盥,揖让升。宾西阶上疑立。主人坐取爵,实之宾之席前,西北面献宾。宾西阶上拜,主人少退。宾进受爵,以复位。主人阼阶上拜送爵,宾少退。荐脯醢。宾升席,自西方。乃设折俎。主人阼阶东疑立。宾坐,左执爵,祭脯醢,奠爵于荐西,兴;右手取肺,却左手执本,坐,弗缭,右绝末以祭,尚左手,哜之,兴;加于俎,坐梲手,遂祭酒,兴;席末坐,啐酒,降席,坐奠爵,拜,告旨,执爵兴。主人阼阶上答拜。宾西阶上北面坐,卒爵,兴;坐奠爵,遂拜,执爵兴。主人阼阶上答拜。

宾降洗,主人降。宾坐奠爵,兴辞,主人对。宾坐取爵,适洗南,北面。主人阼阶东,南面辞洗。宾坐奠爵于篚,兴对。主人复阼阶东,西面。宾东北面盥,坐取爵,卒洗,揖让如初,升。主人拜洗。宾答拜,兴,降盥,如主人礼。宾实爵主人之席前,东南面酢主人。主人阼阶上拜,宾少退。主人进受爵,复位,宾西阶上拜送爵。荐脯醢。主人升席自北方。设折俎。祭如宾礼,不告旨。自席前适阼阶上,北面坐卒爵,兴,坐奠爵,遂拜,执爵兴。宾西阶上答拜。主人坐奠爵于序端,阼阶上北面再拜崇酒。宾西阶上答拜。

主人坐取觯于篚,降洗。宾降,主人辞降。宾不辞洗,立当西序,东面。卒洗,揖让升。宾西阶上疑立。主人实觯酬宾,阼阶上北面坐奠觯,遂拜,执觯兴。宾西阶上答拜。坐祭,遂饮,卒觯,兴;坐奠觯,遂拜,执觯兴。宾西阶上答拜。主人降洗;宾降辞,如献礼,升,不拜洗。宾西阶上立;主人实觯宾之席前,北面;宾西阶上拜;主人少退,卒拜进,坐奠觯于荐西;宾辞,坐取觯,复位;主人阼阶上拜送;宾北面坐奠觯于荐东,复位。

主人揖,降。宾降立于阶西,当序,东面。主人以介揖让升,拜如宾礼。主人坐取爵于东序端,降洗;介降,主人辞降;介辞洗,如宾礼,升,不拜洗。介西阶上立。主人实爵介之席前,西南面献介。介西阶上北面拜,主人少退;介进,北面受爵,复位。主人介右北面拜送爵,介少退。主人立于西阶东。荐脯醢。介升席自北方,设折俎。祭如宾礼,不哜肺,不啐酒,不告旨,自南方降席,北面坐卒爵,兴,坐奠爵,遂拜,执爵兴。主人介右答拜。

介降洗,主人复阼阶,降辞如初。卒洗,主人盥。介揖让升,授主人爵于两楹之间。介西阶上立。主人实爵,酢于西阶上,介右坐奠爵,遂拜,执爵兴。介答拜。主人坐祭,遂饮,卒爵,兴;坐奠爵,遂拜,执爵兴。介答拜。主人坐奠爵于西楹南,介右再拜崇酒;介答拜。

主人复阼阶,揖降,介降立于宾南。主人西南面三拜众宾,众宾皆答壹拜。主人揖升,坐取爵于西楹下;降洗,升实爵,于西阶上献众宾。众宾之长升拜受者三人,主人拜送。坐祭,立饮,不拜既爵;授主人爵,降复位。众宾献,则不拜受爵,坐祭,立饮。每一人献,则荐诸其席。众宾辩有脯醢。主人以爵降,奠于篚。

揖让升,宾厌介升,介厌众宾升,众宾序升,即席。一人洗,升,举觯于宾。实觯,西阶上坐奠觯,遂拜,执觯兴,宾席末答拜;坐祭,遂饮,卒觯,兴,坐奠觯,遂拜,执觯兴,宾答拜。降洗,升,实觯,立于西阶上宾拜;进坐奠觯于荐西,宾辞,坐受以兴。举觯者西阶上拜送,宾坐奠觯于所。举觯者降。

设席于堂廉,东上。工四人,二瑟,瑟先。相者二人,皆左何瑟,后首,挎越,内弦,右手相。乐正先升,立于西阶东。工入,升自西阶。北面坐。相者东面坐,遂授瑟,乃降。工歌《鹿鸣》、《四牡》、《皇皇者华》。卒歌,主人献工。工左瑟,一人拜,不兴,受爵。主人阼阶上拜送爵。荐脯醢。使人相祭。工饮,不拜既爵,授主人爵。众工则不拜,受爵,祭,饮辩有脯醢,不祭。大师则为之洗。宾、介降,主人辞降。工不辞洗。

笙入堂下,磬南,北面立,乐《南陔》、《白华》、《华黍》。主人献之于西阶上。一人拜,尽阶,不升堂,受爵,主人拜送爵。阶前坐祭,立饮,不拜既爵,升授主人爵。众笙则不拜,受爵,坐祭,立饮;辩有脯醢,不祭。

乃间歌《鱼丽》,笙《由庚》;歌《南有嘉鱼》,笙《崇丘》;歌《南山有台》,笙《由仪》。

乃合乐:《周南·关雎》、《葛覃》、《卷耳》,《召南·鹊巢》、《采蘩》、《采苹》。工告于乐正曰:「正歌备。」乐正告于宾,乃降。

主人降席自南方,侧降;作相为司正。司正礼辞,许诺。主人拜,司正答拜。主人升,复席。司正洗觯,升自西阶,阼阶上北面受命于主人。主人曰:「请安于宾。」司正告于宾,宾礼辞,许。司正告于主人。主人阼阶上再拜,宾西阶上答拜。司正立于楹间以相拜,皆揖,复席。

司正实觯,降自西阶,阶间北面坐奠觯;退共,少立;坐取觯,不祭,遂饮,卒觯兴,坐奠觯,遂拜;执觯兴,盥洗;北面坐奠觯于其所,退立于觯南。

宾北面坐取俎西之觯,阼阶上北面酬主人。主人降席,立于宾东。宾坐奠觯,遂拜,执觯兴,主人答拜。不祭,立饮,不拜,卒觯,不洗,实觯,东南面授主人。主人阼阶上拜,宾少退。主人受觯,宾拜送于主人之西。宾揖,复席。

主人西阶上酬介。介降席自南方,立于主人之西,如宾酬主人之礼。主人揖,复席。

司正升相旅,曰:「某子受酬。」受酬者降席。司正退立于序端,东面。受酬者自介右,众受酬者受自左,拜、兴、饮,皆如宾酬主人之礼。辩,卒受者以觯降,坐奠于篚。司正降,复位。

使二人举觯于宾、介,洗,升,实觯于西阶上,皆坐奠觯,遂拜,执觯兴。宾、介席末答拜。皆坐祭,遂饮,卒觯兴,坐奠觯,遂拜,执觯兴,宾、介席末答拜。逆降,洗,升,实觯,皆立于西阶上。宾、介皆拜。皆进,荐西奠之,宾辞,坐取觯以兴。介则荐南奠之,介坐受以兴。退,皆拜送,降。宾、介奠于其所。

司正升自西阶,受命于主人。主人曰:「请坐于宾。」宾辞以俎。主人请彻俎,宾许。司正降阶前,命弟子俟彻俎。司正升,立于序端。宾降席,北面。主人降席,阼阶上北面。介降席,西阶上北面。遵者降席,席东南面。宾取俎,还授司正;司正以降,宾从之。主人取俎,还授弟子;弟子以降自西阶,主人降自阼阶。介取俎,还授弟子;弟子以降,介从之。若有诸公、大夫,则使人受俎,如宾礼。众宾皆降。

说屦,揖让如初,升,坐。乃羞。无算爵。无算乐。

宾出,奏《陔》。主人送于门外,再拜。

宾若有遵者:诸公、大夫,则既一人,举觯,乃入。席于宾东,公三重,大夫再重。公如大夫,入,主人降,宾、介降,众宾皆降,复初位。主人迎,揖让升。公升如宾礼,辞一席,使一人去之。大夫则如介礼,有诸公,则辞加席,委于席端,主人不彻;无诸公,则大夫辞加席,主人对,不去加席。

明日,宾服乡服以拜赐,主人如宾服以拜辱。主人释服,乃息司正。无介,不杀,荐脯醢,羞唯所有。征唯所欲,以告于先生、君子可也。宾、介不与。乡乐唯欲。

记。乡,朝服而谋宾、介,皆使能,不宿戒。蒲筵,缁布纯。尊綌幂,宾至彻之。其牲,狗也。亨于堂东北。献用爵,其他用觯。荐脯,五挺,横祭于其上,出自左房。俎由东壁,自西阶升。宾俎,脊、胁、肩、肺。主人俎,脊、胁、臂、肺。介俎,脊、胁、肫、胳、肺。肺皆离。皆右体,进腠。以爵拜者不徒作。坐卒爵者拜既爵,立卒爵者不拜既爵。凡奠者于左,将举于右。众宾之长,一人辞洗,如宾礼。立者东面北上;若有北面者,则东上。乐正与立者,皆荐以齿。凡举爵,三作而不徒爵。乐作,大夫不入。献工与笙,取爵于上篚;既献,奠于下篚。其笙,则献诸西阶上;磬,阶间缩霤,北面鼓之。主人、介,凡升席自北方,降自南方。司正,既举觯而荐诸其位。凡旅,不洗。不洗者,不祭。既旅,士不入。彻俎:宾、介,遵者之俎,受者以降,遂出授从者;主人之俎,以东。乐正命奏《陔》,宾出,至于阶,《陔》作。若有诸公,则大夫于主人之北,西面。主人之赞者,西面北上,不与,无算爵,然后与。

\hypertarget{header-n28}{%
\subsection{乡射礼}\label{header-n28}}

乡射之礼。主人戒宾,宾出迎,再拜。主人答再拜,乃请。宾礼辞,许。主人再拜,宾答再拜。主人退;宾送,再拜。无介。

乃席宾,南面,东上。众宾之席,继而西。席主人于阼阶上,西面。尊于宾席之东,两壶,斯禁,左玄酒,皆加勺。篚在其南,东肆。设洗于阼阶东南,南北以堂深,东西当东荣。水在洗东,篚在洗西,南肆。县于洗东北,西面。乃张侯,下纲不及地武。不系左下纲,中掩束之。乏参侯道,居侯党之一,西五步。

羹定。主人朝服,乃速宾;宾朝服出迎,再拜;主人答再拜,退;宾送,再拜。宾及众宾遂从之。

及门,主人一相出迎于门外,再拜;宾答再拜。揖众宾。主人以宾揖,先入。宾厌众宾,众宾皆入门左,东面北上。宾少进,主人以宾三揖,皆行。及阶,三让,主人升一等,宾升。主人阼阶上当楣北面再拜,宾西阶上当楣北面答再拜。

主人坐取爵于上篚,以降。宾降。主人阼阶前西面坐奠爵,兴辞降。宾对。主人坐取爵,兴,适洗,南面坐奠爵于篚下,盥洗。宾进,东北面辞洗。主人坐奠爵于篚,兴对,宾反位。主人卒洗,壹揖,壹让,以宾升。宾西阶上北面拜洗。主人阼阶上北面奠爵,遂答拜,乃降。宾降,主人辞降,宾对。主人卒盥,壹揖壹让升;宾升,西阶上疑立。主人坐取爵,实之宾席之前,西北面献宾。宾西阶上北面拜,主人少退。宾进受爵于席前,复位。主人阼阶上拜送爵,宾少退。荐脯醢。宾升席,自西方。乃设折俎。主人阼阶东疑立。宾坐,左执爵,右祭脯醢,奠爵于荐西,兴取肺,坐,绝祭,尚左手,哜之,兴,加于俎,坐梲手,执爵,遂祭酒,兴,席末坐啐酒,降席,坐尊爵,拜,告旨,执爵兴。主人阼阶上答拜。宾西阶上北面坐卒爵,兴,坐奠爵,遂拜,执爵兴。主人阼阶上答拜。

宾以虚爵降。主人降。宾西阶前东面坐奠爵,兴,辞降;主人对。宾坐取爵,适洗,北面坐奠爵于篚下,兴,盥洗。主人阼阶之东,南面辞洗。宾坐奠爵于篚,兴对。主人反位。宾卒洗,揖让如初,升。主人拜洗,宾答拜,兴,降盥,如主人之礼。宾升,实爵主人之席前,东南面酢主人。主人阼阶上拜,宾少退。主人进受爵,复位,宾西阶上拜送爵。荐脯醢。主人升席自北方。乃设折俎。祭如宾礼,不告旨,自席前适阼阶上,北面坐卒爵,兴,坐奠爵,遂拜,执爵兴。宾西阶上北面答拜。主人坐奠爵于序端,阼阶上再拜崇酒,宾西阶上答再拜。

主人坐取觯于篚,以降。宾降,主人奠觯辞降,宾对,东面立。主人坐取觯,洗,宾不辞洗。卒洗,揖让升。宾西阶上疑立。主人实觯,酬之,阼阶上北面坐奠觯,遂拜,执觯兴。宾西阶上北面答拜。主人坐祭,遂饮,卒觯,兴,坐奠觯,遂拜,执觯兴。宾西阶上北面答拜。主人降洗。宾降辞,如献礼,升,不拜洗。宾西阶上立。主人实觯宾之席前,北面。宾西阶上拜。主人坐奠觯于荐西。宾辞,坐取觯以兴,反位。主人阼阶上拜送。宾北面坐奠觯于荐东,反位。

主人揖降。宾降,东面立于西阶西,当西序。主人西南面三拜众宾,众宾皆答一拜。主人揖升,坐取爵于序端,降洗;升实爵,西阶上献从宾。众宾之长升拜受者三人,主人拜送。坐祭,立饮,不拜;既爵,授主人爵;降复位。众宾皆不拜,受爵,坐祭,立饮。每一人献,则荐诸其席。众宾辩有脯醢。主人以虚爵降,奠于篚。

揖让升。宾厌众宾升,众宾皆升,就席。一人洗,举觯于宾;升实觯,西阶上坐奠觯;拜,执觯兴。宾席末答拜。举觯者坐祭,遂饮,卒觯,兴;坐奠觯,拜,执觯兴;宾答拜。降洗,升实之,西阶上北面。宾拜。举觯者进,坐奠觯于荐西。宾辞,坐取以兴,举觯者西阶上拜送。宾反奠于其所。举觯者降。

大夫若有遵者,则入门左。主人降。宾及众宾皆降,复初位。主人揖让,以大夫升,拜至,大夫答拜。主人以爵降,大夫降。主人辞降。大夫辞洗,如宾礼,席于尊东。升,不拜洗。主人实爵,席前献于大夫。大夫西阶上拜,进受爵,反位。主人大夫之右拜送。大夫辞加席。主人对,不去加席。乃荐脯醢。大夫升席。设折俎。祭如宾礼,不哜肺,不啐酒,不告旨,西阶上卒爵,拜。主人答拜。大夫降洗,主人复阼阶,降辞如初。卒洗。主人盥,揖让升。大夫授主人爵于两楹间,复位。主人实爵,以酢于西阶上,坐奠爵,拜,大夫答拜。坐祭,卒爵,拜,大夫答拜。主人坐奠爵于西楹南,再拜崇酒,大夫答拜。主人复阼阶,揖降。大夫降,立于宾南。主人揖让,以宾升,大夫及众宾皆升,就席。

席工于西阶上,少东。乐正先升,北面立于其西。工四人,二瑟,瑟先,相者皆左何瑟,面鼓,执越,内弦。右手相,入,升自西阶,北面东上。工坐。相者坐授瑟,乃降。笙入,立于县中,西面。乃合乐:《周南·关雎》、《葛覃》、《卷耳》,《召南·鹊巢》、《采蘩》、《采苹》。工不兴,告于乐正,曰:「正歌备。」乐正告于宾,乃降。

主人取爵于上篚,献工。大师则为之洗。宾降,主人辞降。工不辞洗。卒洗,升实爵。工不兴,左瑟,一人拜受爵。主人阼阶上拜送爵。荐脯醢。使人相祭。工饮,不拜既爵,授主人爵。众工不拜,受爵,祭饮,辩有脯醢。不祭,不洗。遂献笙于西阶上。笙一人拜于下,尽阶,不升堂。受爵,主人拜送爵。阶前坐祭,立饮,不拜既爵,升,授主人爵。众笙不拜,受爵,坐祭,立饮,辩有脯醢,不祭。主人以爵降,尊于篚,反升,就席。

主人降席自南方,侧降,作相为司正。司正礼辞,许诺。主人再拜,司正答拜。主人升就席。司正洗觯,升自西阶,由楹内适阼阶上,北面受命于主人;西阶上北面请安于宾。宾礼辞,许。司正告于主人,遂立于楹间以相拜。主人阼阶上再拜,宾西阶上答再拜,皆揖就席。司正实觯,降自西阶,中庭北面坐奠觯,兴,退,少立;进,坐取觯,兴;反坐,不祭,遂卒觯,兴;坐奠觯,拜,执觯兴;洗,北面坐奠于其所,兴;少退,北面立于觯南。未旅。

三耦俟于堂西,南面东上。司射适堂西,袒决遂,取弓于阶西,兼挟乘矢,升自西阶。阶上北面告于宾,曰:「弓矢既具,有司请射。」宾对曰:「某不能。为二三子。」许诺。司射适阼阶上,东北面告于主人,曰:「请射于宾,宾许。」

司射降自西阶;阶前西面,命弟子纳射器。乃纳射器,皆在堂西。宾与大夫之弓倚于西序,矢在弓下,北括。众弓倚于堂西,矢在其上。主人之弓矢,在东序东。

司射不释弓矢,遂以比三耦于堂西。三耦之南,北面,命上射曰:「某御于子。」命下射曰:「子与某子射。」

司正为司马,司马命张侯,弟子说束,遂系左下纲。司马又命获者:「倚旌于侯中。」获者由西方,坐取旌,倚于侯中,乃退。

乐正适西方,命弟子赞工,迁乐于下。弟子相工,如初入;降自西降,阼阶下之东南,堂前三笴,西面北上坐。乐正北面立于其南。

司射犹挟乘矢,以命三耦:「各与其耦让取弓矢,拾!」三耦皆袒决遂。有司左执弣,右执弦,而授弓,遂授矢。三耦皆执弓,搢三而挟一个。司射先立于所设中之西南,东面。三耦皆进,由司射之西,立于其西南,东面北上而俟。

司射东面立于三耦之北,搢三而挟一个,揖进;当阶,北面揖;及阶,揖;升常,揖;豫则钩楹内,堂则由楹外。当左物,北面揖;及物,揖。左足履物,不方足,还;视侯中,俯正足。不去旌。诱射,将乘矢。执弓不挟,右执弦。南面揖,揖如升射;降,出于其位南;适堂西,改取一个,挟之。遂适阶西,取扑,搢之,以反位。

司马命获者执旌以负侯,获者适侯,执旌负侯而俟。司射还,当上耦西面,作上耦射。司射反位。上耦揖进,上射在左,并行;当阶,北面揖;及阶,揖。上射先升三等,下射从之,中等。上射升堂,少左;下射升,上射揖,并行。皆当其物,北面揖;及物,揖。皆左足履物,还视侯中,合足而俟。司马适堂西,不决遂,袒执弓,出于司射之南,升自西阶;钩楹,由上射之后,西南面立于物间;右执箫,南扬弓,命去侯。获者执旌许诺,声不绝,以至于乏;坐,东面偃旌,兴而俟。司马出于下射之南,还其后,降自西阶;反由司射之南,适堂西,释弓,袭,反位,立于司射之南。司射进,与司马交于阶前,相左;由堂下西阶之东,北面视上射,命曰:「无射获,无猎获!」上射揖。司射退,反位。乃射,上射既发,挟弓矢;而后下射射,拾发,以将乘矢。获者坐而获,举旌以宫,偃旌以商;获而未释获。卒射,皆执弓不挟,南面揖,揖如升射。上射降三等,下射少右,从之,中等;并行,上射于左。与升射者相左,交于阶前,相揖。由司马之南,适堂西,释弓,说决拾,袭而俟于堂西,南面,东上。三耦卒射,亦如之。司射去扑,倚于西阶之西,升堂,北面告于宾,曰:「三耦座射。」宾揖。

司射降,搢扑,反位。司马适堂西,袒执弓,由其位南,进;与司射交于阶前,相左;升自西阶,钩楹,自右物之后,立于物间;西南面,揖弓,命取矢。获者执旌许诺,声不绝,以旌负侯而俟。司马出于左物之南,还其后,降自西阶;遂适堂前,北面立于所设楅之南,命弟子设楅,乃设楅于中庭,南当洗,东肆。司马由司射之南,退,释弓于堂西,袭,反位。弟子取矢,北面坐委于楅;北括,乃退。司马袭进,当楅南,北面坐,左右抚矢而乘之。若矢不备,则司马又袒执弓如初,升命曰:「取矢不索!」弟子自西方应曰:「诺!」乃复求矢,加于楅。

司射倚扑于阶西,升,请射于宾,如初。宾许诺。宾、主人、大夫若皆与射,则遂告于宾,适阼阶上告于主人,主人与宾为耦;遂告于大夫,大夫虽众,皆与士为耦。以耦告于大夫,曰:「某御于子。」西阶上,北面作众宾射。司射降,搢扑,由司马之南适堂西,立,比众耦。众宾将与射者皆降,由司马之南适堂西,继三耦而立,东上。大夫之耦为上,若有东面者,则北上。宾、主人与大夫皆未降,司射乃比众耦辩。

遂命三耦拾取矢,司射反位。三耦拾取矢,皆袒决遂,执弓,进立于司马之西南。司射作上耦取矢,司射反位。上耦揖进;当楅北面揖,及楅揖。上射东面,下射西面。上射揖进,坐,横弓;却手自弓下取一个,兼诸弣,顺羽,且兴;执弦而左还,退反位,东面揖。下射进,坐,横弓;覆手自弓上取一个,兴;其他如上射。既拾取乘矢,揖,皆左还;南面揖,皆少进;当楅南,皆左还,北面,搢三挟一个;揖,皆左还,上射于右;与进者相左,相揖;退反位。三耦拾取矢,亦如之。后者遂取诱射之矢,兼乘矢而取之,以授有司于西方,而后反位。

众宾未拾取矢,皆袒决遂,执弓,搢三挟一个;由堂西进,继三耦之南而立,东面,北上。大夫之耦为上。

司射作射如初,一耦揖升如初。司马命去侯,获者许诺。司马降,释弓反位。司射犹挟一个,去扑,与司马交于阶前,升,请释获于宾;宾许。降,搢扑,西面立于所设中之东;北面命释获者设中,遂视之。释获者执鹿中,一人执算以从之。释获者坐设中,南当楅,西当西序,东面;兴受算,坐实八算于中,横委其馀于中西,南末;兴,共而俟。司射遂进,由堂下,北面命曰:「不贯不释!」上射揖。司射退反位。释获者坐取中之八算,改实八算于中,兴,执而俟。

乃射,若中,则释获者坐而释获,每一个释一算。上射于右,下射于左,若有馀算,则反委之。又取中之八算,改实八算于中,兴,执而俟。三耦卒射。

宾、主人、大夫揖,皆由其阶降揖。主人堂东袒决遂,执弓,搢三挟一个。宾于堂西亦如之。皆由其阶,阶下揖,升堂揖。主人为下射,皆当其物,北面揖,及物揖,乃射;卒,南面揖;皆由其阶,阶上揖,降阶揖。宾序西,主人序东,皆释弓,说决拾,袭,反位;升,及阶揖,升堂揖,皆就席。

大夫袒决遂,执弓,搢三挟一个,由堂西出于司射之西,就其耦。大夫为下射,揖进;耦少退。揖如三耦。及阶,耦先升。卒射,揖如升射,耦先降。降阶,耦少退。皆释弓于堂西,袭。耦遂止于堂西,大夫升就席。

众家继射,释获皆如初。司射所作,唯上耦。卒射,释获者遂以所执余获,升自西阶,尽阶,不升堂。告于宾曰:「左右卒射。」降,反位,坐委余获于中西;兴,共而俟。

司马袒决执弓,升命取矢,如初。获者许诺,以旌负侯,如初。司马降,释弓,反位。弟子委矢,如初。大夫之矢,则兼束之以茅,上握焉。司马乘矢如初。

司射遂适西阶西,释弓,去扑,袭;进由中东,立于中南,北面视算。释获者东面于中西坐,先数右获。二算为纯,一纯以取,实于左手;十纯则缩而委之,每委异之;有馀纯,则横于下。一算为奇,奇则又缩诸纯下。兴,自前适左,东面;坐,兼敛算,实于左手;一纯以委,十则异之,其馀如右获。司射复位。释获者遂进取贤获,执以升,自西阶,尽阶不升堂,告于宾。若右胜,则曰:「右贤于左。」若左胜,则曰:「左贤于右。」以纯数告;若有奇者,亦曰奇。若左右钧,则左右皆执一算以告,曰:「左右钧。」降复位,坐,兼敛算,实八算于中,委其馀于中西;兴,共而俟。

司射适堂西,命弟子设丰。弟子奉丰升,设于西楹之西,乃降。胜者之弟子洗觯,升酌,南面坐奠于丰上;降,袒执弓,反位。司射遂袒执弓,挟一个,搢扑,北面于三耦之南,命三耦及众宾:「胜者皆袒决遂,执张弓。不胜者皆袭,说决拾,却左手,右加弛弓于其上,遂以执弣。」司射先反位。三耦及众射者皆与其耦进立于射位,北上。司射作升饮者,如作射。一耦进,揖如升射,及阶,胜者先升,升堂,少右。不胜者进,北面坐取丰上之觯;兴,少退,立卒斛;进,坐奠于丰下;兴,揖。不胜者先降,与升饮者相左,交于阶前,相揖;出于司马之南,遂适堂西;释弓,袭而俟。有执爵者。执爵者坐取觯,实之,反奠于丰上。升饮者如初。三耦卒饮。宾、主人、大夫不胜,则不执弓,执爵者取觯,降洗,升实之,以授于席前,受觯,以适西阶上,北面立饮;卒觯,授执爵者,反就席。大夫饮,则耦不升。若大夫之耦不胜,则亦执弛弓,特升饮。众宾继饮,射爵者辩,乃彻丰与觯。

司马洗爵,升实之以降,献获者于侯。荐脯醢,设折俎,俎与荐皆三祭。获者负侯,北面拜受爵,司马西面拜送爵。获者执爵,使人执其荐与俎从之;适右个,设荐俎。获者南面坐,左执爵,祭脯醢;执爵兴,取肺,坐祭,遂祭酒;兴,适左个;中亦如之。左个之西北三步,东面设荐俎,获者荐右东面立饮,不拜既爵,司马受爵,奠于篚,复位。获者执其荐,使人执俎从之,辟设于乏南。获者负侯而俟。

司射适阶西,释弓矢,去扑,说决拾,袭;适洗,洗爵;升实之,以降,献释获者于其位,少南。荐脯醢,折俎,有祭。释获者荐右东面拜受爵,司射北面拜送爵。释获者就其荐坐,左执爵,祭脯醢;兴,取肺,坐祭,遂祭酒;兴,司射之西,北面立饮,不拜既爵。司射受爵,奠于篚。释获者少西辟荐,反位。

司射适堂西,袒决遂,取弓于阶西,挟一个,搢扑,以反位。司射去扑,倚于阶西,升请射于宾,如初。宾许。司射降,搢扑,由司马之南适堂西,命三耦及众宾:「皆袒决遂,执弓就位!」司射先反位。三耦及众宾皆袒决遂,执弓,各以其耦进,反于射位。

司射作拾取矢。三耦拾取矢如初,反位。宾、主人、大夫降揖如初。主人堂东,宾堂西,皆袒决遂,执弓;皆进阶前揖,及楅揖,拾取矢如三耦。卒,北面搢三挟一个,揖退。宾堂西,主人堂东,皆释弓矢,袭;及阶揖,升堂揖,就席。大夫袒决遂,执弓,就其耦;揖皆进,如三耦。耦东面,大夫西面。大夫进坐,说矢束,兴反位。而后耦揖进坐,兼取乘矢,顺羽而兴,反位,揖。大夫进坐,亦兼取乘矢,如其耦,北面,搢三挟一个,揖退。耦反位。大夫遂适序西,释弓矢,袭;升即席。众宾继拾取矢,皆如三耦,以反位。

司射犹挟一个以进,作上射如初。一耦揖升如初。司马升,命去侯,获者许诺。司马降,释弓反位。司射与司马交于阶前,去扑,袭;升,请以乐乐于宾。宾许诺。司射降,搢扑,东面命乐正,曰:「请以乐乐于宾,宾许。」司射遂适阶间,堂下北面命曰:「不鼓不释!」上射揖。司射退反位。乐正东面命大师,曰:「奏《驺虞》,间若一。」大师不兴,许诺。乐正退反位。

及奏《驺虞》以射。三耦卒射,宾、主人、大夫、众宾继射,释获如初。卒射,降。释获者执余获,升告左右卒射,如初。

司马升,命取矢,获者许诺。司马降,释弓反位。弟子委矢,司马乘之,皆如初。司射释弓视算,如初;释获者以贤获与钧告,如初。降复位。

司射命设丰,设丰、实觯如初;遂命胜者执张弓,不胜者执弛弓,升饮如初。

司射犹袒决遂,左执弓,右执一个,兼诸弦,面镞;适堂西,以命拾取矢,如初。司射反位。三耦及宾、主人、大夫、众宾皆袒决遂,拾取矢,如初;矢不挟,兼诸弦弣以退,不反位,遂授有司于堂西。辩拾取矢,揖,皆升就席。

司射乃适堂西,释弓,去扑,说决拾,袭,反位。司马命弟子说侯之左下纲而释之,命获者以旌退,命弟子退楅。司射命释获者退中与算,而俟。

司马反为司正,退,复觯南而立。乐正命弟子赞工即位。弟子相工,如其降也,升自西阶,反坐。宾北面坐,取俎西之觯,兴,阼阶上北面酬主人。主人降席,立于宾东。宾坐奠觯,拜;执觯兴;主人答拜。宾不祭,卒觯,不拜,不洗,实之,进东南面。主人阼阶上北面拜,宾少退。主人进受觯,宾主人之西北面拜送。宾揖,就席。主人以觯适西阶上酬大夫;大夫降席,立于主人之西,如宾酬主人之礼。主人揖,就席。若无大夫,则长受酬,亦如之。司正升自西阶,相旅,作受酬者曰:「某酬某子。」受酬者降席。司正退立于西序端,东面。众受酬者拜、兴、饮,皆如宾酬主人之礼。辩,遂酬在下者;皆升,受酬于西阶上。卒受者以觯降,奠于篚。

司正降复位,使二人举觯于宾与大夫。举觯者皆洗觯,升实之;西阶上北面,皆坐奠觯,拜,执觯兴。宾与大夫皆席末答拜。兴觯者皆坐祭,遂饮,卒觯,兴;坐奠觯,拜,执觯兴。宾与大夫皆答拜。举觯者逆降,洗,升实觯,皆立于西阶上,北面,东上。宾与大夫拜。举觯者皆进,坐奠于荐右。宾与大夫辞,坐受觯以兴。举觯者退反位,皆拜送,乃降。宾与大夫坐,反奠于其所,兴。若无大夫,则唯宾。

司正升自西阶,阼阶上受命于主人,适西阶上,北面请坐于宾,宾辞以俎。反命于主人,主人曰:「请彻俎。」宾许。司正降自西阶,阶前命弟子俟彻俎。司正升立于序端。宾降席,北面。主人降席自南方,阼阶上北面。大夫降席,席东南面。宾取俎,还授司正。司正以降自西阶,宾从之降,遂立于阶西,东面。司正以俎出,授从者。主人取俎,还授弟子。弟子受俎,降自西阶以东。主人降自阼阶,西面立。大夫取俎,还授弟子;弟子以降自西阶,遂出授从者;大夫从之降,立于宾南。众宾皆降,立于大夫之南,少退,北上。

主人以宾揖让,说屦,乃升。大夫及众宾皆说屦,升,坐。乃羞。无算爵。使二人举觯。宾与大夫不兴,取奠觯饮,卒觯,不拜。执觯者受觯,遂实之。宾觯以之主人,大夫之觯长受,而错,皆不拜。辩,卒受者兴,以旅在下者于西阶上。长受酬,酬者不拜,乃饮,卒觯,以实之。受酬者不拜受。辩旅,皆不拜。执觯者皆与旅。卒受者以虚觯降奠于篚;执觯者洗,升实觯,反奠于宾与大夫。无算乐。

宾兴,乐正命奏《陔》。宾降及阶,《陔》作。宾出,众宾皆出,主人送于门外,再拜。

明日,宾朝服以拜赐于门外,主人不见。如宾服,遂从之,拜辱于门外,乃退。

主人释服,乃息司正。无介。不杀。使人速。迎于门外,不拜;入,升。不拜至,不拜洗。荐脯醢,无俎。宾酢主人,主人不崇酒,不拜众宾;既献众宾,一人举觯,遂无算爵。无司正。宾不与。征唯所欲,以告于乡先生、君子可也。羞唯所有。乡乐唯欲。

记。大夫与,则公士为宾。使能,不宿戒。其牲,狗也。亨于堂东北。尊,綌幂。宾至,彻之。蒲筵,缁布纯。西序之席,北上。献用爵,其他用觯。以爵拜者,不徒作。荐,脯用笾,五胑,祭半胑,横于上。醢以豆,出自东房。胑长尺二寸。俎由东壁,自西阶升。宾俎,脊、胁、肩、肺。主人俎:脊、胁、臂、肺。肺皆离。皆右体也。进腠。凡举爵,三作而不徒爵。凡奠者于左,将兴者于右。众宾之长,一人辞洗,如宾礼。若有诸公,则如宾礼,大夫如介礼。无诸公,则大夫和宾礼。乐作,大夫不入。乐正,与立者齿。三笙一和而成声。献工与笙,取爵于上篚。既献,奠于下篚。其笙,则献诸西阶上。立者,东面北上。司正既举觯,而荐诸其位。三耦者,使弟子。司射前戒之。司射之弓矢与扑,倚于西阶之西。司射既袒决遂而升,司马阶前命张侯,遂命倚旌。凡侯:天子熊侯,白质;诸侯麋侯,赤质;大夫布侯,画以虎豹;士布侯,画以鹿豕。凡画者,丹质。射自楹间,物长如笴。其间容弓,距随长武。序则物当栋,堂则物当楣,命负侯者,由其位。凡适堂西,皆出入于司马之南。唯宾与大夫降阶,遂西取弓矢。旌,各以其物。无物,则以白羽与朱羽糅。杠长三仞,以鸿脰韬上,二寻。凡挟矢,于二指之间横之。司射在司马之北。司马无事不执弓。始射,获而未释获;复,释获;复,用乐行之。上射于右。楅长如笴,博三寸,厚寸有半,龙首,其中蛇交,韦当。楅,髹,横而拳之,南面坐而奠之,南北当洗。射者有过,则挞之。众宾不与射者,不降。取诱射之矢者,既拾取矢,而后兼诱射之乘矢而取之。宾、主人射,则司射摈升降,卒射即席,而反位卒事。鹿中,髹,前足跪,凿背容八算。释获者奉之,先首。大夫降,立于堂西以俟射。大夫与士射,袒薰襦。耦少退于物。司射释弓矢视算,与献释获者释弓矢。礼射不主皮。主皮之射者,胜者又射,不胜者降。主人亦饮于西阶上。获者之俎,折脊、胁、肺、臑。东方谓之右个。释获者之俎,折脊、胁、肺,皆有祭。大夫说矢束,坐说之。歌《驺虞》,若《采苹》,皆五终。射无算。古者于旅也语。凡旅,不洗。不洗者,不祭。既旅,士不入。大夫后出。主人送于门外,再拜。乡侯,上个五寻,中十尺。侯道五十弓,弓二寸以为侯中。倍中以为躬,倍躬以为左右舌。下舌半上舌。箭筹八十。长尺有握,握素。楚扑长如笴。刊本尺。君射,则为下射。上射退于物一笴,既发,则答君而俟。君,乐作而后就物。君,袒朱襦以射。小臣以巾执矢以授。若饮君,如燕,则夹爵。君,国中射,则皮树中,以翿旌获,白羽与朱羽糅;于郊,则闾中,以旌获;于竟,则虎中,龙旃。大夫,兕中,各以其物获。士,鹿中,翿旌以获。唯君有射于国中,其馀否。君在,大夫射,则肉袒。

\hypertarget{header-n32}{%
\subsection{燕礼}\label{header-n32}}

燕礼。小臣戒与者。膳宰具官馔于寝东。乐人县。设洗、篚于阼阶东南,当东霤。罍水在东,篚在洗西,南肆。设膳篚在其北,西面。司宫尊于东楹之西,两方壶,左玄酒,南上。公尊瓦大两,有丰,幂用綌若锡,在尊南,南上。尊士旅食于门西,两圆壶。司宫筵宾于户西,东上,无加席也。射人告具。

小臣设公席于阼阶上,西乡,设加席。公升,即位于席,西乡。小臣纳卿大夫,卿大夫皆入门右,北面东上。士立于西方,东面北上。祝史立于门东,北面东上。小臣师一人在东堂下,南面。士旅食者立于门西,东上。公降立于阼阶之东南,南乡尔卿,卿西面北上;尔大夫,大夫皆少进。

射人请宾。公曰:「命某为宾。」射人命宾,宾少进,礼辞。反命。又命之,宾再拜稽首,许诺,射人反命。宾出立于门外,东面。公揖卿大夫,乃升就席。

小臣自阼阶下,北面,请执幂者与羞膳者。乃命执幂者,执幂者升自西阶,立于尊南,北面,东上。膳宰请羞于诸公卿者。

射人纳宾。宾入,及庭,公降一等揖之。公升就席。

宾升自西阶,主人亦升自西阶,宾右北面至再拜,宾答再拜。主人降洗,洗南,西北面。宾降,阶西,东面。主人辞降,宾对。主人北面盥,坐取觚洗。宾少进,辞洗。主人坐奠觚于篚,兴对。宾反位。主人卒洗,宾揖,乃升。主人升。宾拜洗。主人宾右奠觚答拜,降盥。宾降,主人辞。宾对,卒盥。宾揖升。主人升,坐取觚。执幂者举幂,主人酌膳,执幂者反幂。主人筵前献宾。宾西阶上拜,筵前受爵,反位。主人宾右拜送爵。膳宰荐脯醢,宾升筵。膳宰设折俎。宾坐,左执爵,右祭脯醢,奠爵于荐右,兴;取肺,坐绝祭,哜之,兴加于俎;坐梲手,执爵,遂祭酒,兴;席末坐啐酒,降席,坐奠爵,拜,告旨,执爵兴。主人答拜。宾西阶上北面坐卒爵,兴;坐奠爵,遂拜。主人答拜。

宾以虚爵降,主人降。宾洗南坐奠觚,少进,辞降。主人东面对。宾坐取觚,奠于篚下,盥洗。主人辞洗。宾坐奠觚于篚,兴,对。卒洗,及阶,揖,升。主人升,拜洗如宾礼。宾降盥,主人降。宾辞降,卒盥,揖升,酌膳,执幂如初,以酢主人于西阶上。主人北面拜受爵,宾主人之左拜送爵。主人坐祭,不啐酒,不拜酒,不告旨;遂卒爵,兴;坐奠爵,拜,执爵兴。宾答拜。主人不崇酒,以虚爵降尊于篚。

宾降,立于西阶西。射人升宾,宾升立于序内,东面。主人盥,洗象觚,升实之,东北面献于公。公拜受爵。主人降自西阶,阼阶下北面拜送爵。士荐脯醢,膳宰设折俎,升自西阶。公祭如宾礼,膳宰赞授肺。不拜酒,立卒爵,坐奠爵,拜,执爵兴。主人答拜,升受爵以降,奠于膳篚。

更爵,洗,升酌膳酒以降;酢于阼阶下,北面坐奠爵,再拜稽首。公答再拜。主人坐祭,遂卒爵,再拜稽首。公答再拜,主人奠爵于篚。

主人盥洗,升,媵觚于宾,酌散,西阶上坐奠爵,拜宾。宾降筵,北面答拜。主人坐祭,遂饮,宾辞。卒爵,拜,宾答拜。主人降洗,宾降,主人辞降,宾辞洗。卒洗,揖升。不拜洗。主人酌膳。宾西阶上拜,受爵于筵前,反位。主人拜送爵。宾升席,坐祭酒,遂奠于荐东。主人降复位。宾降筵西,东南面立。

小臣自阼阶下请媵爵者,公命长。小臣作下大夫二人媵爵。媵爵者阼阶下,皆北面再拜稽首;公答再拜。媵爵者立于洗南,西面北上,序进,盥洗角觯;升自西阶,序进,酌散;交于楹北,降;阼阶下皆奠觯,再拜稽首,执觯兴。公答再拜。媵爵者皆坐祭,遂卒觯,兴;坐奠觯,再拜稽首,执觯兴。公答再拜。媵爵者执觯待于洗南。小臣请致者。若君命皆致,则序进,奠觯于篚,阼阶下皆再拜稽首;公答再拜。媵爵者洗象觯,升实之;序进,坐奠于荐南,北上;降,阼阶下皆再拜稽首,送觯。公答再拜。

公坐取大夫所媵觯,兴以酬宾。宾降,西阶下再拜稽首。公命小臣辞,宾升成拜。公坐奠觯,答再拜,执觯兴,立卒觯。宾下拜,小臣辞。宾升,再拜稽首。公坐奠觯,答再拜,执觯兴。宾进受虚爵,降奠于篚,易觯洗。公有命,则不易不洗,反升酌膳觯,下拜。小臣辞。宾升,再拜稽首。公答再拜。宾以旅酬于西阶上,射人作大夫长升受旅。宾大夫之右坐奠觯,拜,执觯兴;大夫答拜。宾坐祭,立饮,卒觯不拜。若膳觯也,则降更觯洗,升实散。大夫拜受。宾拜送。大夫辩受酬,如受宾酬之礼,不祭。卒受者以虚觯降尊于篚。

主人洗,升,实散,献卿于西阶上。司宫兼卷重席,设于宾左,东上。卿升,拜受觚;主人拜送觚。卿辞重席,司宫彻之,乃荐脯醢。卿升席坐,左执爵,右祭脯醢,遂祭酒,不啐酒;降席,西阶上北面坐卒爵,兴;坐奠爵,拜,执爵兴。主人答拜,受爵。卿降复位。辩献卿,主人以虚爵降,奠于篚。射人乃升卿,卿皆升就席。若有诸公,则先卿献之,如献卿之礼;席于阼阶西,北面东上,无加席。

小臣又请媵爵者,二大夫媵爵如初。请致者。若命长致,则媵爵者奠觯于篚,一人待于洗南。长致,致者阼阶下再拜稽首,公答再拜。洗象觯,升,实之,坐奠于荐南,降,与立于洗南者二人皆再拜稽首送觯,公答再拜。

公又行一爵,若宾,若长,唯公所酬。以旅于西阶上,如安。大夫卒受者以虚觯降奠于篚。

主人洗,升,献大夫于西阶上。大夫升,拜受觚,主人拜送觚。大夫坐祭,立卒爵,不拜既爵。主人受爵。大夫降复位。胥荐主人于洗北。西面,脯醢,无脀。辩献大夫,遂荐之,继宾以西,东上。卒,射人乃升大夫,大夫皆升,就席。

席工于西阶上,少东。乐正先升,北面立于其西。小臣纳工,工四人,二瑟。小臣左何瑟,面鼓,执越,内弦,右手相。入,升自西阶,北面东上坐。小臣坐授瑟,乃降。工歌《鹿鸣》、《四牡》、《皇皇者唬攥一人拜受爵,主人西阶上拜送爵。荐脯醢。使人相祭。卒爵,不拜。主人受爵。众工不拜受爵,坐祭,遂卒爵。辩有脯醢,不祭。主人受爵,降奠于篚。

公又举奠觯。唯公所赐。以旅于西阶上,如初。

卒,笙入,立于县中。奏《南陔》、《白华》、《华黍》。

主人洗,升,献笙于西阶上。一人拜,尽阶,不升堂,受爵,降;主人拜送爵。阶前坐祭,立卒爵,不拜既爵,升,授主人。众笙不拜受爵,降;坐祭,立卒爵。辩有脯醢,不祭。

乃间:歌《鱼丽》,笙《由庚》;歌《南有嘉鱼》,笙《崇丘》;歌《南山有台》,笙《由仪》。遂歌乡乐:《周南·关雎》、《葛覃》、《卷耳》,《召南·鹊巢》、《采蘩》、《采苹》。大师告于乐正曰:「正歌备。」乐正由楹内、东楹之东,告于公,乃降复位。

射人自阼阶下,请立司正,公许。射人遂为司正。司正洗角觯,南面坐奠于中庭;升,东楹之东受命,西阶上北面命卿、大夫:「君曰以我安!」卿、大夫皆对曰:「诺!敢不安?」司正降自西阶,南面坐取觯,升酌散,降,南面坐奠觯,右还,北面少立,坐取觯,兴,坐不祭,卒觯,奠之,兴,再拜稽首,左还,南面坐取觯,洗,南面反奠于其所,升自西阶,东楹之东,请彻俎降,公许。告于宾,宾北面取俎以出。膳宰彻公俎,降自阼阶以东。卿、大夫皆降,东面北上。宾反入,及卿、大夫皆说屦,升就席。公以宾及卿、大夫皆坐,乃安。羞庶羞。大夫祭荐。司正升受命,皆命:君曰:「无不醉!」宾及卿、大夫皆兴,对曰:「诺!敢不醉?」皆反坐。

主人洗,升,献士于西阶上。士长升,拜受觯,主人拜送觯。士坐祭,立饮,不拜既爵。其他不拜,坐祭,立饮。乃荐司正与射人一人、司士一人、执幂二人,立于觯南,东上。辩献士。士既献者立于东方,西面北上。乃荐士。祝史,小臣师,亦就其位而荐之。主人就旅食之尊而献之。旅食不拜,受爵,坐祭,立饮。

若射,则大射正为司射,如乡射之礼。

宾降洗,升媵觚于公,酌散,下拜。公降一等,小臣辞。宾升,再拜稽首,公答再拜。宾坐祭,卒爵,再拜稽首,公答再拜。宾降洗象觯,升酌膳,坐奠于荐南,降拜。小臣辞。宾升成拜,公答再拜。宾反位。公坐取宾所媵觯,兴。唯公所赐。受者如初受酬之礼,降更爵洗,升酌膳,下拜。小臣辞。升成拜,公答拜。乃就席,坐行之。有执爵者。唯受于公者拜。司正命执爵者爵辩,卒受者兴以酬士。大夫卒受者以爵兴,西阶上酬士。士升,大夫奠爵拜,士答拜。大夫立卒爵,不拜,实之。士拜受,大夫拜送。士旅于西阶上,辩。士旅酌。卒。

主人洗,升自西阶,献庶子于阼阶上,如献士之礼。辩,降洗,遂献左右正与内小臣,皆于阼阶上,如献庶子之礼。

无算爵。士也,有执膳爵者,有执散爵者。执膳爵者酌以进公,公不拜,受。执散爵者酌以之公,命所赐。所赐者兴受爵,降席下,奠爵,再拜稽首。公答拜。受赐爵者以爵就席坐,公卒爵,然后饮。执膳爵者受公爵,酌,反奠之。受赐爵者兴,授执散爵,执散爵者乃酌行之。唯受爵于公者拜。卒受爵者兴,以酬士于西阶上。士升,大夫不拜,乃饮,实爵。士不拜,受爵。大夫就席。士旅酌,亦如之。公有命彻幂,则卿大夫皆降,西阶下北面东上,再拜稽首。公命小臣辞。公答再拜,大夫皆辟。遂升,反坐。士终旅于上,如初。无算乐。

宵,则庶子执烛于阼阶上,司宫执烛于西阶上,甸人执大烛于庭,阍人为大烛于门外。宾醉,北面坐取其荐脯以降。奏《陔》。宾所执脯以赐钟人于门内霤,遂出。卿、大夫皆出。公不送。

公与客燕。曰:「寡君有不腆之酒,以请吾子之与寡君须臾焉。使某也以请。」对曰:「寡君,君之私也。君无所辱赐于使臣,臣敢辞。」「寡君固曰不腆,使某固以请!」「寡君,君之私也。君无所辱赐于使臣,臣敢固辞!」「寡君固曰不腆,使某固以请!」「某固辞,不得命,敢不从?」致命曰:「寡君使某,有不腆之酒,以请吾子之与寡君须臾焉!」「君贶寡君多矣,又辱赐于使臣,臣敢拜赐命!」

记。燕,朝服,于寝。其牲,狗也,亨于门外东方。若与四方之宾燕,则公迎之于大门内,揖让升。宾为苟敬,席于阼阶之西,北面,有脀,不哜肺,不啐酒。其介为宾。无膳尊,无膳爵。与卿燕,则大夫为宾。与大夫燕,亦大夫为宾。羞膳者与执幂者,皆士也。羞卿者,小膳宰也。若以乐纳宾,则宾及庭,奏《肆夏》;宾拜酒,主人答拜,而乐阕。公拜受爵,而奏《肆夏》;公卒爵,主人升,受爵以下,而乐阕。升歌《鹿鸣》,下管《新宫》,笙入三成,遂合乡乐。若舞,则《勺》。唯公与宾有俎。献公,曰:「臣敢奏爵以听命。」凡公所辞,皆栗阶。凡栗阶,不过二等。凡公所酬,既拜,请旅侍臣。凡荐与羞者,小膳宰也。有内羞。君与射,则为下射,袒朱襦,乐作而后就物。小臣以巾授矢,稍属。不以乐志。既发,则小臣受弓以授弓人。上射退于物一笴,既发,则答君而俟。若饮君,燕,则夹爵。君在,大夫射,则肉袒。若与四方之宾燕,媵爵,曰:「臣受赐矣。臣请赞执爵者。」相者对曰:「吾子无自辱焉。」有房中之乐。

\hypertarget{header-n36}{%
\subsection{大射仪}\label{header-n36}}

大射之仪。君有命戒射,宰戒百官有事于射者。射人戒诸公、卿、大夫射,司士戒士射与赞者。

前射三日,宰夫戒宰及司马、射人宿视涤。司马命量人量侯道与所设乏以貍步,大侯九十,参七十,干五十,设乏各去其侯西十、北十。遂命量人、巾车张三侯。大侯之崇,见鹄于参;参见鹄于干,干不及地,不系左下纲。设乏西十、北十,凡乏用革。

乐人宿县于阼阶东,笙磬西面,其南笙钟,其南鑮,皆南陈。建鼓在阼阶西,南鼓,应鼙在其东,南鼓。西阶之西,颂磬东面,其南钟,其南鑮,皆南陈。一建鼓在其南,东鼓,朔鼙在其北。一建鼓在西阶之东,南面。簜在建鼓之间,□倚于颂磬西紘。

厥明,司宫尊于东楹之西,两方壶,膳尊两甒在南。有丰。幂用锡若絺,缀诸箭。盖幂加勺,又反之。皆玄尊。酒在北。尊士旅食于西鑮之南,北面,两圜壶。又尊于大侯之乏东北,两壶献酒。设洗于阼阶东南,罍水在东,篚在洗西,南陈。设膳篚在其北,西面。又设洗于获者之尊西北,水在洗北。篚在南,东陈。小臣设公席于阼阶上,西乡。司宫设宾席于户西,南面,有加席。卿席宾东,东上。小卿宾西,东上。大夫继而东上,若有东面者,则北上。席工于西阶之东,东上。诸公阼阶西,北面,东上。官馔。羹定。

射人告具于公,公升,即位于席,西乡。小臣师纳诸公、卿、大夫,诸公、卿、大夫皆入门右,北面东上。士西方,东面北上。大史在干侯之东北,北面东上。士旅食者在士南,北面东上。小臣师从者在东堂下,南面西上。公降,立于阼阶之东南,南乡。小臣师诏揖诸公、卿大夫,诸公、卿大夫西面北上。揖大夫,大夫皆少进。大射正摈。摈者请宾,公曰:「命某为宾。」摈者命宾,宾少进,礼辞。反命,又命之。宾再拜稽首,受命。摈者反命。宾出,立于门外,北面。公揖卿、大夫,升就席。小臣自阼阶下北面,请执幂者与羞膳者。乃命执幂者。执幂者升自西阶,立于尊南,北面东上。膳宰请羞于诸公卿者。摈者纳宾,宾及庭,公降一等揖宾,宾辟,公升,即席。

奏《肆夏》,宾升自西阶。主人从之,宾右北面,至再拜。宾答再拜。主人降洗,洗南,西北面。宾降阶西,东面。主人辞降,宾对。主人北面盥,坐取觚,洗。宾少进,辞洗。主人坐奠觚于篚,兴对。宾反位。主人卒洗。宾揖,乃升。主人升,宾拜洗。主人宾右奠觚答拜,降盥。宾降,主人辞降,宾对。卒盥。宾揖升。主人升,坐取觚。执幂者举幂,主人酌膳,执幂者盖幂。酌者加勺,又反之。筵前献宾。宾西阶上拜,受爵于筵前,反位。主人宾右拜送爵。宰胥荐脯醢。宾升筵。庶子设折俎。宾坐,左执觚,右祭脯醢,奠爵于荐右;兴取肺,坐绝祭,哜之;兴加于俎,坐兑手,执爵,遂祭酒,兴,席末坐啐酒,降席,坐奠爵,拜,告旨,执爵兴。主人答拜。乐阕。宾西阶上北面坐,卒爵,兴;坐奠爵,拜,执爵兴。主人答拜。

宾以虚爵降。主人降。宾洗南西北面坐奠觚,少进,辞降。主人西阶西东面少进对。宾坐取觚,奠于篚下,盥洗。主人辞洗。宾坐奠觚于篚,兴对,卒洗,及阶,揖升。主人升,拜洗如宾礼。宾降盥,主人降。宾辞降,卒盥,揖升。酌膳、执幂如初,以酢主人于西阶上。主人北面拜受爵。宾主人之左拜送爵。主人坐祭,不啐酒,不拜酒,遂卒爵,兴,坐奠爵,拜,执爵兴。宾答拜。主人不崇酒,以虚爵降,奠于篚。宾降,立于西阶西,东面。摈者以命升宾。宾升,立于西序,东面。

主人盥,洗象觚,升酌膳,东北面献于公。公拜受爵,乃奏《肆夏》。主人降自西阶,阼阶下北面拜送爵。宰胥荐脯醢,由左房。庶子设折俎,升自西阶。公祭,如宾礼,庶子赞授肺。不拜酒,立卒爵;坐奠爵,拜,执爵兴。主人答拜,乐阕。升受爵,降奠于篚。

更爵,洗,升,酌散以降;酢于阼阶下,北面坐奠爵,再拜稽首。公答拜。主人坐祭,遂卒爵,兴,坐奠爵,再拜稽首。公答拜。主人奠爵于篚。

主人盥洗,升媵觚于宾,酌散,西阶上坐奠爵,拜。宾西阶上北面答拜。主人坐祭,遂饮。宾辞。卒爵兴,坐奠爵,拜,执爵兴。宾答拜。主人降洗,宾降。主人辞降,宾辞洗。卒洗。宾揖升,不拜洗。主人酌膳。宾西阶上拜,受爵于筵前,反位。主人拜送爵。宾升席,坐祭酒,遂奠于荐东。主人降,复位。宾降筵西,东南面立。

小臣自阼阶下请媵爵者,公命长。小臣作下大夫二人媵爵。媵爵者阼阶下皆北面再拜稽首。公答拜。媵爵者立于洗南,西面北上,序进,盥洗角觯,升自西阶,序进,酌散,交于楹北,降,适阼阶下,皆奠觯,再拜稽首,执觯兴。公答拜。媵爵者皆坐祭,遂卒觯,兴,坐奠觯,再拜稽首,执觯兴。公答再拜。媵爵者执觯待于洗南。小臣请致者。若命皆致,则序进,奠觯于篚,阼阶下皆北面再拜稽首。公答拜。媵爵者洗象觯,升实之;序进,坐奠于荐南,北上;降,适阼阶下,皆再拜稽首送觯。公答拜。媵爵者皆退反位。

公坐取大夫所媵觯,兴以酬宾。宾降,西阶下再拜稽首。小臣正辞,宾升成拜。公坐奠觯,答拜,执觯兴。公卒觯,宾下拜,小臣正辞。宾升,再拜稽首。公坐奠觯,答拜,执觯兴。宾进,受虚觯,降,奠于篚,易觯,兴洗,公有命,则不易不洗。反升酌膳,下拜。小臣正辞。宾升,再拜稽首。公答拜。宾告于摈者,请旅诸臣。摈者告于公,公许。宾以旅大夫于西阶上。摈者作大夫长升受旅。宾大夫之右坐奠觯,拜,执觯兴。大夫答拜。宾坐祭,立卒觯,不拜。若膳觯也,则降、更觯,洗,升实散。大夫拜受。宾拜送,遂就席。大夫辩受酬,如受宾酬之礼,不祭酒。卒受者以虚觯降,奠于篚,复位。

主人洗觚,升实散,献卿于西阶上。司宫兼卷重席,设于宾左,东上。卿升,拜受觚。主人拜送觚。卿辞重席,司宫彻之。乃荐脯醢。卿升席。庶子设折俎。卿坐,左执爵,右祭脯醢,奠爵于荐右,兴,取肺,坐,绝祭,不哜肺,兴,加于俎,坐梲手,取爵,遂祭酒,执爵兴,降席,西阶上北面坐卒爵,兴,坐奠爵,拜,执爵兴。主人答拜,受爵。卿降,复位。辩献卿。主人以虚爵降,奠于篚。摈者升卿,卿皆升,就席。若有诸公,则先卿献之,如献卿之礼,席于阼阶西,北面东上,无加席。

小臣又请媵爵者,二大夫媵爵如初。请致者。若命长致,则媵爵者奠觯于篚,一人待于洗南,长致者阼阶下再拜稽首,公答拜。洗象觯,升实之,坐奠于荐南,降,与立于洗南者二人皆再拜稽首送觯。公答拜。

公又行一爵,若宾,若长,唯公所赐。以旅于西阶上,如初。大夫卒受者以虚觯降,奠于篚。

主人洗觚,升,献大夫于西阶上。大夫升,拜受觚。主人拜送觚。大夫坐祭,立卒爵,不拜既爵。主人受爵。大夫降复位。胥荐主人于洗北,西面。脯醢,无脀。辩献大夫,遂荐之,继宾以西,东上,若有东面者,则北上。卒,摈者升大夫。大夫皆升,就席。

乃席工于西阶上,少东。小臣纳工,工六人,四瑟。仆人正徒相大师,仆人师相少师,仆人士相上工。相者皆左何瑟,后首,内弦,挎越,右手相。后者徒相入。小乐正从之。升自西阶,北面东上。坐授瑟,乃降。小乐正立于西阶东。乃歌《鹿鸣》三终。主人洗,升实爵,献工。工不兴,左瑟;一人拜受爵。主人西阶上拜送爵。荐脯醢。使人相祭。卒爵,不拜。主人受虚爵。众工不拜,受爵,坐祭,遂卒爵。辩有脯醢,不祭。主人受爵,降奠于篚,复位。大师及少师、上工皆降,立于鼓北,群工陪于后。乃管《新宫》三终。卒管。大师及少师、上工皆东坫之东南,西面北上,坐。

摈者自阼阶下请立司正。公许,摈者遂为司正。司正适洗,洗角觯,南面坐奠于中庭,升,东楹之东受命于公,西阶上北面命宾、诸公、卿、大夫。公曰:「以我安!」宾、诸公、卿、大夫毕对曰:「诺!敢不安?」司正降自西阶,南面坐取觯,升、酌散、降,南面坐奠觯、兴,右还,北面少立、坐取觯,兴、坐,不祭,卒觯,奠之,兴、再拜稽首,左还,南面坐取觯,洗、南面反奠于其所,北面立。

司射适次,袒决遂,执弓,挟乘矢,于弓外见镞于弣,右巨指钩弦。自阼阶前曰:「为政请射。」遂告曰:「大夫与大夫,士御于大夫。」遂适西阶前,东面右顾,命有司纳射器,射器皆入。君之弓矢适东堂。宾之弓矢与中、筹、丰,皆止于西堂下。众弓矢不挟。手忽众弓矢、楅,皆适次而俟。工人、士与梓人升自北阶,两楹之间。疏数容弓,若丹,若墨,度尺而午。射正莅之。卒画,自北阶下。司宫扫所画物,自北阶下。大史俟于所设中之西,东面以听政。司射西面誓之曰:「公射大侯,大夫射参,士射干。射者非其侯,中之不获!卑者与尊者为耦,不异侯!」大史许诺。遂比三耦。三耦俟于次北,西面北上。司射命上射,曰:「某御于子。」命下射,曰:「子与某子射。」卒,遂命三耦取弓矢于次。

司射入于次,搢三挟一个,出于次,西面揖,当阶北面揖,及阶揖,升堂揖,当物北面揖,及物揖,由下物少退,诱射。射三侯,将乘矢,始射干,又射参,大侯再发。卒射,北面揖。及阶,揖降,如升射之仪。遂适堂西,改取一个挟之。遂取扑搢之,以立于所设中之西南,东面。

司马师命负侯者:「执旌以负侯。」负侯者皆适侯,执旌负侯而俟。司射适次,作上耦射。司射反位。上耦出次,西面揖进。上射在左,并行。当阶北面揖,及阶揖。上射先升三等,下射从之,中等。上射升堂,少左。下射升,上射揖,并行。皆当其物北面揖,及物揖。皆左足履物,还,视侯中,合足而俟。司马正适次,袒决遂,执弓,右挟之,出,升自西阶,适下物,立于物间,左执弣,右执箫,南扬弓,命去侯。负侯皆许诺,以宫趋,直西,及乏南,又诺以商,至乏,声止,授获者,退立于西方。获者兴,共而俟。司马正出于下射之南,还其后,降自西阶,遂适次,释弓,说决拾,袭,反位。司射进,与司马正交于阶前,相左,由堂下西阶之东北面视上射,命曰:「毋射获!毋猎获!」上射揖。司射退,反位。乃射,上射既发,挟矢,而后下射射,拾发以将乘矢。获者坐而获,举旌以宫,偃旌以商,获而未释获。卒射,右挟之,北面揖,揖如升射。上射降三等,下射少右,从之,中等;并行,上射于左。与升射者相左,交于阶前,相揖。适次,释弓,说决拾,袭,反位。三耦卒射如之。司射去扑,倚于阶西,适阼阶下,北面告于公,曰:「三耦座射。」反,搢扑,反位。

司马正袒、决、遂,执弓,右挟之,出;与司射交于阶前,相左。升自西阶,自右物之后,立于物间;西南面,揖弓,命取矢。负侯许诺,如初去侯,皆执旌以负其侯而俟。司马正降自西阶,北面命设楅。小臣师设楅。司马正东面,以弓为毕。既设楅,司马正适次,释弓,说决拾,袭,反位。小臣坐委矢于楅,北括;司马师坐乘之,卒。若矢不备,则司马正又袒执弓,升,命取矢如初,曰:「取矢不索!」乃复求矢,加于楅。卒,司马正进坐,左右抚之,兴,反位。

司射适西阶西,倚扑;升自西阶,东面请射于公。公许。遂适西阶上,命宾御于公,诸公、卿则以耦告于上,大夫则降,即位而后告。司射自西阶上,北面告于大夫,曰:「请降!」司射先降,搢扑,反位。大夫从之降,适次,立于三耦之南,西面北上。司射东面于大夫之西,北耦。大夫与大夫,命上射曰:「某御于子。」命下射曰:「子与某子射。」卒,遂比众耦。众耦立于大夫之南,西面北上。若有士与大夫为耦,则以大夫之耦为上,命大夫之耦曰:「子与某子射。」告于大夫曰:「某御于子。」命众耦,如命三耦之辞。诸公、卿皆未降。

遂命三耦各与其耦拾取矢,皆袒、决、遂,执弓,右挟之。一耦出,西面揖,当楅北面揖,及楅揖。上射东面,下射西面。上射揖进,坐横弓,却手自弓下取一个,兼诸弣,兴,顺羽且左还,毋周,反面揖。下射进,坐横弓,覆手自弓上取一个,兼诸弣,兴;顺羽,且左还,毋周,反面揖。既拾取矢,捆之。兼挟乘矢,皆内还,南面揖;适楅南,皆左还,北面揖;搢三挟一个。揖,以耦左还,上射于左。退者与进者相左,相揖。退释弓矢于次,说决拾,袭,反位。二耦拾取矢,亦如之。后者遂取诱射之矢,兼乘矢而取之,以授有司于次中。皆袭,反位。

司射作射如初。一耦揖、升如初。司马命去侯,负侯许诺如初。司马降,释弓,反位。司射犹挟一个,去扑;与司马交于阶前,适阼阶下,北面请释获于公;公许,反,搢扑;遂命释获者设中;以弓为毕,北面。大史释获。小臣师执中,先首,坐设之;东面,退。大史实八筭于中,横委其馀于中西,兴,共而俟。司射西面命曰:「中离维纲,扬触,捆复,公则释获,众则不与!唯公所中,中三侯皆获。」释获者命小史,小史命获者。司射遂进由堂下,北面视上射,命曰:「不贯不释!」上射揖。司射退,反位。释获者坐取中之八筭,改实八筭,兴,执而俟。乃射。若中,则释获者每一个释一筭,上射于右,下射于左。若有馀筭,则反委之。又取中之八筭,改实八筭于中。兴,执而俟。三耦卒射。

宾降,取弓矢于堂西。诸公、卿则适次,继三耦以南。公将射,则司马师命负侯,皆执其旌以负其侯而俟,司马师反位。隶仆人扫侯道。司射去扑,适阼阶下,告射于公,公许,适西阶东告于宾,遂搢扑,反位。小射正一人,取公之决拾于东坫上,一小射正授弓拂弓,皆以俟于东堂。公将射,则宾降,适堂西,袒决遂,执弓,搢三挟一个,升自西阶,先待于物北,一笴,东面立。司马升,命去侯如初;还右,乃降,释弓,反位。公就物,小射正奉决拾以笥,大射正执弓,皆以从于物。小射正坐奠笥于物南,遂拂以巾,取决,兴,赞设决、朱极三。小臣正赞袒,公袒朱襦,卒袒,小臣正退俟于东堂。小射正又坐取拾,兴。赞设拾,以笥退奠于坫上,复位。大射正执弓,以袂顺左右隈,上再下一,左执弣,右执箫,以授公。公亲揉之。小臣师以巾内拂矢,而授矢于公,稍属。大射正立于公后,以矢行告于公。下曰留,上曰扬,左右曰方。公既发,大射正受弓而俟,拾发以将乘矢。公卒射,小臣师以巾退,反位,大射正受弓,小射正以笥受决拾,退奠于坫上,复位。大射正退,反司正之位。小臣正赞袭。公还而后宾降,释弓于堂西,反位于阶西东面。公即席,司正以命升宾。宾升复筵而后卿大夫继射。

诸公、卿取弓矢于次中,袒决遂,执弓,搢三挟一个,出,西面揖,揖如三耦,升射、卒射、降如三耦,适次,释弓,说决拾,袭,反位。众皆继射,释获皆如初。卒射,释获者遂以所执余获,适阼阶下,北面告于公,曰:「左右卒射。」反位,坐委余获于中西,兴,共而俟。

司马袒执弓,升,命取矢如初。负侯许诺,以旌负侯如初。司马降,释弓如初。小臣委矢于楅,如初。宾、诸公、卿、大夫之矢皆异束之以茅,卒,正坐左右抚之,进束,反位。宾之矢,则以授矢人于西堂下。司马释弓,反位,而后卿、大夫升就席。

司射适阶西,释弓,去扑,袭;进由中东,立于中南,北面视筭。释获者东面于中西坐,先数右获。二筭为纯,一纯以取,实于左手。十纯则缩而委之,每委异之。有馀纯,则横诸下。一筭为奇,奇则又缩诸纯下。兴,自前适左,东面坐,坐,兼敛筭,实于左手,一纯以委,十则异之,其馀如右获。司射复位。释获者遂进取贤获,执之,由阼阶下,北面告于公。若右胜,则曰右贤于左。若左胜,则曰左贤于右。以纯数告;若有奇者,亦曰奇。若左右钧,则左右各执一算以告,曰左右钧。还复位,坐,兼敛算,实八算于中,委其馀于中西,兴,共而俟。

司射命设丰。司官士奉丰,由西阶升,北面坐设于西楹西,降复位。胜者之弟子洗觯,升酌散,南面坐奠于丰上,降反位。司射遂袒执弓,挟一个,搢扑,东面于三耦之西,命三耦及众射者:「胜者皆袒决遂,执张弓。不胜者皆袭,说决拾,却左手,右加弛弓于其上,遂以执弣。」司射先反位。三耦及众射者皆升饮射爵于西阶上。小射正作升饮射爵者,如作射。一耦出,揖如升射,及阶,胜者先升,升堂少右。不胜者进,北面坐取丰上之觯,兴;少退,立卒觯,进;坐奠于丰下,兴,揖。不胜者先降,与升饮者相左,交于阶前,相揖;适次,释弓,袭,反位。仆人师继酌射爵,取觯实之,反奠于丰上,退俟于序端。升饮者如初。三耦卒饮。若宾、诸公、卿、大夫不胜,则不降,不执弓,耦不升。仆人师洗,升实觯以授;宾、诸公、卿、大夫受觯于席,以降,适西阶上,北面立饮,卒觯,授执爵者,反就席。若饮公,则侍射者降,洗角觯,升酌散,降拜;公降一等,小臣正辞,宾升、再拜稽首,公答再拜;宾坐祭,卒爵,再拜稽首,公答再拜;宾降,洗象觯,升酌膳以致,下拜,小臣正辞,升、再拜稽首,公答再拜;公卒觯,宾进受觯,降洗散觯,升实散,下拜,小臣正辞,升、再拜稽首,公答再拜;坐,不祭,卒觯,降奠于篚,阶西东面立。摈者以命升宾,宾升就席。若诸公、卿、大夫之耦不胜,则亦执弛弓,特升饮。众皆继饮射爵,如三耦。射爵辩,乃彻丰与觯。

司宫尊侯于服不之东北,两献酒,东面南上,皆加勺设洗于尊西北,篚在南,东肆,实一散于篚。司马正洗散,遂实爵,献服不。服不侯西北三步,北面拜受爵。司马正西面拜送爵,反位。宰夫有司荐,庶子设折俎。卒错,获者适右个,荐俎从之。获者左执爵,右祭荐俎,二手祭酒;适左个,祭如右个,中亦如之。卒祭,左个之西北三步,东面。设荐俎,立卒爵。司马师受虚爵,洗献隶仆人与巾车、获者,皆如大侯之礼。卒,司马师受虚爵,奠于篚。获者皆执其荐,庶子执俎从之,设于乏少南。服不复负侯而俟。

司射适阶西,去扑,适堂西,释弓,说决拾,袭,适洗,洗觚,升,实之,降,献释获者于其位,少南。荐脯醢、折俎,皆有祭。释获者荐右东面拜受爵。司射北面拜送爵。释获者就其荐坐,左执爵,右祭脯醢,兴取肺,坐祭,遂祭酒;兴,司射之西,北面立卒爵,不拜既爵。司射受虚爵,奠于篚。释获者少西辟荐,反位。司射适堂西,袒决遂,取弓,挟一个,适阶西,搢扑以反位。

司射倚扑于阶西,适阼阶下,北面请射于公,如初。反搢扑,适次,命三耦皆袒决遂,执弓,序出取矢。司射先反位。三耦拾取矢如初,小射正作取矢如初。三耦既拾取矢,诸公、卿、大夫皆降如初位,与耦入于次,皆袒决遂,执弓,皆进当楅,进坐,说矢束。上射东面,下射西面,拾取矢如三耦。若士与大夫为耦,士东面,大夫西面。大夫进坐,说矢束,退反位。耦揖进坐,兼取乘矢,兴,顺羽,且左还,毋周,反面揖。大夫进坐,亦兼取乘矢,如其耦;北面搢三挟一个,揖进。大夫与其耦皆适次,释弓,说决拾,袭,反位。诸公、卿升就席。众射者继拾取矢,皆如三耦,遂入于次,释弓矢,说决拾,袭,反位。

司射犹挟一个以作射,如初。一耦揖升如初。司马升,命去侯,负侯许诺。司马降,释弓反位。司射与司马交于阶前,倚扑于阶西,适阼阶下,北面请以乐于公。公许。司射反,搢扑,东面命乐正曰:「命用乐!」乐正曰:「诺。」司射遂适堂下,北面视上射,命曰:「不鼓不释!」上射揖。司射退反位。乐正命大师,曰:「奏《狸首》,间若一!」大师不兴,许诺。乐正反位。奏《狸首》以射,三耦卒射。宾待于物如初。公乐作而后就物,稍属,不以乐志。其他如初仪,卒射如初。宾就席。诸公、卿、大夫、众射者皆继射,释获如初。卒射,降反位。释获者执余获进告:「左右卒射。」如初。

司马升,命取矢,负侯许诺。司马降,释弓反位。小臣委矢,司马师乘之,皆如初。司射释弓、视筭,如初。释获者以贤获与钧告,如初。复位。

司射命设丰、实觯,如初。遂命胜者执张弓,不胜者执弛弓,升、饮如初。卒,退丰与觯,如初。

司射犹袒决遂,左执弓,右执一个,兼诸弦,面镞,适次,命拾取矢,如初。司射反位。三耦及诸公、卿、大夫、众射者,皆袒决遂以拾取矢,如初。矢不挟,兼诸弦,面镞;退适次,皆授有司弓矢,袭,反位。卿、大夫升就席。

司射适次,释弓,说决拾,去扑,袭,反位。司马正命退楅解纲。小臣师退楅,巾车、量人解左下纲。司马师命获者以旌与荐俎退。司射命释获者退中与筭而俟。

公又举奠觯,唯公所赐。若宾,若长,以旅于西阶上,如初。大夫卒受者以虚觯降,奠于篚,反位。

司正升自西阶,东楹之东,北面告于公,请彻俎,公许。遂适西阶上,北面告于宾。宾北面取俎以出。诸公、卿取俎如宾礼,遂出,授从者于门外。大夫降复位。庶子正彻公俎,降自阼阶以东。宾、诸公、卿皆入,东面北上。司正升宾。宾、诸公、卿、大夫皆说屦,升就席。公以宾及卿、大夫皆坐,乃安,羞庶羞。大夫祭荐。司正升受命,公曰:「众无不醉!」宾及诸公、卿、大夫皆兴,对曰:「诺!敢不醉?」皆反位坐。

主人洗、酌,献士于西阶上。士长升,拜受觯,主人拜送。士坐祭,立饮,不拜既爵。其他不拜,坐祭,立饮。乃荐司正与射人于觯南,北面东上,司正为上。辩献士。士既献者立于东方,西面北上。乃荐士。祝史、小臣师亦就其位而荐之。主人就士旅食之尊而献之。旅食不拜,受爵,坐祭,立饮。主人执虚爵,奠于篚,复位。

宾降洗,升,媵觯于公,酌散,下拜。公降一等,小臣正辞。宾升再拜稽首,公答再拜。宾坐祭,卒爵,再拜稽首。公答再拜。宾降,洗象觚,升酌膳,坐奠于荐南,降拜。小臣正辞。宾升成拜,公答拜。宾反位。公坐取宾所媵觯,兴。唯公所赐。受者如初受酬之礼。降,更爵,洗;升酌膳;下,再拜稽首。小臣正辞,升成拜。公答拜。乃就席,坐行之,有执爵者。唯受于公者拜。司正命「执爵者爵辩,卒受者兴以酬士。」大夫卒受者以爵兴,西阶上酬士。士升,大夫奠爵拜,士答拜。大夫立卒爵,不拜,实之。士拜受,大夫拜送。士旅于西阶上,辩。士旅酌。

若命曰:「复射!」则不献庶子。司射命射,唯欲。卿、大夫皆降,再拜稽首。公答拜。一发,中三侯皆获。

主人洗,升自西阶,献庶子于阼阶上,如献士之礼。辩献。降洗,遂献左右正与内小臣,皆于阼阶上,如献庶子之礼。

无算爵。士也,有执膳爵者,有执散爵者。执膳爵者酌以进公;公不拜,受。执散爵者酌以之公,命所赐。所赐者兴受爵,降席下,奠爵,再拜稽首;公答再拜。受赐爵者以爵就席坐,公卒爵,然后饮。执膳爵者受公爵,酌,反奠之。受赐者兴,授执散爵者。执散爵者乃酌行之。唯受于公者拜。卒爵者兴以酬士于西阶上。士升。大夫不拜乃饮,实爵;士不拜,受爵。大夫就席。士旅酌,亦如之。公有命彻幂,则宾及诸公、卿、大夫皆降,西阶下北面东上,再拜稽首。公命小臣正辞,公答拜。大夫皆辟。升,反位。士旅于上,如初。无算乐。

宵,则庶子执烛于阼阶上,司宫执烛于西阶上,甸人执大烛于庭,阍人为烛于门外。宾醉,北面坐取其荐脯以降。奏《陔》。宾所执脯,以赐钟人于门内霤,遂出。卿、大夫皆出,公不送。公入,《骜》。

\hypertarget{header-n40}{%
\subsection{聘礼}\label{header-n40}}

聘礼。君与卿图事,遂命使者,使者再拜稽首辞,君不许,乃退。既图事,戒上介,亦如之。宰命司马戒众介,众介皆逆命,不辞。

宰书币,命宰夫官具。及期,夕币。使者朝服,帅众介夕。管人布幕于寝门外。官陈币,皮北首,西上,加其奉于左皮上;马则北面,奠币于其前。使者北面,众介立于其左,东上。卿、大夫在幕东,西面北上。宰入,告具于君。君朝服出门左,南乡。史读书展币。宰执书,告备具于君,授使者。使者受书,授上介。公揖入。官载其币,舍于朝。上介视载者、所受书以行。

厥明,宾朝服释币于祢。有司筵几于室中。祝先入,主人从入。主人在右,再拜,祝告,又再拜。释币,制玄纁束,奠于几下,出。主人立于户东。祝立于牖西,又入,取币,降,卷币,实于□,埋于西阶东。又释币于行。遂受命。上介释币亦如之。

上介及众介俟于使者之门外。使者载旃,帅以受命于朝。君朝服,南乡。卿、大夫西面北上。君使卿进使者。使者入,及众介随入,北面东上。君揖使者,进之,上介立于其左,接闻命。贾人西面坐启椟,取圭垂缫,不起而授宰。宰执圭屈缫,自公左授使者。使者受圭,同面,垂缫以受命。既述命,同面授上介。上介受圭屈缫,出,授贾人,众介不从。受享束帛加璧,受夫人之聘璋,享玄纁束帛加琮,皆如初。遂行,舍于郊,敛旃。

若过邦,至于竟,使次介假道,束帛将命于朝,曰:「请帅。」奠币。下大夫取以入告,出许,遂受币。饩之以其礼,上宾大牢,积唯刍禾,介皆有饩。士没其竟。誓于其竟,宾南面,上介西面,众介北面东上,史读书,司马执策立于其后。

未入竟,壹肄。为壝坛,画阶,帷其北,无宫。朝服无主,无执也。介皆与,北面西上。习享,士执庭实习夫人聘享,亦如之。习公事,不习私事。

及竟,张旃,誓。乃谒关人。关人问从者几人,以介对。君使士请事,遂以入竟。

入竟,敛旃,乃展。布幕,宾朝服立于幕东,西面,介皆北面东上。贾人北面,坐拭圭,遂执展之。上介北面视之,退复位。退圭。陈皮,北首,西上,又拭璧,展之,会诸其币,加于左皮上。上介视之,退。马则幕南、北面,奠币于其前。展夫人之聘享,亦如之,贾人告于上介,上介告于宾。有司展群币以告。及郊,又展,如初。及馆,展币于贾人之馆,如初。

宾至于近郊,张旃。君使下大夫请行,反。君使卿朝服,用束帛劳。上介出请。入告。宾礼辞,迎于舍门之外,再拜。劳者不答拜。宾揖,先入,受于舍门内。劳者奉币入,东面致命。宾北面听命,还,少退,再拜稽首,受币。劳者出。授老币,出迎劳者。劳者礼辞。宾揖,先入,劳者从之。乘皮设。宾用束锦傧劳者,劳者再拜稽首受。宾再拜稽首,送币。劳者揖皮出,乃退。宾送再拜。夫人使下大夫劳以二竹簋方,玄被纁里,有盖,其实枣蒸栗择,兼执之以进。宾受枣,大夫二手授栗。宾之受,如初礼。傧之如初。下大夫劳者遂以宾入。

至于朝,主人曰:「不腆先君之祧,既拚以俟矣。」宾曰:「俟间。」大夫帅至于馆,卿致馆。宾迎,再拜。卿致命,宾再拜稽首。卿退,宾送再拜。宰夫朝服设飧:饪一牢,在西,鼎九,羞鼎三;腥一牢,在东,鼎七。堂上之馔八,西夹六。门外米、禾皆二十四,薪刍倍禾。上介:饪一牢,在西,鼎七,羞鼎三;堂上之馔六;门外米、禾皆十车,薪刍倍禾。众介皆少牢。

厥明,讶宾于馆。宾皮弁聘,至于朝。宾入于次,乃陈币。卿为上摈,大夫为承摈,士为绍摈。摈者出请事。公皮弁,迎宾于大门内。大夫纳宾。宾入门左,公再拜,宾辟,不答拜。公揖入,每门每曲揖。及庙门,公揖入,立于中庭;宾立接西塾。几筵既设,摈者出请命。贾人东面坐启椟,取圭垂缫,不起而授上介。上介不袭,执圭屈缫,授宾。宾袭,执圭。摈者入告,出辞玉。纳宾,宾入门左。介皆入门左,北面西上。三揖,至于阶,三让。公升二等,宾升,西楹西,东面。摈者退中庭。宾致命。公左还,北乡。摈者进。公当楣再拜。宾三退,负序。公侧袭,受玉于中堂与东楹之间。摈者退,负东塾而立。宾降,介逆出。宾出。公侧授宰玉,裼,降立。摈者出请。宾裼,奉束帛加璧享。摈者入告,出许。庭实,皮则摄之,毛在内;内摄之,入设也。宾入门左,揖让如初,升致命,张皮。公再拜受币。士受皮者自后右客;宾出,当之坐摄之。公侧授宰币,皮如入,右首而东。聘于夫人,用璋,享用琮,如初礼。若有言,则以束帛,如享礼。摈者出请事,宾告事毕。

宾奉束锦以请觌。摈者入告,出辞,请礼宾。宾礼辞,听命。摈者入告。宰夫彻几改筵。公出,迎宾以入,揖让如初。公升,侧受几于序端。宰夫内拂几三,奉两端以进。公东南乡,外拂几三,卒,振袂,中摄之,进,西乡。摈者告。宾进,讶受几于筵前,东面俟。公壹拜送。宾以几辟,北面设几,不降,阶上答再拜稽首。宰夫实觯以醴,加柶于觯,面枋。公侧受醴。宾不降,壹拜,进筵前受醴,复位。公拜送醴。宰夫荐笾豆脯醢,宾升筵,摈者退负东塾。宾祭脯醢,以柶祭醴三,庭实设。降筵,北面,以柶兼诸觯,尚擸,坐啐醴。公用束帛。建柶,北面奠于荐东。摈者进相币。宾降辞币,公降一等辞。栗阶升,听命,降拜,公辞。升,再拜稽首,受币,当东楹,北面,退,东面俟。公壹拜,宾降也。公再拜。宾执左马以出。上介受宾币,从者讶受马。

宾觌,奉束锦,总乘马,二人赞。入门右,北面奠币,再拜稽首。摈者辞。宾出。摈者坐取币出,有司二人牵马以从,出门,西面于东塾南。摈者请受。宾礼辞,听命。牵马,右之。入设。宾奉币,入门左,介皆入门左,西上。公揖让如初,升。公北面再拜。宾三退,反还负序。振币进授,当东楹北面。士受马者,自前还牵者后,适其右,受。牵马者自前西,乃出。宾降阶东拜送。君辞。拜也,君降一等辞。摈者曰:「寡君从子,虽将拜,起也。」栗阶升。公西乡。宾阶上再拜稽首。公少退。宾降出。公侧授宰币。马出。

公降立。摈者出请。上介奉束锦,士介四人皆奉玉锦束,请觌。摈者入告,出许。上介奉币,俪皮,二人赞;皆入门右,东上,奠币,皆再拜稽首。摈者辞,介逆出。摈者执上币,士执众币;有司二人举皮,从其币。出请受。委皮南面;执币者西面北上。摈者请受。介礼辞,听命。皆进,讶受其币。上介奉币,皮先,入门左,奠皮。公再拜。介振币,自皮西进,北面授币,退复位,再拜稽首送币。介出。宰自公左受币,有司二人坐举皮以东。摈者又纳士介。士介入门右,奠币,再拜稽首。摈者辞,介逆出。摈者执上币以出,礼请受,宾固辞。公答再拜。摈者出,立于门中以相拜,士介皆辟。士三人,东上,坐取币,立。摈者进。宰夫受币于中庭,以东,执币者序从之。

摈者出请,宾告事毕。摈者入告,公出送宾。及大门内,公问君。宾对,公再拜。公问大夫,宾对。公劳宾,宾再拜稽首,公答拜。公劳介,介皆再拜稽首,公答拜。宾出,公再拜送,宾不顾。

宾请有事于大夫,公礼辞,许。宾即馆。卿、大夫劳宾,宾不见。大夫奠雁再拜,上介受。劳上介,亦如之。

君使卿韦弁,归饔饩五牢。上介请事,宾朝服礼辞。有司入陈。饔,饪一牢,鼎九,设于西阶前,陪鼎当内廉,东面北上,上当碑,南陈。牛、羊、豕、鱼、腊,肠、胃同鼎,肤、鲜鱼、鲜腊,设扃鼏。膷、臐、膮,盖陪牛、羊、豕。腥二牢,鼎二七,无鲜鱼、鲜腊,设于阼阶前,西面,南陈如饪鼎,二列。堂上八豆,设于户西,西陈,皆二以并,东上韭菹,其南醓醢,屈。八簋继之,黍其南稷,错。六鉶继之,牛以西羊、豕,豕南牛,以东羊、豕。两簠继之,粱在北,八壶设于西序,北上,二以并,南陈。西夹六豆,设于西墉下,北上韭菹,其东醓醢,屈。六簋继之,黍其东稷,错。四鉶继之,牛以南羊,羊东豕,豕以北牛。两簠继之,粱在西。皆二以并,南陈。六壶西上,二以并,东陈。馔于东方,亦如之,西北上。壶东上,西陈。醯醢百瓮,夹碑,十以为列,醯在东。饩二牢,陈于门西,北面东上。牛以西羊、豕,豕西牛、羊、豕。米百筥,筥半斛,设于中庭,十以为列,北上。黍、粱、稻皆二行,稷四行。门外,米三十车,车秉有五籔。设于门东,为三列,东陈;禾三十车,车三秅。设于门西,西陈。薪刍倍禾。

宾皮弁迎大夫于外门外,再拜,大夫不答拜。揖入。及庙门,宾揖入。大夫奉束帛,入,三揖,皆行。至于阶,让,大夫先升一等。宾从,升堂,北面听命。大夫东面致命,宾降,阶西再拜稽首,拜饩亦如之。大夫辞,升成拜。受币堂中西,北面。大夫降,出。宾降,授老币,出迎大夫。大夫礼辞,许。入,揖让如初。宾升一等,大夫从,升堂。庭实设,马乘。宾降堂,受老束锦,大夫止。宾奉币西面,大夫东面。宾致币。大夫对,北面当楣,再拜稽首,受币于楹间,南面,退,东面俟。宾再拜稽首送币。大夫降,执左马以出。宾送于外门外,再拜。明日,宾拜于朝,拜饔与饩,皆再拜稽首。上介饔饩三牢。饪一牢在西,鼎七,羞鼎三。腥一牢,在东,鼎七。堂上之馔六,西夹亦如之。筥及瓮,如上宾。饩一牢。门外米、禾视死牢,牢十车,薪刍倍禾。凡其实与陈,如上宾。下大夫韦弁,用束帛致之。上介韦弁以受,如宾礼。傧之两马束锦。士介四人,皆饩大牢,米百筥,设于门外。宰夫朝服,牵牛以致之。士介朝服,北面再拜稽首受。无傧。宾朝服问卿。卿受于祖庙。下大夫摈。摈者出请事;大夫朝服迎于外门外,再拜。宾不答拜,揖。大夫先入,每门每曲揖。及庙门,大夫揖入。摈者请命。庭实设四皮。宾奉束帛入。三揖,皆行,至于阶,让。宾升一等;大夫从,升堂,北面听命。宾东面致命。大夫降,阶西再拜稽首。宾辞,升成拜。受币堂中西,北面。宾降,出。大夫降,授老币,无傧。

摈者出请事。宾面,如觌币。宾奉币,庭实从,入门右。大夫辞。宾遂左。庭实设,揖让如初。大夫升一等,宾从之。大夫西面,宾称面。大夫对,北面当楣再拜,受币于楹间,南面,退,西面立。宾当楣再拜送币,降,出。大夫降,授老币。

摈者出请事。上介特面,币如觌。介奉币。皮,二人赞。入门右,奠币,再拜。大夫辞。摈者反币。庭实设,介奉币入,大夫揖让如初。介升,大夫再拜受。介降拜,大夫降辞。介升,再拜送币。摈者出请。众介面,如觌币,入门右,奠币,皆再拜。大夫辞,介逆出。摈者执上币出,礼请受,宾辞。大夫答再拜。摈者执上币,立于门中以相拜,士介皆辟。老受摈者币于中庭,士三人坐取群币以从之。摈者出请事。宾出,大夫送于外门外,再拜。宾不顾。摈者退,大夫拜辱。

下大夫尝使至者,币及之。上介朝服、三介,问下大夫,下大夫如卿受币之礼。其面,如宾面于卿之礼。

大夫若不见,君使大夫各以其爵为之受,如主人受币礼,不拜。

夕,夫人使下大夫韦弁归礼。堂上笾豆六,设于户东,西上,二以并,东陈。壶设于东序,北上,二以并,南陈。醙、黍、清,皆两壶。大夫以束帛致之。宾如受饔之礼,傧之乘马束锦。上介四豆、四笾、四壶,受之如宾礼;傧之两马束锦。明日,宾拜礼于朝。

大夫饩宾大牢,米八筐。宾迎,再拜。老牵牛以致之,宾再拜稽首受。老退,宾再拜送。上介亦如之。众介皆少牢,米六筐,皆士牵羊以致之。

公于宾,壹食,再飨。燕与羞,俶献,无常数。宾介皆明日拜于朝。上介壹食壹飨。若不亲食,使大夫各以其爵、朝服致之以侑币。如致饔,无傧。致飨以酬币,亦如之。大夫于宾,壹飨壹食。上介,若食,若飨;若不亲飨,则公作大夫致之以酬币,致食以侑币。

君使卿皮弁,还玉于馆。宾皮弁,袭,迎于外门外,不拜;帅大夫以入。大夫升自西阶,钩楹。宾自碑内听命,升自西阶,自左,南面受圭,退负右房而立。大夫降中庭。宾降,自碑内,东面,授上介于阼阶东。上介出请,宾迎,大夫还璋,如初入。宾裼,迎。大夫贿用束纺。礼玉、束帛、乘皮,皆如还玉礼。大夫出,宾送,不拜。

公馆宾,宾辟,上介听命。聘享,夫人之聘享,问大夫,送宾,公皆再拜。公退,宾从,请命于朝。公辞,宾退。

宾三拜乘禽于朝,讶听之。遂行,舍于郊。公使卿赠,如觌币。受于舍门外,如受劳礼,无傧。使下大夫赠上介,亦如之。使士赠众介,如其觌币。大夫亲赠,如其面币,无傧,赠上介亦如之。使人赠众介,如其面币。士送至于竟。

使者归,及郊,请反命。朝服,载旃,禳,乃入。乃入陈币于朝,西上。上宾之公币、私币皆陈,上介公币陈,他介皆否。束帛各加其庭实,皮左。公南乡。卿进使者,使者执圭垂缫,北面;上介执璋屈缫,立于其左。反命,曰:「以君命聘于某君,某君受币于某宫,某君再拜。以享某君,某君再拜。」宰自公左受玉。受上介璋,致命亦如之。执贿币以告,曰:「某君使某子贿。」授宰。礼玉亦如之。执礼币,以尽言赐礼。公曰:「然。而不善乎!」授上介币,再拜首,公答再拜。私币不告。君劳之,再拜稽首,君答再拜。若有献,则曰:「某君之赐也。君其以赐乎?」上介徒以公赐告,如上宾之礼。君劳之。再拜稽首。君答拜。劳士介亦如之。君使宰赐使者币,使者再拜稽首。赐介,介皆再拜稽首。乃退,介皆送至于使者之门,乃退揖。使者拜其辱。

释币于门,乃至于祢,筵几于室,荐脯醢。觞酒陈。席于阼,荐脯醢,三献。一人举爵,献从者,行酬,乃出。上介至,亦如之。

聘遭丧,入竟,则遂也。不郊劳。不筵几。不礼宾。主人毕归礼,宾唯饔饩之受。不贿,不礼玉,不赠。遭夫人、世子之丧,君不受,使大夫受于庙,其他如遭君丧。遭丧,将命于大夫,主人长衣练冠以受。

聘,君若薨于后,入竟则遂。赴者未至,则哭于巷,衰于馆;受礼,不受飨食。赴者至,则衰而出。唯稍,受之。归,执圭覆命于殡,升自西阶,不升堂。子即位,不哭。辩覆命,如聘。子臣皆哭。与介入,北乡哭。出,袒括发。入门右,即位踊。

若有私丧,则哭于馆,衰而居,不飨食。归。使众介先,衰而从之。

宾入竟而死,遂也。主人为之具,而殡。介摄其命。君吊,介为主人。主人归礼币,必以用。介受宾礼,无辞也。不飨食。归,介覆命,柩止于门外。介卒覆命,出,奉柩送之。君吊,卒殡。若大夫介卒,亦如之。士介死,为之棺敛之,君不吊焉。若宾死,未将命,则既敛于棺,造于朝,介将命。若介死,归覆命,唯上介造于朝。若介死,虽士介,宾既覆命,往,卒殡乃归。

小聘曰问。不享,有献,不及夫人,主人不筵几,不礼。面不升。不郊劳。其礼,如为介,三介。

记。久无事,则聘焉。若有故,则卒聘。束帛加书将命,百名以上书于策,不及百名书于方。主人使人与客读诸门外。客将归,使大夫以其束帛反命于馆。明日,君馆之。既受行,出,遂见宰,问几月之资。使者既受行日,朝同位。出祖,释軷,祭酒脯,乃饮酒于其侧。所以朝天子,圭与缫皆九寸,剡上寸半,厚半寸,博三寸,缫三采六等,朱白仓。问诸侯,朱绿缫,八寸。皆玄纁系,长尺,绚组。问大夫之币,俟于郊,为肆。又□皮马。辞无常,孙而说。辞多则史,少则不达。辞苟足以达,义之至也。辞曰:「非礼也。敢?」对曰:「非礼也。敢辞?」卿馆于大夫,大夫馆于士,士馆于工商。管人为客,三日具沐,五日具浴。飧不致,宾不拜,沐浴而食之。卿,大夫讶。大夫,士讶。士,皆有讶。宾即馆,讶将公命,又见之以其挚。宾既将公事,复见讶以其挚。凡四器者,唯其所宝,以聘可也。宗人授次。次以帷。少退于君之次。上介执圭,如重,授宾。宾入门,皇;升堂,让;将授,志趋;授如争承,下如送;君还,而后退。下阶,发气,怡焉;再三举足,又趋。及门,正焉。执圭,入门,鞠躬焉,如恐失之。及享,发气焉,盈容。众介北面,跄焉。私觌,愉愉焉。出,如舒雁。皇,且行;入门主敬,升堂主慎。凡庭实,随入,左先,皮马相间,可也。宾之币,唯马出,其馀皆东。多货,则伤于德。币美,则没礼。贿,在聘于贿。凡执玉,无藉者袭。礼,不拜至。醴尊于东箱,瓦大一,有丰。荐脯五胑,祭半胑横之。祭醴,再扱始扱一祭,卒再祭。主人之庭实,则主人遂以出,宾之士讶受之。既觌,宾若私献,奉献,将命。摈者入告,出礼辞。宾东面坐奠献,再拜稽首。摈者东面坐取献,举以入告,出礼请受。宾固辞,公答再拜。摈者立于阈外以相拜,宾辟。摈者授宰夫于中庭。若兄弟之国,则问夫人。若君不见,使大夫受。自下听命,自西阶升受,负右房而立。宾降亦降。不礼。币之所及,皆劳,不释服。赐饔,唯羹饪。筮一尸,若昭若穆。仆为祝,祝曰:「孝孙某,孝子某,荐嘉礼于皇祖某甫、皇考某子。」如馈食之礼。假器于大夫。盼及廋车。聘日致饔。明日,问大夫。夕,夫人归礼。既致饔,旬而稍,宰夫始归乘禽,日如其饔饩之数。士中日则二双。凡献,执一双,委其馀于面。禽羞,俶献。比归大礼之日,既受饔饩,请观。讶帅之,自下门入。各以其爵,朝服。士无饔。无饔者无傧。大夫不敢辞,君初为之辞矣。凡致礼,各以其爵,朝服。皆用其飨之加笾豆。无饔者无飨礼。凡饩,大夫黍、粱、稷,筐五斛。既将公事,宾请归。凡宾拜于朝,讶听之。燕,则上介为宾,宾为苟敬。宰夫献。无行,则重贿反币。曰:「子以君命在寡君,寡君拜君命之辱。」君以社稷故,在寡小君,拜。」「君贶寡君,延及二三老,拜。」又拜送。宾于馆堂楹间,释四皮束帛。宾不致,主人不拜。大夫来使,罪,飨之;过,则饩之。其介为介。有大客后至,则先客不飨食,致之。唯大聘有几筵。十斗曰斛,十六斗曰籔,十薮曰秉,二百四十斗,四秉曰筥,十筥曰稯,十稯曰秅,四百秉为一秅。

\hypertarget{header-n44}{%
\subsection{公食大夫礼}\label{header-n44}}

公食大夫之礼。使大夫戒,各以其爵。上介出请,入告。三辞。宾出,拜辱。大夫不答拜,将命。宾再拜稽首。大夫还,宾不拜送,遂从之。宾朝服即位于大门外,如聘。

即位,具。羹定,甸人陈鼎七,当门,南面西上,设扃鼏,鼏若束若编。设洗如飨。小臣具槃匜,在东堂下。宰夫设筵,加席、几。无尊。饮酒、浆饮,俟于东房。凡宰夫之具,馔于东房。

公如宾服迎宾于大门内。大夫纳宾。宾入门左,公再拜;宾辟,再拜稽首。公揖入,宾从。及庙门,公揖入。宾入,三揖。至于阶,三让。公升二等,宾升。大夫立于东夹南,西面北上。士立于门东,北面西上。小臣,东堂下,南面西上。宰,东夹北,西面南上。内官之士在宰东北,西面南上。介,门西,北面西上。公当楣北乡,至壹拜,宾降也,公再拜。宾,西阶东,北面答拜。摈者辞,拜也;公降一等。辞曰:「寡君从子,虽将拜,兴也!」宾粟阶升,不拜,命之成拜,阶上北面再拜稽首。

士举鼎,去鼏于外,次于。陈鼎于碑南,南面西上。右人抽扃,坐奠于鼎西南,顺出自鼎西,左人待载。雍人以俎入,陈于鼎南。旅人南面加匕于鼎,退。大人长盥洗东南,西面北上,序进盥。退者与进者交于前。卒盥,序进,南面匕。载者西面。鱼腊饪。载体进奏。鱼七,缩俎,寝右。肠、胃七,同俎。伦肤七。肠、胃、肤,皆横诸俎,垂之。大夫既匕,匕奠于鼎,逆退,复位。

公降盥,宾降,公辞。卒盥,公壹揖壹让。公升,宾升。宰夫自东房授醯酱,公设之。宾辞,北面坐迁而东迁所。公立于序内,西乡。宾立于阶西,疑立。宰夫自东房荐豆六,设于酱东,西上,韭菹,以东醓醢、昌本;昌本南麋臡以西菁菹、鹿臡。士设俎于豆南,西上,牛、羊、豕,鱼在牛南,腊、肠、胃亚之,肤以为特。旅人取匕,甸人举鼎,顺出,奠于其所。宰夫设黍、稷六簋于俎西,二以并,东北上。黍当牛俎,其西稷,错以终,南陈。大羹湆,不和,实于镫。宰右执镫,左执盖,由门入,升自阼阶,尽阶,不升堂,授公,以盖降,出,入反位。公设之于酱西,宾辞,坐迁之。宰夫设鉶四于豆西,东上,牛以西羊,羊南豕豕以东牛。饮酒,实于觯,加于丰。宰夫右执觯,左执丰,进设于豆东。宰夫东面,坐启簋会,各却于其西。赞者负东房,南面,告具于公。

公再拜,揖食,宾降拜,公辞,宾升,再拜稽首。宾升席,坐取韭菹,以辩擩于醢,上豆之间祭。赞者东面坐取黍,实于左手,辩,又取稷,辩,反于右手,兴,以授宾,宾祭之。三牲之肺不离,赞者辩取之,壹以授宾。宾兴爱,坐祭。兑手,扱上鉶以柶,辩擩之,上鉶之间祭。祭饮酒于上豆之间。鱼、腊、酱、湆不祭。

宰夫授公饭粱,公设之于湆西。宾北面辞,坐迁之。公与宾皆复初位。宰夫□善稻于粱西。士羞庶羞,皆有大、盖,执豆如宰。先者反之,由门入,升自西阶。先者一人升,设于稻南簋西,间容人。旁四列,西北上,膷以东,臐、膮、牛炙。炙南醢以西,牛胾、醢、牛鮨,鮨南羊炙,以东羊胾、醢、豕炙,炙南醢,以西豕胾、芥酱、鱼脍。众人腾羞者尽阶、不升堂,授,以盖降,出。赞者负东房,告备于公。

赞升宾。宾坐席末,取粱,即稻,祭于酱湆间。赞者北面坐,辩取庶羞之大,兴,一以授宾。宾受,兼壹祭之。宾降拜,公辞。宾升,再拜稽首。公答再拜。

宾北面自间坐,左拥簠粱,右执湆,以降。公辞。宾西面坐奠于阶西,东面对,西面坐取之;栗阶升,北面反奠于其所;降辞公。公许,宾升,公揖退于箱。摈者退,负东塾而立。宾坐,遂卷加席,公不辞。宾三饭以湆酱。宰夫执觯浆饮与其丰以进。宾兑手,兴受。宰夫设其丰于稻西。庭实设。宾坐祭,遂饮,奠于丰上。

公受宰夫束帛以侑,西乡立。宾降筵,北面。摈者进相币。宾降辞币,升听命,降拜。公辞。宾升,再拜稽首,受币,当东楹,北面;退,西楹西,东面立。公壹拜,宾降也,公再拜。介逆出。宾北面揖,执庭实以出。公降立。上介受宾币,从者讶受皮。

宾入门左,没霤,北面再拜稽首。公辞,揖让如初,升。宾再拜稽首,公答再拜。宾降辞公,如初。宾升,公揖退于箱。宾卒食会饭,三饮,不以酱湆。

兑手,兴,北面坐,取粱与酱以降,西面坐奠于阶西,东面再拜稽首。公降,再拜。介逆出,宾出。公送于大门内,再拜。宾不顾。

有司卷三牲之俎,归于宾馆。鱼腊不与。

明日,宾朝服拜赐于朝,拜食与侑币,皆再拜稽首。讶听之。

上大夫八豆,八簋,六鉶,九俎,鱼腊皆二俎;鱼,肠胃,伦肤,若九,若十有一,下大夫则若七,若九。庶羞,西东毋过四列。上大夫,庶羞二十,加于下大夫,以雉、兔、鹑、鴽。

若不亲食,使大夫各以其爵、朝服以侑币致之。豆实,实于瓮,陈于楹外,二以并,北陈。簋实,实于筐,陈于楹内、两楹间,二以并,南陈。庶羞陈于碑内,庭实陈于碑外。牛、羊、豕陈于门内,西方,东上。宾朝服以受,如受饔礼。无摈。明日,宾朝服以拜赐于朝。讶听命。

大夫相食,亲戒速。迎宾于门外,拜至,皆如飨拜。降盥。受酱、湆、侑币--束锦也,皆自阼阶降堂受,授者升一等。宾止也。宾执粱与湆,之西序端。主人辞,宾反之。卷加席,主人辞,宾反之。辞币,降一等,主人从。受侑币,再拜稽首。主人送币,亦然。辞于主人,降一等,主人从。卒食,彻于西序端;东面再拜,降出。其他皆如公食大夫之礼。

若不亲食,则公作大夫朝服以侑币致之。宾受于堂。无摈。

记。不宿戒,戒不速。不授几。无阼席。亨于门外东方。司宫具几,与蒲筵常缁布纯,加萑席寻玄帛纯,皆卷自末。宰夫筵,出自东房。宾之乘车在大门外西方,北面立。鉶芼,牛藿,羊苦,豕薇,皆有滑。赞者盥,从俎升。簠有盖幂。凡炙无酱。上大夫:蒲筵加萑席。其纯,皆如下大夫纯。卿摈由下。上赞,下大夫也。上大夫,庶羞。酒饮,浆饮,庶羞可也。拜食与侑币,皆再拜稽首。

\hypertarget{header-n48}{%
\subsection{觐礼}\label{header-n48}}

觐礼。至于郊,王使人皮弁用璧劳。侯氏亦皮弁迎于帷门之外,再拜。使者不答拜,遂执玉,三揖。至于阶,使者不让,先升。侯氏升听命,降,再拜稽首,遂升受玉。使者左还而立,侯氏还璧,使者受。侯氏降,再拜稽首,使者乃出。侯氏及止使者,使者乃入。侯氏与之让升。侯氏先升,授几。侯氏拜送几;使者设几,答拜。侯氏用束帛、乘马傧使者,使者再拜受。侯氏再拜送币。使者降,以左骖出。侯氏送于门外,再拜。侯氏遂从之。

天子赐舍,曰:「伯父,女顺命于王所,赐伯父舍!」侯氏再拜稽首,傧之束帛、乘马。

天子使大夫戒,曰:「某日,伯父帅乃初事。」侯氏再拜稽首。

诸侯前朝,皆受舍于朝。同姓西面北上,异姓东面北上。

侯氏裨冕,释币于祢。乘墨车,载龙旂、弧韣乃朝以瑞玉,有缫。

天子设斧依于户牖之间,左右几。天子衮冕,负斧依。啬夫承命,告于天子。天子曰:「非他,伯父实来,予一人嘉之。伯父其入,予一人将受之。」侯氏入门右,坐奠圭,再拜稽首。摈者谒。侯氏坐取圭,升致命。王受之玉。侯氏降,阶东北面再拜稽首。摈者延之,曰:「升!」升成拜,乃出。

四享皆束帛加璧,庭实唯国所有。奉束帛,匹马卓上,九马随之,中庭西上,奠币,再拜稽首。摈者曰:「予一人将受之。」侯氏升,致命。王抚玉。侯氏降自西阶,东面授宰币,西阶前再拜稽首,以马出,授人,九马随之。事毕。

乃右肉袒于庙门之东。乃入门右,北面立,告听事。摈者谒诸天子。天子辞于侯氏,曰:「伯父无事,归宁乃邦!」侯氏再拜稽首,出,自屏南适门西,遂入门左,北面立,王劳之。再拜稽首。摈者延之,曰:「升!」升成拜,降出。

天子赐侯氏以车服。迎于外门外,再拜。路先设,西上,路下四,亚之,重赐无数,在车南。诸公奉箧服,加命书于其上,升自西阶,东面,大史是右。侯氏升,西面立。大史述命。侯氏降两阶之间;北面再拜稽首,升成拜。大史加书于服上,侯氏受。使者出。侯氏送,再拜,傧使者,诸公赐服者,束帛、四马,傧大史亦如之。

同姓大国则曰伯父,其异姓则曰伯舅。同姓小邦则曰叔父,其异姓小邦则曰叔舅。

飨,礼,乃归。

诸侯觐于天子,为宫方三百步,四门,坛十有二寻、深四尺,加方明于其上。方明者,木也,方四尺,设六色,东方青,南方赤,西方白,北方黑,上玄,下黄。设六玉,上圭,下璧,南方璋,西方琥,北方璜,东方圭。上介皆奉其君之旂,置于宫,尚左。公、侯、伯、子、男,皆就其旂而立。四传摈。天子乘龙,载大旂,像日月、升龙、降龙;出,拜日于东门之外,反祀方明。礼日于南门外,礼月与四渎于北门外,礼山川丘陵于西门外。

祭天,燔柴。祭山、丘陵,升。祭川,沉。祭地,瘗。

记。几,俟于东箱。偏驾不入王门。奠圭于缫上。

\hypertarget{header-n52}{%
\subsection{丧服}\label{header-n52}}

丧服,斩衰裳,苴絰杖,绞带,冠绳缨,菅屦者。诸侯为天子,君,父为长子,为人后者。妻为夫,妾为君,女子子在室为父,布总,箭笄,髽,衰,三年。子嫁,反在父之室,为父三年。公士、大夫之众臣,为其君布带、绳屦。

疏衰裳齐,牡麻絰,冠布缨,削杖,布带,疏屦三年者,父卒则为母,继母如母,慈母如母,母为长子。

疏衰裳齐,牡麻絰,冠布缨,削杖,布带,疏屦,期者,父在为母,妻,出妻之子为母。出妻之子为父后者则为出母无服。父卒,继母嫁,从,为之服,报。

不杖,麻屦者。祖父母,世父母,叔父母;大夫之适子为妻,昆弟;为众子,昆弟之子;大夫之庶子为适昆弟适孙。为人后者,为其父母,报。女子子适人者为其父母、昆弟之为父后者,继父同居者,为夫之君。姑、姊妹、女子子适人无主者,姑、姊妹报。为君之父、母、妻、长子、祖父母。妾为女君。妇为舅姑,夫之昆弟之子。公妾、大夫之妾为其子。女子子为祖父母。大夫之子为世父母、叔父母、子、昆弟、昆弟之子,姑、姊妹、女子子无主者,为大夫命妇者,唯子不报。大夫为祖父母、适孙为士者。公妾以及士妾为其父母。

疏衰裳齐,牡麻絰,无受者。寄公为所寓,丈夫、妇人为宗子、宗子之母、妻,为旧君、君之母、妻,庶人为国君;大夫在外,其妻、长子为旧国君;继父不同居者,曾祖父母,大夫为宗子,旧君。曾祖父母为士者如众人,女子子嫁者、未嫁者为曾祖父母。

大功布衰裳,牡麻絰,无受者:子、女子子之长殇、中殇,叔父之长殇、中殇,姑、姊妹之长殇、中殇,昆弟之长殇、中殇,夫之昆弟之子、女子子之长殇、中殇,适孙之长殇、中殇,大夫之庶子为适昆弟子之长殇、中殇,公子之长殇、中殇,大夫为适子之长殇、中殇。其长殇皆九月,缨絰;其中殇,七月,不缨絰。

大功布衰裳,牡麻絰缨,布带,三月。受以小功衰,即葛,九月者:姑、姊妹、女子子适人者,从父昆弟;为人后者为其昆弟,庶孙;适妇,女子子适人者为众昆弟;侄丈夫妇人,报。夫之祖父母、世父母、叔父母,大夫为世父母、叔父母、子、昆弟、昆弟之子为士者;公之庶昆弟、大夫之庶子为母、妻、昆弟,皆为其从父昆弟之为大夫者;为夫之昆弟之妇人子适人者;大夫之妾为君之庶子;女子子嫁者、未嫁者,为世父母、叔父母、姑、姊妹,大夫、大夫之妻、大夫之子、公之昆弟为姑、姊妹、女子子嫁于大夫者,君为姑、姊妹、女子子嫁于国君者。

繐衰裳,牡麻絰,既葬除之者。诸侯之大夫为天子。

小功布衰裳,澡麻带絰,五月者。叔父之下殇,适孙之下殇,昆弟之下殇,大夫庶子为适昆弟之下殇,为姑、姊妹、女子子之下殇,为人后者为其昆弟、从父昆弟之长殇,为夫之叔父之长殇;昆弟之子、女子子、夫之昆弟之子、女子子之下殇;为侄、庶孙丈夫妇人之长殇;大夫、公之昆弟、大夫之子,为其昆弟、庶子、姑、姊妹、女子子之长殇;大夫之妾为庶子之长殇。

小功布衰裳,牡麻絰,即葛,五月者。从祖祖父母,从祖父母,报;从祖昆弟,从父姊妹、孙适人者,为人后者为其姊妹适人者,为外祖父母;从母,丈夫妇人报;夫之姑、姊妹,娣、姒妇,报;大夫、大夫之子、公之昆弟为从父昆弟,庶孙,姑、姊妹、女子子适士者;大夫之妾为庶子适人者;庶妇;君之父母、从母;君子子为庶母慈己者。

缌麻,三月者。族曾祖父母,族祖父母,族父母,族昆弟;庶孙之妇,庶孙之中殇;从祖姑、姊妹适人者,报;从祖父、从祖昆弟之长殇;外孙,从父昆弟侄之下殇,夫之叔父之中殇、下殇;从母之长殇,报;庶子为父后者,为其母;士为庶母;贵臣、贵妾;乳母,从祖昆弟之子,曾孙,父之姑,从母昆弟,甥,婿,妻之父母,姑之子,舅,舅之子;夫之姑姊妹之长殇;夫之诸祖父母,报;君母之昆弟;从父昆弟之子长殇,昆弟之孙之长殇,为夫之从父昆弟之妻。

记。公子为其母,练冠,麻,麻衣縓缘;为其妻,縓冠,葛絰,带,麻衣縓缘。皆既葬除之。大夫、公之昆弟,大夫之子,于兄弟降一等。为人后者,于兄弟降一等,报;于所为后之子、兄弟,若子。兄弟皆在他邦,加一等。不及知公母,与兄弟居,加一等。朋友皆在他邦,袒免,归则已。朋友,麻。君之所为兄弟服,室老降一等。夫之所为兄弟服,妻降一等。庶子为后者,为其外祖父母、从母、舅,无服。不为后,如邦人。宗子孤为殇,大功衰,小功衰,皆三月。亲,则月算如邦人。改葬,缌。童子,唯当室缌。凡妾为私兄弟,如邦人。大夫吊于命妇,锡衰。命妇吊于大夫,亦锡衰。女子子适人者为其父母,妇为舅姑,恶笄有首以髽。卒哭,子折笄首以笄,布总。

妾为女君、君之长子,恶笄有首,布总。凡衰,外削幅;裳,内削幅,幅三衣包。若齐,裳内,衰外。负,广出于适寸。适,博四寸,出于衰。衰,长六寸,博四寸。衣带,下尺。衽,二尺有五寸。袂,属幅。衣,二尺有二寸。祛,尺二寸。衰三升,三升有半。其冠六升。以其冠为受,受冠七升。齐衰四升,其冠七升。以其冠为受,受冠八升。繐衰四升有半,其冠八升。大功八升,若九升。小功十升,若十一升。

\hypertarget{header-n56}{%
\subsection{士丧礼}\label{header-n56}}

士丧礼。死于适室,幠用敛衾。复者一人以爵弁服,簪裳于衣,左何之,扱领于带;升自前东荣、中屋,北面招以衣,曰:「皋某复!」三,降衣于前。受用箧,升自阼阶,以衣尸。复者降自后西荣。

楔齿用角柶。缀足用燕几。奠脯醢、醴酒。升自阼阶,奠于尸东。帷堂。

乃赴于君。主人西阶东,南面,命赴者,拜送。有宾,则拜之。

入,坐于床东。众主人在其后,西面。妇人侠床,东面。亲者在室。众妇人户外北面,众兄弟堂下北面。

君使人吊。彻帷。主人迎于寝门外,见宾不哭,先入,门右北面。吊者入,升自西阶,东面。主人进中庭,吊者致命。主人哭,拜稽颡,成踊。宾出,主人拜送于外门外。

君使人襚。彻帷。主人如初。襚者左执领,右执要,入,升致命。主人拜如初。襚者入衣尸,出。主人拜送如初。唯君命,出,升降自西阶。遂拜宾,有大夫则特拜之。即位于西阶下,东面,不踊。大夫虽不辞,入也。

亲若襚,不将命,以即陈。庶兄弟襚,使人以将命于室,主人拜于位,委衣于尸东床上。朋友襚,亲以进,主人拜,委衣如初,退,哭,不踊。彻衣者,执衣如襚,以适房。

为铭,各以其物。亡,则以缁长半幅,\textless{}赤巠\textgreater{}末长终幅,广三寸。书铭于末,曰:「某氏某之柩。」竹杠长三尺,置于宇西阶上。

甸人掘坎于阶间,少西。为垼于西墙下,东乡。新盆,槃,瓶,废敦,重鬲,皆濯,造于西阶下。

陈袭事于房中,西领,南上,不綪。明衣裳,用布。\textless{}髟会\textgreater{}笄用桑,长四寸,紌中。布巾,环幅,不凿。掩,练帛广终幅,长五尺,析其末。瑱,用白纩。幎目,用缁,方尺二寸,\textless{}赤巠\textgreater{}里,着,组系。握手,用玄,纁里,长尺二寸,广五寸,牢中旁寸,着,组系。决,用正王棘,若檡棘,组系,纩极二。冒,缁质,长与手齐,\textless{}赤巠\textgreater{}杀,掩足。爵弁服、纯衣、皮弁服、褖衣、缁带、韎韐、竹笏。夏葛屦,冬白屦,皆繶缁絇纯,组綦系于踵。庶襚继陈,不用。

贝三,实于笄。稻米一豆,实于筐。沐巾一,浴巾二,皆用谷,于□。栉,于箪。浴衣,于箧。皆馔于西序下,南上。

管人汲,不说繘,屈之。祝淅米于堂,南面,用盆。管人尽阶,不升堂,受潘,煮于垼,用重鬲。祝盛米于敦,奠于贝北。士有冰,用夷槃可也。外御受沐入。主人皆出,户外北面。乃沐,栉,挋用巾,浴,用巾,挋用浴衣。渜濯弃于坎。蚤,揃如他日。\textless{}髟会\textgreater{}用组,扱笄,设明衣裳。主人入,即位。

商祝袭祭服,褖衣次。主人出,南面,左袒,扱诸面之右,盥于盆上,洗贝,执以入。宰洗柶,建于米,执以从。商祝执巾从入,当牖北面,彻枕,设巾,彻楔,受贝,奠于尸西。主人由足西,床上坐,东面。祝又受米,奠于贝北。宰从立于床西,在右。主人在扱米,实于右,三,实一贝。左、中亦如之。又实米,唯盈。主人袭,反位。

商祝掩,瑱,设幎目,乃屦,綦结于跗,连絇。乃袭,三称。明衣不在算。设韐、带,搢笏。设决,丽于腕,自饭持之,设握,乃连腕。设冒,櫜之,幠用衾。巾、柶、鬊、蚤埋于坎。

重木,刊凿之。甸人置重于中庭,三分庭,一在南。夏祝鬻馀饭,用二鬲于西墙下。幂用疏布,久之,系用靲,县于重,幂用苇席,北面,左衽,带用靲,贺之,结于后。祝取铭置于重。

厥明,陈衣于房,南领,西上,綪,绞横三缩一,广终幅,析其末。缁衾,赬里,无紞。祭服次,散衣次,凡十有九称,陈衣继之,不必尽用。

馔于东堂下,脯醢醴酒。幂奠用功布,实于箪,在馔东。设盆盥于馔东,有巾。

苴絰,大鬲,下本在左,要絰小焉;散带垂,长三尺。牡麻絰,右本在上,亦散带垂。皆馔于东方。妇人之带,牡麻结本,在房。

床第,夷衾,馔于西坫南。西方盥,如东方。

陈一鼎于寝门外,当东塾,少南,西面。其实特豚,四鬄,去蹄,两胉,脊、肺。设扃鼏,鼏西末。素俎在鼎西,西顺,覆匕,东柄。

士盥,二人以并,东面立于西阶下。布席于户内,下莞上簟。商祝布绞衾、散衣、祭服。祭服不倒,美者在中。士举迁尸,反位。设床第于两楹之间,衽如初,有枕。卒敛,彻帷。主人西面冯尸,踊无算;主妇东面冯,亦如之。主人髺发,袒,众主人免于房。妇人髽于室。士举,男女奉尸,侇于堂,幠无夷衾。男女如室位,踊无算。主人出于足,降自西阶。众主人东即位。妇人阼阶上西面。主人拜宾,大夫特拜,士旅之,即位踊,袭絰于序东,复位。

乃奠。举者盥,右执匕,却之,左执俎,横摄之,入,阼阶前西面错,错俎北面。右人左执匕,抽扃予左手,兼执之,取鼏,委于鼎北,加扃,不坐。乃朼,载。载两髀于两端,两肩亚,两□白亚,脊、肺在于中,皆覆。进柢,执而俟。夏祝及执事盥,执醴先,酒、脯、醢、俎从,升自阼阶。丈夫踊。甸人彻鼎,巾待于阼阶下。奠于尸东,执醴酒,北面西上。豆错,俎错于豆东。立于俎北,西上。醴酒错于豆南。祝受巾,巾之,由足降自西阶。妇人踊。奠者由重南,东。丈夫踊。宾出,主人拜送于门外。

乃代哭,不以官。

有襚者,则将命,摈者出请,入告。主人待于位。摈者出,告须,以宾入。宾入中庭,北面致命。主人拜稽颡。宾升自西阶,出于足,西面委衣如于室礼,降,出。主人出,拜送。朋友亲襚,如初仪,西阶东,北面哭,踊三,降,主人不踊。襚者以褶,则必有裳,执衣如初。彻衣者亦如之,升,降自西阶,以东。

宵,为燎于中庭。厥明,灭燎。陈衣于房,南领,西上,綪。绞紟,衾二。君襚,祭服,散衣,庶襚,凡三十称,紟不在算。不必尽用。东方之馔,两瓦甒,其实醴酒,角觯,木柶;毼豆两,其实葵菹芋、蠃醢,两笾,无縢,布巾,其实栗,不择,脯四脡。奠席在馔北,敛席在其东。掘肂见衽。棺入,主人不哭。升棺用轴,盖在下。熬黍稷各二筐,有鱼腊,馔于西坫南。陈三鼎于门外,北上。豚合升,鱼鱄鲋九,腊左胖,髀不升,其他皆如初。烛俟于馔东。

祝、彻、盥于门外,入,升自阼阶,丈夫踊。祝彻巾,授执事者以待。彻馔,先取醴酒,北面。其馀取先设者,出于足,降自西阶,妇人踊。设于序西南,当西荣,如设于堂。醴酒位如初,执事豆北,南面东上。乃适馔。

帷堂。妇人尸西东面。主人及亲者升自西阶,出于足,西面袒。士盥位如初。布席如初。商祝布绞、紟、衾、衣,美者在外,君襚不倒。有大夫,则告。士举迁尸,复位。主人踊无算。卒敛,彻帷。主人冯如初,主妇亦如之。

主人奉尸敛于棺,踊如初,乃盖。主人降,拜大夫之后至者,北面视肂。众主人复位。妇人东复位。设熬,旁一筐,乃涂。踊无算。卒涂,祝取铭置于□聿。主人复闪位,踊,袭。

乃奠。烛升自阼阶,祝执巾,席从,设于奥,东面。祝反降,及执事执馔。士盥,举鼎入,西面北上,如初。载,鱼左首,进鬐,三列,腊进柢。祝执醴如初,酒、豆、笾、俎从,升自阼阶,丈夫踊。甸人彻鼎。奠由楹内入于室。醴酒北面。设豆,右菹,菹南栗,栗东脯。豚当豆。鱼次腊特于俎北,醴酒在笾南。巾如初。既错者出,立于户西,西上。祝后,阖户,先由楹西,降自西阶,妇人踊。奠者由重南东,丈夫踊。

宾出,妇人踊,主人拜送于门外,入,及兄弟北面哭殡。兄弟出,主人拜送于门外。众主人出门,哭止,皆西面于东方。阖门。主人揖,就次。

君若有赐焉,则视敛。既布衣,君至,主人出迎于外门外,见马首,不哭,还,入门右,北面,及众主人袒。巫止于庙门外,祝代之。小臣二人执戈先,二人后。君释采,入门,主人辟。君升自阼阶,西乡。祝负墉,南面,主人中庭。君哭。主人哭,拜稽颡,成踊,出。君命反行事,主人复位。君升主人,主人西楹东,北面。升公卿大夫,继主人,东上。乃敛。卒,公卿大夫逆降,复位。主人降,出。君反主人,主人中庭。君坐抚,当心。主人拜稽颡,成踊,出。君反之,复初位。众主人辟于东壁,南面。君降,西乡,命主人冯尸。主人升自西阶,由足,西面冯尸,不当君所,踊。主妇东面冯,亦如之。奉尸敛于棺,乃盖,主人降,出。君反之,入门左,视涂。君升即位,众主人复位,卒涂,主人出,君命之反奠。入门右,乃奠,升自西阶。君要节而踊,主人从踊。卒奠,主人出,哭者止。君出门,庙中哭。主人不哭,辟。君式之。贰车毕乘,主人哭,拜送。袭,入即位,众主人袭,拜大夫之后至者,成踊。宾出,主人拜送。

三日,成服,杖,拜君命及众宾。不拜棺中之赐。

朝夕哭,不辟子卯。妇人即位于堂,南上,哭。丈夫即位于门外,西面北上;外兄弟在其南,南上;宾继之,北上。门东,北面西上;门西,北面东上;西方,东面北上。主人即位,辟门。妇人拊心,不哭。主人拜宾,旁三,右还,入门,哭。妇人踊。主人堂下,直东序,西面。兄弟皆即位,如外位。卿大夫在主人之南。诸公门东,少进。他国之异爵者门西,少进。敌,则先拜他国之宾。凡异爵者,拜诸其位。彻者盥于门外,烛先入,升自阼阶。丈夫踊。祝取醴,北面;取酒,立于其东;取豆、笾、俎,南面西上。祝先出,酒、豆、笾、俎序从,降自西阶。妇人踊。设于序西南,直西荣。醴酒北面西上,豆西面,错。立于豆北,南面。笾、俎既错,立于执豆之西,东上。酒错,复位,醴错于西,遂先,由主人之北适馔。乃奠,醴、酒、脯、醢升。丈夫踊,入。如初设,不巾。错者出,立于户西,西上。灭烛,出。祝阖门,先降自西阶。妇人踊。奠者由重南,东。丈夫踊。宾出,妇人踊,主人拜送。众主人出,妇人踊。出门,哭止。皆复位。阖门。主人卒拜送宾,揖众主人,乃就次。

朔月,奠用特豚、鱼腊,陈三鼎如初。东方之馔亦如之。无笾,有黍稷。用瓦敦,有盖,当笾位。主人拜宾,如朝夕哭。卒彻,举鼎入,升,皆如初奠之仪。卒朼,释匕于鼎,俎行。朼者逆出,甸人彻鼎。其序,醴酒、菹醢、黍稷、俎。其设于室、豆错,俎错,腊特,黍稷当笾位。敦启会,却诸其南。醴酒位如初。祝与执豆者巾,乃出。主人要节而踊,皆如朝夕哭之仪。月半不殷奠。有荐新,如朔奠。彻朔奠,先取醴酒,其馀取先设者。敦启会,面足。序出,如入。其设于外,如于室。

筮宅,冢人营之。掘四隅,外其壤。掘中,南其壤。既朝哭,主人皆往,兆南北面,免絰。命筮者在主人之右。筮者东面,抽上韇,兼执之,南面受命。命曰:「哀子某,为其父某甫筮宅。度兹幽宅,兆基无有后艰?」筮人许诺,不述命,右还,北面,指中封而筮。卦者在左。卒筮,执卦以示命筮者。命筮者受视,反之,东面。旅占,卒,进告于命筮者与主人:「占之曰从。」主人絰,哭,不踊。若不从,筮择如初仪。归,殡前北面哭,不踊。

既井椁,主人西面拜工,左还椁,反位,哭,不踊。妇人哭于堂。献材于殡门外,西面北上,綪。主人遍视之,如哭椁。献素、献成亦如之。

卜日,既朝哭,皆复外位。卜人先奠龟于西塾上,南首,有席。楚焞置于燋,在龟东。族长莅卜,及宗人吉服立于门西,东面南上。占者三人在其南,北上。卜人及执燋、席者在塾西。阖东扉,主妇立于其内。席于闑西阈外。宗人告事具。主人北面,免絰,左拥之。莅卜即位于门东,西面。卜人抱龟燋,先奠龟,西首,燋在北。宗人受卜人龟,示高。莅卜受视,反之。宗人还,少退,受命。命曰:「哀子某,来日某,卜葬其父某甫。考降,无有近悔?」许诺,不述命;还即席,西面坐;命龟,兴;授卜人龟,负东扉。卜人坐,作龟,兴。宗人受龟,示莅卜。莅卜受视,反之。宗人退,东面。乃旅占,卒,不释龟,告于莅卜与主人:「占曰某日从。」授卜人龟。告于主妇,主妇哭。告于异爵者。使人告于众宾。卜人彻龟。宗人告事毕。主人絰,入,哭,如筮宅。宾出,拜送,若不从,卜择如初仪。

\hypertarget{header-n60}{%
\subsection{既夕礼}\label{header-n60}}

既夕哭,请启期,告于宾。

夙兴,设盥于祖庙门外。陈鼎皆如殡,东方之馔亦如之。夷床馔于阶间。

二烛俟于殡门外。丈夫髽,散带垂,即位如初。妇人不哭。主人拜宾,入,即位,袒。商祝免袒,执功布入,升自西阶,尽阶,不升堂。声三,启三,命哭。烛入。祝降,与夏祝交于阶下。取铭置于重。踊无算。商祝拂柩用功布,幠用夷衾。

迁于祖,用轴。重先,奠从,烛从,柩从,烛从,主人从。升自西阶。奠俟于下,东面北上。主人从升,妇人升,东面。众主人东即位。正柩于两楹间,用夷床。主人柩东,西面。置重如初。席升设于柩西。奠设如初,巾之,升降自西阶。主人踊无算,降,拜宾,即位,踊,袭。主妇及亲者由足,西面。

荐车,直东荣,北輈。质明,灭烛。彻者升自阼阶,降自西阶。乃奠如初,升降自西阶。主人要节而踊。荐马,缨三就,入门,北面,交辔,圉人夹牵之。御者执策立于马后。哭成踊,右还,出。宾出,主人送于门外。

有司请祖期。曰:「日侧。」主人入,袒。乃载,踊无算。卒束。袭。降奠,当前束。商祝饰柩,一池,纽前\textless{}赤巠\textgreater{}后缁,齐三采,无贝。设披。属引。

陈明器于乘车之西。折,横覆之。抗木,横三,缩二。加抗席三。加茵,用疏布,缁翦,有幅,亦缩二横三。器西南上,綪。茵。苞二。筲三,黍,稷,麦。瓮三,醯,醢,屑。幂用疏布。甒二,醴,酒。幂用功布。皆木桁,久之。用器:弓矢,耒耜,两敦,两杅,槃,匜。匜实于槃中,南流。无祭器。有燕乐器可也。役器,甲,胄,干,笮。燕器,杖,笠,翣。

彻奠,巾席俟于西方。主人要节而踊,袒。商祝御柩,乃祖。踊,袭,少南,当前束。妇人降,即位于阶间。祖,还车不还器。祝取铭,置于茵。二人还重,左还。布席,乃奠如初,主人要节而踊。荐马如初。宾出。主人送,有司请葬期。入,复位。

公賵玄纁束,马两。摈者出请,入告。主人释杖,迎于庙门外,不哭。先入门右,北面,及众主人袒。马入设。宾奉币,由马西当前辂,北面致命。主人哭,拜稽颡,成踊。宾奠币于栈左服,出。宰由主人之北,举币以东。士受马以出。主人送于外门外,拜,袭,入复位,杖。

宾賵者将命,摈者出请,入告,出告须。马入设,宾奉币。摈者先入,宾从,致命如初。主人拜于位,不踊。宾奠币如初,举币、受马如按。摈者出请。若奠,入告,出,以宾入,将命如初。士受羊,如受马。又请。若赙,入告。主人出门左,西面。宾东面将命,主人拜,宾坐委之;宰由主人之北,东面举之,反位。若无器,则捂受之。又请,宾告事毕,拜送入。赠者将命,摈者出请,纳宾如初。宾奠币如初。若就器,则坐奠于陈。凡将礼,必请而后拜送。兄弟,賵、奠可也。所知,则賵而不奠。知死者赠,知生者赙。书賵于方,若九,若七,若五。书遣于策。乃代哭,如初。宵,为燎于门内之右。

厥明,陈鼎五于门外,如初。其实。羊左胖,髀不升,肠五,胃五,离肺。豕亦如之,豚解,无肠胃。鱼、腊、鲜兽,皆如初。东方之馔:四豆,脾析,蜱醢,葵菹,蠃醢;四笾,枣,糗,栗,脯;醴,酒。陈器。灭燎。执烛,侠辂,北面。宾入者,拜之。彻者入,丈夫踊。设于西北,妇人踊。彻者东,鼎入,乃奠。豆南上,綪。笾,蠃醢南,北上,綪。俎二以成,南上,不綪。特鲜兽。醴、酒在笾西,北上。奠者出,主人要节而踊。

甸人抗重。出自道,道左倚之。荐马,马出自道,车各从其马,驾于门外,西面而俟,南上。彻者入,踊如初。彻巾,苞牲,取下体。不以鱼腊。行器,茵、苞、器序从,车从。彻者出。踊如初。

主人之史请读賵,执算从。柩东,当前束,西面。不命毋哭,哭者相止也。唯主人主妇哭。烛在右,南面。读书,释算则坐。卒,命哭,灭烛,书与算执之以逆出。公史自西方,东面,命毋哭,主人、主妇皆不哭。读遣,卒,命哭,灭烛,出。

商祝执功布以御柩。执披。主人袒。乃行。踊无算。出宫,踊,袭。至于邦门,公使宰夫赠玄纁束。主人去杖,不哭,由左听命。宾由右致命。主人哭,拜稽颡。宾升,实币于盖,降。主人拜送,复位,杖。乃行。

至于圹。陈器于道东西,北上。茵先入。属引。主人袒。众主人西面,北上。妇人东面。皆不哭。乃窆。主人哭,踊无算。袭,赠用制币,玄纁束,拜稽颡,踊如初。卒,袒,拜宾。主妇亦拜宾;即位,拾踊三,袭。宾出,则拜送。藏器于旁,加见。藏苞筲于旁。加折,却之。加抗席,覆之。加抗木。实土三。主人拜乡人。即位,踊,袭,如初。

乃反哭,入,升自西阶,东面。众主人堂下东面,北上。妇人入,丈夫踊,升自阼阶。主妇入于室,踊,出即位,及丈夫拾踊,三。宾吊者升自西阶,曰:「如之何!」主人拜稽颡。宾降,出。主人送于门外,拜稽颡。遂适殡宫,皆如启位,拾踊三。兄弟出,主人拜送。众主人出门,哭止,阖门。主人揖众主人,乃就次。

犹朝夕哭,不奠。三虞。卒哭。明日,以其班祔。

记。士处适寝,寝东首于北墉下。有疾,疾者齐。养者皆齐,彻琴瑟。疾病,外内皆扫。彻亵衣,加新衣。御者四人,皆坐持体。属纩,以俟绝气。男子不绝于妇人之手,妇人不绝于男子之手。乃行祷于五祀。乃卒。主人啼,兄弟哭。设床第,当牖。衽,下莞上簟,设枕。迁尸。复者朝服,左执领,右执要,招而左。楔,貌如轭,上两末。缀足用燕几,校在南,御者坐持之。即床而奠,当腢,用吉器。若醴,若酒,无巾柶。赴曰:「君之臣某死。」赴母、妻、长子,则曰:「君之臣某之某死。」室中,唯主人、主妇坐。兄弟有命夫命妇在焉,亦坐。尸在室,有君命,众主人不出。襚者委衣于床,不坐。其襚于室,户西北面致命。夏祝淅米,差盛之。御者四人,抗衾而浴,示亶第。其母之丧,则内御者浴,鬠无笄。设明衣,妇人则设中带。卒洗,贝反于笄,实贝,柱右齻左齻塞耳。掘坎,南顺,广尺,轮二尺,深三尺;南其壤。垼,用块。明衣裳,用幕布,袂属幅,长下膝。有前后裳,不辟,长及觳。縓綼緆。缁纯。设握,里亲肤,系钩中指,结于腕。甸人筑坅坎。隶人涅厕。既袭,宵为燎于中庭。厥明,灭燎,陈衣。凡绞紟用布,伦如朝服。设棜于东堂下,南顺,齐于坫。馔于其上两甒醴、酒,酒在南。篚在东,南顺,实角觯四,木柶二,素勺二。豆在甒北,二以并,笾亦如之。凡笾豆,实具设,皆巾之。觯,俟时而酌,柶覆加之,面枋;及错,建之。小敛,辟奠不出室。无踊节。既冯尸,主人袒,髺发,绞带;众主人布带。大敛于阼。大夫升自西阶,阶东,北面东上。既冯尸,大夫逆降,复位。巾奠,执烛者灭烛出,降自阼阶,由主人之北,东。既殡,主人说髦。三日绞垂。冠六升,外縪,缨条属,厌。衰三升。履外纳。杖下本,竹桐一也。居倚庐,寝苫枕块。不说絰带。哭昼夜无时。非丧事不言。歠粥,朝一溢米,夕一溢米。不食菜果。主人乘恶车,白狗幦,蒲蔽,御以蒲菆,犬服,木錧,约绥,约辔,木镳,马不齐髦。主妇之车亦与之,疏布示炎。贰车,白狗摄服,其仓皆如乘车。

朔月,童子执帚,却之,左手奉之,从彻者而入。比奠,举席,扫室,聚诸\{宀交\},布席如初。卒奠,扫者执帚,垂末内鬣,从执烛者而东。燕养、馈羞、汤沐之馔,如他日。朔月若荐新,则不馈于下室。筮宅,冢人物土。卜日吉,告从于主妇;主妇哭,妇人皆哭;主妇升堂,哭者皆止。启之昕,外内不哭。夷床,輁轴,馔于西阶东。其二庙,则馔于祢庙,如小敛奠;乃启。朝于祢庙,重止于门外之西,东面。柩入,升自西阶。正柩于两楹间。奠止于西阶之下,东面北上。主人升,柩东,西面。众主人东即位,妇人从升,东面。奠升,设于柩西,升降自西阶,主人要节而踊。烛先入者,升堂,东楹之南,西面;后入者,西阶东,北面,在下。主人降,即位。彻,乃奠,乃降自西阶,主人踊如初。祝及执事举奠,巾席从而降,柩从、序从如初适祖。荐乘车,鹿浅幦,干,笮,革靾,载旃,载皮弁服,缨、辔、贝勒县于衡。道车,载朝服。稿车,载蓑笠。将载,祝及执事举奠,户西,南面东上。卒束前而降,奠席于柩西。巾奠,乃墙。抗木,刊。茵着,用荼,实绥泽焉。苇苞,长三尺,一编。菅筲三,其实皆瀹。祖,还车不易位。执披者,旁四人。凡赠币,无常。凡糗,不煎。唯君命,止柩于堩,其馀则否。车至道左,北面立,东上。柩至于圹,敛服载之。卒窆而归,不驱。君视敛,若不待奠,加盖而出;不视敛,则加盖而至,卒事。既正柩,宾出,遂、匠纳车于阶间。祝馔祖奠于主人之南,当前辂,北上,巾之。弓矢之新,沾功。有弭饰焉,亦张可也。有柲。设依挞焉。有韣。猴矢一乘,骨镞,短卫。志矢一乘,轩輖中,亦短卫。

\hypertarget{header-n64}{%
\subsection{士虞礼}\label{header-n64}}

士虞礼。特豕馈食,侧亨于庙门外之右,东面。鱼腊爨亚之,北上。饎爨在东壁,西面。设洗于西阶西南,水在洗西,篚在东。尊于室中北墉下,当户,两甒醴、酒,酒在东。无禁,幂用絺布,加勺,南枋。素几,苇席,在西序下。苴刌茅,长五寸,束之,实于篚,馔于西坫上。馔两豆菹、醢于西楹之东,醢在西,一鉶亚之。从献豆两亚之,四笾亚之,北上。馔黍稷二敦于阶间,西上,藉用苇席。匜水错于槃中,南流,在西阶之南,箪巾在其东。陈三鼎于门外之右,北面,北上,设扃鼏。匕俎在西塾之西。羞燔俎在内西塾上,南顺。

主人及兄弟如葬服,宾执事者如吊服,皆即位于门外,如朝夕临位。妇人及内兄弟服、即位于堂,亦如之。祝免,澡葛絰带,布席于室中,东面,右几,降,出,及宗人即位于门西,东面南上。宗人告有司具,遂请拜宾。如临,入门哭,妇人哭。主人即位于堂,众主人及兄弟、宾即位于西方,如反哭位。祝入门,左,北面。宗人西阶前北面。

祝盥,升,取苴降,洗之,升,入设于几东席上,东缩,降,洗觯,升,止哭。主人倚杖,入。祝从,在左,西面。赞荐菹醢,醢在北。佐食及执事盥,出举,长在左。鼎入,设于西阶前,东面北上。匕俎从设。左人抽扃、鼏、匕,佐食及右人载。卒,朼者逆退复位。俎入,设于豆东,鱼亚之,腊特。赞设二敦于俎南,黍,其东稷。设一鉶于豆南。佐食出,立于户西。赞者彻鼎。祝酌醴,命佐食启会。佐食许诺,启会,却于敦南,复位。祝奠觯于鉶南。复位。主人再拜稽首。祝飨,命佐食祭。佐食许诺,钩袒,取黍稷,祭于苴三,取肤祭,祭如初。祝取奠觯,祭,亦如之;不尽,益,反奠之。主人再拜稽首。祝祝卒,主人拜如初,哭,出复位。

祝迎尸,一人衰絰,奉篚,哭从尸。尸入门,丈夫踊,妇人踊。淳尸盥,宗人授巾。尸及阶,祝延尸。尸升,宗人诏踊如初。尸入户,踊如初,哭止。妇人入于房。主人及祝拜妥尸。尸拜,遂坐。

从者错篚于尸左席上,立于其北。尸取奠,左执之,取菹,擩于醢,祭于豆间。祝命佐食堕祭。佐食取黍稷肺祭,授尸,尸祭之。祭奠,祝祝,主人拜如初。尸尝醴,奠之。佐食举肺脊授尸。尸受,振祭,哜之,左手执之。祝命佐食迩敦。佐食举黍,错于席上。尸祭鉶,尝鉶,泰羹湆自门入,设于鉶南;j3四豆,设于左。尸饭,播余于篚。三饭,佐食举干;尸受,振祭,哜之,实于篚。又三饭。举胳,祭如初。佐食举鱼腊,实于篚。又三饭,举肩,祭如初。举鱼腊俎,俎释三个。尸卒食。佐食受肺脊,实于篚。反于篚,反黍如初设。

主人洗废爵,酌酒酳尸。尸拜受爵,主人北面答拜。尸祭酒,尝之。宾长以肝从,实于俎,缩,右盐。尸左执爵,右取肝,擩盐,振祭,哜之,加于俎。宾降,反俎于西塾,复位。尸卒爵,祝受,不相爵。主人拜,尸答拜。祝酌授尸,尸以醋主人,主人拜受爵,尸答拜。主人坐祭,卒爵,拜,尸签拜。筵祝,南面。主人献祝,祝拜,坐受爵,主人答拜。荐菹醢,设俎。祝左执爵,祭荐,奠爵,兴,取肺,坐祭,哜之,兴;加于俎,祭酒,尝之。肝从。祝取肝擩盐,振祭,哜之,加于俎,卒爵,拜。主人答拜。祝坐授主人。主人酌献佐食,佐食北面拜,坐受爵,主人答拜。佐食祭酒,卒爵,拜。主人答拜,受爵,出,实于篚,升堂复位。

主妇洗足爵于房中,酌,亚献尸,如主人仪。自反两笾枣、栗,设于会南,枣在西。尸祭笾,祭酒,如初。宾以燔从,如初。尸祭燔,卒爵,如初。酌献祝,笾、燔从,献佐食,皆如初。以虚爵入于房。

宾长洗繶爵,三献,燔从,如初仪。

妇人复位。祝出户,西面告利成。主人哭,皆哭。祝入,尸谡。从者奉篚哭,如初。祝前尸。出户,踊如初;降堂,踊如初;出门亦如之。

祝反,入彻,设于西北隅,如其设也。几在南,〈厂非〉用席。祝荐席彻入于房。祝自执其俎出。赞阖牖户。

主人降,宾出。主人出门,哭止,皆复位。宗人告事毕。宾出,主人送,拜稽颡。

记。虞,沐浴,不栉。陈牲于庙门外,北首,西上,寝右。日中而行事。杀于庙门西,主人不视。豚解。羹饪,升左肩、臂、臑、肫、胳、脊、胁,离肺。肤祭三,取诸左膉上,肺祭一,实于上鼎;升鱼鱄鲋九,实于中鼎;升腊,左胖,髀不升,实于下鼎。皆设扃鼏,陈之。载犹进柢,鱼进鬐。祝俎,髀、脰、脊、胁,离肺,陈于阶间,敦东。淳尸盥。执槃,西面。执匜,东面。执巾在其北,东面。宗人授巾,南面。主人在室,则宗人升,户外北面。佐食无事,则出户,负依南面。鉶□攥用苦,若薇,有滑。夏用葵,冬用〈艸亘〉,有柶。豆实,葵菹,菹以西,蠃醢。笾,枣烝,栗择。尸入,祝从尸。尸坐不说屦。尸谡。祝前,乡尸;还,出户,又乡尸;还,过主人,又乡尸;还,降阶,又乡尸;降阶,还,及门,如出户。尸出,祝反,入门左,北面复位,然后宗人诏降。尸服卒者之上服。男,男尸,女,女尸;必使异姓,不使贱者。无尸,则礼及荐馔皆如初。既飨,祭于苴,祝祝卒,不绥祭,无泰羹湆、胾、从献。主人哭,出复位。祝阖牖户,降,复位于门西;男女拾踊三;如食间。祝升,止哭;声三,启户。主人入,祝从,启牖、乡,如初。主人哭,出复位。卒彻,祝、佐食降,复位。宗人诏降如初。始虞用柔日,曰:「哀子某,哀显相,夙兴夜处不宁。敢用絜牲、刚鬣、香合、嘉荐、普淖、明齐溲酒,哀荐祫事,适尔皇祖某甫。飨!」再虞,皆如初,曰「哀荐虞事」。三虞、卒哭、他,用刚日,亦如初,曰「哀荐成事。」献毕,未彻,乃饯。尊两甒于庙门外之右,少南。水尊在酒西,勺北枋。洗在尊东南,水在洗东,篚在西。馔笾豆,脯四脡。有乾肉折俎,二尹缩,祭半尹,在西塾。尸出,执几从,席从。尸出门右,南面。席设于尊西北,东面。几在南。宾出,复位。主人出,即位于门东,少南;妇人出,即位于主人之北;皆西南,哭不止。尸即席坐。唯主人不哭,洗废爵,酌献尸,尸拜受。主人拜送,哭,复位。荐脯醢,设俎于荐东,朐在南。尸左执爵,取脯擩醢,祭之。佐食授哜。尸受,振祭,哜,反之。祭酒,卒爵,奠于南方。主人及兄弟踊,妇人亦如之。主妇洗足爵,亚献如主人仪,无从,踊如初。宾长洗繶爵,三献,如亚献,踊如初。佐食取俎,实于篚。尸谡,从者奉篚,哭从之。祝前,哭者皆从,及大门内,踊如初。尸出门,哭者止。宾出,主人送,拜稽颡。主妇亦拜宾。丈夫说絰带于庙门外。入彻,主人不与。妇人说首絰,不说带。无尸,则不饯。犹出,几席设如初,拾踊三。哭止,告事毕,宾出。死三日而殡,三月而葬,遂卒哭。将旦而祔,则荐。卒辞曰:「哀子某,来日某,隮祔尔于尔皇祖某甫。尚飨!」女子,曰「皇祖妣某氏。」妇,曰「孙妇于皇祖姑某氏」。其他辞,一也。飨辞曰:「哀子某,圭为而哀荐之飨!」明日,以其班祔。沐浴,栉,搔翦。用专肤为折俎,取诸脰膉。其他如馈食。用嗣尸。曰:「孝子某,孝显相,夙兴夜处,小心畏忌。不惰其身,不宁。用尹祭、嘉荐、曾淖、普荐、溲酒,适尔皇祖某甫,以隮祔尔孙某甫。尚飨。」期而小祥,曰:「荐此常事。」又期而大祥,曰:「荐此祥事。」中月而禫。是月也。吉祭,犹未配。

\hypertarget{header-n68}{%
\subsection{特牲馈食礼}\label{header-n68}}

特牲馈食之礼。不诹日。及筮日,主人冠端玄,即位于门外,西面。子姓兄弟如主人之服,立于主人之南,西面北上。有司群执事,如兄弟服,东面北上。席于门中,闑西阈外。筮人取筮于西塾,执之,东面受命主人。宰自主人之左赞命,命曰:「孝孙某,筮来日某,诹此某事,适其皇祖某子。尚飨!」筮者许诺,还,即席,西面坐。卦者在左。卒筮,写卦。筮者执以示主人。主人受视,反之,筮者还,东面。长占,卒,告于主人:「占曰迹攥则筮远日,如初仪。宗人告事毕。

前期三日之朝,筮尸,如求日之仪。命筮曰:「孝孙某,诹此某事,适其皇祖某子,筮某之某为尸。尚飨!」

乃宿尸。主人立于尸外门外。子姓兄弟立主人之后,北面东上。尸如主人服,出门左,西面。主人辟,皆东面,北上。主人再拜。尸答拜。宗人摈辞如初,卒曰:「筮子为某尸,占曰吉,敢宿!」祝许诺,致命。尸许诺,主人再拜稽首。尸入主人退。

宿宾。宾如主人服,出门左,西面再拜。主人东面,答再拜。宗人摈,曰:「某荐岁事,吾子将莅之,敢宿!」宾曰:「某敢不敬从!」主人再拜,宾答拜,主人退,宾拜送。

厥明夕,陈鼎于门外,北面北上,有鼏。棜在其南,南顺实兽于其上,东首。牲在其西,北首,东足。设洗于降阶东南,壶、禁在东序,豆、笾、鉶在东房,南上。几、席、两敦在西堂。主人及子姓兄弟即位于门东,如初。宾及众宾即位于门西,东面北上。宗人、祝立于宾西北,东面南上。主人再拜,宾答再拜。三拜众宾,众宾答再拜。主人揖入,兄弟从,宾及众宾从,即位于堂下,如外位。宗人升自西阶,视壶濯及豆笾,反降,东北面告濯、具。宾出,主人出,皆复外位。宗人视牲,告充。雍正作豕。宗人举兽尾,告备;举鼎鼏。请期,曰「羹饪」。告事毕,宾出,主人拜送。

夙兴,主人服如初,立于门外东方,南面,视侧杀。主妇视饎爨于西堂下。亨于门外东方,西面北上。羹饪,实鼎,陈于门外,如初。尊于户东,玄酒在西。实豆、笾、鉶,陈于房中,如初。执事之俎,陈于阶间,二列,北上。盛两敦,陈于西堂,藉用萑,几席陈于西堂,如初。尸盥匜水,实于槃中,箪巾,在门内之右。祝筵几于室中,东面。主妇纚笄,宵衣,立于房中,南面。主人及宾、兄弟、群执事,即位于门外,如初。宗人告有司具。主人拜宾如初,揖入,即位,如初,佐食北面立于中庭。

主人及祝升,祝先入,主人从,西面于户内。主妇盥于房中,荐两豆,葵菹、蜗醢,醢在北。宗人遣佐食及执事盥,出。主人降,及宾盥,出。主人在右,及佐食举牲鼎。宾长在右,及执事举鱼腊鼎。除鼏。宗人执毕先入,当作阶,南面。鼎西面错,右人抽扃,委于鼎北。赞者错俎,加匕,乃朼。佐食升肵俎,鼏之,设于阼阶西。卒载,加匕于鼎。主人升,入复位。俎入,设于豆东。鱼次,腊特于俎北。主妇设两敦黍稷于俎南,西上,及两鉶芼设于豆南,南陈。祝洗,酌奠,奠于鉶南,遂命佐食启会,佐食启会,却于敦南,出,立于户西,南面。主人再拜稽首。祝在左,卒祝,主人再拜稽首。

祝迎尸于门外。主人降,立于阼阶东。尸入门左,北面盥。宗人授巾。尸至于阶,祝延尸。尸升,入,祝先,主人从。尸即席坐,主人拜妥尸。尸答拜,执奠;祝飨,主人拜如初。祝命挼祭。尸左执觯,右取菹手耎于醢,祭于豆间。佐食取黍、稷、肺祭,授尸。尸祭之,祭酒,啐酒,告旨。主人拜,尸奠觯,答拜。祭鉶,尝之,告旨。主人拜,尸答拜,祝命尔敦。佐食尔黍稷于席上,设大羹湆于醢北,举肺脊以授尸。尸受,振祭,哜之,左执之,乃食,食举。主人羞肵俎于腊北。尸三饭,告饱。祝侑,主人拜。佐食举干,尸受,振祭,哜之。佐食受,加于肵俎。举兽干、鱼一,亦如之。尸实举于菹豆。佐食羞庶羞四豆,设于左,南上有醢。尸又三饭,告饱。祝侑之,如初,举骼及兽、鱼,如初,尸又三饭,告饱。祝侑之如初,举肩及兽、鱼如初。佐食盛肵俎,俎释三个,举肺脊加于肵俎反黍稷于其所。

主人洗角,升酌,酳尸。尸拜受,主人拜送。尸祭酒,啐酒,宾长以肝从。尸左执角右取肝手耎于盐,振祭,哜之,加于菹豆,卒角。祝受尸角,曰:「送爵!皇尸卒爵。」主人拜,尸答拜。祝酌授尸,尸以醋主人。主人拜受角,尸拜送。主人退,佐食授挼祭。主人坐,左执角,受祭祭之,祭酒,啐酒,进听嘏。佐食抟黍授祝,祝授尸。尸受以菹豆,执以亲嘏主人。主人左执角,再拜稽首受,复位,诗怀之,实于左袂,挂于季指,卒角,拜。尸答拜。主人出,写啬于房,祝以笾受。筵祝,南面。主人酌献祝,祝拜受角,主人拜送。设菹醢、俎。祝左执角,祭豆,兴取肺,坐祭,哜之,兴加于俎,坐祭酒,啐酒,以肝从。祝左执角,右取肝手耎于盐,振祭,哜之,加于俎,卒角,拜。主人答拜,受角,酌献佐食。佐食北面拜受角,主人拜送。佐食坐祭,卒角,拜。主人答拜,受角,降,反于篚,升,入复位。

主妇洗爵于房,酌,亚献尸。尸拜受,主妇北面拜送。宗妇执两笾,户外坐。主妇受,设于敦南。祝赞笾祭。尸受,祭之,祭酒,啐酒。兄弟长以燔从。尸受,振祭,哜之,反之。羞燔者受,加于肵,出。尸卒爵,祝受爵,命送如初。酢,如主人仪。主妇适房,南面。佐食挼祭。主妇左执爵,右抚祭,祭酒,啐酒,入,卒爵,如主人仪。献祝,笾燔从,如初仪。及佐食,如初。卒,以爵入于房。

宾三献,如初。燔从如初。爵止。席于户内。主妇洗爵,酌,致爵于主人。主人拜受爵,主妇拜送爵。宗妇赞豆如初,主妇受,设两豆两笾。俎入设。主人左执爵,祭荐,宗人赞祭。奠爵,兴取肺,坐绝祭,哜之,兴加于俎,坐兑手,祭酒,啐酒,肝从。左执爵,取肝手耎于盐,坐振祭,哜之。宗人受,加于俎。燔亦如之。兴,席末坐卒爵,拜。主妇答拜,受爵,酌醋,左执爵,拜,主人答拜。坐祭,立饮,卒爵,拜,主人答拜。主妇出,反于房。主人降,洗,酌,致爵于主妇,席于房中,南面。主妇拜受爵,主人西面答拜。宗妇荐豆、俎,从献皆如主人。主人更爵酌醋,卒爵,降,实爵于篚,入复位。三献作止爵。尸卒爵,酢。酌献祝及佐食。洗爵,酌致于主人、主妇、燔从皆如初。更爵,酢于主人,卒,复位。

主人降阼阶,西面拜宾,如初。洗,宾辞洗。卒洗,揖让升,酌,西阶上献宾。宾北面拜受爵。主人在右,答拜。荐脯醢。设折俎。宾左执爵,祭豆,奠爵,兴,取肺,坐绝祭,哜之,兴,加于俎,坐兑手,祭酒,卒爵,拜。主人答拜,受爵,酌酢,奠爵,拜。宾答拜。主人坐祭,卒爵,拜。宾答拜,揖,执祭以降,西面奠于其位;位如初。荐、俎从设。众宾升,拜受爵,坐祭,立饮。荐、俎设于其位,辩。主人备答拜焉,降,实爵于篚。尊两壶阼阶东,加勺,南枋,西方亦如之。主人洗觯,酌于西方之尊,西阶前北面酬宾,宾在左。主人奠觯拜,宾答拜。主人坐祭,卒觯,拜。宾答拜。主人洗觯,宾辞,主人对。卒洗,酌,西面。宾北面拜。主人奠觯于荐北。宾坐取觯,还,东面,拜。主人答拜。宾奠觯于荐南。揖复位。主人洗爵,献长兄弟于阼阶上。如宾仪。洗,献众兄弟,如众宾仪。洗,献内兄弟于房中,如献众兄弟之仪。主人西面答拜,更爵酢,卒爵,降,实爵于篚,入复位。

长兄弟洗觚为加爵,如初仪,不及佐食,洗致如初,无从。

众宾长为加爵,如初,爵止。

嗣举奠,盥入,北面再拜稽首。尸执奠,进受,复位,祭酒,啐洒。尸举肝。举奠左执觯,再拜稽首,进受肝,复位,坐食肝,卒觯,拜。尸备答拜焉。举奠洗酌入,尸拜受,举奠答拜。尸祭酒,啐酒,奠之。举奠出,复位。

兄弟弟子洗酌于东方之尊,阼阶前北面,举觯于长兄弟,如主人酬宾仪。宗人告祭脀,乃羞。宾坐取觯,阼阶前北面酬长兄弟;长兄弟在右。宾奠觯拜,长兄弟答拜。宾立卒觯,酌于其尊,东面立。长兄弟拜受觯。宾北面答拜,揖,复位。长兄弟西阶前北面,众宾长自左受旅,如初,长兄弟卒觯,酌于其尊,西面立。受旅者拜受。长兄弟北面答拜,揖,复位。众宾及众兄弟交错以辩。皆如初仪。为加爵者作止爵,如长兄弟之仪。长兄弟酬宾,如宾酬史弟之仪,以辩。卒者实觯于篚。宾弟子及兄弟弟子洗,各酌于其尊,中庭北面西上,举觯于其长,奠觯拜,长皆答拜。举觯者祭,卒觯,拜,长皆答拜。举觯者洗,各酌于其尊,复初位。长皆拜。举觯者皆奠觯于荐右。长皆执以兴,举觯者皆复位答拜。长皆奠觯于其所,皆揖其弟子,弟子皆复其位。爵皆无算。

利洗散,献于尸,酢,及祝,如初仪。降,实散于篚。

主人出,立于户外,西南面。祝东面告利成。尸谡,祝前,主人降。祝反,及主人入,复位。命佐食彻尸俎,俎出于庙门。彻庶羞,设于西序下。

筵对席,佐食分簋鉶。宗人遣举奠及长兄弟盥,立于西阶下,东面北上。祝命尝食。餕者,举奠许诺,升,入,东面。长兄弟对之,皆坐。佐食授举,各一肤。主人西面再拜,祝曰:「餕,有以也。」两餕奠举于俎,许诺,皆答拜。若是者三。皆取举,祭食,祭举乃食,祭鉶,食举。卒食。主人降洗爵,宰赞一爵。主人升酌,酳上餕,上餕拜受爵,主人答拜;酳下餕,亦如之。主人拜,祝曰:「酳,有与也。」如初仪。两餕执爵拜,祭酒,卒爵,拜。主人答拜。两餕皆降,实爵于篚,上餕洗爵,升酌,酢主人,主人拜受爵。上餕即位,坐答拜。主人答拜。主人坐祭,卒爵,拜。上餕答拜,受爵,降,实于篚。主人出,立于户外,西面。

祝命彻阼俎、豆、笾,设于东序下。祝执其俎以出,东面于户西。宗妇彻祝豆、笾入于房,彻主妇荐、俎。佐食彻尸荐、俎、敦,设于西北隅,几在南,厞用筵,纳一尊。佐食阖牖户,降。祝告利成,降,出。主人降,即位。宗人告事毕。宾出,主人送于门外,再拜。佐食彻阼俎。堂下俎毕出。

记。特牲馈食,其服皆朝服,玄冠、缁带、缁□。唯尸、祝、佐食玄端,玄裳、黄裳、杂裳可也,皆爵□。设洗,南北以堂深,东西当东荣。水在洗东。篚在洗西,南顺,实二爵、二觚、四觯、一角、一散。壶、棜禁,馔于东序,南顺。覆两壶焉,盖在南;明日卒奠,幂用谷;即位而彻之,加勺。笾,巾以谷也,纁里,枣〓,栗择。鉶芼,用苦,若薇,皆有滑,夏葵,冬荁。棘心匕,刻。牲爨在庙门外东南,鱼腊爨在其南,皆西面,饎爨在西壁。肵俎心舌皆去本末,午割之,实于牲鼎,载心立、舌缩俎。宾与长兄弟之荐,自东房,其馀在东堂。沃尸盥者一人,奉槃者东面,执匜者西面淳沃,执巾者在匜北。宗人东面取巾,振之三,南面授尸;卒,执巾者受。尸入,主人及宾皆辟位,出亦如之。嗣举奠,佐食设豆盐。佐食当事,则户外南面,无事,则中庭北面。凡祝呼,佐食许诺。宗人,献与旅齿于从宾。佐食,于旅齿于兄弟。尊两壶于房中西墉下,南上。内宾立于其北,东面南上。宗妇北堂东面,北上。主妇及内宾、宗妇亦旅,西面。宗妇赞荐者,执以坐于户外,授主妇。尸卒食,而祭饎爨、雍爨。宾从尸,俎出庙门,乃反位。尸俎,左肩、臂、臑、肫、胳,正脊二骨,横脊,长胁二骨,短胁。肤三,离肺一,刌肺三,鱼十有五。腊如牲骨。祝俎,髀、脡脊二骨,胁二骨。肤一,离肺一。阼俎:臂,正脊二骨,横脊,长胁二骨,短胁。肤一,离肺一。主妇俎,觳折,其馀如阼俎。佐食俎,觳折,脊,胁。肤一,离肺一。宾,骼。长兄弟及宗人,折:其馀如佐食俎。众宾及众兄弟、内宾、宗妇,若有公有司、私臣,皆殽脀,肤一,离肺一。公有司门西,北面东上,献次众宾。私臣门东,北面西上,献次兄弟。升受,降饮。

\hypertarget{header-n72}{%
\subsection{少牢馈食礼}\label{header-n72}}

少牢馈食之礼。日用丁己。筮旬有一日。筮于庙门之外。主人朝服,西面于门东。史朝服,左执筮,右取上韇,兼与筮执之,东面受命于主人。主人曰:「孝孙某,来日丁亥,用荐岁事于皇祖伯某,以某妃配某氏。尚飨!」史曰:「诺!」西面于门西,抽下韇,左执筮,右兼执韇以击筮,遂述命曰:「假尔大筮有常。孝孙某,来日丁亥,用荐岁事于皇祖伯某,以某妃配某氏。尚飨!」乃释韇立筮。卦者在左坐,卦以木。卒筮,乃书卦于木,示主人,乃退占。吉,则史韇筮,史兼执筮与封以告于主人:「占曰从。」乃官戒,宗人命涤,宰命为酒,乃退。若不吉,则及远日,又筮日如初。

宿。前宿一日,宿戒尸。明日,朝服筮尸,如筮日之礼。命曰:「孝孙某,来日丁亥,用荐岁事于皇祖伯某,以某妃配某氏。以某之某为尸。尚飨!」筮、卦占如初。吉,则乃遂宿尸。祝摈,主人再拜稽首。祝告曰:「孝孙某,来日丁亥,用荐岁事于皇祖伯某,以某妃配某氏。敢宿!」尸拜,许诺,主人又再拜稽首。主人退,尸送,揖,不拜。若不吉,则遂改筮尸。

既宿尸,反,为期于庙门之外。主人门东,南面。宗人朝服北面,曰:「请祭期。」主人曰:「比于子。」宗人曰:「旦明行事。」主人曰:「诺!」乃退。

明日,主人朝服,即位于庙门之外,东方南面。宰、宗人西面,北上。牲北首东上。司马圭刀羊,司士击豕。宗人告备,乃退。雍人摡鼎、匕、俎于雍爨,雍爨在门东南,北上。廪人摡甑甗、匕与敦于廪爨,廪爨在雍爨之北。司宫摡豆、笾、勺、爵、觚、觯、几、洗、篚于东堂下,勺、爵、觚、觯实于篚;卒摡,馔豆、笾与篚于房中,放于西方;设洗于阼阶东南,当东荣。

羹定,雍人陈鼎五,三鼎在羊镬之西,二鼎在豕镬之西。司马升羊右胖。髀不升,肩、臂、臑、□、骼,正脊一、横脊短胁一、正胁一、代胁一,皆二骨以并,肠三、胃三、举肺一、祭肺三,实于一鼎。司士升豕右胖。髀不升,肩、臂、臑、□骼,正脊一、横脊一、短胁一、正胁一、代胁一,皆二骨以并,举肺一、祭肺三,实于一鼎。雍人伦肤九,实于一鼎。司士又升鱼、腊,鱼十有五而鼎,腊一纯而鼎,腊用麋。卒脀,皆设扃幂,乃举,陈鼎于庙门之外,东方,北面,北上。司宫尊两甒于房户之间,同棜,皆有幂,甒有玄酒。司宫设罍水于洗东,有枓,设篚于洗西,南肆。改馔豆、笾于房中,南面,如馈之设,实豆、笾之实。小祝设槃、匜与箪、巾于西阶东。

主人朝服,即位于阼阶东,西面。司宫筵于奥,祝设几于筵上,右之。主人出迎鼎,除鼏。士盥,举鼎,主人先入。司宫取二勺于篚,洗之,兼执以升,乃启二尊之盖幂,奠于棜上。加二勺于二尊,覆之,南柄。鼎序入。雍正执一匕以从,雍府执四匕以从,司士合执二俎以从。司士赞者二人,皆合执二俎以相,从入。陈鼎于东方,当序,南于洗西,皆西面,北上,肤为下。匕皆加于鼎。东枋。俎皆设于鼎西,西肆。肵俎在羊俎之北,亦西肆。宗人遣宾就主人,皆盥于洗,长朼。佐食上利升牢心舌,载于肵俎。心皆安下切上,午割勿没,其载于肵俎,末在上。舌皆切本末,亦午割勿没;其载于肵,横之。皆如初为之于爨也。佐食迁肵俎于阼阶西,西缩,乃反。佐食二人。上利升羊,载右胖,髀不升,肩、臂、臑、□骼;正脊一、横脊一、短胁一、正胁一、代胁一,皆二骨以并;肠三、胃三,长皆乃俎拒;举肺一,长终肺,祭肺三,皆切。肩、臂、臑、□、骼在两端,脊、胁、肺,肩在上。下利升豕,其载如羊,无肠胃。体其载于俎,皆进下。司士三人,升鱼、腊、肤。鱼用鲋十有五而俎,缩载,右首,进腴。腊一纯而俎,亦进下,肩在上。肤九而俎,亦横载,革顺。

卒脀,祝盥于洗,升自西阶。主人盥,升自阼阶。祝先入,南面。主人从,户内西面。主妇被锡,衣侈袂,荐自东房,韭、菹、醓、醢,坐奠于筵前。主妇赞者一人,亦被锡。衣侈袂。执葵菹、蠃醢,以授主妇。主妇不兴,遂受,陪设于东,韭菹在南,葵菹在北。主妇兴,入于房。佐食上利执羊俎,下利执豕俎,司士三人执鱼,腊、肤俎,序升自西阶,相,从入。设俎,羊在豆东,豕亚其北,鱼在羊东,腊在豕东,特肤当俎北端。主妇自东房,执一金敦黍,有盖,坐设于羊俎之南。妇赞者执敦稷以授主妇。主妇兴受,坐设于鱼俎南;又兴受赞者敦黍,坐设于稷南;又兴受赞者敦稷,坐设于黍南。敦皆南首。主妇兴,入于房。祝酌,奠,遂命佐食启会。佐食启会盖,二以重,设于敦南。主人西面,祝在左,主人再拜稽首。祝祝曰:「孝孙某,敢用柔毛、刚鬣、嘉荐、普淖,用荐岁事于皇祖伯某,以某妃配某氏。尚飨!」主人又再拜稽首。

祝出,迎尸于庙门之外。主人降立于阼阶东,西面。祝先,入门右。尸入门左。宗人奉槃,东面于庭南。一宗人奉匜水,西面于槃东。一宗人奉箪、巾,南面于槃北。乃沃尸,盥于槃上。卒盥,坐奠箪,取巾,兴,振之三,以授尸,坐取箪,兴,以受尸巾。祝延尸。尸升自西阶,入,祝从。主人升自阼阶,祝先入,主人从。尸升筵,祝、主人西面立于户内,祝在左。祝、主人皆拜妥尸,尸不言尸答拜,遂坐,祝反南面。

尸取韭菹,辩手耎于三豆,祭于豆间。上佐食取黍稷于四敦。下佐食取牢一切肺于俎,以授上佐食。上佐食兼与黍以授尸。尸受,同祭于豆祭。上佐食举尸牢肺、正脊以授尸。上佐食尔上敦黍于筵上,右之。主人羞肵俎,升自阼阶,置于肤北。上佐食羞两鉶,取一羊鉶于房中,坐设于韭菹之南。下佐食又取一豕鉶于房中以从。上佐食受,坐设于羊鉶之南。皆芼,皆有柶。尸扱以柶,祭羊鉶,遂以祭豕鉶,尝羊鉶,食举,三饭。上佐食举尸牢干,尸受,振祭,哜之。佐食受,加于肵。上佐食羞胾两瓦豆,有醢,亦用瓦豆,设于荐豆之北。尸又食,食胾。上佐食举尸一鱼,尸受,振祭,哜之。佐食受,加于肵,横之。又食。上佐食举尸腊肩,尸受,振祭,哜之,上佐食受,加于肵。又食。上佐食举尸牢胳,如初。又食。尸告饱。祝西面于主人之南,独侑不拜。侑曰:「皇尸未实,侑!」尸又食。上佐食举尸牢肩,尸受,振祭,哜之,佐食受加于肵。尸不饭,告饱。祝西面于主人之南。主人不言,拜侑。尸又三饭。上佐食受尸牢肺、正脊,加于肵。

主人降,洗爵,升,北面酌酒,乃酳尸。尸拜受,主人拜送。尸祭酒,啐酒。宾长羞牢肝,用俎,缩执俎,肝亦缩,进末,盐在右。尸左执爵,右兼取肝,手耎于俎盐,振祭,哜之,加于俎豆,卒爵。主人拜。祝受尸爵。尸答拜。

祝酌授尸,尸醋主人。主人拜受爵,尸答拜。主人西面奠爵,又拜。上佐食取四敦黍稷,下佐食取牢一切肺,以授上佐食。上佐食以绥祭。主人左执爵,右受佐食,坐祭之,又祭酒,不兴,遂啐酒。祝与二佐食皆出,盥于洗,入。二佐食各取黍于一敦。上佐食兼受,抟之,以授尸,尸执以命祝。卒命祝,祝受以东,北面于户西,以嘏于主人,曰:「皇尸命工祝,承致多福无疆于女孝孙。来女孝孙,使女受禄于天,宜稼于田,眉寿万年,勿替引之。」主人坐奠爵,兴;再拜稽首,兴;受黍,坐振祭,哜之;诗怀之,实于左袂,挂于季指,执爵以兴;坐卒爵,执爵以兴;坐奠爵,拜。尸答拜。执爵以兴,出。宰夫以笾受啬黍。主人尝之,纳诸内。

主人献祝,设席南面。祝拜于席上,坐受。主人西面答拜。荐两豆菹、醢。佐食设俎,牢髀,横脊一、短胁一、肠一、胃一、肤三,鱼一横之,腊两髀属于尻。祝取菹手耎于醢,祭于豆间。祝祭俎,祭酒,啐酒。肝牢从。祝取肝手耎于盐,振祭,哜之,不兴,加于俎,卒爵,兴。

主人酌,献上佐食。上佐食户内牖东北面拜,坐受爵。主人西面答拜。佐食祭酒,卒爵,拜,坐授爵,兴。俎设于两阶之间,其俎,折,一肤。主人又献下佐食,亦如之。其脀亦设于阶间,西上,亦折,一肤。

有司赞者取爵于篚以升,授主妇赞者于房庐。妇赞者受,以授主妇。主妇洗于房中,出酌,入户,西面拜,献尸。尸拜受。主妇主人之北西面拜送爵。尸祭酒,卒爵。主妇拜。祝受尸爵。尸答拜。

易爵,洗,酌,授尸。主妇拜受爵,尸答拜。上佐食绥祭。主妇西面,于主人之北受祭,祭之,其绥祭如主人之礼,不嘏,卒爵,拜。尸答拜。

主妇以爵出。赞者受,易爵于篚,以授主妇于房中。主妇洗,酌,献祝。祝拜,坐受爵。主妇答拜于主人之北。卒爵,不兴,坐授主妇。

主妇受,酌,献上佐食于户内。佐食北面拜,坐受爵,主妇西面答拜。祭酒,卒爵,坐授主妇。主妇献下佐食,亦如之。主妇受爵以入于房。

宾长洗爵献于尸,尸拜受爵。宾户西北拜送爵。尸祭酒,卒爵。宾拜。祝受尸爵,尸答拜。

祝酌授尸,宾拜受爵,尸拜送爵。宾坐奠爵,遂拜,执爵以兴,坐祭,遂饮,卒爵,执爵以兴,坐奠爵,拜。尸答拜。

宾酌献祝。祝拜,坐受爵。宾北面答拜。祝祭酒,啐酒,奠爵于其筵前。

主人出立于阼阶上,西面。祝出立于西阶上,东面。祝告曰:「利成。」祝入,尸谡。主人降立于阼阶东,西面。祝先,尸从,遂出于庙门。

祝反,复位于室中。主人亦入于室,复位。祝命佐食彻肵俎,降设于堂下阼阶南。司宫设对食,乃四人餕。上佐食盥升,下佐食对之,宾长二人备。司士进一敦于上佐食,又进一敦黍于下佐食,皆右之于席上。资黍于羊俎两端,两下是餕。司士乃辩举,餕者皆祭黍、祭举。主人西面,三拜餕者。餕者奠举于俎,皆答拜,皆反,取举。司士进一鉶于上餕,又进一鉶于次餕,又进二豆湆于两下。乃皆食,食举,卒食。主人洗一爵,升酌,以授上餕。赞者洗三爵,酌。主人受于户内,以授次餕,若是以辩。皆不拜,受爵。主人西面,三拜餕者。餕者奠爵,皆答拜,皆祭酒,卒爵,奠爵,皆拜。主人答壹拜。餕者三人兴,出,上餕止。主人受上餕爵,酌以酢于户内,西面坐奠爵,拜,上餕答拜。坐祭酒,啐酒。上餕亲嘏,曰:「主人受祭之福,胡寿保建家室。」主人兴,坐奠爵,拜,执爵以兴,坐卒爵,拜,上餕答拜。上餕兴,出。主人送,乃退。

\hypertarget{header-n76}{%
\subsection{有司}\label{header-n76}}

有司彻,扫堂。司宫摄酒。乃燅尸俎,卒燅,乃升羊、豕、鱼三鼎,无腊与肤,乃设扃鼏,陈鼎于门外,如初。

乃议侑于宾,以异姓。宗人戒侑。侑出,俟于庙门之外。

司宫筵于户西,南面;又筵于西序,东面。尸与侑,北面于庙门之外,西上。主人出迎尸,宗人摈。主人拜,尸答拜。主人又拜侑,侑答拜。主人揖,先入门,右。尸入门,左;侑从,亦左。揖,乃让。主人先升自阼阶,尸、侑升自西阶,西楹西,北面东上。主人东楹东,北面拜至,尸答拜。主人又拜侑,侑答拜。

乃举,司马举羊鼎,司士举豕鼎、举鱼鼎,以入。陈鼎如初。雍正执一匕以从,雍府执二匕以从,司士合执二俎以从,司士赞者亦合执二俎以从。匕皆加于鼎,东枋。二俎设于羊鼎西,西缩。二俎皆于二鼎西,亦西缩。雍人合执二俎,陈于羊俎西,并皆西缩。覆二疏匕于其上,皆缩俎,西枋。

主人降,受宰几。尸、侑降,主人辞,尸对。宰授几,主人受,二手横执几,揖尸。主人升,尸、侑升,复位。主人西面,左手执几,缩之,以右袂推拂几三,二手横执几,进授尸于筵前。尸进,二手受于手间,主人退。尸还几,缩之,右手执外廉,北面奠于筵上,左之,南缩,不坐。主人东楹东,北面拜。尸复位,尸与侑皆北面答拜。主人降洗,尸、侑降,尸辞洗。主人对,卒洗,揖。主人升,尸、侑升,尸西楹西北面拜洗。主人东楹东北面奠爵答拜,降盥。尸、侑降,主人辞,尸对。卒盥。主人揖,升,尸、侑升。主人坐取爵,酌献尸。尸北面拜受爵,主人东楹东北面拜送爵。主妇自东房荐韭、菹、醢,坐奠于筵前,菹在西方。妇赞者执昌、苴、醢以授主妇。主妇不兴,受;陪设于南,昌在东方。兴,取笾于房,麷、蕡坐设于豆西,当外列,麷在东方。妇赞者执白、黑以授主妇。主妇不兴,受,设于初笾之南,白在西方;兴,退。乃升。司马朼羊,亦司马载。载右体,肩、臂、肫、胳、臑,正脊一、脡脊一、横脊一,短胁一、正胁一、代胁一,肠一、胃一、祭肺一,载于一俎。羊肉湆:臑折、正脊一、正胁一、肠一、胃一、哜肺一,载于南俎。司士朼豕,亦司士载,亦右体:肩、臂、肫、胳、臑,正脊一、脡脊一、横脊一,短胁一、正胁一、代胁一,肤五、哜肺一,载于一俎。侑俎:羊左肩、左肫、正脊一、胁一、肠一、胃一、切肺一,载于一俎。侑俎:豕左肩折、正脊一、胁一、肤三、切肺一,载于一俎。阼俎:羊肺一,祭肺一,载于一俎。羊肉湆:臂一、脊一、胁一、肠一、胃一、哜肺一,载于一俎。豕脀:臂一、脊一、胁一、肤三、哜肺一,载于一俎。主妇俎:羊左臑、脊一、胁一、肠一、胃一、肤一、哜羊肺一,载于一俎。司士朼鱼,亦司士载,尸俎五鱼,横载之,侑、主人皆一鱼,亦横载之,皆加膴祭于其上。卒升。宾长设羊俎于豆南,宾降。尸升筵自西方,坐,左执爵,右取韭、菹手耎于三豆,祭于豆间。尸取麷、蕡,宰夫赞者取白、黑以授尸。尸受,兼祭于豆祭。雍人授次宾疏匕与俎。受于鼎西,左手执俎左廉,缩之,却右手执匕枋,缩于俎上,以东面受于羊鼎之西。司马在羊鼎之东,二手执桃匕枋以挹湆,注于疏匕,若是者三。尸兴,左执爵,右取肺,坐祭之,祭酒,兴,左执爵。次宾缩执匕俎以升,若是以授尸。尸却手受匕枋,坐祭,哜之,兴,覆手以授宾。宾亦覆手以受,缩匕于俎上以降。尸席末坐啐酒,兴,坐奠爵,拜,告旨,执爵以兴。主人北面于东楹东,答拜。司马羞羊肉湆,缩执俎。尸坐奠爵,兴取肺,坐绝祭,哜之,兴,反加于俎。司马缩奠俎于羊湆俎南,乃载于羊俎,卒载俎,缩执俎以降。尸坐执爵以兴。次宾羞羊燔,缩执俎,缩一燔于俎上,盐在右。尸左执爵,受燔,手耎于盐,坐振祭,哜之,兴,加于羊俎。宾缩执俎以降。尸降筵,北面于西楹西,坐卒爵,执爵以兴,坐奠爵,拜,执爵以兴。主人北面于东楹东答拜。主人受爵。尸升筵,立于筵末。

主人酌,献侑。侑西楹西北面拜受爵。主人在其右,北面答拜。主妇荐韭菹醢,坐奠于筵前,醢在南方。妇赞者执二笾麷、蕡,以授主妇。主妇不兴,受之,奠麷于醢南,蕡在麷东。主妇入于房。侑升筵自北方。司马横执羊俎以升,设于豆东。侑坐,左执爵,右取菹手耎于醢,祭于豆间,又取麷、蕡同祭于豆祭,兴,左执爵,右取肺,坐祭之,祭酒,兴,左执爵。次宾羞羊燔,如尸礼。侑降筵自北方,北面于西楹西,坐卒爵,执爵以兴,坐奠爵,拜。主人答拜。

尸受侑爵,降洗。侑降立于西阶西,东面。主人降自阼阶,辞洗。尸坐奠爵于篚,兴对,卒洗。主人升,尸升自西阶。主人拜洗。尸北面于西楹西,坐奠爵,答拜,降盥。主人降,尸辞,主人对。卒盥。主人升。尸升,坐取爵,酌。司宫设席于东序,西面。主人东楹东北面拜受爵,尸西楹西北面答拜。主妇荐韭、菹、醢,坐奠于筵前,菹在北方。妇赞者执二笾麷、蕡,主妇不兴,受,设麷于菹西北,蕡在麷西。主人升筵自北方,主妇入于房。长宾设羊俎于豆西。主人坐,左执爵,祭豆笾,如侑之祭,兴,左执爵,右取肺,坐祭之,祭酒,兴。次宾羞匕湆。如尸礼。席末坐啐酒,执爵以兴。司马羞羊肉湆,缩执俎。主人坐,奠爵于左,兴,受肺,坐绝祭,哜之,兴,反加于湆俎。司马缩奠湆俎于羊俎西,乃载之,卒载,缩执虚俎以降。主人坐取爵以兴。次宾羞燔,主人受,如尸礼。主人降筵自北方,北面于阼阶上,坐卒爵,执爵以兴,坐奠爵,拜,执爵以兴。尸西楹西答拜。主人坐奠爵于东序南。侑升。尸、侑皆北面于西楹西。主人北面于东楹东,再拜崇酒。尸、侑皆答再拜。主人及尸、侑皆升就筵。

司宫取爵于篚,以授妇赞者于房东,以授主妇。主妇洗爵于房中,出实爵,尊南,西面拜献尸。尸拜,于筵上受。主妇西面于主人之席北,拜送爵,入于房,取一羊鉶,坐奠于韭菹西。主妇赞者执豕鉶以从,主妇不兴,受,设于羊鉶之西,兴,入于房,取糗与腶修,执以出,坐设之,糗在蕡西。修在白西,兴,立于主人席北。西面。尸坐,左执爵,祭糗修,同祭于豆祭,以羊鉶之柶挹羊鉶,遂以挹豕鉶,祭于豆祭,祭酒。次宾羞豕匕湆,如羊匕湆之礼。尸坐啐酒,左执爵,尝上鉶,执爵以兴,坐奠爵,拜,主妇答拜。执爵以兴。司士羞豕脀。尸坐奠爵,兴受,如羊肉湆之礼,坐取爵,兴。次宾羞豕燔。尸左执爵,受燔,如羊燔之礼,坐卒爵,拜。主妇答拜。

受爵,酌,献侑。侑拜受爵,主妇主人之北西面答拜。主妇羞糗、修,坐奠糗于麷南,修在蕡南。侑坐,左执爵,取糗、修兼祭于豆祭。司士缩执豕脀以升。侑兴取肺,坐祭之。司士缩奠豕脀于羊俎之东,载于羊俎,卒,乃缩执俎以降。侑兴。次宾羞豕燔,侑受如尸礼,坐卒爵,拜。主妇答拜。

受爵,酌以致于主人。主人筵上拜受爵,主妇北面于阼阶上答拜。主妇设二鉶与糗、修,如尸礼。主人其祭糗、修,祭鉶,祭酒,受豕匕湆,拜啐酒,皆如尸礼。尝鉶不拜。其受豕脀,受豕燔,亦如尸礼。坐卒爵,拜。主妇北面答拜,受爵。

尸降筵,受主妇爵以降。主人降,侑降。主妇入于房。主人立于洗东北,西面。侑东面于西阶西南。尸易爵于篚,盥洗爵,主人揖尸、侑。主人升。尸升自西阶,侑从。主人北面立于东楹东,侑西楹西北面立。尸酌。主妇出于房。西面拜,受爵。尸北面于侑东答拜。主妇入于房。司宫设室于房中,南面。主妇立于席西。妇赞者荐韭、菹、醢,坐奠于筵前,菹在西方。妇人赞者执麷、蕡以授妇赞者,妇赞者不兴,受,设麷于菹西,蕡在麷南。主妇升筵。司马设羊俎于豆南。主妇坐,左执爵,右取菹手耎于醢,祭于豆间;又取麷、蕡兼祭于豆祭。主妇奠爵,兴取肺,坐绝祭,哜之;兴加于俎,坐梲手,祭酒,啐酒。次宾羞羊燔。主妇兴,受燔,如主人之礼。主妇执爵以出于房,西面于主人席北,立卒爵,执爵拜。尸西楹西北面答拜。主妇入立于房。尸、主人及侑皆就筵。

上宾洗爵以升,酌,献尸。尸拜受爵。宾西楹西北面拜送爵。尸奠爵于荐左。宾降。

主人降,洗爵,尸、侑降。主人奠爵于篚,辞。尸对。卒洗,揖。尸升,侑不升。主人实爵酬尸,东楹东,北面坐奠爵,拜。尸西楹西北面答拜。坐祭,遂饮,卒爵拜。尸答拜。降洗。尸降辞。主人奠爵于篚,对,卒洗。主人升。尸升。主人实爵,尸拜受爵。主人反位,答拜。尸北面坐,奠爵于荐左。

尸、侑、主人皆升筵。乃羞,宰夫羞房中之羞于尸、侑、主人、主妇,皆右之,司士羞庶羞于尸、侑、主人、主妇,皆左之。

主人降,南面拜众宾于门东,三拜。众宾门东,北面,皆答壹拜。主人洗爵,长宾辞。主人奠爵于篚,兴对,卒洗,升酌,献宾于西阶上。长宾升,拜受爵。主人在其右,北面答拜。宰夫自东房荐脯、醢,醢在西。司士俎于豆北,羊胳一,肠一,胃一,切肺一,肤一。宾坐,左执爵,右取脯,手耎于醢,祭之,执爵兴,取肺,坐祭之,祭酒,遂饮,卒爵,执爵以兴,坐奠爵,拜,执爵以兴。主人答拜,受爵,宾坐取祭以降,西面坐委西阶西南。宰夫执荐以从,设于祭东;司士执俎以从,设于荐东。

众宾长升,拜受爵,主人答拜。坐祭,立饮,卒爵,不拜既爵。宰夫赞主人酌,若是以辩。辩受爵。其荐脯、醢与脀,设于其位。其位继上宾而南,皆东面。其脀体,仪也。

乃升长宾,主人酌,酢于长宾,西阶上北面,宾在左。主人坐奠爵,拜,执爵以兴,宾答辩。坐祭,遂饮,卒爵,执爵以兴,坐奠爵,拜。宾答拜。宾降。

宰夫洗觯以升。主人受酌,降酬长宾于西阶南,北面。宾在左。主人坐奠爵,拜,宾答拜。坐祭,遂饮,卒爵拜。宾答拜。主人洗,宾辞。主人坐奠爵于篚,对,卒洗,升酌,降复位。宾拜受爵,主人拜送爵。宾西面坐,奠爵于荐左。

主人洗,升酌,献兄弟于阼阶上。兄弟之长升,拜受爵。主人在其右答拜。坐祭,立饮,不拜既爵,皆若是以辩。辩受爵,其位在洗东,西面北上。升受爵,其荐脀设于其位。其先生之脀,折,胁一,肤一。其众,仪也。

主人洗,献内宾于房中。南面拜受爵,主人南面于其右答拜。坐祭,立饮,不拜既爵。若是以辩,亦有荐脀。

主人降洗,升献私人于阼阶。拜于下,升受,主人答其长拜。乃降,坐祭,立饮,不拜既爵。若是以辩。宰夫赞主人酌。主人于其群私人,不答拜。其位继兄弟之南,亦北上,亦有荐脀。主人就筵。

尸作三献之爵。司士羞湆鱼,缩执俎以升。尸取膴祭祭之,祭酒,卒爵。司士缩奠俎于羊俎南,横载于羊俎,卒,乃缩执俎以降。尸奠爵拜。三献北面答拜,受爵,酌献侑。侑拜受,三献北面答拜。司士羞湆鱼一,如尸礼。卒爵拜。三献答拜,受爵,酌致主人。主人拜受爵,三献东楹东北面答拜。司士羞一湆鱼,如尸礼。卒爵拜。三献答拜,受爵。尸降筵,受三献爵,酌以酢之。三献西楹西北面拜受爵,尸在其右以授之。尸升筵,南面答拜,坐祭,遂饮,卒爵拜。尸答拜。执爵以降,实于篚。

二人洗觯,升实爵,西楹西,北面东上,坐奠爵,拜,执爵以兴,尸、侑答拜。坐祭,遂饮,卒爵,执爵以兴,坐奠爵,拜,尸、侑答拜。皆降洗,升酌,反位。尸、侑皆拜受爵,举觯者皆拜送。侑奠觯于右。尸遂执觯以兴,北面于阼阶上酬主人。主人在右。坐奠爵,拜,主人答拜。不祭,立饮,卒爵,不拜既爵,酌,就于阼阶上酬主人。主人拜受爵。尸拜送。尸就筵,主人以酬侑于西楹西,侑在左。坐奠爵,拜。执爵兴,侑答拜。不祭,立饮,卒爵,不拜既爵,酌,复位。侑拜受,主人拜送。主人复筵,乃升长宾。侑酬之,如主人之礼。至于众宾,遂及兄弟,亦如之,皆饮于上。遂及私人,拜受者升受,下饮,卒爵,升酌,以之其位,相酬辩。卒饮者实爵于篚。乃羞庶羞于宾、兄弟、内宾及私人。

兄弟之后生者举觯于其长。洗,升酌,降,北面立于阼阶南,长在左。坐奠爵,拜,执爵以兴,长答拜。坐祭,遂饮,卒爵,执爵以兴,坐奠爵,拜,执爵以兴,长答拜。洗,升酌,降。长拜受于其位,举爵者东面答拜。爵止。

宾长献于尸,如初,无湆,爵不止。

宾一人举爵于尸,如初,亦遂之于下。

宾及兄弟交错其酬,皆遂及私人,爵无算。

尸出,侑从。主人送于庙门之外,拜,尸不顾,拜侑与长宾,亦如之。众宾从。司士归尸、侑之俎。主人退,有司彻。

若不宾尸,则祝、侑亦如之。尸食,乃盛俎、臑、臂、肫、脡脊、横脊、短胁、代胁,皆牢;鱼七;腊辩。无髀。卒盛,乃举牢肩。尸受,振祭,哜之。佐食受,加于肵。

佐食取一俎于堂下以入,奠于羊俎东。乃摭于鱼、腊俎,俎释三个。其馀皆取之,实于一俎以出。祝、主人之鱼、腊取于是。尸不饭,告饱。主人拜侑,不言,尸又三饭。佐食受牢举,如傧。

主人洗、酌,酳尸,宾羞肝,皆如傧礼。卒爵,主人拜,祝受尸爵,尸答拜。祝酌授尸,尸以醋主人,亦如傧。其绥祭,其嘏,亦如傧。其献祝与二佐食,其位,其荐脀,皆如傧。

主妇其洗献于尸,亦如傧。主妇反取笾于房中,执枣、糗,坐设之,枣在稷南,糗在枣南。妇赞者执栗、脯,主妇不兴,受,设之,栗在糗东,脯在枣东。主妇兴。反位。尸左执爵,取枣、糗。祝取栗、脯以授尸。尸兼祭于豆祭,祭酒,啐酒。次宾羞牢燔,用俎,盐在右。尸兼取燔手耎于盐,振祭,哜之。祝受,加于肵。卒爵。主妇拜。祝受尸爵。尸答拜。祝易爵,洗,酌,授尸。尸以醋主妇,主妇主人之北拜受爵,尸答拜。主妇反位,又拜。上佐食绥祭,如傧。卒爵拜,尸答拜。主妇献祝,其酌如傧。拜,坐受爵。主妇主人之北答拜。宰夫荐枣、糗,坐设枣于菹西,糗在枣南。祝左执爵,取枣、糗祭于豆祭,祭酒,啐酒。次宾羞燔,如尸礼。卒爵。主妇受爵,酌献二佐食,亦如傧。主妇受爵,以入于房。

宾长洗爵,献于尸。尸拜受。宾户西北面答拜。爵止。主妇洗于房中,酌,致于主人。主人拜受,主妇户西北面拜送爵。司宫设席。主妇荐韭、菹、醢,坐设于席前,菹在北方。妇赞者执枣、糗以从,主妇不兴,受,设枣于菹北,糗在枣西。佐食设俎,臂、脊、胁、肺皆牢,肤三,鱼一,腊臂。主人左执爵,右取菹手耎于醢,祭于豆间,遂祭笾,奠爵,兴,取牢肺,坐绝祭,哜之,兴,加于俎,坐梲手,祭酒,执爵以兴,坐卒爵,拜。主妇答拜,受爵,酌以醋,户内北面拜,主人答拜。卒爵,拜。主人答拜。主妇以爵入于房。尸作止爵,祭酒,卒爵。宾拜。祝受爵。尸答拜。祝酌授尸。宾拜受爵,尸拜送。坐祭,遂饮,卒爵拜。尸答拜。献祝及二佐食。洗,致爵于主人。主人席上拜受爵,宾北面答拜。坐祭,遂饮,卒爵,拜。宾答拜,受爵,酌,致爵于主妇。主妇北堂。司宫设席,东面。主妇席北东面拜受爵,宾西面答拜。妇赞者荐韭、菹、醢,菹在南方。妇人赞者执枣、糗,授妇赞者;妇赞者不兴,受,设枣于菹南,糗在枣东。佐食设俎于豆东,羊臑,豕折,羊脊、胁,祭肺一,肤一,鱼一,腊臑。主妇升筵,坐,左执爵,右取菹手耎于醢,祭之,祭笾,奠爵,兴取肺,坐绝祭,哜之,兴加于俎,坐梲手,祭酒,执爵兴,筵北东面立卒爵,拜。宾答拜。宾受爵,易爵于篚,洗、酌,醋于主人,户西北面拜,主人答拜。卒爵,拜,主人答拜。宾以爵降奠于篚。乃羞。宰夫羞房中之羞,司士羞庶羞于尸、祝、主人、主妇,内羞在右,庶羞在左。

主人降,拜众宾,洗,献众宾。其荐脀,其位,其酬醋,皆如傧礼。主人洗,献兄弟与内宾,与私人,皆如傧礼。其位,其荐脀,皆如傧礼。卒,乃羞于宾、兄弟、内宾及私人,辩。

宾长献于尸,尸醋,献祝,致,醋。宾以爵降,实于篚。

宾、兄弟交错其酬。无算爵。

利洗爵,献于尸,尸醋。献祝,祝受,祭酒,啐酒,奠之。

主人出,立于阼阶上,西面。祝出,立于西阶上,东面。祝告于主人曰;「利成。」祝入。主人降,立于阼阶东,西面。尸谡,祝前,尸从,遂出于庙门。祝反,复位于室中。祝命佐食彻尸俎。佐食乃出尸俎于庙门外,有司受,归之。彻阼荐俎。

乃餕,如傧。

卒餕,有司官彻馈,馔于室中西北隅,南面,如馈之设,右几,厞用席。纳一尊于室中。司宫扫祭。主人出,立于阼阶上。西面。祝执其俎以出,立于西阶上,东面。司宫阖牖户。祝告利成,乃执俎以出于庙门外,有司受,归之。众宾出。主人拜送于庙门外,乃反。妇人乃彻,彻室中之馔。

\end{document}
