\PassOptionsToPackage{unicode=true}{hyperref} % options for packages loaded elsewhere
\PassOptionsToPackage{hyphens}{url}
%
\documentclass[]{article}
\usepackage{lmodern}
\usepackage{amssymb,amsmath}
\usepackage{ifxetex,ifluatex}
\usepackage{fixltx2e} % provides \textsubscript
\ifnum 0\ifxetex 1\fi\ifluatex 1\fi=0 % if pdftex
  \usepackage[T1]{fontenc}
  \usepackage[utf8]{inputenc}
  \usepackage{textcomp} % provides euro and other symbols
\else % if luatex or xelatex
  \usepackage{unicode-math}
  \defaultfontfeatures{Ligatures=TeX,Scale=MatchLowercase}
\fi
% use upquote if available, for straight quotes in verbatim environments
\IfFileExists{upquote.sty}{\usepackage{upquote}}{}
% use microtype if available
\IfFileExists{microtype.sty}{%
\usepackage[]{microtype}
\UseMicrotypeSet[protrusion]{basicmath} % disable protrusion for tt fonts
}{}
\IfFileExists{parskip.sty}{%
\usepackage{parskip}
}{% else
\setlength{\parindent}{0pt}
\setlength{\parskip}{6pt plus 2pt minus 1pt}
}
\usepackage{hyperref}
\hypersetup{
            pdfborder={0 0 0},
            breaklinks=true}
\urlstyle{same}  % don't use monospace font for urls
\setlength{\emergencystretch}{3em}  % prevent overfull lines
\providecommand{\tightlist}{%
  \setlength{\itemsep}{0pt}\setlength{\parskip}{0pt}}
\setcounter{secnumdepth}{0}
% Redefines (sub)paragraphs to behave more like sections
\ifx\paragraph\undefined\else
\let\oldparagraph\paragraph
\renewcommand{\paragraph}[1]{\oldparagraph{#1}\mbox{}}
\fi
\ifx\subparagraph\undefined\else
\let\oldsubparagraph\subparagraph
\renewcommand{\subparagraph}[1]{\oldsubparagraph{#1}\mbox{}}
\fi

% set default figure placement to htbp
\makeatletter
\def\fps@figure{htbp}
\makeatother


\date{}

\begin{document}

\hypertarget{header-n0}{%
\section{公孙龙子}\label{header-n0}}

\begin{center}\rule{0.5\linewidth}{\linethickness}\end{center}

\tableofcontents

\begin{center}\rule{0.5\linewidth}{\linethickness}\end{center}

\hypertarget{header-n10}{%
\subsection{迹府}\label{header-n10}}

公孙龙,六国时辩士也。疾名实之散乱,因资材之所长,为``守白''之论。
假物取譬,以``守白''辩,谓白马为非马也。白马为非马者,言白所以名色,言
马所以名形也;色非形,形非色也。夫言色则形不当与,言形则色不宜从,今合
以为物,非也。如求白马于厩中,无有,而有骊色之马,然不可以应有白马也。
不可以应有白马,则所求之马亡矣;亡则白马竟非马。欲推是辩,以正名实而化
天下焉。

龙于孔穿会赵平原君家。穿曰:``素闻先生高谊,愿为弟子久,但不取先生
以白马为非马耳!请去此术,则穿请为弟子。''

龙曰:``先生之言悖。龙之所以为名者,乃以白马之论尔!今使龙去之,则
无以教焉。且欲师之者,以智与学不如也。今使龙去之,此先教而后师也;先教
而后师之者,悖。

``且白马非马,乃仲尼之所取。龙闻楚王张繁弱之弓,载亡归之矢,以射蛟
口于云梦之圃,而丧其弓。左右请求之。王曰:`止。楚人遗弓,楚人得之,又
何求乎?'仲尼闻之曰:`楚王仁义而未遂也。亦曰人亡弓,人得之而已,何必
楚?'若此,仲尼异`楚人'与所谓`人'。夫是仲尼异`楚人'与所谓`人',
而非龙`白马'于所谓`马',悖。''

``先生修儒术而非仲尼之所取,欲学而使龙去所教,则虽百龙,固不能当前
矣。''孔穿无以应焉。

公孙龙,赵平原君之客也;孔穿,孔子之孙也。穿与龙会。穿谓龙曰:``臣
居鲁,侧闻下风,高先生之智,说先生之行,愿受益之日久矣,乃今得见。然所
不取先生者,独不取先生之以白马为非马耳。请去白马非马之学,穿请为弟子。''

公孙龙曰:``先生之言悖。龙之学,以白马为非马者也。使龙去之,则龙无
以教;无以教而乃学于龙也者,悖。且夫欲学于龙者,以智与学焉为不逮也。今
教龙去白马非马,是先教而后师之也;先教而后师之,不可。''

``先生之所以教龙者,似齐王之谓尹文也。齐王之谓尹文曰:`寡人甚好士,
以齐国无士,何也?'尹文曰:`愿闻大王之所谓士者。'齐王无以应。尹文曰:
`今有人于此,事君则忠,事亲则孝,交友则信,处乡则顺,有此四行,可谓士
乎?'齐王曰:`善!此真吾所谓士也。'尹文曰:`王得此人,肯以为臣乎?'
王曰:`所愿而不可得也。'''

``是时齐王好勇。于是尹文曰:`使此人广众大庭之中,见侵侮而终不敢斗,
王将以为臣乎?'王曰:`钜士也?见侮而不斗,辱也!辱则寡人不以为臣矣。'
尹文曰:`唯见侮而不斗,未失其四行也。是人失其四行,其所以为士也然。而
王一以为臣,一不以为臣,则向之所谓士者,乃非士乎?'齐王无以应。''

``尹文曰:`今有人君,将理其国,人有非则非之,无非则亦非之;有功则
赏之,无功则亦赏之,而怨人之不理也,可乎?'齐王曰:`不可。'尹文曰:
`臣口观下吏之理齐,齐方若此矣。'王曰:`寡人理国,信若先生之烟,人虽
不理,寡人不敢怨也。意未至然与?'

``尹文曰:`言之敢无说乎?王之令曰:`杀人者死,伤人者刑。'人有畏
王之令者,见侮而终不敢斗,是全王之令也。而王曰:`见侮而不斗者,辱也。
'谓之辱,非之也。无非而王非之,故因除其籍,不以为臣也。不以为臣者,罚
之也。此无而王罚之也。且王辱不敢斗者,必荣敢斗者也;荣敢斗者,是而王是
之,必以为臣矣。必以为臣者,赏之也。彼无功而王赏之。王之所赏,吏之所诛
也;上之所是,而法之所非也。赏罚是非,相与四谬,虽十黄帝,不能理也。'
齐王无以应。''

``故龙以子之言有似齐王。子知难白马之非马,不知所以难之说,以此,犹
好士之名,而不知察士之类。''

\hypertarget{header-n13}{%
\subsection{白马论}\label{header-n13}}

``白马非马'',可乎?

曰:可。

曰:何哉?

曰:马者,所以命形也;白者,所以命色也。命色者非名形也。故曰:
``白马非马''。

曰:有马不可谓无马也。不可谓无马者,非马也?有白马为有马,白
之,非马何也?

曰:求马,黄、黑马皆可致;求白马,黄、黑马不可致。是白马乃马
也,是所求一也。所求一者,白者不异马也,所求不异,如黄、黑马有可有不
可,何也?可与不可,其相非明。如黄、黑马一也,而可以应有马,而不可以
应有白马,是白马之非马,审矣!

曰:以马之有色为非马,天下非有无色之马。天下无马可乎?

曰:马固有色,故有白马。使马无色,有马如已耳,安取白马?故白
马非马也。白马者,马与白也。黑与白,马也?故曰白马非马业。

曰:马未与白为马,白未与马为白。合马与白,复名白马。是相与以
不相与为名,未可。故曰:白马非马未可。

曰:以``有白马为有马'',谓有白马为有黄马,可乎?

曰:未可。

曰:以``有马为异有黄马'',是异黄马与马也;异黄马与马,是以黄为
非马。以黄马为非马,而以白马为有马,此飞者入池而棺椁异处,此天下之悖
言辞也。

以``有白马不可谓无马''者,离白之谓也;不离者有白马不可谓有马
也。故所以为有马者,独以马为有马耳,非以白马为有马耳。故其为有马也,
不可以谓``白马''也。

以``白者不定所白'',忘之而可也。白马者,言白定所白也,定所白
者非白也。马者,无去取于色,故黄、黑皆所以应;白马者,有去取于色,黄、
黑马皆所以色去,故唯白马独可以应耳。无去者非有去也,故曰:``白马非马''。

\hypertarget{header-n14}{%
\subsection{指物论}\label{header-n14}}

物莫非指,而指非指。

天下无指,物无可以谓物。非指者天下,而物可谓指乎?

指也者,天下之所无也;物也者,天下之所有也。以天下之所有,为天下之所无,未可。

天下无指,而物不可谓指也。不可谓指者,非指也?非指者,物莫非指也。

天下无指而物不可谓指者,非有非指也。非有非指者,物莫非指也。物莫非指者,而指非指也。

天下无指者,生于物之各有名,不为指也。不为指而谓之指,是无部为指。以有不为指之无不为指,未可。

以``指者天下之所无''。天下无指者,物不可谓无指也;不可谓无指者。非有非指也;非有非指者,物莫非指、指非非指也,指与物非指也。

使天下无物指,谁径谓非指?天下无物,谁径谓指?天下有指无物指,谁径谓非指、径谓无物非指?

且夫指固自为非指,奚待于物而乃与为指?

\hypertarget{header-n15}{%
\subsection{名实论}\label{header-n15}}

天地与其所产焉,物也。物以物其所物而不过焉,实也。实以实其所实
而不旷焉,位也。出其所位,非位,位其所位焉,正也。

以其所正,正其所不正;以其所不正,疑其所正。其正者,正其所实也
;正其所实者,正其名也。

其名正则唯乎其彼此焉。谓彼而彼不唯乎彼,则彼谓不行;谓此而此不
唯乎此,则此谓不行。其以当不当也。不当而当,乱也。

故彼彼当乎彼,则唯乎彼,其谓行彼;此此当乎此,则唯乎此,其谓行
此。其以当而当也。以当而当,正也。

故彼彼止于彼,此此止于此,可。彼此而彼且此,此彼而此且彼,不可。

夫名,实谓也。知此之非此也,知此之不在此也,则不谓也;知彼之非
彼也,知彼之不在彼也,则不谓也。

至矣哉,古之明王。审其名实,慎其所谓。至矣哉,古之明王。

\hypertarget{header-n16}{%
\subsection{通变论}\label{header-n16}}

曰:二有一乎?

曰:二无一。

曰:二有右乎?

曰:二无右。

曰:二有左乎?

曰:二无左。

曰:右可谓二乎?

曰:不可。

曰:左可谓二乎?

曰:不可。

曰:左与右可谓二乎?

曰:可。

曰:谓变非变,可乎?

曰:可。

曰:右有与,可谓变乎?

曰:可。

曰:变奚?

曰:右。

曰:右苟变,安可谓右?

曰:苟不变,安可谓变?

曰:二苟无左,又无右,二者左与右,奈何?

曰:羊合牛非马,牛合羊非鸡。

曰:何哉?

曰:羊与牛唯异,羊有齿,牛无齿,而牛之非羊也、羊之非牛也,未可
。是不俱有而或类焉。

羊有角,牛有角,牛之而羊也,未可。是俱有而类之不同也。羊牛有角
,马无角,马有尾,羊牛无尾,故曰羊合牛非马也。非马者,无马也。无马者,羊不二,牛不二,而羊牛二,是而羊而牛非马,可也。

若举而以是,犹类之不同,若左右,犹是举。

牛羊有毛,鸡有羽。谓鸡足一,数足二,二而一故三;谓牛羊足一,数
足四,四而一故五。牛羊足五,鸡足三,故曰牛合羊非鸡。非有以非鸡也。

与马以鸡宁马。材不材,其无以类,审矣!举是乱名,是谓狂举。

曰:他辩。

曰:青以白非黄,白以青非碧。

曰:何哉?

曰:青白不相与而相与,反对也;不相邻而相邻,不害其方也。不害其
方也。不害其方者,反而对,各当其所,若左右不骊。故一于青不可,一于白不可,恶乎其有黄矣哉?黄其正矣,是正举也。其有君臣之于国焉,
故强寿矣!

而且青骊乎白而白不胜也。白足之胜矣而不胜,是木贼金也。木贼金者
碧,碧则非正举矣。青白不相与而相与,不相胜则两明也。争而明,其色碧也。

与其碧宁黄。黄其马也,其与类乎,碧其鸡也,其与暴乎!

暴则君臣争而两明也。两明者,昏不明,非正举也。非正举者,名实无
当,骊色章焉,故曰两明也。两明而道丧,其无有以正焉。

\hypertarget{header-n17}{%
\subsection{坚白论}\label{header-n17}}

『坚白石三』可乎?

曰:不可。

曰:二可乎?

曰:可。

曰:何哉?

曰:无坚得白,其举也二;无白得坚,其举也二。

曰:得其所白,不可谓无白;得其所坚,不可谓无坚。而之石也,之于
然也,非三也?

曰:视不得其所坚而得,其所白〔得其所白〕者,无坚也;拊不得其所
白而得其所坚,得其〔所〕坚〔者〕,无白也。

曰:天下无白,不可以视石;天下无坚,不可以谓石。坚、白不相外,
藏三,可乎?

有自藏也,非藏而藏也。

曰:其白也,其坚也,而石必得以相盈,其自藏奈何?

曰:得其白,得其坚,见与不见谓之离;不见离,一二不相盈,故离。
离也者,藏也。

曰:石之白,石之坚,见与不见,二与三,若广修而相盈也。其非举乎?

曰:物白焉,不定其所白;物坚焉,不定其所坚。不定者,兼。恶乎其 石也?

曰:循石,非彼无石,非石无所取坚白。〔坚白石〕不相离也,固乎然
,其无已!

曰:于石,一也;坚白,二也,而在于石。故有知焉,有不知焉;有见
焉,〔有不见焉〕。故知与不知相与离,见与不见相与藏。藏故,孰谓之不离?

曰:目不能坚,手不能白,不可谓无坚,不可谓无白。其异任也,其无
以代也。坚白域与石,恶乎离?

曰:坚未与石为坚,而物兼,未与〔物〕为坚而坚必坚。其不坚石物而
坚,天下未有若坚而坚藏。

白固不能自白,恶能白石物乎?若白者必白,则不白物而白焉。黄、黑 与之然。

石其无有。恶取坚白石乎?故离也。离也者,因是。

力与知果,不若,因是。

且犹白以目〔见〕,〔目〕以火见,而火不见,则火与目不见而神见;
神不见,而见离。

坚以手〔知〕,而手以捶〔知〕,是捶与手知而不知,而神与不知。

神乎!是之谓离焉。离也者,天下故独而正。

\end{document}
