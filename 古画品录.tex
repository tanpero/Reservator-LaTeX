\PassOptionsToPackage{unicode=true}{hyperref} % options for packages loaded elsewhere
\PassOptionsToPackage{hyphens}{url}
%
\documentclass[]{article}
\usepackage{lmodern}
\usepackage{amssymb,amsmath}
\usepackage{ifxetex,ifluatex}
\usepackage{fixltx2e} % provides \textsubscript
\ifnum 0\ifxetex 1\fi\ifluatex 1\fi=0 % if pdftex
  \usepackage[T1]{fontenc}
  \usepackage[utf8]{inputenc}
  \usepackage{textcomp} % provides euro and other symbols
\else % if luatex or xelatex
  \usepackage{unicode-math}
  \defaultfontfeatures{Ligatures=TeX,Scale=MatchLowercase}
\fi
% use upquote if available, for straight quotes in verbatim environments
\IfFileExists{upquote.sty}{\usepackage{upquote}}{}
% use microtype if available
\IfFileExists{microtype.sty}{%
\usepackage[]{microtype}
\UseMicrotypeSet[protrusion]{basicmath} % disable protrusion for tt fonts
}{}
\IfFileExists{parskip.sty}{%
\usepackage{parskip}
}{% else
\setlength{\parindent}{0pt}
\setlength{\parskip}{6pt plus 2pt minus 1pt}
}
\usepackage{hyperref}
\hypersetup{
            pdfborder={0 0 0},
            breaklinks=true}
\urlstyle{same}  % don't use monospace font for urls
\setlength{\emergencystretch}{3em}  % prevent overfull lines
\providecommand{\tightlist}{%
  \setlength{\itemsep}{0pt}\setlength{\parskip}{0pt}}
\setcounter{secnumdepth}{0}
% Redefines (sub)paragraphs to behave more like sections
\ifx\paragraph\undefined\else
\let\oldparagraph\paragraph
\renewcommand{\paragraph}[1]{\oldparagraph{#1}\mbox{}}
\fi
\ifx\subparagraph\undefined\else
\let\oldsubparagraph\subparagraph
\renewcommand{\subparagraph}[1]{\oldsubparagraph{#1}\mbox{}}
\fi

% set default figure placement to htbp
\makeatletter
\def\fps@figure{htbp}
\makeatother


\date{}

\begin{document}

\hypertarget{header-n0}{%
\section{古画品录}\label{header-n0}}

\begin{center}\rule{0.5\linewidth}{\linethickness}\end{center}

\tableofcontents

\begin{center}\rule{0.5\linewidth}{\linethickness}\end{center}

\hypertarget{header-n10}{%
\subsection{序文}\label{header-n10}}

夫画品者,盖众画之优劣也。图绘者,莫不明劝戒、著升沉,千载寂寥,披图可鉴。虽画有六法,罕能尽该。而自古及今,各善一节。六法者何?一,气韵生动是也;二,骨法用笔是也;三,应物象形是也;四,随类赋彩是也;五,经营位置是也;六,传移模写是也。唯陆探微、卫协备该之矣。然迹有巧拙,艺无古今,谨依远近,随其品第,裁成序引。故此所述不广其源,但传出自神仙,莫之闻见也。

\hypertarget{header-n16}{%
\subsection{第一品}\label{header-n16}}

陆探微。事五代宋明帝,吴人。穷理尽性,事绝言象。包前孕后,古今独立。非复激扬所以称赞,但价之极乎上上品之外,无他寄言,故屈标第一等。

曹不兴。五代吴时事孙权,吴兴人。不兴之迹,殆莫复传。唯秘阁之内一龙而已。观其风骨,名岂虚成!

卫协。五代晋时。占画之略,至协始精。六法之中,迨为兼善。虽不说备形妙,颇得壮气。陵跨群雄,旷代绝笔。

张墨、荀((曰助))五代晋时。风范气候,极妙参神。但取精灵,遗其骨法。若拘以物体,则未见精粹。若取之外,方厌高腴,可谓微妙也。

\hypertarget{header-n20}{%
\subsection{第二品}\label{header-n20}}

顾骏之。神韵气力,不逮前贤;精微谨细,有过往哲。始变古则今,赋彩制形,皆创新意。如包牺始更卦体,史籀初改画法。常结构层楼,以为画所。风雨炎燠之时,故不操笔;天和气爽之日方乃染毫。登楼去梯,妻子罕见。画蝉雀,骏之始也。宋大明中,天下莫敢竞矣。

陆绥。体韵遒举,风彩飘然。一点一拂,动笔皆奇。传世盖少,所谓希见卷轴,故为宝也。\\
~\\
袁((艹倩))。比方陆氏,最为高逸。象人之妙,亚美前贤。但志守师法,更无新意。然和璧微玷,岂贬十城之价也。

\hypertarget{header-n24}{%
\subsection{第三品}\label{header-n24}}

姚昙度。画有逸方,巧变锋出,((鬼音))魁神鬼,皆能绝妙。奇正咸宜,雅郑兼善,莫不俊拔出人意表,天挺生知非学所及。虽纤微长短,往往失之。而舆皂之中,莫与为匹。岂直栋梁萧艾可搪突((王与))((王番))者哉!

顾恺之。五代晋时晋陵无锡人。字长康,小字虎头。除体精微,笔无妄下。但迹不逮意,声过其实。

毛惠远。画体周赡,无适弗该,出入穷奇,纵黄逸笔,力遒韵雅,超迈绝伦。其挥霍必也极妙,至于定质,块然未尽。其善神鬼及马,泥滞于体,颇有拙也。

夏瞻。虽气力不足,而精彩有余。擅名远代,事非虚美。

戴逵。情韵连绵,风趣巧拔。善图贤圣,百工所范。荀、卫以后,实为领袖。及乎子((禺页))能继其美。

江僧宝。斟酌袁陆,亲渐朱蓝。用笔骨梗,甚有师法。像人之外,非其所长也。

吴((日东))。体法雅媚,制置才巧。擅美当年,有声京洛。

张则。意思横逸,动笔新奇。师心独见,鄙于综采。变巧不竭,若环之无端,景多触目,谢题徐落云此二人后不得预焉。

陆杲。体制不凡,跨迈流欲。时有合作,往往出人点画之间。动流恢服,传于后者,殆不盈握。桂枝一芳,足征本性。流液之素,难效其功。

\hypertarget{header-n28}{%
\subsection{第四品}\label{header-n28}}

蘧道愍。章继伯。并善寺壁,兼长画扇,人马分数,毫厘不失,别体之妙,亦为入神。

顾宝先。全法陆家,事之宗禀。方之袁((艹倩)),可谓小巫。

王微。史道硕。五代晋时。并师荀、卫,各体善能。然王得其细,史传以似真。细而论之,景玄为劣。

\hypertarget{header-n32}{%
\subsection{第五品}\label{header-n32}}

刘顼。用意绵密,画体简细,而笔迹困弱。形制单省。其于所长,妇人为最。但纤细过度,翻更失真,然观察祥审,甚得姿态。

晋明帝。讳绍,元帝长子,师王厉。虽略于形色,颇得神气。笔迹超越,亦有奇观。

刘绍祖。善于传写,不闲其思。至于雀鼠笔迹,历落往往出群。时人为之语,号曰移画,然述而不作,非画所先。

\hypertarget{header-n69}{%
\subsection{第六品}\label{header-n69}}

宋炳。炳明于六法,迄无适善,而含毫命素,必有损益,迹非准的,意足师放。

丁光。虽擅名蝉雀,而笔迹轻羸。非不精谨,乏于生气。

\end{document}
