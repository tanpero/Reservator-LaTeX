\PassOptionsToPackage{unicode=true}{hyperref} % options for packages loaded elsewhere
\PassOptionsToPackage{hyphens}{url}
%
\documentclass[]{article}
\usepackage{lmodern}
\usepackage{amssymb,amsmath}
\usepackage{ifxetex,ifluatex}
\usepackage{fixltx2e} % provides \textsubscript
\ifnum 0\ifxetex 1\fi\ifluatex 1\fi=0 % if pdftex
  \usepackage[T1]{fontenc}
  \usepackage[utf8]{inputenc}
  \usepackage{textcomp} % provides euro and other symbols
\else % if luatex or xelatex
  \usepackage{unicode-math}
  \defaultfontfeatures{Ligatures=TeX,Scale=MatchLowercase}
\fi
% use upquote if available, for straight quotes in verbatim environments
\IfFileExists{upquote.sty}{\usepackage{upquote}}{}
% use microtype if available
\IfFileExists{microtype.sty}{%
\usepackage[]{microtype}
\UseMicrotypeSet[protrusion]{basicmath} % disable protrusion for tt fonts
}{}
\IfFileExists{parskip.sty}{%
\usepackage{parskip}
}{% else
\setlength{\parindent}{0pt}
\setlength{\parskip}{6pt plus 2pt minus 1pt}
}
\usepackage{hyperref}
\hypersetup{
            pdfborder={0 0 0},
            breaklinks=true}
\urlstyle{same}  % don't use monospace font for urls
\setlength{\emergencystretch}{3em}  % prevent overfull lines
\providecommand{\tightlist}{%
  \setlength{\itemsep}{0pt}\setlength{\parskip}{0pt}}
\setcounter{secnumdepth}{0}
% Redefines (sub)paragraphs to behave more like sections
\ifx\paragraph\undefined\else
\let\oldparagraph\paragraph
\renewcommand{\paragraph}[1]{\oldparagraph{#1}\mbox{}}
\fi
\ifx\subparagraph\undefined\else
\let\oldsubparagraph\subparagraph
\renewcommand{\subparagraph}[1]{\oldsubparagraph{#1}\mbox{}}
\fi

% set default figure placement to htbp
\makeatletter
\def\fps@figure{htbp}
\makeatother


\date{}

\begin{document}

\hypertarget{header-n2}{%
\section{古诗十九首}\label{header-n2}}

\begin{center}\rule{0.5\linewidth}{\linethickness}\end{center}

\tableofcontents

\begin{center}\rule{0.5\linewidth}{\linethickness}\end{center}

\hypertarget{header-n6}{%
\subsection{行行重行行}\label{header-n6}}

行行重行行,与君生别离。\\
相去万余里,各在天一涯。\\
道路阻且长,会面安可知?\\
胡马依北风,越鸟巢南枝。\\
相去日已远,衣带日已缓。\\
浮云蔽白日,游子不顾返。\\
思君令人老,岁月忽已晚。\\
弃捐勿复道,努力加餐饭。

\hypertarget{header-n10}{%
\subsection{青青河畔草}\label{header-n10}}

青青河畔草,郁郁园中柳。\\
盈盈楼上女,皎皎当窗牖。\\
娥娥红粉妆,纤纤出素手。\\
昔为倡家女,今为荡子妇。\\
荡子行不归,空床难独守。

\hypertarget{header-n14}{%
\subsection{青青陵上柏}\label{header-n14}}

青青陵上柏,磊磊涧中石。\\
人生天地间,忽如远行客。\\
斗酒相娱乐,聊厚不为薄。\\
驱车策驽马,游戏宛与洛。\\
洛中何郁郁,冠带自相索。\\
长衢罗夹巷,王侯多第宅。\\
两宫遥相望,双阙百余尺。\\
极宴娱心意,戚戚何所迫?

\hypertarget{header-n18}{%
\subsection{今日良宴会}\label{header-n18}}

今日良宴会,欢乐难具陈。\\
弹筝奋逸响,新声妙入神。\\
令德唱高言,识曲听其真。\\
齐心同所愿,含意俱未申。\\
人生寄一世,奄忽若飙尘。\\
何不策高足,先据要路津。\\
无为守穷贱,坎坷长苦辛。

\hypertarget{header-n22}{%
\subsection{西北有高楼}\label{header-n22}}

西北有高楼,上与浮云齐。\\
交疏结绮窗,阿阁三重阶。\\
上有弦歌声,音响一何悲!\\
谁能为此曲,无乃杞梁妻。\\
清商随风发,中曲正徘徊。\\
一弹再三叹,慷慨有余哀。\\
不惜歌者苦,但伤知音稀。\\
愿为双鸿鹄,奋翅起高飞。

\hypertarget{header-n26}{%
\subsection{涉江采芙蓉}\label{header-n26}}

涉江采芙蓉,兰泽多芳草。\\
采之欲遗谁,所思在远道。\\
还顾望旧乡,长路漫浩浩。\\
同心而离居,忧伤以终老。

\hypertarget{header-n30}{%
\subsection{明月何皎皎}\label{header-n30}}

明月何皎皎,照我罗床帏。\\
忧愁不能寐,揽衣起徘徊。\\
客行虽云乐,不如早旋归。\\
出户独彷徨,愁思当告谁!\\
引领还入房,泪下沾裳衣。

\hypertarget{header-n34}{%
\subsection{冉冉孤生竹}\label{header-n34}}

冉冉孤生竹,结根泰山阿。\\
与君为新婚,菟丝附女萝。\\
菟丝生有时,夫妇会有宜。\\
千里远结婚,悠悠隔山陂。\\
思君令人老,轩车来何迟!\\
伤彼蕙兰花,含英扬光辉。\\
过时而不采,将随秋草萎。\\
君亮执高节,贱妾亦何为!

\hypertarget{header-n38}{%
\subsection{庭中有奇树}\label{header-n38}}

庭中有奇树,绿叶发华滋。\\
攀条折其荣,将以遗所思。\\
馨香盈怀袖,路远莫致之。\\
此物何足贵,但感别经时。

\hypertarget{header-n42}{%
\subsection{迢迢牵牛星}\label{header-n42}}

迢迢牵牛星,皎皎河汉女。\\
纤纤擢素手,札札弄机杼。\\
终日不成章,泣涕零如雨。\\
河汉清且浅,相去复几许!\\
盈盈一水间,脉脉不得语。

\hypertarget{header-n46}{%
\subsection{回车驾言迈}\label{header-n46}}

回车驾言迈,悠悠涉长道。\\
四顾何茫茫,东风摇百草。\\
所遇无故物,焉得不速老。\\
盛衰各有时,立身苦不早。\\
人生非金石,岂能长寿考?\\
奄忽随物化,荣名以为宝。

\hypertarget{header-n50}{%
\subsection{东城高且长}\label{header-n50}}

东城高且长,逶迤自相属。\\
回风动地起,秋草萋已绿。\\
四时更变化,岁暮一何速!\\
晨风怀苦心,蟋蟀伤局促。\\
荡涤放情志,何为自结束?\\
燕赵多佳人,美者颜如玉。\\
被服罗裳衣,当户理清曲。\\
音响一何悲!弦急知柱促。\\
驰情整巾带,沉吟聊踯躅。\\
思为双飞燕,衔泥巢君屋。

\hypertarget{header-n54}{%
\subsection{驱车上东门}\label{header-n54}}

驱车上东门,遥望郭北墓。\\
白杨何萧萧,松柏夹广路。\\
下有陈死人,杳杳即长暮。\\
潜寐黄泉下,千载永不寤。\\
浩浩阴阳移,年命如朝露。\\
人生忽如寄,寿无金石固。\\
万岁更相送,贤圣莫能度。\\
服食求神仙,多为药所误。\\
不如饮美酒,被服纨与素。

\hypertarget{header-n58}{%
\subsection{去者日以疏}\label{header-n58}}

去者日以疏,来者日以亲。\\
出郭门直视,但见丘与坟。\\
古墓犁为田,松柏摧为薪。\\
白杨多悲风,萧萧愁杀人。\\
思还故里闾,欲归道无因。

\hypertarget{header-n62}{%
\subsection{生年不满百}\label{header-n62}}

生年不满百,常怀千岁忧。\\
昼短苦夜长,何不秉烛游!\\
为乐当及时,何能待来兹?\\
愚者爱惜费,但为後世嗤。\\
仙人王子乔,难可与等期。

\hypertarget{header-n66}{%
\subsection{凛凛岁云暮}\label{header-n66}}

凛凛岁云暮,蝼蛄夕鸣悲。\\
凉风率已厉,游子寒无衣。\\
锦衾遗洛浦,同袍与我违。\\
独宿累长夜,梦想见容辉。\\
良人惟古欢,枉驾惠前绥。\\
愿得常巧笑,携手同车归。\\
既来不须臾,又不处重闱。\\
亮无晨风翼,焉能凌风飞?\\
眄睐以适意,引领遥相睎。\\
徙倚怀感伤,垂涕沾双扉。

\hypertarget{header-n70}{%
\subsection{孟冬寒气至}\label{header-n70}}

孟冬寒气至,北风何惨栗。\\
愁多知夜长,仰观众星列。\\
三五明月满,四五蟾兔缺。\\
客从远方来,遗我一书札。\\
上言长相思,下言久离别。\\
置书怀袖中,三岁字不灭。\\
一心抱区区,惧君不识察。

\hypertarget{header-n74}{%
\subsection{客从远方来}\label{header-n74}}

客从远方来,遗我一端绮。\\
相去万余里,故人心尚尔。\\
文采双鸳鸯,裁为合欢被。\\
著以长相思,缘以结不解。\\
以胶投漆中,谁能别离此?

\hypertarget{header-n78}{%
\subsection{明月皎夜光}\label{header-n78}}

明月皎夜光,促织鸣东壁。\\
玉衡指孟冬,众星何历历。\\
白露沾野草,时节忽复易。\\
秋蝉鸣树间,玄鸟逝安适。\\
昔我同门友,高举振六翮。\\
不念携手好,弃我如遗迹。\\
南箕北有斗,牵牛不负轭。\\
良无盘石固,虚名复何益?

\end{document}
