\PassOptionsToPackage{unicode=true}{hyperref} % options for packages loaded elsewhere
\PassOptionsToPackage{hyphens}{url}
%
\documentclass[]{article}
\usepackage{lmodern}
\usepackage{amssymb,amsmath}
\usepackage{ifxetex,ifluatex}
\usepackage{fixltx2e} % provides \textsubscript
\ifnum 0\ifxetex 1\fi\ifluatex 1\fi=0 % if pdftex
  \usepackage[T1]{fontenc}
  \usepackage[utf8]{inputenc}
  \usepackage{textcomp} % provides euro and other symbols
\else % if luatex or xelatex
  \usepackage{unicode-math}
  \defaultfontfeatures{Ligatures=TeX,Scale=MatchLowercase}
\fi
% use upquote if available, for straight quotes in verbatim environments
\IfFileExists{upquote.sty}{\usepackage{upquote}}{}
% use microtype if available
\IfFileExists{microtype.sty}{%
\usepackage[]{microtype}
\UseMicrotypeSet[protrusion]{basicmath} % disable protrusion for tt fonts
}{}
\IfFileExists{parskip.sty}{%
\usepackage{parskip}
}{% else
\setlength{\parindent}{0pt}
\setlength{\parskip}{6pt plus 2pt minus 1pt}
}
\usepackage{hyperref}
\hypersetup{
            pdfborder={0 0 0},
            breaklinks=true}
\urlstyle{same}  % don't use monospace font for urls
\setlength{\emergencystretch}{3em}  % prevent overfull lines
\providecommand{\tightlist}{%
  \setlength{\itemsep}{0pt}\setlength{\parskip}{0pt}}
\setcounter{secnumdepth}{0}
% Redefines (sub)paragraphs to behave more like sections
\ifx\paragraph\undefined\else
\let\oldparagraph\paragraph
\renewcommand{\paragraph}[1]{\oldparagraph{#1}\mbox{}}
\fi
\ifx\subparagraph\undefined\else
\let\oldsubparagraph\subparagraph
\renewcommand{\subparagraph}[1]{\oldsubparagraph{#1}\mbox{}}
\fi

% set default figure placement to htbp
\makeatletter
\def\fps@figure{htbp}
\makeatother


\date{}

\begin{document}

\hypertarget{header-n2186}{%
\section{天工开物}\label{header-n2186}}

\begin{center}\rule{0.5\linewidth}{\linethickness}\end{center}

\tableofcontents

\begin{center}\rule{0.5\linewidth}{\linethickness}\end{center}

\hypertarget{header-n2191}{%
\subsection{上篇}\label{header-n2191}}

\hypertarget{header-n2193}{%
\subsubsection{乃粒}\label{header-n2193}}

宋子曰:上古神农氏若存若亡,然味其徽号,两言至今存矣。生人不能久生而五谷生之,五谷不能自生而生人生之。土脉历时代而异,种性随水土而分。不然,神农去陶唐,粒食已千年矣。耒耜之利,以教天下,岂有隐焉。而纷纷嘉种,必待后稷详明,其故何也?纨裤之子,以赭衣视笠蓑;经生之家,以农夫为诟詈。晨炊晚饷,知其味而忘其源者众矣!夫先农而系之以神,岂人力之所为哉!

\textbf{○总名}

凡谷无定名,百谷指成数言。五谷则麻、菽、麦、稷、黍,独遗稻者,以著书圣贤起自西北也。今天下育民人者,稻居什七,而来、牟、黍、稷居什三。麻、菽二者,功用已全入蔬饵膏馔之中,而犹系之谷者。从其朔也。

\textbf{○稻}

凡稻种最多。不粘者,禾曰亢,米曰粳。粘者,禾曰余,米曰糯。(南方无粘黍,酒皆糯米所为。)质本粳而晚收带粘(俗名婺源光之类)不可为酒,只可为粥者,又一种性也。凡稻谷形有长芒、短芒、(江南名长芒者曰浏阳早,短芒者曰吉安早。)长粒、尖粒、圆顶、扁圆面不一,其中米色有雪白、牙黄、大赤、半紫、杂黑不一。

湿种之期,最早者春分以前,名为社种,(遇天寒有冻死不生者。)最迟者后于清明。凡播种,先以稻麦稿包浸数日,俟其生芽,撒于田中,生出寸许,其名曰秧。秧生三十日即拨起分栽。若田亩逢旱干、水溢,不可插秧。秧过期,老而长节,即栽于亩中,生谷数粒,结果而已。凡秧田一亩所生秧,供移栽二十五亩。凡秧既分栽后,早者七十日即收获,(粳有救公饥、喉下急,糯有金包银之类,方语百千,不可殚述。)最迟者历夏及冬二百日方收获。其冬季播种、仲夏即收者,则广南之稻,地无霜雪故也。

凡稻旬日失水,即愁旱干。夏种冬收之谷,必山间源水不绝之亩,其谷种亦耐久,其土脉亦寒,不催苗也。湖滨之田,待夏潦已过,六月方栽者,其秧立夏播种,撒藏高亩之上,以待时也。南方平原,田多一岁两栽两获者。其再栽秧,俗名晚糯,非粳类也。六月刈初禾,耕治老膏田,插再生秧。其秧清明时已偕早秧撒布。早秧一日无水即死,此秧历四五两月,任从烈日干无忧,此一异也。

凡再植稻遇秋多晴,则汲灌与稻相终始。农家勤苦,为春酒之需也。凡稻旬日失水则死期至,幻出旱稻一种,粳而不粘者,即高山可插,又一异也。香稻一种,取其芳气以供贵人,收实甚少,滋益全无,不足尚也。

\textbf{○稻宜}

凡稻,土脉焦枯,则穗实萧索。勤农粪田,多方以助之。人畜秽遗、榨油枯饼、(枯者,以去膏而得名也。胡麻、莱菔子为上,芸苔次之,大眼桐又次之,樟、桕、棉花又次之。)草皮木叶,以佐生机,普天之所同也。(南方磨绿豆粉者,取溲浆灌田肥甚。豆贱之时,撒黄豆于田,一粒烂土方寸,得谷之息倍焉。)土性带冷浆者,宜骨灰蘸秧根,(凡禽兽骨。)石灰淹苗足,向阳暖土不宜也。土脉坚紧者,宜耕陇,叠块压薪而烧之,填坟松土不宜也。

\textbf{○稻工 (耕 耙 磨耙 耘耔 具图)}

凡稻田刈获不再种者,土宜本秋耕垦,使宿稿化烂,敌粪力一倍。或秋旱无水及怠农春耕,则收获损薄也。凡粪田若撒枯浇泽,恐霖雨至,过水来,肥质随漂而去。谨视天时,在老农心计也。凡一耕之后,勤者再耕、三耕,然后施耙,则土质匀碎,而其中膏脉释化也。

凡牛力穷者,两人以扛悬耜,项背相望而起土。两人竟日仅敌一牛之力。若耕后牛穷,制成磨耙,两人肩手磨轧,则一日敌三牛之力也。凡牛,中国惟水、黄两种。水牛力倍于黄。但畜水牛者,冬与土室御寒,夏与池塘浴水,畜养心计亦倍于黄牛也。凡牛春前力耕汗出,切忌雨点,将雨则疾驱入室。候过谷雨,则任从风雨不惧也。

吴郡力田者,以锄代耜,不藉牛力。愚见贫农之家,会计牛值与水草之资,窃盗死病之变,不若人力亦便。假如有牛者,供办十亩。无牛用锄而勤者半之。既已无牛,则秋获之后,田中无复刍牧之患,而菽麦麻蔬诸种,纷纷可种,以再获偿半荒之亩,似亦相当也。

凡稻分秧之后数日,旧叶萎黄而更生新叶。青叶既长,则耔可施焉。(俗名挞禾。)植杖于手,以足扶泥壅根,并屈宿田水草,使不生也。凡宿田菵草之类,遇耔而屈折。而ㄗ、稗与荼、蓼非足力所可除者,则耘以继之。耘者苦在腰手,辨在两眸。非类既去,而嘉谷茂焉。从此泄以防潦,溉以防旱,旬月而``奄观銍刈''矣。

\textbf{○稻灾}

凡早稻种,秋初收藏,当午晒时烈日火气在内,入仓廪中关闭太急,则其谷粘带暑气。(勤农之家,偏受此患。)明年田有粪肥,土脉发烧,东南风助暖,则尽发炎火,大坏苗穗,此一灾也。若种谷晚凉入廪,或冬至数九天收贮雪水、冰水一瓮,(交春即不验。)清明湿种时,每石以数碗激洒,立解暑气,则任从东南风暖,而此苗清秀异常矣。(祟在种内,反怨鬼神。)

凡稻撒种时,或水浮数寸,其谷未即沉下,骤发狂风,堆积一隅,此二灾也。谨视风定而后撒,则沉匀成秧矣。凡谷种生秧之后,防雀聚食,此三灾也。立标飘扬鹰俑,则雀可驱矣。凡秧沉脚未定,阴雨连绵,则损折过半,此四灾也。邀天晴霁三日,则粒粒皆生矣。凡苗既函之后,亩上肥泽连发,南风薰热,函内生虫,(形似蚕茧。)此五灾也。邀天遇西风雨一阵,则虫化而谷生矣。凡苗吐穑之后,暮夜``鬼火''游烧,此六灾也。此火乃朽木腹中放出。凡木母火子,子藏母腹,母身未坏,子性千秋不灭。每逢多雨之年,孤野坟墓多被狐狸穿塌。其中棺板为水浸,朽烂之极,所谓母质坏也。火子无附,脱母飞扬。然阴火不见阳光,直待日没黄昏,此火冲隙而出,其力不能上腾,飘游不定,数尺而止。凡禾穑叶遇之立刻焦炎。逐火之人则他处树根放光,以为鬼也。奋梃击之,反有鬼变枯柴之说。不知向来鬼火见灯光而已化矣。(凡火未经人间传灯者,总属阴火,故见灯即灭。)

凡苗自函活以至颖栗,早者食水三斗,晚者食水五斗,失水即枯,(将刈之时少水一升,谷数虽存,米粒缩小,入碾臼中亦多断碎。)此七灾也。汲灌之智,人巧已无余矣。凡稻成熟之时,遇狂风吹粒殒落,或阴雨竟旬,谷粒沾湿自烂,此八灾也。然风灾不越三十里,阴雨灾不越三百里,偏方厄难亦不广被。风落不可为。若贫困之家,苦于无霁,将湿谷升于锅内,燃薪其下,炸去糠膜,收炒糗以充饥,亦补助造化之一端矣。

\textbf{○水利 (筒车 牛车 踏车 拔车 桔槔 皆具图)}

凡稻防旱藉水,独甚五谷。厥土沙、泥、硗、腻,随方不一。有三日即干者,有半月后干者。天泽不降,则人力挽水以济。凡河滨有制筒车者,堰陂障流,绕于车下,激轮使转,挽水入筒,一一倾于枧内,流入亩中。昼夜不息,百亩无忧。(不用水时,拴木碍止,使轮不转动。)其湖池不流水,或以牛力转盘,或聚数人踏转。车身长者二丈,短者半之。其内用龙骨拴串板,关水逆流而上。大抵一人竟日之力,灌田五亩,而牛则倍之。

其浅池、小浍不载长车者,则数尺之车,一人两手疾转,竟日之功可灌二亩而已。扬郡以风帆数扇,俟风转车,风息则止。此车为救潦,欲去泽水以便栽种。盖去水非取水也,不适济旱。用桔槔、辘轳,功劳又甚细已。

\textbf{○麦}

凡麦有数种,小麦曰来,麦之长也;大麦曰牟、曰广;杂麦曰雀、曰荞;皆以播种同时、花形相似、粉食同功而得麦名也。四海之内,燕、秦、晋、豫、齐鲁诸道,民粒食,小麦居半,而黍、稷、稻、粱仅居半。西极川、云,东至闽、浙,吴、楚腹焉,方长六千里中种小麦者,二十分而一,磨面以为捻头、环饵、馒首、汤料之需,而饔飧不及焉。种余麦者五十分而一,闾阎作苦以充朝膳,而贵介不与焉。

广麦独产陕西,一名青稞,即大麦,随土而变。而皮成青黑色者,秦人专以饲马,饥荒人乃食之。(大麦亦有粘者,河洛用以酿酒。)雀麦细穗,穗中又分十数细子,间亦野生。荞麦实非麦类,然以其为粉疗饥,传名为麦,则麦之而已。

凡北方小麦,历四时之气,自秋播种,明年初夏方收。南方者种与收期,时日差短。江南麦花夜发,江北麦花昼发,亦一异也。大麦种获期与小麦相同,荞麦则秋半下种,不两月而即收。其苗遇霜即杀,邀天降霜迟迟,则有收矣。

\textbf{○麦工 (北耕种 耨 具图)}

凡麦与稻初耕垦土则同,播种以后则耘耔诸勤苦皆属稻,麦惟施耨而已。凡北方厥土坟垆易解释者,种麦之法耕具差异,耕即兼种。其服牛起土者,耒不用耕,并列两铁于横木之上,其具方语曰镪。镪中间盛一小斗,贮麦种于内,其斗底空梅花眼。牛行摇动,种子即从眼中撒下。欲密而多,则鞭牛疾走,子撒必多;欲稀而少,则缓其牛,撒种即少。既播种后,用驴驾两小石团,压土埋麦。凡麦种紧压方生。南地不与北同者,多耕多耙之后,然后以灰拌种,手指拈而种之。种过之后,随以脚根压土使紧,以代北方驴石也。

耕种之后,勤议耨锄。凡耨草用阔面大,麦苗生后,耨不厌勤,(有三过四过者。)余草生机尽诛锄下,则竟亩精华尽聚嘉实矣。功勤易耨,南与北同也。凡粪麦田,既种以后,粪无可施,为计在先也。陕、洛之间忧虫蚀者,或以砒霜拌种子,南方所用惟炊烬也。(俗名地灰。)南方稻田有种肥田麦者,不粪麦实。当春小麦、大麦青青之时,耕杀田中,蒸罨土性,秋收稻谷必加倍也。

凡麦收空隙,可再种他物。自初夏至季秋,时日亦半载,择土宜而为之,惟人所取也。南方大麦有既刈之后乃种迟生粳稻者。勤农作苦,明赐无不及也。凡荞麦,南方必刈稻,北方必刈菽、稷而后种。其性稍吸肥腴,能使土瘦。然计其获入,业偿半谷有余,勤农之家何妨再粪也。

\textbf{○麦灾}

凡麦防患抵稻三分之一。播种以后,雪、霜、晴、潦皆非所计。麦性食水甚少,北土中春再沐雨水一升,则秀华成嘉粒矣。荆、扬以南唯患霉雨。倘成熟之时晴干旬日,则仓禀皆盈,不可胜食。扬州谚云``寸麦不怕尺水'',谓麦初长时,任水灭顶无伤;``尺麦只怕寸水'',谓成熟时寸水软根,倒茎沾泥,则麦粒尽烂于地面也。

江南有雀一种,有肉无骨,飞食麦田数盈千万,然不广及,罹害者数十里而止。江北蝗生,则大之岁也。

\textbf{○黍稷 粱粟}

凡粮食,米而不粉者种类甚多。相去数百里,则色、味、形、质随方而变,大同小异,千百其名。北人唯以大米呼粳稻,而其余概以小米名之。

凡黍与稷同类,粱与粟同类。黍有粘有不粘,(粘者为酒。)稷有粳无粘。凡粘黍、粘粟统名曰秫,非二种外更有秫也。黍色赤、白、黄、黑皆有,而或专以黑色为稷,未是。至以稷米为先他谷熟,堪供祭祀则当以早熟者为稷,则近之矣。凡黍在《诗》、《书》有虋、芑、、丕等名,在今方语有牛毛、燕颔、马革、驴皮、稻尾等名。种以三月为上时,五月熟;四月为中时,七月熟;五月为下时,八月熟。扬花结穗总与来、牟不相见也。凡黍粒大小,总视土地肥硗、时令害育。宋儒拘定以某方黍定律,未是也。

凡粟与粱统名黄米。粘粟可为酒,而芦粟一种名曰高粱者,以其身高七尺如芦、荻也。粱粟种类名号之多,视黍稷犹甚,其命名或因姓氏、山水,或以形似、时令,总之不可枚举。山东人唯以谷子呼之,并不知粱粟之名也。已上四米皆春种秋获,耕耨之法与来、牟同,而种收之候则相悬绝云。

\textbf{○麻}

凡麻可粒可油者,惟火麻、胡麻二种。胡麻即脂麻,相传西汉始自大宛来。古者以麻为五谷之一,若专以火麻当之,义岂有当哉?窈意《诗》、《书》五谷之麻,或其种已灭,或即菽、粟之中别种,而渐讹其名号,皆未可知也。

今胡麻味美而功高,即以冠百谷不为过。火麻子粒压油无多,皮为疏恶布,其值几何?胡麻数龠充肠,移时不馁。Х饵、饴饧得粘其粒,味高而品贵。其为油也,发得之而泽,腹得之而膏,腥膻得之而芳,毒厉得之而解。农家能广种,厚实可胜言哉。

种胡麻法,或治畦圃,或垄田亩。土碎草净之极,然后以地灰微湿,拌匀麻子而撒种之。早者三月种,迟者不出大暑前。早种者花实亦待中秋乃结。耨草之功唯锄是视。其色有黑、白、赤三者。其结角长寸许有四棱者,房小而子少,八棱者房大而子多。皆因肥瘠所致,非种性也。收子榨油每石得四十斤余,其枯用以肥田。若饥荒之年,则留供人食。

\textbf{○菽}

凡菽种类之多,与稻、黍相等,播种收获之期,四季相承。果腹之功在人日用,盖与饮食相终始。

一种大豆,有黑、黄两色,下种不出清明前后。黄者有五月黄、六月爆、冬黄三种。五月黄收粒少,而冬黄必倍之。黑者刻期八月收。淮北长征骡马必食黑豆,筋力乃强。

凡大豆视土地肥硗、耨草勤怠、雨露足悭,分收入多少。凡为豉、为酱、为腐,皆于大豆中取质焉。江南又有高脚黄,六月刈早稻方再种,九十月收获。江西吉郡种法甚妙:其刈稻田竟不耕垦,每禾稿头中拈豆三四粒,以指扌及之,其稿凝露水以滋豆,豆性充发,复浸烂稿根以滋。已生苗之后,遇无雨亢干,则汲水一升以灌之。一灌之后,再耨之余,收获甚多。凡大豆入土未出芽时,防鸠雀害,驱之惟人。

一种绿豆,圆小如珠。绿豆必小暑方种,未及小暑而种,则其苗蔓延数尺,结荚甚稀。若过期至于处暑,则随时开花结荚,颗粒亦少。豆种亦有二,一曰摘绿,荚先老者先摘,人逐日而取之。一曰拔绿,则至期老足,竟亩拔取也。凡绿豆磨澄晒干为粉,荡片搓索,食家珍贵。做粉溲浆灌田甚肥。凡畜藏绿豆种子,或用地灰、石灰、马蓼,或用黄土拌收,则四五月间不愁空蛀。勤者逢晴频晒,亦免蛀。凡已刈稻田,夏秋种绿豆,必长接斧柄,击碎土块,发生乃多。

凡种绿豆,一日之内遇大雨扳土则不复生。既生之后,防雨水浸,疏沟浍以泄之。凡耕绿豆及大豆田地,耒耜欲浅,不宜深入。盖豆质根短而苗直,耕土既深,土块曲压,则不生者半矣。``深耕''二字不可施之菽类。此先农之所未发者。

一种豌豆,此豆有黑斑点,形圆同绿豆,而大则过之。其种十月下,来年五月收。凡树木叶迟者,其下亦可种。

一种蚕豆,其荚似蚕形,豆粒大于大豆。八月下种,来年四月收。西浙桑树之下遍环种之。盖凡物树叶遮露则不生,此豆与豌豆,树叶茂时彼已结荚而成实矣。襄、汉上流,此豆甚多而贱,果腹之功不啻黍稷也。

一种小豆,赤小豆入药有奇功,白小豆(一名饭豆)当冫食助嘉谷。夏至下种,九月收获,种盛江淮之间。

一种(音吕),此豆古者野生田间,今则北土盛种。成粉荡皮可敌绿豆。燕京负贩者,终朝呼豆皮,则其产必多矣。

一种白藊豆,乃沿篱蔓生者,一名蛾眉豆。

其他豇豆、虎斑豆、刀豆,与大豆中分青皮、褐色之类,间繁一方者,犹不能尽述。皆充蔬代谷以粒民者,博物者其可忽诸!

\hypertarget{header-n2250}{%
\subsubsection{乃服}\label{header-n2250}}

宋子曰:人为万物之灵,五官百体,赅而存焉。贵者垂衣裳,煌煌山龙,以治天下。贱者衤豆褐、裳,冬以御寒,夏以蔽体,以自别于禽兽。是故其质则造物之所具也。属草木者为、麻、苘、葛,属禽兽与昆虫者裘褐、丝绵。各载其半,而裳服充焉矣。

天孙机杼,传巧人间。从本质而见花,因绣濯而得锦。乃杼柚遍天下,而得见花机之巧者,能几人哉?``治乱''、``经纶''字义,学者童而习之,而终身不见其形象,岂非缺憾也!先列饲蚕之法,以知丝源之所自。盖人物相丽,贵贱有章,天实为之矣。

\textbf{○蚕种}

凡蛹变蚕蛾,旬日破茧而出,雌雄均等。雌者伏而不动,雄者两翅飞扑,遇雌即交,交一日、半日方解。解脱之后,雄者中枯而死,雌者即时生卵。承藉卵生者,或纸或布,随方所用。(嘉、湖用桑皮厚纸,来年尚可再用。)一蛾计生卵二百余粒,自然粘于纸上,粒粒匀铺,天然无一堆积。蚕主收贮,以待来年。

\textbf{○蚕浴}

凡蚕用浴法,唯嘉、湖两郡。湖多用天露、石灰,嘉多用盐卤水。每蚕纸一张,用盐仓走出卤水二升,参水浸于盂内,纸浮其面(石灰仿此)。逢腊月十二即浸浴,至二十四,计十二日,周即漉起,用微火烘干。从此珍重箱匣中,半点风湿不受,直待清明抱产。其天露浴者,时日相同。以篾盘盛纸,摊开屋上,四隅小石镇压,任从霜雨、风雨、雷电,满十二日方收。珍重待时如前法。盖低种经浴,则自死不出,不费叶故,且得丝亦多也。晚种不用浴。

\textbf{○种忌}

凡蚕纸用竹木四条为方架,高悬透风避日梁枋之上,其下忌桐油、烟煤火气。冬月忌雪映,一映即空。遇大雪下时,即忙收贮,明曰雪过,依然悬挂,直待腊月浴藏。

\textbf{○种类}

凡蚕有早、晚二种。晚种每年先早种五六日出,(川中者不同。)结茧亦在先,其茧较轻三分之一。若早蚕结茧时,彼已出蛾生卵,以便再养矣。(晚蛹戒不宜食。)凡三种浴种,皆谨视原记。如一错误,或将天露者投盐浴,则尽空不出矣。凡茧色唯黄、白二种。川、陕、晋、豫有黄无白,嘉、湖有白无黄。若将白雄配黄雌,则其嗣变成褐茧。黄丝以猪胰漂洗,亦成白色,但终不可染漂白、桃红二色。

凡茧形亦有数种。晚茧结成亚腰葫卢样,天露茧尖长如榧子形,又或圆扁如核桃形。又一种不忌泥涂叶者,名为贱蚕,得丝偏多。

凡蚕形亦有纯白、虎斑、纯黑、花纹数种,吐丝则同。今寒家有将早雄配晚雌者,幻出嘉种,一异也。野蚕自为茧,出青州、沂水等地,树老即自生。其丝为衣,能御雨及垢污。其蛾出即能飞,不传种纸上。他处亦有,但稀少耳。

\textbf{○抱养}

凡清明逝三日,蚕少即不偎衣衾暖气,自然生出。蚕室宜向东南,周围用纸糊风隙,上无棚板者宜顶格,值寒冷则用炭火于室内助暖。凡初乳蚕,将桑叶切为细条。切叶不束稻麦镐为之,则不损刀。摘叶用瓮坛盛,不欲风吹枯悴。

二眠以前,腾筐方法皆用尖圆小竹筷提过。二眠以后则不用箸,而手指可拈矣。凡腾筐勤苦,皆视人工。怠于腾者,厚叶与粪湿蒸,多致压死。凡眠齐时,皆吐丝而后眠。若腾过,须将旧叶些微拣净。若粘带丝缠叶在中,眠起之时,恐其即食一口,则其病为胀死。三眠已过,若天气炎热,急宜搬出宽凉所,亦忌风吹。凡大眠后,计上叶十二冫食方腾,太勤则丝糙。

\textbf{○养忌}

凡蚕畏香,复畏臭。若焚骨灰、淘毛圊者,顺风吹来,多致触死。隔壁煎鲍鱼、宿脂,亦或触死。灶烧煤炭,炉沉、檀,亦触死。懒妇便器摇动气侵,亦有损伤。若风则偏忌西南,西南风太劲,则有合箔皆僵者。凡臭气触来,急烧残桑叶烟以抵之。

\textbf{○叶料}

凡桑叶无土不生。嘉、湖用枝条垂压,今年视桑树傍生条,用竹钩挂卧,逐渐近地面,至冬月则抛土压之,来春每节生根,则剪开他栽。其树精华皆聚叶上,不复生葚与开花矣。欲叶便剪摘,则树至七八尺即斩截当顶,叶则婆娑可扳伐,不必乘梯缘木也。其他用子种者,立夏桑葚紫熟时取来,用黄泥水搓洗,并水浇于地面,本秋即长尺余。来春移栽,倘灌粪勤劳,亦易长茂。但间有生葚与开花者,则叶最薄少耳。又有花桑叶薄不堪用者,其树接过,亦生厚叶也。

又有柘叶三种以济桑叶之穷。柘叶浙中不经见,川中最多。寒家用浙种桑叶穷时,仍啖柘叶,则物理一也。凡琴弦、弓弦丝,用柘养蚕,名曰棘茧,谓最坚韧。

凡取叶必用剪,铁剪出嘉郡桐乡者最犀利,他乡未得其利。剪枝之法,再生条次月叶愈茂,取资既多,人工复便。凡再生条叶,仲夏以养晚蚕,则止摘叶而不剪条。二叶摘后,秋来三叶复茂,浙人听其经霜自落,片片扫拾以饲绵羊,大获绒毡之利。

\textbf{○食忌}

凡蚕大眠以后,径食湿叶。雨天摘来者,任从铺地加冫食;晴日摘来者,以水洒湿而饲之,则丝有光泽。未大眠时,雨天摘叶用绳悬挂透风檐下,时振其绳,待风吹干。若用手掌拍干,则叶焦而不滋润,他时丝亦枯色。凡食叶,眠前必令饱足而眠,眠起即迟半日上叶无妨也。雾天湿叶甚坏蚕,其晨有雾,切勿摘叶。待雾收时,或晴或雨,方剪伐也。露珠水亦待于干而后剪摘。

\textbf{○病症}

凡蚕卵中受病,已详前款。出后湿热积压,妨忌在人。初眠腾时,用漆合者不可盖掩逼出气水。凡蚕将病,则脑上放光,通身黄色,头渐大而尾渐小;并及眠之时,游走不眠,食叶又不多者,皆病作也。急择而去之,勿使败群。凡蚕强美者必眠叶面,压在下者或力弱或性懒,作茧亦薄。其作茧不知收法,妄吐丝成阔窝者,乃蠢蚕,非懒蚕也。

\textbf{○老足}

凡蚕食叶足候,只争时刻。自卵出少多在辰巳二时,故老足结茧亦多辰巳二时。老足者,喉下两夹通明,捉时嫩一分则丝少。过老一分,又吐去丝,茧壳必薄。捉者眼法高,一只不差方妙。黑色蚕不见身中透光,最难捉。

\textbf{○结茧 (山箔 具图)}

凡结茧必如嘉、湖,方尽其法。他国不知用火烘,听蚕结出,甚至丛杆之内,箱匣之中,火不经,风不透。故所为屯、漳等绢,豫、蜀等绸,皆易朽烂。若嘉、湖产丝成衣,即入水浣濯百余度,其质尚存。其法析竹编箔,其下横架料木约六尺高,地下摆列炭火(炭忌爆炸),方圆去四五尺即列火一盆。初上山时,火分两略轻少,引他成绪,蚕恋火意,即时造茧,不复缘走。

茧绪既成,即每盆加火半斤,吐出丝来随即干燥,所以经久不坏也。其茧室不宜楼板遮盖,下欲火而上欲风凉也,凡火顶上者不以为种,取种宁用火偏者。其箔上山用麦稻稿斩齐,随手纠捩成山,顿插箔上。做山之人最宜手健。箔竹稀疏用短稿略铺洒,妨蚕跌坠地下与火中也。

\textbf{○取茧}

凡茧造三日,则下箔而取之。其壳外浮丝一名丝匡者,湖郡老妇贱价买去,(每斤百文。)用铜钱坠打成线,织成湖绸。去浮之后,其茧必用大盘摊开架上,以听治丝、扩绵。若用厨箱掩盖,则郁而丝绪断绝矣。

\textbf{○物害}

凡害蚕者,有雀、鼠、蚊三种。雀害不及茧,蚊害不及早蚕丝,鼠害则与之相终始。防驱之智是不一法,唯人所行也。(雀屎粘叶,蚕食之立刻死烂。)

\textbf{○择茧}

凡取丝必用圆正独蚕茧,则绪不乱。若双茧并四五蚕共为茧,择去取绵用。或以为丝则粗甚。

\textbf{○造绵}

凡双茧并缫丝锅底零余,并出种茧壳,皆绪断乱不可为丝,用以取绵。用稻灰水煮过,(不宜石灰。)倾入清水盆内。手大指去甲净尽,指头顶开四个,四四数足,用拳顶开又四四十六拳数,然后上小竹弓。此《庄子》所谓纟光也。

湖绵独白净清化者,总缘手法之妙。上弓之时惟取快捷,带水扩开。若稍缓水流去,则结块不尽解,而色不纯白矣。其治丝余者名锅底绵,装绵衣衾内以御重寒,谓之挟纩。凡取绵人工,难于取丝八倍,竟日只得四两余。用此绵坠打线织湖绸者,价颇重。以绵线登花机者名曰花绵,价尤重。

\textbf{○治丝 ( 缫车 具图)}

凡治丝先制丝车,其尺寸器具开载后图。锅煎极沸汤,丝粗细视投茧多寡,穷日之力一人可取三十两。若包头丝,则只取二十两,以其苗长也。凡绫罗丝,一起投茧二十枚,包头丝只投十余枚。凡茧滚沸时,以竹签拨动水面,丝绪自见。提绪入手,引入竹针眼,先绕星丁头,(以竹棍做成,如香筒样。)然后由送丝竿勾挂,以登大关车。断绝之时,寻绪丢上,不必绕接。其丝排匀不堆积者,全在送丝竿与磨木之上。川蜀丝车制稍异,其法架横锅上,引四五绪而上,两人对寻锅中绪,然终不若湖制之尽善也。

凡供治丝薪,取极燥无烟湿者,则宝色不损。丝美之法有六字:一曰``出口干'',即结茧时用炭火烘。一曰``出水干'',则治丝登车时,用炭火四五两盆盛,去车关五寸许。运转如风转时,转转火意照干,是曰出水干也。(若晴光又风色,则不用火。)

\textbf{○调丝}

凡丝议织时,最先用调。透光檐端宇下以木架铺地,植竹四根于上,名曰络笃。丝匡竹上,其傍倚柱高八尺处,钉具斜安小竹偃月挂钩,悬搭丝于钩内,手中执\{矍\}旋缠,以俟牵经织纬之用。小竹坠石为活头,接断之时,扳之即下。

\textbf{○纬络 (纺车 具图)}

凡丝既\{矍\}之后,以就经纬。经质用少而纬质用多,每丝十两,经四纬六,此大略也。凡供纬\{矍\},以水沃湿丝,摇车转铤而纺于竹管之上。(竹用小箭竹。)

\textbf{○经具 (溜眼 掌扇 经耙 印架 皆具图)}

凡丝既\{矍\}之后,牵经就织。以直竹竿穿眼三十余,透过篾圈,名曰溜眼。竿横架柱上,丝从圈透过掌扇,然后缠绕经耙之上。度数既足,将印架捆卷。既捆,中以交竹二度,一上一下间丝,然后扌及于筘内。(此筘非织筘。)扌及筘之后,然的杠与印架相望,登开五七丈。或过糊者,就此过糊。或不过糊,就此卷于的杠,穿综就织。

\textbf{○过糊}

凡糊用面\textless{}角力\textgreater{}内小粉为质。纱罗所必用,绫绸或用或不用。其染纱不存素质者,用牛胶水为之,名曰清胶纱。糊浆承于筘上,推移染透,推移就干。天气晴明,顷刻而燥,阴天必藉风力之吹也。

\textbf{○边维}

凡帛不论绫罗,皆别牵边,两傍各二十余缕。边缕必过糊,用筘推移梳干。凡绫罗必三十丈、五立十丈一穿,以省穿接繁苦。每匹应截画墨于边丝之上,即知其丈尺之足。边丝不登的杠,别绕机梁之上。

\textbf{○经数}

凡织帛,罗纱筘以八百齿为率。绫绢筘以一千二百齿为率。每筘齿中度经过糊者,四缕合为二缕,罗纱经计三千二百缕,绫绸经计五千六千缕。古书八十缕为一升,今绫绢厚者,古所谓六十升布也。凡织花文必用嘉、湖出口、出水皆干丝为经,则任从提挈,不忧断接。他省者即勉强提花,潦草而已。

\textbf{○花机式 (具全图)}

凡花机通身度长一丈六尺,隆起花楼,中托衢盘,下垂衢脚。(水磨竹棍为之,计一千八百根。)对花楼下掘坑二尺许,以藏衢脚。(地气湿者,架棚二尺代之。)提花小厮坐立花楼架木上。机末以的杠卷丝,中间叠助木两枝,直穿二木,约四尺长,其尖插于筘两头。

叠助,织纱罗者,视织绫绢者减轻十余斤方妙。其素罗不起花纹,与软纱绫绢踏成浪梅小花者,视素罗只加桄二扇。一人踏织自成,不用提花之人,闲住花楼,亦不设衢盘与衢脚也。其机式两接,前一接平安,自花楼向身一接斜倚低下尺许,则叠助力雄。若织包头细软,则另为均平不斜之机。坐处斗二脚,以其丝微细,防遏叠助之力也。

\textbf{○腰机式 (具图)}

凡织杭西、罗地等绢,轻素等绸,银条、巾帽等纱,不必用花机,只用小机。织匠以熟皮一方置坐下,其力全在腰尻之上,故名腰机。普天织葛、苎、棉布者,用此机法,布帛更整齐坚泽,惜今传之犹未广也。

\textbf{○结花本}

凡工匠结花本者,心计最精巧。画师先画何等花色于纸上,结本者以丝线随画量度,算计分寸杪忽而结成之。张悬花楼之上,即织者不知成何花色,穿综带经,随甚尺寸度数提起衢脚,梭过之后居然花现。盖绫绢以浮轻而见花,纱罗以纠纬而见花。绫绢一梭一提,纱罗来梭提,往梭不提。天孙机杼,人巧备矣。

\textbf{○穿经}

凡丝穿综度经,必用四人列坐。过筘之人,手执筘耙先插以待丝至。丝过筘则两指执定,足五七十筘,则绦结之。不乱之妙,消息全在交竹。即接断,就丝一扯即长数寸。打结之后,依还原度,此丝本质自具之妙也。

\textbf{○分名}

凡罗,中空小路以透风凉,其消息全在软综之中。衮头两扇打综,一软一硬。凡五梭三梭(最厚者七梭)之后,踏起软综,自然纠转诸经,空路不粘。若平过不空路而仍稀者曰纱,消息亦在两扇衮头之上。直至织花绫绸,则去此两扇,而用桄综八扇。

凡左右手各用一梭交互织者,曰绉纱。凡单经曰罗地,双经曰绢地,五经曰绫地。凡花分实地与绫地,绫地者光,实地者暗。先染丝而后织者曰缎。(北土土屯绢,亦先染丝。)就丝绸机上织时,两梭轻,一梭重,空出稀路者,名曰秋罗,此法亦起近代。凡吴越秋罗,闽广怀素,皆利绅当暑服,屯绢则为外官、卑官逊别锦绣用也。

\textbf{○熟练}

凡帛织就犹是生丝,煮练方熟。练用稻稿灰入水煮。以猪胰脂陈宿一晚,入汤浣之,宝色烨然。或用乌梅者,宝色略减。凡早丝为轻、晚丝为纬者,练熟之时每十两轻去三两。经纬皆美好早丝,轻化只二两。练后日干张急,以大蚌壳磨使乖钝,通身极力刮过,以成宝色。

\textbf{○龙袍}

凡上供龙袍,我朝局在苏、杭。其花楼高一丈五尺,能手两人扳提花本,织来数寸即换龙形。各房斗合,不出一手。赭黄亦先染丝,工器原无殊异,但人工慎重与资本皆数十倍,以效忠敬之谊。其中节目微细,不可得而详考云。

\textbf{○倭缎}

凡倭缎制起东夷,漳、泉海滨效法为之。丝质来自川蜀,商人万里贩来,以易胡椒归里。其织法亦自夷国传来。盖质已先染,而斫绵夹藏经面,织过数寸即刮成黑光。北虏互市者见而悦之。但其帛最易朽污,冠弁之上顷刻集灰,衣领之间移日损坏。今华夷皆贱之,将来为弃物,织法可不传云。

\textbf{○布衣 (赶 弹 纺 具图)}

凡棉布御寒,贵贱同之。棉花古书名麻,种遍天下。种有木棉、草棉两者,花有白、紫二色。种者白居十九,紫居十一。凡棉春种秋花,花先绽者逐日摘取,取不一时。其花粘子于腹,登赶车而分之。去子取花,悬弓弹化。(为挟纩温衾袄者,就此止功。)弹后以木板擦成长条以登纺车,引绪纠成纱缕。然后绕\{矍\}牵经就织。凡纺工能者一手握三管纺于铤上。(捷则不坚。)

凡棉布寸土皆有,而织造尚松江,浆染尚芜湖。凡布缕紧则坚,缓则脆。碾石取江北性冷质腻者,(每块佳者值十余金。)石不发烧,则缕紧不松泛。芜湖巨店首尚佳石。广南为布薮而偏取远产,必有所试矣。为衣敝浣,犹尚寒砧捣声,其义亦犹是也。

外国朝鲜造法相同,惟西洋则未核其质,并不得其机织之妙。凡织布有云花、斜文、象眼等,皆仿花机而生义。然既曰布衣,太素足矣。织机十室必有,不必具图。

\textbf{○著}

凡衣衾挟纩御寒,百有之中止一人用茧绵,余皆著。古袍今俗名胖袄。棉花既弹化,相衣衾格式而入装之。新装者附体轻暖,经年板紧,暖气渐无,取出弹化而重装之,其暖如故。

\textbf{○夏服}

凡苎麻无土不生。其种植有撒子、分头两法。(池郡每岁以草粪压头,其根随土而高。广南青麻撒子种田茂甚。)色有青、黄两样。每岁有两刈者,有三刈者,绩为当暑衣裳、帷帐。

凡苎皮剥取后,喜日燥干,见水即烂。破析时则以水浸之,然只耐二十刻,久而不析则亦烂。苎质本淡黄,漂工化成至白色。(先用稻灰、石灰水煮过,入长流水再漂,再晒,以成至白。)纺苎纱能者用脚车,一女工并敌三工,惟破析时穷日之力只担三五铢重。织苎机具与织棉者同。凡布衣缝线,革履串绳,其质必用苎纠合。

凡葛蔓生,质长于苎数尺。破析至细者,成布贵重。又有м麻一种,成布甚粗,最粗者以充丧服。即苎布有极粗者,漆家以盛布灰,大内以充火炬。又有蕉纱,乃闽中取芭蕉皮析缉为之,轻细之甚,值贱而质枵,不可为衣也。

\textbf{○裘}

凡取兽皮制服统名曰裘。贵至貂、狐,贱至羊、麂,值分百等。貂产辽东外徼建州地及朝鲜国。其鼠好食松子,夷人夜伺树下,屏息悄声而射取之。一貂之皮方不盈尺,积六十余貂仅成一裘。服貂裘者立风雪中,更暖于宇下。眯入目中,拭之即出,所以贵也。色有三种,一白者曰银貂,一纯黑,一黯黄。(黑而毛长者,近值一帽套已五十金。)凡狐、貂亦产燕、齐、辽、汴诸道。纯白狐腋裘价与貂相仿,黄褐狐裘值貂五分之一,御寒温体功用次于貂。凡关外狐取毛见底青黑,中国者吹开见白色以此分优劣。

羊皮裘母贱子贵。在腹者名曰胞羔,(毛文略具。)初生者名曰乳羔,(皮上毛似耳环脚。)三月者曰跑羔,七月者曰走羔,(毛文渐直。)胞羔、乳羔为裘不膻。古者羔裘为大夫之服,今西北绅亦贵重之。其老大羊皮硝熟为裘,裘质痴重,则贱者之服耳,然此皆绵羊所为。若南方短毛革,硝其享如纸薄,止供画灯之用而已。服羊裘者,腥膻之气习久而俱化,南方不习者不堪也。然寒凉渐杀,亦无所用之。

麂皮去毛,硝熟为袄裤御风便体,袜靴更佳。此物广南繁生外,中土则积集聚楚中,望华山为市皮之所。麂皮且御蝎患,北人制衣而外,割条以缘衾边,则蝎自远去。虎豹至文,将军用以彰身;犬豕至贱,役夫用以适足。西戎尚獭皮,以为毳衣领饰。襄黄之人穷山越国射取而远货,得重价焉。殊方异物如金丝猿,上用为帽套;扯里狲御服以为袍,皆非中华物也。兽皮衣人此其大略,方物则不可殚述。飞禽之中有取鹰腹、雁胁毳毛,杀生盈万乃得一裘,名天鹅绒者,将焉用之?

\textbf{○褐毡}

凡绵羊有二种,一曰蓑衣羊,剪其毳为毡、为绒片,帽袜遍天下,胥此出焉。古者西域羊未入中国,作褐为贱者服,亦以其毛为之。褐有粗而无精,今日粗褐亦间出此羊之身。此种自徐、淮以北州郡无不繁生。南方唯湖郡饲畜绵羊,一岁三剪毛。(夏季稀革不生。)每羊一只,岁得绒袜料三双。生羔牝牡合数得二羔,故北方家畜绵羊百只,则岁入计百金云。

一种\{矛ぁ\}\{艹刁\}羊(番语,)唐末始自西域传来,外毛不甚蓑长,内毳细软,取织绒褐,秦人名曰山羊,以别于绵羊。此种先自西域传入临洮,今兰州独盛,故褐之细者皆出兰州。一曰兰绒,番语谓之孤古绒,从其初号也。山羊毳绒亦分两等,一曰ㄐ绒,用梳栉ㄐ下,打线织帛,曰褐子、把子诸名色。一曰拔绒,乃毳毛精细者,以两指甲逐茎ㄎ下,打线织绒褐。此褐织成,揩面如丝帛滑腻。每人穷日之力打线只得一钱重,费半载工夫方成匹帛之料。若ㄐ绒打线,日多拔绒数倍。凡打褐绒线,冶铅为锤,坠于绪端,两手宛转搓成。

凡织绒褐机大于布机,用综八扇,穿经度缕,下施四踏轮,踏起经隔二抛纬,故织出文成斜现。其梭长一尺二寸,机织、羊种皆彼时归夷传来,(名姓再详。)故至今织工皆其族类,中国无典也。凡绵羊剪毳,粗者为毡,细者为绒。毡皆煎烧沸汤投于其中搓洗,俟其粘合,以木板定物式,铺绒其上,运轴赶成。凡毡绒白黑为本色,其余皆染色。其氍俞、氆鲁等名称,皆华夷各方语所命。若最粗而为毯者,则驽马诸料杂错而成,非专取料于羊也。

\hypertarget{header-n2344}{%
\subsubsection{彰施}\label{header-n2344}}

宋子曰:霄汉之间云霞异色,阎浮之内花叶殊形。天垂象而圣人则之,以五彩彰施于五色,有虞氏岂无所用其心哉?飞禽众而凤则丹,走兽盈而麟则碧,夫林林青衣,望阙而拜黄朱也,其义亦犹是矣。君子曰:``甘受和,白受采。''世间丝、麻、裘、褐皆具素质,而使殊颜异色得以尚焉,谓造物不劳心者,吾不信也。

\textbf{○诸色质料}

大红色、(其质红花饼一味,用乌梅水煎出。又用碱水澄数次,或稻稿灰代碱,功用亦同。澄得多次,色则鲜甚。染房讨便宜者,先染芦木打脚。凡红花最忌沉、麝,袍服与衣香共收,旬月之间其色即毁。凡红花染帛之后,若欲退转,但浸湿所染帛,以碱水、稻灰水滴上数十点,其红一毫收转,仍还原质。所收之水藏于绿豆粉内,放出染红,半滴不耗。染家以为秘诀,不以告人。)莲红、桃红色、银红、水红色、(以上质亦红花饼一味,浅深分两加减而成。是四色皆非黄茧丝所可为,必用白丝方现。)木红色、(用苏木煎水,入明矾、子。)紫色、(苏木为地,青矾尚之。)赭黄色、(制未详。)鹅黄色、(黄蘖煎水染,靛水盖上。)金黄色、(芦木煎水染,复用麻稿灰淋,碱水漂。)茶褐色、(莲子壳煎水染,复用青矾水盖。)大红官绿色、(槐花煎水染,蓝淀盖,浅深皆用明矾。)豆绿色、(黄蘖水染,靛水盖。今用小叶苋蓝煎水盖者,名草豆绿,色甚鲜。)油绿色、(槐花薄染,青矾盖。)天青色、(入靛缸浅染,苏木水盖。)蒲萄青色、(入靛缸深染,苏木水深盖。)蛋青色、(黄蘖水染,然后入靛缸。)翠蓝、天蓝(二色俱靛水分深浅。)玄色、(靛水染深青,芦木、杨梅皮等分煎水盖。又一法,将蓝芽叶水浸,然后下青矾、子同浸,令布帛易朽。)月白草色二色、(俱靛水微染,今法用苋蓝煎水,半生半熟染。)象牙色、(芦木煎水薄染,或用黄土。)藕褐色。(苏木水薄染,入莲子壳,青矾水薄盖。)

附:染包头青色。(此黑不出蓝靛,用栗壳或莲子壳煎煮一日,漉起,然后入铁砂、皂矾锅内,再煮一宵即成深黑色。)

附:染毛青布色法。(布青初尚芜湖千百年矣。以其浆碾成青光,边方外国皆贵重之。人情久则生厌。毛青乃出近代,其法取松江美布染成深青,不复浆碾,吹干,用胶水参豆浆水一过。先蓄好靛,名曰标缸。入内薄染即起,红焰之色隐然。此布一时重用。

\textbf{○蓝淀}

凡蓝五种,皆可为淀。茶蓝即菘蓝,插根活;蓼蓝、马蓝、吴蓝等皆撒子生。近又出蓼蓝小叶者,俗名苋蓝,种更佳。

凡种茶蓝法冬月割获,将叶片片削下,入窖造淀。其身斩去上下,近根留数寸。薰干,埋藏土内。春月烧净山土使极肥松,然后用锥锄,(其锄勾末向身长八寸许。)刺土打斜眼,插入于内,自然活根生叶。其余蓝皆收子撒种畦圃中。暮春生苗,六月采实,七月刈身造淀。

凡造淀,叶者茎多者入窖,少者入桶与缸。水浸七日,其汁自来。每水浆一石下石灰五升,搅冲数十下,淀信即结。水性定时,淀沉于底。近来出产,闽人种山皆茶蓝,其数倍于诸蓝。山中结箬篓,输入舟航。其掠出浮沫晒干者曰靛花。凡靛入缸必用稻灰水先和,每日手执竹棍搅动,不可计数,其最佳者曰标缸。

\textbf{○红花}

红花场圃撒子种,二月初下种,若太早种者,苗高尺许即生虫如黑蚁,食根立毙。凡种地肥者,苗高二三尺。每路打橛,缚绳横阑,以备狂风拗折。若瘦地尺五以下者,不必为之。

红花入夏即放绽,花下作求汇多,刺花出求上。采花者必侵晨带露摘取。若日高露旰,其花即已结闭成实,不可采矣。其朝阴雨无露,放花较少,旰摘无妨,以无日色故也,红花逐日放绽,经月乃尽。入药用者不必制饼。若入染家用者,必以法成饼然后用,则黄汁净尽,而真红乃现也。其子煎压出油,或以银箔贴扇面,用此油一刷,火上照干,立成金色。

\textbf{○造红花饼法}

带露摘红花,捣熟以水淘,布袋绞去黄汁。又捣以酸粟或米泔清。又淘,又绞袋去汁,以青蒿覆一宿,捏成薄饼,阴干收贮。染家得法,我朱孔扬,所谓猩红也,(染纸吉礼用,亦必紫矿,不然全无色。)

\textbf{○附:燕脂}

燕脂古造法以紫矿染绵者为上,红花汁及山榴花汁者次之。近济宁路但取染残红花滓为之,值甚贱。其滓干者名曰紫粉,丹青家或收用,染家则糟粕弃也。

\textbf{○槐花}

凡槐树十余年后方生花实。花初试未开者曰槐蕊,绿衣所需,犹红花之成红也。取者张度籅稠其下而承之。以水煮一沸,漉干捏成饼,入染家用。既放之。花色渐入黄,收用者以石灰少许晒拌而藏之。

\hypertarget{header-n2366}{%
\subsubsection{粹精}\label{header-n2366}}

宋子曰:天生五谷以育民,美在其中,有黄裳之意焉。稻以糠为甲,麦以\textless{}麦夫\textgreater{}为衣,粟、粱、黍、稷毛羽隐然。播精而择粹,其道宁终秘也。饮食而知味者,食不厌精。杵臼之利,万民以济,盖取诸《小过》。为此者岂非人貌而天者哉?

\textbf{○攻稻(击禾 轧禾 风车 水碓 石碾 臼 碓 筛 皆具图)}

凡稻刈获之后,离稿取粒。束稿于手而击取者半,聚稿于场而曳牛滚石以取者半。凡束手而击者,受击之物或用木桶,或用石板。收获之时雨多霁少,田稻交湿,不可登场者,以木桶就田击取。晴霁稻干,则用石板甚便也。

凡服牛曳石滚压场中,视人手击取者力省三倍。但作种之谷,恐磨去壳尖,减削生机。故南方多种之家,场禾多藉牛力,而来年作种者则宁向石板击取也。

凡稻最佳者九穰一秕,倘风雨不时,耘耔失节,则六穰四秕者容有之。凡去秕,南方尽用风车扇去;北方稻少,用扬法,即以扬麦、黍者扬稻,盖不若风车之便也。

凡稻去壳用砻,去膜用舂、用碾。然水碓主舂,则兼并砻功。燥干之谷入碾亦省砻也。凡砻有二种:一用木为之,截木尺许,(质多用松。)斫合成大磨形,两扇皆凿纵斜齿,下合植┺穿贯上合,空中受谷。木砻攻米二千余石,其身乃尽。凡木砻,谷不甚燥者入砻亦不碎,故入贡军国漕储千万,皆出此中也。

一土砻析竹匡围成圈,实洁净黄土于内,上下两面各嵌竹齿。上合ド空受谷,其量倍于木砻。谷稍滋湿者入其中即碎断。土砻攻米二百石,其身乃朽。凡木砻必用健夫,土砻即孱妇弱子可胜其任。庶民饔飧皆出此中也。

凡既砻,则风扇以去糠秕,倾入筛中团转。谷未剖破者浮出筛面,重复入砻。凡筛大者围五尺,小者半之。大者其中心偃隆而起,健夫利用。小者弦高二寸,其中平洼,妇子所需也。凡稻米既筛之后,入臼而舂,臼亦两种。八口以上之家堀地藏石臼其上,臼量大者容五斗,小者半之。横木穿插碓头,(碓嘴冶铁为之,用醋滓合上。)足踏其末而舂之。不及则粗,太过则粉,精粮从此出焉。晨炊无多者,断木为手杵,其臼或木或石以受舂也。既舂以后,皮膜成粉,名曰细糠,以供犬豕之豢。荒歉之岁,人亦可食也。细糠随风扇播扬分去,则膜尘净尽而粹精见矣。

凡水碓,山国之人居河滨者之所为也。攻稻之法省人力十倍,人乐为之。引水成功,即筒车灌田同一制度也。设臼多寡不一。值流水少而地窄者,或两三臼。流水洪而地室宽者,即并列十臼无忧也。

江南信郡水碓之法巧绝。盖水碓所愁者,埋臼之地卑则洪潦为患,高则承流不及。信郡造法即以一舟为地,橛桩维之。筑土舟中,陷臼于其上,中流微堰石梁,而碓已造成,不烦木壅坡之力也。又有一举而三用者,激水转轮头,一节转磨成面,二节运碓成米,三节引水灌于稻田,此心计无遗者之所为也。凡河滨水碓之国,有老死不见砻者,去糠去膜皆以臼相终始,惟风筛之法则无不同也。

凡砌石为之,承藉、转轮皆用石。牛犊、马驹惟人所使,盖一牛之力日可得五人。但入其中者,必极燥之谷,稍润则碎断也。

\textbf{○攻麦 ( 磨 罗 具图)}

凡小麦其质为面。盖精之至者,稻中再舂之米;粹之至者,麦中重罗之面也。

小麦收获时,束稿击取如击稻法。其去秕法北土用,盖风扇流传未遍率土也。凡不在宇下,必待风至而后为之。风不至,雨不收,皆不可为也。

凡小麦既之后,以水淘洗尘垢净尽,又复晒干,然后入磨。凡小麦有紫、黄二种,紫胜于黄。凡佳者每石得面一百二十斤,劣者损三分之一也。

凡磨大小无定形,大者用肥健力牛曳转,其牛曳磨时用桐壳掩眸,不然则眩晕。其腹系桶以盛遗,不然则秽也。次者用驴磨,斤两稍轻。又次小磨,则止用人推挨者。

凡力牛一日攻麦二石,驴半之。人则强者攻三斗,弱者半之。若水磨之法,其详已载《攻稻》``水碓''中,制度相同,其便利又三倍于牛犊也。

凡牛、马与水磨,皆悬袋磨上,上宽下窄。贮麦数斗于中,溜入磨眼。人力所挨则不必也。

凡磨石有两种,面品由石而分。江南少粹白上面者,以石怀沙滓,相磨发烧,则其\textless{}麦夫\textgreater{}并破,故黑参和面中,无从罗去也。江北石性冷腻,而产于池郡之九华山者美更甚。以此石制磨,石不发烧,其\textless{}麦夫\textgreater{}压至扁秕之极不破,则黑疵一毫不入,而面成至白也。凡江南磨二十日即断齿,江北者经半载方断。南磨破\textless{}麦夫\textgreater{}得面百斤,北磨只得八十斤,故上面之值增十之二,然面\textless{}角力\textgreater{}、小粉皆从彼磨出,则衡数已足,得值更多焉。

凡麦经磨之后,几番入罗,勤者不厌重复。罗匡之底用丝织罗地绢为之。湖丝所织者,罗面千石不损,若他方黄丝所为,经百石而已朽也。凡面既成后,寒天可经三月,春夏不出二十日则郁坏。为食适口,贵及时也。

凡大麦则就舂去膜,炊饭而食,为粉者十无一焉。荞麦则微加舂杵去衣,然后或舂或磨以成粉而后食之。盖此类之视小麦,精粗贵贱大径庭也。

\textbf{○攻黍稷粟粱麻菽 (小碾 枷 具图)}

凡攻治小米,扬得其实,舂得其精,磨得其粹。风扬、车扇而外,簸法生焉。其法篾织为圆盘,铺米其中,挤匀扬播。轻者居前,簸弃地下;重者在后,嘉实存焉。凡小米舂、磨、扬、播制器,已详《稻》、《麦》之中。唯小碾一制在《稻》、《麦》之外。北方攻小米者,家置石墩,中高边下,边沿不开槽。铺米墩上,妇子两人相向,接手而碾之。其碾石图长如牛赶石,而两头插木柄。米堕边时随手以小扫上。家有此具,杵臼竟悬也。

凡胡麻刈获,于烈日中晒干,束为小把,两手执把相击。麻粒绽落,承藉以簟席也。凡麻筛与米筛小者同形,而目密五倍。麻从目中落,叶残角屑皆浮筛上而弃之。

凡豆菽刈获,少者用枷,多而省力者仍铺场,烈日晒干,牛曳石赶而压落之。凡打豆枷,竹木竿为柄,其端锥圆眼,拴木一条长三尺许,铺豆于场,执柄而击之。

凡豆击之后,用风扇扬去荚叶,筛以继之,嘉实洒然入禀矣。是故舂磨不及麻,碾不及菽也。

\hypertarget{header-n2396}{%
\subsubsection{作咸}\label{header-n2396}}

宋子曰:天有五气,是生五味。润下作咸,王访箕子而首闻其义焉。口之于味也,辛酸甘苦经年绝一无恙。独食盐禁戒旬日,则缚鸡胜匹倦怠恹然。岂非``天一生水'',而此味为生人生气之源哉?四海之中,五服而外,为蔬为谷,皆有寂灭之乡,而斥卤则巧生以待。孰知其所已然。

\textbf{○盐产}

凡盐产最不一,海、池、井、土、崖、砂石,略分六种,而东夷树叶,西戎光明不与焉。赤县之内,海卤居十之八,而其二为井、池、土碱。或假人力,或由天造。总之,一经舟车穷窘,则造物应付出焉。

\textbf{○海水盐}

凡海水自具咸质,海滨地高者名潮墩,下者名草荡,地皆产盐。同一海卤传神,而取法则异。一法高堰地,潮波不没者,地可种盐。种户各有区画经界,不相侵越。度诘朝无雨,则今日广布稻麦稿灰及芦茅灰寸许于地上,压使平匀。明晨露气冲腾,则其下盐茅勃发,日中晴霁,灰、盐一并扫起淋煎。一法潮波浅被地,不用灰压。候潮一过,明日天晴,半日晒出盐霜,疾趋扫起煎炼。一法逼海潮深地,先掘深坑,横架竹木,上铺席苇,又铺沙于苇席上。俟潮灭顶冲过,卤气由沙渗下坑中,撤去沙、苇,以灯烛之,卤气冲灯即灭,取卤水煎炼。总之功在晴霁,若淫雨连旬,则谓之盐荒。又淮场地面有日晒自然生霜如马牙者,谓之大晒盐。不由煎炼,扫起即食。海水顺风飘来断草,勾取煎炼名蓬盐。

凡淋煎法,掘坑二个,一浅一深。浅者尺许,以竹木架芦席于上,将扫来盐料(不论有灰无灰,淋法皆同)铺于席上。四围隆起作一堤当形,中以海水灌淋,渗下浅坑中。深者深七八尺,受浅坑所淋之汁,然后入锅煎炼。

凡煎盐锅古谓之牢盆,亦有两种制度。其盆周阔数丈,径亦丈许。用铁者以铁打成叶片,铁钉栓合,其底平如盂,其四周高尺二寸,其合缝处一以卤汁结塞,永无隙漏。其下列灶燃薪,多者十二三眼,少者七八眼,共煎此盘。南海有编竹为者,将竹编成阔丈深尺,糊以蜃灰,附于釜背。火燃釜底,滚沸延及成盐。亦名盐盆,然不若铁叶镶成之便也。凡煎卤未即凝结,将皂角椎碎,和粟米糠二味,卤沸之时投入其中搅和,盐即顷刻结成。盖皂角结盐犹石膏之结腐也。

凡盐淮扬场者,质重而黑。其他质轻而白。以量较之。淮场者一升重十两,则广、浙、长芦者只重六七两。凡蓬草盐不可常期,或数年一至,或一月数至。凡盐见水即化,见风即卤,见火愈坚。凡收藏不必用仓廪,盐性畏风不畏湿,地下叠稿三寸,任从卑湿无伤。周遭以土砖泥隙,上盖茅草尺许,百年如故也。

\textbf{○池盐}

凡池盐,宇内有二,一出宁夏,供食边镇;一出山西解池,供晋、豫诸郡县。解池界安邑、猗氏、临晋之间,其池外有城堞,周遭禁御。池水深聚处,其色绿沉。土人种盐者池傍耕地为畦陇,引清水入所耕畦中,忌浊水,参入即淤淀盐脉。

凡引水种盐,春间即为之,久则水成赤色。待夏秋之交,南风大起,则一宵结成,名曰颗盐,即古志所谓大盐也。以海水煎者细碎,而此成粒颗,故得大名。其盐凝结之后,扫起即成食味。种盐之人。积扫一石交官,得钱数十文而已。其海丰、深州引海水入池晒成者,凝结之时扫食不加人力,与解盐同。但成盐时日,与不藉南风则大异也。

\textbf{○井盐}

凡滇、蜀两省远离海滨,舟车艰通,形势高上,其咸脉即韫藏地中。凡蜀中石山去河不远者,多可造井取盐。盐井周围不过数寸,其上口一小盂覆之有余,深必十丈以外乃得卤性,故造井功费甚难。

其器冶铁锥,如碓嘴形,其尖使极刚利,向石山舂凿成孔。其身破竹缠绳,夹悬此锥。每舂深入数尺,则又以竹接其身使引而长。初入丈许,或以足踏锥梢,如舂米形。太深则用手捧持顿下。所舂石成碎粉,随以长竹接引,悬铁盏挖之而上。大抵深者半载,浅者月余,乃得一井成就。

盖井中空阔,则卤乞游散,不克结盐故也。井及泉后,择美竹长丈者,凿净其中节,留底不去。其喉下安消息,吸水入筒,用长ㄌ系竹沉下,其中水满。井上悬桔槔、辘轳诸具,制盘驾牛。牛曳盘转,辘轳绞ㄌ,汲水而上。入于釜中煎炼,(只用中釜,不用牢盆。)顷刻结盐,色成至白。

西川有火井,事奇甚。其井居然冷水,绝无火气,但以长竹剖开去节合缝漆布,一头插入井底,其上曲接,以口紧对釜脐,注卤水釜中。只见火意烘烘,水即滚沸。启竹而视之,绝无半点焦炎意。未见火形而用火神,此世间大奇事也,凡川、滇盐井逃课掩盖至易,不可穷诘。

\textbf{○末盐}

凡地碱煎盐,除并州末盐外,长芦分司地土人,亦有刮削煎成者,带杂黑色,味不甚佳。

\textbf{○崖盐}

凡西省阶、凤等州邑,海井交穷。其岩穴自生盐,色如红土,恣人刮取,不假煎炼。

\hypertarget{header-n2420}{%
\subsubsection{甘嗜}\label{header-n2420}}

宋子曰:气至于芳,色至于\textless{}青色\textgreater{},味至于甘,人之大欲存焉。芳而烈,\textless{}青色\textgreater{}而艳,甘而甜,则造物有尤异之思矣。世间作甘之味什八产于草木,而飞虫竭力争衡,采取百花酿成佳味,使草木无全功。孰主张是,而颐养遍于天下哉?

\textbf{○蔗种}

凡甘蔗有二种,产繁闽、广间,他方合并得其什一而已。似竹而大者为果蔗,截断生啖,取汁适口,不可以造糖。似荻而小者为糖蔗,口啖即棘伤唇舌,人不敢食,白霜、红砂皆从此出。凡蔗古来中国不知造糖,唐大历间,西僧邹和尚游蜀中遂宁始传其法。今蜀中种盛,亦自西域渐来也。

凡种荻蔗,冬初霜将至将蔗斫伐,去杪与根,埋藏土内。(土忌洼聚水湿处。)雨水前五六日,天色晴明即开出,去外壳,斫断约五六寸,以两个节为率。密布地上,微以土掩之,头尾相枕,若鱼鳞然。两芽平放,不得一上一下,致芽向土难发。芽长一二寸,频以清粪水浇之,俟长六七寸,锄起分栽。

凡栽蔗必用夹沙土,河滨洲土为第一。试验土色,掘坑尺五许,将沙土入口尝味,味苦者不可栽蔗。凡洲土近深山上流河滨者,即土味甘,亦不可种。盖山气凝寒,则他日糖味亦焦苦。去山四五十里,平阳洲土择佳而为之。(黄泥脚地毫不可为。)

凡栽蔗治畦,行阔四尺,犁沟深四寸。蔗栽沟内,约七尺列三丛,掩土寸许,土太厚则芽发稀少也。芽发三四个或六七个时,渐渐下土,遇锄耨时加之。加土渐厚,则身长根深,庶免欹倒之患。凡锄耨不厌勤过,浇粪多少视土地肥硗。长至一二尺,则将胡麻或芸苔枯浸和水灌,灌肥欲施行内。高二三尺则用牛进行内耕之。半月一耕,用犁一次垦土断傍根,一次掩土培根,九月初培土护根,以防斫后霜雪。

\textbf{○蔗品}

凡荻蔗造糖,有凝冰、白霜、红砂三品。糖品之分,分于蔗浆之老嫩。凡蔗性至秋渐转红黑色,冬至以后由红转褐,以成至白。五岭以南无霜国土,蓄蔗不伐以取糖霜。若韶、雄以北十月霜侵,蔗质遇霜即杀,其身不能久待以成白色,故速伐以取红糖也。凡取红糖,穷十日之力而为之。十日以前其浆尚未满足,十日以后恐霜气逼侵,前功尽弃。故种蔗十亩之家,即制车釜一付以供急用。若广南无霜,迟早惟人也。

\textbf{○造糖 (具图)}

凡造糖车,制用横板二片,长五尺,厚五寸,阔二尺,两头凿眼安柱,上┺出少许,下┺出板二三尺,埋筑土内,使安稳不摇。上板中凿二眼,并列巨轴两根,(木用至坚重者。)轴木大七尺围方妙。两轴一长三尺,一长四尺五寸,其长者出┺安犁担。担用屈木,长一丈五尺,以便驾牛团转走。轴上凿齿分配雌雄,其合缝处须直而圆,圆而缝合。夹蔗于中,一轧而过,与棉花赶车同义。蔗过浆流,再拾其滓,向轴上鸭嘴扌及入,再轧,又三轧之,其汁尽矣,其滓为薪。其下板承轴,凿眼,只深一寸五分,使轴脚不穿透,以便板上受汁也。其轴脚嵌安铁锭于中,以便捩转。

凡汁浆流板有槽,枧汁入于缸内。每汁一石下石灰五合于中。凡取汁煎糖,并列三锅如``品''字,先将稠汁聚入一锅,然后逐加稀汁两锅之内。若火力少束薪,其糖即成顽糖,起沫不中用。

\textbf{○造白糖}

凡闽、广南方经冬老蔗,用车同前法。榨汁入缸,看水花为火色。其花煎至细嫩,如煮羹沸,以手捻试,粘手则信来矣。此时尚黄黑色,将桶盛贮,凝成黑沙。然后以瓦溜(教陶家烧造)置缸上。共溜上宽下尖,底有一小孔,将草塞住,倾桶中黑沙于内。待黑沙结定,然后去孔中塞草,用黄泥水淋下。其中黑滓入缸内,溜内尽成白霜。最上一层厚五寸许,洁白异常,名曰洋糖,(西洋糖绝白美,故名。)下者稍黄褐。

造冰糖者将洋糖煎化,蛋青澄去浮滓,候视火色。将新青竹破成篾片,寸斩撒入其中。经过一霄,即成天然冰块。造狮、象、人物等,质料精粗由人。凡白糖有五品,石山为上,团枝次之,瓮鉴次之,小颗又次,沙脚为下。

\textbf{○饴饧}

凡饴饧,稻、麦、黍、粟皆可为之。《洪范》云:``稼穑作甘。''及此乃穷其理。其法用稻麦之类浸湿,生芽暴干,然后煎炼调化而成。色以白者为上,赤色者名曰胶饴,一时宫中尚之,含于口内即溶化,形如琥珀。南方造饼饵者谓饴饧为小糖,盖对蔗浆而得名也。饴饧人巧千方以供甘旨,不可枚述。惟尚方用者名``一窝丝'',或流传后代不可知也。

\textbf{○蜂蜜}

凡酿蜜蜂普天皆有,唯蔗盛之乡则蜜蜂自然减少。蜂造之蜜出山岩土穴者十居其八,而人家招蜂造酿而割取者,十居其二也。凡蜜无定色,或青或白,或黄或褐,皆随方土花性而变。如菜花蜜、禾花蜜之类,百千其名不止也。

凡蜂不论于家于野,皆有蜂王。王之所居造一台如桃大,王之子世为王。王生而不采花,每日群蜂轮值,分班采花供王。王每日出游两度,(春夏造蜜时。)游则八蜂轮值以待。蜂王自至孔隙口,四蜂以头顶腹,四蜂傍翼飞翔而去,游数刻而返,翼顶如前。

畜家蜂者或悬桶檐端,或置箱牖下,皆锥圆孔眼数十,俟其进入。凡家人杀一蜂二蜂皆无恙,杀至三蜂则群起螫人,谓之蜂反。凡蝙蝠最喜食蜂,投隙入中,吞噬无限。杀一蝙蝠悬于蜂前,则不敢食,俗谓之枭令。凡家蓄蜂,东邻分而之西舍,必分王之子去而为君,去时如铺扇拥卫。乡人有撒洒糟香而招之者。

凡蜂酿蜜,造成蜜脾,其形鬣鬣然。咀嚼花心汁吐积而成。润以人小遗,则甘芳并至,所谓臭腐神奇也。凡割脾取蜜,蜂子多死其中。其底则为黄蜡。凡深山崖石上有经数载未割者,其蜜已经时自熟,土人以长竿刺取,蜜即流下。或未经年而攀缘可取者,割炼与家蜜同也。土穴所酿多出北方,南方卑湿,有崖蜜而无穴蜜。凡蜜脾一斤炼取十二两。西北半天下,盖与蔗浆分胜云。

\textbf{○附:造兽糖}

凡造兽糖者,每巨釜一口受糖五十斤。其下发火慢煎,火从一角烧灼,则糖头滚旋而起。若釜心发火,则尽尽沸溢于地。每釜用鸡子三个,去黄取清,入冷水五升化解。逐匙滴下用火糖头之上,则浮沤黑滓尽起水面,以笊篱捞去,其糖清白之甚。然后打入铜铫,下用自风慢火温之,看定火色然后入模。凡狮象糖模,两合如瓦为之,杓写糖入,随手覆转倾下。模冷糖烧,自有糖一膜靠模凝结,名曰享糖,华筵用之。

\hypertarget{header-n2446}{%
\subsection{中篇}\label{header-n2446}}

\hypertarget{header-n2448}{%
\subsubsection{陶埏}\label{header-n2448}}

宋子曰:水火既济而土合。万室之国,日勤千有而不足,民用亦繁矣哉。上栋下室以避风雨,而瓴建焉。王公设险以守其国,而城垣雉堞,寇来不可上矣。泥瓮坚而醴酒欲清,瓦登洁而醯醢以荐。商周之际俎豆以木为之,毋亦质重之思耶。后世方土效灵,人工表异,陶成雅器,有素肌玉骨之象焉。掩映几筵,文明可掬,岂终固哉?

\textbf{○瓦}

凡埏泥造瓦,掘地二尺余,择取无沙粘土而为之。百里之内必产合用土色,供人居室之用。凡民居瓦形皆四合分片,先以圆桶为模骨,外画四条界。调践熟泥,叠成高长方条。然后用铁线弦弓,线上空三分,以尺限定,向泥\{一个\}平戛一片,似揭纸而起,周包圆桶之上。待其稍干,脱模而出,自然裂为四片。凡瓦大小古无定式,大者纵横八九寸,小者缩十之三。室宇合沟中,则必需其最大者,名曰沟瓦,能承受淫雨不溢漏也。

凡坯既成,干燥之后,则堆积窑中燃薪举火,或一昼夜或二昼夜,视窑中多少为熄火久暂。浇水转氵幼(音右)与造砖同法。其垂于檐端者有滴水,不于脊沿者有云瓦,瓦掩覆脊者有抱同,镇脊两头者有鸟兽诸形象,皆人工逐一做成,载于窑内受水火而成器则一也。

若皇家宫殿所用,大异于是。其制为琉璃瓦者,或为板片,或为宛筒。以圆竹与斫木为模逐片成造,其土必取于太平府(舟运三千里方达京师,参沙之伪,雇役掳舡之扰,害不可极。即承天皇陵亦取于此,无人议正)造成。先装入琉璃窑内,每柴五千斤浇瓦百片。取出,成色以无名异、棕榈毛等煎汁涂染成绿,黛赭石、松香、蒲草等涂染成黄。再入别窑,减杀薪火,逼成琉璃宝色。外省亲王殿与仙佛宫观间亦为之,但色料各有配合,采取不必尽同,民居则有禁也。

\textbf{○砖}

凡埏泥造砖,亦掘地验辨土色,或蓝或白,或红或黄,(闽、广多红泥,蓝者名善泥,江、浙居多。)皆以粘而不散、粉而不沙者为上。汲水滋土,人逐数牛错趾,踏成稠泥,然后填满木匡之中,铁线弓戛平其面,而成坯形。

凡郡邑城雉民居垣墙所用者,有眠砖、侧砖两色。眠砖方长条,砌城郭与民人饶富家,不惜工费直垒而上。民居算计者则一眠之上施侧砖一路,填土砾其中以实之,盖省啬之义也。凡墙砖而外地者名曰方墁砖。榱桷上用以承瓦者曰皇板砖。圆鞠小桥梁与圭门与窀穸墓穴者曰刀砖,又曰鞠砖。凡刀砖削狭一偏面,相靠挤紧,上砌成圆,车马践压不能损陷。

造方墁砖,泥入方匡中,平板盖面,两人足立其上,研转而坚固之,浇成效用。石工磨斫四沿,然后地。刀砖之直视墙砖稍溢一分,皇板砖则积十以当墙砖之一,方墁砖则一以敌墙砖之十也。

凡砖成坯之后,装入窑中,所装百钧则火力一昼夜,二百钧则倍时而足。凡烧砖有柴薪窑,有煤炭窑。用薪者出火成青黑色,用煤者出火成白色。凡柴薪窑巅上偏侧凿三孔以出烟,火足止薪之候,泥固塞其孔,然后使水转氵幼。凡火候少一两则氵幼色不光,少三两则名嫩火砖。本色杂现,他日经霜冒雪,则立成解散,仍还土质。火候多一两则砖面有裂纹,多三两则砖形缩小拆裂,屈曲不伸,击之如碎铁然,不适于用。巧用者以之埋藏土内为墙脚,则亦有砖之用也。凡观火候,从窑门透视内壁,土受火精,形神摇荡,若金银熔化之极然,陶长辨之。

凡转氵幼之法,窑巅作一平田样,四围稍弦起,灌水其上。砖瓦百钧用水四十石。水神透入土膜之下,与火意相感而成。水火既济,其质千秋矣。若煤炭窑视柴窑深欲倍之,其上圆鞠渐小,并不封顶。其内以煤造成尺五径阔饼,每煤一层隔砖一层,苇薪垫地发火。

若皇居所用砖,其大者厂在临清,工部分司主之。初名色有副砖、券砖、平身砖、望板砖、斧刃砖、方砖之类,后革去半。运至京师,每漕舫搭四十块,民舟半之。又细料方砖以正殿者,则由苏州造解。其琉璃砖色料已载《瓦》款。取薪台基厂,烧由黑窑云。

\textbf{○罂瓮}

凡陶家为缶属,其类百千。大者缸瓮,中者钵孟,小者瓶罐,款制各从方土,悉数之不能。造此者必为圆而不方之器。试土寻泥之后,仍制陶车旋盘。工夫精熟者视器大小掐泥,不甚增多少,两人扶泥旋转,一捏而就。其朝迁所用龙凤缸(窑在真定曲阳与扬州仪真)与南直花缸,则厚积其泥,以俟雕镂,作法全不相同,故其直或百倍或五十倍也。

凡罂缶有耳嘴者皆另为合,上以氵幼水涂粘。陶器皆有底,无底者则陕以西炊甑用瓦不用木也。凡诸陶器精者中外皆过釉,粗者或釉其半体。惟沙盆齿钵之类其中不釉,存其粗涩,以受研擂之功。沙锅沙罐不釉,利于透火性以熟烹也。

凡釉质料随地而生,江、浙、闽、广用者蕨蓝草一味。其草乃居民供灶之薪,长不过三尺,枝叶似杉木,勒而不棘人。(其名数十,各地不同。)陶家取来燃灰,布袋灌水澄滤,去其粗者,取其绝细。每灰二碗参以红土泥水一碗,搅令极匀,蘸涂坯上,烧出自成光色。北方未详用何物。苏州黄罐釉亦别有料。惟上用龙凤器则仍用松香与无名异也。

凡瓶窑烧小器,缸窑烧大器。山西、浙江省分缸窑、瓶窑,余省则合一处为之。凡造敞口缸,旋成两截,接合处以木椎内外打紧,匝口、坛瓮亦两截,接合不便用椎,预于别窑烧成瓦圈如金刚圈形,托印其内,外以木椎打紧,土性自合。

凡缸、瓶窑不于平地,必于斜阜山冈之上,延长者或二三十丈,短者亦十余丈,连接为数十窑,皆一窑高一级。盖依傍山势,所以驱流水湿滋之患,而火气又循级透上。其数十方成窑者,其中苦无重值物,合并众力众资而为之也。其窑鞠成之后,上铺覆以绝细土,厚三寸许。窑隔五尺许则透烟窗,窑门两边相向而开。装物以至小器,装载头一低窑,绝大缸瓮装在最末尾高窑。发火先从头一低窑起,两人对面交看火色。大抵陶器一百三十费薪百斤。火候足时,掩闭其门,然后次发第二火。以次结竟至尾云。

\textbf{○白瓷 (附:青瓷)}

凡白土曰垩土,为陶家精美器用。中国出惟五六处,北则真定定州、平凉华亭、太原平定、开封禹州,南则泉郡德化、(土出永定,窑在德化。)徽郡婺源、祁门。(他处白土陶范不粘,或以扫壁为墁。)德化窑惟以烧造瓷仙、精巧人物、玩器,不适实用;真、开等郡瓷窑所出,色或黄滞无宝光,合并数郡不敌江西饶郡产。浙省处州丽水、龙泉两邑,烧造过釉杯碗,青黑如漆,名曰处窑,宋、元时龙泉琉山下,有章氏造窑出款贵重,古董行所谓哥窑器者即此。

若夫中华四裔驰名猎取者,皆饶郡浮梁景德镇之产也。此镇从古及今为烧器地,然不产白土。土出婺源、祁门两山:一名高梁山,出粳米土,其性坚硬;一名开化山,出糯米土,其性粢软。两土和合,瓷器方成。其土作成方块,小舟运至镇。造器者将两土等分入臼舂一日,然后入缸水澄,其上浮者为细料,倾跌过一缸,其下沉底者为粗料。细料缸中再取上浮者,倾过为最细料,沉底者为中料。既澄之后,以砖砌方长塘,逼靠火窑以借火力。倾所澄之泥于中,吸干然后重用清水调和造坯。

凡造瓷坯有两种,一曰印器,如方圆不等瓶瓮炉合之类,御器则有瓷屏风、烛台之类。先以黄泥塑成模印,或两破或两截,亦或囫囵。然后埏白泥印成,以釉水涂合其缝,浇出时自圆成无隙。一曰圆器,凡大小亿万杯盘之类乃生人日用必需,造者居十九,而印器则十一。造此器坯先制陶车。车竖直木一根,埋三尺入土内使之安稳,上高二尺许,上下列圆盘,盘沿以短竹棍拨运旋转,盘顶正中用檀木刻成盔头冒其上。

凡造杯盘无有定形模式,以两手棒泥盔冒之上,旋盘使转,拇指剪去甲,按定泥底,就大指薄旋而上,即成一杯碗之形。(初学者任从作废,破坯取泥再造。)功多业熟,即千万如出一范。凡盔冒上造小杯者不必加泥,造中盘、大碗则增泥大其冒,使干燥而后受功。凡手指旋成坯后,覆转用盔冒一印,微晒留滋润,又一印,晒成极白干,入水一汶,漉上盔冒,过利刀二次,(过刀时手脉微振,烧出即成雀口。)然后补整碎缺,就车上旋转打圈。圈后或画或书字,画后喷水数口,然后过釉。

凡为碎器与千钟粟与褐色杯等,不用青料。欲为碎器,利刀过后,日晒极热。入清水一蘸而起,烧出自成裂纹。千钟粟则釉浆捷点,褐色则老茶叶煎水一抹也。(古碎器日本国极珍重,真者不惜千金。古香炉碎器不知何代造,底有铁钉,其钉掩光色不锈。)

凡饶镇白瓷釉用小港嘴泥浆和桃竹叶灰调成,似清泔汁,(泉郡瓷仙用松毛水调泥浆,处郡青瓷釉未详所出。)盛于缸内。凡诸器过釉,先荡其内,外边用指一蘸涂弦,自然流遍。凡画碗青料总一味无名异。(漆匠煎油,亦用以收火色。)此物不生深土,浮生地面,深者掘下三尺即止,各省直皆有之。亦辨认上料、中料、下料,用时先将炭火丛红煅过。上者出火成翠毛色,中者微青,下者近土褐。上者每斤煅出只得七两,中下者以次缩减。如上品细料器及御器龙凤等,皆以上料画成,故其价每石值银二十四两,中者半之,下者则十之三而已。

凡饶镇所用,以衢、信两郡山中者为上料,名曰浙料,上高诸邑者为中,丰城诸处者为下也。凡使料煅过之后,以乳钵极研,(其钵底留粗,不转釉。)然后调画水。调研时色如皂,入火则成青碧色。凡将碎器为紫霞色杯者,用胭脂打湿,将铁线纽一兜络,盛碎器其中,炭火炙热,然后以湿胭脂一抹即成。凡宣红器乃烧成之后出火,另施工巧微炙而成者,非世上殊砂能留红质于火内也。(宣红元末已失传。正德中历试复造出。)

凡瓷器经画过釉之后,装入匣钵。(装时手拿微重,后日烧出即成坳口,不复周正。)钵以粗泥造,其中一泥饼托一器,底空处以沙实之。大器一匣装一个,小器十余共一匣钵。钵佳者装烧十余度,劣者一二次即坏。凡匣钵装器入窑,然后举火。其窑上空十二圆眼,名曰天窗。火以十二时辰为足。先发门火十个时,火力从下攻上,然后天窗掷柴烧两时,火力从上透下。器在火中其软如棉絮,以铁叉取一以验火候之足。辨认真足,然后绝薪止火。共计一坯工力,过手七十二方克成器,其中微细节目尚不能尽也。

\textbf{○附:窑变 回青}

正德中,内使监造御器。时宣红失传不成,身家俱丧。一人跃入自焚。托梦他人造出,竞传窑变,好异者遂妄传烧出鹿、象诸异物也。又回青乃西域大青,美者亦名佛头青。上料无名异出火似之,非大青能入洪炉存本色也。

\hypertarget{header-n2481}{%
\subsubsection{冶铸}\label{header-n2481}}

宋子曰:首山之采,肇自轩辕,源流远矣哉。九牧贡金,用襄禹鼎,从此火金功用日异而月新矣。夫金之生也,以土为母,及其成形而效用于世也,母模子肖,亦犹是焉。精粗巨细之间,但见钝者司舂,利者司垦,薄其身以媒合水火而百姓繁,虚其腹以振荡空灵而八音起。愿者肖仙梵之身,而尘凡有至象。巧者夺上清之魄,而海宇遍流泉,即屈指唱筹,岂能悉数!要之,人力不至于此。

\textbf{○鼎}

凡铸鼎,唐虞以前不可考。唯禹铸九鼎,则因九州贡赋壤则已成,入贡方物岁例已定,疏浚河道已通,禹贡业已成书。恐后世人君增赋重敛,后代侯国冒贡奇淫,后日治水之人不由其道,故铸之于鼎。不如书籍之易去,使有所遵守,不可移易,此九鼎所为铸也。年代久远,末学寡闻,如珠、暨鱼、狐狸、织皮之类皆其刻画于鼎上者,或漫灭改形未可知,陋者遂以为怪物。故《春秋传》有使知神奸、不逢魑魅之说也。此鼎入秦始亡。而春秋时郜大鼎、莒二方鼎,皆其列国自造,即有刻画必失禹贡初旨。此但存名为古物,后世图籍繁多,百倍上古,亦不复铸鼎,特并志之。

\textbf{○钟}

凡钟为金乐之首,其声一宣,大者闻十里,小者亦及里之余。故君视朝、官出署必用以集众,而乡饮酒礼必用以和歌,梵宫仙殿必用以明摄谒者之城,幽起鬼神之敬。

凡铸钟高者铜质,下者铁质。今北极朝钟则纯用响铜,每口共费铜四万七千斤、锡四千斤、金五十两、银一百二十两于内。成器亦重二万斤,身高一丈一尺五寸,双龙蒲牢高二尺七寸,口径八尺,则今朝钟之制也。

凡造万钧钟与铸鼎法同,掘坑深丈几尺,燥筑其中如房舍,埏泥作模骨,用石灰、三和土筑,不使有丝毫隙拆。干燥之后以牛油、黄蜡附其上数寸。油蜡分两:油居什八,蜡居什二,其上高蔽抵晴雨。(夏月不可为,油不冻结。)油蜡墁定,然后雕镂书文、物象,丝发成就。然后舂筛绝细土与炭末为泥,涂墁以渐而加厚至数寸,使其内外透体干坚,外施火力炙化其中油蜡,从口上孔隙熔流净尽,则其中空处即钟鼎托体之区也。

凡油蜡一斤虚位,填铜十斤。塑油时尽油十斤,则备铜百斤以俟之。中既空净,则议熔铜。凡火铜至万钧,非手足所能驱使。四面筑炉,四面泥作槽道,其道上口承接炉中,下口斜低以就钟鼎入铜孔,槽傍一齐红炭织围。洪炉熔化时,决开槽梗,(先泥土为梗塞住。)一齐如水横流,从槽道中枧注而下,钟鼎成矣。凡万钧铁钟与炉、釜,其法皆同,而塑法则由人省啬也。若千斤以内者则不须如此劳费,但多捏十数锅炉。炉形如箕,铁条作骨,附泥做就。其下先以铁片圈筒直透作两孔,以受杠穿。其炉垫于土墩之上,各炉一齐鼓鞲熔化。化后以两杠穿炉下,轻者两人,重者数人抬起,倾注模底孔中。甲炉既倾,乙炉疾继之,丙炉又疾继之,其中自然粘合。若相承迂缓,则先入之质欲冻,后者不粘,衅所由生也。

凡铁钟模不重费油蜡者,先埏土作外模,剖破两边形或为两截,以子口串合,翻刻书文于其上。内模缩小分寸,空其中体,精美而就。外模刻文后以牛油滑之,使他日器无粘,然后盖上,泥合其缝而受铸焉。巨磬、云板,法皆仿此。

\textbf{○釜}

凡釜储水受火,日用司命系焉。铸用生铁或废铸铁器为质。大小无定式,常用者径口二尺为率,厚约二分。小者径口半之,厚薄不减。其模内外为两层,先塑其内,俟久日干燥,合釜形分寸于上,然后塑外层盖模。此塑匠最精,差之毫厘则无用。

模既成就干燥,然后泥捏冶炉,其中如釜,受生铁于中,其炉背透管通风,炉面捏嘴出铁。一炉所化约十釜、二十釜之料。铁化如水,以泥固纯铁柄杓从嘴受注。一杓约一釜之料,倾注模底孔内,不俟冷定即揭开盖模,看视罅绽未周之处。此时釜身尚通红未黑,有不到处即浇少许于上补完,打湿草片按平,若无痕迹。

凡生铁初铸釜,补绽者甚多,唯废破釜铁熔铸,则无复隙漏。(朝鲜国俗破釜必弃之山中,不以还炉。)凡釜既成后,试法以轻杖敲之,响声如木者佳,声有差响则铁质未熟之故,他日易为损坏。海内丛林大处,铸有千僧锅者,煮糜受米二石,此真痴物也。

\textbf{○像}

凡铸仙佛铜像,塑法与朝钟同。但钟鼎不可接,而像则数接为之,故写时为力甚易,但接模之法分寸最精云。

\textbf{○炮}

凡铸炮,西洋、红夷、佛郎机等用熟铜造,信炮、短提铳等用生熟铜兼半造,襄阳、盏口、大将军、二将军等用铁造。

\textbf{○镜}

凡铸镜,模用灰沙,铜用锡和。(不用倭铅。)《考工记》亦云:``金锡相半,渭之鉴、燧之剂。''开面成光,则水银附体而成,非铜有光明如许也。唐开元宫中镜尽以白银与铜等分铸成,每口值银数两者以此故。朱砂斑点乃金银精华发现。(古炉有入金于内者。)我朝宣炉亦缘某库偶灾,金银杂铜锡化作一团,命以铸炉。(真者错现金色。)唐镜、宣炉皆朝廷盛世物云。

\textbf{○钱}

凡铸铜为钱以利民用,一面刊国号通宝四字,工部分司主之。凡钱通利者,以十文抵银一分值。其大钱当五、当十,其弊便于私铸,反以害民,故中外行而辄不行也。

凡铸钱每十斤,红铜居六七,倭铅(京中名水锡)居三四,此等分大略。倭铅每见烈火必耗四分之一。我朝行用钱高色者,唯北京宝源局黄钱与广东高州炉青钱,(高州钱行盛漳泉路。)其价一文敌南直江、浙等二文。黄钱又分二等,四火铜所铸曰金背钱,二火铜所铸曰火漆钱。

凡铸钱熔铜之罐,以绝细土末(打碎干土砖妙)和炭末为之。(京炉用牛蹄甲,未详何作用,)罐料十两,土居七而炭居三,以炭灰性暖,佐土使易化物也。罐长八寸,口径二寸五分。一罐约载铜、铅十斤,铜先入化,然后投铅,洪沪扇合,倾入模内。

凡铸钱模以木四条为空匡。(木长一尺一寸,阔一寸二分。)土炭末筛令极细,填实匡中,微洒杉木炭灰或柳木炭灰于其面上,或熏模则用松香与清油,然后以母百文(用锡雕成)或字或背布置其上。又用一匡如前法填实合盖之。既合之后,已成面、背两匡,随手覆转,则母钱尽落后匡之上。又用一匡填实,合上后匡,如是转覆,只合十余匡,然后以绳捆定。其木匡上弦原留入铜眼孔,铸工用鹰嘴钳,洪炉提出熔罐,一人以别钳扶抬罐底相助,逐一倾入孔中。冷定解绳开匡,则磊落百丈,如花果附枝。模中原印空梗,走铜如树枝样,挟出逐一摘断,以待磨钅差成钱。凡钱先错边沿,以竹木条直贯数百文受钅差,后钅差平面则逐一为之。

凡钱高低以铅多寡分,其厚重与薄削,则昭然易见。铅贱铜贵,私铸者至对半为之,以之掷阶石上,声如木石者,此低钱也。若高钱铜九铅一,则掷地作金声矣。凡将成器废铜铸钱者,每火十耗其一。盖铅质先走,其铜色渐高,胜于新铜初化者。若琉球诸国银钱,其模即凿锲铁钳头上,银化之时入锅夹取,淬于冷水之中,即落一钱其内。图并具后。

\textbf{○附:铁钱}

铁质贱甚,从古无铸钱。起于唐藩镇魏博诸地,铜货不通,始冶为之,盖斯须之计也。皇家盛时则冶银为豆,杂伯衰时则铸铁为钱。并志博物者感慨。

\hypertarget{header-n2512}{%
\subsubsection{舟车}\label{header-n2512}}

宋子曰:人群分而物异产,来往懋迁以成宇宙。若各居而老死,何藉有群类哉?人有贵而必出,行畏周行;物有贱而必须,坐穷负贩。四海之内,南资舟而北资车。梯航万国,能使帝京元气充然。何其始造舟车者不食尸祝之报也。浮海长年,视万顷波如平地,此与列子所谓御泠风者无异。传所称奚仲之流,倘所谓神人者非耶!

\textbf{○舟}

凡舟古名百千,今名亦百千,或以形名,(如海鳅、江鳊、山梭之类。)或以量名,(载物之数。)或以质名,(各色木料。)不可殚述。游海滨者得见洋船,居江湄者得见漕舫。若局趣山国之中,老死平原之地,所见者一叶扁舟、截流乱筏而已。粗载数舟制度,其余可例推云。

\textbf{○漕舫}

凡京师为军民集区,万国水运以供储,漕舫所由兴也。元朝混一,以燕京为大都。南方运道由苏州刘家港、海门黄连沙开洋,直抵天津,制度用遮洋船。永乐间因之。以风涛多险,后改漕运。

平江伯陈某始造平底浅船,则今粮船之制也。凡船制底为地,枋为宫墙,阴阳竹为覆瓦。伏狮前为阀阅,后为寝堂。桅为弓弩,弦、篷为翼,橹为车马,\{\}纤为履鞋,纟聿索为鹰雕筋骨,招为先锋,舵为指挥主帅,锚为扎车营寨。

粮船初制,底长五丈二尺,其板厚二寸,采巨木楠为上,栗次之。头长九尺五寸,梢长九尺五寸。底阔九尺五寸,底头阔六尺,底梢阔五尺,头伏狮阔八尺,梢伏狮阔七尺,梁头一十四座。龙口梁阔一丈,深四尺,使风梁阔一丈四尺,深三尺八寸。后断水梁阔九尺,深四尺五寸。两廒共阔七尺六寸。此其初制,载米可近二千石。(交兑每只止足五百石。)

后运军造者私增身长二丈,首尾阔二尺余,其量可受三千石。而运河闸口原阔一丈二尺,差可度过。凡今官坐船,其制尽同,第窗户之间宽其出径,加以精工彩饰而已。

凡造船先从底起,底面傍靠樯,上承栈,下亲地面。隔位列置者曰梁。两傍峻立者曰樯。盖樯巨木曰正枋,枋上曰弦。梁前竖桅位曰锚坛,坛底横木夹桅本者曰地龙,前后维曰伏狮,其下曰拿狮,伏狮下封头木曰连三枋。船头面中缺一方曰水井。(其下藏缆索等物。)头面眉际树两木以系缆者曰将车柱。船尾下斜上者曰草鞋底,后封头下曰短枋,枋下曰挽脚梁,船梢掌舵所居其上者野鸡篷。(使风时,一人坐篷巅,收守篷索。)

凡舟身将十丈者,立桅必两,树中桅之位,折中过前二位,头桅又前丈余。粮船中桅长者以八丈为率,短者缩十之一二。其本入窗内亦丈余,悬篷之位约五六丈。头桅尺寸则不及中桅之半,篷纵横亦不敌三分之一。苏、湖六郡运米,其船多过石瓮桥下,且无江汉之险,故桅与篷尺寸全杀。若湖广、江西省舟,则过湖冲江无端风浪,故锚、缆、篷、桅必极尽制度而后无患。凡风篷尺寸,其则一视全舟横身,过则有患,不及则力软。

凡船篷其质乃析篾成片织就,夹维竹条,逐块折叠,以俟悬挂。粮船中桅篷合并十人力方克凑顶,头篷则两人带之有余。凡度篷索先系空中寸圆木关捩于桅巅之上,然后带索腰间缘木而上,三股交错而度之。凡风篷之力其末一叶,敌其本三叶。调匀和畅顺风则绝顶张篷,行疾奔马。若风力氵存至,则以次减下。(遇风鼓急不下,以钩搭扯。)狂甚则只带一两叶而已。

凡风从横来名曰抢风。顺水行舟,则挂篷之玄游走,或一抢向东,止寸平过,甚至却退数十丈。未及岸时捩舵转篷,一抢向西,借贷水力兼带风力轧下,则顷刻十余里。或湖水平而不流者亦可缓轧。若上水舟则一步不可行也。凡船性随水,若草从风,故制舵障水使不定向流,舵板一转,一泓从之。

凡舵尺寸,与船腹切齐。其长一寸,则遇浅之时船腹已过,其梢尼舵使胶住,设风狂力劲,则寸木为难不可言。舵短一寸则转运力怯,回头不捷。凡舵力所障水,相应及船头而止,其腹底之下俨若一派急顺流,故船头不约而正,其机妙不可言。

舵上所操柄名曰关门棒,欲船北则南向捩转,欲船南则北向捩转。船身太长而风力横劲,舵力不甚应手,则急下一偏披水板以抵其势。凡舵用直木一根(粮船用者围三尺,长丈余)为身,上截衡受棒,下截界开衔口,纳板其中如斧形,铁钉固拴以障水。梢后隆起处,亦名曰舵楼。

凡铁锚所以沉水系舟。一粮船计用五六锚,最雄者曰看家锚,重五百斤内外,其余头用二枝,梢用二枝。凡中流遇逆风不可去又不可泊,(或业已近岸,其下有石非沙,亦不可泊,惟打锚深处。)则下锚沉水底,其所系纟聿缠绕将军柱上,锚爪一遇泥沙扣底抓住,十分危急则下看家锚。系此锚者名曰本身,盖重言之也。或同行前舟阻滞,恐我舟顺势急去有撞伤之祸,则急下梢锚提住,使不迅速流行。风息开舟则以云车纹缆提锚使上。

凡船板合隙缝以白麻斫絮为筋,钝凿扌及入,然后筛过细石灰,和桐油舂杵成团调念。温、台、闽、广即用蛎灰。凡舟中带篷索,以火麻秸(一名大麻)绞。粗成径寸以外者即系万钧不绝。若系锚缆则破析青篾为之,其篾线入釜煮熟然后纠绞。拽缱\{\}亦煮熟篾线绞成十丈以往,中作圈为接区,遇阻碍可以掐断。凡竹性直,篾一线千钧。三峡入川上水舟,不用纠绞\{\}缱,即破竹阔寸许者,整条以次接长,名曰火杖。盖沿崖石棱如刃,惧破篾易损也。

凡木色桅用端直杉木,长不足则接,其表铁箍逐寸包围。船窗前道皆当中空阙,以便树桅。凡树中桅,合并数巨舟承载,其未长缆系表而起。梁与枋樯用楠木、槠木、樟木、榆木、槐木。(樟木春夏伐者,久则粉蛀。)栈板不拘何木。舵杆用榆木、榔木、槠木。关门棒用周木、榔木。橹用杉木、桧木、楸木。此其大端云。

\textbf{○海舟}

凡海舟,元朝与国初运米者曰遮洋浅船,次者曰钻风船(即海鳅。)所经道里,止万里长滩、黑水洋、沙门岛等处,皆无大险。与出使琉球、日本暨商贾爪哇、笃泥等船制度,工费不及十分之一。

凡遮洋运船制,视漕船长一丈六尺,阔二尺五寸,器具皆同,唯舵杆必用铁力木,念灰用鱼油和桐油,不知何义。凡外国海舶制度大同小异,闽、广(闽由海澄开洋,广由香奥)洋船截竹两破排栅,树于两傍以抵浪。登、莱制度又不然,倭国海舶两傍列橹手栏板抵水,人在其中运力。朝鲜制度又不然。

至其首尾各安罗经盘以定方向,中腰大横梁出头数尺,贯插腰舵,则皆同也。腰舵非与梢舵形同,乃阔板斫成刀形插入水中,亦不捩转,盖夹卫扶倾之义。其上仍横柄栓于梁上,而遇浅则提起,有似乎舵,故名腰舵也。凡海舟以竹筒贮淡水数石,度供舟内人两日之需,遇岛又汲。其何国何岛合用何向,针指示昭然,恐非人力所祖。舵工一群主佐,直是识力造到死生浑忘地,非鼓勇之谓也。

\textbf{○杂舟}

江汉课船。身甚狭小而长,上列十余仓,每仓容止一人卧息。首尾共桨六把,小桅篷一座。风涛之中恃有多桨挟持。不遇逆风,一昼夜顺水行四百余里,逆水亦行百余里,国朝盐课淮、扬数颇多,故设此运银,名曰课船。行人欲速者亦买之。其船南自章、贡,西自荆、襄,达于瓜、仪而止。

三吴浪船。凡浙西、平江纵横七百里内尽是深沟小水湾环,浪船(最小者曰塘船)以万亿计。其舟行人贵贱来往以代马车、履。舟即小者必造窗牖堂房,质料多用杉木。人物载其中,不可偏重一石,偏即欹侧,故俗名天平船。此舟来往七百里内,或好逸便者径买,北达通、津,只有镇江一横渡,俟风静涉过,又渡清江浦,溯黄河浅水二百里则入闸河安稳路矣。至长江上流风浪,则没世避而不经也。浪船行力在梢后,巨橹一枝两三人推轧前走,或恃缱\{\}。至于风篷,则小席如掌所不恃也。

东浙西安船。浙东自常山至钱塘八百里,水径入海,不通他道,故此舟自常山、开化、遂安等小河起,至钱塘而止,更无他涉。舟制箬篷如卷瓦为上盖。缝布为帆,高可二丈许,绵索张带。初为布帆者,原因钱塘有潮涌,急时易于收下。此亦未然,其费似侈于篾席,总不可晓。

福建清流、梢篷船。其船自光泽、崇安两小河起,达于福州洪塘而止,其下水道皆海矣。清流船以载货物、客商,梢篷船大差可坐卧,官贵家属用之。其船皆以杉木为地。滩石甚险,破损者其常,遇损则急舣向岸搬物掩塞。船梢径不用舵,船首列一巨招,捩头使转。每帮五只方行,经一险滩则四舟之人皆从尾后曳缆,以缓其趋势。长年即寒冬不裹足,以便频濡。风篷竟悬不用云。

四川八橹等船。凡川水源通江、汉,然川船达荆州而止,此下则更舟矣。逆行而上,自夷陵入峡,挽缱者以巨竹破为四片或六片,麻绳约接,名曰火杖。舟中鸣鼓若竞渡,挽人从山石中闻鼓声而咸力。中夏至中秋川水封峡,则断绝行舟数月。过此消退,方通往来。其新滩等数极险处,人与货尽盘岸行半里许,只余空舟上下。其舟制腹圆而首尾尖狭,所以辟滩浪云。

黄河满篷梢。其船自河入淮,自淮溯汴用之。质用楠木,工价颇优。大小不等,巨者载三千石,小者五百石。下水则首颈之际,横压一梁,巨橹两枝,两傍推轧而下。锚、缆、\{\}、帆制与江、汉相仿云。

广东黑楼船、盐船。北自南雄,南达会省,下此惠、潮通漳、泉则由海汊乘海舟矣。黑楼船为官贵所乘,盐船以载货物。舟制两傍可行走。风帆编蒲为之,不挂独竿桅,双柱悬帆不若中原随转。逆流冯藉缱力,则与各省直同功云。

黄河秦船(俗名摆子船。)造作多出韩城,巨者载石数万钧顺流而下,供用淮、徐地面。舟制首尾方阔均等,仓梁平下不甚隆起,急流顺下,巨橹两傍夹推,来往不冯风力。归舟挽缱多至二十余人,甚有弃舟空返者。

\textbf{○车}

凡车利行平地,古者秦、晋、燕、齐之交,列国战争必用车,故千乘、万乘之号起自战争国。楚、汉血争而后日辟。南方则水战用舟,陆战用步马,北膺胡虏交使铁骑,战车逐无所用之。但今服马驾车以运重载,则今日骡车即同彼时战车之义也。

凡骡车之制有四轮者,有双轮者,其上承载支架,皆从轴上穿斗而起。四轮者前后各横轴一根,轴上短柱起架直梁,梁上载箱。马止脱驾之时,其上平整,如居屋安稳之象。若两轮者驾马行时,马曳其前则箱地平正,脱马之时则以短木从地支撑而住,不然则欹卸也。

凡车轮一曰辕。(俗名车陀。)其大车中毂(俗名车脑)长一尺五寸,(见《小戎》车注。)所谓外受辐、中贯轴者。辐计三十片,其内插毂,其外接辅。车轮之中内集轮外接辋,圆转一圈者是曰辅也。辋际尽头则曰轮辕也。凡大车脱时则诸物星散收藏。驾则先上两轴,然后以次间架。凡轼、衡、轸、轭皆从轴上受基也。

凡四轮大车量可载五十石,骡马多者或十二挂或十挂,少亦八挂。执鞭掌御者居箱之中,立足高处。前马分为两班,(战车四马一班,分骖、服。)纠黄麻为长索分系马项,后套总结收入衡内两旁。掌御者手执长鞭,鞭以麻为绳,长七尺许,竿身亦相等,察视不力者鞭及其身。箱内用二人踹绳,须识马性与索性者为之。马行太紧则急起踹绳,否则翻车之祸从此起也。凡车行时遇前途行人应避者,则掌御者急以声呼,则群马皆止。凡马索总系透衡入箱处,皆以牛皮束缚,《诗经》所谓``胁驱''是也。

凡大车饲马不入肆舍,车上载有柳盘,解索而野食之。乘车人上下皆缘小梯。凡过桥梁中高边下者,则十马之中择一最强力者系于车后。当其下坂,则九马从前缓曳,一马从后竭力抓住,以杀其驰趋之势,不然则险道也。凡大车行程,遇河亦止,遇山亦止,遇曲径小道亦止。徐、兖、汴梁之交或达三百里者,无水之国所以济舟楫之穷也。

凡车质惟先择长者为轴,短者为毂,其木以槐、枣、檀、榆(用榔榆)为上。檀质太久劳则发烧,有慎用者合抱枣、槐,其至美也。其余轸、衡、箱、轭则诸木可为耳。此外,牛车以载刍粮,最盛晋地。路逢隘道则牛颈系巨铃,名曰报君知,犹之骡车群马尽系铃声也。

又北方独辕车,人推其后,驴曳其前,行人不耐骑坐者,则雇觅之。鞠席其上以蔽风日。人必两旁对坐,否则欹倒。此车北上长安、济宁径达帝京。不载人者,载货约重四五石而止。其驾牛为轿车者,独盛中州。两旁双轮,中穿一轴,其分寸平如水。横架短衡列轿其上,人可安坐,脱驾不欹。其南方独轮推车,则一人之力是视。容载两石,遇坎即止,最远者止达百里而已。其余难以枚述。但生于南方者不见大车,老于北方者不见巨舰,故粗载之。

\hypertarget{header-n2554}{%
\subsubsection{锤锻}\label{header-n2554}}

宋子曰:金木受攻而物象曲成。世无利器,即般、亻垂安所施其巧哉?五兵之内,六乐之中,微钳锤之奏功也,生杀之机泯然矣。同出洪炉烈火,大小殊形。重千钧者系巨舰于狂渊。轻一羽者透绣纹于章服。使冶铸鼎之巧,束手而让神功焉。莫邪、干将,双龙飞跃,毋其说亦有征焉者乎?

\textbf{○治铁}

凡治铁成器,取已炒熟铁为之。先铸铁成砧,以为受锤之地。谚云``万器以钳为祖'',非无稽之说也。凡出炉熟铁名曰毛铁。受锻之时,十耗其三为铁华、铁落。若已成废器未锈烂者名曰劳铁,改造他器与本器,再经锤煅,十止耗去其一也。凡炉中炽铁用炭,煤炭居十七,木炭居十三。凡山林无煤之处,锻工先择坚硬条木烧成火墨。(俗名火矢,扬烧不闭穴火。)其炎更烈于煤。即用煤炭,也别有铁炭一种,取其火性内攻,焰不虚腾者,与炊炭同形而有分类也。

凡铁性逐节粘合,涂上黄泥于接口之上,入火挥槌,泥滓成枵而去,取其神气为媒合。胶结之后,非灼红斧斩,永不可断也。凡熟铁、钢铁已经炉锤,水火未济,其质未坚。乘其出火时,入清水淬之,名曰健钢、健铁。言乎未健之时,为钢为铁,弱性犹存也。凡焊铁之法,西洋诸国别有奇药。中华小焊用白铜末,大焊则竭力挥锤而强合之,历岁之久终不可坚。故大炮西番有锻成者,中国惟恃冶铸也。

\textbf{○斤斧}

凡铁兵薄者为刀剑,背厚而面薄者为斧斤。刀剑绝美者以百炼钢包裹其外,其中仍用无钢铁为骨。若非钢表铁里,则劲力所施即成折断。其次寻常刀斧,止嵌钢于其面。即重价宝刀可斩钉截凡铁者,经数千遭磨砺,则钢尽而铁现也。倭国刀背阔不及二分许,架于手指之上不复欹倒,不知用何锤法,中国未得其传。

凡健刀斧皆嵌钢、包钢,整齐而后入水淬之。其快利则又在砺石成功也。凡匠斧与椎,其中空管受柄处,皆先打冷铁为骨,名曰羊头,然后热铁包裹,冷者不粘,自成空隙。凡攻石椎日久四面皆空,熔铁补满平填,再用无弊。

\textbf{○锄}

凡治地生物,用锄、之属,熟铁锻成,熔化生铁淋口,入水淬健,即成刚劲。每锹、锄重一斤者,淋生铁三钱为率,少则不坚,多则过刚而折。

\textbf{○鎈}

凡铁鎈纯钢为之,未健之时钢性亦软。以已健钢钅斩划成纵斜文理,划时斜向入,则文方成焰。划后浇红,退微冷,入水健。久用乖平,入水退去健性,再用钅斩划。凡鎈开锯齿用茅叶鎈,后用快弦鎈。治铜钱用方长牵鎈,锁钥之类用方条鎈,治骨角用剑面鎈。(朱注所谓钅虑锡。)治木末则锥成圆眼,不用纵斜文者,名曰香鎈。(划鎈纹时,用羊角末和盐醋先涂。)

\textbf{○锥}

凡锥熟铁锤成,不入钢和。治书编之类用圆钻,攻皮革用扁钻。梓人转索通眼、引钉合木者,用蛇头钻。其制颖上二分许,一面圆,一面剜入,傍起两棱,以便转索。治铜叶用鸡心钻,其通身三棱者名旋钻,通身四方而末锐者名打钻。

\textbf{○锯}

凡锯熟铁锻成薄条,不钢,亦不淬健。出火退烧后,频加冷锤坚性,用鎈开齿。两头衔木为梁,纠篾张开,促紧使直。长者刮木,短者截木,齿最细者截竹。齿钝之时,频加鎈锐而后使之。

\textbf{○刨}

凡刨磨砺嵌钢寸铁,露刃秒忽,斜出木口之面,所以平木,古名曰准。巨者卧准露刃,持木抽削,名曰推刨,圆桶家使之。寻常用者横木为两翅,手执前推。梓人为细功者,有起线刨,刃阔二分许。又刮木使极光者名蜈蚣刨,一木之上,衔十余小刀,如蜈蚣之足。

\textbf{○凿}

凡凿熟铁锻成,嵌钢于口,其本空圆,以受木柄。(先打铁骨为模,名曰羊头,杓柄同用。)斧从柄催,入木透眼,其末粗者阔寸许,细者三分而止。需圆眼者则制成剜凿为之。

\textbf{○锚}

凡舟行遇风难泊,则全身系命于锚。战船、海船有重千钧者,锤法先成四爪,以次逐节接身。其三百斤以内者用径尺阔砧,安顿炉傍,当其两端皆红,掀去炉炭,铁包木棍夹持上砧。若千斤内外者则架木为棚,多人立其上共持铁链。两接锚身,其末皆带巨铁圈链套,提起捩转,咸力锤合。合药不用黄泥,先取陈久壁土筛细,一人频撒接口之中,浑合方无微罅。盖炉锤之中,此物其最巨者。

\textbf{○针}

凡针先锤铁为细条。用铁尺一根,锥成线眼,抽过条铁成线,逐寸剪断为针。先鎈其末成颖,用小槌敲扁其本,钢锥穿鼻,复鎈其外。然后入釜,慢火炒熬。炒后以土末入松木火矢,豆豉三物罨盖,下用火蒸。留针二三口插于其外,以试火候。其外针入手捻成粉碎,则其下针火候皆足。然后开封,入水健之。凡引线成衣与刺绣者,其质皆刚。惟马尾刺工为冠者,则用柳条软针。分别之妙,在于水火健法云。

\textbf{○治铜}

凡红铜升黄而后熔化造器,用砒升者为白铜器,工费倍难,侈者事之。凡黄铜,原从炉甘石升者不退火性受锤;从倭铅升者出炉退火性,以受冷锤。凡响铜入锡参和(法具《五金》卷)成乐器者,必圆成无焊。其余方圆用器,走焊、炙火粘合。用锡末者为小焊,用响铜末者为大焊。(碎铜为末,用饭粘和打,入水洗去饭。铜末具存,不然则撒散。)若焊银器,则用红铜末。

凡锤乐器,锤钲(俗名锣)不事先铸,熔团即锤。锤镯(俗名铜鼓)与丁宁,则先铸成圆片,然后受锤。凡锤钲、镯皆铺团于地面。巨者众共挥力,由小阔开,就身起弦声,俱从冷锤点发。其铜鼓中间突起隆炮,而后冷锤开声。声分雌与雄,则在分厘起伏之妙。重数锤者,其声为雄。凡铜经锤之后,色成哑白,受鎈复现黄光。经锤折耗,铁损其十者,铜只去其一。气腥而色美,故锤工亦贵重铁工一等云。

\hypertarget{header-n2584}{%
\subsubsection{燔石}\label{header-n2584}}

宋子曰:五行之内,土为万物之母。子之贵者,岂惟五金哉。金与水相守而流,功用谓莫尚焉矣。石得燔而成功,盖愈出而愈奇焉。水浸淫而败物,有隙必攻,所谓不遗丝发者。调和一物以为外拒,漂海则冲洋澜,粘则固城雉。不烦历候远涉,而至宝得焉。燔石之功,殆莫之与京矣。至于矾现五金色之形,硫为群石之将,皆变化于烈火。巧极丹铅炉火,方士纵焦劳唇舌,何尝肖像天工之万一哉!

\textbf{○石灰}

凡石灰经火焚炼为用。成质之后,入水永劫不坏。亿万舟楫,亿万垣墙,窒隙防淫,是必由之。百里内外,土中必生可燔石,石以青色为上,黄白次之。石必掩土内二三尺,掘取受燔,土面见风者不用。燔灰火料煤炭居什九,薪炭居什一。先取煤炭泥和做成饼,每煤饼一层叠石一层,铺薪其底,灼火燔之。最佳者曰矿灰,最恶者曰窑滓灰。火力到后,烧酥石性,置于风中久自吹化成粉。急用者以水沃之,亦自解散。

凡灰用以固舟缝,则桐油、鱼油调厚绢、细罗,和油杵千下塞念。用以砌墙石,则筛去石块,水调粘合。墁则仍用油灰。用以垩墙壁,则澄过入纸筋涂墁。用以襄墓及贮水池,则灰一分,入河沙、黄土二分,用糯粳米、羊桃藤汁和匀,轻筑坚固,永不隳坏,名曰三和土。其余造淀造纸。功用难以枚述。凡温、台、闽、广海滨石不堪灰者,则天生蛎蚝以代之。

\textbf{○蛎灰}

凡海滨石山傍水处,咸浪积压,生出蛎房,闽中曰蚝房。经年久者长成数丈,阔则数亩,崎岖如石假山形象。蛤之类压入岩中,久则消化作肉团,名曰蛎黄,味极珍美。凡燔蛎灰者,执椎与凿,濡足取来,(药铺所货牡蛎,即此碎块。)叠煤架火燔成,与前石灰共法。粘砌成墙、桥梁,调和桐油造舟,功皆相同。有误以蚬灰(即蛤粉)为蛎灰者,不格物之故也。

\textbf{○煤炭}

凡煤炭普天皆生,以供锻炼金石之用。南方秃山无草木者,下即有煤,北方勿论。煤有三种,有明煤、碎煤、末煤。明煤大块如斗许,燕、齐、秦、晋生之。不用风箱鼓扇,以木炭少许引燃,炽达昼夜。其傍夹带碎屑,则用洁净黄土调水作饼而烧之。碎煤有两种,多生吴、楚。炎高者曰饭炭,用以炊烹;炎平者曰铁炭,用以治锻。入炉先用水沃湿,必用鼓鞲后红,以次增添而用。末煤如面者,名曰自来风。泥水调成饼,入于炉内,既灼之后,与明煤相同,经昼夜不灭,半供炊爨。半供熔铜、化石、升朱。至于燔石为灰与矾、硫,则三煤皆可用也。

凡取煤经历久者,从土面能辨有无之色,然后掘挖,深至五丈许方始得煤。初见煤端时,毒气灼人。有将巨竹凿去中节,尖锐其末,插入炭中,其毒烟从竹中透上,人从其下施拾取者。或一井而下,炭纵横广有,则随其左右阔取。其上枝板,以防压崩耳。

凡煤炭取空而后,以土填实其井,以二三十年后,其下煤复生长,取之不尽。其底及四周石卵,土人名曰铜炭者,取出烧皂矾与硫黄。(详见后款)凡石卵单取硫黄者,其气薰甚,名曰臭煤,燕京房山、固安、湖广荆州等处间有之。凡煤炭经焚而后,质随火神化去,总无灰滓。盖金与土石之间,造化别现此种云。凡煤炭不生茂草盛木之乡,以见天心之妙。其炊爨功用所不及者,唯结腐一种而已。(结豆腐者用煤炉则焦苦。)

\textbf{○矾石 白矾}

凡矾燔石而成。白矾一种,亦所在有之。最盛者山西晋、南直无为等州,值价低贱,与寒水石相仿。然煎水极沸,投矾化之,以之染物,则固结肤膜之间,外水永不入,故制糖饯与染画纸、红纸者需之。其末干撒,又能冶浸淫恶水,故湿疮家亦急需之也。

凡白矾,掘土取磊块石,层叠煤炭饼锻炼,如烧石灰样。火候已足,冷定入水。煎水极沸时,盘中有溅溢如物飞出,俗名蝴蝶矾者,则矾成矣。煎浓之后,入水缸内澄,其上隆结曰吊矾,洁白异常。其沉下者曰缸矾。轻虚如棉絮者曰柳絮矾,烧汁至尽,白如雪者,谓之巴石。方药家锻过用者曰枯矾云。

\textbf{○青矾 红矾 黄矾 胆矾}

凡皂、红、黄矾,皆出一种而成,变化其质。取煤炭外矿石(俗名铜炭)子,每五百斤入炉,炉内用煤炭饼(自来风不用鼓鞲者)千余斤,周围包裹此石。炉外砌筑土墙圈围,炉巅空一圆孔如茶碗口大,透炎直上,孔傍以矾滓厚罨。(此滓不知起自何世,欲作新炉者,非旧滓罨盖则不成。)然后从底发火,此火度经十日方熄。其孔眼时有金色光直上。(取硫,详后款。)

锻经十日后,冷定取出。半酥杂碎者另拣出,名曰时矾,为煎矾红用。其中清淬如矿灰形者,取入缸中浸三个时,漉入釜中煎炼。每水十石煎至一石,火候方足。煎干之后,上结者皆佳好皂矾,下者为矾滓。(后炉用此盖。)此皂矾染家必需用。中国煎者亦惟五六所。原石五百斤成皂矾二百斤,其大端也。其拣出时矾(俗又名鸡屎矾)每斤入黄土四两,入罐熬炼,则成矾红。圬墁及油漆家用之。

其黄矾所出又奇甚,乃即炼皂矾炉侧土墙,春夏经受火石精气,至霜降、立冬之交,冷静之时,其墙上自然爆出此种,如淮北砖墙生焰硝样。刮取下来,名曰黄矾,染家用之。金色淡者涂炙,立成紫赤也。其黄矾自外国来,打破,中有金丝者,名曰波斯矾,别是一种。

又山、陕烧取硫黄山上,其滓弃地,二三年后雨水浸淋,精液流入沟麓之中,自然结成皂矾。取而货用,不假煎炼。其中色佳者,人取以混石胆云。

石胆一名胆矾者,亦出晋、隰等州,乃山石穴中自结成者,故绿色带宝光。烧铁器淬于胆矾水中,即成铜色也。

《本草》载矾虽五种,并未分别原委。其昆仑矾状如黑泥,铁矾状如赤石脂者,皆西域产也。

\textbf{○硫黄}

凡硫黄,乃烧石承液而结就。著书者误以焚石为矾石,逐有矾液之说。然烧取硫黄,石半出特生白石,半出煤矿烧矾石,此矾液之说所由混也。

又言中国有温泉处必有硫黄,今东海、广南产硫黄处又无温泉,此因温泉水气似硫黄,故意度言之也。

凡烧硫黄石,与煤矿石同形。掘取其石,用煤炭饼包裹丛架,外筑土作炉。炭与石皆载千斤于内,炉上用烧硫旧渣罨盖,中顶隆起,透一圆孔其中。火力到时,孔内透出黄焰金光。先教陶家烧一钵盂,其盂当中隆起,边弦卷成鱼袋样,覆于孔上。石精感受火神,化出黄光飞走,遇盂掩住不能上飞,则化成汁液靠着盂底,其液流入弦袋之中,其弦又透小眼流入冷道灰槽小池,则凝结而成硫黄矣。

其炭煤矿石浇取皂矾者,当其黄光上走时,仍用此法掩盖以取硫黄。得硫一斤则减去皂矾三十余斤,其矾精华已结硫黄,则枯滓逐为弃物。

凡火药,硫为纯阳,硝为纯阴,两精逼合,成声成变,此乾坤幻出神物也。

硫黄不产北狄,或产而不知炼取亦不可知。至奇炮出于西洋与红夷,则东徂西数万里,皆产硫黄之地也。其琉球土硫黄、广南水硫黄,皆误纪也。

\textbf{○砒石}

凡烧砒霜,质料似土而坚,似石而碎,穴土数尺而取之。江西信郡、河南信阳州皆有砒井,故名信石。近则出产独盛衡阳,一厂有造至万钧者。凡砒石井中,其上常有浊绿水,先绞水尽,然后下凿。砒有红、白两种,各因所出原石色烧成。

凡烧砒,下鞠土窑,纳石其上,上砌曲突,以铁釜倒悬覆突口。其下灼炭举火。其烟气从曲突内熏贴釜上。度其已贴一层厚结寸许,下复息火。待前烟冷定,又举次火,熏贴如前。一釜之内数层已满,然后提下,毁釜而取砒。故今砒底有铁沙,即破釜滓也。凡白砒止此一法。红砒则分金炉内银铜脑气有闪成者。

凡烧砒时,立者必于上风十余丈外,下风所近,草木皆死。烧砒之人经两载即改徙,否则须发尽落。此物生人食过分厘立死。然每岁千万金钱速售不滞者,以晋地菽麦必用拌种,且驱田中黄鼠害,宁、绍郡稻田必用蘸秧根,则丰收也。不然火药与染铜需用能几何哉!

\hypertarget{header-n2619}{%
\subsubsection{膏液}\label{header-n2619}}

宋子曰:天道平分昼夜,而人工继晷以襄事,岂好劳而恶逸哉?使织女燃薪,书生映雪,所济成何事也。草木之实,其中韫藏膏液,而不能自流。假媒水火,冯藉木石,而后倾注而出焉。此人巧聪明,不知于何禀度也。

人间负重致远,恃有舟车。乃车得一铢而辖转,舟得一石而罅完,非此物之为功也不可行矣。至\{艹俎\}蔬之登釜也,莫或膏之,犹啼儿失乳焉。斯其功用一端而已哉?

\textbf{○油品}

凡油供馔食用者,胡麻(一名脂麻)、莱菔子、黄豆、菘菜子(一名白菜)为上,苏麻(形似紫苏,粒大于胡麻)、芸苔子(江南名菜子)次之,茶子(其树高丈余,子如金罂子,去壳取仁)次之,苋菜子次之,大麻仁(粒如胡荽子,剥取其皮,为纟聿索用者)为下。

燃灯则桕仁内水油为上,芸苔次之,亚麻子(陕西所种,俗名壁虱脂麻,气恶不堪食)次之,棉花子次之,胡麻次之,(燃灯最易竭)。桐油与桕混油为下。(桐油毒气熏人,桕油连皮膜则冻结不清。)造烛则桕皮油为上,蓖麻子次之,桕混油每斤入白蜡结冻次之,白蜡结冻诸清油又次之,樟树子油又次之,(其光不减,但有避香气者。)冬青子油又次之。(韶郡专用,嫌其油少,故列次。)北土广用牛油,则为下矣。

凡胡麻与蓖麻子、樟树子,每石得油四十斤。莱菔子每石得油二十七斤。(甘美异常,益人五脏。)芸苔子每石得油三十斤,其耨勤而地沃、榨法精到者,仍得四十斤。(陈历一年,则空内而无油。)茶子每石得油一十五斤。(油味似猪脂,甚美,其枯则止可种火及毒鱼用。)桐子仁每石得油三十三斤。桕子分打时,皮油得二十斤,水油得十五斤,混打时共得三十三斤,(此须绝净者。)冬青子每石得油十二斤。黄豆每石得油九斤。(吴下取油食后,以其饼充豕粮。)菘菜子每石得油三十斤。(油出清如绿水。)棉花子每百斤得油七斤。(初出甚黑浊,澄半月清甚。)苋菜子每石得油三十斤。(味甚甘美,嫌性冷滑。)亚麻、大麻仁每石得油二十余斤。此其大端,其他未穷究试验,与夫一方已试而他方未知者,尚有待云。

\textbf{○法具}

凡取油,榨法而外,有两镬煮取法,以治蓖麻与苏麻。北京有磨法,朝鲜有舂法,以治胡麻。其余则皆从榨出也。凡榨木巨者围必合抱,而中空之。其木樟为上,檀与杞次之。(杞木为者,防地湿,则速朽。)此三木者脉理循环结长,非有纵直文。故竭力挥椎,实尖其中,而两头无璺拆之患,他木有纵文者不可为也。中土江北少合抱木者,则取四根合并为之。铁箍裹定,横拴串合而空其中,以受诸质,则散木有完木之用也。

凡开榨,空中其量随木大小。大者受一石有余,小者受五斗不足。凡开榨,辟中凿划平槽一条,以宛凿入中,削圆上下,下沿凿一小孔,犀刂一小槽,使油出之时流入承藉器中。其平槽约长三四尺,阔三四寸,视其身而为之,无定式也。实槽尖与枋唯檀木、柞子木两者宜为之,他木无望焉。其尖过斤斧而不过刨,盖欲其涩,不欲其滑,惧报转也。撞木与受撞之尖,皆以铁圈裹首,惧披散也。

榨具已整理,则取诸麻菜子入釜,文火慢炒(凡桕、桐之类属树木生者,皆不炒而碾蒸)透出香气,然后碾碎受蒸。凡炒诸麻菜子,宜铸平底锅,深止六寸者,投子仁于内,翻拌最勤。若釜底太深,翻拌疏慢,则火候交伤,减丧油质。炒锅亦斜安灶上,与蒸锅大异。凡碾埋槽土内,(木为者以铁片掩之。)其上以木竿衔铁陀,两人对举而椎之。资本广者则砌石为牛碾,一牛之力可敌十人。亦有不受碾而受磨者,则棉子之类是也。既碾而筛,择粗者再碾,细者则入釜甑受蒸。蒸气腾足,取出以稻秸与麦秸包裹如饼形。其饼外圈箍,或用铁打成,或破篾绞刺而成,与榨中则寸相稳合。

凡油原因气取,有生于无。出甑之时,包裹怠缓,则水火郁蒸之气游走,为此损油。能者疾倾,疾裹而疾箍之,得油之多,诀由于此,榨工有自少至老而不知者。包裹既定,装入榨中,随其量满,挥撞挤轧,而流泉出焉矣。包内油出滓存,名曰枯饼。凡胡麻、莱菔、芸苔诸饼,皆重新碾碎,筛去秸芒,再蒸、再裹而再榨之。初次得油二分,二次得油一分。若桕、桐诸物,则一榨已尽流出,不必再也。

若水煮法,则并用两釜。将蓖麻、苏麻子碾碎,入一釜中,注水滚煎,其上浮沫即油。以杓掠取,倾于干釜内,其下慢火熬干水气,油即成矣。然得油之数毕竟减杀。北磨麻油法,以粗麻布袋捩绞,其法再详。

\textbf{○皮油}

凡皮油造浊法起广信郡,其法取洁净桕子,囫囵入釜甑蒸,蒸后倾入臼内受舂。其臼深约尺五寸,碓以石为身,不用铁嘴,石以深山结而腻者,轻重斫成限四十斤,上嵌衡木之上而舂之。其皮膜上油尽脱骨而纷落,挖起,筛于盘内再蒸,包裹入榨皆同前法。皮油已落尽,其骨为黑子。用冷腻小石磨不惧火者,(此磨亦从信郡深山觅取。)以红火矢围壅锻热,将黑子逐把灌入疾磨。磨破之时,风扇去其黑壳,则其内完全白仁,与梧桐子无异。将此碾蒸,包裹入榨,与前法同。榨出水油清亮无比,贮小盏之中,独根心草燃至天明,盖诸清油所不及者。入食馔即不伤人,恐有忌者,宁不用耳。

其皮油造烛,截苦竹筒两破,水中煮涨,(不然则粘带。)小篾箍勒定,用鹰嘴铁杓挽油灌入,即成一枝。插心于内,顷刻冻结,捋箍开筒而取之。或削棍为模,裁纸一方,卷于其上而成纸筒,灌入亦成一烛。此烛任置风尘中,再经寒暑,不敝坏也。

\hypertarget{header-n2638}{%
\subsubsection{杀青}\label{header-n2638}}

宋子曰:物象精华,乾坤微妙,古传今而华达夷,使后起含生,目授而心识之,承载者以何物哉?君与民通,师将弟命,冯藉占占口语,其与几何?持寸符,握半卷,终事诠旨,风行而冰释焉。覆载之间之藉有楮先生也,圣顽咸嘉赖之矣。身为竹骨为与木皮,杀其青而白乃见,万卷百家基从此起。其精在此,而其粗效于障风、护物之间。事已开于上古,而使汉、晋时人擅名记者,何其陋哉!

\textbf{○纸料}

凡纸质用楮树(一名谷树)皮与桑穰、芙蓉膜等诸物者为皮纸,用竹麻者为竹纸。精者极其洁白,供书文、印文、柬启用;粗者为火纸、包裹纸。所谓``杀青'',以斩竹得名;``汗青''以煮沥得名;``筒''即已成纸名,乃煮竹成筒。后人遂疑削竹片以纪事,而又误疑韦编为皮条穿竹札也。秦火未经时,书籍繁甚,削竹能藏几何?如西番用贝树造成纸叶,中华又疑以贝叶书经典。不知树叶离根即焦,与削竹同一可哂也。

\textbf{○造竹纸}

凡造竹纸,事出南方,而闽省独专其盛。当笋生之后,看视山窝深浅,其竹以将生枝叶者为上料。节界芒种,则登山斫伐。截断五七尺长,就于本山开塘一口,注水其中漂浸。恐塘水有涸时,则用竹枧通引,不断瀑流注入。浸至百日之外,加功槌洗,洗去粗壳与青皮,(是名杀青。)其中竹穰形同苎麻样。用上好石灰化汁涂浆,入皇桶下煮,火以八日八夜为率。

凡煮竹,下锅用径四尺者,锅上泥与石灰捏弦,高阔如广中煮盐牢盆样,中可载水十余石。上盖皇桶,其围丈五尺,其径四尺余。盖定受煮八日已足。歇火一日,揭皇取出竹麻,入清水漂塘之内洗净。其塘底面、四维皆用木板合缝砌完,以防泥污。(造粗纸者不须为此。)洗净,用柴灰浆过,再入釜中,其上按平,平铺稻草灰寸许。桶内水滚沸,即取出别桶之中,仍以灰汁淋下。倘水冷,烧滚再淋。如是十余日,自然臭烂。取出入臼受舂,(山国皆有水碓。)舂至形同泥面,倾入槽内。

凡抄纸槽,上合方斗,尺寸阔狭,槽视帘,帘视纸。竹麻已成,槽内清水浸浮其面三寸许。入纸药水汁于其中,(形同桃竹叶,方语无定名。)则水干自成洁白。凡抄纸帘,用刮磨绝细竹丝编成。展卷张开时,下有纵横架框。两手持帘入水,荡起竹麻入于帘内。厚薄由人手法,轻荡则薄,重荡则厚。竹料浮帘之顷,水从四际淋下槽内。然后覆帘,落纸于板上,叠积千万张。数满则上以板压。俏绳入棍。如榨酒法,使水气净尽流干。然后以轻细铜镊逐张揭起焙干。凡焙纸先以土砖砌成夹巷地面,下以砖盖地面,数块以往,即空一砖。火薪从头穴烧发,火气从砖隙透巷外。砖尽热,湿纸逐张贴上焙干,揭起成帙。

近世阔幅者名大四连,一时书文贵重。其废纸洗去朱墨污秽,浸烂入槽再造,全省从前煮浸之力,依然成纸,耗亦不多。南方竹贱之国,不以为然,北方即寸条片角在地,随手拾取再造,名曰还魂纸。竹与皮,精与细,皆同之也。若火纸、糙纸,斩竹煮麻,灰浆水淋,皆同前法。唯脱帘之后不用烘焙,压水去湿,日晒成干而已。

盛唐时鬼神事繁,以纸钱代焚帛,(北方用切条,名曰板钱。)故造此者名曰火纸。荆楚近俗,有一焚侈至千斤者。此纸十七供冥烧,十三供日用。其最粗而厚者曰包裹纸,则竹麻和宿田晚稻稿所为也。若铅山诸邑所造柬纸,则全用细竹料厚质荡成。以射重价。最上者曰官柬,富贵之家通刺用之。其纸敦厚而无筋膜,染红为吉柬,则先以白矾水染过,后上红花汁云。

\textbf{○造皮纸}

凡楮树取皮,于春末夏初剥取。树已老者,就根伐去,以土盖之。来年再长新条,其皮更美。凡皮纸,楮皮六十斤,仍入绝嫩竹麻四十斤,同塘漂浸,同用石灰浆涂,入釜煮糜。近法省啬者,皮竹十七而外,或入宿田稻稿十三,用药得方,仍成洁白。凡皮料坚固纸。其纵文扯断绵丝,故曰绵纸,衡断且费力。其最上一等,供用大内糊窗格者,曰棂纱纸。此纸自广信郡造,长过七尺,阔过四尺。五色颜料先滴色汁槽内和成,不由后染。其次曰连四纸,连四中最白者曰红上纸。皮名而竹与稻稿参和而成料者,曰揭贴呈文纸。

芙蓉等皮造者统曰小皮纸,在江西则曰中夹纸。河南所造,未详何草木为质,北供帝京,产亦甚广。又桑皮造者曰桑穰纸,极其敦厚,东浙所产,三吴收蚕种者必用之。凡糊雨伞与油扇,皆用小皮纸。

凡造皮纸长阔者,其盛水槽甚宽,巨帘非一人手力所胜,两人对举荡成。若棂纱,则数人方胜其任。凡皮纸供用画幅,先用巩水荡过,则毛茨不起。纸以逼帘者为正面,盖料即成泥浮其上者,粗意犹存也。朝鲜白︴纸,不知用何质料。倭国有造纸不用帘抄者,煮料成糜时,以巨阔青石覆于炕面,其下火,使石发烧。然后用糊刷蘸糜,薄刷石面,居然顷刻成纸一张,一揭而起。其朝鲜用此法与否,不可得知。中国有用此法者亦不可得知也。永嘉蠲糨纸,亦桑穰造。四川薛涛笺,亦芙蓉皮为料煮糜,入芙蓉花末汁。或当时薛涛所指,遂留名至今。其美在色,不在质料也。

\hypertarget{header-n2655}{%
\subsection{下篇}\label{header-n2655}}

\hypertarget{header-n2657}{%
\subsubsection{五金}\label{header-n2657}}

宋子曰:人有十等,自王公至于舆台,缺一焉而人纪不立矣。大地生五金以利用天下与后世,其义亦犹是也。贵者千里一生,促亦五六百里而生;贱者舟车稍艰之国,其土必广生焉。黄金美者,其值去黑铁一万六千倍,然使釜、ň、斤、斧不呈效于日用之间,即得黄金,直高而无民耳。懋迁有无,货居《周官》泉府,万物司命系焉。其分别美恶而指点重轻,孰开其先而使相须于不朽焉?

\textbf{○黄金}

凡黄金为五金之长,熔化成形之后,住世永无变更。白银入洪炉虽无折耗,但火候足时,鼓鞲而金花闪烁,一现即没,再鼓则沉而不现。惟黄金则竭力鼓鞲,一扇一花,愈烈愈现,其质所以贵也。凡中国产金之区,大约百余处,难以枚举。山石中所出,大者名马蹄金,中者名橄榄金、带胯金,小者名瓜子金。水沙中所出,大者名狗头金,小者名\textless{}麦夫\textgreater{}麦金、糠金。平地掘井得者,名面沙金,大者名豆粒金。皆待先淘洗后冶炼而成颗块。

金多出西南,取者穴山至十余丈见伴金石,即可见金。其石褐色,一头如火烧黑状。水金多者出云南金沙江,(古名丽水,)此水源出吐蕃,绕流丽江府,至于北胜州,回环五百余里,出金者有数截。又川北潼川等州邑与湖广沅陵、溆浦等,皆于江沙水中淘沃取金。千百中间有获狗头金一块者,名曰金母,其余皆\textless{}麦夫\textgreater{}麦形。入冶煎炼,初出色浅黄,再炼而后转赤也。儋、崖有金田,金杂沙土之中,不必深求而得,取太频则不复产,经年淘炼,若有则限。然岭南夷獠洞穴中金,初出如黑铁落,深挖数丈得之黑焦石下。初得时咬之柔软,夫匠有吞窃腹中者亦不伤人。河南蔡、矾等州邑,江西乐平、新建等邑,皆平地掘深井取细沙淘炼成,但酬答人功所获亦无几耳。大抵赤县之内隔千里而一生。《岭南录》云居民有从鹅鸭屎中淘出片屑者,或日得一两,或空无所获。此恐妄记也。

凡金质至重,每铜方寸重一两者,银照依其则,寸增重三钱。银方寸重一两者,金照依其则,寸增重二钱。凡金性又柔,可屈折如枝柳。其高下色,分七青、八黄、九紫、十赤。登试金石上,(此石广信郡河中甚多,大者如斗,小者如拳,入鹅汤中一煮,光黑如漆。)立见分明。凡足色金参和伪售者,唯银可入,余物无望焉。欲去银存金,则将其金打成薄片剪碎,每块以土泥裹涂,入坩锅中硼砂熔化,其银即吸入土内,让金流出以成足色。然后入铅少许,另入坩锅内,勾出土内银,亦毫厘具在也。

凡色至于金,为人间华美贵重,故人工成箔而后施之。凡金箔每金七分造方寸金一千片,粘铺物面,可盖纵横三尺。凡造金箔,既成薄片后,包入乌金纸内,竭力挥椎打成。(打金椎,短柄,约重八斤。)凡乌金纸由苏、杭造成,其纸用东海巨竹膜为质。用豆油点灯,闭塞周围,止留针孔通气,熏染烟光而成止纸。每纸一张打金箔五十度,然后弃去,为药铺包朱用,尚末破损,盖人巧造成异物也。凡纸内打成箔后,先用硝熟猫皮绷急为小方板,又铺线香灰撒墁皮上,取出乌金纸内箔覆于其上,钝刀界画成方寸。口中屏息,手执轻杖,唾湿而挑起,夹于小纸之中。以之华物,先以熟漆布地,然后粘贴。(贴字者多用楮树浆。)秦中造皮金者,硝扩羊皮使最薄,贴金其上,以便剪裁服饰用,皆煌煌至色存焉。凡金箔粘物,他日敝弃之时,刮削火化,其金仍藏灰内。滴清油数点,伴落聚底,淘洗入炉,毫厘无恙。

凡假借金色者,杭扇以银箔为质,红花子油刷盖,向火熏成。广南货物以蝉蜕壳调水描画,向火一微炙而就,非真金色也。其金成器物呈分浅淡者,以黄矾涂染,炭火炸炙,即成赤宝色。然风尘逐渐淡去,见火又即还原耳。(黄矾详《燔石》卷。)

\textbf{○银}

凡银中国所出,浙江、福建旧有坑场,国初或采或闭。江西饶、信、瑞三郡有坑从末开。湖广则出辰州,贵州则出铜仁,河南则宜阳赵保山、永宁秋树坡、卢氏高嘴儿、嵩县马槽山,与四川会川密勒山、甘肃大黄山等,皆称美矿。其他难以枚举。然生气有限,每逢开采,数不足则括派以赔偿。法不严则窃争而酿乱,故禁戒不得不苛。燕、齐诸道,则地气寒而石骨薄,不产金、银。然合八省所生,不敌云南之半,故开矿煎银,唯滇中可永行也。

凡云南银矿,楚雄、永昌、大理为最盛,曲靖、姚安次之,镇沅又次之。凡石山硐中有铆砂,其上现磊然小石,微带褐色者,分丫成径路。采者穴土十丈或二十丈,工程不可日月计。寻见土内银苗,然后得礁砂所在。凡樵砂藏深土,如枝分派别,各人随苗分径横挖而寻之。上耆横板架顶,以防崩压。采工篝灯逐径施,得矿方止。凡土内银苗,或有黄色碎石,或土隙石缝有乱丝形状,此即去矿不远矣。凡成银者曰礁,至碎者如砂,其面分丫若枝形者曰铆,其外包环石块曰矿。矿石大者如斗,小者如拳,为弃置无用物。其礁砂形如煤炭,底衬石而不甚黑,其高下有数等。(商民凿穴得砂,先呈官府验辨,然后定税。)出土以斗量,付与冶工,高者六七两一斗,中者三四两,最下一二两。(其礁砂放光甚者,精华泄露,得银偏少。)

凡礁砂入炉,先行拣净淘洗。其炉土筑巨墩,高五尺许,底铺瓷屑、炭灰,每炉受礁砂二石。用栗木炭二百斤,周遭丛架。靠炉砌砖墙一朵,高阔皆丈余。风箱安置墙背,合两三人力,带拽透管通风。用墙以抵炎热,鼓鞲之人方克安身。炭尽之时,以长铁叉添入。风火力到,礁砂溶化成团。此时银隐铅中,尚未出脱,计礁砂二石溶出团约重百斤。

冷定限出,另入分金炉(一名虾蟆炉)内,用松木炭匝围,透一门以辨火色。其炉或施风箱,或使交Ψ。火热功到,铅沉下为底子。(其底已成陀僧样,别入炉炼,又成扁担铅。)频以柳枝从门隙入内燃照,铅气净尽,则世宝凝然成象矣。此初出银,亦名生银。倾定无丝纹,即再经一火,当中止现一点圆星,滇人名曰``茶经''。逮后入铜少许,重以铅力熔化,然后入槽成丝。(丝必倾槽而现,以四围匡住,宝气不横溢走散。)其楚雄所出又异,彼硐砂铅气甚少,向诸郡购铅佐炼。每礁百斤,先坐铅二百斤于炉内,然后煽炼成团。其再入虾蟆炉沉铅结银,则同法也。此世宝所生,更无别出。方书、本草,无端妄想妄注,可厌之甚。

大抵坤元精气,出金之所三百里无银,出银之所三百里无金,造物之情亦大可见。其贱役扫刷泥尘,入水漂淘而煎者,名曰淘厘锱。一日功劳轻者所获三分,重者倍之。其银俱日用剪、斧口中委余,或鞋底粘带布于衢市,或院宇扫屑弃于河沿,其中必有焉,非浅浮土面能生此物也。

凡银为世用,惟红铜与铅两物可杂入成伪。然当其合琐碎而成钣锭,去疵伪而造精纯,高炉火中,坩锅足炼。撒硝少许,而铜、铅尽滞锅底,名曰银锈。其灰池中敲落者,名曰炉底。将锈与底同入分金炉内,填火土甑之中,其铅先化,就低溢流,而铜与粘带余银,用铁条逼就分拨,井然不紊。人工、天工亦见一斑云。炉式并具于后。

\textbf{○附朱砂银}

凡虚伪方士以炉火惑人者,唯朱砂银愚人易惑。其法以投铅、朱砂与白银等分,入罐封固,温养三七日后,砂盗银气,煎成至宝。拣出其银,形有神丧,块然枯物。入铅煎时,逐火轻折,再经数火,毫忽无存。折去砂价、炭资、愚者贪惑犹不解,并志于此。

\textbf{○铜}

凡铜供世用,出山与出炉只有赤铜。以炉甘石或倭铅参和,转色为黄铜,以砒霜等药制炼为白铜;矾、硝等药制炼为青铜;广锡参和为响铜;倭铅和写为铸铜。初质则一味红铜而已。

凡铜坑所在有之。《山海经》言出铜之山四百六十七,或有所考据也。今中国供用者,西自四川、贵州为最盛。东南间自海舶来,湖广武昌、江西广信皆饶铜穴。其衡、瑞等郡,出最下品曰蒙山铜者,或入冶铸混入,不堪升炼成坚质也。

凡出铜山夹土带石,穴凿数丈得之,仍有矿包其外,矿状如姜石,而有铜星,亦名铜璞,煎炼仍有铜流出,不似银矿之为弃物。凡铜砂在矿内,形状不一,或大或小,或光或暗,或如石,或如姜铁。淘洗去土滓,然后入炉煎炼,其熏蒸傍溢者,为自然铜,亦曰石髓铅。

凡铜质有数种。有全体皆铜,不夹铅、银者,洪炉单炼而成。有与铅同体者,其煎炼炉法,傍通高低二孔,铅质先化从上孔流出,铜质后化从下孔流出。东夷铜又有托体银矿内者,入炉炼时,银结于面,铜沉于下。商舶漂入中国,名曰日本铜,其形为方长板条。漳郡人得之,有以炉再炼,取出零银,然后泻成薄饼,如川铜一样货卖者。

凡红铜升黄色为锤锻用者,用自风煤炭(此煤碎如粉,泥糊作饼,不用鼓风,通红则自昼达夜。江西则产袁郡及新喻邑)百斤,灼于炉内,以泥瓦罐载铜十斤,继入炉甘石六斤坐于炉内,自然熔化。后人因炉甘石烟洪飞损,改用倭铅。每红铜六斤,入倭铅四斤,先后入罐熔化,冷定取出,即成黄铜,唯人打造。

凡用铜造响器,用出山广锡无铅气者入内。钲(今名锣)、镯(今名铜鼓)之类,皆红铜八斤,入广锡二斤。铙、钹、铜与锡更加精炼。凡铸器,低者红铜、倭铅均平分两,甚至铅六铜四。高者名三火黄铜、四火熟铜,则铜七而铅三也。

凡造低伪银者,唯本色红铜可入。一受倭铅、砒、矾等气,则永不和合。然铜入银内,使白质顿成红色,洪炉再鼓,则清浊浮沉立分,至于净尽云。

\textbf{○附:倭铅}

凡倭铅古书本无之,乃近世所立名色。其质用炉甘石熬炼而成。繁产山西太行山一带,而荆、衡为次之。每炉甘石十斤,装载入一泥罐内,封裹泥固以渐砑干,勿使见火拆裂。然后逐层用煤炭饼垫盛,其底铺薪,发火煅红,罐中炉甘石熔化成团,冷定毁罐取出。每十耗去其二,即倭铅也。此物无铜收伏,入火即成烟飞去。以其似铅而性猛,故名之曰倭云。

\textbf{○铁}

凡铁场所在有之,其铁浅浮土面,不生深穴,繁生平阳、冈埠,不生峻岭高山。质有土锭、碎砂数种。凡土锭铁,土面浮出黑块,形似枰锤。遥望宛然如铁,之则碎土。若起冶煎炼,浮者拾之,又乘雨湿之后牛耕起土,拾其数寸土内者。耕垦之后,其块逐日生长,愈用不穷。西北甘肃,东南泉郡,皆锭铁之薮也。燕京、遵化与山西平阳,则皆砂铁之薮也。凡砂铁一抛土膜即现其形,取来淘洗,入炉煎炼,熔化之后与锭铁无二也。

凡铁分生、熟,出炉未炒则生,既炒则熟。生熟相和,炼成则钢。凡铁炉用盐做造,和泥砌成。其炉多傍山穴为之,或用巨木匡围,朔造盐泥,穷月之力不容造次。盐泥有罅,尽弃全功。凡铁一炉载土二千余斤,或用硬木柴,或用煤炭,或用木炭,南北各从利便。扇炉风箱必用四人、六人带拽。土化成铁之后,从炉腰孔流出。炉孔先用泥塞。每旦昼六时,一时出铁一陀。既出即叉泥塞,鼓风再熔。

凡造生铁为冶铸用者,就此流成长条、圆块,范内取用。若造熟铁,则生铁流出时相连数尺内,低下数寸筑一方塘,短墙抵之。其铁流入塘内,数人执持柳木棍排立墙上,先以污潮泥晒干,舂筛细罗如面,一人疾手撒扌艳,众人柳棍疾搅,即时炒成熟铁。其柳棍每炒一次,烧折二三寸,再用则又更之。炒过稍冷之时,或有就塘内斩划成方块者,或有提出挥椎打圆后货者。若济阳诸冶,不知出此也。

凡钢铁炼法,用熟铁打成薄片如指头阔,长寸半许,以铁片束尖紧,生铁安置其上,(广南生铁名堕子生钢者妙甚。)又用破草履盖其上,(粘带泥土者,故不速化。)泥涂其底下。洪炉鼓鞲,火力到时,生钢先化,渗淋熟铁之中,两情投合,取出加锤。再炼再锤,不一而足。俗名团钢,亦曰灌钢者是也。

凡倭夷刀剑有百炼精纯、置日光檐下则满室辉曜者,不用生熟相和炼,又名此钢为下乘云。夷人又有以地溲淬刀剑者,(地溲乃石脑油之类,不产中国。)云钢可切玉,亦末之见也。凡铁内有硬处不可打者名铁核,以香油涂之即散。凡产铁之阴,其阳出慈石,第有数处不尽然也。

\textbf{○锡}

凡锡中国偏出西南郡邑,东北寡生。古书名锡为``贺''者,以临贺郡产锡最盛而得名也。今衣被天下者,独广西南丹、河池二州居其十八,衡、永则次之。大理、楚雄即产锡甚盛,道远难致也。

凡锡有山锡、水锡两种。山锡中又有锡瓜、锡砂两种,锡瓜块大如小瓠,锡砂如豆粒,皆穴土不甚深而得之。间或土中生脉充刃,致山土自颓,恣人拾取者。水锡衡、永出溪中,广西则出南丹州河内,其质黑色,粉碎如重罗面。南丹河出者,居民旬前从南淘至北,旬后又从北淘至南。愈经淘取,其砂日长,百年不竭。但一日功劳淘取煎炼不过一斤。会计炉炭资本,所获不多也。南丹山锡出山之阴,其方无水淘洗,则接连百竹为枧,从山阳枧水淘洗土滓,然后入炉。

凡炼煎亦用洪炉,入砂数百斤,丛架木炭亦数百斤,鼓鞲熔化。火力已到,砂不即熔,用铅少许勾引,方始沛然流注。或有用人家炒锡剩灰勾引者。其炉底炭末、瓷灰铺作平地,傍安铁管小槽道,熔时流出炉外低池。其质初出洁白,然过刚,承锤即拆裂。入铅制柔,方充造器用。售者杂铅太多,欲取净则熔化,入醋淬八九度,铅尽化灰而去。出锡唯此道。方书云马齿苋取草锡者,妄言也;谓砒为锡苗者,亦妄言也。

\textbf{○铅}

凡产铅山穴,繁于铜、锡。其质有三种,一出银矿中,包孕白银。初炼和银成团,再炼脱银沉底,曰银矿铅,此铅云南为盛。一出铜矿中,入烘炉炼化,铅先出,铜后随,曰铜山铅,此铅贵州为盛。一出单生铅穴,取者穴山石,挟油灯寻脉,曲折如采银矿,取出淘洗煎炼,名曰草节铅,此铅蜀中嘉、利等州为盛。其余雅州出钓脚铅,形如皂荚子,又如蝌斗子,生山涧沙中。广信郡上饶、饶郡乐平出杂铜铅,剑州出阴平铅,难以枚举。

凡银矿中铅,炼铅成底,炼底复成铅。草节铅单入烘炉煎炼,炉傍通管注入长条土槽内,俗名扁担铅,亦曰出山铅,所以别于凡银炉内频经煎炼者。凡铅物值虽贱,变化殊奇,白粉、黄丹,皆其显像。操银底于精纯,勾锡成其柔软,皆铅力也。

\textbf{○附:胡粉}

凡造胡粉,每铅百斤,熔化,削成薄片,卷作筒,安木甑内。甑下甑中各安醋一瓶,外以盐泥固济,纸糊甑缝。安火四两,养之七日。期足启开,铅片皆生霜粉,扫入水缸内。未生霜者,入甑依旧再养七日,再扫,以质尽为度,其不尽者留作黄丹料。

每扫下霜一斤,入豆粉二两、蛤粉四两,缸内搅匀,澄去清水,用细灰按成沟,纸隔数层,置粉于上。将干,截成瓦定形,或如磊鬼,待干收货。此物古因辰、韶诸郡专造,故曰韶粉(俗误朝粉)。今则各省直饶为之矣。其质入丹青,则白不减。揸妇人颊,能使本色转青。胡粉投入炭炉中,仍还熔化为铅,所谓色尽归皂者。

\textbf{○附:黄丹}

凡炒铅丹,用铅一斤,土硫黄十两,硝石一两。熔铅成汁,下醋点之。滚沸时下硫一块,少顷入硝少许,沸定再点醋,依前渐下硝、黄。待为末,则成丹矣。其胡粉残剩者,用硝石、矾石炒成丹,不复用醋也。欲丹还铅,用葱白汁拌黄丹馒炒,金汁出时,倾出即还铅矣。

\hypertarget{header-n2705}{%
\subsubsection{佳兵}\label{header-n2705}}

宋子曰:兵非圣人之得已也。虞舜在位五十载,而有苗犹弗率。明王圣帝,谁能去兵哉?弧矢之利,以威天下,其来尚矣。为老氏者,有葛天之思焉。其词有曰:``佳兵者,不详之器。''盖言慎也。

火药机械之窍,其先凿自西番与南裔,而后乃及于中国。变幻百出,日盛月新。中国至今日,则即戎者以为第一义,岂其然哉?虽然,生人纵有巧思,乌能至此极也?

\textbf{○弧矢}

凡造弓,以竹与牛角为正中干质,(东北夷无竹,以柔木为之。)桑枝木为两梢。弛则竹为内体,角护其外;张则角向内而竹居外。竹一条而角两接,桑肖则其末刻锲,以受弦区,其本则贯插接笋于竹丫,而光削一面以贴角。

凡造弓,先削竹一片,(竹宜秋冬伐,春夏则朽蛀。)中腰微亚小,两头差大,约长二尺许。一面粘胶靠角,一面铺置牛筋与胶而固之。牛角当中牙接,(北边无修长牛角,则以羊角四接而束之。广弓则黄牛明角亦用,不独水牛也。)固以筋胶。胶外固以桦皮,名曰暖靶。凡桦木关外产辽阳,北土繁生遵化,西陲繁生临洮郡,闽、广、浙亦皆有之。其皮护物,手握如软绵,故弓靶所必用。即刀柄与枪干亦需用之。其最薄者,则为刀剑鞘室也。

凡牛脊梁每只生筋一方条,约重三十两。杀取晒干,复浸水中,析破如苎麻丝。北边无蚕丝,弓弦处皆纠合此物为之。中华则以之铺护弓干,与为棉花弹弓弦也。凡胶乃鱼脬杂肠所为,煎治多属宁国郡,其东海石首鱼,浙中以造白鲞者,取其脬为胶,坚固过于金铁。北边取海鱼脬煎成,坚固与中华无异,种性则别也。天生数物,缺一而良弓不成,非偶然也。

凡造弓初成坯后,安置室中梁阁上,地面勿离火意。促者旬日,多者两月,透干其津液,然后取下磨光,重加筋胶与漆,则其弓良甚。货弓之家,不能俟日足者,则他日解释之患因之。

凡弓弦取食柘叶蚕茧,其丝更坚韧。每条用丝线二十余根作骨,然后用线横缠紧约。缠丝分三停,隔七寸许则空一二分不缠,故弦不张弓时,可折叠三曲而收之。往者北边弓弦,尽以牛筋为质,故夏月雨雾,妨其解脱,不相侵犯。今则丝弦亦广有之。涂弦或用黄蜡,或不用亦无害也。凡弓两肖系区处,或以最厚牛皮,或削柔木如小棋子,钉粘角端,名曰垫弦,义同琴轸。放弦归返时,雄力向内,得此而抗止,不然则受损也。

凡造弓,视人力强弱为轻重,上力挽一百二十斤,过此则为虎力,亦不数出。中力减十之二三,下力及其半。彀满之时皆能中的。但战阵之上洞胸彻札,功必归于挽强者。而下力倘能穿杨贯虱,则以巧胜也。凡试弓力,以足踏弦就地,称钩搭挂弓腰,弦满之时,推移称锤所压,则知多少。其初造料分两,则上力挽强者,角与竹片削就时,约重七两。筋与胶、漆与缠约丝绳,约重八钱。此其大略。中力减十之一二,下力减十之二三也。

凡成弓,藏时最嫌霉湿。(霉气先南后北,岭南谷雨时,江南小满,江北六月,燕、齐七月。然淮、扬霉气独盛。)将士家或置烘厨、烘箱,日以炭火置其下。(春秋雾雨皆然,不但霉气。)小卒无烘厨,则安顿灶突之上。稍怠不勤,立受朽解之患也。(近岁命南方诸省造弓解北,纷纷驳回,不知离火即坏之故,亦无人陈说本章者。)

凡箭,中国南方竹质,北方萑柳质,北边桦质,随方不一。竿长二尺,簇长一寸,其大端也。凡竹箭削竹四条或三条,以胶粘合,过刀光削而圆成之。漆丝缠约两头,名曰"三不齐"箭杆。浙与广南有生成箭竹,不破合者。柳与桦杆,则取彼圆直枝条而为之,微费刮削而成也。凡竹箭其体自直,不用矫揉。木杆则燥时必曲,削造成时以数寸之木,刻槽一条,名曰箭端。将木杆逐寸戛拖而过,其身乃直。即首尾轻重,亦由过端而均停也。

凡箭,其本刻衔口以驾弦,其末受镞。凡镞冶铁为之。(《禹贡》石乃方物,不适用。)北边制如桃叶枪尖,广南黎人矢镞如平面铁铲,中国则三棱锥象也。响箭则以寸木空中锥眼为窍,矢过招风而飞鸣,即《庄子》所谓嚆矢也。凡箭行端斜与疾慢,窍妙皆系本端翎羽之上。箭本近衔处剪翎直贴三条,其长三寸,鼎足安顿,粘以胶,名曰箭羽。(此胶亦忌霉湿,故将卒勤者,箭亦时以火烘。)

羽以雕膀为上,(雕似鹰而大,尾长翅短。)角鹰次之,鸱鹞又次之。南方造箭者,雕无望焉,即鹰、鹞亦难得之货,急用塞数,即以雁翎,甚至鹅翎亦为之矣。凡雕翎箭行疾过鹰、鹞翎,十余步而端正,能抗风吹。北边羽箭多出此料。鹰、鹞翎作法精工,亦恍惚焉。若鹅、雁之质,则释放之时,手不应心,而遇风斜窜者多矣。面箭不及北,由此分也。

\textbf{○弩}

凡弩为守营兵器,不利行阵。直者名身,衡者名翼,弩牙发弦者名机。斫木为身,约长二尺许,身之首横拴度翼。其空缺度翼处,去面刻定一分,(稍厚则弦发不应节。)去背则不论分数。面上微刻直槽一条以盛箭。其翼以柔木一条为者名扁担弩,力最雄。或一木之下加以竹片叠承(其竹一片短一片),名三撑弩,或五撑、七撑而止。身下截刻锲衔弦,其衔傍活钉牙机,上剔发弦。上弦之时唯力是视。一人以脚踏强弩而弦者,《汉书》名曰``蹶张材官''。弦送矢行,其疾无与比数。

凡弩弦以苎麻为质,缠绕以鹅翎,涂以黄蜡。其弦上翼则谨,放下仍松,故鹅翎可扌及首尾于绳内。弩箭羽以箬叶为之。析破箭本,衔于其中而缠约之。其射猛兽药箭,则用草乌一味,熬成浓胶,蘸染矢刃。见血一缕则命即绝,人畜同之。凡弓箭强者行二百余步,弩箭最强者五十步而止,即过咫尺,不能穿鲁缟矣。然其行疾则十倍于弓,而入物之深亦倍之。

国朝军器造神臂弩、克敌弩,皆并发二矢、三矢者。又有诸葛弩,其上刻直槽,相承函十矢,其翼取最柔木为之。另安机木随手扳弦而上,发去一矢,槽中又落一矢,则又扳木上弦而发。机巧虽工,然其力绵甚,所及二十余步而已。此民家妨窃具,非军国器。其山人射猛兽者名曰窝弩,安顿交迹之衢,机傍引线,俟兽过,带发而射之。一发所获,一兽而已。

\textbf{○干}

凡干戈名最古,干与戈相连得名者,后世战卒,短兵驰骑者更用之。盖右手执短刀,左手执干以蔽敌矢。古者车战之上,则有专司执干,并抵同人之受矢者。若双手执长戈与持戟、槊,则无所用之也。凡干长不过三尺,杞柳织成尺径圈置于项下,上出五寸,亦锐其端,下则轻竿可执。若盾名中干,则步卒所持以蔽矢并拒槊者,俗所谓傍牌是也。

\textbf{○火药料}

火药、火器,今时妄想进身博官者,人人张目而道,著书以献,未必尽由试验。然亦粗载数叶,附于卷内。凡火药以硝石、硫黄为主,草木灰为铺。硝性至阴,硫性至阳,阴阳两神物相遇于无隙可容之中。其出也,人物膺之,魂散惊而魄齑粉。凡硝性主直,直击者硝九而硫一。硫性主横,爆击者硝七而硫三。其佐使之灰,则青杨、枯杉、桦根、箬叶、蜀葵、毛竹根、茄秸之类,烧使存性,而其中箬叶为最燥也。

凡火攻有毒火、神火、法火、烂火、喷火。毒火以白砒、砂为君,金汁、银锈、人粪和制。神火以朱砂、雄黄、雌黄为君。烂火以硼砂、磁末、牙皂、秦椒配合。飞火以朱砂、石黄、轻粉、草乌、巴豆配合。劫营火则用桐油、松香。此其大略。其狼粪烟昼黑夜红,迎风直上,与江豚灰能逆风而炽,皆须试见而后详之。

\textbf{○硝石}

凡硝,华夷皆生,中国则专产西北。若东南贩者不给官引,则以为私货而罪之。硝质与盐同母,大地之下潮气蒸成,现于地面。近水而土薄者成盐,近山而土厚者成硝。以其入水即硝熔,故名曰``硝''。长淮以北,节过中秋,即居室之中,隔日扫地,可取少许以供煎炼。

凡硝三所最多:出蜀中者曰川硝,生山西者俗呼盐硝,生山东者俗呼土硝。凡硝刮扫取时,(墙中亦或迸出。)入缸内水浸一宿,秽杂之物浮于面上,掠取去时,然后入釜,注水煎炼。硝化水干,倾于器内,经过一宿,即结成硝。其上浮者曰芒硝,芒长者曰马牙硝,(皆从方产本质幻出。)其下猥杂者曰朴硝。欲去杂还纯,再入水煎炼。入莱菔数枚同煮熟,倾入盆中,经宿结成白雪,则呼盆硝。

凡制火药,牙硝、盆硝功用皆同。凡取硝制药,少者用新瓦焙,多者用土釜焙,潮气一干,即成研末。凡研硝不以铁碾入石臼,相激火生,则祸不可测,凡硝配定何药分两,入黄同研,木灰则从后增入。凡硝既焙之后,经久潮性复生。使用巨泡,多从临期装载也。

\textbf{○硫黄(详见《燔石》卷)}

凡硫黄配硝,而后火药成声。北狄无黄之国,空繁硝产,故中国有严禁,凡燃炮拈硝与木灰为引线,黄不入内,入黄即不透关。凡碾黄难碎,每黄一两,和硝一钱同碾,则立成微尘细末也。

\textbf{○火器}

西洋炮熟铜铸就,圆形若铜鼓。引放时,半里之内,人马受惊死。(平地引炮有关捩,前行遇坎方止。点引之人反走坠入深坑内,炮声在高头,放者方不丧命。)红夷炮铸铁为之,身长丈许,用以守城。中藏铁弹并火药数斗,飞激二里,膺其锋者为齑粉。凡炮引内灼时,先往后坐千钧力,其位须墙抵住,墙崩者其常。

大将军 二将军(即红夷之次,在中国为巨物。) 佛郎机(水战舟头用。)

三眼铳 百子连珠炮

地雷埋伏土中,竹管通引,冲土起击,其身从其炸裂。所谓横击,用黄多者。(引线用矾油,炮口覆以盆。)

混江龙,漆固皮囊炮沉于水底,岸上带索引机。囊中悬吊火石、火镰,索机一动,其中自发。敌舟行过,遇之则败。然此终痴物也。

鸟铳。凡鸟铳长约三尺,铁管载药,嵌盛木棍之中,以便手握。凡锤鸟铳,先以铁梃一条大如箸为冷骨,裹红铁锤成。先为三接,接口炽红,竭力撞合。合后以四棱钢锥如箸大者,透转其中,使极光净,则发药无阻滞。其本近身处,管亦大于末,所以容受火药。每铳约载配消一钱二分,铅铁弹子二钱。发药不用信引,(岭南制度,有用引者。)孔口通内处露消分厘,捶熟苎麻点火。左手握铳对敌,右手发铁机逼苎火于消上,则一发而去。鸟雀遇于三十步内者,羽肉皆粉碎,五十步外方有完形,若百步则铳力竭矣。鸟枪行远过二百步,制方仿佛鸟铳,而身长药多,亦皆倍此也。

万人敌。凡外郡小邑乘城却敌,有炮力不具者,即有空悬火炮而痴重难使者,则万人敌近制随宜可用,不必拘执一方也。盖消、黄火力所射,千军万马立时糜烂。其法:用宿干空中泥团,上留小眼筑实消、黄火药,参入毒火、神火,由人变通增损。贯药安信而后,外以木架匡围,或有即用木桶而塑泥实其内郭者,其义亦同。若泥团必用木匡,所以妨掷投先碎也。敌攻城时,燃灼引信,抛掷城下。火力出腾,八面旋转。旋向内时,则城墙抵信,不伤我兵。旋向外时,则敌人马皆无幸。此为守城第一器。而能通火药之性、火器之方者,聪明由人。作者不上十年,守土者留心可也。

\hypertarget{header-n2745}{%
\subsubsection{丹青}\label{header-n2745}}

宋子曰:斯文千古之不坠也,注玄尚白,其功孰与京哉?离火红而至黑孕其中,水银白而至红呈其变。造化炉锤,思议何所容也。五章遥降,朱临黑而大号彰。万卷横披,墨得朱而天章焕。文房异宝,珠玉何为?至画工肖像万物,或取本姿,或从配合,而色色咸备焉。夫亦依坎附离,而共呈五行变态,非至神孰能与于斯哉?

\textbf{○朱}

凡朱砂、水银、银朱,原同一物,所以异名者,由精细老嫩而分也。上好朱砂出压辰、锦(今名麻阳)与西川者,中即孕Е,然不以升炼。盖光明、箭镞、镜面等砂,其价重于水银三倍,故择出为朱砂货鬻。若以升水,反降贱值。唯粗次朱砂方以升炼水银,而水银又升银朱也。

凡朱砂上品者,穴土十余丈乃得之。始见其苗,磊然白石,谓之朱砂床。近床之砂,有如鸡子大者。其次砂不入药,只为研供画用与升炼水银者。其苗不必白石,其深数丈即得。外床或杂青黄石,或间沙土,土中孕满,则其外沙石多自折裂。此种砂贵州思、印、铜仁等地最繁,而商州、秦州出亦广也。

凡次砂取来,其通坑色带白嫩者,则不以研朱,尽以升Е。若砂质即嫩而烁视欲丹者,则取来时,入巨铁碾槽中,轧碎如微尘,然后入缸,注清水澄浸。过三日夜,跌取其上浮者,倾入别缸,名曰二朱。其下沉结者,晒干即名头朱也。

凡升水银,或用嫩白次砂,或用缸中跌出浮面二朱,水和槎成大盘条,每三十斤入一釜内升Е,其下炭质亦用三十斤。凡升Е,上盖一釜,釜当中留一小孔,釜傍盐泥紧固。釜上用铁打成一曲弓溜管,其管用麻绳缠通梢,仍用盐泥涂固。煅火之时,曲溜一头插入釜中通气,(插处一丝固密。)一头以中罐注水两瓶,插曲溜尾于内,釜中之气在达于罐中之水而止。共煅五个时辰,其中砂末尽化成Е,布于满釜。冷定一日,取出扫下。此最妙玄,化全部天机也。(《本草》胡乱注,``凿地一孔,放碗一个盛水''。)

凡将水银再升朱用,故名曰银朱。其法或用磬口泥罐,或用上下釜。每水银一斤入石亭脂(即硫黄制造者)二斤,同研不见星,炒作青砂头,装于罐内。上用铁盏盖定,盏上压一铁尺。铁线兜底捆缚,盐泥固济口缝,下用三钉插地鼎足盛罐。打火三炷香久,频以废笔蘸水擦盏,则银自成粉,贴于罐上,其贴口者朱更鲜华。冷定揭出,刮扫即用。其石亭脂沉下罐底,可取再用也。每升水银一斤得朱十四两,次朱三两五钱,出数藉硫质而生。

凡升朱与研朱,功用亦相仿。若皇家、贵家画彩,则即同辰、锦丹砂研成者,不用此朱也。凡朱,文房胶成条块,石砚则显,若磨于锡砚之上,则立成皂汁。即漆工以鲜物彩,唯入桐油调则显,入漆亦晦也。凡水银与朱更无他出,其Е海、草Е之说无端狂妄,耳食者信之。若水银已升朱,则不可复还为Е,所谓造化之巧已尽也。

\textbf{○墨}

凡墨烧烟凝质而为之。取桐油、清油、猪油烟为者居十之一,取松烟为者居十之九。凡造贵重墨者,国朝推重徽郡人,或以载油之艰,遣人僦居荆、襄、辰、沅,就其贱值桐油点烟而归。其墨他日登于纸上,日影横射有红光者,则以紫草汁浸染灯心而燃炷者也。

凡油取烟,每油一斤得上烟一两余。手力捷疾者,一人供事灯盏二百付。若刮取怠缓则烟老,火燃质料并丧也。其余寻常用墨,则先将松树流去胶香,然后伐木。凡松香有一毛未净尽,其烟造墨,终有滓结不解之病。凡松树流去香,木根凿一小孔,炷灯缓炙,则通身膏液就暖倾流而出也。

凡烧松烟,伐松斩成尺寸,鞠篾为圆屋如舟中雨篷式,接连十余丈。内外与接口皆以纸及席糊固完成。隔位数节,小孔出烟,其下掩土砌砖先为通烟道路。燃薪数日,歇冷入中扫刮。凡烧松烟,放火通烟,自头彻尾。靠尾一二节者为清烟,取入佳墨为料。中节者为混烟,取为时墨料。若近头一二节,只刮取为烟子,货卖刷印书文家,仍取研细用之。其余则供漆工、垩工之涂玄者。

凡松烟造墨,入水久浸,以浮沉分清悫。其和胶之后,以捶敲多寡分脆坚。其增入珍料与漱金、衔麝,则松烟、油烟增减听人。其余《墨经》、《墨谱》,博物者自详,此不过粗纪质料原因而已。

\textbf{○附}

胡粉(至白色,详《五金》卷。)

黄丹(红黄色,详《五金》卷。)

淀花(至蓝色,详《彰施》卷。)

紫粉(纟辰红色,贵重者用胡粉、银朱对和,粗者用染家红花滓汁为之。)

大青(至青色,详《珠玉》卷。)

铜绿(至绿色,黄铜打成板片,醋涂其上,裹藏糠内,微藉暖火气,逐日刮取。)

石绿(详《珠玉》卷。)

代赭石(殷红色,处处山中有之,以代郡者为最佳。)

石黄(中黄色,外紫色,石皮内黄,一名石中黄子。)

\hypertarget{header-n2772}{%
\subsubsection{曲蘖}\label{header-n2772}}

宋子曰:狱讼日繁,酒流生祸,其源则何辜!祀天追远,沉吟《商倾》、《周雅》之间,若作酒醴之资曲蘖也,殆圣作而明述矣。惟是五谷菁华变幻,得水而凝,感风而化,供用岐黄者神其名,而坚固食羞者丹其色。君臣自古配合日新,眉寿介而宿痼怯,其功不可殚述。自非炎黄作祖、末流聪明,乌能竟其方术哉。

\textbf{○酒母}

凡酿酒必资曲药成信。无曲即佳米珍黍,空造不成。古来曲造酒,蘖造醴,后世厌醴味薄,遂至失传,则并蘖法亦亡。凡曲,麦、米、面随方土造,南北不同,其义则一。凡麦曲,大、小麦皆可用。造者将麦连皮,井水淘净,晒干,时宜盛暑天。磨碎,即以淘麦水和作块,用楮叶包扎,悬风处,或用稻秸罨黄,经四十九日取用。

造面曲用白面五斤、黄豆五升,以蓼汁煮烂,再用辣蓼末五两、杏仁泥十两和踏成饼,楮叶包悬与稻秸罨黄,法亦同前。其用糯米粉与自然蓼汁溲和成饼,生黄收用者,罨法与时日,亦无不同也。其入诸般君臣草药,少者数味,多者百味,则各土各法,亦不可殚述。近代燕京,则以薏苡仁为君,入曲造薏酒。浙中宁、绍则以绿豆为君,入曲造豆酒。二酒颇擅天下佳雄。(别载《酒经》。)

凡造酒母家,生黄未足,视候不勤,盥拭不洁,则疵药数丸动辄败人石米。故市曲之家必信著名闻,而后不负酿者。凡燕、齐黄酒曲药,多从淮郡造成,载于舟车北市。南方曲酒,酿出即成红色者,用曲与淮郡所造相同,统名大曲。但淮郡市者打成砖片,而南方则用饼团。其曲一味,蓼身为气脉,而米、麦为质料,但必用已成曲、酒糟为媒合。此糟不知相承起自何代,犹之烧矾之必用旧矾滓云。

\textbf{○神曲}

凡造神曲所以入药,乃医家别于酒母者。法起唐时,其曲不通酿用也。造者专用白面,每百斤入青蒿自然汁、马蓼、苍耳自然汁相和作饼,麻叶或楮叶包罨如造酱黄法。待生黄衣,即晒收之。其用他药配合,则听好医者增入,苦无定方也。

\textbf{○丹曲}

凡丹曲一种,法出近代。其义臭腐神奇,其法气精变化。世间鱼肉最朽腐物,而此物薄施涂抹,能固其质于炎暑之中,经历旬日蛆蝇不敢近,色味不离初,盖奇药也。

凡造法用灿稻米,不拘早晚。舂杵极其精细,水浸一七日,其气臭恶不可闻,则取入长流河水漂净。(必用山河流水,大江者不可用。)漂后恶臭犹不可解,入甑蒸饭则转成香气,其香芬甚。凡蒸此米成饭,初一蒸半生即止,不及其熟。出离釜中,以冷水一沃,气冷再蒸,则令极熟矣。熟后,数石共积一堆拌信。

凡曲信必用绝佳红酒糟为料,每糟一斗入马蓼自然汁三升,明矾水和化。每曲饭一石入信二斤,乘饭热时,数人捷手拌匀,初热拌至冷。候视曲信入饭,久复微温,则信至矣。凡饭拌信后,倾入箩内,过矾水一次,然后分散入篾盘,登架乘风。后此风力为政,水火无功。

凡曲饭入盘,每盘约载五升。其屋室宜高大,防瓦上暑气侵逼。室面宜向南,防西晒。一个时中翻拌约三次。候视者七日之中,即坐卧盘架之下,眠不敢安,中宵数起。其初时雪白色,经一二日成至黑色。黑转褐,褐转赭,赭转红,红极复转微黄。目击风中变幻,名曰生黄曲,则其价与入物之力皆倍于凡曲也。凡黑色转褐,褐转红,皆过水一度。红则不复入水。凡造此物,曲工盥手与洗净盘簟,皆令极洁。一毫滓秽,则败乃事也。

\hypertarget{header-n2788}{%
\subsubsection{珠玉}\label{header-n2788}}

宋子曰:玉韫山辉,珠涵水媚,此理诚然乎哉,抑意逆之说也?大凡天地生物,光明者昏浊之反,滋润者枯涩之仇,贵在此则贱在彼矣。合浦、于阗行程相去二万里,珠雄于此,玉峙于彼,无胫而来,以宠爱人寰之中,而辉煌廊庙之上,使中华无端宝藏折节而推上坐焉。岂中国辉山、媚水者,萃在人身,而天地菁华止有此数哉?

\textbf{○珠}

凡珍珠必产蚌腹,映月成胎,经年最久,乃为至宝。其云蛇蝮、龙颔、鲛皮有珠者,妄也。凡中国珠必产雷、廉二池。三代以前,淮扬亦南国地,得珠稍近《禹贡》``淮夷珠'',或后互市之便,非必责其土产也。金采蒲里路,元采杨村直沽口,皆传记相承之妄,何尝得珠。至云忽吕古江出珠,则夷地,非中国也。

凡蚌孕珠,乃无质而生质。他物形小而居水族者,吞噬弘多,寿以不永。蚌则环包坚甲,无隙可投,即吞腹,囫囵不能消化,故独得百年千年,成就无价之宝也。凡蚌孕珠,即千仞水底,一逢圆月中天,即开甲仰照,取月精以成其魄。中秋月明,则老蚌犹喜甚。若彻晓无云,则随月东升西没,转侧其身而映照之。他海滨无珠者,潮汐震撼,蚌无安身静存之地也。

凡廉州池自乌泥、独揽沙至于青鸾,可百八十里。雷州池自对乐岛斜望石城界,可百五十里。户采珠每岁必以三月,时牲杀祭海神,极其虔敬。户生啖海腥,入水能视水色,知蛟龙所在,则不敢侵犯。

凡采珠舶,其制视他舟横阔而圆,多载草荐于上。经过水漩,则掷荐投之,舟乃无恙。舟中以长绳系没人腰,携篮投水。凡没人以锡造弯环空管,其本缺处对掩没人口鼻,令舒透呼吸于中,别以熟皮包络耳项之际。极深者至四五百尺,拾蚌篮中。气逼则撼绳,其上急提引上,无命者或葬鱼腹。凡没人出水,煮热毳急覆之,缓则寒栗死。

宋朝李招讨设法以为耩,最后木柱扳口,两角坠石,用麻绳作兜如囊状。绳系舶两傍,乘风扬帆而兜取之,然亦有漂溺之患。今户两法并用之。

凡珠在蚌,如玉在璞。初不识其贵贱,剖取而识之。自五分至一寸一分经者为大品。小平似覆釜,一边光彩微似镀金者,此名珠,其值一颗千金矣。古来``明月''、``夜光'',即此便是。白昼晴明,檐下看有光一线闪烁不定,``夜光''乃其美号,非真有昏夜放光之珠也。次则走珠,置平底盘中,圆转无定歇,价亦与珠相仿。(化者之身受含一粒,则不复朽坏,故帝王之家重价购此。)次则滑珠,色光而形不甚圆。次则累珠,次官雨珠,次税珠,次葱符珠。幼珠如粱粟,常珠如豌豆。卑而碎者曰玑。自夜光至于碎玑,譬均一人身而王公至于氓隶也。

凡珠生止有此数,采取太频,则其生不继。经数十年不采,则蚌乃安其身,繁其子孙而广孕宝质。所谓珠徙珠还,此煞定死谱,非真有清官感召也。(我朝弘治中,一采得二万八千两。万历中,一采止得三千两,不偿所费。)

\textbf{○宝}

凡宝石皆出井中,西番诸域最盛,中国惟出云南金齿卫与丽江两处。凡宝石自大至小,皆有石床包其外,如玉之有璞。金银必积土其上,韫结乃成,而宝则不然,从井底直透上空,取日精月华之气而就,故生质有光明。如玉产峻湍,珠孕水底,其义一也。

凡产宝之井即极深无水,此乾坤派设机关。但其中宝气如雾,氤氲井中,人久食其气多致死。故采宝之人,或结十数为群,入井者得其半,而井上众人共得其半也。下井人以长绳系腰,腰带叉口袋两条,及泉近宝石,随手疾拾入袋。(宝井内不容蛇虫。)腰带一巨铃,宝气逼不得过,则急摇其铃,井上人引ㄌ提上,其人即无恙,然已昏瞢。止与白滚汤入口解散,三日之内不得进食粮,然后调理平复。其袋内石,大者如碗,中者如拳,小者如豆,总不晓其中何等色。付与琢工钅虑错解开,然后知其为何等色也。

属红黄种类者,为猫精、羯芽、星汉砂、琥珀、木难、酒黄、喇子。猫精黄而微带红。琥珀最贵者名曰\{玉\}(音依,此值黄金五倍价,)红而微带黑,然昼见则黑,灯光下则红甚也。木难纯黄色,喇子纯红。前代何妄人,于松树注茯苓,又注琥珀,可笑也。

属青绿种类者,为瑟瑟珠、且母绿、鸦鹘石、空青之类。(空青既取内质,其膜升打为空青。)至玫瑰一种如黄豆、绿豆大者,则红、碧、青、黄数色皆具。宝石有玫瑰,如珠之有玑也。星汉砂以上,犹有煮海金丹。此等皆西番产,亦间气出。滇中井所无。

时人伪造者,唯琥珀易假。高者煮化硫黄,低者以殷红汁料煮入牛羊明角,映照红赤隐然,今亦最易辨认。(琥珀磨之有浆。)至引灯草,原惑人之说,凡物借人气能引拾轻芥也。自来《本草》陋妄,删去毋使灾木。

\textbf{○玉}

凡玉入中国,贵重用者尽出于阗、(汉时西国号,后代或名别失八里,或统服赤斤蒙古,定名未详。)葱岭。所谓蓝田,即葱岭出玉别地名,而后世误以为西安之蓝田也。其岭水发源名阿耨山,至葱岭分界两河,一曰白玉河,一曰绿玉河。后晋人高居海作《于阗国行程记》载有乌玉河,此节则妄也。

玉璞不藏深土,源泉峻急激映而生。然取者不于所生处,以急湍无着手。俟其夏月水涨,璞随湍流徙,或百里,或二三百里,取之河中。凡玉映月精光而生,故国人沿河取玉者,多于秋间明月夜,望河候视。玉璞堆聚处,其月色倍明亮。凡璞随水流,仍错杂乱石浅流之中,提出辨认而后知也。

白玉河流向东南,绿玉河流向西北。亦力把力地,其地有名望野者,河水多聚玉。其俗以女人赤身没水而取者,云阴气相召,则玉留不逝,易于捞取,此或夷人之愚也。(夷中不贵此物,更流数百里,途远莫货,则弃而不用。)

凡玉唯白与绿两色。绿者中国名菜玉。其赤玉、黄玉之说,皆奇石、琅之类,价即不下于玉,然非玉也。凡玉璞根系山石流水,未推出位时,璞中玉软如棉絮,推出位时则已硬,入尘见风则愈硬。谓世间琢磨有软玉,则又非也。凡璞藏玉,其外者曰玉皮,取为砚托之类,其值无几。璞中之玉有纵横尺余无瑕玷者,古者帝王取以为玺。所谓连城之璧,亦不易得。其纵横五六寸无瑕者,治以为杯,此亦当世重宝也。

此外惟西洋琐里有异玉,平时白色,晴日下看映出红色。阴雨时又为青色,此可谓之玉妖,尚方有之。朝鲜西北太尉山有千年璞,中藏羊脂玉,与葱岭美者无殊异。其他虽有载志,闻见则未经也。凡玉由彼地缠头回,(其俗人首一岁裹布一层,老则臃肿之甚,故名缠头回子。其国王亦谨不见发。问其故,则云见发则岁凶荒,可笑之甚。)或溯河舟,或驾橐驼,经庄浪入嘉峪,而至于甘州与肃州。中国贩玉者,至此互市而得之,东入中华,卸萃燕京。玉工辨璞高下定价,而后琢之。(良玉虽集京师,工巧则推苏郡。)

凡玉初剖时,冶铁为圆盘,以盆水盛沙,足踏圆盘使转,添沙剖玉,逐忽划断。中国解玉沙,出顺天玉田与真定邢台两邑,其沙非出河中,有泉流出,精粹如面,藉以攻玉,永无耗折。既解之后,别施精巧工夫,得镔铁刀者,则为利器也。(镔铁亦出西番哈密卫砺石中,剖之乃得。)

凡玉器琢余碎,取入钿花用。又碎不堪者,碾筛和灰涂琴瑟,琴有玉音,以此故也。凡镂刻绝细处,难施锥刃者,以蟾酥填画而后锲之。物理制服,殆不可晓。凡假玉以充者,如锡之于银,昭然易辨。近则捣舂上料白瓷器,细过微尘,以白敛诸汁调成为器,干燥玉色烨然,此伪最巧云。

凡珠玉、金银,胎性相反。金银受日精,必沉埋深土结成。珠玉、宝石受月华,不受土寸掩盖。宝石在井上透碧空,珠在重渊,玉在峻滩,但受空明、水色盖上。珠有螺城,螺母居中,龙神守护,人不敢犯。数应入世用者,螺母推出人取。玉初孕处,亦不可得。玉神推徙入河,然后恣取,与珠宫同神异云。

\textbf{○附:玛瑙 水晶 琉璃}

凡玛瑙非石非玉,中国产处颇多,种类以十余计。得者多为簪\{度\}、钩(音扣)结之类,或为棋子,最大者为屏风及棹面。上品者产宁夏外徼羌地砂碛中,然中国即广有,商贩者亦不远涉也。今京师货者多是大同、蔚州九空山、宣府四角山所产,有夹胎玛瑙、截子玛瑙、锦红玛瑙,是不一类。而神木、府谷出浆水玛瑙、锦缠玛瑙,随方货鬻,此其大端云。试法以砑木不热者为真。伪者虽易为,然真者值原不甚贵,故不乐售其技也。

凡中国产水晶,视玛瑙少杀,今南方用者多福建漳浦产(山名铜山,)北方用者多宣府黄尖山产,中土用者多河南信阳州(黑色者最美)与湖广兴国州(潘家山)产,黑色者产北不产南。其他山穴本有之而采识未到,与已经采识而官司厉禁封闭(如广信惧中官开采之类)者尚多也。凡水晶出深山穴内瀑流石罅之中,其水经晶流出,昼夜不断,流出洞门半里许,其面尚如油珠滚沸。凡水晶未离穴时如棉软,见风方坚硬。琢工得宜者,就山穴成粗坯,然后持归加功,省力十倍云。

凡琉璃石,与中国水精、占城火齐其类相同,同一精光明透之义。然不产中国,产于西域。其石五色皆具,中华人艳之,遂竭人巧以肖之。于是烧瓴\textless{}商瓦\textgreater{}转釉成黄绿色者曰琉璃瓦。煎化羊角为盛油与笼烛者为琉璃碗。合化硝、铅写珠铜线穿合者为琉璃灯。捏片为琉璃瓶袋。(硝用煎炼上结马牙者。)各色颜料汁任从点染。凡为灯、珠皆淮北齐地人,以其地产硝之故。

凡硝见火还空,其质本无,而黑铅为重质之物。两物假火为媒,硝欲引铅还空,铅欲留硝住世,和同一釜之中,透出光明形象。此乾坤造化隐现于容易地面。《天工》卷末,著而出之。

\end{document}
