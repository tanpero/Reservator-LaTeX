\PassOptionsToPackage{unicode=true}{hyperref} % options for packages loaded elsewhere
\PassOptionsToPackage{hyphens}{url}
%
\documentclass[]{article}
\usepackage{lmodern}
\usepackage{amssymb,amsmath}
\usepackage{ifxetex,ifluatex}
\usepackage{fixltx2e} % provides \textsubscript
\ifnum 0\ifxetex 1\fi\ifluatex 1\fi=0 % if pdftex
  \usepackage[T1]{fontenc}
  \usepackage[utf8]{inputenc}
  \usepackage{textcomp} % provides euro and other symbols
\else % if luatex or xelatex
  \usepackage{unicode-math}
  \defaultfontfeatures{Ligatures=TeX,Scale=MatchLowercase}
\fi
% use upquote if available, for straight quotes in verbatim environments
\IfFileExists{upquote.sty}{\usepackage{upquote}}{}
% use microtype if available
\IfFileExists{microtype.sty}{%
\usepackage[]{microtype}
\UseMicrotypeSet[protrusion]{basicmath} % disable protrusion for tt fonts
}{}
\IfFileExists{parskip.sty}{%
\usepackage{parskip}
}{% else
\setlength{\parindent}{0pt}
\setlength{\parskip}{6pt plus 2pt minus 1pt}
}
\usepackage{hyperref}
\hypersetup{
            pdfborder={0 0 0},
            breaklinks=true}
\urlstyle{same}  % don't use monospace font for urls
\setlength{\emergencystretch}{3em}  % prevent overfull lines
\providecommand{\tightlist}{%
  \setlength{\itemsep}{0pt}\setlength{\parskip}{0pt}}
\setcounter{secnumdepth}{0}
% Redefines (sub)paragraphs to behave more like sections
\ifx\paragraph\undefined\else
\let\oldparagraph\paragraph
\renewcommand{\paragraph}[1]{\oldparagraph{#1}\mbox{}}
\fi
\ifx\subparagraph\undefined\else
\let\oldsubparagraph\subparagraph
\renewcommand{\subparagraph}[1]{\oldsubparagraph{#1}\mbox{}}
\fi

% set default figure placement to htbp
\makeatletter
\def\fps@figure{htbp}
\makeatother


\date{}

\begin{document}

\hypertarget{header-n0}{%
\section{明季南略}\label{header-n0}}

\begin{center}\rule{0.5\linewidth}{\linethickness}\end{center}

\tableofcontents

\begin{center}\rule{0.5\linewidth}{\linethickness}\end{center}

\hypertarget{header-n10}{%
\subsection{序}\label{header-n10}}

呜呼!有明自南渡以后,小朝廷事难言之矣。当时北都倾覆,海内震惊;即薪胆弥厉,未知终始。乃马、阮之徒犹贿赂公行,处堂自喜;不逾载而金瓯尽缺,罪胜诛哉!唐藩起闽中,势如危卵;而郑氏以骄奢贪纵,转之日与鲁藩为难,唇亡齿寒之义谓何?桂藩立粤东,僻处海隅;一逼于成栋、再逼于三王、三逼于孙可望,遁走不常,舟居靡定,是时君不君、国不国矣;虽有瞿桂林留守四载,无济时艰。至于杜允和、李定国辈,益难支矣。若成功、煌言出没风涛,徒扰民耳;亦何益乎!

岁辛亥仲夏,予编``南略''一书;始于甲申五月、止于康熙乙巳,凡二十余年事,分十八卷。虽叙次不伦、见闻各异,而笔之所至,雅俗兼收,有明之微绪余烬,皆毕于是矣。嗟嗟!祸乱之作,天之所以开皇清也;岂人力欤!爰是识数言于左。

康熙十年(辛亥)季冬八日(乙酉),无锡计六奇题于社王氏之书斋。

\hypertarget{header-n17}{%
\subsection{卷一·南都甲乙纪}\label{header-n17}}

福王本末

福嗣王讳由崧,神宗之孙、光宗之侄、思宗之兄也;建号宏光。乙酉,南都陷,北奔。浙东鲁藩监国,谥为赧皇帝;及闽中唐王立,遥上尊号为圣安皇帝;永明王立,谥为安宗简皇帝。而我朝则削其年号,止称福藩而已。王之父讳常洵,郑贵妃所出;神宗第二子。万历四十二年,封藩河南府。崇祯十有四年正月,流贼李自成破河南,福王遇害,世子逾城免。十七年二月初三日(壬戌),怀庆府夜变;同母邹氏走出东门,弃母兵间,狼狈走卫辉府依潞王。

纪云:`福藩后奏``王宝''实系无存;盖为世子时自称窃以送贼者'。

``甲乙史''云:`三月,福、周、潞、崇四王各弃藩南走;此初四日也。十八日,寓淮安湖嘴杜光绍园。二十九日,淮上始传京师陷。四月二十七日,百司公启迎王于仪真。三十日,南京诸臣见王于舟次'。
五月纪

初一戊子朔,王自三山门登陆。至孝陵,乘马从西门入。

初二(己丑),诸臣谒王于行宫。

初三(庚寅),百官朝服,王行告天礼;祝文飘入云霄,众以为异。魏国公徐宏基进``监国之宝''。

初五(壬辰),以张应元为承天总兵。

初六(癸巳),河北、山东府州县各杀贼所署伪官,咸称起义。居庸巡抚何谦自北亡命过德州,济王留之共事;寻送之南行。临清铺商留阁部监纪凌駉起义,旧侍郎张凤翔亦起义东昌。

初七(甲午),史可法议防江:设水师五万,添二镇将,画地分守;仍以文臣操江协事。衡王杀伪官于青州。成国勋卫宋元官浦口渡江,自言杂担夫出京来者。杨士聪家眷出北城,门生方大猷以家丁护送(大猷,蓟州监事;随吴三桂降大清,令守通州也)。

初八(乙未),江南抚郑瑄奏报:`江北刘泽清兵连骑数万,皆欲渡江;三吴百姓呼吸变乱。臣驻师于江,遗书高、刘二帅,不肯止兵;请敕操江武臣速援京口'。凤阳参将戈士凯报:`泽清兵沿途杀劫,逼攻临清';敕御史祁彪佳等分行安抚。楚督袁继咸请入觐,止之。起刘宗周左都御史。

初九(丙申),瑞王避兵入重庆;奏闻。

初十(丁酉),楚抚何志坚奏:`鄂岳恢复,方国安冒功混报'。又奏:`左良玉复德、随'。户科罗万象劾方孔照屯抚河北,寇至踉跄遁归;又蒙面补官。

十一(戊戌),奠安帝后御容,遣太监韩赞周、卢九德行礼;奠安二祖御容,遣魏国公徐宏基、安远侯柳祚冒等行礼。尚书张慎言陈十议。命赵光远镇守四川。贵省民何兆仰作乱。吴中士民焚掠仕贼官项煜、钱位坤、宋学显、汤有庆四家。群臣三次劝进。

十二(己亥),史可法请增文武重臣经理招讨。济宁乡官潘士良约回兵入城,杀伪将伪道兵回。杨科奏藩为总河而自为总兵。

十五(壬寅),王即帝位于武英殿;诏以明年为宏光元年。仕贼臣项煜自北逃归,混入朝班。

十六(癸卯),故御史汪承诏自言伪政府点用,坚拒南奔。命马士英掌兵部,仍入直佐理。史可法自请督师江北,以避士英。

十七(甲辰),伪将刘暴随伪镇董学礼出抚敕五道送高杰、黄得功、刘伊盛、刘肇基、徐大受;得功执以闻。

十八(乙巳),史可法辞朝。通政刘士祯请严封驳参治之令;时行宫前章奏杂投,御史朱国昌亦言班制宜肃。祭先恭王太妃于行宫。进封黄得功、左良玉为侯,高杰、刘泽清、刘良佐为伯。史可法请发铜甲、铜锅、倭刀、团牌、红衣炮并色绢、白布一应军需诣户部即给。

十九(丙午),史可法请以刘肇基、于永绶、李栖凤、卜从善、金声桓随征,俱隶标下。马士英奏大计四款:一、圣母流离,可密谕高杰部下将卫迎;一、皇考追尊位号,迁梓宫南来;一、皇子未生,即敕慎选淑女;一、诸藩失国,恐有奸宄挟之,不利社稷,宜迎置京师。

二十(丁未),刘孔昭言:`封疆失守各官,不在``逆案''之例;吏部毋得混推'。

二十一(戊申),礼部请补历官。

二十二(己酉),令应天府祈雨。

二十三(庚戌),早朝毕,刘孔昭大骂张慎言;欲逐去之。
二十五(壬子),淮抚路振飞颁``登极诏书''、``国政二十五款''于民间。常熟土民焚掠仕贼官时敏家,三代四棺俱劈毁。

二十七(甲寅),命部司清查十七年练饷尽数起解;明年全免。

二十八(乙卯),马士英奏吴三桂之捷,命封蓟国公,世袭;户部发银五万两、米十万担,责令沈廷扬送与之。御史陈良弼言:`科臣李沾荐人调停,从来误国宿套'。

张慎言上疏求去。侍郎贺世寿言:`今日更化善治,莫若肃纪纲而慎刑赏。口头报国、河上拥兵,恩数已盈;功名不立,人主轻此名器矣。至于草泽语难,实繁有徒。未见兵勇杀贼,但见兵来虐民;小民不恨贼而恨兵,甘心合顺而从逆。不肖有司,日刑剥其民:而求为保障,必不可得'!

二十九(丙辰),御史朱国昌论山东督抚邱祖德轻弃地方。以陈子壮为礼部尚书,徐汧、吴伟业少詹,管绍宁詹事,陈盟右庶子。

文震亨``实录''云:`初三日,传百官止服青锦绣朝拜。百官以监国典礼重大,俱朝服。礼毕,即以张慎言为吏部尚书,传旨会推阁员。疏上,先用史可法进东阁大学士,兼兵部尚书如故,高宏图礼部尚书,进东阁大学士,即入阁办事。召工部侍郎周堪赓为户部尚书;马士英进东阁大学士兼兵部尚书右都御史,总督凤阳如故;姜曰广、王铎俱进东阁大学士'。

六月纪

初一丁巳朔,大学士高宏图请暂辍阁务,督收漕粮江上;许之。禁讹言、匿名揭帖。允马士英言,淮扬增兵三万。上大行皇帝尊谥曰烈皇帝,庙号思宗;皇后尊谥曰烈皇后。

初二(戊午),命铸金玺代玉。前巡抚王永祚遵旨就逮,下刑部。

初三(己未),旧大学士蒋德璟北归,奏贺;尚书张国维在途,入贺。德安王侨居广陵。

初四(庚申),夏允彝、余飏、严锡命、文德翼补吏部郎。

初五(辛酉),马士英奏北信诛伪功,命加黎玉田兵部尚书、卢世漼太仆卿、旧辅谢升上柱国(时讹传谢陛为谢升也)。

初六(壬戌),起钱谦益为礼部尚书协理詹事,杜宏域提督大教场,杨振宗安庆总兵。马士英荐``案逆''阮大铖,命来京陛见。

初七(癸亥),赵光远提督川、陕。工科李清疏请谥陶安、方孝孺及蒋钦、李应升等;从之。

初八(甲子),史可法奏:`扬州悍民惨杀乡绅郑元勋'。吉王子慈煃报:`吉王播迁而薨'。命护送潞王于杭州。工部尚书程注致仕。命逮治从逆诸臣光时亨、周钟等。

初九(乙丑),刘泽清、高杰等公举陈淇范仍以原官驻瓜州、泰兴。原任侍郎吴履中自理。惠王寓于肇庆。

初十(丙寅),张慎言致仕。侍郎张有誉到任。礼部请立中官;诏以国雠未报,不许。马士英荐起张捷。

十三(己巳),鲁王泊舟京口,请附京简僻地方安顿。顾锡畴言:`大祀莫如郊社,合祀、分祀后先互异。但仪礼于今,物力告匮,当删繁就简,禀从高皇合祀之制为便'。

十四(庚午),御史朱国昌劾在逃巡抚郭景昌泊舟清凉门外欺饰辩疏,且论抚楚、抚晋种种恶孽;命御史驱逐。释高墙罪宗前唐王聿键等七十五案,凡三百四十一人。

十五(辛未),蜀王告急。户科罗万象奏``惊见内员催征''。先是,命太监王肇基督催闽、浙金花银。肇基名坤,即崇祯时肆恶于淮、扬者。高宏图以方争阮大铖事,不便执奏,请身往督催;因过肇基言之,肇基悟,即上疏辞止。

十六(壬申),詹事管绍宁请遴内阁诰敕房官,各以赀纳授。

十七(癸酉),吕大器引疾去;顾锡畴署吏部印。詹兆恒进``钦定逆案''。

十八(甲戌),蒋德璟疏辞内召。

十九(乙亥),旧兵部侍郎徐人龙自请除用。

二十一(丁丑),左懋第疏请北行。

二十三(己卯),赵之龙纠高宏图议思宗庙号之失,请改正;诏仍旧。黄澍奏:`王聚奎弃数千里之地逃回至省,惟日催赃罚'。

二十五(辛巳),诏迎母后邹氏。

二十六(壬午),史可法奏报扬州已安;特奖慰之。何楷户右侍郎;程世昌佥都,抚应天。吕大器辞朝奏谢,谕以``挑激''二字勿言。通政使刘士桢参监生陆浚源为兄奏辨,词牵国本三案。祁彪佳请留漕米十万担贮镇江。巡按御史王燮奏皇太子、定王、永王俱遇害;即以燮为都察院右佥都,巡抚山东。

二十九(乙酉),给募兵御史陈荩``令''字牌。

七月纪

初一丙戌〔朔〕,选郎倪嘉庆改户科。命鲁王暂驻处州、崇王处台州。命选净身男子。

初二(丁亥),起张采仪制主事、陈龙正祠祭员外郎。旧辅孔贞运卒。

初三(戊子),追尊皇考福恭王为恭皇帝、妣姚氏为皇太后。

初五(庚寅),命考选科道中行评博推知各减俸。行取知县杨文骢自荐边材(马士英甥婿也)。左懋第经理河北、关东军务;马绍瑜为太常寺少卿,加陈洪范太子太保,齎白金十万两、金千万、缎绢万匹偕使大清。

初六(辛卯),蒋德璟献``中兴三策'';上嘉纳之。疏辞召用,加恩礼予归。高宏图、姜曰广奉旨迎太后。

初七(壬辰),惠、桂二王驻广西,鲁、潞、周、崇四王驻浙东。

初八(癸巳),刘子渤佥都,抚四川;范矿佥都,抚贵州。御史米寿图按四川。礼部尚书顾锡畴请谥文震孟、姚希孟、罗喻义、吕维祺,又请削温体仁谥;从之。

初九(甲午),发十万米给山东抚镇。定从逆诸臣六等罪。

十三(戊戌),抚宁侯朱国弼以不预会推冢臣,疏争非制;上谕;`出何会典'?

十四(己亥),魏国公徐宏基、抚宁侯朱国弼、安远侯柳祚昌、灵璧侯汤国祚、忭城伯赵之龙、东宁伯焦梦熊、南和伯方一元、诚意伯刘孔昭、成安伯郭祚永各进彩缎恭贺;上命该衙门察收。

十五(庚子),上诞日,百官朝贺。驾出内官监,服黄袍,十六校尉抬棕轿,进坐武英殿;文武朝见庆贺毕,仍回内官监。以开封推官陈潜夫为御史,巡按河南。改黔督为抚;设川黔云广总督,镇荆、襄。

十六(辛丑),吏部尚书徐石麟到任。朱国弼、刘孔昭条陈新政:一、吏部用人,必勋臣商确;一、各部行政,必勋臣面定;一、呈上图治,必勋臣召对。兵科陈子龙纠庄应会督漕狼籍。

二十(乙巳),用御史郑友元言,削夺温体仁、周延儒、薛国观官衔荫子,以为作奸不忠之戒。

二十三(戊申),朱国弼、刘孔昭各请增设家丁营将,祈户部给粮。

二十四(己酉),刘孔昭荐举循良卓异;内有冯大任,即户科所参赃私狼戾者。

二十六(辛亥),尽释高墙罪宗为庶人。命经筵择吉,钱谦益、管绍宁、陈监充讲官。苏按周一敬请表故举人张世伟、顾云鸿学行以风世;诏可。

是月,黄元吉奏大清兵南征。

八月纪

初一丙辰朔,日有食之。命锦衣冯可宗遣役缉事。

初二(丁巳),亲祀孔子。

初三(戊午),以杨鹗为兵侍郎,总督贵州、湖广、广西。易应昌协院副都御史,王延坦、管绍宁礼部左、右侍郎。

初四(己未),贺世寿总督仓场。

初五(庚申),史部尚书徐石麒推举朱大典、王永吉;有旨`永吉身任督师,致北都沦陷;朱大典赃私狼籍,先帝严追未给:何得朦胧推举'?士英以贿不至,故拟旨切责。寻贿至,而擢用无碍。

初六(辛酉),加翼戴新恩:史可法少保,马士英太子少师,高宏图、姜曰广、王铎太子少保。起丁魁楚兵部侍郎佥都,巡抚承、襄。

初八(癸亥),谕户、兵部:`向差内官催省、直军饷并内库钱粮,因辅臣高宏图、科臣罗万象谏止。今需用甚急,该部再严催,限八月全完'。
初九(甲子),李遽加职方司衔、沈胤培太常少卿、徐一范鸿胪卿。张献忠陷成都,蜀王遇害。

初十(乙丑),侍郎管绍宁疏请遣使告先帝后梓宫、访问东宫二王消息。

十一(丙寅),长安街遍粘匿名谤帖,指谤吴甡、刘宗周;皆李沾所为。

十三(戊辰),太后至自河南,自仪凤门入;遣灵璧侯汤国祚告于南郊。

十六(辛未),袁枢、郭正申为兖东西道兵备。

十七(壬申),越其杰巡抚河南;其杰罢闲家金陵,以马士英妹夫起。朱之臣刑部、练国事兵部、刘士桢工部各侍郎,文安之詹事,樊一衡总督川、陕,凌駉东昌兵备。

十九(甲戌),周王准于苏州城外寄居。刘孔昭请操营额饷,着常州府解。

二十(乙亥),太监孙象贤自北来归,温旨留用。命吏部察废员及举贡监生才品堪用愿效力危疆者,考选二、三十名,咨发督辅军前,以补地方缺官。

二十一(丙子),内批:张捷补吏部左侍郎;由勋臣荐。下项煜于狱,逮周镳、陈以谦等。赠吴三桂父勷辽国公。

二十三(戊寅),进士王曰俞请褒诸生许琰(琰,长州人)。

二十四(己卯),赠李邦华少保,荫子。高宏图、何应瑞合词请王永吉;允之。

二十五(庚辰),王心一工部右侍郎、高倬刑部左侍郎、王荣右通政、马兆羲礼部、成勇福建道。通政使刘士祯因病求去。太监卢九德请营制钱粮。命选淑女及内员;廷臣交章谏,不听。

二十六(辛巳),赐北京殉节大学士范景文、户部尚书倪元璐、左都御史李邦华等二十二人赠谥、祭葬有差。

二十七(壬午),姚思孝大理少卿。吏科章正宸言:`内批用张捷非制';有旨:`前解学龙荐叶廷秀亦经批升,何以寂无一言'?

二十八(癸未),故辅王应熊改兵部尚书,总督云、贵、川、湖军务;赐蟒、剑。申绍芳督饷侍郎,王志道、沈犹龙户、兵右侍郎,郭维经右佥都。封郑芝龙为南安伯。命停文武官荐举,禁非言官而上疏者。

二十九(甲申),礼科袁彭年言:`伪吏政侍郎喻上猷将荆州绅衿开荐,江陵举人陈万策、李开先在所荐中不受伪檄,万策自经、开先触墙死'。考选推知胡时享、吴适等拟授科道部属等官。

三十日(乙酉),中旨以阮大铖为兵部右侍郎,巡阅江防;刘宗周劾奏,不听。大清遣将杨万兴下济宁。

九月纪

九月丙戌朔,冯起纶福建布政使、孙朝让按察使、瞿式耜应天府丞、萧士玮光禄少卿。命王杨基、李干德各带罪往王应熊军前理饷。追理桃红坝功,夺张伦优偿。加田仰兵部尚书,锦衣指挥世荫。礼科张希夏请停荐举幸门。太监苏养性请自往催金花逋欠。太监李承芳催发年例公费。

初二(丁亥),内批蒋鸣士、梁应奇补科,郑瑜、秦镛补道。黄得功趋扬州,高杰以兵袭仪征。谕史可法:`清在河北、贼在河南,大兵继渡,或亦未便。徐、宿之师直抵汴、梁,御寇防河尚可;兼顾海宁、归德,去寇尚远。大兵前行,当抵何处?兵由楚、豫,饷就江、淮,则势分道远;东事如急,能否四应?详酌缓急,以为进取'。

初三(戊子),高宏图请开馆修史。赐北京殉难文臣二十一人、勋臣二人、戚臣一人谥,先后补予开国诸臣谥、建文死难诸臣谥、正德朝死谏诸臣谥、天启朝死珰难诸臣谥。广西巡抚方震孺言:`狼兵善火器、药弩,以副将朱之胤统千人入卫'。

初四(己丑),内旨:授福建副使郭之司为詹事。马士英奏补张成礼都督佥事、山东河北总兵。高杰请瓜州泰兴、邵伯盐税助军。纂修``玉牒''。

初五(庚寅),谕通政司:`凡故官子孙陈乞,不许封进'。

初六(辛卯),上始御经筵,柳祚冒乞侍经筵。命驱逐黄正宾。命撰``起居注''。

初七(壬辰),高宏图请设起居注。补荫故侍郎沈子才一人入监。责左光先滥荐多人,必贿嘱;着从重议处。裁各省右布政使。

初八(癸巳),刘若金通政司参议。史可法请督饷万元吉专驻扬州。逮御史黄澍,不至。命修``思宗实录''。

初九(甲午),辅臣姜曰广致仕回籍。侍郎练国事、阮大铖见朝。徐之坦补御史、余飏文选主事。左都御史刘宗周罢。

初十(乙未),郧阳守臣朱翊辨自称孤城抗贼,其子尝洪捐生;命优叙。总兵黄斌卿驻九江、郑鸿逵驻镇江、黄蜚驻采石。

十一(丙申),淮安生员谈正逢自陈守淮功求叙,不许。予故辅何如宠谥``文端''。

十二(丁酉),考功郎梁羽明自言昔日雒邸册封,着准其优叙。王之纲为荡寇将军、河南总兵官。

十四(己亥),柯楷户部左侍郎。

十六(辛丑),内阁题补中书多人。王溁右佥都,巡抚登莱、江东。太监袁升请催各钞关税银。遣行人洪维干催督钱粮。牟文绶总兵荆州。移黄得功驻庐州、高杰驻徐州。

十七(壬寅),叶重华广西按察使。陆朗复讦冢臣说谎。

十八(癸卯),吏部章正宸大理丞。录梅殷后一人为散骑。召降贼刘侨,补锦衣。命刘泌宣谕西蜀,即留王应熊军前赞画。越其杰奏□□银。给楚藩朱华渫空名札一百。令王允成镇岳州。

十九(甲辰),曹勋詹事、程正揆右谕德。黄道周礼部尚书协理詹事,陈盟、谢德溥并侍郎詹事。马士英奏张亮永城战功。刘泽清荐张凤翔、李栖凤可预重兵之选。马士英奏童生输银,免府县试。

二十(乙巳),命乡官与监生齐民较田多寡,一体当差,不得擅立官户。

二十一(丙午),万元吉还冏寺。命黄得功、刘良佐合兵驻凤、寿。\\
二十二(丁未),宗敦一、张鼎廷左右通政,周汝玑福建左布政。加何腾蛟兼抚湖北,催范矿、杨鹗、越其杰赴任。奉化布衣方翼明直言``政祈克终'';着送刑部问罪。称皇考福恭王陵曰熙陵。开佐工事例。

二十三(戊申),命鸿胪官宣谕高宏图入直、杨文骢京口监军。加左良玉太子太傅。郑鸿逵、黄蜚、黄斌卿各请战船月饷。

二十四(己酉),怀远侯常延龄予一子文荫入监。抚宁侯朱国弼进爵保国公。张凤翔添设兵部右侍郎。给越其杰饷银十万两。
二十五(庚戌),议恭皇帝建特庙。再赏定策功,加李沾左都御史;沾因奏吕大器当日沮难,革职逮问。

二十六(辛亥),太监谷国珍奏:要知府总兵而下悉行属礼。停宗室换授。

二十七(壬子),都督黄友义领黄河水师,金声桓改豫、楚援剿。以李成栋镇守徐州。再命刑部逮问黄澍,亦不至。

二十八(癸丑),起葛寅亮太常卿。谕北京旧官南来吏、兵部报名量用。刘安行佥都,提督浙江直市舶、屯田;刘若金提督闵、广屯舶,兼珠池海防。

二十九(甲寅),给驸马齐赞元千金。张捷条陈数事;上奖之。御史黄耳鼎初奉差陕西巡抚,不肯到任;因马士英见朝复班,自言无路入秦。已而例转,遂疏:`昔之按秦,陈演陷臣不测;今之外转,徐石麒朋谋暗害'。又奏:`刘宗周妄议从逆';有旨:`宗周持论孟浪,着察明'!

十月纪

大清世祖章皇帝顺治元年,定鼎燕京。

十月乙卯朔,吏部尚书徐石麒罢。马士英欲用张捷,使陆朗、黄耳鼎连疏诋之,遂致仕去;捷因署部事。周延儒子奕封乞恩免赃,马士英拟旨:`奕封赦免罪辅赃贿,系亲弟正仪指骗;正仪既故,未完赃六万着于汪曙名下追入'。曙系徽商,最富;士英先年假贷不应,故恨之也。

初二(丙辰),禁诸臣酬接宴会;马士英、阮大铖、刘孔昭、朱国弼仍每夕醵饮为常。起梁云构添设兵部右侍郎、钱元悫太仆少卿。百户魏楝等自言扈卫劳,各升一级。淮漕米□纳每担加尖一斗二升。

初三(丁巳),命铸``宏光钱''。

初四(戊午),应天府尹禳旱。减吴昌时赃银十之五。锦衣冯可宗捕得江阴知县行贿王李沽者;马士英为之请,诏勿问。马士英欲起用蔡奕琛、杨维垣,恐物论不容,乃趋一大僚荐之;荐词有``魁垒男子''语。奕琛不善,飏言于朝曰:`我自宜录用,何藉某之荐牍诮我'。闻者鄙笑之。

初五(己未),张孙振补四川道御史。

初七(辛酉),遣内官孙元德往浙、直、闽三处催金花缎价一应年额、商关税银、两浙盐储随解。赐北京死节太监王承恩等九人赠谥、祭葬、予荫有差。命于杭州选淑女。

初八(壬戌),刘泽清举用文臣黄国琦。

初十(甲子),楚抚何腾蛟加兵部右侍郎。抄没朱一冯家私。凤阳地震(丙寅再震)。

十一(乙丑),户科陆朗论徐石麒贪邪,即王思任为赵之龙所荐何得擅置察中?

十三(丁卯),张捷选授中书多人,又题监纪、通判、推官多人。张有誉言御用需迫,请差内员各处催征。

十四(戊辰),令崇王次子爚寓温州。

十五(己巳),南和伯方一元概为贼戮诸公侯伯十五人请恤。照磨张明弼奏周镳险恶。何楷兼工部左侍郎。

十六(庚午),职方杨文骢请宏佛教,以扶王化。监生蒋佐上``累朝实录''。

十七(辛未),戴英补兵科给事中、张采精缮员外郎。刑科梁某奏周仲涟卑污无耻,命提问;御史郑泰、李乔素着清能,复官。盖仲涟于贼入京时削发潜遁,不受伪辱;而乔则在疆弃城,严逮逃匿者。

十八(壬申),张捷升吏部尚书,彭遇颽改御史。遇颽敢为大言,谓马士英曰:`岳飞言大误!文官若不爱钱,高爵、厚禄何以劝人?武臣必惜死养其身,以有待'。

十九(癸酉),丁魁楚总督两广。管绍宁请予行人谢于宣祭葬;盖被贼追赃夹死者。

二十一(乙亥),张秉贞巡抚浙江。勒王永吉驻徐州。戚臣李诚臣奏``要典''始末。

二十二(丙子),停冬至郊祀。颁户部印单给州县实填赎锾。

二十三(丁丑),解学龙刑部尚书、陈盟吏部左侍郎、杨维垣通政使。阮大铖奏雷縯祚不忠不孝;下法司严讯。河南劝农尚书丁启睿罢。

二十四(戊寅),御史霍达巡漕。命停今年决囚。

二十五(己卯),张凤翔复尚书,管侍郎事。

二十六(庚辰),复以黄耳鼎为御史。

二十七(辛巳),鸿胪寺少卿高梦箕北来复任,谢恩。

二十八(壬午),赠故祭酒许士柔詹事。士柔常熟人,与文震孟、倪元璐同年友善,正谊相勖;温体仁恶之,阻其入阁。摘其旧撰高攀龙诰语,降调之;朝论共愤。至是,吏、礼部为请命,照四品例全给。

二十九(癸未),谕吏部:`郝明征原非行贿,准复原官'。

三十(甲申),张作楫提督四夷馆。张孙振追劾吴甡、郑三俊、刘宗周、祁彪佳。

十一月纪

初一乙酉朔,予李邦华、王章荫锦衣世官。周藩安乡王居无锡。

初二(丙戌),蔡奕琛吏部左侍郎。

初四(戊子),西宫旧园落成,赐名慈禧殿。桂王薨,谥曰``端''。着候勘黄澍回籍。

初五(己丑),凤阳皇陵灾,松柏俱烬。陈僭夫私自回籍,着按抚察之。御史何纶按淮。

初六(庚寅),越其杰赴任河南,有旨慰之。行人庄则敬自言曾事福恭王;命与考选。命文武官俸尽支本色。命开屯海中玉环等山。太监韩赞周请西洋大炮。命唐庶人聿键居广西平乐。

初七(辛卯),常应俊荐许定国实心恢复;着铸印给之。命生员纳银充贡。总兵官邱磊有罪,下狱死。

初八(壬辰),吉贞王子慈煃嗣封。寄流寓诸生于淮安府学。总兵马进忠镇荆州。

初九(癸已),设起居注六员,轮珥笔以记实事。驸马齐赞元称颂刘孔昭翼戴有功,赏不足酬;着吏、礼部再议。王骥为太仆卿。居辽王于海宁。
初十(甲午),改太仆寺署于南都。居祁阳王于邵武。陆朗言:`徐石麒以巧诈文其贪、刘宗周以迂腐托其正,必有真才、真品者如王骥、郑渝,畀以节钺,当无多让'。左良玉奏承德将士饿死。

十一(乙未),夜,端门外火。大清兵破海州,入宿迁;山东及丰、沛尽降。

十二(丙申),琉球世子尚贤入贡告袭。命郑鸿逵节制京口至海门。

十三(丁酉),右佥都郭维经恳辞职;内旨责其欺卸。应天府祁彪佳罢。

十四(戊戌),大理卿郑瑄罢。奖高起潜冒险来归,忠义可嘉。

十五(己亥),朱继祚少詹事。刘泽清请安流寓青衿,以便科举。工科李某为降贼被杀诸臣顾鋐、彭琯、李逢申请恤。郑芝龙奏黔兵万里荷戈,三月缺饷;上切责兵部。

十六(庚子),升李永茂巡抚南赣。屈勋补吏科给事中。户科罗万象以回奏掩饰,罚俸一年。

十七(辛丑),追论江右功,解学龙世袭锦衣千户。奉先殿上梁。沈廷扬加光禄少卿,宋劼、李犹龙太仆少卿。周藩临汝王寓武进。孙维城袭怀宁侯,补铁券。予故举人归子慕、张世伟、顾云鸿等翰林待诏。给浙江总兵王之仁``镇倭将军''印。

十八(壬寅),陈潜夫奏张缙彦、陵潜南渡;着安插河南,不必入觐。

十九(癸卯),兵科戴英自辨被谤情由。

二十(甲辰),曹勋礼部侍郎,管翰林院;沈延嘉、刘同升、陈之遴、刘正宗各转坊官。赠故山东巡按宋学洙大理卿;学洙潜家二年始故,马士英奏其殉难,因得恤赠。西鄂王寓宁国。谕苏抚大瞿山屯田。吏科张某言:`臣乡来者言贼久踞平阳,人亡过半'。吏科抄参安远侯柳祚昌所荐程士逵富豪蠢竖,非可与举贡同例。

二十一(乙巳),鲁王移居台州。戒宗室换授。

二十二(丙午),李沾请分台员从逆真枉。颍州生员卢鸿上七政历。

二十三(丁未),长至节,上受朝贺。张凤翔兵部尚书,巡抚苏、松四府;卢若腾巡抚凤阳。申绍芳言江北需饷急;命户部于附近府州县措二十万付之。刘洪起加总兵衔。淮安地震。

二十四(戊申),刘孔昭以定策功进封侯爵;不受,特旨奖之。奖阮大铖役民修筑敌台。谕吏部:`王孙蕃与李沾定策同事有劳,一体优叙'。谕兵部:`职方监纪幸滥,俱不准'。谕礼部:`求恩滥予可厌,宗室呼吁难凭;宜慎辨之'!

二十五(己酉),马士英请榷酒助饷;下部行之。九江总兵黄斌卿侦知左良玉难制,请改驻皖、池;从之。

二十六(庚戌),黄斌卿改驻安庆。命许定国镇守开封,与王之纲合剿。高杰请籍没周延儒财产;谕``不忍''。

二十七(辛亥),命王永吉议塞汴口。吴希哲补工科、鲁倜补山东道。王国宾光禄卿。黄升请牛种兴屯。杨文骢请金山、圌山建城;从之。

二十九(癸丑),命马士英大阅。

三十(甲寅),起杨公翰太仆卿、马鸣霆湖广参议。汀州分守夏尚絅进万金助饷;有旨:`以道臣而捐万金,操守可知;玩寇猖獗,贻祸地方。着革职提问'。

自五月不雨,至于是月。

十二月纪

初一己卯朔,加练国事兵部尚书.白贻清太子太保。御史沈向巡抚湖广。命荆王驻九江。

初二(丙辰),琉球使臣金应元入朝。

初三(丁巳),马士英奏刘孔昭实心定策,刘泽清、张文光密议效忠;命二刘进侯爵,文光加宫衔。刘泽清奏请禁巡按访拏奸党。

初四(戊午),录国初功臣冯国用、冯胜各世袭指挥。

初五(己未),加刘承胤右都督。马士英保荐胡国贞等,悉加总兵衔。

初六(庚申),凌駉交纳伪凭、伪契。大清兵围邳州,凡三日。

初七(辛酉),凌駉实授御史。命何腾蛟以兵部侍郎总督川、湖、云、贵、广西;召杨鹗回部。安远侯柳祚昌自言定策功高;斥之。以巢湖民船为保甲。

初八(壬戌),高杰荐旧臣黄道周、黄志道、解学龙、刘同升、赵上春、章正宸为众正,吴甡、郑三俊为万世瞻仰,金光宸、熊开元、姜采无愧社稷臣,金声、沈正宗夙储经济。

初九(癸亥),吴国华右谕德。刑部奏偏沅抚陈睿谟失守封疆事;着助三万金收赎。定勇卫营万五千人。谕太监高起潜:`阁臣已在河上,尔驻浦口,无事便于提调、有事相机应援'。

初十(甲子),命太监卢九德丈量芦洲升课。许桂王妃王氏扶王柩回衡。大清兵入河南府,总兵李际遇降。

十一(乙丑),齐藩宗长知墭等请换授官;不许。

十二(丙寅),吏科张某奏:`督抚所荐司道、推知、贡监、生员巧诈毕现,无非骗官'。有旨:命严覈参处。

十三(丁卯),马士英以定策功,加张文光太常少卿;又以尹伸、顾光祖添注少卿。又奏:沽酒之家,每斤定税一文。

十四(戊辰),李希沆添设兵部右侍郎、高斗枢巡抚湖广。奖阮大铖筑鸭矶堡之劳。监军宋劼请采矿铜陵。史可法奏请锆弹三万筋、生铁十三万筋、铜甲叶五百副;命部给之。又荐举人韩诗等。

十五(己巳),通政使杨维垣言``三朝要典''为党人所毁;命礼部购付史馆。陈洪范北使还,左懋第不屈被执,马绍瑜留;和议不成。行税法。颠僧大悲至京,自称齐王、又称潞王;下镇抚司鞫讯。

十六(庚午),丁启睿加太子太保、了魁楚进兵部尚书。赠李邦华太保。

十八(壬申),进马士英少师。义阳王居太仓。尚书黄道周、太常卿葛寅亮、尚宝丞邹之麟见朝。命王永吉防河北、张缙彦防河南,分许定国、王之纲信地。

十九(癸酉),陈燕翼吏科右、钱增兵科左。旧阁臣钱士升加太子太保,荫孙焘中书舍人。谕都督牟文绶鼓锐先赴施州。

二十一(乙亥),允部议,诏封于谦临安伯;遣太仆主簿陈济生致祭。

二十三(丁丑),命治旧顺天抚陈祖芑失城之罪。开文武职官诰命事例。大清兵自孟津渡河,命高杰进屯归德以备之。

二十四(戊寅),张缙彦分诸将王之纲等防河。巡抚陈潜夫获太康伪知县安中外等、副将刘铉、郭从宽等,杀贼六百余级;擒鄢陵伪知县王度、许州伪巡捕王法唐。总兵王之纲斩贼都司虞世杰。总兵刘洪起获汝宁府伪官祝永苞、上海伪知县冯世遇,斩三百七十级;又于襄城斩贼二千二百七十六级,擒贼二百三十一名。总兵许定国获陈州伪官惠在公等。各加级;以洪起斩获独多,仍加二级。

二十五(己卯),念郧阳孤危固守,加徐起元兵部侍郎、高斗枢副都御史、朱翊辨京堂缺用。唐庶人聿键求复王爵;不允,命居广东之平乐。

二十六(庚辰),命妇入贺。复姚思江、王水吉原官,倪嘉庆刑科右。

二十七(辛巳),驸马齐赞元掌宗人。

二十八(壬午),瞿式耜巡抚广西、马干巡抚四川。搜取宁波洹课七千两。

二十九(癸未),布衣何光显上书乞诛马士英、刘孔昭;诏戮于市,籍其家。
三十(甲申),太监孙象贤、孙珍世锦衣佥事。吏科抄参``逆案''陈尔翼颂珰,有``内外诸臣心珰心''之语;聂慎行久挂吏议,内计处分;杨屯升亦系察处之人:近皆荐起,抄出议之。贾登联四川总兵。禁四六俪文。

\hypertarget{header-n22}{%
\subsection{卷二·南都甲乙纪(续)}\label{header-n22}}

正月纪

大清顺治二年(乙酉)、宏光元年正月初一乙酉朔,上御殿,受朝贺。

初六(庚寅),加史可法太师、马士英少师、王铎少保,予荫;以士英掌文渊阁印,充首辅办事。可法辞太师,许之。

初八(壬辰),流星入紫微垣。方允元、杨兆升为吏、兵科,冯志京、张茂梧、袁宏勋、周昌晋补御史,余飏为稽勋员外郎。史可法奏荐赞画刘湘客;又奏择将守邳。马士英奏撰张捷、卢九德敕;又奏除禖官九十五员。阮大铖报治江筑堡,上嘉之;又请黄蜚、杜宏域联络水路。刘泽清请添水兵。制丹阳陆路视良乡例,给邮符。

初九(癸巳),监军卫胤文奏:已冒雪抵徐。吏侍郎陈盟奏:川事溃裂。贵抚李若星奏:川贼势甚猖獗。赣抚李永茂奏:寇扰汀州。钟斗添注太常少卿,郭如闇、方士亮补户、刑科。进丽江土知府木增太仆卿。总兵刘洪起击贼于襄城,俘斩五百余人。马士英请赐陆献明抚黔功;予荫,子入监。御史沈某请举郊祀;命俟之。命黄得功、刘良佐进屯颍、亳;受命不行。高杰提兵直抵关、雒,进据虎牢。运司解银万两渡江,为镇江都督郑彩截留;诏谕彩勿擅。

初十(甲午),修奉先殿及午门、左右掖门。邹之麟为应天府丞。四川参议耿庭录改遵义监军。御史凌駉巡按河南,给吏部空札三十张、兵部空札一百,以待矢义南归者。户部尚书张有誉奏:江北各藩新旧兵饷额本有定,今所增万不能支;令督辅议察。工部请裁御前料价以供楚饷;上不许。侍郎何楷定各镇鼓铸。太监高起潜言边将不宜内转;又请银市马,命给太仆寺银五万两。

十一(乙未),马士英奏杨御蕃五载战功;着进左都督及马进忠、王允成并加太子太保。晋众臣迎驾之劳,补指挥、千户等官。命各府推官稽察官役冒工料。允刑科钟某言,凡监纪等官猾棍白丁借题募府骗钱者,悉行驱逐。许定国诱杀兴平伯高杰于睢州。

十二(丙申),高允滋补御史。安抚黄某荐废籍官李乔等。御史游有伦极言朝臣、镇将背公植党。部院剧震,分请马士英饮酒。刑部尚书解学龙奏从逆六案;以登极初,停刑。

十三(丁酉),户科陆某请覈学田输谷裕国;从之。河南副将郭从宽缚长葛县伪令来献。

十四(戊戌),叶廷秀添注光禄少卿。户部尚书张有誉言:旧制钱粮各处必解部,派发于外,宜着为令;从之。禁宗室入京朝见。太监高起潜请佃丹阳练湖,可岁得五万金;从之。又奏浦口增建墩台;着工部估价鸠工。太监韩赞周告退;与其定策大功,不允辞。田仰奏叙效劳将领。凌駉请早定恢复大计;命专畀刘泽清、王永吉。太监孙某劾奏盐臣李挺欠银二十六万;不许其报竣。

十五(己亥),刘泽清报年终措饷给兵;温旨奖其忠义,又允行间事不中制。蔡秋卿广东海北道。杨振宗奏皖兵缺饷。

十六(庚子),钱增为刑科。松江知府陈亨为四府兵道。张有誉酌定白粮每石折价一两三钱。

十七(辛丑),吏部侍郎蔡奕琛兼东阁大学士,入阁办事。

十八(壬寅),左良玉请留抚臣何腾蛟;有旨:`五省总督之设,不惟恢复荆、襄,且以接应巴蜀。腾蛟俟高斗枢到任,方行移镇'。

十九(癸卯),刘孔昭请革内地监纪,并汰武弁。又言:`未尝到王孙蕃榻前商量定策,孙蕃前奏欺妄,大为无耻;刘宪章闻变逋逃,自当与余日新同议'。贡生韩诗予职方主事。工科李清辨其祖思诚误入魏党``逆案'';命下部议。申绍芳为祖特行陈当年回护宫闱旧情;有旨慰勉。真人张应京入朝。史可程自北庭南奔。

二十(甲辰),马思理添注左通政、张时畅尚宝司丞。主事李尔育奉旨宣谕刘洪起、李际遇二人,俱无见;遇张缙彦,即至睢阳而回。命删定``三朝要典''。朱国弼、张孙振劾解学龙。

二十一(乙巳),荫故山东巡抚陈应元子入监。郎中赵明铎为云南提学、黎永庆为贵州提学。赐侍郎阮大铖蟒服。雪推官周之夔罪。谕吏部:`邹之麟清修自守,着起用'。谕刑部:`朱一冯身为大臣,多藏厚亡,大丧缙绅之体;其入官七万外,田宅所值几何?九千六百亩之外有无余产?察明'!夺解学龙职。

二十二(丙午),起唐世济左都御史,管右都事。葛寅亮为大理卿、戴英为兵科左。荫故辅丁绍轼子入监。吏侍郎陈盟辞任。太平推官胡尔恺辨罪;有旨:`壬午南闱关节滥行,缙绅子弟几干半榜;公议沸腾,何止周正仪一人。尔恺已经薄处,姑不究'。

二十三(丁未),刘孔昭请汰多官。尚宝丞耿奉光辨父如杞勤王之祸;上念其首倡可怜,下部察。

二十四(戊申),安远侯柳祚昌荫子入监。尚书黄道周、侍郎梁云构到任。兵科王之晋奏南阳为贼所踞,家乡难归。

二十五(己酉),御史黄耳鼎兼巡上、下江。上林监丞贺儒修论管绍宁贪髦阴奸;诏不问。议修徐州城。

二十六(庚戌),刘应宾太常卿、王梦阳浙江按察使、文士昂云南布政使。赵之龙言章服违制;上是之。令武臣自公侯伯而下,非赐肩舆,并遵骑马;坐蟒、斗牛非奉赐,麒麟、白泽非勋爵,不许借用。御史刘光斗请鉴别大臣;诏衰颓庸钝者,自行引退。

二十七(辛亥),戎政张国维给假归;李希沆代署。前参政陈尧言:`尝任待诏,侍福恭王有旧劳';下部寝之。先,贵阳杨师孔与陈同侍,竟得礼部侍郎;盖马士英戚也。加卫胤文兵部左侍郎,总督兴平标下镇将,经略开、归防剿事务。

二十八(壬子),荫徐大绶子入监。吴希哲为工科。赠邱禾嘉左副都御史、冯任右都御史,各荫一子入监。

二月纪

初一甲寅朔,命于嘉兴、绍兴二府选淑女。

初二(乙卯),时东川侯勋卫胡家奴作横;兵部言:`东川久已革袭。又戚畹向无勋卫,草创滥冒'。命清厘之。覈北京锦衣卫官南奔实迹,不许轻题。荫杜锵太仓百户。袁继咸报郧镇重围。刑科梁某奏:全蜀已无完土。

初三(丙辰),王骥右副都御史,巡抚湖广。李清添注大理丞、徐复阳御史,甘惟、邢大忠云南、广东各按察使。谭振举苏松粮储道、田有年贵州驿传道。严究司库侵欺。谥桂王曰``端''。高起潜请开纳银赎罪之例,上以`纳银免死,则富豪墨吏何所不至;流罪以下或可赎';下部酌议。

初四(丁巳),太监王肇基条奏京城购捕方略。钱继登、周端豹各添注尚宝少卿,陆怀玉福建按察使。胡世宗自称越公八世孙,求附勋卫。

初五(戊午),荫故辅未国祚子入监。赠许土柔詹事兼侍读学士,荫子入监。行人朱统\{金类\}讦御史周粲;命勿究。工科吴某荐起被察官李永昌、周文夔。安庐抚程某奏:`获假弁王梦旭,自称藩府都司;抢掠民商,辱及关吏。又有铜陵县盗大船,牌额上写``天子一家'''。

初六(己未),阮大铖升兵部尚书,协理部事;仍管巡阅江防。高倬刑部尚书,吏部陈盟改左侍郎、王志道右侍郎。吴本泰添注尚宝丞、关守箴广西布政使。调浙江巡按彭遇颽于淮扬,以淮扬按何纶移浙(遇颽癸未进士,避乱南渡,首附马士英,诞说蜂涌,遂授职方主事,改御史。身任募兵十万,或问饷安出?曰:`搜括可办也'。才抵任,即移家入杭,纵强奴掠市钱。抚臣张秉贞以问,士英以遇颽边才调用)。有上书言开化、德兴有云雾山为先朝封禁,开之可以助国;命太监李国辅会同抚按勘视。

初七(庚申),赠冯垣登太仆少卿、邹逢吉太仆丞。李长春添注太仆少卿。太监孙元德覈报苏州七年欠饷七十四万两、金花银七万两。

初八(辛酉),朱国弼核勋臣世系无容幸袭;命饬之(天启、崇祯之际冒袭最多,惟有力者得之;如王先通以王守仁异毋弟之后、刘孔昭之父荩臣系刘尚忠出婢外生之子,竟自夺嫡,莫之敢问)。顾元镜为广东岭西道、孙时伟浙江驿传道。遣户科倪嘉庆、中书胡承善掣盐于瓜、仪,加盐课每引五分。

初九(壬戌),杭州机匠疏称旧抚潘汝桢旧泽难忘,□建逆祠系前任事;上以会稿甚明,不允(盖汝桢事,久有言其误者)。

初十(癸亥),马士英以京师水陆各营杂务,全造小印号色分别。高起潜奏分汛筑台事宜。点用云南、贵州试差徐复仪、林志远等。

十一(甲子),兵科戴英论陈洪范所请叙录从行员役,有何劳绩?滥予非宜;上是之。太常卿张元始请虔祀社稷。陆康稷改文选郎中,加沈廷扬参议。宫继阑、曹烨广东、江西副使。叶绍颙太仆卿。考选林有本、沈应昌、张利民、韩祖、钱源、徐方来、庄则敬为给事中,王锡衮、刘襄、夏维虞、郝锦、王大捷、毕十臣、张兆熊、王养、郭贞一为御史。谥思宗皇太子曰``献愍''、定王曰``哀''、永王曰``悼''。

十二(乙丑),上始御经筵。阮大铖请江上筑堡助工;命张亮、程世昌严督州县经营。中书陈爊自陈拥护有劳,愿与考选;不许。故巡抚蔡懋德之子为父求恤;内批:`懋德纵贼渡河,一死何赎'?不允。户部言兵饷日增;有旨:`各督折兵十八万,一切旧兵应并销入数内'。都督杨振宗请裁见糜各饷,以供鼓铸。太监高起潜请饷;着于浙、闽增派二十万。孙元德催解军前饷。史可法请用高杰部将李本深为提督;不许。遣黄道周祭告禹陵。张孙振奏劾礼部尚书顾锡畴。

十三(丙寅),靖江王亨嘉表贺登极,因奏全、永、连三州皆为土贼所踞,抚按匿不以闻。兵部右侍郎徐人龙罢。谕祭兵部尚书张希武。命于苏州织造大婚冠服。

十四(丁卯),谕:`都督牟文绶久任江上,大肆骚扰。户部所欠之饷,何不速发?坐视流毒。即征盐抵补,催兵起行'。御史郑瑜劾前总督朱大典侵赃百万;上谓`大典创立军府,所养士马岂容枵腹?岁饷几何,不必妄计'!命衰劣在京诸臣俱着自陈。赐罪诛内官刘元斌、王裕茂祭葬,荫子锦衣卫、指挥使。旧府厨役各授百户。姚思孝、沈胤培大理左、右少卿。荫方孝孺裔孙树节五经博士。撤高杰部兵回。遣太监高起潜安抚兴平营将士驻扬州。

十五(戊辰),史可法奏擒贼臣程维孔;又奏左懋第抗节。

十六(己巳),谕部:`捐助原听民乐输,抄没乃朝廷偶行;岂刁民献媚报仇之事。宗藩、勋戚、武臣须敬礼士夫,与地方相安;不得听奸人拨置,非法罔利'!李嗣京御史。

十七(庚午),谕吏部:`吏贪民困,全由抚按婪贿。林挚、李仲熊互讦事情,延搁已经十月,虚实应与立剖;何必复行外勘,以滋延卸'!予罪谴尚书刘荣嗣昭雪。予苏松殉节王钟彦、宋文显、施溥祭葬。太常卿张元始请改皇考谥号。

十八(辛未),马士英请免朱一冯籍产。``逆案''杨维垣起用,补通政使。奖卢九德营粮就绪。

十九(壬申),蔡奕琛进尚书文渊阁。起朱大典、吴光义、易应昌户、兵、工部各左侍郎,陈洪谥太仆少卿。侍郎钱谦益请即家开局修史;不允。奖刘廷元保全慈孝有功,特予优恤。王骥惊闻滇信,辞任;不许。

二十(癸酉),令刘良佐驻归德。马士英请褫中书唐允甲。李维樾为兵科。存问大学士钱士升。兵部侍郎练国事罢。张亮请立盐税局于皖城;不允。
二十一(甲戌),改谥先帝``毅宗烈皇帝''。王铎六请告归。

二十二(乙亥),谕阮大铖:`江上奸人出没、乱兵纵横,以致商旅梗塞;不可不严备'。太监孙元德搜覈常州府欠金花银九万五千、积欠三饷至三十三万;命勒限严征。

二十三(丙子),卫胤文奏:`柳城土寨金高自筑土城、集勇壮,不受伪官;乞授何□进、钱式命、葛舍馨考功郎,陈瑞大理寺副。□以副总兵职,以六筋四两为准'。

二十四(丁丑),张承志袭惠安伯。来方炜添注太仆少卿、吴适兵科右。吏科马嘉植转岭西道、御史沈荃苏松兵备道、御史高允兹湖南道、文选主事余飏广东水利道。户科熊维典奏:`四府逋欠三年内三百三十一万八千五百,皆属应征;又已征不解九十五万六千有奇'。又奏:`正项辄借支赎锾侵那弊薮,至批详才下、提差已至,抚按身先不法'等语。又户部王某奏:`守令失职,赋额不清;飞派朦胧,火耗太虐'。袁宏勋疏攻袁继咸,左良玉救之,并言``要典''宜焚;谕解之。

二十五(戊寅),贵督李若星奏以兵勤王,谕止;如已到常德,即留隶何腾蛟。户科熊维典察覈嘉定漕折胥吏侵匿至五万两。管绍宁于寓所失去部印。李自成走承天。

二十六(己卯),奉安御容于武英殿。吏部恭报剪除群贼,加马士英太保、王铎少傅。

二十七(庚辰),朱国弼请治郭维经庇逆。卢九德等九员加级。

二十八(辛巳),大兴伯邹存义请加提举公署。

二十九(壬午),马士英殉管绍宁之私,请更铸各衙门印,去``南京''字;其旧印悉令缴入。进都督赵民怀太子太保,荫子世锦衣百户。陆朗、吴希哲为户、工科左。刘孔昭请益操江书役俸粮。吴希哲奏都城五方杂处,假宗、冒戚、伪勋、奸弁横行不道,虐民戾商;有旨严编。

三十(癸未),起熊化太仆少卿、水佳胤尚宝司丞,皆添注。僧大悲伏诛。李向中嘉湖道。鸿胪少卿高梦箕密奏先帝皇太子自北来;遣内臣踪迹之。
三月纪

初一甲申朔,上受朝贺,始御日讲。命高起潜安抚扬州。御史徐复阳讦吏部以文德翼、夏元彝匿表升补;上功责之。刑部郎中申继揆请严责左光先抗题。内臣自杭州送北来太子至京,驻兴善寺;夜移至锦衣卫都督同知冯可京邸,□□□视。大学士王铎叱为假,严究主使;自供王之明,旋下中城兵马司狱。

初二(乙酉),御史袁某请起罪废诸人;谕史\{范土\}、陈启新、张文郁不准。福府旧役乞恩者百余人。吏科张希夏升太常少卿。

初三(丙戌),钱谦益进宫保兼翰林学士,陈燕翼、杨兆升为礼、工科右。

初四(丁亥),吏部尚书张捷奏故辅温体仁清忠谨恪,当复``文忠''之谥;顾锡畴以私憾,议削文震孟宜改谥。上命温允复、文免议。

初五(戊子),命太监乔上总理两淮盐课,严察兵马粮饷。李自成逼承天府,左良玉遣使告急;命督臣何腾蛟等御之。大清兵取郾城,又取西平。

初八(辛卯),刘泽清自陈弃家南奔,予注鸿胪卿。右都唐世济到。大清兵取上蔡。

初九(壬辰),马士英自剖诛盗程继孔之功;又奏李天培等各锦衣指挥世袭。耿廷箓巡抚四川、朱之臣添注兵部左侍郎、刘应宾通政使、吴希哲吏科都。汝宁镇将刘洪起以无饷,撤兵还楚。工科杨兆升奏江南有司既征本色在仓,不肯还民;又征漕折。命百官会审王之明、高梦箕、穆虎于午门外。藩邸元妃童氏在河南自东,刘良佐送至京;上怒,目为妖妇,下锦衣卫狱。李自成兵寇潜江。

初十(癸巳),礼部请恤甲申殉难诸臣;有旨:`阁部大臣谋国无能,致兹颠覆;虽殉节堪怜,赠恤已渥。先帝斩焉不永,诸臣延世加恩,臣谊何安?通着另议'。刘理顺、成德准荫子入监。户部尚书张有誉请于文武廪禄外,各加公费;不许。加郑芝龙太子太保;其弟及将士二千人各升授。御史郝某奏各镇分队于村落打粮,刘泽清尤狠,扫荡民舍几尽。又奏:官买私赂,量出剩余助公,以佐民急。时买官者,大县多至二十余家,少亦有数家。然止两殿中书及改贡各有事例,其职方、待诏、监纳、追荫、起废皆向权门投纳,故郝言之。锦衣卫请添旗役。遥祭诸陵。

十一(甲午),李守贞荫都督同知。停八、九品官移封及援纳待诏等官。

十二(乙未),史可法自劾师久无功。马士英请荫内官三人各锦衣千户世袭。阮大铖荐马锡充总兵,仍莅京营;锡即士英长子,以白衣径仕。左懋第抗节死。左佥都郭维经告病去,江中遭寇甚惨;人皆惜之(或云阮大铖密遣兵劫之也)。

十三(丙申),庐抚张亮飞报闯贼分股南来,求解职;放归。贺世寿、曹勋回籍。

十四(丁酉),起罪废陈于鼎掌翰林院。张捷奏:嘉靖间侍郎瞿景淳补荫。李若星加一品服、李干德加一级、于元炜八人纪录。李希沆兵部左侍郎。户部张有誉奏郧兵三千,先解五万两,运至九江,交袁继咸送去;又奏:浙省银十二万、闽省银八万解至高起潜军前开销。

十五(戊戌),复会审太子。

十六(己亥),徙崇王居福州。命黄得功移镇庐州,与刘良佐合力防御。

十九(壬寅),思宗忌辰,上于宫中举哀;百官于太平门外设坛遥祭。

二十(癸卯),命三法司覆审太子,毁黄得功疏以绝奸谋。

二十一(甲辰),封黄九鼎雒中伯;其弟金鼎都督同知。许定国前哨抵归德,王之纲屯宿州。

二十二(乙巳),黔将包琳为其下所杀。黄希宪以擅弃封疆,逮戍。大清豫王从河南下,是日取归德;巡按御史凌駉及其子润生死之。

二十三(丙午),朱大典尚书,提督江上。兵科戴英讼故罪辅薛国观之冤,株累叶有声、林栋诸臣;上是之,下部议覆。许定国降大清,封平南侯。张天福请于史可法,回扬安顿家口;留防之兵,遂离象山,几至瓦解。罢安庆巡抚。

二十四(丁未),方国安佩``镇南将军''印。张有誉酌议卖官、赎罪纳银事例。

二十五(戊申),左良玉举兵,以奉太子密旨诛奸臣马士英为名;焚武昌东下。

二十六(己酉),刘良佐奏荆州失陷。钱维登佥都御史,总理两淮盐法。赠故兴平伯高杰太子太保,荫一子锦衣百户。

二十七(庚戌),登莱巡抚王溁缴纳印敕。大清兵向徐州,总兵李成栋登舟南遁。

二十八(辛亥),赠故辅李标少傅。王国宾太常少卿,提督四夷馆。

二十九(壬子),进李本深太子少保、左都督。荫左良玉世锦衣指挥使。大清兵取颍州、太和,刘良佐檄各路兵防寿州。

四月纪

初一癸丑朔,颁各官新印。王永吉报大清兵已过河,自归德以达象山共八百里,无一兵防,扬、泗、邳、徐势同鼎沸;命史可法驰扼徐、泗。黔兵杀掠徽境,徽人汪爵率众御之,杀其凶者数人,诏擒爵抵罪;御史黄耳鼎请赦,不允。詹有桓混入宫门秽言辱骂:着杖一百。

初二(甲寅),罢练河屯兵太监高起潜。湖抚何腾蛟请解任;不允。

初三(乙卯),马士英告退;慰留之。

初四(丙辰),遣内官守十三门,禁各官家眷不许出京城。徐元爵嗣魏公。惠安伯张养志论选郎陆康稷贪污;诏勿问。御史毕十臣言:`孟夏享太庙,文武班陪祀寥寥';命戒其后。左良玉兵陷九江府,寻死,其子梦庚自称``留后'';命阮大铖、刘孔昭率师出御。

初五(丁巳),左梦庚兵陷建德。追恤三案诸臣刘廷元等二十人,并复原官,仍各荫赠有差。

初六(戊午),左梦庚兵陷彭泽。梁云构、李乔皆兵部左侍郎。逮前巡抚陈潜夫于家。

初七(己未),左梦庚兵陷东流,京师戒严;以公侯分守长安等门及都城十三门,征靖南、广昌、东平镇兵入卫,命史可法至江北调度。祁逢吉总督仓场侍郎;王骥加侍郎,仍巡抚。周宗文光禄少卿、刘呈瑞御史。兵科钱某奏:警报日至,刘泽清、刘良佐退兵近郊。镇兵避大清南迁,占夺民房民物。

初八(庚申),马士英自出五十金,委黄金种招募健卒;即以同知补用。马士英荐白衣李毓新知兵,即补职方主事。卫胤文以紧急辞任。

初九(辛酉),决从逆贼臣光时亨、周钟、武愫于市,周镳、雷演祥勒令自尽;余尽革职放还。路振飞自明守淮有功,朱国弼复论之;有旨慰国弼、责振飞。王时敏起太常少卿。李沾请听民搬运柴米入京。大清兵自归德分道:一趋亳州、一趋砀山、徐州;总兵李成栋奔扬州。

初十(壬戌),御史何某请禁四六文章、坊间社草。封常澄为襄王,命居汀州。都督黄斌〔卿〕等与左兵战于铜陵之灰河败之,明日复沈其船三千艘;命劳诸将银币。

十一(癸亥),马士英奏大清兵与西寇并急,请征皖饷。户部奏催各府兵练饷。

十二(甲子),户部请催徽、宁等府预征来年之银。刘泽清陈文武考察旧例不许借题罗织,驱除异己。皇亲黄九鼎言皇亲满街作横,请查核之。
十三(乙丑),左梦庚陷安庆。大清兵取泗州。

十四(丙寅),大清兵渡淮,史可法退保扬州。刘良佐提兵入卫。

十五(丁卯),太监屈尚忠奏催大礼措办;着部火速那借。马士英言开洋之船,每只或二百、三百金,设太监给批放行;于崇明等处起税,如临清关例。选淑女于元晖殿。潞王在杭州,上书请僻静一郡。

十八(庚午),王永吉改总河兼督淮、安、凤、庐,钱继登兼抚扬州,田仰撤回另用,卫胤文事定再定议。浙按柯伦束装赴任,着门军放行。太仆丞张如惠丁忧,着留其携赀充饷。左梦庚兵至池州;诏暴良玉罪状。

十九(辛未),大清兵围扬州。范凤翌光禄少卿、黄国琦试兵科。御史刘某奏缉奸严密,丁役四出扰害。命申绍芳亲至浙直催饷。德清县大荒之后,一县民逃亡殆尽;实征三万三千两,着有司那借。

二十(壬申),命参政马鸣霆驻江阴、副使印司奇往京、杨文骢专监镇军;凡逃军南渡,用炮打回,不许过江一步。王永吉奏弃徐万分可惜,乞敕刘泽清固守淮安,勿托勤王移镇。命差风力科臣督催``江南赋役全书''。令廪生输银准贡。高起潜言闯贼尾后、我兵击前,左党穷蹙,自当授首,不烦过虑。前山东提学翁鸿业之子求父褒恤;下部察议(国亡后鸿业尚存,逾年乃殁)。

二十一(癸酉),刘泽清大掠淮安,席卷辎重西奔。

二十二(甲戌),刘泽清南奔,大清兵渡淮。

二十五(丁丑),大清兵攻陷扬州,督师史可法死之;知府任民育、知县周志畏等不屈死,总兵刘肇基战死,原任兵部侍郎张伯鲸、都督兵部右侍郎卫胤文、监纪主事何刚先后殉难。

五月纪

初一壬午朔,进封黄得功为靖国公,世袭;诸将升荫有差。李彬为佥都御史,巡抚河南。

初二(癸未),移惠王于嘉兴。遣京营兵二百迎黄得功移守板子矶;得功击左兵于板子矶,败之。大清兵驻瓜州。

初五(丙戌),百官进贺,上不视朝;以串戏无暇也。

初六(丁亥),午后,马士英入大内,与韩赞周、卢九德二监商议,传令各门下闸,辰开申闭。刘泽清率兵至浦口。分苏、松、常、镇为二巡抚。杨文骢佥都御史,巡抚常、镇二府兼辖扬州沿海等处军务。

初七(戊子),集百官清议堂议事,预坐者十六人;马士英、王铎、蔡奕琛、陈于鼎、张捷、陈盟、张有誉、钱谦益、李乔、李沾、唐世济、杨维垣、秦镳、张孙振、钱增、赵之龙各窃窃偶语,百官皆不与闻。临散,李乔、唐世济相和曰:`便降志夺身,也说不得了'。后有叩诸大僚者,皆云:`大清信虽急,如今不妨了'(盖密议藉之龙纳款于大清也)。是日昼晦,大风猛雨,人心汹汹。

初九(庚寅),大清豫王渡江,援师悉溃;杨文骢走苏州、郑鸿逵走福建。

初十(辛卯),闭京师各城门。二鼓后,上奉太后、一妃与内官四、五十人从通济门走出。刘孔昭斩关遁。上如黄得功营;旋如芜湖,命扬州府同知李继晟巡抚安庆,命阮大铖、朱大典督师。

十一(壬辰),马士英奔浙。京城溃乱,兵民拥立王之明。

\hypertarget{header-n27}{%
\subsection{卷三·南都甲乙纪(续)}\label{header-n27}}

议立福藩

四月初三日(庚申),时潞、周藩泊淮上者,各以宫眷随;独福王孑然,与常应俊等数人流离飘泊。凤阳总督马士英阴使人导之,借淮抚路振飞舟南行。

十四日(辛未),有内官至南京,府部科道等官始知北京被陷确信,上殉社稷;大小惊惶。史可法、张慎言等集高宏图寓,议所尊奉。时潞王、福王并在淮上,姜曰广属意在福王;史可法曰:`在藩不忠、不孝,恐难以主天下'。逡巡而散。二十一日(戊寅),时新主未定,人望皆在潞王。高杰、刘泽清移书路振飞,问所奉;振飞云:`议贤则乱、议亲则一,现在惟有福王。有劝某随去南京扶立者,此时某一动,则淮、扬不守,天下事去矣。此功自让与开国元勋居之,必待南部议定;不然,我奉王入而彼不纳,必且互争,自不待闯贼至而自相残败事矣'!南京文武大臣齐集中军都督魏国公徐宏基第,议推戴讨贼。时惠王、桂王道远难至,潞王、福王、周世孙各避贼舟次淮安。马士英独念福王昏庸,可利为之;内贿刘孔昭、外贿刘泽清,同心推戴,必欲立之。移书史可法及礼部侍郎吕大器,谓:`以序、以贤,无如福王。已传谕将士奉为三军主,请奉为帝';且责史可法当主其议。可法、大器持不可。二十二日(己卯),可法治兵于浦口。二十六日(癸未),高宏图、姜曰广、李沾、郭维经、诚意伯刘孔昭、司礼太监韩赞周等复捱次进见,对勋臣恸哭;素衣角带,群臣行礼,皆以手扶,待茶款话,极其温和。言及迎立,即力辞;言`封疆大计,惟仗众先生主持'。五月戊子朔辰刻,福王自三山门登陆。由城外至孝陵,乘马;入西门享殿祭告,以东门内御路也。拜谒罢,问懿文太子寝园?遂诣瞻拜。既毕,从臣集朝内会议;独大器后至。时以潞王伦次稍疏,福王有在邸不类事,莫之敢决;李沾奋袂厉声曰:`今日有异议者,以死殉之'!刘孔昭、韩赞周力持之;孔昭又面骂大器不得出言摇惑。议遂定,乃以福王告庙。因先修武英殿,是日即具公启迎王;而可法督师江上犹未回。

四月二十七日(甲申),南京礼部司务齎百司公启迎福王于仪真,王得启即行。二十八日(乙酉),徐宏基等迎王于浦口。二十九日(丙戌),午刻,王舟泊燕子矶。三十日(丁亥),南京诸臣见王于舟次。王角巾葛衣,坐寝榻上;旧枕敝衾,孑影空囊。从行田成诸人,布袍葛履,不胜其困。王答兵部书谓:`国母尚无消息,宫眷未携一人;初意欲避难浙东僻地,迎立决不敢当'等语。从正阳门进城至东华门,步行过皇极殿,谒奉先殿,出西华门;以内守备府为行宫,驻跸焉。文武进见,王惶赧欲避。史可法`言殿下宜正受';遂行四拜礼。王传上殿,商战守之策;刘孔昭暨诸勋侯甚有德色。可法奏对良久,魏国公徐宏基、内守备各有奏;群臣乃退。是日,王辇所至,都民聚观;生员及在籍官,沿途皆有拱迎者。有云:先一日,两大星夹日;本日五色云见。

大星夹日、五色云见,似为南都之祥;而后事如此,是时摄政王初入燕,星云殆大清朝之瑞乎?

福王登极

五月初二日(己丑),南京诸臣谒福王于行宫。灵璧侯汤国祚以户部措饷不发为言,其词愤怒;太监韩赞周叱之起。吕大器呵言:`此非对君礼'!御史祁彪佳言:`纲纪法度,为立国之本'。吏科李沾言:`朝班宜肃'!彪佳又言宜早颁大号、敬天法祖诸事。王皆允之。群臣退,俱会议于内守备家;议监国、登极,咸以先行监国为便。张慎言曰:`国虚无人,可遂即位'。可法密曰:`太子存亡未卜,倘北将挟以来,奈何'?刘孔昭曰:`今日既定,谁敢更移?请正位'!可法曰:`缓几日无妨'。彪佳曰:`监国名正;盖愈推让,益彰王之贤德。且总师讨贼,申复国耻,示海内无因以自利之心。而江北诸大将使共预推戴,则将士亦欢欣;然后择吉登大宝,布告天下'。吕大器、徐宏基皆然之。遂定监国,以金铸``监国之宝''。

是日,诸大臣面奏劝进,王公百官升殿议。王辞让愈坚,谓`人生以忠孝为本。今大仇未报,是不能事君;父遭惨死,母无消息,是不能事亲:断无登位之理'。言讫涕泣。又言:`东宫及永、定二王见在贼中,或可致之;且桂、惠、瑞三王皆本王之叔,听诸先生择贤迎立'。科道官奏迎立之意,彪佳以人心天意为言。王逊谢如前,令百官退;止留兵部及内守备进内议事。少顷,再入班,上劝进第一笺,吕大器跪奏。王传旨:`暂领监国'。百官退。少顷,又进第二笺;王命传进,乃手书批答:`仍领监国,余所请不敢当'。

初三日(庚寅),百官朝服,王行告天礼。其祝文焚时,飘入云霄;众以为异。王升殿,百官行四拜礼,魏国公徐宏基跪进``监国之宝'';王既受讫,再行四拜礼,乃退。早间,有传得后日即登极者;史可法以人言摇摇,亦欲再劝进。祁彪佳力争,谓`监国不两日即登极,何以服人心'!乃止(``编年''、``遗闻''俱载初四日监国;而``甲乙史''及``日记''又载初三日进``监国宝'',姑从之)。

十一日(戊戌),群臣第三次劝进;王令旨:`这所启,予屡谕甚明,何又连章劝进!先生等惓惓忠爱,无非从宗社起见;予不忍固违,勉从所请。候择吉举行。该部知道'。十五日(壬寅)辰时,福王即帝位于武英殿;诏以明年为``宏光元年''。

附记:`时予入城,或问曰:`闻新皇帝止有八个月天下,信乎'?予曰:`未知也'。及明年五月帝遁,甫一载。而豫王率师南下,则春月也;天命所归,实止八月耳'。传为无锡费国瑄语;瑄颇通天文,顺治乙丑进士,今选余杭令。

宏光诏书

奉天承运皇帝诏曰:`我国家受天鸿祚,弈世滋昌;保大定功,重熙累洽。自高皇帝龙飞奠鼎,而已卜无疆之历矣。朕嗣守藩服,播迁江、淮;群臣百姓共推继序,跋涉来迎,请正位号。予暂允监国,摄理万几。乃累笺劝进,拒辞弗获;谨于五月十五日祗告天地、宗庙、社稷,即皇位于南都。猥以藐躬,荷兹神器。惟我大行皇帝英明振古,勤俭造邦;殚宵旰以经营,希荡平之绩效。乃潢池盗弄,钟虡震惊;燕畿扫地以蒙尘,龙驭宾天而上涉。三灵共愤,万姓同仇。朕凉德弗胜,遗弓抱痛;敢辞薪胆之瘁,誓图俘馘之功!尚赖亲贤僇力劻勷,助予敌忾。其以明年为宏光元年,与民更始,大赦天下。所有合行事宜,开示于后:

国政二十五款

一、在京文武各官,俱照原官加一级;无级可加者,晋勋阶一级,给与新衔诰命。在外督抚、监司、守令,俱照新衔给与。应得诰命,有碍于典制、封典不得自遂者,准请明移封。

一、前朝文武大臣,有劳绩可纪、品行可师而幽光未阐、谥荫未全者,该部即类题补。

一、在籍阁臣暨六卿之长,年六十以上者存问;其有遣配及闲住者,俱复原职。该抚按奏明存亡,三品以下先行豁罪;其中创艾日久、情可矜原者,着吏部行文抚按从公采访,列名报闻,分别酌用。该部亦发访单,确行察覈公论佥同,据实奏闻报用;不得藉端燃灰,致滋幸滥。

一、诸藩有流寓失所者,行各抚按善为安置;除南直不列藩封外,如浙之台州、处州、闽之邵武、汀州、广之南雄、韶州等郡,酌议来说。
一、宗室在南京者,名粮宜按时给发。其管理约束有年,准加敕优奖。

一、公侯伯共五十余人,北都沦陷,亡者甚多;今现在不过十余人。所有应得常禄,往日本、折三七关支或本、折中半兼支者,俱于折色中各给本色一半,每石折银七钱,以示厚意。

一、累朝及现在公王所出子孙,各荫一子入监读书;如无当荫者,准于原荫武职上量加一级,兵部即与题复。

一、七十以上年高有德者,府县申报抚按。已冠带者,仍作旌异;未冠带者,给与冠带。其细民,于元年量给膳米,以称朕养老至意。但不得因而徇滥,因而诈扰。
一、忠义殉难者,该抚按确察题明,准与赠恤、荫谥;还与建祠有年远未沾恩者,照例一体行;不得需索留难。
一、举人以字句蒙摈及停科者,俱准于宏光元年给凭赴部会试;其行止有碍、无关伦理者,该抚按奏明,一体会试。

一、各府州县廪生例得恩贡,务收真才,以需后用;不拘年例。

一、赴京举贡、生监等道途寇阻、资斧维艰者,合行考录,以疏淹滞。五年则减一年,举贡、监生仍照旧例行。

一、换授、保举、副榜特用等项,以后尽行停止。系副榜、廪生、监生出身或经荐过者,照旧量用;不得过抑,以塞贤路。

一、山林草泽下僚贱吏,有真正奇才异能、堪以匡时御乱者,除前谕已颁抚按行各属从公察报外,今仍着在京科道、阁部等衙门一体从公保举,确定人数,以凭拔用。如徇情贿嘱,事后发觉及试验罔效者,举主连坐。

一、北直、山东、河南、山西、陕西、辽东等处文武官生,义不从贼在南者,除文官现任废籍听吏部察明推升赴用外,其生员流寓无归、进取无路者,俱赴礼部报名;仍取乡官印结及各生互相保结,照各省直地方廪、增、附名色分寄应天府学,学臣一体考试作养,以示优恤士子之意。其武弁指挥、千百户等,如有真正袭替号纸脚色,许赴本部察验明确,准附在京各卫寄俸,勿令失所。

一、陷贼各官,本当戮窜;恐绝其自新,暂开一面:有能返邪归正者,宽其前罪;有能杀贼自效者,准以军功论。

一、钱粮屡赦宥,民未沾恩;在民者惟利顽户,在官者惟饱奸胥:朕甚悯焉!今于宏光元年,不论本色、折色,量蠲一分;其本色,仍改折二分。除北直、山西、陕西全免五年,山东、河南全免三年外,其江北、湖广蠲十分之五;其江西曾蹂躏地方,抚按察明,照四川蠲十分之三;其辽饷名色尽行蠲免。南粮作何归并?该地方官从长计议,务苏民用。俟大仇既复,朝廷尚有浩荡之恩。

一、漕粮原系永折地方,非比暂折、灾折内有虚粮、沙瘠、滩江等情。当时议折之故,已经酌处三、四;今后当还改折。其有罚兑副米等弊,尽行厘革。

一、新诏宽民间交易,如买置田产、房屋等项,皆民脂膏。先年税契不过每两二分、三分,今加至五分,吏胥索买契尾,又索加耗;且业主屡更,重复报税,不一而足。自后准以五年推收、十年大造为则,每两止取旧额三分。如未至期者,不许奸胥妄报,指诈害民。

一、开垦屯粮,屡旨激劝,未见成效;皆因新垦未熟,而催科迫之。自后不论军民人等,有能垦废为业、不费在官一文一粒者,即三年成熟后,据亩升科;止照本县额定升合,征取一半、永减一半,以示鼓舞招抚,永着为令。

一、内外监追、还官入官给主赃物问已成案已经完纳者,依例减等发落;其真正犯监追已故家属代禁、财产尽绝者,开其所犯情罪,奏请定夺,系给主赃径行豁免。

一、内外衙门现监囚犯,有情可矜疑及人命在辜限外者,覆审是实,比照热审例俱免死,发边卫充军;军改徒、徒改杖、杖释放。

一、小民罪犯各有正律,除真正强盗、人命法不应赎外,其余徒、杖、笞者折赎,原不定例。近因军兴费繁,院道府动云有司设处,凡一切讼狱,不论事之大小、曲直,但以犯者身家厚薄为差等,借题措饷。院道府官动罚千金、万金,州县官动罚数百石米或百石折银,以充军需;究无实济,致小民倾家破产,性命随之。如此虐政,宜痛革。

一、上供柴炭,该部酌量数目,专官采办;不许派扰商民。其供祀孝陵及诸祀典煎盐等项柴斤,仍照洪武二十六年例,龙江、瓦屑二关抽分;不得多抽,以滋商民之害。

一、恩赦以登极为准;诏到日,各抚按星速颁行各郡县,务令挂榜通知;仍刊刻成册,里甲人给一本。如官胥猾吏匿隐虚情支饰以图侵盗,诏差官同巡按御史访明究问。

于戏!宏济艰难,用宣九伐平邦之政;覃敷闿泽,并沛三驱解网之仁。新綍涣颁,前徽益懋;布告天下,咸使闻知。

崇祯十七年五月。

先是,初二日,诸臣议赦书蠲免。史可法曰:`今天下半坏,正赋有限;军饷繁费,恐未可尽除'。二十二日,淮抚路振飞宣``登极诏书''于民间,有新旧钱粮赦免之条;众情欢腾。

会推阁臣、冢臣及诸臣升擢

五月初二日,摄吏部史可法邀请诸臣会议阁员及冢臣。旧例:五府不入班行;时恐不和,乃共商之。群推可法及高宏图、姜曰广为相,而以冢宰属张慎言。慎言曰:`吾老矣!愿安于总宪'。徐宏基曰:`张公内阁、高公冢宰,似极相宜'。刘孔昭攘臂欲得内阁,可法曰:`本朝无勋臣入阁例'。孔昭曰:`即我不可,马士英有何不可'!诸臣默然。又议起废,竞推刘宗周、徐石麒等。孔昭特举阮大铖等;可法曰:`此先帝钦定``逆案'',勿庸再言'!初三日,马士英率高杰等拥兵临江,称十万众,欲威劫留都诸臣。文武臣会推史可法、高宏图、马土英皆东阁大学士,张慎言吏部、周堪赓户部各尚书;又推词林姜曰广、王铎为东阁。高、刘二帅书至,请可法渡江,欲其卸权于士英也。

初六日,兵侍郎吕大器转吏部;起练国事户部、贺世寿刑部、何应瑞工部各侍郎,刘士祯通政使。

初八日,起刘宗周左都御史。张慎言荐起顾锡畴等;起倪嘉庆、华允诚、叶廷秀补吏部司官。

初九日,马士英自请入朝,拜疏即行。

初十日,李沾、张元始、沈胤培、左懋第、钟斗、李清为都给事中,罗万象、陆朗、熊维典、张希夏、钱增、姜应甲、马嘉植、王士鑅、黄云师为给事中。

十三日,张国维原官协理戎政。起徐石麒左都御史、解学龙兵左侍郎、张有誉督仓侍郎、王廷梅应天府尹、郭维经府丞、朱之臣太常卿、左懋第少卿、李沾提督四夷馆少卿。吏部欲以李沾为操江,沾故善刘孔昭,惧分其任,乃求可法得典属国。维经积劳于扞掫,都人赖之;骤难其代,令仍摄巡视。沾嗾维经劾冢宰有私,旋知误引罪;而沾憾冢宰不已。以其清望,不敢显忤,因加色于少宰;吕大器思逐之。召王重补稽勋;时铨曹乏员,张慎言以重在金坛可立至,故推之。李沾言:`不可。是受我贽四十金者'。

慎言曰:`仆起家三十年,所贽十三金而止;公安得以四十金贽乎?仆老矣,须旧铨郎乃解事。又地近,其人廉否,仆自有提衡,不能混也'!沾益衔之。召谪籍科道章正宸、杨时化、裘恺、庄鳌献、熊开元、姜采、马兆羲、詹尔选、李长春、张瑄、郑友元、李模、乔可聘、李日辅等原官起用。

十四日,起许誉卿光禄卿。

十五日,进内官韩赞周秉笔司礼、卢九德提督京营。

十六日,命士英掌兵部。士英未尝奉召,自入朝;不欲出镇。史可法知其意,自请督师江北以避之。起顾锡畴礼部尚书,黄道周、高倬吏、工部各右侍郎,罗大任祭酒。左懋第右佥都,巡抚应天;侯峒曾左通政、郑瑄大理卿。
十七日,忻城伯赵之龙提督戎政。起田仰抚淮、扬。召楚督袁继咸陛见。

十八日,史可法辞朝,命文武官郊饯。自可法离京,刘孔昭略无忌惮,而高、张俱不能安其位矣。

二十日,可法开荐举人李遽主事、何刚军前监纪。

二十四日,设勇卫营,太监韩赞周节制之。都督徐大受兼总兵,郑彩分管水陆舟师。

二十五日,加恩翼戴诸臣:公徐宏基、伯刘孔昭、方一元、焦梦熊、郭祚永,侯朱国弼、柳祚昌、汤国祚,太监韩赞周、卢九德各陛赏世荫。

二十九日,以陈子壮为礼部尚书,徐汧、吴伟业少詹,管绍宁詹事,陈盟右庶子(``甲乙史'')。

``遗闻''云:以宗敦一为山东道、郑坤贞山西道、黄耳鼎广西道、梁士济江西道、周灿浙江道、周一敬福建道、潘世奇湖广道、王燮河南道、杨仁愿云南道、邓起龙贵州道、黄澍四川道(为楚监军)、白抱一陕西道;又贺登选、陆清原、任天成、霍达、左光先、李挺、刘达、吴文瀛、陈丹衷、阮正中、郑封、刘文渤、杨羽化、成勇等各点用。又调总兵官郑鸿逵、黄蜚镇守镇江,吴志葵驻防吴淞,黄斌卿驻防上江;敕御史祁彪佳等分行安抚江、浙。

马士英,字瑶草;贵州府贵阳县人,崇祯辛未进士。本广西梧州府藤县人。与袁崇焕同里,居北门街;又同辛卯年生。士英本姓李,五岁时,为贩槟榔客马姓者螟蛉而去,故遂从其姓。明末,予邑人有亲见马建坊于藤县,尚未就。其为人手长智短、耳软眼瞎者。

王铎,字觉斯;河南孟津籍,山西平阳府洪洞县人。

张慎言,字藐姑;山西阳城人,万历庚寅进士。
姜采,字乡野;山东莱阳人,崇祯辛未进士。初为仪真令。

张煊,字葆光;山西介休人,崇祯戊辰进士。陕西道御史。

白抱一,字函二;北直南和人,保举恩贡。林县令。

乔可聘,字圣臣;南直宝应人,天启壬戌进士。

陈丹衷,号涉江;应天人,崇祯癸未进士。御史。

史可法请设四镇

五月十二日(庚子),史可法言:`从来守江南者,必于江北当酌地利,急设四藩。以淮、扬、泗、庐自守,而以徐、滁、凤、六为进取之基;兵马、钱粮,皆听自行征取。而四藩即用黄得功、高杰、刘泽清、刘良佐,为我藩屏,固守江北;则江南之人情自安。黄得功已封伯,杰、泽清、良佐似应封伯;左良玉恢复楚疆,应照得功封侯。马士英合诸镇之功,爵赏似难异同;卢九德事同一体,听司礼监察叙'。十七日(甲辰),可法又奏``四不可无'';疏曰:`臣与高宏图、姜曰广、马士英谨议得新增文臣,有协理戎政、协理操江二员;新增武臣,有京口、九江二镇。此外,则上江抚臣,现议增设。又议得江北与贼接壤,遂为冲边,议设四镇,分辖其地。有四镇,不可无督师;督师应屯驻扬州,居中调遣。其四镇,则设于淮扬、徐泗、凤寿、滁和,各自画地。封总兵官刘泽清东平伯,辖淮、海,驻于淮北;以山阳、清和、桃源、宿迁、海州、沛县、赣榆、盐城、安东、邳州、睢宁十一州县隶之,经理山东一带讨招事。封总兵官高杰兴平伯,辖徐、泗,驻于泗水;以徐州、萧县、砀山、丰县、沛县、泗州、盱眙、五河、虹县、灵璧、宿州、蒙城、亳州、怀远十四州县隶之,经理河北、河南开、归一带招讨事。总兵官刘良佐广昌伯,辖凤、寿,驻于临淮;以凤阳、临淮、颍上、颍州、寿州、太和、定远、六安、霍邱九州县隶之;经理河南陈、杞一带招讨事。晋靖南伯黄得功靖南侯,辖滁、和,驻于庐州;以滁州、和州、全椒、来安、含山、江浦、六合、合肥、巢县、无为州十一县领之,经理光、固一带招讨事。各设监军一员,一切军民皆听统辖、州县有司皆听节制、营卫原存旧兵皆听归并整理。所辖各将,听督师荐举题用。荒芜田土,皆听开垦;山泽有利,皆听采开。仍许各于境内招商收税,以供军前买马、制器之用。每镇额兵三万人,岁供本色米二十万、折色银四十万,悉听各属自行征取。所取中原城池,即归统辖。寰宇恢复,爵为上公与开国元勋同,世袭。贼在河北,则各镇合力协防淮、徐;贼在河南,则各镇协守泗、凤;贼在河北、河南并犯,则各镇严兵固守。其凤阳总兵,应改副将一员。计共六百余万,及察每岁所入约米二百四十万、约银五六百万,除各兵支用外,所存亦自无多也。所望诸臣核实兵、实饷之中,为实战、实守之计;御于门庭之外,以贻堂奥之安。则中兴大业,即在于此矣'。

此亦寓调于进取之意。愚谓即效古藩镇法,亦当在大河以北开屯设府;岂堂奥之内,而遽以藩篱视之!

时高、刘等封伯爵,而黄、左晋侯爵,仍荫一子锦衣卫正千户世袭。又旨云:`马士英保障东南,肤功更着;着加太子太保,荫一子锦衣卫指挥佥事世袭。卢九德功一体,着司礼监从优议叙'。

黄得功
黄得功,字虎山。貌伟,胡髯两颐倒竖;膂力绝伦。微时,驱驴为生计。有贵州举人杨文骢、周祚新北上,于浦口雇其驴;初不知为豪杰也。道经关山,突遇响马六人。文骢、祚新等亦闲弓马,欲与之敌;得功大呼曰:`公等勿动,我往御之'。时杨家人亦颇材武,已于驴背跃下,行李与牲口重数百斤,得功一手挟驴,一手提行囊,突扑响马,响马大惊,乞止之;且曰:`有言相告'。得功不听,扑击如故。响马急齐下马罗拜,`老兄真英雄,吾辈愿拜下风,勿失义气'!得功方止;亦拜曰:`我不愿为此,只放吾等过去可也'!响马请姓氏,得到坚不与言;既而曰:`姓黄,呼为黄大'。响马遗以金;得功不受,乃去。杨、周两孝廉见其勇而有志,待如兄弟。及南回,告于马士英。士英觅至,为之婚娶;延武士,教以兵法。及莅任凤阳,即用为旗鼓。堵截流寇,建功河北,升副总戎。军中尝乘黑驴,呼为``黄大刀'',甚畏之。于是庐、凤一带,贼不敢久驻。

附记遗事:大学士蒋德璟曰:`挂印总兵爵虽大,然庭参宰相时,相不出迎,居位受拜;拜讫,相始出接,延入后堂,乃行宾主之礼。时高、刘三镇皆行此,独得功来见,拜入后堂,仍行跪礼。此固忠义之气,亦以昔日在吾门下故也'。盖武臣曾于文臣门下奔走出身者,后虽贵显,必行跪见礼。太祖旧制:凡宰相阅边,虽总兵封侯必戎服庭参,揖于槛外;所以尊相体也。宏光时,史相督师,四镇将谒,私议见礼;得功曰:`有旧制在'。高、刘等曰:`吾辈已封侯伯矣'。得功戎服先入,高、刘不得已,亦戎服继之。于此知得功勇而知义。

得功贫时,豢鸭池塘,其数日减,久之几尽。得功怒,将水戽竭,捕一巨鳝,长可数尺、围五寸许;烹食之。体貌倏易,顷成伟丈夫,亦不自觉力多也。及浴,手绞巾布,忽断裂;始知之。予叔君衡公昔在仪真,闻门外喧闹,出见所舁铁鞭,鞭每重三十斤、双鞭则重六十斤矣;此得功在马上所运者。得功有猎犬三十随马走,甚捷。在六合时,每使小卒以金锣戴额上,得功射之,百发百中,而人不伤。众呼为``小由基''。

得功善饮细酒和火酒,可饮五十斤。临阵时,以扎巾紧缚,目瞳突出;饮半酣,方入阵,所向无前。扬人云:`得功驻仪真,治兵严整。曾遣四十骑白事于史相,道经高营被劫;得功怒,卒兵驰扬,杰与战。时杰兵盛,得功被围;适弟黄蜚等至,杰兵始退。已复战,得功临阵,杰部将号黑虎子者最骁勇,出战;蜚发烟铳,黑虎子目眩,得功鞭碎其首而死。杰惧收兵;适史相至,杰偿得功马,始罢兵'。战场在扬州城外荒地埂子上;然埂子街进城即是,非城外也。

初,仪真举人李洪甲宦囊甚厚,营建壮丽。有相者曰:`此屋必出一封侯者居之'。传至于孙,适得功莅镇,居其宅凡九阅月;而形家之言始验,亦异矣。得功贱时,有饭肆老妪厚遇之;得功感其意,拜为母。及贵,挈至仪真。妪卒,葬于方山,四镇合兵数万送之;旌旗蔽野,仪卫眩目,郡邑荣之。初,得功在河北,阵前马惊几蹶;适一人持之得不堕。得功问之,对曰:`小卒任姓'。问其名,对曰:`无'。得功见其严冬尚无裤,即名之曰``有裤'';意欲厚酬之也。及得功镇仪真,任已为锦衣守备,驻六合矣。未几,升参将;继升副总戎,赐蟒玉。

刘泽清

刘泽清,字鹤洲。白面朱唇,貌颇美。崇祯时,为总兵官。癸未七月,请于青、登诸山开矿煎银;着巡抚设法。甲申二月,移镇彰德。贼警急,召吴三桂、唐通与泽清等将兵入援;三桂、泽清不奉诏。三月,兵科韩如愈奉差至东昌,泽清遣兵杀之;曰:`尚能论我主将否也'?为山东总兵,虚报捷,赏银五十两;又诡言堕马被伤,复赏药资四十两。命即扼真定,泽清不从;即于是日大掠临清。统兵南下,所至焚劫一空。三月十六日上泣,方、魏二相请对封清为安东伯,上不应。五月十二日,泽清以数百人大掠瓜州。淮安自路振飞、王燮同心戮力,颇成巩固。振飞去后,泽清突来盘踞,散遣义士,桀骜者藉之部下,抢劫村落一空。与淮抚田仰,日肆欢饮。北兵南下,有问其如何御者?泽清曰:`吾拥立福王而来,以此供我休息;万一有事,吾自择江南一郡去耳'。八月,泽清大兴土木,造宅淮安,极其壮丽;四时之宝俱备,僭拟皇居。休卒淮上,无意北往。田仰犹屡为请饷;宏光以东南饷额不满五百万,江北已给三百六十万,岂能以有限之财、供无已之求!命仰与泽清通融措辨。

泽清曾杀其叔副总兵刘孔和。孔和,故大学士鸿训子;泽清初为其狎客,及后势盛,反抑孔和属役之。一日,以所作诗示和曰:`好否'?孔和戏曰:`不作尤好'。泽清色变。无何,遣孔和以二千人渡河;忽檄召至,斩之。所部二千人,汹汹不服;令别将击斩之,无一人存者。其凶暴如此。

昔霍去病云:`匈奴未灭,无以家为'!李西平云:`天子何在,敢言家乎'?宜其立大功、成天下大名也。泽清当乾坤颠覆、大敌在前之时,即卧薪尝胆,犹惧不济;乃大兴土木,真处堂燕雀耳。愚昧若此,堪为将乎?他事抑勿论矣!

刘良佐

刘良左,字明辅;大同左卫人。初与高杰同居李自成麾下,杰护内营、良左护外营。后杰降,良左亦有归朝意。未几,降。崇祯十四年,曾破贼袁时中数万众,历官至总戎。素乘花马,故世号``花马刘''里。

先君子云:`昔刘良佐未显时,居督抚朱大典部下。忽为所知,加以殊恩,屡以军功荐拔,遂至总戎;亦一遇也'。

甲申六月六日(壬戌),刘良佐开镇临淮,士民张羽明等不服,临淮士民戈尚友等亦奏叛镇环攻;命抚按调和之。

高杰

高杰,字英吾;米脂人。初为李自成先锋;后与自成后妻邢氏通,惧诛,遂偕以归降,隶秦将贺人龙麾下。孙传庭督秦中,令杰与白广恩为前锋。二将各不相下,遂溃;潼关不守。

甲申春,调赴李建泰军前;未至,闻建泰兵溃,遂抢河东一带,由山西、河北率兵南下,大肆劫掠。抵扬,欲入城,扬人畏惧,为罢市,登陴死守。杰攻之,多杀掠。四月二十八日(乙酉),杰围扬州困之。五月初五日(壬辰),杰兵大掠江北,声言欲送家眷安顿江南;约刘泽清刻日南渡。史可法议发户部一万两,遣职方郎万元吉前谕各镇,分别犒赏。初六日(癸已),太仆少卿万元吉言:`扬州、临淮、六合,所在兵民相角。在兵素少纪律,在民近更乖张。遂致一城之隔,民以兵为贼,死守不容;兵以民为叛,环攻弗释。猝有寇至,民必至于惊窜,真今日莫大之忧也。江北郡邑接连山东、河南,贼骑处处可到,势必需兵堵剿;臣等虽有爱民之心,无销兵之术。就中调停,惟是官兵经过驻札地方,使城外居民尽移城内,空下房屋,听各将领派兵住宿,严禁毁伤;其蔬菜等项,仍谕城内居民尽出城外有无贸迁,有司会同各将领共相防护,严禁抢掠。如此立法,自然民不苦兵、兵不恨民。臣前监军楚、蜀时,行之甚效。其在今,何独不然'云云。

万疏,``大事记''载六月二十四日;而此则从``甲乙史''也。予闻史督辅行师时,亦令贤能将领预往歇宿去处,将房屋料定,安置兵将若干;分贴标明,书``某营某将宿此''。到则认标而止,无有抢攘;此良法也。

五月七日(甲午),扬州士绅王傅龙奏:`东省附逆,河北悉为贼有,淮、扬人自为守。不意贼警未至,而高兵先乱。自杰渡河掠徐,至泗、至扬,四厢之民何啻百万;杀人则积尸盈野、污淫则辱及幼女。新旧城环围,绝粮已经月余。何不恢已失之州邑而杀自有之良民也'!十六日(癸卯),杰屯兵扬州城下。淮抚黄家瑞漫无主张,守道马鸣騄昼夜督民守城,集众议事;进士郑元勋与杰善,亲诣高营解纷。遂入城,劝家瑞放高兵入城,便可帖然。谓杰有福王札,命驻扬州;宜善御之,毋撄其暴乱。士民哗曰:`城下杀人如是,元勋不见耶'?元勋强为杰辨,众怒指为杰党;且曰:`不杀元勋,城不可守'。遂寸斩之城楼。鸣騄疾走泰州。杰恨;攻益力。史可法以义喻解之,始移驻瓜州。及设四镇,杰卒驻扬:泽清驻淮,良佐驻凤、泗,黄得功驻庐。得功心薄之,因提兵争淮、扬,与杰战;不胜。朝廷闻之,升万元吉太仆少卿,监江北军解之,始各罢兵。隶杰于史可法标下,为前部总兵官。

``甲乙史''云:五月十八日(乙巳),万元吉言:`臣奉命犒师,沿途兵言构祸,寸步皆阻;扬州民尤甚,闭城登陴已十余日。乃兵与民相杀,民又与兵相杀;成何纪律?顷接水营将张士仪言:``寇奔清河,官兵击烧贼舡殆尽。若高、刘、黄将潜师以济,一鼓歼之,即可称中兴第一功也'''。初,黄得功分地扬州,高杰、刘泽清以繁富争之;纵兵淫掠,扬人大哄。得功兵至天长,杰、泽清欲拒;又值李栖凤、高文昌兵至,众心汹汹。元吉移得功书,期共戮力王室;得功自明无他,欲联络各镇鼓勇杀贼。元吉以得功书驰示杰等,始肯相戢。然杰部悍,终不自制。

二十三日(庚戌),高杰疏言:`奉旨分防,扬仪人登陴、罢市,抚道不出;伪将董学礼又入宿迁。臣进退无所,乞赐应住何地'?

六月初二日(戊午),扬州难民盛运开奏扬民横遭焚劫;上谕以`百姓当仰体朝廷不得已之意;该镇忠勇名帅,督辅既到自妥'。

初六日(壬戌),史可法以高杰悍不可制,身入其营谕之。见留不能出,尽夺其兵,仆从多散。自是,章奏俱经邀阅,权遂不振。

``大事记''云:六月初八日,史可法奏``悍民惨杀乡绅疏''曰:`镇臣高杰之率兵南下也,扬人实未预知。初到之时,不无骚扰;及镇臣既至,取犯兵斩以徇,日不下数十人,地方官可以谅矣。乃抚臣黄家瑞漫无主张、道臣马鸣騄一味偏徇,听城中百姓日守河边草际,取零兵而杀之;用是结衅愈深,竟不可解。乡绅郑元勋亲到高营,所以为百姓之心无所不至。而百姓反谓通同播害,乘元勋一言之误,当抚臣坐次操戈而群杀之,至于碎其身首;抚臣之威令谓何?至于道臣始则乖张、复又畏缩,今避于泰州矣。骂兵、杀兵以为爱民,而不知适以害民!臣于二臣,不能无憾。伏乞敕下处分,以谕三军、以谕百姓;一面察其首恶,一重创之,庶几纲常不至尽坏'。上谕:`该部议处'。

黄、马二公为地方受过,父老诣阙保任;上优诏恕之。已复乱臣正法,黄公为元勋恳恤;尤见厚道。

二十六日(壬午),史可法奏``兵民两便书''曰:`镇臣高杰之兵,奉旨驻扬,而扬人坚不肯纳。盖从前既有仇隙,则向后不无提防;虽严令驱之,不能动也。臣前急于渡江,原欲了当此事,即当讨贼西行;不意兵民扞格,竟不能解。扬人惟利兵去,各兵惟愿驻扬;而好事者遂造为不根之言。如镇臣黄得功到仪真,本为安插家眷;而谓之曰:`此乃与尔兵为难者'!于是高兵移扎于野以待之。及臣至。则又谓之曰:`此来非真心为尔'!以致兵疑臣、将疑臣,即镇臣杰亦似疑臣;臣惟处之以坦、待之以诚。数日之间,镇臣杰亦似谅臣心事矣。昨与臣面议,将兵尽驻城外,止镇臣家眷入城,携二、三百人自护;臣以为可行矣。而城内之人终不允;臣正踌蹰无计,适有为移驻瓜州之说者。瓜州距扬州仅四十里,即江都县所辖也;驻瓜州,犹之驻扬州。且有城、有水可以自卫,而资给日用较之扬州尤便焉。惟时为镇臣刘泽清标下官兵所驻,必刘兵移住淮上,而后高兵可来。臣商之镇臣,镇臣遂诺;盖深感皇上恩遇之厚,不欲以家口之故,致成兵民水火之形、耽误练兵剿贼之事也。镇臣用意如此,臣甚重之。镇臣在瓜州、臣在扬,调停于兵民之间,渐为释其猜嫌,同归于好,未必扬州之必不可居也'。

郑元勋,字超宗;南直歙县人,籍扬州。天启甲子,乡魁。崇祯癸未,进士第三人;旋假归。高杰至扬,扬人闭门拒守;杰怒,将攻城。公单骑往谒,陈说大义;杰掣兵于五里外,以待犒赏后行。越日,暂启两门;乃好事者复取城外游兵翦之,以利其橐。杰益怒,积不可解。公请迎原任蓟督王永吉至郡,往为纠纷。杰以扬民先杀起衅为辞,且请与中丞约:曲在兵者,镇斩之;若曲在民者,抚斩之。永吉以杰言传覆公。二十五日,公登城南,与抚道议事,万众俱集。公谓如高杰言先杀启衅,诚当禁惩;否则,构祸且不测。众以高兵杀人,罪不容逭;公曰:`亦有杨诚戕贼者,岂尽由高镇耶'?言未毕,渠魁张自强、王柱万、陈尝等大呼:`郑宦通贼,曲为解宽;吾侪若不下手,势必尽遭屠灭'!于是利刃攒集,遂遇害。义仆殷报以身护主,同被创死。盖营将有杨诚者,标兵多不法,往往杀越人于货,故指及之;而众误认``杨诚''为``扬城'',公遂及于难。先五日,南都授公兵部职方主事,竟未及拜官云。史可法疏参;越数日,抚臣斩前三渠魁于市,并杖其党毙之。自后,扬人常夜见公于城上,峨冠、绯袍指挥而过,若天神然。诗画妙天下,所著有``读史论赞''、``英雄令终录''、``英雄恨''、``左国类函''、``文娱初、二集''行世。

附记:杰得城内百姓则杀之,若居城外者截右耳,杀人甚众;米物腾贵,民不聊生。扬之属邑泰兴,故抚朱一冯家在焉。杰兵入,启地三层,得藏金八十万而去;朱以是贫困,将所居宅庐悉鬻于同邑绅士李寓庸云。
刘宗周论时事

甲申六月,起刘宗周都察院左都御史。初十日(丙寅),宗周三抗疏论时事,止称``草莽孤臣'',不署衔。首言大铖进退,关系江左兴亡。又言:`讨贼之法有四:一曰据形胜以规进取。江左非偏安之业,请进而图江北;今淮安、凤阳、安庆、襄阳等处虽各立重镇,尤当重在凤阳而驻以陛下亲征之师。中都固天下之枢也,东扼淮、徐,北控豫州,西顾荆、襄,而南去金陵亦不远。以此渐恢渐进,秦、晋、燕、齐当必响应。兼开一面之网,听其杀贼自效,贼势益孤、贼党日尽矣。一曰重屏藩以资弹压。地方之见贼而逃也,总由督抚非才,不能弹压。远不具论,即如淮、扬数百里之间,两节钺不能御乱贼之南下,致淮北一块土拱手而授之贼。尤可恨者,路振飞坐守淮城,以家眷浮舟于远地;是倡之逃也。于是镇臣刘泽清、高杰遂相率有家属寄江南之说,尤而效之;又何诛焉!按军法:临阵脱逃者斩;臣谓一抚、二镇皆可斩也。一曰慎爵赏以肃军情。今天下兵事不竞极矣,将悍兵骄已非一日。今请陛下亲征所至,亟问士卒甘苦而身与共之,乃得渐资腾饱、徐张挞伐。一面分别各帅之封赏,孰应孰滥?轻则量收侯爵、重则并夺伯爵。军功既核,军法益伸;左之右之,无不用命。夫以左帅恢复焉而封,高、刘败逃也而亦封;又谁为不封者?武臣既滥,文臣随之;外廷既滥,中珰从之。臣恐天下闻而解体也。一曰覈旧官以立臣纪。燕京既破,有受伪官而逃者、有在封守而逃者、有在使命而逃者,于法皆在不赦;急宜分别定罪。至于伪命南下,徘徊于顺逆之间者,实繁有徒;尤当显示诛绝。行此数者,于讨贼复仇之法亦略俱是矣。若夫邦本之计,贪官当逮、酷吏当诛;循良卓异,当破格旌异,则有安抚之使在。而臣更有不忍言者,当此国破君亡之际,普天臣子皆当致死;幸而不死,反膺升级,能无增天谴!除滥典不宜概行外,此后一切大小铨除,仍请暂称``行在'',少存臣子负罪引慝之诚'!又疏言:`贼兵入秦逾晋,直逼京师;大江以南,固晏然无恙也。而二、三督抚曾不闻遣一人一骑北进,以壮声援;贼遂得长驱犯阙,坐视君父危亡而不之救:则封疆诸臣之宜诛者一。既而大行之凶问确矣,敷天痛愤;奋戈而起,决一战以赎前愆,又当不俟朝食。方且仰声息于南中,争言固圉之事;卸兵权于阃外,首图定策之功。督抚诸臣仍复安坐地方,不移一步:则封疆诸臣之宜诛者二。然犹或曰事无禀承;迨新朝既立,自应立遣北伐之师。不然,而亟驰一介,使齎蜡丸间道北进,或檄燕中父老共激仇耻,哭九庙之灵,奉安梓宫;兼访诸皇子的耗,苟效包胥之义,虽逆贼未始无良心。而诸臣计不出此也!又不然,亟起闽帅郑芝龙,以海师直捣燕都;令九边督镇卷甲衔枚,出其不意,事或可几。而诸臣又不出此!纷纷制作,尽属体面。天假之灵,仅令吴镇诸臣一奏燕京之捷,将置我南中面目于何地?则举朝谋国不忠之宜诛者三。而更有难解者,先帝升遐,颁行丧诏,何等大典!而迟滞日久,距今月余,未至臣乡;在浙如此,远省可知。时移事换,舛谬错出;即成服祗成名邑,是先帝终无服于天下也:则今日典礼诸臣之宜诛者四。至罪废诸臣量从昭雪,自应援先帝遗诏而及之。乃一概竟用新恩,即光帝诛珰铁案,前后诏书蒙混,势必彪虎之类尽从平反而后已。君、父,一也;``三年无改''之谓何?嗟乎!已矣。先帝十七年之忧勤,念念可以对皇天、泣后土;一旦身殉社稷,罹古今未有之惨,而食报于臣工乃如此之薄!仰惟陛下再发哀痛之诏、立兴问罪之师,请自中外诸臣之不职者始'。

七月十一日(丙午),刘泽清、高杰劾奏刘宗周:`劝上亲征,以动摇帝祚;夺诸将封,以激变军心。不仁不智,获罪名教'!又三十日(已卯),刘良佐、刘泽清各疏参刘宗周劝主上亲征,为有逆谋。八月初二日(丁已),高杰等公疏,请加宗周重僇,谓疏自称草莽孤臣为不臣。既上,泽清以稿示杰;杰惊曰:`吾辈武人,乃预朝中事乎'!疏列黄得功名,得功又疏辨实不预闻;马士英阴尼之,不得上。士英仍拟旨云:`宪臣平日原以议论取重';盖刺之也。廷议欲谴高、刘而莫可谁何,欲罪宗周而难违清议;史可法因疏,两解之曰:`廷臣论是非、宪臣论功罪,各不相碍'。二十日(乙亥),刘泽清复捏四镇公疏纠姜曰广、刘宗周谋危社稷。九月初十日(乙未),宗周致仕。次日,户科陆朗内批留用。先是,陆朗与御史黄耳鼎以例外转兵备佥事,计无所出;乃疏攻宗周,因而复职。尚书徐石麒言:`朗赃私应劣转、交通内臣,传留非法'。朗即劾`石麒结党欺君,把持朝政,无人臣礼'。宗周于七月十九日(甲辰)到任,至九月初十日致任;凡任都察院左都御史四十九日。

刘孔昭陵侮张慎言

刘孔昭,号复阳,浙江人;袭诚意伯,官操江。孔昭弑其祖母胡氏。胡为刘尚忠继妻,生莱臣;而孔昭父荩臣,为出婢莫氏巧云所生。刘尚忠没,莱臣应袭嫡嗣;以幼,为荩臣僭袭。荩臣没,孔昭又冒袭之,遂赠莫氏为伯夫人。及官操江,遂捕莱臣毙之狱;恶胡氏出揭,并缢杀之。真大逆不道者哉!

至甲申五月,议起废,孔昭故善阮大铖,特举之;史可法不从。及十八日(乙已)可法离京,二十二日(乙酉)马士英入直,孔昭必欲起大铖。自诏有``逆党不得轻议''之语,而张慎言秉政持正,孔昭度难破例,置酒酌诸侯伯廷论之,必欲廷论逐去冢臣,而后可惟我所欲为;灵璧侯、忻城伯皆诺之。时慎言条议:`北来诸臣虽屈膝腼颜,事或胁从,情非委顺;如能自拔南来,酌定用之之法'。因荐原任督师大学士吴甡、吏部尚书郑三俊;有旨:`赦甡罪,陛见;三俊俟另议'。二十三日(庚戌),早朝毕,孔昭挈汤国祚、赵之龙诸勋臣呼大小九卿科道于廷,大骂慎言,欲逐之去;谓`雪耻除凶、防江防河,举朝臣子,全副精神宜注于此。乃今日讲推官、明日讲升官,排忽武臣、专选文臣,结党行私。所荐吴甡,有悖成宪;真奸臣也'!慎言立班不辨。大学士高宏图言`冢臣自有本末,何遽殿争'!上谕:`文武官各和衷,勿偏竞'!孔昭袖中取出小刃,逐慎言于班;泣陈丑詈,必欲手刃之。太监韩赞周叱之,言曰:`从古无此朝规'!乃止。御史王孙蕃曰:`先帝裁文操江、归武操江,亦未见作何事业;且吏部职司用人,除推官、升官外,别无职掌'。喧争殿上。慎言即引疾乞休。孔昭退,奏:`慎言推补幸滥,举荐吴甡、郑三俊更为可异。慎言原有二心,告庙定策,阻难奸辨,不可不诛。乞大奋干断,收回吴甡``陛见''之命,重处慎言欺君误国之戒'。二十四日(辛亥),高宏图奏言:`文武各官有职掌,毋得侵犯;即文臣中各部不得夺吏部之权。今用人乃慎言事,孔昭一手握定;非其所私,即谓之奸,臣等皆属赘员矣。慎言荐甡,勋臣知为不可,臣不能知;票拟实出臣手。又三俊清刚,系五朝人望,臣终以为不可不用;是臣罪不减慎言。窃念朝廷之尊,尊于李勉;天子之贵,贵以叔孙。臣忝辅弼,坐视宸陛几若讼庭,愧死无地。请赐罢斥'!姜曰广引疾求去。上遣鸿胪官各谕留。

二十六日(癸丑),上召辅臣高宏图、姜曰广、马士英于行宫;上谓宏图曰:`国家多故,倚赖良多;先生何言去也'?对曰:`臣非敢轻去;第用人一事,臣谓可、勋臣则谓不可,是非淆乱,臣何能在位'。上曰:`朕于行政用人未习,卿等所言无一不从,勿疑有他'。宏图曰:`冢臣张慎言清正有品,吏部以用人为职,无一日不用人,是无一日不修职也。如推刘宗周、黄道周,使勋臣处之,亦必藉重;何独以为罪?吴甡前任抚按,俱有声名清望,先帝简在内阁;督师稍缓致谴,先帝杀延儒、不杀甡,即可知其人。假先帝在今日,亦必用之;何勋臣以此罪冢臣也'!宏图又言近臣贪黩状;上曰:`朕固闻之;诸臣通赂出之袖中,诚可嗤也'!时屡召对,先后无虚日,或一日再召;上亦有意为明主。至马士英当国,直高拱听之,不复知外边事矣。

二十七日(甲寅),张慎言请亟求罢斥,以服世臣之心。李沾言:`勋臣愤激有因。当中府聚会,马士英手札移吕大器迎立皇上,赞周、孔昭无不允协。黎明集议,大器绾礼、兵二部,纡回不前。臣等十九人以名帖延之,从容后至。议至日中不决,孔昭怒形于色、臣与郭经维、陈良弼、周元泰、朱国昌历阶而上,面折大器。赞周云:``快取笔来'';因得俛首就盟。清晨迎驾,大器尚欲停待;而赞周已登舟矣,偕行者徐宏基、陈良弼、朱国昌也。孔昭拥戴有功,文臣启事屡登、武臣封爵未定,所以有殿上之争'。二十八日(乙卯),慎言具疏求去云:`臣按河南时,曾劾布政冯明盛倡逃。其子冯铨作相,嗾其门生曹钦程参周宗建、李应升、黄尊素以及臣;三臣皆死狱,而臣戍肃州。先帝召升刑部侍郎,以拟狱不当,闲住十余年而复起。今待罪铨曹二十日,遂为孔昭所指;止有一去而已。吴甡、郑三俊阁臣荐于前,科臣荐于后;两人者行己有耻,臣能保之。孔昭指为小人,亦硁硁小人,非反覆之小人也。伪官至阳城,臣子履旋投崖而死,孤孙尚幼。国难家变,恸无生理;臣当与缁黄为侣矣'。

六月初二日(戊午),上命吏部司官敦促慎言视事。

初六日(壬戌),史可法言:`先帝用人原无成心,傅宗龙、孙传廷起自累囚,张凤翔、袁继咸、马士英起自戍藉。当吴甡奉命南征,以候唐通兵不至,迟则过之;所可原者,国难之作,勋臣之殉国者谁?孔昭何不思之!慎言以七旬冢宰,一举吴甡,便以为罪;不益轻朝廷而长祸乱耶'?初八日(甲子),奖谕刘孔昭``功在社稷''。

初十日(丙寅),张慎言致仕。上谕曰:`晋疆不复,卿已无家可归;沿途侨寓需召'。慎言遂止于宁国。孤孙间关来侍,慎言曰:`祖孙相聚足矣'!国亡后,慎言郁郁卒;孙扶榇反葬故里。

甚矣!刘孔昭之狂悖小人也。始也,弑叔、弑祖母,固已绝灭人理矣。既乃以武操江,欲手刃铨部于朝;其无忌惮若此,将置南国君臣于何地?犹赖王孙蕃、韩赞周等正言折之耳。然孔昭之敢于有此举,亦由马士英为之助耳。

路振飞、王燮镇抚淮安
甲申春,山西逃兵南下,江北震恐;淮抚路振飞遣金声桓等十七将率兵分道防河及守徐州。三月十三日,言`淮徐道何腾蛟整顿徐方有功,今升楚抚;有同知范鸣珂可补缺'。
二十七日(乙卯),路振飞令淮安七十二坊各集义兵,每家或三、或五,刀杖俱自备。每坊一生员为社长、一为副;自为操演,贵持久、戒作辍。日则团练,夜则鱼贯巡逻,以备非常。是日大阅,举人汤调鼎等咸易戎服。

二十九日(丁巳),淮上始传京城陷。振飞分设壮丁守城,拈分守门官;日则各守一门,夜宿城楼。四月(戊午)朔,淮城义士到军门过堂,振飞赏以花红每人银一两;人人踊跃,耀武于河上。适有北来逃兵骚扰,见之辟易避去。

初九日(丙寅),振飞集淮城绅士议事。至则出塘报于袖中,言`京城已陷,代我者即至;将缚我出迎乎?抑勉力一守乎'?言毕泪下,众皆泣。散漕粮四千石于民,擒伪官胡来贺、宋自诚、李魁春,沈之于河;斩叛将赵洪祯等,又擒癸未进士伪防御使武愫,解京。伪制将军董学礼袭据宿迁,振飞遣监城王守备率兵击破之;获学礼及从者十三人,悉斩之。乃与按臣王燮同心固守。

燮字雷臣,顺天宛平籍,湖广王陂人。崇祯庚午举人、丁丑进士;三代锦衣卫指挥同知。通``春秋'',夏允彝常称其有经纬大才。初任河南祥符令,三守危城,才识、胆力无不超绝。甲申三月初九日莅任淮安,与振飞并着劳绩。有伪选淮安知府巩克顺行牌至淮上,写``永昌元年二月二十二日给'';燮碎其牌,捆责其人,逐之淮口。擒克顺,斩以徇众。燮自任守河,托振飞守城,士民恃以屹然。

三月二十一日,刘泽清兵顿宿迁、高杰兵顿徐州,各声言南侵;淮民大怒。燮自谓与泽清有识,轻身诣之,劝其迁辕北上。泽清不肯,大声云:`即不扰贵治,请假道赴扬州'!燮不可,曰:`即不得已,迂道从天长、六合,则非我所知也'。泽清允之,淮城得免涂炭。

四月初三日(庚申),伪防御使吕弼周遣牌至淮代振飞,燮捆责其人。弼周者,原任河南驿传道,为燮座师也。十五日(壬申),弼以师生视燮,携伪参将王富赴任。游击骆举知燮本意,乃迎之于中途;火猝缚之。燮叱使跪,弼周骂曰:`人也!不认'?燮曰:`乱臣贼之,我认得谁'!令左右截其耳;细鞫其贼事,并问以圣上、东宫,弼周一字不答。解至抚院,振飞命留驿亭,悬示四门,令善射者竞集。振飞举觞劳骆举,簪花旁立。缚吕弼周、王富于柱,射者二十步外,五人为耦,人发一矢;不中者退,中者报名赏银牌一。射者尽,乃命剐之。众悦,诣肆快饮。

五月初五日,淮坊义士擒兵三十余人,振飞不敢问;纵之。十三日,马士英官兵由淮赴江达南京,共一千二百船;王燮驻清江浦,令淮坊义士排立两涯,不许一舟停泊、一人登岸,凡三日而毕。二十二日午刻,太监卢九德引兵一千欲进城中,士民大震;振飞再三求免。

二十三日,刘泽清奉旨驻淮安;未至,士民皆惧。二十四日,泽清驻兵盱眙;抚按集议,振飞、燮不行。

二十五日,伪官武愫解至抚院,振飞于愫有旧,不忍遽杀;乃下之狱。

二十九日,振飞大享士于淮安府学中,叙向来有功文武官八十余员;振飞与燮亲安席,观者鼓舞。

已而振飞为马士英所论,得旨提合,阁城不平;寻以士民公疏,得免。旋丁艰去。王燮又为御史陈丹衷荐,升巡抚山东;士民夺气。刘泽清遂营窟于淮城中,田仰与之猫鼠;山东又不可往,王燮逡巡于河上而已。田仰,士英之私人;五月十七日,起抚淮扬,以阮大铖力荐堪任节钺也。

史可法奏``淮人忠义疏'':`闯贼自入关以后,声势逼人,假借安民,煽动海内;伪官一到,争思奉迎。甚至督抚手握兵权,不能碎一伪牌、斩一伪使;人心之坏,至此极矣。惟有淮安官民固守,伪牌到则碎之、伪使到则斩之;贼骑逼河上,则邀击败退之。贼将如董学礼、白邦政等,皆踯躅而不敢前。民间义兵集至一、二十万,声势之壮,犹若长城。顷又报恢复宿迁,伪官遁走,维持疆事,江南乃安;其有功于国家甚大。然淮人之敢于此者,实地方官鼓舞之力也。抚按诸臣亲在河干,与民共守;碎牌斩使,断而行之。密遣各兵多所斩获,故能振将卒同仇之忾、坚民间死守之心。东南奠安,实赖此举。伏乞敕下该部院,将按臣王燮优擢示劝;抚臣路振飞已经解任,另候优议。其余地方官、乡绅、士民及行间有功将士,并行按臣察确具题,特为旌叙:庶忠义之士感奋,而他处投贼、避贼偷生苟免者皆知所愧耻矣'。

伪淮扬防御使武愫至宿迁,伪将董学礼、伪漕储方允昌、伪督饷白邦政等俱置酒晏之,遂留连数日;借董兵千人,所过地方骚扰不堪。伪示传至徐州,举人阎尔梅大骂,碎其牒;武愫拘之,下狱。尔梅赋诗曰:`死国非轻死逆轻,鸿毛敢与泰山争!楚衰未必无三户,夏复由来起一成。日月有时经晦蚀,乾坤何旦不皇明!宠新岂是承恩者,空自将身买贼名'!遣人驰示武愫,愫大怒,密欲令人斩之。

``路文贞公传'':公讳振飞,字见白,号皓月;广平曲周人。天启乙丑进士,授泾阳知县。不建逆奄祠,多惠政;县人皆绘图祀之。崇祯辛未,召入为四川道御史。疏劾宜兴、乌程、巴县三相国、湖州冢宰及山东二抚臣,举朝惮之。

\hypertarget{header-n32}{%
\subsection{卷四·南都甲乙纪(续)}\label{header-n32}}

马士英特举阮大铖

阮大铖,字集之,号圆海;怀宁人。天启时,为太常少卿。以魏党,思庙钦定``逆案''禁锢。大铖本士英之房师;既被废,寄居金陵,与孔昭、士英及太监李承芳交密。士英抚宣大,以总监王坤论罪。及周延儒再相,大铖、士英同馈万金求复官,夺于物议,仅起士英兵部左侍郎,提督凤阳:此崇祯壬午四月也。至是,士英思所以酬之;孔昭殿争,因士英而发也。

六月初六日(壬戌),士英奏:`冒罪特举知兵之阮大铖,当赦其往罪,即补臣部右侍郎';许之。时士英乘高宏图督漕未入,即自拟旨:`赐冠带陛见'。举朝大骇。

初八日(甲子),高宏图曰:`大铖可用,必须九卿会议'。士英曰:`会议,则大铖必得用'。宏图曰:`臣非阻大铖;旧制:京堂必会议,乃于大铖更光明'。士英曰:`臣非受其贿,何所不光明'?宏图曰:`何必受贿;一付廷议,国人皆曰贤,然后用之耳'。宏图出,即乞休。姜曰广``辞归疏''云:`臣前见文武纷竞,既惭无术调和;近见``逆案''掀翻,又愧无能豫寝。遂使先帝十七年之定力,顿付逝波;陛下数日前之明诏,竟同覆雨。梓宫未冷,增龙驭之凄凉;制墨未干,骇四方之听闻。惜哉维新!遂有此举。臣所惜者,朝廷之典礼;所畏者,千秋之清议而已'!

初九日(乙丑),士英为大铖奏辩;言`魏忠贤之逆之非闯贼可比'。且力攻宏图、曰广、吕大器诸人护持局面;谓`于所爱而登之天者,即曰先皇帝原无成心也;于所忌而锢之渊者,即曰先皇帝定案不可翻也。其妄莫甚'!

十一日(丁卯),给事中罗万象奏曰:`辅臣荐用大铖,或以愧世无知兵者;然而大铖实未知兵,恐``燕子笺''、``春灯谜''即枕上之阴符而袖中之黄石也。伏望许其陛见,以成辅臣吐握之意;禁其复用,以杜邪人觊觎之端'。御史詹兆恒奏曰:`钦案诸人久图翻局,幸先帝神明内断,确不可移;陛下跸驻龙江,痛心先帝异变,与诸臣抱头痛哭!百姓遂莫不洒血捶胸,愿思一报。近闻燕、齐之间,士绅皆白衣冠,吁先帝而呼天;驱杀伪守,各守关隘。此诚先帝德泽在人、国愤非常有以激发其忠义耳。今梓宫夜雨,一坏未干;太子、诸王,六尺安在?国仇未复,而忽召见大铖,还以冠带;岂不上伤在天之灵、下短忠臣之气'?

十三日(己已),吕大器奏曰:`先帝血肉未寒,爰书凛若日星;而士英悍然不顾,请用大铖!不惟视吏部如刍狗,抑且视陛下为弁髦'。又言:`近年温、周擅权,老成凋谢;一时庸奸偾事,中原陆沈。皇上中兴,一时云蒸霞起;乃不意马士英浊乱朝政!夫士英,非以贿败问遣、借途知兵而为凤督哉?乃挟重兵入朝,腼颜政地。南国从来蔼蔼,一唆拨而殿陛喑哑叱咤,藐主尊为赘旒矣。``逆案''一书,先帝定为乱贼大防。而士英拉大铖于尊前,径授司马,布列私人;越其杰、杨文骢等有何劳绩,倏而尚书、宫保、内阁?倏而金吾世荫也'?郭维经奏:```逆案''成先帝之手,今``实录''将修,若将此案抹杀不书,则赫赫英灵,恐有余恫,非陛下所以待先帝;若书之,而与起用大铖对照,则显显今古未免少愆,并非辅臣所以爱陛下也!惟愿陛下爱祖宗之法,因爱先帝,并爱先帝丝纶'!

十四日(庚午),兵部郎中尹民兴言:`熹庙时崔、魏煽逆,士大夫丧耻忘君,几成苞孽之固;遂至先帝末载天子下席,诸臣或匍伏而拜爵、或献策以梯荣,皆忠孝不明之流祸也。申罪讨逆,司马职也。今抗颜堂上者,一``逆案''之阮大铖;即行檄四方,何以消跋扈将军之氛?古者破格求才,惟曰使贪、使诈,不曰使逆。``逆案''可翻,崔、魏亦可恤,周、钟诸逆皆可使才宥过矣'!

十七日(癸酉),御史左光先言:`阮大铖线索逆党野子傅应星,杀臣兄光斗及魏大中、杨琏;士英冒罪特举,明知无复有罪之者。皇上不改先帝之政,臣忍忘不共之仇耶'?

十八日(甲戌),詹兆恒进魏党钦案原本,御史陈良弼谏阻勿翻``逆案''。时怀远侯常廷龄、太仆少卿万元吉、御史王孙蕃等各言``逆案''不可翻、大铖不可用;俱不听。马士英自辩在兵言兵;上慰士英,切责科道。

``甲乙史''云:`阮大铖于六月初八日入见,备陈见枉之由'。``编年''云:`大铖召对,具联络、控扼、进取、接应四策,又陈``长江两合、三要、十四隙'',俱称旨;竟用为江防兵部尚书'。

九月初一日,柳祚昌催补阮大铖官,即命添注兵部右侍郎;仍禁朝臣不得把持阻谏。刘宗周云云,上切责之。

黄澍笏击马士英背

黄澍,字仲霖;徽州人。丙子举浙闱,丁丑登进士;授河南开封推官。以固守功,擢御史,巡按湖广;监左良玉军。

甲申宏光立,六月二十日(丙子),澍同承天守备太监何志孔入朝,求召对。既入见,澍面纠马士英权奸误国,泪随语下。上大感动,顾高宏图曰:`黄澍言言有理,卿识之'!召入御座前,澍益数其罪;士英不能辩一语。志孔复前佐澍,言士英无上诸事。秉笔太监韩赞周叱志孔退,曰:`御史言事是其职;内臣操议,殊伤国体'。士英亦跪求处分;适跪澍前,澍以笏击其背曰:`愿与奸臣同死'!士英号呼曰:`陛下视之'!上摇首不言;良久,谓澍曰:`卿且出'!赞周命执志孔;上私谕赞周云:`马阁老宜自退避'!士英遂称疾,尽移直庐器具以出。以金币分馈福邸旧阉田成、张执中两人,向上泣曰:`皇上非马公不得立;若逐马公,天下皆议皇上背恩矣。且马公在阁,诸事不烦皇上;可以优闲自在。马公一去,谁复有念皇上者'?上默然;田成即谕士英疾趋入直。随有旨:`何志孔本当重处,首辅亟为求宽,具见雅度;姑饶他为民'。谣曰:`要纵奸,须种田;欲装哑,莫问马'!

黄澍论马士英十大罪

奸督有十可斩之罪,谨详列以求圣断、以质公论事。

痛自乱贼猖狂,宗社失守。幸皇上应运中兴,大张挞伐。臣小臣也,缄口苟容,岂不自保禄位!顾臣受国厚恩,禀性刚烈,不顾利害,致捋虎须。臣今日言亦死,不言亦死;言则马士英必杀臣,不言而苟且偷生,臣不死于贼、必死于兵。均之死也,臣敢冒死言之。奸督自任数年以来,有功无罪,臣谓可斩之罪有十焉。

凤陵一坏土,国家发祥之地;士英受知先帝,自宜生死以之。巧卸重担,居然本兵;万世而下,贻皇上以轻弃祖宗之名:是谓不忠。不忠者,可斩也。

国难初定,人办必死之志,为先帝复仇。士英总督两年,居肥拥厚,有何劳苦?明圣之前,动云劳苦多年:是谓骄蹇。骄蹇者,可斩也。

奉命讨献,而足未尝跨出蕲、黄一步;奉命讨闯,而足未尝跨出寿春一步。耽延岁月,以致贼势猖狂,不可收拾:是谓误封疆。误封疆者,可斩也。
献贼兵部尚书周文江引贼破楚省,教献下江南;及左镇恢复蕲、黄之后,周文江□金朝以入而参将之荐夕以上:朦胧先帝,□祸地方:是谓通贼。通贼者,可斩也。

市棍黄鼎无以报德,用其参谋冯应庚私铸闯贼银印一颗,上篆``果毅将军印'',托言夺自贼手,飞报先帝;士英蒙厚赏,黄鼎等俱加副将。今麻城士民有``假印不去、真官不来''之谣:是谓欺君。欺君者,可斩也。

皇上中兴,人归天与。士英□施然以为``非我莫能为'';始而居功,后必蔑上:其目中无朝廷久矣。金陵之人有``若要天下平,除非杀了马士英''之谣:是谓失众亡等。失众亡等者,可斩也。

生平至污、至贪,清议不齿。幸以手足圆滑,偶脱名于``逆案''。其精神满腹,无日忘之;一朝得志,遂特荐同心逆党阮大铖。大铖居朝为逆贼、居家为匪类,三尺之童见其过市,必唾骂之。士英首登启事,对人云:`我要操朝权,必先自用大铖始'。魏党贻祸,至今为烈;敢于蔑侮前朝,矫诬先帝。迹其所为,恨不起逆党于地下而与之同谋:是谓造叛。造叛者,可斩也。

减克兵粮,家肥兵瘦;平素不能行恩,临事岂能用武!一旦有急,挟君父而要之;借皇上之名,为请罪之夤缘。在各镇忠义自奋,人人愿报明主;皇上念民间劳苦,破格殊恩。士英动云:`都是我在皇上面前奏的'。``善则归君'',其义谓何?是谓招摇骗诈。招摇骗诈者,可斩也。

宸居寥落,长江浩浩;士英不闻严御警跸,紧防江流。而马疋、兵械札营私居,以防不测;以保金帛何其智?以守陵园何其怯?以壮甲第何其横?是谓不道。不道者,可斩也。

上得罪于二祖、列宗,下得罪于兆民百姓;举国欲杀,犬彘弃余。以奸邪济跋扈之私,以要君为卖国之渐:十可斩也。

士英有此十大罪,皇上即念其新功、待以不死,当削去职衔,责之速赴原任,广联声援;庶可以慰祖宗在天之灵、谢亿兆人之口。而奸狡日深、巧言狂逞,此岂一日可容于尧、舜之世哉!伏乞大奋乾纲,下臣言于五府、六部、九卿、科道从公参议。如臣一言涉欺,皇上即诛臣以为嫉功害能、衊诬大臣之戒;如臣言不谬,亦乞立诛士英以为奸邪误国、大逆不忠者之戒。抑臣更有说焉:臣昨赴都,见吏部侍郎吕大器曾疏参士英,臣尚未见全抄;要之,大器亦非无罪人也。悻戾自用,反覆阴阳。臣曩在都门,与台臣王燮曾交章参之;臣到九江,甚鄙其为人。昨士英指臣有党,今必以臣党大器为题;故为明白拈破。臣言官也,明知害之所在,与死为邻;职掌所关,不敢不争。士英即旦夕杀臣,臣甘之如饴矣。因补疏直陈颠末,字稍逾格;惟皇上干断施行!

七月初二日(丁亥),着黄澍星回地方料理恢复承、襄。时澍连上十疏,内多纠士英者。宏光不得已,屡谕其赴楚;乃去。总览前后诸疏,逼真古名臣奏议,有胆、有识,落笔妙天下者也。然其侃侃而谈,无少顾忌者,挟良玉以为重也;而士英之不敢遽斥澍者,亦畏良玉耳。不然,吕大器一参士英,即有旨`予告去',或刑部逮问矣;亦何爱乎澍、何惮乎澍而纵之之楚耶?

黄澍辩疏

七月二十二日(丁未),黄澍辩马士英见诬疏云:`麻城劣生周文江为献贼兵部尚书,引献贼破省。有锦衣遣戍刘侨托文江进美妾、玉杯、古玩数万金于献,即用侨为锦衣大堂。比左良玉恢复蕲、黄,侨削发、私遁;寻送赤金三千两、女乐十二人于士英。今年四月,士英委黄鼎署印麻城,麻城汹汹几乱。乡绅请臣弹压,侨献银三千两助军;臣批云:``正苦无粮,真可愧挟资以媚贼者。仰即收贮''!臣言隐而讽矣。既还武昌,黄鼎代为解银一千两、玉带二围、殊冠一顶;臣又批云:``军中无妇人,何用珠冠?大功未成,不须玉带。仰即变价济饷'。臣巡方衙门收支,皆有司存。士英以侨私书为言,试命将臣原书呈览,则清浊立见矣'。

九月二十六日(辛亥),楚宗朱盛浓疏诬黄澍毁制、辱宗、贪贿、激变;士英喜,擢盛浓池州府推官。内批:`逮澍刑部提问'。澍不至。

十月初八日(壬戌),黄澍奏辩;内皆:`朱盛浓害非剥肤,何至千里叩阍'?

逮澍而澍不至,士英之权势不能行于南楚之臣矣。次年良玉举兵之事,已兆于此。

附记:乙酉大兵下徽州,闽相黄道周拒于徽州之高堰桥;自晨至暮,斩获颇多。澍以本部邑人习知桥下水深浅不齐,密引大清骑三十由浅渚渡,突去闽兵后;骤见骇甚,遂溃。徽人无不唾骂澍者。后官于闽,谋捣郑成功家属以致边患,遂罢。

朱统\{金类\}诬诋姜曰广

七月二十六日(辛亥),南昌建安王府镇国中尉吏部候考朱统\{金类\}上书,诬诋大学士姜曰广秽迹,定策时显有异志;词连史可法、张慎言、吕大器等。盖马士英欲挤可法以独居定策之功、刘孔昭欲去可法以专任田仰,为一网打尽之计;阮大铖属草,授统\{金类\}上之。疏入,高宏图票拟`究治'。上坐内殿,召辅臣入;上厉声曰:`统\{金类\}吾一家,何重拟也'!且责宏图疏召可法还朝为非是。宏图抗辩,士英独默。上每语必左顾田成,明有指授者。

二十九日,朱统\{金类\}参姜曰广谋逆;高宏图、姜曰广皆引疾杜门。礼科给事袁彭年驳奏曰:`祖制:中尉奏请,必先具启亲王参详可否,然后给批齎奏。若候吏部,则与外吏等;应从通政司封进。今何径、何窦,直达御前?微刺显攻,捕风捉影;陛下宜加禁戢。臣,礼垣也,事涉宗藩,皆得执奏'。不问。通政司刘士祯言:`曰广劲骨戆性,守正不阿;居乡立朝,皆有公论。统\{金类\}何人?扬波喷血、掩耳盗铃,飞章越奏,不由职司;此真奸险之尤者,岂可容于圣世'!皆不听。

刘泽清捏四镇公疏纠姜曰广、刘宗周谋危社稷;朱统\{金类\}复讦奏姜曰广、雷演祚、周镳,其疏仍出阮大铖草。马士英拟旨:`逮演祚、镳等'。时演祚居忧,侨金陵;镳为大铖最恨人。有自比于孔昭者,显示辣手于同邑大僚,一时阴挤;而士英借是以迫宏图、曰广之去耳。

陆朗、黄耳鼎疏攻姜曰广、徐石麒、刘宗周结党欺君、把持朝政,无人臣礼;曰广、石麒、宗周寻各予告而去。户科吴适疏言:`曰广、宗周历事五朝,贞心亮节,久而弥劲;应亟赐留'。不听。

熊汝霖论异同恩怨

吏科熊汝霖言:`臣观目前大势,即偏安亦未可稳。``兵饷战守''四字,改为``异同恩怨''四字;朝端之上,元黄交战。即一、二人之用舍,而始以勋臣,继以方镇。固圉恢境之术,全然不讲;惟舌锋笔锷之是务,真可笑也。且以匿帖而逐旧臣矣,俄又以疏藩而参宰辅矣,继又喧传复厂卫人心皇皇矣。辅臣曰广忠诚正直,海内共钦;乃么么小臣,为谁驱除?听谁主使?且闻上章不由通政,结纳当在何途?内外交通,飞章告密;墨敕斜封,端自此始。事不严行诘究,用杜将来,必至厂卫之害:横者借以树威、黠者用以牟利,人人可为叛逆、事事可作营求。缙绅惨祸,所不必言;小民鸡犬,亦无宁日:此尚可为国乎?先帝十七年忧勤,曾无失德;而一旦受此奇惨,止有厂卫一节,未免府怨臣民。今日缔造之初,如育婴孩,调护为难;岂可便行摧折?陛下试思先朝之何以失,即知今日之何以得?始先帝笃念宗藩,而闻寇先逃,谁死社稷?保举换授,尽是殃民;则今何以使跃冶不萌而维城有赖?先帝隆重武臣,而死绥敌忾十无一二,叛降跋扈肩背相踵;则今何以使赏罚必当而惠威易行?先帝委任勋臣,而官舍选练一任饱飏,京营锐卒徒为寇籍;则今何以使父书有用,客氛是屏?先帝简任内臣,而小忠小信原无足用,开门延敌且噪传门;则今何以使柄无旁操而恩有余地?先帝擢用文臣,而边才督抚谁为捍御,超迁宰执罗拜贼廷;则今何以使用者必贤而贤者必用'?疏入,士英票旨云:`这厮指朕为何如主?重处;姑罚俸三月'。

九月初九日,姜曰广致仕回籍。十月二十日,予统\{金类\}京官,寻补行人;以疏逐曰广也。统\{金类\}曰:`须还我总宪'!

吴适陈维新五事

吴适上言维新五事:`一曰信诏旨:朝廷之有丝纶,所以彰示臣民,俾知遵守。迩因事变错出,前后悬殊。用人之途,始慎而继以杂;诛逆之典,初严而终以宽。禁陈乞矣,而矜功、诵冤者章日上;重爵赏矣,而请荫、乞封者望日奢。镇帅屡责进取,而逡巡不前;军需频督转输,而庚癸如故。欲斯画一,宜重王言。今后凡奉明旨,务俾上作而臣下尽遵,毋致游移。一曰核人才:人才为治道所从出,将为其终、先谨其始。顷者,典籍无稽,钱神有径,人思跃冶。初仕辄冀清华,官多借题行间;每增监纪,膻逐之谋愈切、卸担之术偏工。起废而薰莸并进,悬缺则暮夜是求;以致荐牍日广、启事日勤。今后求才务宽\\
、用人务覈,宁重严于姑进,毋进恨于偾辕。一曰储边才:将帅之略,岂必尽出武途;如唐之节度,文武兼用而内外互迁,盖储之者素耳。请饬中外蓬荜之彦,非韬钤之略勿讲;辟举之选,非军旅之才勿登。技勇骑射日日请求,共激同仇,以振积懦。一曰伸国法:陷北诸臣已有定案,但恐此辈辇金求翻。既以宽其不死者,昭皇上之浩荡;尤当以继其觊用者,明臣子之大防。一曰明言责:祖宗设立六垣,与六部相表里;是故纠弹之外,复有抄参。补阙拾遗,务期殚虑。倘掖垣仅取充位,则白简止贵空悬;则抄发本章,一胥吏事:岂先王设官意哉!望陛下亟进谠言,见诸施行;毋致批答徒勤,而实效罔着,所裨非浅'。疏入,不省。

马嘉植陈立国本

吏科马嘉植陈立国本事:`一、改葬梓宫;一、迎养国母;一、访求东宫、二王;一、祭告燕山陵寝。在君父力自贬损,尊养原非乐受;在臣子痛加悔艾,富贵岂所相期!茅茨虽陋,可勿翦也;有以劳人费财导者,勿听。经武以外,亦可概节也;有以处优晏衍进者,勿听'。

贺世奇言慎行赏,刑部侍郎贺世奇上言刑赏宜慎:`如吴三桂奋勇血战,李、郭同功,拜爵方无愧色。若夫口头报国,岂其遂是干城;河上拥兵,曷不以之敌忾?恩数已盈,勋名不立,冒滥莫甚'!疏上,俱报闻而已(``遗闻''载贺世寿)。

李谟奏明臣谊

国子监典籍李谟奏曰:`今日诸臣能刻刻认先帝之罪臣,方能纪常勒卣,蔚为陛下之功臣。日者庙廷之争,几成哄市;恐传闻遐迩,不免开轻视朝廷之意。原拥立之事,皇上不以得位为利,诸臣何敢以定策为名;而甚至轻加镇将,于义未安。镇将事先帝,未闻收桑榆之效;事陛下,未闻彰汗马之绩。按其实,亦在戴罪之科;而予之定策,其何以安?倘谓劝进有章,足当夹辅;抑以勖勉敌忾,无嫌溢称。然而名实之辨,何容轻假!夫建武之邓禹,犹悬受任无功;唐肃宗之郭子仪,尚自诣阙请贬。愿陛下敕谕诸大臣:立志以倡率中外,力图赎罪;必大慰先帝殉国之灵,庶堪膺陛下延世之赏。至一概勋爵,俱应辞免,以明臣谊。至于丝纶有体,勿因大僚而过繁;拜下宜严,勿因泰交而稍越;繁缨可惜,勿因近侍而稍宽:然后纲维不堕而威福日隆也'。

陈子龙请广忠益、慎名器、用贤勿二

兵科给事中陈子龙疏请``广忠益''谓:`当黄道周触忌权佞,构陷至深;先帝震怒,祸将不测。群工百官相戒结舌,独涂仲吉以孤童担囊走万里外,上书北阙;予杖下狱。狱吏希迎,拷掠荼酷;至死不屈,以明道周之冤。此虽王调贯械以讼李固、杜仲杀身以救李云,亦不过是。幸先帝圣明,得以俱免。宪臣刘宗周昔以廷诤去国,孝廉祝渊毅然请留;先帝已轻议罚。迨后奸臣挑激,复征槛军;虽与仲吉得祸轻重有殊,然为国惜贤,舍生取义,其揆一也。当仲吉赴戍之时、祝渊征逮之日,臣皆得与接对。仲吉凝静深远,绝不以立名自喜;祝渊谦抑温恭,惟出位引咎。间有投赠,锱铢不纳。若置之台谏之班,必有以上补衮职、下剔奸邪'。

``遗闻''云:`以兵部侍郎解学龙疏荐,内批:``升原任户部主事叶廷秀为都察院堂上官,监生涂仲吉、生员诸永明为翰林院待诏''。盖廷秀、仲吉、永明者皆侠节士,先帝时申救道周下狱杖戍者也'。``甲乙史''云:`七月二十六日(辛亥),仲吉、永明并授待诏'。

子龙又疏请``慎名器''谓:`陛下间关南返,从官几何?卫士奄尹寥寥无几。今大位既登,来者何众!不遏其流,何所底止。必将人夸翼赞之功,家切从龙之念;伤体害政,非国之福。夫劝功诱善,惟在爵赏;一为轻滥,后将无极。丰沛故人、文墨小吏,自昔为嫌;朱紫盈门、貂蟒满座,尤乖国典。立政之始,惟愿陛下慎持之!嗣后果系服劳有功,但当赏之金帛、不应授以爵位,以贻``曹风''``不称''之讥、犯``大易''``负乘''之戒'!

又疏请``用贤勿二、爵人宜公'':`一在宪臣之宜召也:宪臣老成清直,海内尽知。今入国门,寄居萧寺,不得一望天颜。在陛下以方谕大臣和衷共济,恐宪臣戆直,奏对之际,复生异同。然臣以陛下``疑畏君子''之机,从此而生,恐君子有携手同归之志;黄道周之流,皆踯躅而不前矣。陛下谁与共济天下哉?一为计臣之特用也:计臣清端敏练,百僚所服。但古制爵人于朝,与众共之。墨敕斜封,覆辙可鉴!万一异日有奸邪乘间、左右先容,铨司不及议、宰辅不及知,而竟以内降出之。臣等不争,则幸门日开;臣等争之,则已有前例。立国之始,臣愿陛下慎持之也'。

疏入,俱不听。

疏内``宪臣''疑指刘宗周,而``计臣''则指江阴张有誉也。``甲乙史''云:`七月二十五日(庚戌),户部尚书周堪赓久不到仕,中旨传升吏部侍郎张有誉为户部尚书大学士。高宏图以不经会推缴命;得旨:``特用出自朕裁''。盖有誉清慎,为人所称;马士英借以开传升之幸门,为阮大铖地也。吏部给事中章正宸封还中旨,力争;不听。故姜曰广、陈子龙诸君子俱极论之学。

姜曰广论中旨

祖宗会推之典,立法万世无弊;斜封墨敕,覆辙俱在。臣观先帝之善政虽多,而以坚持``逆案''为盛美;先帝之害政亦间出,而以频出中旨为乱阶。用阁臣内传矣,用部臣、勋臣内传矣,选大将、言官亦内传矣。他无足数,论其尤者:其所得阁臣,则逢君殃民、奸俭刻毒之周延儒、温体仁、杨嗣昌、偷生从贼之魏藻德等也;其所得部臣,则阴邪贪猾之陈新甲等也;其所得勋臣,则力阻南迁、尽撤守御之李国桢也;其所得大将,则纨绔支离之王朴、倪宠辈也;其所得言官,则贪婪无赖之史\{范土\}、陈启新也。凡此,皆力排众议,简自中旨者也;乃其后效亦可睹矣。且陛下亦知内传之故乎?总由鄙夫热心仕进,一见摈于公论,遂乞哀于内廷。线索关通,中自有窍;门户摧折,巧为之词。内廷但见其可怜之状、听其一面之词,遂不能无耸动;间以其事密闻于上,又得上之意旨转而授之。于是平台召对,片语投机,立谈取官,有若登场之戏。臣昔痛心此弊,亦于讲艺敷陈;但以未及畅语,至今隐恨。先帝既误,陛下岂堪再误哉!

天威在上,密勿深严;臣安得事事而争之。但愿陛下深宫有暇,温习经书,间取``大学衍义''、``资治通鉴''视之;如周宣、汉光之何以竟恢远烈?晋元、宋高之何以终狃偏安?武侯之出师南蛮,何惓惓以``亲君子必远小人''为说?李纲之受命御敌,亦何以切切``信君子勿问小人''为言?苟能思维,必能发明圣性。陛下与其用臣之身,不若行臣之言;不行其言而但用其身,是犹兽畜之以供人力俎也!

吴适请忧勤节爱

户科吴适疏请``忧勤节爱''。言:`国耻未雪,陵寝成墟;豫东之收复无期,楚、蜀之摧残弥甚。旧部草创,一事未举;万孔千疮,忧危丛集。又况畿南各省,到处旱灾;兼之臣邻消长多虞、将帅元黄构衅。伏惟陛下始终竞惕,兼仿祖制:早、午、晚三朝勤御经筵而谘时政;亲近儒臣,朝期无更传免。而又躬崇俭约,尚茅茨而省工作、严爵赏而重名器;锱铢必恤,俾佐军兴。诸凡无艺之征,一概报罢;被灾之地,确覈酌缓。墨吏必惩,蠹胥必殛。根本之计,孰大于此'!

沈胤培请立中宫、举经筵、定朝仪

礼科沈胤培疏请``立中宫、举经筵、定朝仪''谓:`今永巷无脱簪之儆,崆峒鲜问道之谟。濒笑或假借于从龙,而帘远堂高之义不着;是非或混淆于市虎,而阴阳消长之关可虞!陛下诚思此身为祖宗付托之身,先帝之大仇一日未复,即九庙之神灵一日怨恫。而正朝廷以正百官、正万民,先自宫闱始;则选立中宫为第一义。经筵业奉明旨,尤祈汲汲举行。或召词臣询经史、或召部臣考政治,而时令台谏之臣陈得失。宫中万几之暇,披览``资治通鉴''及本朝``宝训''等书,以知前代兴亡之迹、祖宗致治之由。至于朝仪多阙,大典未光:如朝门不应奏乐而奏乐,各衙门应奏事而不奏事。凡若此类,并宜申饬'。

章正宸论铨政

吏科章正宸指陈铨政:`一、名器宜慎:定策者既懋厥赏,其余人自请叙,则十倍增官。辇金不供刻印,宁免瓜果之诮!一、职掌宜专:用人独归吏部;今有咨送者、有荐拔者、有径自奏讨者,冢臣所职几何?一、封疆宜肃:文武共寄封疆,不斩误国之臣,不激报国之气。一、废臣宜饬:爵重则人乃劝,法守则士知恩。累累起废,不自静听;岂不闻律有``罢吏不入国门''乎'(``甲乙史'')?

宋劼疏略

监军佥事宋劼上言:`臣民苟安江界,恐非所以保江界;诸臣苟存富贵,恐非所以保富贵也'。又言:`人生止有此时日、人身止有此精神。古贤惜分阴,运甓舞鸡,皆劳筋骨于有用'。\\
祁彪佳请革三弊政

御史祁彪佳疏论时政谓:`洪武初,官民有犯,或收系锦衣卫。高皇帝因有非法凌虐,二十年遂焚其刑具,移送刑部审理:是祖制原无诏狱也。后乃以锻炼为功、以罗织为事,虽曰朝廷之爪牙,实为权奸之鹰狗。口词从迫勒而来,罪案听指挥而定;即举朝尽知其枉,而法司谁雪其宽!酷惨等于来、周,平反从无徐、杜:此诏狱之弊也。洪武十五年,改銮仪司为锦衣卫,专掌直驾、侍卫等事;未尝有缉事也。迨后东厂设立,始有告密之端:用银而打事件、得贿而鬻刑章;飞诬多及善良,赤棍立成巨万。招承皆出于吊拷,怨愤充塞于京畿。欲绝苞苴,而苞苴托之愈盛;欲究奸宄,而奸宄未能稍清:此缉事之弊也。若夫刑不加大夫,原祖宗忠厚立国之本;乃夫逆瑾用事,始有去衣受杖者。刑章不归司政,扑责多及直臣;本无可杀之罪,乃致必杀之刑。况乎朝廷徒受拒谏之名,天下反归忠义之誉。盖当血溅玉阶、肉飞金陛,班行削色,气短神摇;即恤录随颁,已魂惊骨削矣。是岂明盛之休风,大失君臣之分谊:此廷杖之弊也。伏乞陛下严行禁革'!

袁彭年请革厂卫

八月初七日,礼科袁彭年疏言:`高皇帝时,不闻有厂。相传文皇帝十八年始立东厂,命内官主之;此不见正史。惟大学士万安行之,亦不闻特以缉事着。嗣后一盛于成化。然西厂汪直、逾年辄罢;东厂尚铭,有罪辄斥:当时不得称纯治矣。再盛于正德。邱聚、谷大用等相继用事,皆倚逆瑾煽虐;酿十六年之祸,天下骚然。一盛于天启。逆魏之祸,几危社稷:近事之明鉴也。自此而外,列圣无闻。夫即厂卫之兴废,而世运之治乱因之。顷先帝朝亦尝任厂卫访缉矣,乃当世决无不营而得之官,中外亦有不胫而走之贿。故逃网之方,即从密网之地而布;作奸之事,又资发奸之人以行。始犹帕仪交际,为人情所有之常;后乃赃贿万千,成极重莫返之势。岂非以奥援之途愈秘而专,传送之关愈曲而费乎?究竟刁风所煽,官长不能行法于胥吏、徒隶可以迫胁其尊上,不可不革'。疏入,上责其狂悖沽名;降三级,外调浙江按察司照磨。
陈子龙疏略

十八日,兵科陈子龙言:`中兴之主,莫不身先士卒,故能光复旧物。陛下入国门再旬矣,人情泄沓,无异升平之时;清歌漏舟之中、痛饮焚屋之下:臣诚不知所终矣!其始皆起于姑息一、二武臣,以至凡百政令,皆因循遵养;臣甚为之寒心也'!

史可法请行征辟

史可法请行征辟之法,以通铨政之穷;疏曰:`国家设四藩于江北,非为江左偏安计也。将欲立定根基,养成气力;北则为恢复神京之计、西则为澄清关陕之图,一举而遂归全盛耳。圣明在上,忠义在人;君父之仇耻特深,海宇之群心竞奋。在师武臣,无不以灭贼复仇为念者。乘时大举,扫荡可期。特所虑者,兵戈扰攘之中,不复有百姓耳;无百姓,何利于有疆土。故此时择吏不缓于择将,而救乱莫先于救民。所谓得一贤守,如得胜兵万人;得一贤令,如得胜兵三千人:正今日之谓也。然而今日之守令难言;虽以前北都未陷,求牧方殷,非不有破格之升除,何曾收得人之实效。地有难易、缺有炎冷,无所不用其营避。而兵荒破残之区,卒举而授之庸人,此岂白面书生所能胜任?目今人才告乏、资格为拘,东南缺员正自不少,安能复填西北之缺,使无致叹于晨星;则铨选法穷,不得不改为征辟。往时保举多系慕膻,故捷足蝇营,真才裹足。今西北则危地也,危则人人思避;而真从君父起念,誓图除凶雪耻垂功名于千载,乃始投袂而相从、请缨而奋起。臣以为宜仿保举之法,通行省直抚按、司道及在京九卿、科道官,果有才胆过人、堪极危乱者,不拘资格,各举一人,起送到京;资以路费,赴臣军前效用,酌补守令缺员。二年考满,平升善地;三年考选,优擢京曹。有靖乱恢疆、功能殊异者,立以节钺京堂,用示酬劝。如各官避嫌不举,即听该科指参,重行罚治。若有坏才思逞,赴臣军前者,验其真才,一体录用。再如江北、山东、河南一带,有能保护一方、为民推服者,即系桑梓之邦,亦可权宜径用。总求天恩破格,假臣便宜;决不敢滥用匪人,自误进取。闻逆贼所至,常带多人,得一州,即设一州官;得一县,即设一县官。小人不识顺逆,为所用者恒多。况际国祚重新,贼寇垂尽;则必有桓桓德心之士,辐辏而翼中兴。臣拭目望之矣'!

千古良法,所虑奉行非人,杂之以私,旋举而旋废耳。

李清奏国用不支

工科李清言:`天下秦、晋属贼,燕、代属清,兖、豫已成瓯脱,闽、广解京无几;徽、宁力殚于安、芜二抚,常、镇用竭于京口二镇。养兵上供者,仅苏、松、江、浙。且昔以天下供天下,不足;今以一隅供天下,有余乎?营建、仪器事事增出,其何支也'!

张捷论民心国运

十月十五日,张捷言:`先帝末造,民心、兵心、士子之心、将吏之心,无一不坏。要皆在廷诸臣之先坏,而种种因之。重贿所归,使人不知有法纪。以科场为垄断,以文字为纠连。举贪官污吏之所渔猎、豪绅悍士之所诳逼、愤帅骄兵之所淫掠,聚毒于民。民心既去,国运随之;而惨祸及于先帝矣'!

按捷疏甚得当日情景;而立朝后,惟阿党是徇,毒更甚焉。古人所以致慨于目睫也!

吏科奏计典

二十六日(庚辰),吏部张某奏:`近时位署无常、挨举叠进,辇金觅穴,营求不止。如往岁之计典可翻,明岁之计可以不设矣'。

吴适陈日讲、午朝二事

``补遗''云:十月朔,户科吴适疏陈``昭事之实'':`一曰日讲宜行:请敕定期,俾博闻有道之臣,朝夕左右稽询经史,虚衷延纳;更取``祖训''、``大诰''诸书,时时省览,以为蓍鉴。一曰午朝宜举:俾阁部大臣以及台垣散秩,咸得躬膺清问:即于披对之余,采疾苦以疏民隐、覈功罪以劝疆臣、明是非以黜邪佞'。疏入,不省。

游有伦奏国事淆乱

十一月初二日(丙戌),御史游有伦奏:`今日国事淆乱,不知礼义廉耻为何物。明知君子进退不苟,故以含沙之口,激之速去;甚至有常人所不忍道者,渎于君父之前。其视皇上何如主乎!台省中微有纠劾,则指为比党;相戒结舌,真所谓``前有谗而不见、后有贼而不知''也'。

是时黄耳鼎、陆朗、朱统\{金类\}疏攻姜曰广、徐石麒、刘宗周等,各予告去。故有伦奏此,可谓抗疏矣;不知句尤骂得马奸一班小人好。

钱增请浚刘家河

户科钱增疏请``备水利''言:`苏、松、常、镇、杭、嘉、湖七郡之水,以太湖为腹,以大海为尾闾,以三江入海为血脉。盖自吴淞淹塞、东江微细,独存娄江一派。而娄江之委七十里曰刘家河,乃娄江入海之道;东南诸水全恃此以归墟,不至横溢泛滥者,则带水灵长之利也。元时,刘河最深,运艘、市舶走集于此。近口涨沙淤塞,于是东流之水,逆而向西;涓滴不入,灌溉无资。兼之岁岁旱魃,平畴龟坼,人牛立槁;虽复桔槔如林,何从乞灵海若?然此就旱暵言耳。万一大浸嵇天,七郡洪流倾河倒峡,震泽不能受;散漫横溃,势必以七郡之田庐为壑,而城郭人民益不可问。东南数百万财赋尽委逝波,其如国计何哉'?

苏松巡按周元泰亦言刘家河急宜开浚,工部主事叶国华又疏请浚吴淞;俱下旨:`该部察议'(出``遗编'')。

史可法奏官多无益

史可法言:`今日江北有四藩、有督师、有抚按、有屯抚、有总督,不为不多矣。敌寇并至,曾何益毫末哉!臣近至扬州,一时集于城内者,有总督、有提督、有盐科,酬应繁杂,府县皆病。今又添监督,人人可以剥商,商本尽亏;新征不已,利归豪猾。不足之害,朝廷实自受之'。

吴适论云雾山

乙酉二月初六日,太监李国辅请往云雾山开采;命驰驿去。给事中吴适疏言:`云雾山,即名封禁山;纵横数百里,北通徽、池,南连八闽,东抵衢、严,西界信州。唐、宋以来,每为盗薮。其间深谷穷渊,虎狼接述;险阻极目,无径可攀。且地接祖陵龙脉,为神京右臂;历朝禁止樵牧,``封禁''所由名也。英宗初年,遣官采木;于是地方讹棍在相煽惑,而狐假之辈因之攘夺小民,招引匪类,共肆劫掠;兼多内外官属供亿之费,数邑坐困,民不聊生。近山良民,遂鸟兽散。大盗邓茂七等聚众数万,藉以为窟,攻城杀令;合四省兵力以讨之,十四年乃戡定,奉旨``照旧封禁'':往祸盖可鉴也。臣窃以界通四省,境地相岐,内阻峻岭、外多绝谷,绵延重叠,筚路崎岖;封禁既久,开凿维艰:不便一。林莽高深,重嶂叠峰;毒蛇猛兽,生育繁滋。一旦开伐,奔突狂噬,伤人必多:不便二。邃深幽奥,迥绝恒区;水不通舟、陆难移运;纵使输垂再出,畴令神输?不便三。乘传驿骚,有司困于供亿,谁筹正赋?且吏胥假公行私,何所不至。而力田小民弃本逐末,消磨岁月;土田有荒芜之虑,力役多死亡之忧:不便四。兴朝举动,天下仰望以卜安危。今以无益有害之事,而特遣重臣,摇动人心,倾危四省;垂之青史,贻讥后世:不便五。远迩传闻,必且蜂屯蚁聚,竞营巢穴;居奇召祸,约束无力。是使盗贼复生而杀戮再见:不便六。况臣讯之老父,佥云`此山地连陵寝,自正统初开伐,致伤地脉,遂酿土木之难,泄山川灵气:不便七。举此数端,有害无利;伏惟陛下采择'!国辅亦疏请中撤,俱不许;驰视如适言,报罢。

国辅系大司礼韩赞周养子;赞周阉寺中正人也,伤心时事,杜门休沐。国辅时在宫中,每有所匡救;时人以张永目之,马士英则视为眼中之钉。因属所私,以开采事诳国辅,具疏请往;其实,士英竟不在开采也。国辅提督勇卫营操练禁旅,及奉命往浙,士英竟夺营篆,授其子马锡;以乳臭儿绾兵柄,时事可知矣。适疏出,士英遂切恨之。

直言无讳,虽以此忤权相,身轻似叶而名重如山矣。

万元吉疆事疏

太仆少卿万元吉奏``疆事不堪再坏''疏曰:`臣待罪方郎,荷蒙简命监军江北。今陛辞前往,一得之愚,不敢不为皇上陈之。窃惟主术无过宽严,道在兼济;官常无过任议,义贵相资。先皇帝初莅海宇,惩逆党用事,刘削元气;委任臣工,力行宽大。诸臣狃之,争意见之元黄、略绸缪之桑土。大患当前,束手无策。先帝震怒,一时宵壬遂乘间抵隙,中以用严之说:凡告密、廷杖、加派、抽练,新法备行。使在朝者不暇救过,在野者无复聊生。然后号称振作,乃中外不宁,国家多故;十余年小人用严之效,彰彰如是。先帝悔之,于是更崇宽大,悉反前规;天下以为太平可致。诸臣复思竞贿赂、恣欺蒙,每趋愈下;再撄圣怒,诛杀方兴,宗社继没。盖诸臣之孽,每乘于先帝之宽;而先帝之严,亦每激于诸臣之玩:则以宽严之用偶偏也。昨岁孙传庭拥兵关中,识者以为不宜轻出,出则必败;然已有逗挠议之者矣。贼既渡河,臣即与阁臣史可法、姜曰广云:``急撤关宁吴三桂,俾随路迎击,可以一胜'';先帝召对,亦曾及此。然已有蹙地议之者矣。乃贼势薰灼,廷臣劝南迁、劝出储监国南都,语不择音,亦权宜应尔;然已有邪妄议之者矣。由事后而观,咸追恨违者之误国;设事幸不败,必共服议者之守经。天下事无全害、亦无全利,大率类是。当局者心怵无全利之害,谁敢违众独行;旁观者偏见无全害之利,必欲强人从我!年来督抚更置,专视苞苴;封疆功罪,悉从意见:御寇实着,概乎未讲。国事因之大坏莫救,则以任议之途大畸也。臣敢直究前事之失,以为后事之鉴。伏祈皇上,留神省览'!

御寇全疏

万元吉奏曰:`贼今被创入秦,挑精选锐,垂涎东南。转盼秋深,若出汉、商,则径抵襄城;出豫、宋,则直窥江北。两处兵民,积怨深怒;于斯时民必争迎贼以报兵,兵更退疑民而进畏贼。恐将士之在上游者却而趋下,在北岸者急而渡南。金陵重地,武备单弱,何以当此!臣入都将近十日,窃窥人情皆积薪厝火,安寝其上。居功者思为史册之矫诬,见才者不顾公论之注射,舌战徒纷,实备不讲。一旦有急,不识诸臣置陛下于何地?得毋令三桂等窃笑江左人物,功非功而才非才乎?从来战胜首称庙堂,若使在廷无公忠共济之雅,断未有能立功于外者!伏乞皇上申谕中外大小臣工,宜洗前习,猛励后图;毋急不可居之功名,毋冒不可违之清议!捐去成心,收集人望;务萃众志,以报大仇。集群谋以制大胜,社稷身名并受其福矣'!

累朝阙典未行疏

万元吉奏曰:`皇上前者恭谒孝陵,徐问懿文园陵所在,亲为展拜;臣随诸臣后,莫不手额斯举实为三百年来未有盛事也。先臣杨守陈尝议修``建文实录''有云:``国可废,史不可废''。卓哉两语,可称要言不烦。宏治中,布衣缪恭伏阙上书,请复建文时故号,爵其后裔奉祀。时系恭狱,以闻于上;敬皇帝诏勿罪。夫灭曲直不载,不若直陈往事而示之以无可增加也;削庙号弗隆,不若引景帝故事,还懿文当日追尊故号,祀之园寝,而配以建文君也。二事并系大典,伏乞皇上敕下廷臣集议:``建文实录''作何开局纂修?懿文故号、祀典作何厘正?若此举告成,千秋万世之下必传为美谈。抑臣更有请者:靖难死事诸臣,历蒙恩诏褒录;乃谥、荫诸典,尚阙有待。美逊国之君臣何厚,愧此时之节义多亏!良由高皇帝首褒余阙而斥危素,风励备至。靖难以后,正气渐就损削;故酿为今日狯猾卖国之徒,屈膝拜伪、腼颜见人也。请将靖难死事诸臣及北京各省直陷城殉节诸臣,敕下有司细归采录;编成一事分别二等,酌予谥荫、庙祀。仍颁行学宫,广示激劝:庶于晚近人心补救匪浅也'。

请恤死节诸臣疏

万元吉奏曰:`臣前护军四川,追剿献、操二贼,总兵猛如虎、参将刘士杰、游击郭关、守备猛光捷等听臣催督,从芦州至开县为程凡二千余里,日夜靡宁,遇贼即杀;无奈先时故辅不听臣言,早扼归路,致令我兵深入,刘士杰与郭关、猛光捷俱死之。此臣所目击最悉者。后臣丁艰回籍,猛如虎守南阳,闯贼用大炮攻城甚急;如虎以计破之,伤贼精兵数千人。既闻他门失守,如虎始下城,犹持短刀斫杀多人。至唐府国门,望北拜称``负恩'',被贼剸刃。此臣所访问最真者。如虎等阵亡数载,褒录未沾;伏乞皇上敕下兵部速议旌恤,以风示江北颠将。惟时同臣监军关内道副使曹心明调护秦兵,备尝艰险,屡奏俘馘;竟以积劳,尽瘁棉州。使得半通褒纶,荣其身后,差令不同腐草耳。蓟辽旧督赵光抃\textbackslash{}赴召于突骑之冲、受事于破口之后,骤令乌合身先被创,竟与误国督师骈首西市;迄今文武贵贱,莫不抱冤。并望皇上下部议复'。

\hypertarget{header-n37}{%
\subsection{卷五·南都甲乙纪(续)}\label{header-n37}}

南都公檄

四月戊午朔,南京兵部尚书史可法、户部尚书高宏图、工部尚书程注、都察院御史张慎言、兵部侍郎吕大器、翰林院詹事兼侍读学士姜曰广、太常寺卿何应瑞、应天府府尹刘士祯、鸿胪寺卿朱之臣、太常寺寺丞姚思孝、吏科给事中李沾、户科给事中罗万象、河南道御史郭维经、山东道御史陈良弼、广东道御史周元泰、山西道御史米寿图、陕西道御史加升一级王孙蕃、四川道御史朱国昌誓告天地,号召天下臣民起义、勤王、捐赀急事。

维崇祯十七年四月溯,南京参赞机务兵部尚书史可法等谨以宗庙危情、生民至计,布告普天臣子尝被今天子十七年之鸿庥、托高皇帝三百祀之阴骘。其言曰:窃闻遭时有道,类多以文事之盛而绌武功;遘会非常,正可以国恩之洪而征臣节。故天宝乱而常山、睢阳之事着,靖康靡而宗泽、李纲之气烈。彼皆慝从上作,衅可预知。然且侠骨铮铮,与长岳之峰而并厉;义风烈烈,拨霓裳之奏以争鸣。况休命笃于上天、明德光于良史,有若本朝者乎!力扫凶氛,二祖之廓清号同盘古;治从宽简,累朝之熙洽象拟华胥。迺至今土特兴,宏谟益备:孝庙之温恭俨在,世祖之祖武重光。常冲龄而扫恭、显之氛,立清宫府;于召对而发龚、黄之叹,总为编氓。以寇起而用兵,是虐民者寇也,而兵非得已;以兵兴而派饷,是糜饷者兵也,而饷非自私。顾犹诏旨勤颁,有``再累吾民''之语;每遇天灾修省,无``一时自逸''之心。蔬膳布袍,真能以天下之肥而忘己之瘦;蠲逋宥罪,不难以一人之过以就臣之名。是宜大业之宏昌,何意诸艰之骈集!理诚莫解,事有可陈。思为苍生而得人,上之张罗者诚广;责以赤心而报主,下之自矢者难言。家家有半闲之堂,事事同小儿之戏。果能功名比曹武惠,讵妨好官之得钱;竟无肝胆似汉淮阴,曾念一人之推食!成俗大都尔尔,贤者亦并悠悠。壅蔽实繁,担当何状?图之不早,病已成于养痈;局尚可为,涉必穷于灭顶。悲夫!悲夫!边尘未殄,寇焰旋腾;血溅天潢,烽传陵寝。秦称天府,谁能封以一丸;晋有霸图,无复追其三驾。迺者介马横驰夫畿辅,羽书不绝于殿廷;南北之耗莫通,河山之险尽失。天威不测,极知汉天子自有神灵;兵势无常,岂得谢太傅但凭歌啸!留都系四方之率,司马有九伐之经;义不共天,行将指日。克襄大举,实赖同仇。请无分宦游、无分家食,或世贵如王谢、或最胜若金张,或子虚之以赀起、或挽辂之以谈兴;乃至射策孝廉、明经文学,亦往往名斑国士、橐为里雄。合无各抒状谋、各团义旅,仗不需于武库,糗无壅于郇厨;飞附大军,力争一决。但群策直承黄钺,岂贼运得有白头!丑类立歼,普天大脯。此则万代之所瞻仰,虽九庙亦为之鉴临者也。倘策未暇于即戎,必义且先于助饷;多或抵小国之赋,少则割中人之家。幸济危机,何弦高之牛足惜;即非长物,亦曹洪之马是求。各付有司,转输留讨。此则事弥从便,气易为豪。至登垄巨商、联田富室,若与缙绅并举,亦自分谊攸殊。然使平准法行,即杨翟之雄岂得举其奇货;又如手实令在,将处士之号未可保其素封。凡称多筭之有余,总赖圣恩之无外。欲与其为义士,多方亦赖同盟。偶值佳缘,毋忘善诱!譬以同舟之谊,但凡在千八百国,畴非王臣?揆诸恤纬之心,决不至二十四城,遂无男子!呜呼!亲郊乃雍容之事,唐庄尚有崇韬;出塞本徼幸之图,汉武乃逢卜式。矧兹何日,敢曰无徒!不惟社稷之忧,即是身家之筭。始贼之巧于为饵,时亦有优孟之仁;迨我之既入其樊,莫不婴地狱之罚。齐姜、宋子,相率而入平康;珠户、绮窗,所过便成瓯脱。来俊臣之刑具,则公卿之被拷者痛尝;郑监门之画图,与老弱之受害者酷肖。是皆难民所说,足令听者寒心!夫连岁报陷,如西安、太原、武昌等处,皆行省也;其中金穴何止一家。若一时之牛酒不乏,虽八公之草木可驱;只坐一悭,遂成胥溺,岂不冤哉?欲图稳着,须问前车。诚清夜而念上恩,虽何曾之万钱有难下咽;更援古以筹得策,岂王衍之三窟便可藏身?同舟即是一家,破巢必无完卵;可不思之思之又重思之哉!法等智不足以效谋,愤何辞于即死;实切报殳之顾,辄通托钵之呼。人理尚存,我求必应。如缠情阿堵、绝念封疆,睢阳之援竟停,则霁云抽誓言之矢;荆州之粟独拥,则温峤有回指之旗。封章尚达于北辰,夺笔敢驾于南史!是为过计,亦属痴衷;见起君亲,约昭天日。法等无任斫地呼天、捶心沥血之至!

在籍兵部侍郎徐人龙、主事雷演祚移檄远近,浙江台绍道傅云龙与台州知州关继缙、通判杨体元并推官张明弼、知县宋腾熊、在籍陈函辉等亦誓师,临川曾益、吴群诸生王圣风、徐珩等各有檄文。

临海陈函辉讨贼檄

呜呼!故老有未经之变,禾黍伤心;普天同不共之仇,戈矛指发。壮士白衣冠,易水精通虹日;相君素车马,钱塘怒击江涛。呜呼!三月望后之报,此后盘古而蚀日月者也。昔我太祖高皇帝手挽三辰之轴,一扫腥膻;身钟二曜之英,双驱诚、谅。历年二百八纪,何人不沐皇恩?传世一十五朝,寰海尽行统历。迨我皇上崇祯御宇,十有七年于兹矣。始政诛珰,独励震霆作鼓;频年御敌,咸持宵旰为衣。九边寒暑,几警呼庚呼癸之嗟;万姓啼号,时切己溺己饥之痛。虽举朝肉食之多鄙,而一人辰极之未迁;遽至覆瓯,有何失序!呜呼!即尔纷然造逆之辈,畴无累世沐养之恩?乃者焰逼神京,九庙不获安其主;腥流宫寝,先帝不得正其终。罪极海山,贯知已满;惨深天地,誓岂共生!呜呼!谁秉国成?讵无封事!门户膏肓,河北贼置之不问;藩篱破坏,大将军竟苦罔闻!开门纳叛,皆观军容使者之流;卖主投降,尽宏文馆学士之辈。乞归便云有耻,徒死即系忠臣。此则劫运真遭阳九、百六之爻,而凡民并值柱折维裂之会矣。安禄山以番将代汉将,帐中猪早抽刀;李希烈自汴州奔蔡州,丸内鸩先进毒。凤既斩于京口,剖尸之僇安逃;景亦毙于舟中,跛足之凶终尽!无强不折,有逆必诛。又况汉德犹存,周历未过。赤眉、铜马,适开先武之中兴;夷羿、逄蒙,难免少康之并僇。臣子心存报主,``春秋''义大复仇。业赖社稷之灵,九人已推重耳;诚愤汉贼之并,六军必出祁山。呜呼!迁□金人,亦下铜盘之泪;随斑舞马,犹嘶玉陛之魂。矧具须眉,丘叨簪绂。身家非吾有,总属君恩;寝食岂能安,务伸国耻。握拳透爪,气吞一路鼓鼙;啮齿穿龂,声断五更鼓角。共洒申包胥之泪,誓焚百里视之舟。所幸泽、纲张翼宋之旗,协恭在位;愿如恂、禹挟兴汉之钺,磨厉以须。二三子何患无君,金陵咸尊正朔;千八国不期大会,江左赖有夷吾。莫非王土、莫非王臣,各请敌王所忾;岂曰同袍、岂曰同泽,咸歌``与子同仇''。聚神州、赤县之心,直穷巢穴;抒孝子、忠臣之愤,歼厥渠魁!斑马叶乎北风,旗常纪于南极。以赤手而挟神鼎,事在人为;即白衣而效前筹,君不我负!一洗欃枪晦蚀,日月重光;再开带砺山河,朝廷不小。海内共扶正气,神明鉴此血诚!谨檄。

陈璧论贼必灭有八

户部司务陈璧奏曰:`闯逆据秦越晋,破都逼帝;望风讹传者,非谓其智深勇沈、将卒超越,必谓其假仁仗义,百姓乐归。以臣所睹闯贼所为并贼将贼兵之情形决之,贼之必灭,断断无疑也!贼之来也,所过郡县,绝无战功;惟用奸细广布流言,辄云大兵百万、战将千员,顺者秋毫无犯、逆者屠戮全城。致荒残愚民被其煽惑,或望风逃窜、或俛首迎降;贼未至境,城市一空。及贼压境,奸淫掳掠,殆无噍类。民恨其诈,更受其酷,心切同仇:知其必灭者一也。逆贼进京,毫无大志。止张伪示,钩通长班;抄没勋戚,锁押百官追银两,或千金、或万金,昼夜夹打,惨酷万状。文官有钱者,不问才品、止看肥长,仍旧收用。流毒如此、用人如此:知其必灭者二也。贼兵沿门掳掠,抢财物、淫妇女,反复殆尽;仍各移据一家,责供狼餐。道路行人,短褐苟完,即缚拷夹。满城百姓,如在汤火,片刻难存;知其必灭者三也。贼将所号头目数人,各相雄长,目无贼主。闯逆屡欲僭位,其下每相对偶语云:``以响马拜响马,谁甘屈膝''?又言:``我辈汗血杀来天下,不是他的本事''。时聚族殿上,谑浪笑傲,秽亵不堪:知其必灭者四也。逆贼所追官民财物,下将十取二、三以解上将,上将又十取二、三以解闯逆。又有此将押追,彼将攘夺。吏政选用、将府拘囚,上下争利,文武争权:知其必灭者五也。贼兵掳括,腰缠多者千余金,少者亦不下三百、四百金;人人有富足何卿之心、无勇往赴战之气。临敌必至怯亡,平日渐将溃散:知其必灭者六也。燕京所积米麦有限,今贼兵人马作践,指日必尽;东南绝运,西北奇荒。破城不满二十日,米价已腾贵三倍,嗷嗷怨恨。半年之内,燕京内外必至绝粒:知其必灭者七也。贼来道经西鲁,与之市马,仍夺其金。西人痛恨,钩连清兵同总兵吴三桂连兵入讨,贼出兵一万,一阵尽没,仅存七人;贼又络续发兵,兵众怨叹。闯逆不及篡位,四月十二日亲统贼兵应敌;若四方义兵与清兵首尾夹击,知其必灭者八也。更以逆贼所据之势言之,其所据北直、陕西、山西、河南诸处,土地虽广,荒芜不治;人民鲜少,饥困难生:财贿无所出、税赋无所收,此贼势之必穷于内者矣。且逆贼三面距鲁,鲁知逆贼劫聚甚多,无一日忘贼之心。贼若南下,鲁必出大众以捣其巢;贼若守边,我又可出锐师击其后。贼若分头应敌,则兵饷单匮,北制南牵,又贼势之必穷于外者矣。此皆臣身亲目击,段段实境实情。夫贼情如此、贼势如彼,殄灭可期、恢复可□也'。

论列贼之情势,无一语不确。虽托空言,要皆实事。故录而存之。

张献忠杂志

甲申六月初一日,左良玉陈``恢复多城疏''曰:`臣于去年八月初复武昌,旋以江省为忧,约彼抚按备粮,拟即发兵往护;而抚、按二臣严文力拒臣兵,不使得前。贼因入袁州,祸中江西,非臣之过也。臣随选副将吴学礼于十月十三日恢复□□,因粮绝兵回;迨贼复返,臣乃遣马进忠统骑兵陆走江西。十一月二十七日,再复袁州;而江西省曾无颗粒寸草以劳军也。又于本月十三日恢复平乡,十二月初二日恢复万载,初五日恢复湖广澧陵,二十□日恢复长沙、湘潭、湘阴湖南一带城池,获各城伪守等官□苏民等,现在九江营监禁。遣副将马士秀等统步兵由水趋湖广,因于十二月二十四日恢复临湘,即于月有恢复岳州之大捷。又于十七年正月十六日恢复监利,二十二日恢复石首,二月十一日恢复公安。先是,臣又调副将惠登相率兵由均东下会师,于十七年正月二十四日同副将毛显文恢复惠安,又于二十六日乘胜直捣随州。未满三月,恢复府州县一十四处。屡次捷功,俱经臣与监军职方司主事李犹龙先后驰报。近檄袁、岳水陆兵马合进江追贼,而逆献始踉跄窜蜀,江右、湖南尽为宁宇。今图乘隙进复陵寝,方惬臣之本愿。督吕大器驻兵江西省城,从不出一步,乃有恢复吉安之报;顾不思献贼未到吉安,何事恢复?反疏``左兵无心剿贼,皆足为地方之患''等语。

湖按黄澍疏曰:`正月初三日,据郧标右营副将贾一选塘报:献逆于十二月十五日自荆河口搭浮桥渡河,十二月二十四日入荆□城,及老\{犭回\}\{犭回\}合营。先是,荆为闯逆部贼任、孟二贼所据,老\{犭回\}\{犭回\}曾与之争。自献逆渡河,而任、孟杳遁,为分、为合,似未可知'。
献、\{犭回\}在荆、襄,闯逆据承德;楚中入川、入豫要路,我往则寸寸皆□、彼来则处处皆通。今我取得前者,惟青、徐一线,亳、归数武而已。

六月十三日,张献忠陷涪州,再陷泸州。二十二日,献忠冲佛图关,遂围重庆四日。城中力不支,乃破;献忠屠之一城老幼无孑遗者。取壮丁去耳鼻、两臂,驱至各州县;言`兵至而不下者视此。但杀王府官府□绅封籍以待,则秋毫无犯矣'。由是所至,官民自乱,无不破竹下者。巡抚陈士奇时交代未至,与重庆推官王行俭俱死。瑞王避难在渝,阖门遇害。总兵赵光远降贼;士英犹请降敕奖之。八月初五日,献忠围成都。初九日,献忠陷成都;蜀王阖宫遇害,抚臣龙文光暨道、府皆死。三十日,贵抚范矿奏蜀□猖獗。

九月十四日,御史徐养心言:`闯贼使孟长庚筑江陵城,逆献复有□荆州之檄。万一顺流而东,浔阳、芜湖单弱,枢辅尚属筑舍,不几以金陵为孤注耶?总督死者止熊文灿耳,其地一味欺饰,失律之罪为何'?十二月十九日,四川佥事张一甲言:`川事□裂之甚,东则张贼直冲夔门,由忠、万而上,势如破竹;北则李贼渐逼阆中,广元、昭化以南久已降贼,通、巴一带日为摇黄土贼西掠。六月二十一日,张贼陷重庆,瑞王遇害,旧抚陈士奇拷死,将弁伤歼、兵民斫去一手者万计。八月初五日,张贼围省城。初九日,大炮崩城,官兵尽溃;士民惨死,拥尸塞流。蜀王、抚按、总镇三□,俱无下落。而李贼又于七日招安保宁,士民投顺;川北无兵,胆气已为摇黄折尽。自涪、渝继陷,各兵死手放归,见者寒心'。二十日,川督王应熊上言:`重庆、成都二府,凡川民敲骨吸髓,所供殆七、八十万,悉为贼有'。

李自成杂志

甲申七月十三日,李贼出关道雒阳,攻密县李际遇小寨。十八日,伪``顺''行牌至东昌云:`发兵三十万,由曹县至金乡缴'。十九日,参将夏有光报探至台儿庄,知闯贼见在平阳整兵,太原、潞安乡绅富户尽徙西安。二十日,李贼伪将宋朝臣兵至杜胜集,旧兵部职方主事郭献珂微服村居,召标将张成初与战于桃园,贼兵溃,追获朝臣斩之(``遗闻''作``郜献珂'')。八月二十八日,芜湖主事陈道晖奏榷关银被贼入,皆掠尽。十月二十一日,李贼出潼关,三营向归德、三营上裕州、二营踞郏县。十二月初六日,陈潜夫报李贼下河南。

伪官

甲申六月初七日,原任河南劝农兵部尚书丁启睿奏:`弟启光分守睢阳,命副将盛时隆、申吉、白维屏、游击黄承国、都司李定国、马国贞等密会归德知府桑开第,举人丁魁南、郭爌、余正绅等计擒各县伪官,俱于五月十六日一齐擒拿,获得归德府伪登河同知陈奇、商邱伪知县贾士俊父子仆三人、柘城伪知县郭经邦、鹿邑伪知县孙澄、宁陵伪知县许承荫、考城伪知县范倩、夏邑伪知县尚国隽并各伪契一颗。今将所获各贼解京;郭经邦以天暑病死,舁尸浦口俟验'。时济宁都司李允和杀伪官刘浚、尹宗衡、张问行、傅龙等九人,囚原任兖西道副使叛官王世英解京献俘。又开封府推官陈潜夫、寨勇李建知、刘洪起等各杀伪官南往,青州府衡藩率诸生驱杀伪官请徒内地。四川巡按御史刘之渤奏报:合江、仁怀擒杀贼杨腾凤、张见骘等。改潜夫江西道御史,巡按河南;启睿以原官为河南安抚。赐遇知、洪起总兵官,敕之渤下部纪录。

初十日,马士英疏曰:`为请申大逆之诛,以泄神人之愤事。缙绅之贪横无耻,至先帝末年而已极;结党营私、招权纳贿以致国事败坏,祸及宗社。闯贼入都之日,死忠者寥寥。降贼者强半侍从之班、清华之选;素号正人君子之流,皆稽首贼廷。如科臣光时亨力阻南迁之议,而身先迎贼;龚鼎孳降贼后,每见人则曰:``我本要死,小妾不肯''。其他逆臣,不可枚举。台省不纠弹、司寇不行法,巨窃疑焉。更有大逆之尤者,如庶吉士周钟劝进未已,复上书劝贼早定江南;又差人寄其子,称贼为新主,盛夸其英武仁明及恩遇之隆,以摇惑东南。亲友见者,无不愤恨;恨不立毁其家。昨臣病中,东镇刘泽清来见,诵其劝进表联云:``比尧、舜而多武功,迈汤、武而无惭德''。又闻其过先帝梓宫之前,扬扬得意,竟不下马。臣闻之,不胜发指。其伯父周应秋、周维持,皆魏忠贤门下走狗;本犯又为闯贼之臣:枭獍萃于一门,逆恶种于前世。臣按律:谋危社稷,谓之谋反。大逆不道,宜加赤族之诛,以为臣民之戒。今其胞兄周铨尚厕衣冠之列,其亲堂弟周镳俨余寅清之署;均当连坐,以清逆党。伏乞皇上大奋干断,敕下法司:先将本犯家属并周镳、周铨等严行提问,依律正法。其余从贼苟免诸臣,分别定罪。庶国法伸而人心傲,于□政不无小补矣'。

十二日,通政使刘士桢请令北面大小文武职官,俱着归原籍,静听朝廷处分,不得纷然奏辨;上从之。

十五日,叙东省擒杀伪官功,以李元和为首。

七月初二日,令叙山东擒伪功。

初八日,谕北京从逆诸臣效六等定罪。

二十五日,刘孔昭荐钱位坤。言`位坤曾经吴三桂收用,忠实可信;长安所刻``国变录'',为奸徒借题害人,不止龚彝受屈。请亟收用位坤'。

八月初三日,原任□□抚何谦自北逃归。

初四日,朱国昌言:`往者贼入都城,自阁部以至庶僚,无一不青衣小帽叩首贼廷者乎?至贼众已去,又思藏头换面,驾言不屈;潜踪觅线,冀燃死灰。知梁兆旸、何瑞征等,万口唾骂。至若刘大巩等,耻心荡然;当与周钟辈并行正法者也'。

初八日,谕刑部:`所拟从逆之臣,如领兵献策,即在庶僚岂可末减?督抚、总兵降贼,情罪极重,岂可列二等?京堂、科道、翰林受贼伪令,岂可止于绞?封疆大吏闻变倡逃,岂止于流?献女献婢,岂止于徒?诸臣负恩辱国至此,须有定案昭示天下'。

初九日,逮故大同知府张璘然、户侍郎党崇雅、祭酒薛所蕴。

二十日,伪太常丞项煜逮到。

二十七日,御史王孙蕃、方以智自亏臣节,复撰伪书以乱是非;命逮以智。

九月初一日,下部议王先通恤褒。先通,非王守仁后人,冒袭;降贼劝进,为贼声罪所诛。

十六日,浙抚任天成劾`浙属乡宦金汝砺、缪沅身污伪命,张璘然、方允昌为贼亲,一归一未归。李纲、徐家林俱受伪职,庶吉士鲁□、王自超、吴尔埙、魏学濂为贼所留,止学濂痛愤自缢;诸人犹恋身家,臣谊安容'。

十八日,田仰拿解光时亨至。

二十一日,高杰为匍匐南归之臣,请从末减。

十月初五日,降贼故尚书张缙彦自言在河北收义勇、诛伪官;大学士王铎保之。命以原官,总督北直、山西、河南军务;文武委用,给空名札二百。

十二日,御史胡时亨言:`近来章奏、文武升授,皆出勋臣之口;至从逆伪官,借口军前,蒙面求进。武臣不效命,谓文臣掣其肘;今不又武臣掣文臣之肘乎'?又言:`黄国琦、施凤仪补用,臣实骇然。黄则伪吏部掌朱封者;施则管仪仗时,语贼不可用亡国之器,愿自赔十金造者。此何人?而辱班行乎'!

十五日,兵科王□奏李祖述、朱元臣偷生负主,有愧诸勋;下部议。

十九日,谕兵部:`临淮侯李祖述奉命守门,城陷君亡,偷生南窜;该部严议'。命北归庶吉士史可程督辅私寓候议。刘泽清招禁商船为水营,荐黄国琦为监军。

二十日史可法荐北归谕德卫允文兼兵科,命监高杰军。

十一月初五日,丁启睿、丁魁楚合奏:有伪侍郎金之俊保举二人伪抚,遣人持檄文至,为刘良佐所获。二丁合辞待罪。

二十五日,御史沈宸铨劾张缙彦、王永吉、何谦、邱祖德、黄希宪、鲁化龙;命缙彦、永吉勿问,何谦等法司提究。

二十七日,刘泽清荐时敏海外兴屯。令苏京驻庙湾防海。

十二月初五日,春坊韩四维自言未经贼辱,弃家南奔;令复官。四维实愿施银一万求贼司业,而贼降为修撰者。

十一日,光时亨辨罪;不允。

二十日,受伪命李逢甲赠太仆少卿。
二十一日,刘泽清荐受伪命时敏仍以兵科开屯大瞿山。刑部尚书解学龙请宽贬节偷生诸臣如何瑞征、张若麒、杨观光、党崇雅、熊文举二十二人,应候三年定夺。

二十三日,解学龙上从逆诸臣六等罪:一等应磔,宋企郊等十一人;二等应斩拟长系秋决,光时亨等四人;三等应绞拟赎,陈名夏等七人;四等应戍拟赎,王孙蕙等十五人;五等应徒拟赎,宋学显、沈文然等十人;六等应杖拟赎,潘同春等八人;有疑另拟,翁元益等二十八人。保国公朱国弼等合疏纠刑官六失;御史张孙振亦言`从逆一案,明谕法宜从重;大司寇操此三尺,推诿半年,八人出脱。北人诸人乃贼弃之而来,非弃贼而来!解学龙卖法舞文,乞敕公鞫'!革学龙职;以高倬为刑部尚书。

乙酉正月初十日,韩四维逃归,托言前使岷府,不污贼尘;上谓`遣封在四月中,可未及受事,何得欺饰?姑着调用'!工科钱□奏:`科场大开贿赂,何瑞征、项煜公然市买'。

二十三日,苏松按周元奏杨枝起、宋学显、杨汝成、宋之绳、曹谷、朱积、翁元益既受伪官,岂容幸漏;令法司提问。

诛周钟等

四月初九(辛酉),杀从逆光时亨、周钟、武愫,又杀原任武德道佥事雷演祚、礼部主事周镳。镳与钟,从兄弟也,负时誉,与阮大铖有隙。士英参钟从逆,谓镳当坐;照磨张明弼奏镳险恶,朱统\{金类\}摭镳他事:论劾提问。演祚与大铖有怨,刘泽清疏攻吕大器及演祚,大铖复奏演祚不忠、不孝;吏科林有本继之。有旨:`从逆各犯及演祚二案,着法司速行讯结'。光时亨者,与李明睿不同声气,阻驾南迁者;故与四人同死。

周钟遁居嘉兴项仲展家,时遣无锡武举邹浩之往逮;钟见邹,谓之曰:`汝非有年谊乎'?邹曰:`然'。因伪云:`老年翁此去,亦不如不去;晚生妻子已下狱矣'。钟与千金,邹不受。钟曰:`去终是死,亦避不得矣'!遂行。至南京,杀于大中桥。临刑,谓众曰:`今杀我,天下遂太平乎'?时年四十四,万历壬寅生也。

边镇诸将

甲申六月十三日(己巳),以总兵吴志葵镇守吴淞。先是,江北诸镇兵不戢,眈眈思渡。志葵时为游击,随抚臣郑瑄镇京口。志葵悉心守御之,昼夜靡懈,江上以安;故有是擢。

十八日(甲戌),刘泽清请诛吕大器,指其起用王重掌选;又指其比雷演祚,谓`演祚为吴甡走狗,杀周延儒以媚东林'。泽清又荐张捷、邹之麟、张孙振、刘光斗及逃抚郭景昌、王永吉。

十一日(丁丑),塘报陕西全陷。马士英请亟奖赵光远,给以空札一百;时光远已降贼矣。

十三日(己卯),济宁回子兵朱继宗伤杀所署副将杨朴一家,而自为总兵;与李元和共事。

十八日(甲申),北将于永绶等领马兵千人,驻札镇江。浙江都司贾之奎领步兵至,止其地;及京口营兵与各路零卒分札西门外与教场等处。类聚繁杂,平日与市铺交易,未免争较锱铢,遂各怀嫌忿。复因马兵以贱值攫小儿瓜,相持不让,兵伤儿额;道路不平,攒殴之,缚而掷之江。马兵攘臂,欲得首事者甘心焉。问之,则浙兵居多,深以为恨;呼党攻斗,忿然驰马来。路遇浙营守备李大开,呵之不下;大开怒,抽矢射之,中数人。马兵谓浙营兵将皆欺我,群起攻杀;大开中矢,伤重即毙。时浙兵于道上有窜隐民家者,马兵借端挟索,恣其淫掠;焚东门外居民数十里。马兵有云:`四镇以杀抢封伯,吾辈何惮不为'!仇杀不解,几成大乱。祁抚军擐甲驰往压弹,地方以安。而马兵旋为史阁部调去仪真安插,其事得解。事闻,上以于永绶等四将驰千余兵纪律不彰,仇杀骇听,宜速处其首衅者;令赴史可法军前核治。可法下令总镇官,处分起事兵丁一、二名而已;其后兵将调集,悉听本处抚臣节制,着为令。命总兵黄斌卿防御京口、邱磊镇守山东。

七月初三日(戊子),命四镇各率兵由六合驰赴督辅调用;皆不奉诏。

初五日(庚寅),祁彪佳、黄斌卿总兵镇江,命金声桓驻防淮、扬。

初六日(辛卯),张凤翔手书移邱磊,言北兵甚迫,义不可往;已率义勇乡绅离东昌而来。

初九日,加李际遇、刘洪起总兵,防御河南。

初十日,定京营之制悉照北京。以杜宏域、杨御蕃、牟文绶补三大营各总兵,各统一营至五营;卞启光、窦国宁、胡文若补三大营各总兵,各统六营至十营;詹世勋等各补正副号顺。先是,牟文绶协防凤陵,见贼势纵横,捐赀募练义勇数千,以资战守;至是,有京营之命。即与刘良佐议:原兵愿留凤者、不愿留者,各听自便。于是不愿留者,即令原领兵官王先声、袁大仁等统率,并骑甲、器炮俱赴刘镇交什外,其余挈妻子先南走,期以随绶暂驻江浦四千余人。兵将不忍相离,绶奏`该督神机、巡逻二营名虽一万六千,实不及一半;倘隶此四千人于二营,可壮京营守御'。上下其章于所司。诏各镇举用大帅;刘泽清荐水陆大将马化豹、柏永馥,俱听史可法题用。疏入,上嘉其得体,故有是命。御史陈荩募兵云南,先携三万金备饷。

二十六日,杜文焕提督巡捕营。

八月初二日(丁巳),苏抚祁彪佳言:`镇臣黄斌卿躬提一旅至京口,正值乱兵肆劫,才得布置;郑鸿逵欲以上江调换他处,那借之钱粮如何销算'?

初七日,命左良玉开藩武昌;左梦庚、惠登相并都督佥事。

初九日(甲子),王应熊开藩遵义。

十五日,刘良玉移镇寿春。马士英荐汪硕德兄弟招募水师造船。

二十八日,刘泽清上言``进取之计'':`募数十万之兵,储数十万之饷,备十余万之马匹、器械,须整顿一、二年,乃可渡河。今惟恐姜曰广、刘宗周不得党胜为怏怏,臣不能随辅臣急于一渡也。今□已入临清,会兵南下;贼已道雒阳,攻密县。如此光景,寇不至江、清不至河不止也'。

二十九日(甲申),封福建总兵郑芝龙南安伯。都司同知陈谦奉命往闽,请乞召对,面陈军政机宜。并祈臣工尽涤积习,忘尔我门户之私;文武协和、中外交应,以赞中兴之业。谦镇粤时,曾与郑芝龙盟于羊城,矢心报国;近因寇患,陈追剿三策。部议谓其切于时务,可佐前筹;且与闽帅交善,令齎敕谕、金帛往闽奖赉芝龙,兼调其兵六千入防,即与郑鸿逵统领共兄一万之数。俟谦差旋淮浦,以旌贤劳。

顾锡畴请谥

甲申七月初八日,礼部尚书顾锡畴疏称:`文震孟正性直节,望重朝野。当熹庙初,``勤政讲学''一疏,直褫逆珰之魄;以致削夺,几陷危祸。复蒙先帝赐环,劳深讲幄,特荷拔置政府;竟为同列温体仁所抑速去,未至大用,齎志以殁。奉先帝赠恤之旨,而美谥未膺,公论惋惜。罗喻义正气孤标,著述多先儒所未发之旨。为南大司成,擒倡建逆珰之祠者置之法;风毅肃然。后以日讲不附会温体仁进呈讲章,遂致告;朝野重之。姚希孟学问志行,渊纯刚果。少以风节自励;一入仕途,遂为小人侧目,珰祸幸留硕果。后直先帝讲幄最久,启沃功深;又为温体仁所不容,抑郁以死。先帝恤之,已有赠荫,而谥典未举。吕维祺生平忠孝,捐助急公。雒阳陷日,烈烈以死,全大臣不辱之节。已经赠恤,未与易名之典。四臣立朝、殉难本末,近在数年内人人能道之者也。乃应得谥恤而久悬不补,则未免为盛朝之缺事矣。并请削体仁滥邀非分之谥,以正褒诛大义'。上以事关激劝,从之。

``遗闻''云:允顾锡畴议,削温体仁``文忠''谥;寻复之。予大学士文震孟谥``文肃''、刘一璟谥``文端''、贺逢圣谥``文忠''、礼部侍郎罗喻义谥``文介''、詹事姚希孟谥``文肃''、兵部尚书吕维祺谥``忠节''、山西巡抚蔡懋德谥``忠襄''、随州知州王寿一谥``忠愍'';懋德谥寻夺之。

``甲乙史''载:锡畴请谥在六月初十日,十一日予一璟、逢圣谥;及六月二十七日,谕礼部`温体仁贻毒深远,着削去谥,以明公道'。

北京殉难诸臣谥

九月初三日(戊子),赐北京殉难文臣二十一人、勋臣二人、戚臣一人祭葬、赠荫、祠谥有差:阁臣范景文谥``文贞'',户部尚书倪元璐谥``文正''、左都御史李邦华谥``忠文'',兵部侍郎王家彦谥``忠端'',刑部侍郎孟兆祥谥``忠贞'',右都御史施邦曜谥``忠介'',大理寺卿凌义渠谥``忠清'',太常少卿吴麟征谥``忠节'',左春坊庶子周凤翔谥``文节'',左谕德马世奇谥``文忠'',左中允刘理顺谥``文正'',检讨汪伟谥``文烈'',太仆寺丞申佳胤谥``节愍'',户科给事中吴甘来谥``忠节'',御史陈良谟谥``恭愍'',御史陈纯德谥``恭节'',御史王章谥``忠烈'',吏部员外许直谧``忠节'',兵部郎中成德谥``忠毅'',兵部主事金铉谥``忠节'',观政进士孟章明谥``节愍'',惠安伯张庆臻谥``忠武'',襄城伯李国桢谥``贞武'',驸马都尉巩永固谥``贞愍'',立祠南京,赐名``旌忠''。又赠死节诸生许琰官翰林院五经博士,从祀忠臣庙中。

``遗闻''云:赠刘理顺妻万氏、妾李氏、成德母张氏淑人,金铉母章氏、妾王氏、汪伟妻耿氏恭人,马世奇妾朱氏、季氏、陈良谟妾时氏孺人;建坊旌表。

予勋戚新乐侯刘文炳谥``忠壮''、左都督刘文耀谥``忠果''、太监王承恩、王之心谥``忠愍''、李凤翔谥``恭壮''(凤翔以降贼被杀者)、大同巡抚卫景瑗谥``忠毅''、宣府巡抚朱之冯谥``忠壮'';总兵官吴襄谥``忠壮'',特赠辽国公;周遇吉谥``忠武''。工部主事王钟彦、经历施溥、中书舍人宋天显各予祭葬,赠刑部郎中李逢甲太寺少卿、布衣汤琼中书舍人。

``甲乙史''载:王章、汪伟谥在七月初一日,张庆臻、李国桢、巩永固谥在九月十六日。又十月初十日,赐王承恩谥,立祠;以降贼夹死王之心荨七人附祀,各荫锦衣官。

开国诸臣谥

先后补予开国诸臣谥:郢国公冯国用谥``武翼'',济国公丁德兴谥``武襄'',德庆侯廖永忠谥``武勇'',定远侯王弼谥``武威'',长兴侯耿炳文谥``武壮'',永义侯桑世杰谥``忠烈'',河间王俞廷玉谥``武烈'',东胜侯汪兴祖谥``武愍'',□□侯茅成谥``武烈'',济阳侯丁普郎谥``武简'',高阳郡侯韩成谥``忠壮'',东邱郡侯花云谥``忠毅'',丹阳县男孙炎谥``忠愍'',当涂县子王恺谥``壮愍'',高阳郡侯许瑗谥``忠节'',缙云伯胡深谥``襄节'',御史中丞章溢谥``庄敏'',晋府长史桂彦良谥``敬裕'',詹事唐铎谥``敬安''、祭酒刘崧谥``恭介''、东莞伯何真谥``恭清''、平遥训导叶居升谥``忠愍'';姑孰郡公陶安、学士詹俱谥``文献''。

``甲乙史''载:丁德兴、冯国用、廖永忠、王弼、耿文炳谥在七月十五日,傅友德、冯胜、章溢、桂彦良谥在七月十九日,唐铎、刘崧、何真、叶居升谥在七月二十一日,桑世杰、俞廷玉、汪兴祖、茅成、丁普郎、韩成、花云谥在八月二十二日,陶安、詹同、孙炎、王恺、许瑗、胡深谥在九月十八日。

建文朝死难诸臣谥

补予建文朝死难诸臣谥:文学博士方孝孺谥``文正''、兵部尚书齐泰、太常寺卿黄子澄、刑部侍郎张昺、太常寺少卿卢原质、给事中叶福俱谥``节愍'',礼部尚书陈迪、御史大夫景清、大理少卿胡闰俱谥``忠烈''、兵部尚书铁铉谥``忠襄'',修撰王叔英谥``文忠'',礼部侍郎黄观谥``文贞'',户部侍郎卓敬、御史大夫练子宁俱谥``忠贞'',衡府纪善周是修、按察使王良俱谥``贞毅'',编修王艮、太常少卿廖升俱谥``文节''、刑部尚书毕昭、左赞善连枢俱谥``刚烈'',都御史茅大方、御史高翔、教授陈思贤、燕府伴读俞逢辰俱谥``忠愍'',给事黄钺谥``忠献'',御史曾凤诏、参军断事高巍俱谥``忠毅''、左拾遗戴德彝、御史魏冕俱谥``毅直''、知府姚善、知县颜伯瑺俱谥``忠惠''、大理寺丞邹瑾、兵部侍郎谭翌俱谥``忠愍'',都御史陈性善谥``忠节'',燕府长史葛诚谥``果愍'',刑部侍郎胡子昭谥``介愍'',谷府长史刘璟谥``刚节'',御史林英谥``毅节'',魏国公徐辉祖谥``忠贞'',越隽侯俞通渊、都指挥瞿能俱谥``襄烈''、卫卒储福谥``贞义'',都指挥谢贵、庄得俱谥``勇愍'',马宣谥``贞庄''、朱鉴谥``壮烈'';皆允给事中李清请也。

``甲乙史''载云:十二月二十八日,允建文诸臣谥:方孝孺等七十一人、俞通海等十七人;瞿能平阳伯、谢贵英山伯、王得分水伯、马宜全椒伯、朱鉴含山伯。

正德朝死谏诸臣谥

补予正德朝死谏诸臣谥:御史蒋钦谥``忠烈'',刑部员外陆震谥``忠定'',工部主事何遵谥``忠节'',刑部主事刘较谥``孝毅'',大理评事林公黼谥``忠恪'',行人孟阳谥``忠介'',李绍贤谥``忠端'',俞廷缵谥``忠愍'',李翰臣谥``忠毅''、詹轼谥``忠洁''、刘平甫谥``忠质'',给事中周玺谥``忠愍''、指挥张英谥``忠壮''。

``甲乙史''载:诸臣谥在九月二十日;内更有詹寅一人谥``忠宪''。

天启朝死谏难诸臣谥

补予天启朝死珰难诸臣谥:副都御史左光斗、给事中周朝瑞、御史周宗建、袁化中、李应升俱谥``忠毅'',黄尊素谥``忠端'',工部主事万燝谥``忠贞'',副使顾大章谥``忠愍''、苏松巡抚周起元谥``忠襄''。

``甲乙史''载:诸臣谥在九月二十日,内更有缪昌期一人;俱从部请也。

先后补谥

先后补予右都御史沈子木谥``恭靖''、工部尚书沈儆炌谥``褒敏'',副都御史张玮谥``清惠''、礼部尚书董其昌谥``文敏''、大学士何如宠谥``文端'',孙承宗谥``文忠'',太常少卿鹿善继谥``忠节'',大学士孔贞运谥``文忠'',蓟辽总督吴阿衡谥``忠毅'',简讨胡守恒谥``文节''。贞运以国变痛哭不食死,守恒、阿衡皆死难者。又予修撰沈懋学谥``文节''。谕德焦竑谥``文端''、祭酒陈仁锡谥``文庄'',礼部侍郎张邦纪谥``文懿''。仁锡初以忤珰削夺,寻得赐环。典较抡才,横经造士,生平究心钱谷、边屯、河漕、律历等书,著述几千卷,皆千秋金镜。子济生,官太仆主簿;命主祭。故兵部尚书于谦为临安伯。谦奠安宗社,有大功;为奸邪构祸。吏、礼部以恤不酬冤,为之请恤。复左都御史陈于庭原官,赠少保。

``甲乙史''载云:何如宠谥在九月十二日,张玮、董其昌谥在九月十三日,沈子木、沈儆炌谥在九月十六日,陈仁锡、张邦纪谥在十月初八日,沈懋学、焦竑谥在十一月初三日,吴阿衡谥在十一月十二日,胡守恒谥在十一月二十八日。守恒,崇祯戊辰进士,为湖州推官,入翰林;与无锡绅胡之竑通谱。癸未流寇破城,阖门被难。

七月初二日,予故总督卢象升谥``忠烈''。二十五日,予故巡按湖广刘熙祚谥``忠毅''。``遗闻''云:乙酉春,予吏部侍郎顾起元谥``文庄''、都督刘源清谥``武节''。

御史张孙振劾在告礼部尚书顾锡畴险邪,有玷秩宗;以其请削体仁谥而谥文震孟也。命锡畴致仕去,震孟、体仁确议。

补:甲申九月二十一日,赐降贼被杀内臣李凤翔谥``恭壮''。谥法之滥如此。

吴适参驳

十二月十三日(丁卯),户科吴适纠乱政监司:`一为陈之伸以衮东少参闻警潜逃,革职逮问;捏称部覆,朦补佥宪。一为夏万亨中书被察,题补劝农知县,加副使衔;弃地南奔,遂营齎诏之役,称副使;又借题迎护,升江西布政:以邑令半载而登兵牧。一为郭正中以举人罪加责戍,蒙选知州,避兵不赴,借名修历入京,奉旨驱逐;今又借危疆躐得佥宪。内此而捐,则从贼拔用之黄国琦应得昼锦矣'。

``编章''云:吴适抄参忭城伯赵之龙``荐用人才疏''谓:`陈尔翼颂逆有据,且荐崔呈秀为本兵;不可复用'。之龙再疏争之;适特疏言:`祖制:惟科臣专封驳之权,未闻勋爵而参驳正之司。勋臣党邪求胜,将部、科俱可不设;不几背明旨而蔑祖制乎'?是时,张捷秉铨,部务皆阮大铖一手握定,而选郎以贪黩济之,吏道庞杂。惟适办事垣中,抄驳侃侃,不惮权贵。若安远侯柳祚昌荐授程士达督理京营;适抄参`士达非科贡正途,勋臣乃提督大汉,非有标营之责,何得侵枢戎职掌以夺铨部权势'?怀庆知府郭仪凤疏言挂冠勤王,且诬巡抚方震孺贪状;适驳参`郡守无勤王之例,挂冠非入援之名。仪凤不侯宪檄、非奉明纶,擅离职守,饰词妄渎。察抚臣清执有素,仪凤秽迹着闻,必惧题参,先行反噬;自应严究,以杜刁风'。光禄署丞张星疏求考选;适驳参`张星以县令躁进降处,又挂察典;不惟望断清华之梦,亦已身绝仕进之阶。乃无端幻想,侥幸上赏,欺君孰甚!若一为点破,则阖门大典,不几为燃灰之地、向跃之门耶'?保定侯勋卫梁世烈请袭祖爵;适参`国难以来,虽王侯戚里咸喂虎狼,华胄重臣悉罗锋刃;而其间脱身图存、埋名溷俗者固亦不乏。该勋何以逆料其家之必歼,而以子嗣乎?万一本宗匹马来归,将夺诸该勋以授乎?抑姑仍之□而封乎?恐无此法纪也。该勋世受国恩,诚恢复有志,何难倡诸勋旧破家从军,自当直捣燕云,上为先帝复仇、次为诸勋雪耻。尔时访问本支有无存否,然后请诸朝令,光复祖爵,不亦休乎!昔李晟收复长安,下令军中曰:``五日内无得辄通家信''。今长安未复,殊非诸臣问家之日也'。遂安伯勋卫陈浚请袭;适参`自都邑变迁,山河阻绝;世次无凭,单词莫信,业奉明旨严覈。该勋一请、再请,若不能待;直视五等之封,祗同土块之乞。亦与菜佣都督一醉告身为可以弃时拾芥而攘取乎?况遂安勋卫今或遯迹闾阎、或从容归国,安可悬坐鬼录?使后来鞍马遗裔,执途人而可称;攀髯孤忠,裂本支而他续也'!中书舍人张钟龄请给部衔;适参`职方何官、监军何事?妄行陈请。若果报国有心,何官不可自效!而藉口赞画,辄请高衔;躁进尤甚'!他若革职司务朱济之、计处、吏部聂慎行、副使曾应瑞等躐跻营升,或疏劾、或抄参,不少假借。无奈人心日竞、启事日杂,虽经封驳,铨部竟置高阁;旋驳、旋用,使职掌扫地而宵小盈廷矣。

吴适,字幼洪,号静斋;苏之长洲人。崇祯丙子举人,丁丑进士。祖讳之佳,庚辰进士;以抗言国本为民,赠太仆少卿。然则吴黄门殆忠谏世传乎!语云:`鸷鸟累百,不如一鹗'。信然!

公与舅氏有年谊,当行取时,来谒南昌。时先君子在署中,见其年甚少,美丰仪,朱唇,其言明朗若金石声;每语不肖,极赏之。今读其诸参,益叹先君子之藻鉴也。复忆昔侍内父杭济之先生,先生最喜其专稿。是公之文章、政事、人物、家风,俱有大胜于人者。

熊汝霖奏献、闯二贼

吏科熊汝霖言:`献贼已至重庆、闯贼直至成都,破渝不守,意在顺流东下。北使臣所恃以为缓兵之要着也,左懋第请兵、请饷,望眼尚悬;王燮敕申未颁、马价未给为吁。此何时,而尚容姑待乎?皇上既以阮大铖为知兵,即当置之有用之地;若但优游司马,枢辅已饶为之,何须忝此'!

起刘同升等

``补遗''云:以易应昌为都察院副都御史、郭维经为佥都御史。起葛寅亮太常寺卿、成勇福建道御史、文安之詹事府詹事、刘同升翰林院侍讲、赵上春翰林院编修。寅亮、安之清望素着;勇以谏言护罪,直声振天下;同升、上春忠孝世传:皆以劾杨嗣昌夺情,与黄道周昌言去国者也。升贺世寿户部督仓尚书,起王志道吏部侍郎、申绍芳督饷侍郎。志道佐宪,以监视内臣越俎参官,廷诤,革职;绍芳居官清慎,因温体仁欲倾文震孟、许誉卿文致遣戍。至是雪之。

考选科道

考选游有伦、朱统铨、赵进美、沈宸荃、沈应旦、吴春枝、吴铸、吴适、林冲霄、刘天斗、左光明、蒋明玉、汤来贺、李日池、胡时亨为科道部属官,起补张采礼部仪制司主事、熊汝霖户科给事中、章正宸吏科给事中(``补遗'')。

高宏图乞归

十月初六日(庚申),大学士高宏图四疏乞归,允之。先是,章正宸争中旨升张有誉,朱统\{金类\}纠姜曰广。及争起用阮大铖,诸票拟俱不称旨,发改票,再拟再改;宏图力争,不听。至是,具疏请乞,遂予请告去。初,宏图家甚富;山东遭乱后,纤屑无存。惟一幼子自随,欲侨居常熟,不果。寄栖吴门僧寺,幼子附读村馆;已迁之会稽。

阄差

十一月初三日(丁亥),御史王化澄按广东,胡时忠视南京屯田。台规:铁板序差。时有广、闽、江、屯四差,时忠首应差。化澄名次第六,尚未应差;巧拜士英为门生,串谋总宪李沾、掌道张孙振疏创阄差,上下其手。时有旧河南道乔可聘梦与时忠空院奕碁云:`塞翁失马,未必非福'。后得因差归里养亲,莫非数也。

时忠,予舅氏也。初名时亨,耻与光逆同名,遂疏改今名。为御史时,屡言时政得失,京师号曰``冲锋''。时泰、靖两邑突张沙,争杀不已;出巡,立牌分界乃定。民歌思之,不仅遗爱江右也。后当按闽,不果;隐居养母,康熙庚戌春卒。

许都余党复乱

甲申八月十九日,浙抚左光先报士贼勾连逃兵,义乌、东阳许都余党复乱。二十日,批浙抚黄鸣俊奏:`左光先诱杀许都,不行善政,以致煽动;着鸣俊即相机剿抚'。二十三日,谕兵科:`许都初降、终杀、激变遗殃事情,着在朝浙臣直奏'。二十六日,兵科陈子龙言:`东阳再乱,全因县官诛求激变'。九月初三日,吏部奏:`姚孙矩贪酷,激变东阳'。命逮讯之。二十五日,上谕:`姚孙矩贪横激变许都,尚敢搜卖贼产,日事诛求,激成大祸,罪不容诛。左光先力庇贪令,毒流东越:着革职拿问'。

``编年''云:罢浙江巡抚黄鸣俊,降巡按任天成;以许都余党复叛处分未定也。先是,许都变起东阳,两浙汹汹;前任巡按左光先授计绍兴推官陈子龙诱擒斩之。光先为光斗弟,故与阮大铖有仇隙;又首劾大铖、马士英,故借诱降激变并议光先之罪而陷之,朝右无敢直言者。苏松巡抚祁彪佳独言`许都之变突发,东阳、义乌、浦江皆无坚城,光先事竣出境,闻变遄返,一切调兵措饷,皆其拮据;不一月而元凶授首,两□复安。乃今奉旨推求!夫弄兵揭竿,至于破城据邑;其罪岂不当死?当日兵威所迫,贼已穷蹙而后乞命,与阵擒无异,非诱降也。设诛锄不力,养虎遗患,后来国难方张,又不知作何举动矣!岂可反以激变罪之乎'?于是大铖等并切齿彪佳。而御史张孙振论劾彪佳贪奸,且所定策有异议;词连吴甡、郑三俊、刘宗周等。彪佳因罢去。史载:孙振追劾彪佳在十月三日,而彪佳之罢则十一月十三日也。

甚矣!史之难信也。由前说观之,则光先隐孙矩激变之罪,不为无过;由后说观之,则光先授计子龙诱擒之事,不为无功。夫以吴、越联壤,复躬当其时,犹言人人殊如此;况今古异时、四方异地,而欲凭臆以断志之,其为诬可胜道也。

马士英请纳银

八月十八日(癸卯),马士英请免各府州县童生应试,上户纳银六两、中户四两、下户三两,竟送学院收考。时溧阳知县李思谟不令童生纳银,待降五级(李降,乙酉正月二十一日事)。又诏行纳贡例:廪生纳银三百两、增六百两、附七百两。至明年正月十一日,□廪生加纳通判。又立开纳助工例:武英殿中书纳银九百两、文华中书一千五百两、内阁中书二千两、待诏三千两、拔贡一千两、推知衔二千两;监纪、职方,万千不等:皆以助军兴也。时为之语曰:`中书随地有,都督满街走;监纪多如羊,职方贱如狗。荫起千年尘,拔贡一呈首;扫尽江南钱,填塞马家口'。至乙酉二月,输纳富人授翰林、待诏等官;故更云`翰林满街走'也。

是时,士英卖官鬻爵,乡邑哄传。予在书斋,今日闻某挟赀赴京做官矣,明日又闻某鬻产买官矣。一时卖菜儿,莫不腰缠走白下;或云把总衔矣、或游击衔矣,且将赴某地矣。呜乎!此何时也,而小人犹尔梦梦;欲不亡得乎!

五陵注略

十二月二十二日(丙子),禁书坊不许行``五陵注略''。杨士聪曰:```五陵注略''者,许生重熙之所撰也,持论颇异。如叶福清之谥``忠''似谬、方德清之谥``正''似观?,朝论韪之。至书刘伯温□渡江、勋旧袭封出乡,人人推戴,前人已有言之;孔昭一见,大怒。适温相忌倪元□,恐其入阁;孔昭遂以倪锢妻事,与许□□作疏,意重在许,欲开大狱。上不允,亲票旨放归。许之书遂播行'。

新殿推恩

乙酉正月十九日,殿宇鼎新推恩:辅臣马士英、王铎、王应熊、史可法、尚书何应瑞、侍郎高倬、刘士祯、科道李维樾、游有伦、周元泰、主事朱日爃、秦祖襄各赐金币;内官韩赞周、庐九德、刘文忠、屈尚忠、张执中、田成、王肇基、高起潜、孙象贤、车天祥、乔尚、谷国珍、何志孔、赵兴邦、李灿、苏养性、孙珍、诸进朝银币外,各荫子锦衣指挥;李国辅锦衣千户。

三月二十二日(乙巳),殿工落成加恩:史可法、马士英、王铎、高宏图、姜曰广、管绍宁、朱之臣、高倬、刘士桢、何应瑞、陈盟、曹勋、葛寅亮各加官,惟顾锡畴不许叙。二十三日,叙内臣殿工功:加韩赞周、卢九德等三十五人赏赉有差。

朝政浊乱昏淫

时上深居禁中,惟渔幼女、饮火酒、伶官演戏为乐。修兴宁宫、建慈禧殿,大工繁费,宴赏皆不以节,国用匮乏。佃练湖,放洋船瓜、仪,掣盐芦州升课,甚至沽酒之家每筋定税钱一文;利之所在,搜括殆尽。盖马士英当国,与刘孔昭比,浊乱国是;内则韩、卢、张、田,外则张、李、杨、阮,一唱群和。兼有东平、兴平遥制内帐,忻城、抚宁侵挠吏事。边警日逼而主不知,小人弃时射利,识者已知不堪旦夕矣。

韩赞周、卢九德、张执中、田成、张捷、李沾、杨维垣、阮大铖、刘泽清、高杰、赵之龙、朱国弼。

□一月二十九日,上不豫,几殆。辅臣入候,群阉窃窃,有所指画;良久乃退。时上崇饮好内,权在群阉,田成为最;大臣皆因之固宠,致以贿成。时语云:`金刀莫试割,长弓早上弦;求田方得录,买马即□□(?)。

时有自京中来云:`阉人张执中年仅十九,上最嬖之;甚恣。诸臣欲见不得,即偶见,亦甚骄倨。惟马士英登门乃见,或留一清茶;士英已觉荣甚'。

除夕,上在兴宁宫,色忽不怡。韩赞周言:`新宫宜欢'。上曰:`梨园殊少佳者'!赞周泣曰:`臣以陛下令节或思皇考、或念先帝;乃作此想耶'!

赞周泣对,有汲黯、魏征之风。宏光此相,酷似东昏、后主一辈。``甲乙史''载:此为二十四(戊寅)事。予按令节,似除夕为真;故从之。

正月十二日(丙申),传旨天财库,召内竖五十三人进宫演戏、饮酒。上醉后,淫死童女二人;乃旧院雏妓,马、阮选进者。抬出北安门,付鸨儿葬之。嗣后屡有此事。由是曲中少女几尽,久亦不复抬出;而马、阮搜觅六院,亦无遗矣。二十四日(甲辰),复召内竖进宫演戏。

故事:宫中有大变,门夜半鸣钟。一夕,大内钟鸣,外廷闻之大骇,谓有非常。须臾,内竖启门而出,素鬼面头子数十欲演戏耳。可笑如此,安得不亡。时表弟胡鸿仪在屯田署中,亲所闻见者。

苏州有医者郑三山,日以春方进上,多鄙亵;上宠之。

\hypertarget{header-n42}{%
\subsection{卷六·南都甲乙纪(续)}\label{header-n42}}

先帝谥号

六月初六日(壬戌),谥大行皇帝曰``思宗烈皇帝''、皇后曰``孝节皇后''。

``大事纪''云:六月二十三日,御定先帝庙号``思宗''。先是,阁臣高宏图奉旨□拟已经点用,及考据典则,备极徽隆,不必再改。下部久矣:`着即颁诏行'。至七月初七日,遣各官颁行``追尊谥号诏''于天下。而``甲乙史''六月二十一日,忻诚伯赵之龙奏辩`先帝不当庙号曰``思'';``思''字非美'。盖之龙实不识一丁,李沾嗾使排高宏图也;复改``毅宗''。左良玉云:`思宗改谥,明示先帝不足``思'';为马士英第一罪'。永历谥为``威宗''。大清朝谥为``怀宗''。

追尊帝后

六月初六日,尊福恭王为``恭皇帝''、正妃曰``孝诚皇后''、生母邹氏曰``仁寿皇太后''、神庙贵妃郑氏曰``孝宁太皇太后''、上元妃曰``孝哲皇后''。

六月十九日(己亥),追复懿文太子``兴宗孝康皇帝'',追崇建文为``惠宗让皇帝''、景皇帝号``代宗''。

封常应俊

六月二十三日,封福府千户常应俊为襄卫伯,补青浦知县陈爊为中书舍人,予王铎弟镛、子无党世袭锦衣指挥使。盖应俊本革工,值宏光出亡,应俊负之行雪中数十里,脱于难;与镛、爊、无党俱扈从有功者也。

``甲乙史''云:六月初四日(庚申),以常自俊为左都督。``编年''、``遗闻''及``大事记''诸书俱载``应俊'',则志``自俊''或误。

太后至自河南

七月初六日(辛卯)寅刻,阁臣高宏图、姜曰广奉旨出郭迎圣母皇太后。先是,马士英奏曰:`雒阳变后,圣母寓河南郭家寨;有常守义者知之甚确。工臣程注亦向臣言之,当急图迎养。但事须机密,若兴大兵往迎,恐有阻滞。镇臣高杰言有参将王之纲者,曾在河南招抚李际遇,得其欢心;又有兵部王真卿奉命联络河南各山寨,颇有头绪。宜密谕督臣史可法遣王之纲、王真卿等与亲近内员同往李际遇处,密谕其具舟于河,拨兵护送。沿途而东,地方文武具仪卫迎于徐州,庶为妥便'。从之。至是,上命二辅出迎。

八月十三日(戊辰),太后至自河南,从仪凤门入;辰刻,上迎于午门。十四日,谕户、兵、工三部:`太后光临,限三日内搜括万金以备赐赏'。十六日,御用监诸进朝请给工科钱粮龙凤床座及床顶架一应器物并宫殿陈设金玉等项约数十万两。工部尚书何应瑞、侍郎高倬苦点金无术,恳祈崇俭;工科李清亦疏请节省。不听。十七日,高倬言:`臣在署办事,光禄寺开器皿计一万五千七百余件该费六千八百六十余两,厨役衣帽工料银九百四十余两。今日寇势方张,而赏赐银动以千万计,将何支?望皇上一熟筹也'!十九日,谕工部:`行宫湫隘,亟修西宫之园,刻期告成,以居皇太后'。

二十日,圣母南临加恩:士英、可法少傅、少保。二十三日,奖邹存仪力劝圣母有劳,封大兴伯。九月初九日,谕迎圣母有劳刘孔昭等六员荫子锦衣千户。十月初一,太后从人王镛、王无党授世指挥。

太子一案
乙酉三月甲申朔,皇太子至自金华,从石城门入,送止兴善寺。盖东宫旧竖李继周密奉御札,礼迎之至也。先是,吴三桂拥太子离永平,檄中外臣民:将奉入京即位。至榆河,阴逸之民间,使人导入皇姑寺。太监高起潜奔西山,太子自诣之;遂同至天津,浮海而南。八月,抵淮上;闻定王之沉,惧弗敢留,前至扬州。起潜访的中朝之旨,欲加弑害;其侄鸿胪序班〔梦箕〕义不可,挟之渡江,因栖于苏,复转于杭。太子不堪羁旅,渐露贵倨之色;于元夕观灯浩叹,遂为路人所窃指。梦箕惧祸及己,乃赴京密奏,并密启于士英;于是遣内竖李继周持御札召之。继周至杭,闻已诣金华,即往觅之;乃跪曰:`奴婢叩小爷头'。太子云:`我认得汝,但遗忘姓氏'。继周以告;且云:`奉新皇爷旨,迎接小爷进京'。太子云:`迎我进京,让皇帝与我做否'?继周云:`此事奴婢不知'。遂呈御札。时金华诸臣闻之,俱朝见馈礼。越二日,开舟至杭;抚臣张秉贞来朝,与文武百官导之而过。继周进京,先白士英,随奏宏光。时太子止石城门外,上复使北京张、王两内竖觇之;且迎之入城,权居兴善寺。二竖一见太子,即抱足大恸;见天寒衣薄,各解衣以进。上闻之,大怒曰:`真假未辨,何得便尔!太子即真,让位与否,尚须吾意。这厮敢如此'!遂掠二竖俱死,继周亦赐酖死。都人初闻青宫至,踊跃趋谒;文武官投职名帖者络绎不绝。最后,督营太监卢九德至;正视,一时难辨。太子呵之曰:`卢九德!汝何不叩首'?卢不觉叩头曰:`奴婢无礼'。太子曰:`汝隔几时,肥胖至此;可见在南京受用'。卢复叩头曰:`小爷保重'!觳觫辞出;与众曰:`我未尝伏侍东宫,如何里此;看来有些相像,却认不真'。随戒营兵曰:`吾等好好守视!真太子自应护卫;即假者,亦非小小神棍,须防逸去'!寻有旨谕文武官,不许私谒。自此,众不得见。中夜,移太子于大内。

三月初三日(丙戌),阮大铖自江北驰密书于士英;士英密奏,请以太子及从行二人俱下中城兵马〔司〕狱;遂捕高成、穆虎,夜更余肩舆太子入中城狱。时已大醉,狱中有大圈椅,坐其上即睡去。黎明,太子甫醒,见副兵马侍侧,问何人?以官对。太子曰:`汝去,我睡未足'。良久,问兵马曰:`汝何以不去'?兵马曰:`应在此伺候'。又问:`此何地'?曰:`公所'。又问:`纷纷者何'?曰:`行路人'。问:`何故皆蓝缕'?兵马未及答,太子曰:`我知之矣'!兵马以钱一串置几上,曰:`恐爷要用'。太子命彻去;兵马曰:`恐要买物'。太子颔之,令撩之壁间;曰:`你自去'!方出,顷之,校尉四人至前,叩头曰:`校尉伏侍爷的'。太子指壁间钱曰:`持去买香烛来!余钱可四人分之'。香烛至,太子即燃火间南北向再拜,大呼`太祖高皇帝、皇考皇帝'!复再叩首,号泣数声,拭泪就坐,饮泣不已;满狱为之凄然。

杨瑞甫,无锡人;时为校尉,监视太子于狱中。太子语之曰:`昔贼破北京,予趋出欲南走,时贼恐上南行,俱严兵堵截,无些子隙处;东、北二面亦然。独正西一路为贼巢窟,贼之来处兵众稍疏,予遂西走,终日不得食,晚宿野舍开浴堂家。及明,复走。自北七日不食,转而南,遂止于高梦箕家'(邑人口述)。

初五日(戊子),兵科戴英奏:`王之明假冒太子,请多官会审'。先是,杨维垣飏言于众曰:`驸马王昺侄孙王之明之貌,甚类太子'。莫即袭其言入奏。初六日(己丑),会审太子于大明门外。上先召中允刘正宗、李景濂入武英殿,谕之曰:`太子若真,将何容朕!卿等旧讲官,宜细认的'。正宗曰:`恐太子未能来此,臣当以说穷之,使无遁辞'。上悦。群臣先后至识所,太子东向踞坐,人尚不敢以囚礼待之。一官置禁城图于前,问之;曰:`此北京宫殿也'。指承华宫曰:`此我所居';指坤宁宫曰:`此我娘娘所居'。一官前问曰:`公主今何在'?曰:`不知,想必死矣'!一官问:`公主同宫女叩周国舅门'。太子曰:`同宫女叩周国舅门者,即我也'。刘正宗前曰:`我是讲官,汝识否'?太子一视,不答;问以讲所?曰:`文华殿'。问仿何书?曰:`诗句'。问写几行?曰:`写十行'。问讲读先后?曰:`忘之矣'。正宗更多其词以折之,太子笑而不应;曰:`汝以为伪,即伪可耳。我原不想与皇伯夺做皇帝'。诸臣无如何,仍以肩舆送入狱中。正宗遂奏:`眉目全不相似。所言讲所、仿书悉误'。时诸内侍皆谓非妄,特劫于上威,莫敢相剖;主以柄臣、和以讲幄如出一口,中外悲之。兵科戴英奏:`王之明假冒太子,质以先帝曾携之中左门而不答,问以嘉定伯姓名不答。其伪无疑!然稚年何能办此,必有大奸人挟为奇货;务在根究,宜敕法司严讯'。

``遗闻''云:`昔先帝携太子在中左门鞫吴昌时,故戴英问曰;`先帝亲鞫吴昌时于廷,东宫立何地'?对曰:`谁吴昌时'?英乃诘之曰:`汝是诈冒。以实告,当救汝'!即跪请救命。授以纸笔,供称:`高阳人王之明,系驸马王昺之侄孙。家破南奔,遇高梦箕家人穆虎,教以诈冒东宫'。王铎等面奏状,宏光流涕曰:`朕未有子,东宫若真,即东宫矣'!至初八日,集文武百官、举监、生员、耆老于午门外鞫之,梦箕、穆虎具服如之明言。下之明刑部狱;而京师士民谬以太子为非伪也'。此与他书所载大异。据此,则太子的系假冒矣。自供既明,即当如大悲弃市;何须屡次再审,狱久不决也?此非信史可知。

初七日(庚寅),有内官以密疏劝上曰:`东宫足骱异于常形,每骱则双;莫之能诬'。上令卢九德持至马士英寓商之。士英答疏云:`臣病在寓,皇上以竖臣密疏示臣,臣细阅之,其言虽是而疑处甚多。既为东宫,幸脱虎口,不即到官说明,却走绍兴:可疑一也。东宫厚质凝重,此人机变百出:可疑二也。公主现在周奎家,而云已死:可疑三也。左懋第在北,北亦有假太子事;懋第密书贻蔡奕琛,今奕琛抄腾进览。是太子不死于贼,即死于清矣。原日讲官方拱干在刑部狱,密谕来廷辨之。如其假冒,当付法司,与臣民共见而弃之。如其东宫,则祈取入深宫,留养别院;不可分封于外,以启奸人之心'。刑部严讯穆虎、高成,五毒备至,誓死不承假冒。穆虎云:`我家主是忠臣,直言奏闻,一字非谬;我等何得畏死背义'!法司气夺。梦箕复上书自明;并逮治之。

初八日(辛卯),复会审太子于午门。呼拱干在刑部狱;是晨张捷坐刑部尚书高倬家,以名帖召之至。捷曰:`先生恭喜!此番不惟释罪,且可以不次超擢,全在先生一言耳'。拱干唯唯。既谒门,百官集定。各役喝太子跪,太子仍前面西蹲倨。众拥拱干前,王铎指示太子曰:`此何人'?太子一见,即云`方先生在'。拱干惧,即退入人后,不敢复前,亦不敢言真伪。张孙振曰:`汝是王之明'!太子曰:`我南来,从不曾自己说是太子。你等不认罢了,何必改易姓名'?又曰:`李继周持皇伯谕帖来招我,非我自来者'。又曰:`你等不尝立皇考之朝乎!何一旦蒙面至此'?众官窃窃,有赧者、有恨者,莫之敢决。最后,王铎前曰:`千假万假,总是一假。是我一人承任,不必再审'!叱送还狱。应天府官蔡某自朝审出,人问云何?蔡云:`即非真太子,亦是久熟朝内事者'。旁一官云:`汝此言,明日即弃官矣'!自后朝臣不复有敢称太子者。京中谣曰:`若辨太子诈,射人先射马;若要太子强,擒贼先擒王'(一云:审时太子云:`我南来,从不曾说自已东宫;你不认罢了,何必改易姓名'?刑部尚书高倬、给事戴英齐声皆云:`既认王之明,何须再问?亦不必动刑。回奏便了')。

穆虎真义士,马、王辈不如仆隶远矣!

看太子语,原未尝自认王之明;乃高、戴齐声做作上去,众耳众目何在,而有掩盗鼠狗之说;小人真可笑也。至王铎身为大臣,敢云``承任'';真鄙夫、妄人也哉!

初九日(壬辰),中允李景濂奏云:`太子的系假冒,阁臣王铎再加质问,使之供吐姓名'。都察院粘示通衢:`王之明假冒太子'。

十四日(丁酉),谕刑部:`穆虎若非奸人,岂敢挟王之明冒认东宫?正月、二月,所成何局?往闽、往楚,欲干何事?岂高梦箕一人所办!主使附逆,实繁有徒。着法司穷究'!盖士英意在姜〔曰〕广辈,故严旨究问。黄得功上言:`东宫未必假冒,各官逢迎,不知的系何人、辨明何人,定为奸伪。先帝之子,即陛下之子;未有不明不白,付之刑狱。混然雷同,将人臣之义谓何?恐在廷诸臣谄徇者多、抗颜者少,即明白认识,亦谁敢出头取祸乎'!有旨:`王之明假冒系亲口供吐,有何逢迎?不必悬揣过虑'!

十五日(戊戌),复会审太子于朝。左都李沾先令校尉私戒太子,必须直言某。及审时,沾呼``王之明'';不应。喝问何不应?太子曰:`何不呼明之王'!沾喝上拶,太子号呼皇天上帝,声彻于内。士英传催放拶;沾复好言问之。太子曰:`汝令校尉嘱我,校尉自能言之;何必我言。前日追者何处,追者自知;何必问我'!高倬见其言急切,令扶出。将出朝,旧东宫伴读邱致中捧持大恸。上闻,即令擒下,发镇抚司严讯。有题诗于皇城曰:`百神护跸贼中来,会见前星闭后开;海上扶苏原未死,狱中病已又奚猜!安危定自关宗社,忠义何曾到鼎台!烈烈大行何处遇,普天空向棘圜哀'!冯可宗即讯高梦箕,梦箕列〔述〕自北来来历甚详,假冒欺隐至死不承;爰书故久未定。御史陈以瑞奏:`愚民观听易惑,故道路籍籍,皆以诸臣有意倾先帝之血胤'。有旨:`将王之明好生护养,勿骤加刑,以招民谤。俟正告天下,愚夫愚妇皆已明白,然后申法'。

李沾喝拶,与禽兽何异!梦箕至死不认,烈丈夫也。陈以瑞一疏,可云婉而直。
三月二十三日(丙午),刘良佐疏言:`王之明、童氏两案,未协舆论。恳求曲全两朝彝伦,毋贻天下后世口实'!有旨:`童氏妖妇,冒认结发。据供,系某陵王宫人,尚未悉真伪。王之明系驸马王昺之侄孙,避难南来,与梦箕家人穆虎沿途狎眤,冒认东宫,妄图不轨,正在严究。朕与先帝素无嫌怨,不得已从群臣之请,勉承重奇;岂有利天下之心,毒害其血胤!举朝文武,谁非先帝旧臣、谁不如卿,肯昧心至此!法司官即将两案刊布,以息群疑'。

二十八日(辛亥),左良玉具疏,请保全东宫,以安臣民之心。谓`东宫之来,吴三桂实有符验,史可法明知之而不敢言;此岂大臣之道!满朝诸臣,但知逢君,不惜大体。前者李贼逆乱,尚锡王封,不忍遽加刑害;何至一家,反视为仇?明知穷究并无别情,必欲辗转诛求,遂使皇上忘屋乌之德、臣下绝委裘之义!普天同怨,皇上独与二、三奸臣保守天下,无是理也!亲亲而仁民,愿皇上省之'!有旨:`东宫果真,当不失王封。但王之明被穆虎使冒太子,正在根究奸党。其吴三桂、史可法等语,必系讹传。法司将审明节略,宣谕该藩'。

四月初一日(癸丑),工部侍郎何楷奏:`镇臣疏东宫甚明'。有旨:`此疏岂可流传,必非镇臣之意,令提塘官立行追毁;敢有鼓煽者,兵部立擒正法'!

初二日(甲寅),湖广巡抚何腾蛟疏言:`太子到南,何人奏闻?何人物色?既召至京,马士英何以独知其伪?既是王昺之侄孙,何人举发?内官公侯多北来之人,何无一人确认,而泛云自供?梦箕前后二疏,何以不发抄传?明旨愈宣,则臣下愈惑。此事关天下万世是非,不可不慎'!有旨:`王之明自供甚明,百官士民万目昭然,不日即将口词章疏刊行。何腾蛟不必滋扰'!

十三日(乙丑),御史张兆熊奏:`伪太子一案,谤议遍处沸腾'。上命即将口词章疏连夜速刻,即付诏使逐郡宣布。

十六日(戊辰),袁继咸奏:`良玉举兵东下,请赦太子以遏止之'!有旨:`王之明的系假冒,如果先帝遗体,朕岂无慈爱?人臣何即称兵犯阙!继咸身为大臣兼拥兵众,如何说不能堵止'!又``编年''云:`江督袁继咸疏言:``太子居移气养,必非外间儿童所能假袭。王昺原系富族,且高阳未闻屠害,岂无父兄群从,何事只身流转到南?既走绍兴,于朝廷有何关系,遣人踪迹召来?诈冒从何而起?望陛下勿信偏词,使一人免向隅之悲,则宇宙亨荡平之福矣''。有旨:``王之明不刑自认,高梦箕、穆虎合口输情。诸臣无端过疑,何视朕太薄、视廷臣太浅!继咸身为大臣,不得过听讹言,别生臆揣'''!

十七日(己巳),史可法恭请召见,面言东宫处分,以息群□。有旨:`西警方急,卿专心料理;待奏凯后见'。可法叹曰:```奏凯''二字,谈何容易!诚如上所言,面君不知何日矣'!

不要史公回京,其事便有可疑。 北太子一案
先帝共三子,太子年十六,定、永二王皆十三岁。闯入京时大索,惟永王不知所在。自成东出,人见太子马衔尾随后,不见定王。或曰:已先日随闯出京,过通州,马上失一履。有人拾而进,王伸足与着;因问`军乎、民乎'?人以民对。王曰:`军则食我家饭者;民方受征税之苦,有何好事到汝'?其人泣,王亦泣谢之。自成战败西还,不见太子随后;人传太子归吴三桂军中矣。

十月,有男子自诣周中书家求见公主;相抱持大哭,滞留不去。周仆逐之,遂为街道所奏。明日,殿中勘之,言宫中事颇合;以讯内官,莫敢认者(一说嘉定伯周奎家)。

有一杨姓内监在旁;太子曰:`此杨某,曾侍我'。杨即诈曰:`奴婢姓张,先服侍者非我也'!又呼旧侍卫锦衣卒十人讯之,咸曰:`是永王'。有晋王者,山西从闯来,因留京师;独言其伪。于是言真者,皆下狱。刑曹郎钱凤览详讯,遂以真皇子报命。晋王抵览,览勃然语侵晋王。复廷讯之,内阁谢升执以为伪;太子曰:`某事,先生忆之否'?升默然,一揖退。凤览面叱升不臣。正阳门商民数人具疏救皇子,詈谢升禽兽无道;具疏人亦下狱。乙酉正月初十日,摄政王谓廷臣曰:`太子真伪无伤,但晋王明朝宗室、谢升明朝大臣,凤览呵晋王、百姓骂谢升,皆乱民也'。命系狱者尽杀。谢升早朝,见凤览与拱手,颈忽渐垂;时时自语曰:`钱先生饶我'!肿溃即死。四月初六日,凤阳民张三聚众誓救皇子,以杨生员为谋主,生员孙三应之;俱擒杀。初十日,太子遂死。

钱凤览,字子端,号兰台;会稽人,相国麟武公之孙。以祖荫,入中书;烈皇帝授刑部主事。宏光时,以东宫事,北京廷臣俱斥为假。凤览独疏争之,其略曰:`太子危地,死生之权,一在朝廷。据其供词,保者、验者确有凭证。在部五日,悲惧言动,绝无装饰。今责其身大音宏为非真耶;人幼而渺小,至十六而顿长且大,比比也。责以不能书写为非真耶;东宫素无能书之名。若责以不能尽悉宫中事耶;播迁流窜,魂魄未安,人于富贵时多不经意,试问各官朝贺跪拜惟听鸿胪传呼而已,能于仓猝中悉其礼数否?太子在宫中,未寒而衣、未饥而食,随侍者众,安能呼姓名?试问各官书吏、皂役等,几何人能一一悉其姓名、面貌否?当时二王在刘宗敏家,人心止有二王,不知有太子;今诘问时,不能明对者。贵处东宫,何堪挫辱;自不可以民犯同观也。总之,大臣不认,则小臣瞻顾;内员不认,则外员亦箝口。然天地祖宗,不可欺灭;敢以死争之'!疏上,下狱。法吏讽之曰:`苟易汝言,则生'。凤览毅然曰:`我身早办一死耳,言不可易'。竟坐诛死。事闻于南,赠以太仆寺卿,谥``忠毅''。

三皇子一案

大清顺治八年冬月,有人首三皇子在民间,擒捉至马提督府审问。皇子自书供云:`云庵,系崇祯第三子,名慈焕;年二十岁。兄慈烺,即东宫;同为周后所生。弟名慈灿,田妃生。焕居景仁宫,乳母邓、蒋;八岁就外傅,讲读官傅、张。贼犯都时,先帝托予张近侍及指挥黄贵送周皇亲家,不纳;潜藏民间,为贼搜出。随营到山海关;闯败,携至潼关,随营至荆、襄,遇左良玉战,贼败散,即随左营,改姓黄,称为黄贵叔。左兵为黄得功所败,黄蜚掳左兵船,杀贵;张近侍以实告,蜚秘其事。明年五月,得功亡,蜚携走太湖;遇江西乐安王,蜚托之。王携往孝丰,遇瑞昌王;乐安往闽,以予托瑞昌转藏。九月,诣于潜乡官余文渊家,假称宋座师公子。有湖广人陈砥流,时相亲密;砥流改名李王台,算命浪迹,得太平府乡友夏名卿重义,即与名卿同至于潜来接。予在陈监生家,监生与文渊说知而别;予改姓孙,名卿以女字之。四年十二月,余文渊与知县不和,前事遂露;行文太平,查不获。五年五月朔,予削发为僧,号云庵;或称一鉴、或称起云。砥流无定名,随口应人;浪迹江北各庵。砥流访知宁国府秀才沈辰伯好义,六年七月同予往访,遇于船中;一老秀才吕飞六善诗文,辰伯即托飞六留家读书。八年闰二月,辞别沈、吕二人,与砥流复到夏家。三月完姻;因夏贫苦,自租乡村空屋一间居住,渡日维艰。四月,与砥流议往芜湖借银二十两,买细茶同徽商汪礼仙往苏州卖。礼仙与常州杨秀甫、吴中虎邱相识,茶卖毕,同到常州。秀甫言邹介之是好人,到其家住几日;介之又言路迈是好人,即往谒路迈。临行时,送吴中诗扇一具、银五钱。在路迈家住几日,将因夏家,不意吴中私作假札贾利不遂,因出首于抚院。抚院差官先往宁国沈、吕二家跟寻,至芜湖即获砥流;予挺身出,随抚院差官起行。于途遇江宁赵同知、当涂某知县带到太平,随到江宁也'。

太子杂志

甲申六月十八日,刘泽清奏:`有典史顾元龄,系浙江钱塘人;五月初二日出北京,传言皇太子卒于乱军,其定王、永王俱于贼走之日遇害于王府二条巷吴总兵宅内'。

七月十七日,``大事记''载王燮塘报。

八月二十九日,召北来太监高起潜陛见。起潜实奉太子浮海至南,朝论讳之。

九月丙戌朔,朱国弼、赵之龙上太子及定、永二王谥;时太子南来,欲断之也。

二十五日(庚戌),初,袁妃公主受上刃不死,带伤出宫,依老中书周元振家。永王久潜民间,至是自出,求见妃主,抱持大恸。元振惧,奏闻。大清朝使内院谢升验视,执言其伪;下之狱。

十月二十七日(辛已),鸿胪寺少卿高梦箕北来复任,谢恩。

十一月乙酉朔,太子潜居兴教寺,高起潜私闻于马士英,遣人杀之。及至,而太子已先一日渡江南遁矣。

十二月二十四日(戊寅),管绍宁言东宫确遇害;命于明年二月为东宫制服。至乙酉二月十一日(甲子),绍宁请谥皇太子曰``献愍''、永王曰``悼''、定王曰``哀''。时定王已沈于海,皇太子方遁绍兴,上密令内使召之;管绍宁先定谥以绝之也。

东村老人曰:国变后,皇子凡三见:北京则自诣周中书家、南京则自内使召来、太平则有人出首者,人皆以为伪。愚谓不然。在北京,一以为永王、一以为太子;若是太子,则南京信伪矣,马士英已言之。然据士英疏云:`既为东宫,幸脱虎口,不即到官,却走绍兴'。即其言覈之,既非东宫,彼自走绍兴,于朝廷何关利害而遣人追之来?不可解也。初到时安置僧寺,百官递帖;旋谕禁止。多兵杂沓于街,似护似防;遂取入官,越日付之狱:何多周旋也!多官会审不决,王铎一人定假,李沾始喝用刑,确然伪矣;又不加之缧绁,仍以肩舆付狱,一对板前导:不可解也!我不能随人雷同,且存当日之实案耳。

又曰:三皇子,定王也;有可疑者。既依良玉,当左兵东下,必喜得王,何故隐名?迨黄蜚一帆到海,寻依李监奉义阳王;何故舍皇子而戴宗室?事固有不可度者,存疑可耳。

童妃一案

乙酉三月十三日(丙申),有童氏自称旧妃,自越其杰所解至;上命付锦衣卫监候。初,上为郡王,娶妃黄氏,早逝。既为世子,又娶李氏;洛阳遭变,又亡。嗣王之岁,初封童氏为妃,曾生一子,不育;已而遭乱播迁,各不相顾。又弃藩南奔,太妃与妃各依人自活。太妃至南,陈潜夫奏妃故在,上弗召;至是自诣其杰所。其杰不听隐,解至南;上弗善也,系之狱。妃在狱,细书入宫日月、相离情事甚悉;求冯可宗上达,上弃去弗视。至四月初六日,谕襄卫伯常应俊:`朕藩邸事,宜卿所详;童氏生育皇嗣,绝无影响。冯可宗辞审童氏,着太监屈尚忠会同严审'。初七日(已未),以童氏狱词所连于史可法营中逮庶吉士吴尔埙及中军孙秀。

``遗闻''云:童氏本周府宫人,逃乱至尉氏县,遇上于旅邸,相依生一子,已六岁。已而贼破京师,播迁流离,遂相失云。刘良左言童氏非假冒;马士英亦言`苟非至情所关,谁敢与陛下称敌体?宜迎归内。密谕河南巡抚迎致皇子,以尉臣民之望、以消奸宄之心'。上命屈尚忠严刑酷拷,童氏号呼诅骂;寻死狱中。

``野史''云:`马士英语阮大铖曰:`童氏系旧妃,上不肯认;如何'?大铖曰:`吾辈只观上意;上既不认,应置之死'。张捷曰:`太重'!大铖曰:`真则真、假则假,恻隐之心,岂今日作用乎'!士英曰:`真假未辨,从容再处'。

童氏系河南人,知书;与冯可宗云:`吾在尉氏县遇上,即至店中叩首;上手扶起,携置怀中。且云:``我伴无人,李妃不知所在;汝貌好,在此事我''!从之居四十日,闻流寇寖近,上挈我南走。至许州,遇太妃,悲喜交集。州官闻之,给公馆及廪饩。居八月,养一子,弥月即死;时已有内相随侍矣。及李贼破京,地方难容,上又走;中途遇土贼折散'。童氏述至此,呼天大哭;又云:`时同太妃流散甚苦。后闻上为帝,大喜。谁知他负心,止接太妃进宫,不来接我!至此又不肯认,天乎!这短命人,少不得死我眼前。汝为锦衣官,求汝代言;将字与他,视如何答我'!可宗见所陈本末甚详,入奏。上见童氏书,面赤,掷地曰:`吾不认得妖妇,速速严讯'!可宗不敢再奏。次日,呼毛牢子,传谕童氏云云。童氏大哭,且咒且詈;饮食不进,遂染重疾。可宗密奏,竟不批发。时□人詹自植闯入武英门坐御幄,妄语;又有疯癫白应元闯入御殿,肆骂:俱奉旨杖死。牢子等惧,遂不饮食童氏,饿死狱中。

``遗闻''载:`生子六岁,士英疏迎致皇子';而``编年''、``甲乙史''童妃口词,则云`生而不育,弥月即死',似为近之。呜呼!宏光薄行甚矣。

``甲乙史'':`四月初一日,詹有恒混入宫门,秽言辱骂:着打一百'。则是有恒,非自植也;二字或相似而误。

附录``童妃续记'':崇祯十四年,张献忠破福藩,王遇害;世子只身逃出,潜内城脚之厕室。有府皂刘正学者负一危病之母意拟跳城;世子浼之。刘见世子虽青年,体实肥重;跃出,安能逃命。世子曰:`尔母老颓,贼见之必不害;尔能救我出城,后自还尔富贵。吾乃福王嫡子也'!刘为筹之于邻近染坊中。见有旧黄绢伞并衣服等,室皆无(?);又取为世子包衬头面与上身,外以伞里之又用绳紧缚,择城斜垣处滚下。刘再安置其母,复跃出解之,幸不伤寸肤;乃与间道趋野外。约行五十余里,世子困不能前;刘解所衣纱裙一袭易旧破椅,两人舁之。又前往二十里,借宿荒村,流贼之氛远矣;刘诫勿露王府字,但云是教书先生。刘归觅母,果无恙;移母居于乡,再来访世子。众皆谓东渡东河始安;相与步行二百里,渡河至曹州界之新店,见有酒标。居其店之空室;店无男主,孀妪当炉,有一弱子与长女童氏,家颇裕。刘浼之,使世子安其身,因教其子读小书。刘复归;过冬再访,世子已迁入内室,则尽其邻之蒙童而就学矣。刘见其隔内外之木板有隙二、三寸,若内外相视然;已疑及其家之长女。然世子之身已得所,刘遂归。再阅月,李闯又破怀庆府。时亲王之暂栖此城者,为周、璐、崇三王;逃出流离,复各汇集,从水道由曹州南下。时为崇祯十七年二月,又逢京变,挽泊世子所寓近处。世子又会其女之夫家有构衅情,乃趋入舟边,诉履历于三王。又有福藩旧内侍田成、应进二人在内,识故主,遂同舟下淮安。

时三王俱有宫眷,惟福世子葛巾布袍而已。四月初一日,入仪真。北都三月十九日之信已确,留京诸公会议拥立,史可法、高宏图、程注、张慎言、姜曰广、李沾、郭维经、何应瑞等皆属意于潞王。马士英时在凤阳,不欲徇留京诸公意;内贿勋臣刘孔昭、外贿镇臣刘泽清,先阴使人导福世子借漕抚路振飞船在仪真载之过江,即挟诸大僚见之舟次。士英首荐房师阮大铖,谓`亟用此人,方可议中兴之事'。时有应天府生员何光显亦于舟次上揭,有``正国体,以正人心''议;隐制大铖一党应起用也。马、阮甚恨之。福世子五月十六日正位,大赦;改明年为宏光。太后亦自卫辉来;年同世子逃出而失散者一皮匠护藏之,至是封伯。何光显知宏光在曹州有童姓女事,密奏前迎;即遣仪真所来\{舟周\},彩画龙凤,差内官田成等迎接来京。七月二十日到水西门,二十一日拟进大内;合城小民结彩供香,皆谓圣后进朝。而马士英秉政,一凭大铖主裁,以为后之来也,自何光显,后立而光显内助之力巨矣;亟尼之以败乃事。鸾舆已进朝门,忽传太后懿旨:`在藩原配已经死难,并未再婚;今突闻有童氏擅自入京,必系假伪奸棍引诱。着三法勘问'!时阮大铖职总宪事,举朝承风旨,竟加刑讯问;各刑曹今官日上拶、明日上夹。童氏有随来族兄,亦潜逃全命。荒村野居之孤女,权贵以``冒认''二字加之,大内又不出一旨,何从分辨!九月初一日,河南刘正学踉跄而来;先知护太后者已封伯,谓己之功不在皮匠下。乃一入城,便知讯质童女事;倡言其事之真,谓朝官不宜如此诬国,已大触时忌。马、阮闻之,深嫉其人。疏入,留中;见朝,不许。后竟直闯朝堂,攘臂乞陈。宏光但云`候旨';童女亦置于狱。明年五月城破,童女不知随何人而去,刘正学亦逃出城。阮大铖为乱兵索金银,活钉入棺埋之地下。马士英逃至浙江绍兴府,亦为乱兵所杀(按此纪与各书所载不合,不知何所援引?姑存之)。
大悲僧假称楚王

甲申十二月,南京水西门外小民王二至西城兵马司,报一和尚自言当今之亲王,速往报,使彼前迎。兵马司申文巡城御史入奏,宏光批:`着中军都督蔡忠去拿'。忠率营兵四十、家丁二人驰往,见和尚坐草厅;忠入,问曰:`汝何人,敢称亲王?恐得罪'!和尚曰:`汝何人?敢问我'!左右曰:`都督蔡爷'。和尚曰:`既是官儿,亦宜行礼;我亦不较。且问汝来何故?得毋拏我否'?忠曰:`奉圣旨,请汝进去'。和尚即行。忠授马乘之,入城。有旨:`委戎政赵之龙、锦衣掌堂冯可宗在都督府会蔡忠勘问'。是十二月十七日事。和尚供:`我是定王,为国变出家,法名大悲。今潞王贤明,应为天子;欲宏光让位'。又牵出钱、申二大臣,言语支吾。赵之龙和颜授以纸笔,命彼自供。奏闻,宏光命刑部鞫讯,系是齐庶宗诈冒定王;复批九卿科道俱在城隍庙会审,端是诈伪。合词上奏,即斩首西市。

此野史也,他书载乙酉正月事。 诏选淑女

八月初二日(丁己),科臣陈子龙奏:`有中使四出搜采,凡有女子之家,黄纸贴额,持之而去;闾井骚然。明旨未经有司,中使私自搜求,殊非法纪。又前见收选内员,虑市井无籍自宫希进;昨闻果有父子同奄者。先朝若瑾、若贤,皆壮而自宫者也'。又御史朱国昌言:`有北城士民呈称历选宫嫔,必巡司、州县定地开报。今未见官示,忽有棍徒、哨凶擅入人家,不拘长幼,概云``抬去'';但云``大者选侍宫帏,小者教习戏曲''。街坊缄口,不敢一言'。二十二日,群奄肆扰收女;陈子龙言之,命禁讹传,棍徒不许借端诈骗。

二十六日,传皇太后遴选中宫。

九月初九日,选淑女黄氏、郭氏、戴氏送内;命再选。

十八日,韩赞周请大婚礼物;着光禄寺办。二十一日,谕工部:`大婚应用珠冠等如数解进'。二十四日,工科李维樾言:`日来道途鼎沸,不择配而过门;皆云王、田两中贵强取民女,以备宫卫。有方士营杨寡妇少女自刎,母亦投井;亦大不成举动矣'。

十月初八月,韩赞周奏淑女齐集。十二日,赞周请选淑女于杭州。十四日,谕管绍宁:`京城淑女,着博访细选'。又谕内官田成、李国辅分路速选淑女。十七日,谕赞周挨门严访淑女;富室官家隐匿者,邻人连坐。

十一月十二日,限中宫礼冠三万金、常冠一万金,下户部措办。

二月十五月,韩赞周再进淑女六名。二十三日,命礼部广选淑女。一日,士英云`选妃内臣田成有本来报,杭州选淑女程氏'。上见一人,大不乐。已而批旨云:`选婚大典,地方官漫不经心,且以丑恶充数,殊为有罪。责成抚按、道官于嘉兴府加意遴选,务要端淑。如仍前玩忽,一并治罪'。阮大铖曰:`定额三名不可少'。浙江巡抚张秉贞、内官田成得旨,出示嘉兴,合城大惧;昼夜嫁娶,贫富良贱、妍丑老少俱错。合城若狂,行路挤塞。苏州闻之亦然;错配不可胜纪,民间编为笑歌。所选程氏,寄养母家,每日廪给三两,仰仁和、钱塘两县各为护卫;皂快五名,在程门伺候。田成复至嘉兴,从者百人,坐察院恣甚。凡选二十余日,选中两名:一王氏、一李氏,俱小姓女,共程氏淑女三人;乃还南京。

四月初九日,钱谦益奏选到淑女;着于十五日进元辉殿。京选七十人中选阮姓一人、田成浙选五十人中选王姓一人、周书办自献女二人,俱进皇城内。

至五月初十日(辛卯)晨,传旨:`三淑女在经厂者放还母家'。时以大清兵至,是夕将出狩也。

``野史''载:士英语遣选妃内臣往浙,俱云田壮国;而``编年''、``甲乙史''诸书则载田成。

三案、要典、逆案重翻

先是,甲申十二月二十二日(丙子),张捷抄出杨维垣所题,言:`韩爌之再相也,举国皆推之,独臣不肯附和。己巳东变,有一非爌所召者乎!只造一本不公之``逆案'',阮大铖及臣皆不附杨、左而入;乞皇上重复审定!有刘廷元、徐绍言、霍维华、吕纯如、徐大化、贾继春、徐扬先、岳骏声雪之而恤之,周昌晋、徐复阳、虞廷陛、郭如闇、李寓庸、陈以瑞、曹谷雪之而用之;王永光、唐世济、章光岳、许鼎臣、杨兆升、袁宏勋、徐卿伯、水佳胤发愤此案者,亦宜恤之'!

乙酉正月二十日(甲辰),编修吴孔嘉言:```三朝要典''须备列当日奏议,以存其实;删去崔呈秀附和'。命下所司。

二十一日(乙巳),总督袁继咸言``要典''不必重翻;有旨:`皇祖妣、皇考无妄之诬,岂可不雪!事关青史,非存宿憾。群臣当体朕意'。

二十三日(丁未),杨维垣又请重颁``三朝要典'';言张差疯颠,强坐为剌客者,王之采也;李可灼红丸,谓之行鸩者,孙慎行也;李选侍移宫,造以垂帘之谤者,杨涟也。刘鸿训、文震孟只快驱除异己,不顾诬谤君父;此``要典''一事重颁天下,必不容缓也'。

二月初四日,杨维垣请恤三案被罪诸臣。初五日,昭雪珰案编修吴孔嘉。十七日,予逆案徐景濂恤典。二十二日,御史袁洪勋追论梃击、红丸、移宫三案及焚``要典''诸臣罪;因摘吴甡、郑三俊。并言`管绍宁不亟搜``要典''、袁继咸公然忤逆,宜急行究治'。诏勿问。十五日,予逆案徐大化恤典。二十八日(辛巳),刘孔昭言``逆案''尽翻似滥。左良玉言:```要典''治乱所关,勿听邪言,致兴大狱'。有旨:`此朕家事,不必疑揣'!三月初一日,``逆案''杨所修子为父雪罪;允之。

初三日,升杨维垣都察院副都御史;升阮大铖兵部尚书,赐蟒服。十九日,设坛太平门外,百官素服望祭先帝;独阮大铖后至,哭呼先帝而来曰:`致先帝殉社稷者,东林诸臣也。不杀尽东林,不足以谢先帝。今陈名夏、徐汧等俱北走矣'!士英急止之曰:`徐九一现有人在'。大铖日与杨维垣谋,必欲尽杀东林、复社之人。大狱将兴,寻以上游告警始缓。

四月初五日,吏部尚书张捷奏请表章附郑戚诸臣;允之。于是刘廷元、吕纯如、王德完、黄克缵、王永光、杨所修、章光岳、徐大化、范济世各予谥荫、祭葬,徐扬先、刘廷宣、许鼎臣、岳骏声、徐卿伯、姜麟各赠官、予祭葬,王绍徽、徐兆魁、乔应甲、陆澄原各复原官;而唐世济、水佳胤、杨兆升、吴孔嘉、郭如闇、周昌晋、袁宏勋、徐扬、陈以瑞等先后起用。

初七日,御史袁宏勋请究追``三朝要典''诸臣得罪孝宁太后先庄妃者。监生陆浚源又借题三案,疏诋光禄少卿许誉卿。誉卿疏言:`当日诸臣以翊戴光庙为正、今日诸臣以翊戴陛下为正,俱从伦序起见耳。光宗母子无间,先帝身殉社稷,何嫌何疑;而小人无端播弄,假手浚源。先帝久任体仁,养寇酿祸;使得生荣死宠,窃谥``文忠''。陛下追削,万口称快。浚源满口颂其平章之功,何若辈之敢于党奸欺上也'!

史载誉卿疏在甲申八月十七日,而``遗闻''则列于乙酉年。

重提三案,欲伤宫帏骨肉之伦、构清流危亡之祸,此乾坤何等时,而谋杀正人?若非告警,祸正有不可测者。

先是,杨维垣言``要典''为党人所毁。夫小人自为党,而反目君子为党;此从来一网打尽之计。当时被其祸者三十余年,而国亦与之终始矣!

灾异
十月十一日(乙丑),淮督田仰奏凤阳地震。十五日(己已),凤阳祖陵一日三震,有声如吼;太监谷国珍以闻。

二十九日(癸未),长庚星见东方,较昔大异;光芒闪铄,有四角或五角,中有刀剑、旗帜、马影似哄斗象,且倏大倏小、忽长忽缩。

十一月初五日(己丑),太监谷国珍奏凤阳灾。

十一日(乙未),端门西旁舍火。

自秋至冬,烈日如夏,在在赤地。``遗闻''云:`庙门告灾,凤阳祖陵叠火'。

乙酉元旦为乙酉日,天文家云:`太岁值事,不利'。是日,日有蚀之。

中书舍人林翘疏称:`正月初六日雷声自北至西,占在赵、晋之野有兵。日在庚寅,主口角妖言'。翘,江浦人;善星术。马士英在戍日,卜其大用;至是,士英神其术,因荐授中书。寻躐一品武衔,蟒玉趋事。未几,获妖僧大悲,僧系齐庶宗诈冒定王,下法司会审,弃市。

初八日(壬辰)立春,流星入紫薇宫。

初九日,大雷电,雨雹。

张缙彦奏:`十一日(乙未)午刻,河南开封府荣泽县村郭忽现大城,堞门毕具;二时方隐'。天官家云:```广莫之气成城郭'';今河南茫无人烟故也'。

二月二十日(癸酉),钦天监正杨邦庆奏:`近来日月色甚赤'。上云:`是何分野?何无占候?其访术者举用'。

三月初二日(乙酉),杨维垣升左副都御史。时语曰:`马、刘、张、杨,国势速亡'。

七月十三日(乙酉),太白经天。是日,予往四河口候内父,遇秦先生;适姚生至,云甫见日旁一星甚朗。夫金星昼见,变之大者;而诸书不载,何欤?秦之神,无锡华藏人,性至孝;曾于元旦夜梦西城县一牌,大书云:`天下已属之清'。时江南犹无事,与众言之未信;然秦素诚笃,馆于舅氏,予闻而异焉。是春,南京有驴忽作人言云:`造什么桥、修什么路!五月干戈乱,人人路上跑'。既而不语。又是春江南督学朱国昌驻江阴岁试,有奔牛。王生赴试,寓中夜观天象;次日归,不与试。众怪问之,王生曰:`昨晚旌头星已现,大清人不日至矣'!众未之信;未几而南京陷。江阴琉璜乡亦多异鸟,有一鸟身如鹁鸪,口中吐舌长寸许。又一鸟花色可观,头有两角,颇似鹿角;行于地上,见人辄飞。张森之见而问予,予忆古书有`鹝鸟,大如观鹆、头似雉,有时吐物长数寸';有`鵵鸟,有毛角';此非常鸟,天下将乱。鸟能得气之先,此之谓矣(鹝音逆,鵵音屠)。

初,崇祯十三年,一五台僧诣苏州元墓山访道友,语人云:`天马星下界,新天子已降生矣。不久,当有易代事'。时共妄之。不五载,大清果入。

乙酉元旦,微雨、夜风。初二日下午,雨。初三日,雪。初四日,雨。初六日,终日雪。初九夜,大雪。然吾乡元旦阴雨而南京则日蚀,初六日终日雪而南京有雷声;初九日大雪而他处大雷震,雹:阴阳灾异,所在不同如此。

吴适下狱

四月二十一日(癸酉),给事中吴适疏参方国安、牟文绶;疏言:`文绶本无寸功,骤列大帅;乃复纵兵哗掠,致摧陷建德、东流,大属非法。国安受国厚恩,乃铜陵西关及南陵城外聚兵攻击;赤子何辜,遭兹涂炭,益之以深热,其与叛逆何异?陛下宜加禁戢'!蔡奕琛等票旨,切责之云:`左良玉称兵犯顺,连破九江、安庆,文绶实久在南康、国安现在剿逆;吴适讹言乱政,巧为逆臣出脱,是何肺肠'?明日,奕琛具疏特纠,吴适下狱。盖先是,左光先按浙会鞫奕琛一案,适时为衢州司理官,与绍兴司理陈子龙共成是狱。及奕琛入相,乃与阮大铖同心排挤光先以至褫逮,并及于适。实借题以快其夙憾,而国事、封疆俱置不问。御史张孙振又有疏纠参`适为东林嫡派、复社渠魁,宜速正两观之诛'。

东林正人之薮、复社名士之林,以此论罪,荣于华衮矣!

迁都召对

四月二十六日(戊寅),上视朝毕,问群臣迁都计。时礼部钱谦益力言不可,乃退。自左兵檄至、大清兵信急汹汹,上日怨士英强之称帝,因谋所以自全;士英请召黔兵入卫,办走贵阳。工科吴希哲等力陈,乃止。是日,召黔兵一千二百人入城,驻鸡鸣山,践踏僧房殆遍;每夜拨二百名守私宅。二十八日(庚辰),上下寂无一言。良久,上云:`外人皆言朕欲出去'。王铎云:`此语从何得来'?上指一小奄,正色语;铎曰:`外间话,不可传的'!铎因请讲期;上曰:`且过端午'。马士英发黔兵六百赴杨文骢军。是时大清兵渡江甚急,王铎身为大臣,而无一言死守京城,以待缓兵至计;及第请讲期,岂欲赋诗退敌耶?抑欲戎服讲``老子''耶?这都是不知死活人,国家用若辈为辅臣,不亡何待!然铎意已办归大清一着为善后策,故发如此淡话耳。宏光云`且过端午',此语颇冷,使铎多少没趋。君虽庸愦,亦密知大清兵将至矣?

马士英笞驿报

四月二十七日(己卯),龙潭驿探马至,报清兵编木为筏,乘风而下。又一报云:`江中一炮,京口城去四垛'。最后杨文骢令箭至,云`江中有四筏,疑清兵。因架炮于城下,火从后发,震倒颓城大半垛;连发三炮,江筏俱粉碎矣'。士英将前报二人捆打,而重赏杨使。自是,警报寂然。

马士英奔浙

五月十六日黎明,钱谦益肩舆过马士英家,门庭纷然。良久,士英出,小帽、快鞋、上马衣,向钱一拱手云:`诧异、诧异!我有老母,不得随君殉国矣'!即上马去。后随妇女多人,皆上马妆束;家丁百余人。出城至孝陵,诡装其母为太后,召守陵黔兵自卫;黔兵亦半逃。平旦,百姓见宫门不守、宫女乱奔,始知君相俱逃去,惊惶无措;遂乱拥入内宫抢掠,御用物件遗落满街。一时文武逃遁隐窜,各不相顾,洗去门上封示,男女泉涌出城;有出而复返。少顷,忻城伯赵之龙出示安民,有`此土已致大清国大帅'之语,闭各城门以待大兵。黔兵在城者,百姓尽搜杀之;以先受其害也。

附记:士英卫卒三百人从通济门出,门者不放;欲兵之,乃出。私衙元宝三厅,立刻抢尽。有一围屏,玛瑙及诸宝所成,其价无算,乃西洋贡入者;百姓击碎之,各取一小块即值百余金。多藏厚亡,信哉!

黔兵自江上随尹帅还鸡鸣山者,先至二百九十人,随士英出;后至六十人,无归,劫行城中。司城方勇巡警竟夜,乃不敢肆。有潜藏者、有逃出城者,民尽杀之,无一人存。城内栅门盘诘,护马士英中军八人送戎政赵之龙斩之。

马士英寓在西华门;其子马锡寓北门桥都督公署,在鸡鹅巷:百姓焚毁一空。次掠及阮大铖、杨维垣、陈盟家,惟大铖家最富,歌姬甚盛;一时星散。
赵监生立太子

五月十一日午刻,有赵监生率百姓千余人擒王铎到中城狱,群殴之;使认太子。铎呼曰:`非干我事,皆马士英所使'。众笞铎,须发俱尽;太子亟止之,命禁中城狱。百姓拥太子上马入西华门,至武英殿;又拥至西宫,尚未栉沐。时仓卒无备,取戏箱中翊善冠戴首,于武英殿登座,群呼万岁。两日天气阴霾怆惨,月色罕见;是日天晴日朗,众心开悦。各部寺署官见者俱行四拜礼,大僚亦间有至者。太子粘示皇城,略云:`先皇帝丕承大鼎,惟兹臣庶同其甘苦。胡天不佑,惨罹奇祸。凡有血气,裂眦痛耻!泣予小子,分宜殉国;以君父大仇不共戴天、皇祖基业汗血非易,忍垢匿避,图雪国耻。幸文武先生迎立福藩,予惟先帝之哀,奔投南都,实欲哭陈大义。不意巨奸障蔽,至撄桎梏;予虽幽狱,无日不痛绝也。今福王闻兵远遁,先为民望;其如高皇帝之陵寝何!泣予小子,父老人民围抱出狱,拥入皇宫;予自负重冤,岂称尊南面之日乎!谨此布告在京勋旧文武先生士庶人等,念此痛怀,勿惜会议,共抒皇猷!勿以前日有不识予之嫌,惜尔经纶之教也'。左都李沾肩舆微服诣赵之龙家求庇,之龙以令箭护送出城。吏部尚书张捷微行至鸡鸣寺,以佛幡带自缢。左副都御史杨维垣自蹙二妾朱氏、孔氏死;买三棺,旁置二妾,中题``杨某之柩'',并埋中堂;身挈一仆夜遁。至秣陵,为怨家所击杀。数日,仆复迹之,尸为犬食半。

十三日,太子令释王铎,仍为大学士。又召方拱干、高梦箕于狱,并为礼部侍郎、东阁大学士;二人出狱即逃。赵之龙召勇卫营兵入城,城中乘间出者甚众。栅禁稍宽,店肆颇有开张者。文武诸僚集中府会议,齿及太子,皆有难色曰:`前日几番云云,恐有蹈吕、张之祸者;不然,宏光帝复来,将奈何'?赵之龙曰:`此中复立新王,款使北归,其何辞以善后'!众皆然之,哄然而散。各衙门出示安民城守,并不及立新王之事。太子敕封中城狱神为王,差官捧敕。二人行至狱中开读,敕文称``崇祯十八年'';兵马司素服迎之。监生徐瑜、萧某谒赵之龙,劝其早奉太子即位;之龙立叱斩之。差官自北军中回,之龙即入西宫,劝太子避位。冯可宗、陈监、王心一皆弃官逃;高倬、张有誉初传死,后亦逃。李沾既去,李乔自为总宪。

王铎不认太子,罪可斩矣。而太子止其殴、释其狱,仍以为相,其度必有太过人者。惜乎!全躯保妻子之臣之众也。使铎清夜自思,其知愧否?

宋蕙湘题壁

宋蕙湘,金陵人;宏光宫女,年十四岁。为兵掠去,题诗汲县壁云:`风动江空羯鼓催,降旗飘飐凤城开;将军战死君王系,薄命红颜马上来'。`广陌黄尘暗鬓雅,北风吹面落铅华;可怜夜月箜篌引,几度穹庐伴暮笳'!

\hypertarget{header-n47}{%
\subsection{卷七·南都甲乙纪(续)}\label{header-n47}}

高杰论保江南

高杰疏言:`目今大势,守江北以保江南,人人言之。然从曹、单渡,则黄河无险;自颍、归入,则凤、泗可虞:犹或曰有长江天堑在耳。若何而据上游?若何而防海道?岂止瓜、仪、浦、采为江南门户已乎!伏乞和盘打算,定断速行;中兴大业,庶有可观'。杰发总兵李朝云赴泗州,又发参将蒋应雄、许占魁、郭茂荣、李玉赴徐州防守。
时宁南侯左良玉报称:`副将苏荐、游击朱国强斩贼四百余级'。获伪官江一洪,献俘京师。

陈子龙疏募练水师

六月十九日,陈子龙疏言:`一介草茅,四载抱牍,蒙升南京吏部文选司主事,便道还里。因臣祖母高氏老病危笃,而臣以孑身独子,循例乞恩,拜疏终养。风尘阻塞,未达中途,蒙先帝擢署省掖。时寇破恒、代,渐逼京师;臣妄意联络海舟直达,可资应援。因与长乐知县夏允彝、中书舍人宋征璧等捐赀召募。忽闻神京沦陷,先帝升遐;饮血崩心,呼号无地。臣伏思君父之仇,不可不报;中原之地,不可不复。然必保固江、淮,以为中兴之根本。守江之策,莫急水师;海舟之议,更不容缓。幸松江知府陈亨志切同袍、气雄击楫,多方措置,以求成旅。适接兵部尚书史可法、职方司郎中万元吉手书,以``江上守御方殷,望此一军,共为犄角;不妨动支正供,以俟销算。总之,以朝廷之粮养朝廷之兵,无分彼此也''。臣等推职方司主事何刚忠勇性成、清介绝俗,专司募练;而佐以山阴知县钱世贵、举人徐孚远、李素、廪生张密已买沙船二十五只,募材官水卒共一千余名,多堪守战之士。其制造器甲、修船、练药等事,则试中书舍人董庭、都司李时举、生员唐侯等分头经理:一月之内,可以就绪。夫千人在长江,如双凫乘雁,不足为重轻;然使江南诸郡各为门户之计,则万人亦不难致。臣等亦聊尽精卫之心,倡怒蛙之气而已'(出``大事记'')。
张亮奏边防

六月二十九日,安庐巡抚张亮奏``南北止隔一河疏''曰:`贼若从山东来,则淮、徐据黄河之险,我能守之。若从河南来,则我无险可据,必摈河地方防守缜密、盘诘严谨,不容一人、一船私自暗渡。而不知大谬不然者。臣衙门承差程之兖、前抚臣董配元差往北齎奏身陷贼中,四月初九日始得脱出;询之兖从何处渡河,彼云止闻清江浦有防守,彼从宿迁觅船,于白洋河过渡,同行二千人。乡民间有问者,答云``南边逃难人'',辄不为怪也。再询路上有行人否?彼云:`途间遇有车推夏布、扇、茶等物,皆自南而北,赴彼交易'。臣闻之不觉骇然。夫南北止隔衣带水,果能一苇不渡,犹虑取道中州;及今何时也,而去来自若,茫无稽察,致使茶、扇、布箱得饱载而往,于贼巢行垄断之计哉!从来贼用奸细,即以本地之人行之。程之兖系安庆人,又系臣衙门差出,幸而真也;假令人人如此不疑,且如此可渡,即贼之奸细已不知有若干散匿于大江南北矣!滨河者所司何事?而疏玩若此哉!夫迁宿既有伪官,彼已受贼之职,自不禁人之渡;乃河南守土者漫不加意,此何以故?乞饬滨河州县严加盘诘,若真正思汉归南者,有何凭据,务得的确而后许之;若贩卖北送者,仍治以通贼之罪。其于封疆之计非小补也'。
章正宸论时事

七月初二日(丁亥),吏科给事中章正宸上言:`两月以来,闻大吏锡鞶矣,不闻献俘;武臣私斗矣,不闻公战;老成引遯矣,不闻敌忾;诸生卷堂矣,不闻请缨。如此而曰兴朝气象,臣虽愚知其未也。计惟有进取为第一义。进取不锐,则守御不坚。比者河北、山左忠义响应,各结营塞、多杀伪官,为朝廷效死力;不及今电掣星驰,倡义申讨,是劘天下之气而坐失事机也。宜亟檄江北四镇,分渡河、淮,联络诸路齐心协力,互为声援;使两京血脉通而后塞井泾、绝孟津、据武关以攻陇右,恐贼不难旦夕殄也。陛下宜缟素,亲率六师于淮上。但陛下亲征,岂必冒矢石、履行阵哉!声灵所震,人切同仇;虎豹貔貅,勇愤百倍也。今都门部院、寺司各署不称``行在'',而工作仪文;陛下赫然欲为中兴令主,宜严敕诸大臣:速简尔车徒,某旧额、某新增,水几何、陆几何;速备尔刍糗,几何本、几何折,主几费、客几费;选尔将帅,某堪监纛、某堪分阃;审尔形势,某地建镇、某地设堡,某处埋伏、某处出奇;修尔干戈,缮尔城堑:进寸则寸、进尺则尺,阨险处要,大势已得。天下大矣,不患无人;臣未见张、岳、韩、刘之杰不应运而出也'。

洞时事如观火,谈兵械如列眉;而归重亲征,尤为大义。

熊汝霖论封四镇
户科熊汝霖言:`四镇以抢杀封侯,百姓头颈何辜而为此辈功名之地乎!今俨然佐命矣,收拾恢复为中兴名将,岂不更快?况一镇之饷多至六十万,势必不供;何不仿古藩镇法,在大河以北开屯设府,永盟带砺,而逼处堂奥也?万元吉云:``城外之屋应让于兵'';谁非民业,而拱手让乎?近闻辇金求进者,实繁有徒。当事诸臣,亦宜猛省前事、倍涤肺肠也'!

蒋芬请勤王

广西巡抚方震孺、松江知府陈亨、给事中李维樾与兄佥都御史李光泰先后各措饷募兵入卫。而建阳知县蒋芬捐俸资造火器,募勇士朱千斤、刘铁臂等,三请勤王;其词曰:`幸而邀天之幸,迅扫狂氛,指日奏凯,社稷之福。否则,惟有断脰决腹,一瞑而万世不视;以明国家三百年养士之报,亦无负职三十年读书之志'。识者壮之。

``甲乙史'':`七月初二日(丁亥),建宁知县蒋芬自请勤王,具进所造火器;按臣陆清原奏闻'。
王孙蕃论东南形势

御史王孙蕃奏曰:`审天下之势者,贵因乎时;而制一时之宜者,先扼其要。今日定恢复规模,或以区区在东南守备;然必防守固而后可以议攻讨,乃为策之善也。夫大江以南独称安土者,恃此一襟带水耳。于此时,当以屯江为万里长城。兹彭泽、京口已增二镇,可谓识扼险之宜矣。然彭泽有道臣、有督臣,层层弹压、节节关通,上流冲要或无他虑;京口负山枕江,控扼三关、襟带百越,并镇矣而不议设监军道,何以重弹压乎!常镇道鞭长不及,则道臣所宜专设者也。说者谓京口并镇,不如孟河;虽近海口,盐道出没,是一隅之险,而非合筹东南大势也。孟河旧以把总部之,沙唬船二十二只、水陆官兵止九百四十三员名,实存见兵少。水军战舰增设若干,仍于京口并镇为长也。夫金山东连大海、西接神京,去三江会口仅隔一江。昔韩世忠屯兵扼金人于此,``江防考''所载额设官兵三千八百员名、战船百艘。今存见兵六百名、战船十余只,即支持绿林之充斥且不足,何暇巩固皇都而称锁钥重地耶!是亦不可不早计者'。
李向中陈楚省安危

兵部员外李向中言:`臣乡湖广,穷民散乱、军旅空虚,万一逆贼竞武昌,则江南岂得安堵?臣谓荆、襄两处宜速设重镇,募大兵以据其上游,与淮、凤诸处相与犄角;使贼骑不得驰骤汉、黄,庶可保障江南。且襄阳而下、汉黄而上,为承天陵寝重地;按其昭穆迄今仅四世耳,当不忍使祖宗血食为贼出没之区,乞早为整顿。至左镇驻札武昌,自隐有虎据在上之势。而抚臣何腾蛟一腔忠义,千里干城;小民依之若婴儿之求慈母、将士信之若手足之应腹心,亦可谓上下相安而军民各得者矣。近闻有升迁别省之说,乞皇上仍令何腾蛟照旧和衷抚楚。臣思保江南不在逼处江干,而在扼其要领;则臣省荆、襄为急矣。而安臣省者,拒贼犹后而驭兵为先;则知抚臣其不可更矣。伏乞圣明速赐施行'(出``大事记'')。
王应熊请节制楚、郧、贵、广
四川督辅王应熊上言:`蜀境西北接郧,东抵夷陵,西南由建昌通云南,东南由遵义通贵州。今寇踞成都,蜀人殆无孑遗。议者谓李贼在陕,献忠必不北向。然李贼自七月入蜀,虚喝保宁、顺庆之吏民而制之;一且为献忠所驱,则献忠之无顾畏可见矣。川陕总督宜提兵复保宁,牵贼北顾;臣得合滇、黔之力,以捣其空。贼若不南不北,则仍趋正东,未可料也。广西、郧阳许臣节制,则缓急可以呼应;臣名总督四省,而兵止于黔、饷止于滇,不几轻视巨寇乎'?有旨:`命楚、郧、贵、广悉听督辅节制'。

黄耳鼎劾解学龙、张缙彦

御史黄耳鼎言:`解学龙执法大臣,受贿党逆,如光时亨、周钟、方允昌、项煜、陈名夏议缓议赎,岂古人三宥八议之道进于此者!张缙彦俛首贼吏,延喘偷生;皇上重以节钺,优游数月,不恢复寸土。高杰之变,单骑逃避。乞付法司治以误国之罪'!诏勿问。

朱国弼劾路振飞

保国公朱国弼劾旧淮抚路振飞:`贼信日逼,先纵狱囚;天潢洊至,兵拒河上。皇上扁舟不纳入城,且云凤阳有天子气;伪官武愫系进学门生,代为夤缘。乞敕法司逮治'!章下部院。

左良玉参马士英八罪

四月初四日(丙辰),宁南侯左良玉举兵东下,驰疏至云:`窃见逆贼马士英出自苗种,性本凶顽。臣身在行间,无日不闻其罪状、无人不闻其奸邪。先帝皇太子至京,道路汹传陛下屡发矜慈,士英以真为假,必欲置之死而后快其谋。臣前两疏,望陛下从容审处;冀士英夜气稍存,亦当剔肠悔过以存先帝一脉。不意奸谋日甚一日,臣自此义不与奸贼共天日矣!臣已提师在途,将士眦目指发,皆欲食其肉。臣恐百万之众发而难收,震惊宫阙;且声其罪状,正告陛下。仰祈刚断,与天下共弃之!自先帝之变,人人号泣;士英利灾擅权,事事与先帝为仇。``钦案'',先帝手定者,士英首翻之;``要典'',先帝手焚者,士英复修之。思宗改谥``毅宗'',明示先帝不足``思'',以绝天下报仇雪耻之心:罪不容于死者一也。国家提衡文武,全恃名器鼓舞人心。自贼臣窃柄以来,卖官鬻爵,殆无虚刻;都门有``职方贱如狗、都督满街走''之谣。如越其杰以贪罪遣戍,不一年而立升部堂;□□□以贪污绞犯,不数日而夤缘仆少;袁□□与张道浚皆诏狱论罪者也,借起废径复原官;如杨文骢、刘泌、王燧以及赵书办等皆行同犬彘或罪等叛逆,皆用之于当路。凡此之类,直以千计,罄竹难书:罪不容于死者二也。阁臣司票拟、政事归六部,至于兵柄尤不容兼握。士英已为首辅,犹复掌枢;是弁髦太祖法度。又引腹心阮大铖为添设尚书,以济其篡弑之阶;两子枭獍,各操重兵以为呼应,司马昭复生于今日:罪不容于死者三也。陛下选立中宫,典礼攸关。士英居为奇货,先择其尤者以充下陈,罪通于天;而又私买歌女寄养阮大铖家,希图进选,计乱中宫,阴谋叵测:罪不容于死者四也。陛下即位之初,恭俭神明;士英百计诳惑,进优童艳女,损伤盛德。每对人言,恶则归君:罪不容于死者五也。国家遭此大难,须宽仁慈爱以收人心。士英自引阮大铖以来,睚眦杀人,如雷演祚、周镳等锻炼周纳,株连蔓引。尤其甚者,借题三案,深埋陷阱;将生平不快意之人一网打尽,令天下士民重足解体:罪不容于死者六也。九重秘密,岂臣子所敢言;士英遍布私人,凡陛下一言一动,无不窥视。又募死士窜伏皇城,诡名禁军,以观陛下动静,曰``废立由我'':罪不容于死者七也。率土碎心痛号者,先帝殉难,皇子幸存。前此定王之事,四海至今传疑未已;况今皇太子授受不明,士英乃与阮大铖一手拏定,抹杀的确认识之方拱干、而信串通朋谋之刘正宗,不畏天道神明、不畏二祖列宗、不畏天下公议、不畏万古纲常,忍以先帝已立七年之嗣君为四海讴歌讼狱所归者,付之幽囚;天昏地惨,神人共愤。凡有血气者,皆欲寸磔士英、大铖等以谢先帝。此非臣之私言,诸将士之言也;非独臣标将之言,天下忠臣义士、愚夫愚妇之公言也。伏乞陛下立将士英等肆诸市朝,传首四方,用抒公愤。臣等束兵计刻以待,不禁大声疾呼,激切以闻'。
数列八罪,使人摭辨不得、躲闪不得;足褫奸雄之魄矣!

又讨马士英檄

盖闻大义之垂,炳于星日;无礼之逐,严于鹰鹯:天地有至公,臣民不可罔也。奸臣马士英根原赤身,种类蓝面。昔冒九死之罪,业已侨妾作奴、屠发为僧;重荷三代之恩,徒尔孤窟白门、狼吞泗上。会当国家多难之日,侈言拥戴、劝进之功;以今上历数之归,为私家携赠之物。窃弄威福,炀蔽聪明。持兵力以胁人,致天子闭目拱手;张伪旨以詟俗,俾臣民重足寒心!本为报仇而立君,乃事事与先帝为仇,不止矫诬圣德;初因民愿而择主,乃事事拂兆民之愿,何由奠丽民生?幻蜃蔽天,妖蟆障日。卖官必先姻娅,试看七十老囚、三木败类,居然节钺监军;渔色罔识君亲,托言六宫备选、二八红颜,变为桑间濮上。苏、松、常、镇,横征之使肆行;檇李、会稽,妙选之音日下。江南无夜安之枕,言马家便尔杀人;北斗有朝彗之星,谓英君实应图谶。除诰命、赠荫之余无朝政,自私怨旧仇而外无功能。类此之为,何其亟也!而乃冰山发焰,鳄水兴波;群小充斥于朝端,贤良窜逐于崖谷。同己者性侔豺虎、行列猪猳如阮大铖、张孙振、李宏勋等数十巨憝,皆引之为羽翼,以张杀人媚人之赤帜;异己者德并苏黄、才媲房杜如刘宗周、姜曰广、高宏图等数十大贤,皆诬之为朋党,以快如虺如蛇之狠心。道路有口,空怜``职方如狗、都督满街''之谣;神明难欺,最痛``立君由我''、``杀人何妨''之句!呜呼!江汉长流、潇湘尽竹,罄此之罪,岂有极欤!若鲍鱼蓄而日膻,若火木重而愈烈。放崔、魏之瘈狗,遽敢灭伦;收闯、献之狝猴,教以升木。用腹心出镇,太尉朱泚之故智,几几殆有甚焉;募死士入宫,宇文化及之所为,人人而知之矣。是诚河山为之削色,日月倏兮无光。又况皇嗣幽囚,列祖怨恫。海内怀忠之士,谁不愿食其肉;敌国向风之士,咸思操盾其家。
本藩先帝旧臣、招讨重任。频年痛心疾首,愿为鼎边鸡犬以无从;此日履地戴天,誓与君侧豺狼而并命。在昔陶八州靖石头之难,大义于今炳然;迄乎韩蕲王除苗氏之奸,臣职如斯乃尽。是用砺兵秣马,讨罪兴师。当郑畋讨贼之军,意裴度蔽邪之语。谓朝中奸党尽去,则诸贼不讨自平;倘左右凶恶未除,则河北虽平无用。三军之士,戮力同仇。申明仁义之声闻,首严焚戮之隐祸:不敢妄杀一人以伤天心,不敢荒忽一日以忘王室。义旗所指,正明为人臣子不忘君父之心;天意中兴,必有间世英灵天翼皇明之运!

泣告先帝,揭此心肝:愿斩贼臣之首,以复九京;还收阮奴之党,以报四望。倘惑于邪说、诖误流言,或受奸臣之指挥、或树义兵之仇敌,本藩一腔热血,郁为轮囷离奇;势必百万雄兵,化作蛟螭妖孽。玉石俱焚之祸近在目前,水火无情之时追维痛心。敬告苦衷,愿言共事。呜呼!朝无直臣,谁斥李林甫之奸邪!国有同心,尚怀郑虎臣之素志。我祖宗三百年养士之德,岂其决裂于佥壬?大明朝十五国忠义之心,正宜暴白于魂魄。速张殪虎之机,勿作逋猿之薮!燃董卓之腹,膏溢三旬;籍元载之厨,椒盈八百:国人尽快,中外甘心!\\
谨檄。

又檄

左良玉反兵东下,请除君侧之恶;移檄远近,以讨马士英。其略云:`马士英蛮獠无知,贪狠悖义。挟异人为奇货,私嫪毒以种奸;欺虾蟆之不闻,恣鹿马以任意。不难屠灭皇宗,遂敢刑戮太子。效胡溁之名访邋遢,既不使之遯于荒野;踵钱宁之即讯大千,又不容其毙于深宫。群小罗织,比燕啄而已深;中犴幽囚,视雀探而更惨。李沾威拷,何如崔季舒拳殴;王铎喝招,有甚朱友恭塞谤!岂先帝不足复留种,既沈其弟,又灭其兄;将小朝自有一番人,既削其臣,并翦其主。嗟乎!安金藏之不作,丙定侯之已亡!附会成群,谁敢曰``吾君之子''?依违欲了,咸称曰``的系他人''。临江之当乱虎,是可忍也!孑舆之遇蟒毒,尚何言哉!良玉受恩故主,爵忝通侯;宁无食蕊之思,讵忘结草之报。愿共义士,共讨天仇!严虎豹之亟驱,风云气愤;矢鹰鹯之必逐,日月光明。郿坞丰盈,应有然脐之祸;渐台高拥,难逃切舌之灾'!檄下远近传播,惟京中噤口。

前檄出``遗闻'',在初四日(丙辰)下;此檄出``甲乙史''与``编年'',载初三日(乙卯)也。

左兵东下
四月初五日(丁巳),左兵入九江、安庆至于建德,顺流东下。初七日(己未),左兵入东流。良玉沿途遍张告示,称`本藩奉太子密旨,率师赴救'。士英等大惧,京师戒严;士英专理部事,不入直。江督袁继咸请赦太子以遏止之;宏光切责。士英调黄得功、刘良佐离汛,邀刘孔昭、阮大铖、方国安、朱大典御左兵。升大典兵部尚书,国安挂镇南将军印。十四日(丙寅),黄得功兵至江上;着于荻港三山暂驻,有警进前。十五日(丁卯),马士英言:水陆诸军必直抵湖口,与九江、安庆呼吸相通。乃知上游消息,即催阮大铖、朱大典督诸军前发,不得稽延。十七日(己巳),马士英奏上大捷,赏刘孔昭、朱大典、黄得功、阮大铖、黄斌卿、黄蜚、郑彩、方国安、赵民怀、郑鸿逵、卜从善、杜宏域、张鹏翼、杨振宗银币。五月初一日,张捷率百官进贺捷表。时维扬信绝,左兵停留不下;阮大铖、刘孔昭虚报捷音,以愚都人耳目。初五日(丙戌),黄得功与左兵屡战,身中二矢。捷闻,诏封靖国公;遣太监王肇基劳之。并进阮大铖、朱大典并太子太保,总兵张杰、马得功、郑彩、黄蜚并加三级,副将而下各进一级,仍予锦衣世袭。

``遗闻''云:`良玉举兵不数日,即病死。子梦庚东下至采石,为黄得功、方国安所败。寻闻大清兵紧急,遂引还'。

高杰遗大清肃王书

逆闯犯阙,危及君父,痛愤于心。大仇未复,山州俱蒙羞色,岂独臣子义不共天!关东大兵能复我神州、葬我先帝、雪我深怨、救我黎民,前有朝使谨齎金币,稍抒微忱;独念区区一介,未足答高厚万一!兹逆成跳梁西晋,未及授首。凡系臣子及一时豪杰忠义之士,无不西望泣血,欲食其肉而寝其皮;昼夜卧薪尝胆,惟以杀闯逆、报国仇为亟。贵国原有莫大之恩,铭佩不暇;岂敢苟萌异念,自干负义之愆!
杰猥以菲劣,奉旨堵河。不揣绵力,急欲会合劲旅,分道入秦,歼逆成之首,哭奠先帝;则杰之忠血已尽、能事已毕,便当披发入山,不与世间事,一意额祝复我大仇者。兹咫尺光耀,可胜忻仰;一腔积怀,无由而质。若杰本念千言万语,总欲会师剿闯,以成贵国恤邻之名。且逆成凶悖,贵国所恶也;本朝抵死欲报大仇,贵国念其忠义,所必许也。本朝列圣相承,原无失德;正朔承统,天意有在。三百年豢养士民,沦肌浃髓,忠君报国,未尽泯灭;亦祈贵国之垂鉴也。

肃王报书

肃王致书高大将军:钦差官远来,知有投诚之意,正首建功之日也。果能弃暗投明,择主而事,决意躬来过河面会,将军功名不在寻常中矣。若第欲合兵剿闯,其事不与予言;或差官北来,予令人引奏我皇上,予不自主。此复。

先是,大清副将唐起龙其父唐虞时致书杰,劝其早断速行;有`大者王、小者侯,不失如带如砺、世世茅土'语。杰皆不听;身先士卒,沿河筑墙,专力备御。
许定国杀高杰

许定国,河南归德府睢州人;膂力千斤。初,高杰为李自成将时,尝劫定国村,杀其全家老幼,惟定国逃免。至是同为列将,定国衔之;秘而不言,阳与杰好。时杰冒雪防河,疏请重兵驻归德,东西兼顾,联络河南总兵许定国以奠中原。定国在睢,闻杰将至,遣人致书云:`睢州城池完固、器械精良,愿让公驻兵'。杰信而不疑。十二月二十七日,杰在归德,贻定国千金、币百匹。正月初九日,定国约杰会于睢州。初十日,杰抵睢州,定国来见;杰即回谒,各叙思慕意。十二日,定国招杰饮;杰即与张缙彦、监军李升偕部将八人及亲兵数十人坦然赴之。定国设专席于内以宴杰,布列席于外以宴诸将从兵,酒醪悉盛;酣饮竟日,继之以烛。杰醉,定国伏兵于内,饰美妓荐寝。先窃去杰之甲兵;夜半,帐外伏兵四起,大声连呼``高杰''。杰梦寐间闻之,大惊曰:`谁敢呼我名'!急起觅枪甲,已不得。定国持枪直入,剌杰;杰虽短小而勇悍绝人,连折二枪;定国持短刀杀之,剖其腹以祭先灵。张缙彦、李升走免;时八将犹饮于外,闻内变大骇,推倒筵案,逾垣狂走。亲兵被杀者三十余人,余趋出城去。定国持杰首,招抚士卒;士卒以失主将,遇中州人即杀,谓其合谋也。城中如沸,竟夜走空。定国遂以渡河降大清,封为平南侯。既而引大兵入仪封。

闻定国杀高杰有授旨者。

二月初五日,史可法请优恤高杰。十二月,杰妻邢氏率子元爵请恤,可法请以杰部将李本深为提督。有旨:`兴平有子,朕岂以兵马信地遽授他人!加监军卫胤文兵部侍郎,总督杰军。所部将士,仍听邢氏统辖'。既而,再请加李本深太子太保、左都督,提督本镇,赴归德;中权总兵杨承祖赴夏邑,副将唐应虎赴虞城,苗顺甫赴砀山,后劲总兵李翔云赴双沟、右协总兵胡茂贞、左翼总兵郭虎赴泗州驻防。十四日,黄得功尝与杰争扬州而哄,至是闻杰被害,欲向扬州泄忿。史可法驰归镇抚之。请旨,上谕曰:`大臣当先国事而后私憾,得功若向扬州,致高营兵弃河东顾,设清兵乘隙渡河,罪将谁任?诸藩当恪守臣节,不得任意'!又谕史可法:`卿既归扬,解谕黄得功回汛,何必与孤儿寡妇争构?河上防御,责成王永吉、卫胤文料理'。十五日,刘良佐见杰死,欲并其众;疏称溃兵不宜授本深提督。刘泽清、黄得功、刘良佐又合奏`高杰从无寸功,骄横淫杀,上天默除大恶;史可法乃欲其子承袭,又欲李本深为提督,是何肺肠?倘误听加恩太重,臣等实不能相安矣'。宁南侯左良玉有``忠胤将同压卵''之疏,九江总督袁继咸亦有``兴平有可念之劳''疏;诏赠太子太保,许其子袭爵,再荫一子锦衣卫百户,从优祭葬。

阅``不能相安''一语,刘、黄辈挟制朝廷,目中无上久矣。

张缙彦荐卜从善

二月,张缙彦奏:`狄、白二贼流蔓固、汝间,臣委李鼎招安;镇臣王之纲以争地之故,激陷王帅,乃闭门自守,纵兵杀劫。臣以为之纲宜坐镇内地,安享温饱。芜湖卜从善恩威久着河北,有``飞将''之号;调使恢复,则督、抚有臂指之使'。

史可法请饷

九月,史可法言:`臣皇皇渡江,岂真调和四镇哉!朝廷之设四镇,岂真江北数郡哉!四镇岂以江北数州为子孙业哉?高杰言进取开、归,直捣关、雒,其志甚锐;臣于六月请粮,今九月矣,岂有不食之卒可以杀贼乎'?

史可法请恢复
十一月十七日(辛丑),钦命督帅史可法为时事万难分支,中兴一无胜着;密请恢复远略,激励同仇,以收人心、以安天位事:`痛自三月来,陵庙荒芜、山河鼎沸;大仇在目,一矢未加。臣备员督师,死不塞责。晋之末也,其君臣日图中原,而仅保江左;宋之季也,其君臣尽力楚、蜀,而仅固临安。盖偏安者,恢复之退步;未有志在偏安,而遽然自立者也。大变之初,黔黎洒泣、绅士悲歌,痛愤相乘,犹有朝气;今兵骄饷诎,文恬武嬉,顿成暮气矣。屡得北来塘报,皆言清必南窥;水则广调唬船,陆则分布精锐。黄河以北,悉为清有;而我河上之防,百未料理。人心不一,威令不行;复仇之师不及于关陕,讨贼之诏不达于北廷。一似君父之仇,置之膜外者。近见清示,公然以``僭逆''二字加于南;是和议断难成也。一旦寇为清并,必以全力南侵;即使寇势鸱张足以相扼,必转与清合,先犯东南。宗社安危,决于此日。我即卑宫菲食、尝胆卧薪,枕戈待旦、破釜沈舟,尚恐无救于事;以臣观庙堂之作用与百执事之经营,殊有未尽然者。夫将之所以能克敌者,气也;君之所以能驭将者,志也。庙堂之志不奋,则行间之气不鼓。夏之少康,不忘逃出自窦之志;汉之光武,不忘芜薮爇薪之时。臣愿皇上之为光武、少康,不愿左右亵御之臣以唐肃、宋高之说进也!忆前北变初传,人心骇震;臣等恭迎圣驾临莅南都,亿万之人欢声动地。皇上初见臣等,言及先帝,则泪下沾襟;次谒孝陵,赞见高皇帝、高皇后,则泪痕满襟:皇天后土,实式监临。曾几何时,顿忘前事!先帝以圣明罹惨祸,此千古以来所未有之变也。先帝待臣以礼、驭将以恩;且变出非常,在北诸臣死节者寥寥、在南诸臣讨贼者寥寥:此千古以来所未有之耻也。庶民之家,父兄被杀,尚思穴胸断脰得而甘心;况在朝廷,顾可膜置!以臣仰窥圣德、俯察人情,似有初而鲜终、改德而见怨。以清之强若彼而我之弱如此、以清之能行仁政若彼而我之渐失人心如此,臣恐恢复之无期而偏安未可保也!今宜速发讨贼之诏,严责臣与四镇悉简精锐,直指秦关;悬上赏以待有功,假便宜以责成效。丝纶之布,痛切淋漓;庶使海内忠臣义士闻而感激也。国家遭此大变,皇上嗣承大统,原与前代不同;诸臣但〔有〕罪之当诛,实无功之足录。臣于``登极诏''稿,将``加恩''一款特为删除;不意颁发之时,仍复开载!闻清见此示,颇笑之。今恩外加恩,纷纷未已;武臣腰玉,直等寻常:名器滥觞,于斯为极。以后似宜慎重,专待真正战功;庶使行间猛将、劲兵,有所激厉也。至兵行讨贼,最苦无粮;搜括既不可行,劝输亦觉难强。似宜将内库一切尽行催解,凑济军需。其余不急之工役、可已之繁费,一切报罢;朝夕之宴衎、左右之贡献,一切谢绝。即事关典礼,万不容废,亦宜概从俭约。盖盗贼一日不灭,海宇一日不宁。即有宫室,岂能宴处;即有玉食,岂能安享!此时一举一动,皆人心向背所关、邻国窥伺所在也。必皇上念念思祖宗之鸿业、刻刻愤先帝之深仇,振举朝之精神、萃四方之物力,以并于选将练兵、灭寇复仇之一事,庶乎人心犹可鼓、天意犹可回耳!臣待罪戎行,不宜复预朝政;然安内实御外之本,故敢痛切直陈'。

圣旨:`览卿奏疏,具征忠悃。朕于皇考、先帝深仇,朝夕未尝去念;誓师北讨、光复旧业,岂非至愿!但外解不至、百用匮诎。时复亢旱,催科实难;捉衿露肘,徒烦仰屋。西宫、大婚等费,日从省约;内库物料,正在议折。其余的,朕知道了。卿凡有忠谠,不妨密切敷陈。讨贼诏书,候即颁行。该衙门知道'。

``甲乙史''载此疏为十二日奏,而``遗闻''则云`疏入,不省'。予读此疏,酷似贾生痛哭、武侯尽瘁之书。阅之而不发愤为雄者,真下愚之不移也;可为三叹!

大清摄政王致史可法书

甲申九月,清摄政王遣副将唐起龙致史可法书:

清摄政王致书于史老先生文几:予向在沈阳,即知燕山物望,咸推司马;及入关破贼,得与都人士相接见,识介弟于清班,曾托其手勒平安,权致衷绪,未审何时得达?

比闻道路纷纷,多谓金陵有自立者。夫君父之仇,不共戴天。``春秋''之义,有贼不讨,则故君不得书葬、新君不得书即位:所以防乱臣贼子,法至严也。闯贼李自成称兵犯阙,手毒君亲;中国臣民,不闻加遗一矢!平西王吴三桂介在东陲,独效包胥之哭;朝廷感其忠义,念累世之夙好、弃近日之小嫌,爰整貔貅,驱除狗鼠。入京之日,首崇怀宗皇帝后谥号;卜葬山林,悉如典礼。亲王、将军以下,一仍故封,不加改削;勋戚、文武诸臣,咸在朝列,恩礼有加。耕市不惊,秋毫无犯。方拟秋高气爽,遣将西征。传檄江南,联兵河朔;陈师鞠旅,戮力同心:报尔君父之仇,彰我朝廷之德。岂意南州诸君子苟安旦夕,不审事几;聊慕虚名,顿忘实害:予甚惑之!夫国家之抚定燕都,乃得之于闯贼,非取之于明朝也。贼毁明朝之庙主;辱及先人;我国家不惮征缮之劳,悉索敝赋,代为雪耻。仁人君子,当何如感恩图报!兹乃乘逆寇稽诛、王师暂息,即欲雄据江南,坐享渔人之利;揆之情理,岂可谓平?将以为天堑不能飞渡、投鞭不足断流耶?夫闯贼但为明朝祟耳,未尝得罪于我国家也;徒以薄海同仇,特申大义。今若拥号称尊,便是``天有二日'',俨为劲敌。予将简西行之锐,转旆东征;且拟释彼重诛,命为前导。夫以中华全力,受困潢池;而欲以江左一隅兼支大国,胜负之数无待蓍龟矣!
予闻君子之爱人也以德,细人则以姑息。诸君子果识时知命,笃念故主、厚爱贤王,宜劝令削号称藩,永绥福位;朝廷当待以虞宾,统承礼物;带砺河山,位在诸侯王上:庶不负朝廷伸义讨贼、兴灭继绝之初心。至于南州群彦翩然来仪,尔公尔侯、列爵分土,有平西之典例在;惟执事实图维之!晚近士大夫好高树名义而不顾国家之急,每有大事,辄同筑舍;昔宋人议论未定而兵已渡河,可为殷鉴!先生领袖名流,主持至计;必能深维终始,宁忍随俗沉浮!取舍从违,应早审定!兵行在即,可东可西;南国安危,在此一举。愿诸君子同以讨贼为心,毋贪瞬息之荣,致令故国有无穷之祸,为乱臣贼子所笑!予实有厚望焉。

``记''有之:`惟善人能受尽言'。敢布腹心,伫闻明教;江天在望,延跂为劳。书不尽意。
史可法答书

南中向接好音,随遣使讯吴大将军,未敢遽通左右;非委隆谊于草莽也,诚以``大夫无私交'',``春秋''之义。今倥偬之际,忽捧琬瑊之章,不啻从天而降也。讽读再三,慇慇致意;若以逆贼尚稽天讨为贵国忧,法且感且愧。惧左右不察,谓南中臣民偷安江左,顿忘君父之仇;敬为殿下一详陈之。

我大行皇帝敬天法祖、勤政爱民,真尧、舜之主也;以庸臣误国,致有三月十九之变。法待罪南枢,救援无及。师次江上,凶问遂来;地坼天崩,川枯海竭。嗟乎!人孰无君?虽肆法于市朝以为泄泄之戒,亦奚足慰先帝于地下哉!尔时南中臣民哀痛如丧考妣,无不抚膺切齿,欲悉东南之甲,立翦凶仇;而二、三老成,谓``国破君亡,宗社为重'',相与迎立今上,以系中外人心。今上非他,神宗之孙、光宗犹子而大行皇帝之兄也;名正言顺,天与人归。五月朔日,驾临南都,万姓夹道欢呼,声闻数里;群臣劝进,今上悲不自胜,让再、让三,仅允监国。迨臣民伏阙屡请,始于十五日正位南都。从前凤集河清,瑞应非一。即告庙之日,紫气如盖,祝文升霄,万目共瞻,欣传盛事;大江涌出楠梓数万,助修宫殿:是岂非天意哉!越数日,即命法视师江北,刻日西征。忽传我大将军吴三桂假兵贵国,破走逆成;殿下入都,为我先皇帝后发丧成礼,扫清宫阙,抚戢群黎,且罢薙发之令,示不忘本朝:此等举动,振古铄今;凡为大明臣子,无不是跽北向、顶礼加额,岂但如明谕所云``感恩图报''已乎!谨于八月薄治筐篚,遣使犒师;兼欲请命鸿裁,连兵西讨。是以王师既发,复次江、淮。乃辱明诲,引``春秋''大义来相诘责;试推言之,此又为``列国君薨,世子应立;有贼未讨,不忍死其君者''之说耳。若夫天下共主身殉社稷、青宫皇子惨变非常,而犹拘牵``不即位''之文、坐昧``大一统''之义,中原鼎沸,仓卒出师,将何以维系人心、号召忠义?紫阳``纲目'',踵事``春秋''。其间特书,如莽移汉鼎,光武中兴;丕废山阳,昭烈践祚;怀、愍亡国,晋元嗣基;徽、钦蒙尘,宋高缵统:是皆于国雠未翦之日亟正位号,``纲目''未尝斥为自立,卒以正统予之。甚至如玄宗幸蜀,太子即位灵武;议者疵之,亦未尝不许以行权,幸其光复旧物也。本朝传世十六,正统相承,自治冠带之族;继绝存亡,仁风遐被贵国昔在先朝,夙膺封号,载在盟府。后以小人构衅,致启兵端;先帝深痛疾之,旋加诛僇:此殿下之所知也。今痛心本朝之难,驱除乱逆,可谓大义复着于``春秋''矣。若乘我国运中微,一旦视同割据;转欲移师东下,而以前导命元凶,义利兼收、恩仇倏忽,奖乱贼而长寇仇:此不惟孤本朝借力复仇之心,亦甚违殿下仗义扶危之初志矣!昔契丹和宋,止岁输以金缯;回纥助唐,原不利其土地。况贵国笃念世好,兵以义动;万代瞻仰,在此一举。若夫手足齐难,并同秦、越;规此幅员,为德不卒:是以义始而以利终,贻贼人窃笑也,贵国岂其然欤?

往者先帝轸念潢池,不忍尽戮;剿抚互用,贻误至今。今上天纵英武,刻刻以复仇为念;庙堂之上,和衷体国。介胄之士,饮泣枕戈;忠义民兵,愿为国死。窃以天亡逆闯,当不越于斯时矣。语有云:`树德务滋,除恶务尽'。今逆贼未伏天诛,谍知卷土西秦,方图报复;此不独本朝不共戴天之恨,抑亦贵国``除恶未尽''之忧。伏乞坚同仇之谊、全始终之德,合师进讨,问罪秦中;共枭逆成之头,以泄敷天之恨:则贵国义问照耀千秋,本朝图报惟力是视。从此两国世通盟好,传之无穷;不亦休乎!至于牛耳之盟,本朝使臣久已在道;不日抵燕,奉盘盂从事矣。

法北望陵庙,无涕可陨;身陷大戮,罪应万死。所以不即从先帝于地下者,实为社稷之故。``传''曰:`竭股肱之力,继之以忠贞'。法处今日,鞠躬致命,克尽臣节而已。即日奖帅三军长驱渡河,以穷狐鼠之窟,光复神州,以报今上及大行皇帝之恩。贵国即有他命,弗敢与闻;惟殿下实昭鉴之!

宏光甲申九月十五日。

何亮工,南直桐城人;宰相何如宠之孙也。亮工少有逸才,时为史道邻幕宾;此书乃其手笔。顺治丁酉,亮工举孝廉;家于南京武定桥。

史可法奏李际遇降大清
正月初九日(癸巳),史可法上书:`陈潜夫所报清豫王自孟县渡河,约五、六千骑;步卒尚在单、怀,欲往潼关:皆李际遇接引长驱而来,刻日可至。据此,李际遇降附确然矣。况攻邳之日未返济宁,岂一刻忘江北哉!请命高杰提兵二万,与张缙彦直抵开、雒,据虎牢;刘良佐贴防邳、宿'。又言:`御史陈荩往调黔兵五千人,半载杳然;乞催之早到'。上从之,命给闽铳三十枚。又奏:`清兵渡洛阳,河南抚按俱避于颍、寿'。

史可法奏和议不成

十二日(丙申),史可法言:`北使之旋,和议已无成矣。向以全力御寇而不足,今复分以御北矣。唐、宋门户之祸与国始终,以意气相激,化成恩仇。有心之士,方以为危身之场;而无识之人,转以为快意之计。孰有甚于戕我君父、覆我邦家者?不此之仇而修睚眦之微,真不知类矣:此臣所望于庙堂也。先帝之待诸镇何等厚恩、皇上之封诸镇何等隆遇;诸镇之不能救难,何等罪过!释此不问而日寻干戈,于心忍乎?和不成,惟有战;战非诸将之事而谁事也?阃外视庙堂,庙堂视皇上;尤望深思痛愤,无然泄沓!古人言:``不本人情,何由恢复'';今之人情,大可见矣'。至十四日,可法七请接济。时幕客驰金四出,以召集为名,不问所至;而可法躬自俭苦,军需尝乏,人皆惜之。

``和议不成''一疏,``编年''载于甲申十二月下旬。

史可法求退

正月二十四日(戊申),史可法上疏求退,言:`卫胤文揭为一事权,谓臣赘疣应去,欲召臣使还。臣讨贼未效,妄冀还朝;臣虽至愚,计不出此。遭君父之变、膺简命之隆,千难万苦,臣何自安'!上慰勉之。又言:```春秋''即位初年,必称元年;明人君之用也。敬天法祖、任贤使能、节用爱人、勤政讲学,惟皇上力行无斁,将由元年以至亿万年矣'。\\
史可法论军资
二月十四日(丁卯),史可法言:`当日建置四藩,恢复难期,而军资最急;在淮、扬则有税可榷,而庐、凤则否:此得功、良佐所以有偏枯之嗟也。臣每岁饷银有本、折六十万数,内五万养徐州兵、一万五千养泗州兵,官兵间有犒赏。议将淮、扬两关岁征,臣与得功、良佐三股均分。此时北道不通,每季不过五千;若能守住江北,则税归朝廷。否则,地且难存,何从榷税'。

史可法奏泗州将

三月甲申朔,史可法上言:`泗州镇将李世春廉而有威,一病遽亡。其弟遇春队伍精严,地方相安;奉旨用代矣。黄得功坚逐浦口将张天福,部议改张天福于泗州;高营各将以泗州为其分地,天福若来,恐难相安。比伊兄天禄迁家属至,总兵卜从善扼之于泊所;夺其马骡,家眷惊落水中。乞敕部仍用遇春,其天福另用'。上如其言。

史可法北征疏

四月癸丑朔,史可法``北征疏''云:`臣受命督师,无日不以国事为念。而人情难协,事局纷更;睢州大变之后,又有维扬之构。外侮未御,内衅方深;拥节制之虚名、负封疆之大罪,窃自悲也!先是,提督之命未下,高营将士汹汹;臣不得不容之以镇静。本月二十三日,臣议调兵北向,李本深身患痈未起;今臣不得已,先将镇臣胡茂贞进发矣'。

\hypertarget{header-n52}{%
\subsection{卷八·南都甲乙纪(续)}\label{header-n52}}

北事

六月初三日(己未),都督陈洪范请任北使;命来京陛见。史可法乞选臣齎监国、即位二诏及使吴三桂、谢升二敕,抵山东、北直晓谕。时讹传``谢陛''伪``谢升''也。

十五日(辛未),马士英以大清国摄政王所谕南朝官民示奏闻,请遣官诏北行;士英疏曰:`据东镇太子太师东平伯刘泽清揭前事内称:``六月初六日,据北来难民严太、沈绍祖、潘章、张敬山等云:北兵五月初一日追贼至京,出示云:大清国摄政王令旨,谕南朝官绅军民人等知悉:曩者,我国欲尔大明和好,永享太平,屡致书不答;以致四次深入,期尔悔悟耳。岂意坚执不从;今被流贼所灭,事属既往,不必论也。且天下者非一人之天下,有德者居之;军民非一人之军民,有德者主之。我今居此为尔朝雪君父之仇,破釜沉舟,一贼不灭,誓不返辙。所过州县地方,能削发投顺、开城纳款,即予爵禄,世守富贵;如有抗拒不遵,大兵一到,玉石不分,尽行屠戮。有志之士,正干功立业之秋;如有失信,为何以服天下乎!特谕''。看此示,是不知中国已有主矣。理合速差文武二臣颁诏北行,以安中外臣民之心。从此南北,又换一局。臣已遣陈洪范向议和款主事马绍愉往督辅史可法处相机商酌'。十六日(壬申),马士英举陈洪范北行。十九日(乙亥),佥都左懋第以母死于北京,愿同陈洪范北使。

二十六日(壬午),进旧辅谢升上柱国少师、卢世漼工部侍郎、黎玉田兵部尚书、王应华光禄卿,俱充山陵使,祭告先帝后祔葬。二十七日(癸未),大清兵入德州,卢世迎降;济王走死,马元騄奔南京,谢升亦出仕于大清。

二十九日(乙酉),北归诸臣南下;舟次上闸,监军凌駉在舟。时李建泰已作大清辅,駉有抚东之命;与署道于连跃出示,称顺治元年。然駉于南京,亦发疏不绝。

是日,传报济宁固山额真石等奉摄政王令,调兵马巡视山东;所到地界,官民出郭迎接,违者以抗师治罪。大清国平西王吴三桂称摄政王简选虎贲数十万络绎南下,牌仰山东德清一带仰体大清安民德意。七月丙戌朔,有北兵数人持告示至青州,一为摄政王、一为平西王吴,各称安民。又有大清兵部文二角,索一路清册;惟济宁未降,东昌、临清皆服。又临清中军张显荣称:摄政王命固山额真石六家总兵驻德州、侍郎王鳌永招抚山东。

八月初二日(丁亥),张凤翔家眷与杨仕聪同舟遇南京颁诏官,即同南行,借临清兵自卫;凌駉预戒兵丁,言北朝兵无送人南往之礼,到济宁即返。时冯铨、李建泰、谢升俱为大清国内院大学士。

初五日(庚寅),进左懋第兵部右侍郎佥都,经理河北;进郎中马诏愉太仆少卿、陈洪范太子太傅。

大清已除王鳌永总督山东、河南,以方大猷为监军。署巡抚事杨汝成、张维机从陆至,大猷遣牌送至济宁登舟。

初六日(辛卯),视朝毕,召廷臣及左懋第、陈洪范、马绍瑜议北使;遂召对面谕之。尚书顾锡畴恭拟祭告陵园文、祭告大行皇帝后文、吴三桂封爵制书敕命铁券、黎玉田高起潜敕命、谕宣北京人民谕,一一呈览。十四日(己亥),佥都御史左懋第言:`臣衔以经理河北、联络关东为命。夫河北,则山东、北直也;关东,则辽东矣。辽东久为清有,北直为清现居;山东虽杀伪官,遍地皆土贼。臣家人来云:胶州被围贼至十余万,则不皆向化可知也。经理实有封疆之责;以封疆重寄之衔而往议金缯岁币之事,名实相乖:此衔之当议者也。马绍瑜昔年赴清讲款,为清所折,奴颜婢膝,清送之参、貂;台臣陆清源纠之。其与清交情深浅,臣诚不知;但闻其私许金十万、银百二十万,逢人颂扬,臣不便与之同行也'。

十六日(辛丑),史可法奏:`邱茂华所称吴三桂师次庆都,建大清国顺治元年旗号,迫人削发'。

十八日(癸卯),催陈洪范速行。
二十一日(丙午),大清国遣辽人四名到沂州索粮户册。

二十三日(戊申),催左懋第、陈洪范星驰渡淮,银、币令马绍瑜随后护行。

三十日(乙卯),刘泽清请褒封吴襄,使三桂衔感。刘孔昭奏:`三桂父子效忠,宜加殊礼'。时举朝皆知三桂无心本朝,而奸党故欲崇之,已寓卖国之意矣。八月初二日(丁巳),光禄少卿沈廷扬奉命海运十万石饷吴三桂,道梗不可行,祈止之;上不许。二十三日(戊寅),赠吴三桂父襄辽国公。

凌駉在临海佯款大清国,驰奏亟乘机恢复;遂令巡抚王燮、总兵邱磊速赴任山东。改駉巡按山东御史,给空札一百劝功。

三十日(乙酉),兖东道郭正中奏大清骑下东省。

九月十四日(己亥),御史徐养心言人自德州来者,言山东有大清国巡抚方大猷、道臣张安豫牌赴济上;宜敕王燮早行。大清国总河杨方兴驻济宁,传檄山东州县,渐次款服。方兴,辽东贡生,登进士第一;尚主,历官内院。至是来总河,与济宁道朱国柱议取江南,修漕运。

十六日(壬寅),大清兵入宿迁。二十三日(戊申),大清将杨方兴收服土寇扫地王等。二十五日(庚戌),大清国山东抚方大猷承选丰、沛二知县胡增光、钦光到任;二人兄弟也,俱鱼台生员。

二十六日(辛亥),田仰报忻州、郯城、宿迁烽火逼近。

十月初三日(丁巳),大清国牌到济宁,称摄政王发大兵十万南下,谕州县预备粮草。有临清总兵进济宁驻札。初五日(己未),大清国东路兵到沂州、西路兵至濮。初八日(壬戌),大清国取丰县,胡增光入城;前知县刘燧走死。

十三日(丁卯),马士英奏赐王永吉一品斗牛服色,少隆接待北使之礼。刘泽清报:`赣、沭、沛、邳、曹、单、开、归,处处皆有大清兵;陈洪范、左懋第渡河无期,玉燮、邸磊赴任无地。徐州为张成福所守;成福送母至淮,令马化豹代须。今成福还徐、化豹回淮,大清将已在沂、郯;必令邱磊渡海先收登、莱。邳、宿正当南北通衢,令修清河废城,使马化豹、柏承馥防守。如此派定,以待使臣回日定和战'。

十六日(庚午),大清兵入海州。十七日(辛未),大清兵至宿迁界,乡兵羊酒迎之;县民尽逃。

十一月初四日(戊子),总兵邱磊报青州之变;磊于白沙祭海,装家眷、行李于船,将下船北发。初六日(庚寅),邱磊带百余骑至安东,柏承馥、王尊垣召磊进署,突兵擒之。至二十一日(乙巳),王燮为邱磊引罪。

初十日(甲午),大清兵破海州,将狱囚尽放;天明,回兵泗口。大清兵马八万分路南下,一向沭阳、一向邳州、一向宿迁;又牌行邻县,催办粮料。十一日(乙未),大清兵攻邳州,署印推官沈冷之固守待救。``遗闻''云:`史可法统兵抵白洋河'。十二日,大清兵入宿迁;可法提兵救之,随拔营去。

十三日,高杰抵徐州。先是,河南巡按陈潜夫探得大清朝于十月二十五日发兵,一往山东、一往徐州、一往河南,豫王将从孟县过河。杰与泽清书:`清朝发一王子,领兵号二十万,实七、八千;齐驻济宁。近日河南抚镇接踵告警,一夕数至;开封上下北岸,俱大兵问渡甚急。恐一越渡,则天堑失恃;长江南北,尽为战场。时事到此,令人应接不暇。惟有殚心竭力,直前无二,于万难之中求其可济,以报国恩而已'。泽清以闻。

十五日(己亥),刘泽清奏:`清将夏成祖已发济宁;杨方兴在宿迁集铁匠打铁条,为扎筏之用。臣今议分汛防河,三里一保、百步一圈,空处筑墙,挑濠灌水,勒令有司兴工。王燮、田仰、王永吉自安东至徐,萧、砀属督辅,开、归属越其杰,各申报竣;候左懋第回日另图也'。
二十日(甲辰),田仰言:`清将已驻沂、莒二州,哨马至沭、榆;辽人赵福星为宿迁道,兵五千镇守'。

十二月乙卯朔,大清国万骑下河南。

初三日(丁已),王永吉总督防河,刘、高二将联络张缙彦、王燮分布河北,王燮移驻淮上,命黄得功、刘良佐移驻近地以援邳、宿。

十五日(己巳),左都督陈洪范南还;上言:`初,礼部荐臣,以臣与吴三桂同里戚谊,意大清之破贼,必三桂为政;其事殊不然。九月十六日,臣至德州,大清抚方大猷示以摄政王令,有``来使不必敬护,止许百人赴京朝见''。夫曰朝见,则目无天使矣。阁臣主议,以抗节为不辱命;但知三桂借兵于清,未知大势之何如也。锦衣骆养性为之抚,遣兵相迎。二十九日,司务赞画王言齎臣名帖送内院,回言冯铨、谢升等词色甚薄,却帖不收。十月十二日,奉御书入正阳门;臣随宿鸿胪寺,关防甚严,水火不通,饥寒殊苦。十四日,内院刚林偕十余人来视,戎服佩刀,直登寺堂上坐,指地下毡,令臣等坐;大声责臣江南不应更立天子,且曰``毋多言,我将不日下江南''。十五日,刚林来收银,将十万两交讫;蟒缎余币,尚在后也。私计吴三桂不受书,则万金可无与,诸人踊跃抢散。明日,遣兵押行。臣等请祭告诸陵及改葬帝后,皆不许;朗诵檄文。二十七日,促行,防守甚严。十一月朔,至天津;复运缎绢悉押去。疑养性有私于臣,革职逮问。初四日,过沧州;有官来追,执左懋第回京,不容叙别。十六日,过济宁;大清兵乃还。十一日,到徐州,渡河'。洪范入见,言大清兵万分紧急,旦夕必下江南。马士英恶之,曰:`有四镇在,何虑焉'!陈洪范请加恩使北臣,兵科戴英劾止之;言`洪范出使无功,正使身陷异域、下役群聚晋爵,天下闻之,恐哄然窃笑也'!

十八日(壬申),马士英疏言:`清兵虽屯河北,然贼势尚张,不无后虑;岂遂投鞭问渡乎?且强弱何常之有,赤壁三万、淝水八千,一战而江左以定;况国家全盛,兵力万倍于前,廓清底定、痛饮黄龙,愿诸臣刻励之也'。命王永吉防河北、张缙彦防河南,分许定国、王之纲信地。``遗闻''云:`大学士王铎疏请视师江北,以复国仇;不允'。时大清兵至夏镇,别由济宁南渡,攻海州、围邳州。史可法、高杰、刘泽清各请告急;不应。

二十日(甲戌),命史可法会兵援邳州。二十四日,张绪彦分诸将防河:宁陵以东至归德属王之纲、宁陵以西至兰阳属许定国、祥符以西属刘洪起、河雒委李际遇。高杰北征,发徐州。

二十九日(癸未),加高杰太子少傅、史可法太傅。先是,程继孔斩木编筏,勾引北兵渡河,伪投杰降。杰知其诈,因诱斩之,收其众。至是,士英追理其功,故有是命。

使臣左懋第殉节

左懋第,字仲及,号萝石;登州莱阳人。崇祯辛未进士,出陈文庄之门。壬申冬,授韩城令。三年之中,流寇薄城者三、入境者再;皆设法击走之。癸酉,考选户科给事中。寻以吏〔科〕给事中,奉敕察核南京、燕湖等处兵饷;未复命而上崩。

宏光立,入见,陈中兴大计;命视师江上。升佥都御史,巡抚应、安等处。以母死于天律,乞守制。而朝议遣大臣使北通好,营先帝山陵并议割地岁币。公自请北行,因得葬母;升兵部侍郎,齎国书、金币以行。而副之者太子太傅、左都督陈洪范及太仆寺少卿、兵部职方司郎中马绍愉,兵部司务陈用极等从行。

八月,行次沧州,陈洪范遣信先致吴三桂封册;三桂不启封,缄奉摄政王。九月,至杨村。士人曹逊、金镳、孙正疆谒见,言报国之志;公喜,署为参谋。十月,进至张家湾。闻以四夷馆处使臣,行属国进见之礼;洪范无言,参谋陈用极曰:`此事所系甚大'。公争之,乃改鸿胪寺;遣官骑迎入。十四日,内院刚林来,责以朝见;公曰:`敕命先谒陵、后通好;今未拜先帝梓宫,不敢见'。刚屈而去。明日复来,言如前,公终不屈;一一抗拒,声色俱厉。既,持国书、金币去。公遣参谋陈用极以谒陵事请,不得;乃陈太牢于寺厅,率将士哭三日。

二十七日,忽数骑遣行,出永定门。十一月初五日,止沧州里铺;又数骑追执公及绍愉还,而独令洪范南。副将张有才、杨逢春、刘英止沧州;公返北都,拘之太医院,不通出入。上摄政王启,不报,而时令人说之降;公不答。洪承畴谒之,公曰:`鬼也?承畴松、杏败死,先帝赐祭、加醮九坛、锡荫久矣;今日安得更生'!李建泰亦来谒,公曰:`受先帝宠饯,不徇国降贼,又降清;何面目见我耶'!汉臣投谒者皆受骂,亦惮见之。

乙酉正月,刘英及曹逊、金镳入讯,逾垣得见;遂发疏,令金镳及都司杨文泰赴金陵奏之。及至,而金陵已失守矣。曹逊曰:`如何'?公曰:`复何言'!七日不食,恸哭誓必死。

闰六月十五日,以江南即平,再下薙发令。副将艾大选首髡如诏,公杖大选及传浚,大选自经死;浚恐,为蜚语闻。十九日,捕下刑部;公曰:`我自行我法、杀我人,与若何与?可速杀我'!以兵胁公薙发,公大呼`不可'。而参谋兵部主事陈用极字明仲,苏州昆山人;与游击王一斌、王廷佐、张良佐、守备刘统亦大呼`不可'。遂以公等六人下狱。

二十日,摄政王召见,铁锁拥入内朝;公麻衣孝巾,向上长揖,南面坐于庭下。摄政王数以伪立福王、勾引土贼、不投国书、擅杀总兵、当庭抗礼五大罪,而公辩对侃侃,终不屈,惟请一死。命薙发,坚不肯。摄政王问在廷汉臣云:`如何'?吏部侍郎陈名夏曰:`为福王来,不可饶'。公曰:`若中先朝会元,今日何面目在此'!兵部侍之郎金俊曰:`先生何不知兴废'!公曰:`汝何不知羞耻?我今日祗有一死,又何多言'!摄政王挥出斩之。佥都赵开心将起有言,同坐掣其裾而止。公至宣武门外,神气自若;南向四拜,端坐受刑。侩子杨某涕泣稽首,而后行刑。公既出,赵开心始得启王,王将从之;而已报死矣。题绝命诗有曰:`峡坼巢封归路回,片云南下意如何!寸丹冷魄消将尽,荡作寒烟总不磨'!马绍愉率所从将士,悉薙发降;陈用极、王一斌、王廷佐、张良佐、刘统与公同日见杀。忽沙风四起,卷市席棚于云际,屋瓦皆飞;一时罢市。

陈用极之门人咸默,序其事传之。盖国朝以奉使死者,忠文王袆、忠节吴云与公三人而已。公与会稽章大理正宸谊最深;公死,大理亦遯荒野。公之同乡姜给谏采,出其诗以梓于世。

东村老人曰:`萝石之死,比文信公尤烈。有一人而可洗中朝三百年之气,可见读圣贤之书者原有人实践。纷纷盗名之辈,妄言声气,卖降恐后矣'!

左公至北,陈洪范欲以国书畀礼部。公谓馆伴:`必以龙亭出迎;不然,敕书必不可与'!故摄政王责公不投国书。

凌駉自缢于济馆
凌駉,原名云翔,字龙翰;徽州歙县人。崇祯癸未进士;甲申正月,授兵部职方司主事、督辅军前赞画。曲沃兵溃,駉独走至临清,纠合三百人起兵;擒伪防御使王皇极等三人。传檄山东,其略云:`迹今逆贼所恃,无过假义虚声。假义则预免民租,虚声则盛称贼势。以致浮言胥动,举国如狂;愚懦无知,开门揖寇。及至关城一启,即便毒楚交加;一宦而征数万金,一商而派数千两。非刑拷比,罔念尊贤;纵卒奸淫,不遗寡幼。将军出令,先问女人;州县升堂,但求富户。于是,山东、河北各土寨来归者甚众'。

上疏南京,改浙江道监察御史,巡按山东。而大清兵日逼,駉复上疏言:`臣以铅椠书生,未谙军旅;先帝过简,置之行间。遭值危亡,不能以死殉国。乃以万死余生,纠集义师,讨擒伪逆,诚欲自奋其桑榆之效;然不藉尺兵、不资斗粟,徒以``忠义''二字激发人心。方今贼势犹张,东师渐进;臣已上书彼国大臣,反覆恳切,不啻秦庭之哭矣。然使东师独任其劳而我安享其逸、东师克有其土而我坐受其名,恐无以服彼之心而伸我之论。为今日计,暂假臣便宜,权通北好:合兵讨贼,名为西伐,实作东防。俟逆贼既平、国势已立,然后徐图处置之方。若一与之抗,不惟兵力不支;万一弃好引仇,并力南向,其祸必中于江、淮矣!若臣之自为计,则当不出此。臣南人也,即不肖而有功名之想,尚可几幸于南;但恐臣一移足而南,大河之北便非我有。故忍苦支撑于此,以为他日收拾河北、畿南之本。夫有山东,然后有畿南;有畿南,然后有河北。临清者,畿南、河北之枢纽也;与其以天下之饷守淮,不若以两河之饷守东。乞皇上择一不辱君命之使臣,联络北方,以弭后患;宣慰山东州县,以固人心'。时朝廷已遣陈洪范北行,而竟无一兵救山东者。大清兵尽下山东州县,駉南走至大名。大清国以兵科印札招駉,駉悬之陈桥驿中,遂独身至南京。

入对,复差巡按河南。駉受命,疾驰入归德,而大清兵已至城下。大帅王之纲引兵南走,独駉与士兵数百守城中。游击赵擢入城说降,駉斩之以徇。次日,率兵出西门斫营,而守者已开东北门迎降。大清帅传令:必生致駉。駉自刎,为其麾下所持。乃以两印投井中,命参将吴国兴等齎敕旨并具遗疏入奏。即书一官衔帖,与其从子润生单骑诣营。见大清帅豫王,长揖不拜;豫王雅重駉,命具酒馔,亲持金爵饮駉,駉辞以性不饮酒。留营中,另设一幕,赠大帽一、貂裘一、革舄一,駉不受;强留之。一日,夜与侄润生同自缢死。遗豫王书曰:`世受国恩,济之以死,臣义尽矣。愿贵国无负初心,永敦邻好;大江以南,不必进窥。否则,扬子江头凌御史,即昔日钱塘江之伍相国也!承贵国隆礼,人臣义无私交,谨附缴上'。豫王令殡之察院公署,送银百两治丧;城中吏民皆大哭。駉母年七十岁、子四岁,登第后未得一省云。事闻,朝廷壮之,下部议恤;会国亡,不果。自宏光初立,史督辅请分南四镇,遂无一人计收山东者。使乘大清兵未下之日,一旅北出与公犄角,上扼沧、德,下蔽徐、兖,天下事未可知也。

``编年''云:大清兵至范家塞,总兵王之纲邀巡按凌駉南避;駉不听。大清陷睢州,巡按御史凌駉被执,不屈;与侄润生自缢。事闻,赠駉兵部侍郎、润生御史。

大清兵剿青州土贼

大清于正月初六日发兵往青口,又调登州、天津海船巡逻平度州。望高山有土贼作乱,烧莱州西关;有号许王者,兵数万,屯青州。大清兵往剿。

大清豫王晓谕

四月十七日(己巳),大清国摄政王晓谕江南、南京、浙江、江西、湖广等处文武官员军民人等知悉:尔南方诸臣向佐明朝,崇祯皇帝有难,天阙焚毁;国破君亡,不遣一兵、不发一矢,不识流寇一面,如鼠藏穴:其罪一也。及我进战,流寇西奔;尔南方未知京师确信,又无遗诏,擅立福王:其罪二也。流寇为尔大仇,不思征讨;尔诸将各自拥众,扰害良民,自生反侧以起兵端:其罪三也。此乃天下所共愤,王法所不赦;予是以恭承王命,问罪征讨。尔文武官员,速以地方城池投顺者,不论官之大小,各升一级;抗拒不顺者,自身遭戮、妻子受俘。如福王改悔前非,自投军前,面释其罪,与明朝一体优待;福王亲信诸臣亦知罪改过归诚,亦与禄俸。文到之日,士民不必惊慌逃避,农夫照前耕种;城市秋毫无犯,乡村安堵无妨。但所用粮草,预解军前。兵部作速火牌晓谕,毋得迁延,以违军法。咸使闻知。

议御北兵

大清兵攻破徐、砀,又破亳、泗。四月初八日(庚申),史可法三报紧急;宏光曰:`上游急则赴上游、北兵急则赴北兵,自是长策'。可法曰:`上游不过欲除君侧之奸,原不敢〔与〕君父为难;若北兵一至,则宗社可虞!不知辅臣何以朦蔽至此'?乃遗书马士英,恳其选将添兵;大声疾呼。士英惟以左兵为虑,不应。

初九日(辛丑),大清兵至颍州,南将降者、逃者相半。梁云构请合刘泽清、黄得功将兵入卫,黄斌卿请留驻防。

初十日(壬戌),徐、邳告急,令卫胤文、李本深督兵驻泗州。

十四日(丙寅),刘泽清、刘良佐各请将兵入卫;谕以防边为急。

十五日(丁卯),刘洪起奏:`大清兵乘势南下,如同破竹,无人敢遏;恐为南京之忧'。王永吉奏:`徐镇孤危援绝,势不能存。乞敕史可法、卫胤文共保徐州,方可以保全江北'。十七日(己巳),史可法奏:`大清骑分路南下,镇将平日拥兵糜饷,有警一无足恃'!又奏:李成栋弃地南奔。士英亦不应。

时塘报汹汹。十九日(辛未),宏光召对,士英力请亟御良玉,大理寺卿姚思孝、尚宝寺卿李之椿等合词请备淮、扬;工科吴希哲等亦言淮、扬最急,应亟防御。宏光谕士英曰:`左良玉虽不应兴兵逼南京,然看他本上意思,原不曾反叛;如今还该守淮、扬,不可撤江防兵'!士英厉声指诸臣,对曰:`此皆良玉死党,为游说;其言不可听!臣已调得功、良佐渡江矣。宁可君臣皆死于大清,不可死于良玉之手'!瞋目大呼:`有议守淮者斩'!宏光默然,诸臣咸为咋舌。于是北守愈疏矣。礼部尚书钱谦益言:`陈洪范还该收他'。宏光曰:`国家何尝不收人,只是收来不得其用耳'!希哲退曰:`贾似道弃淮、扬矣'!

先君子述舅氏语曰:`宏光召对时,群臣俱请御北兵;宏光然之。独士英大声面斥上曰:`不是这样讲,宁可失国于大清'云云。宏光不敢言。又朱大典含怒入朝堂,曰:`少不得大家要做一个大散场了'!众闻之愕然。
史可法扬州殉节

四月二十二日(甲戌),大清兵渡淮,如入无人之境。二十四日,大清兵猝至扬州,围攻新城。可法力御之,薄有斩获;恐益急。可法书寸纸,驰诣兵部代题请救;不报。二十五日(丁丑),可法开门出战,大清兵破城入;可法拔剑自刎。原任兵部尚书张伯鲸被执不顺,身被数创,自刎死;妻杨氏、媳郝氏从之。伯鲸标下游击龚尧臣被执,不屈死。

``甲乙史''云:大清兵渡淮,是晓猝至扬州,破新城。史可法在旧城,大清檄云:`若好让城,不戮一人也'!可法不为动。丁丑,大清兵诈称黄蜚兵到,可法缒人下城询之;云蜚兵有三千,可留二千在外、放一千入城。可法信之;时大清兵在东门,约以西门入。及进,而反戈击杀。可法立城上见之,即拔剑自刎;左右持救,乃同总兵刘肇基缒城潜去。或云引四骑出北门南走,没于乱军中。或云大清兵锐攻北门,可法震大炮击之,死者甚众;再震而愈聚,攻益锐,已破西门入矣。拥可法见豫王,长揖不屈;遂遇害。

予思甲戌渡淮,是晚猝至扬州,未必如此之速;则疑丙子为是。至于史公死节,其说不一。然豫王入南京,五月二十二日(癸卯)即令建史可法祠,优恤其家。是王之重史公,必在正言不屈;而``缒城潜去''之说非也。更闻江北有史公墓;康熙初年予在淮扬,见公生祠谥为``清惠'',父老犹思慕焉。忆顺治六年仲冬,予入城应试。有浙之嘉兴人同舟,自言久居于扬;问以大清兵破城事,彼云:`我在城逃出,稔知颠末。初,扬人畏高杰淫掠,乡民避入城;后水土不服,欲出城,江都令不许,遂居于城。四月十九日,大清豫王自亳州陆路猝至扬州,兵甚盛,围之。时史可法居城内,兵虽有,能战者少;闭城坚守,不与战。大清以炮攻城,铅弹小者如杯、大者如罍;堞堕,即修讫。如是数次,而炮益甚,不能遽修;将黄草大袋盛泥于中,须臾填起。大清或令一、二火卒侦伺,守兵获之,则皆欢呼请赏,可法赐以银牌;殊不知大清兵甚众。可法日夜待黄得功至;围至六日,乃二十五日(丁丑)也,忽报曰:`黄爷兵到'。望城外旗帜,信然;可法开门迎入。及进城,猝起杀人,知为大清人所绐,大惊;悉弃甲溃走。百姓居新城者,一时哗叫,不知所为;皆走出城,可法不知所终。史公短小精悍,面黑;在军中茹麦粞饭,食不二味。众共怜之。

予按宋恭帝时元右丞相阿杰围扬州日久无成功,筑长围困之。城中食尽,死者枕籍满道。明太祖将缪大亨克扬州,止余民十八家。然则宋、元迄今,扬民三罹劫矣;岂繁华过盛,造化亦忌之耶!

\hypertarget{header-n57}{%
\subsection{卷九·南都甲乙纪(续)}\label{header-n57}}

宏光出奔

五月初十日,京师各城闭门。午后,唤集梨园子弟入大内演戏,上与太监韩赞周、屈尚忠、田成等杂坐酣饮。二鼓后,上奉太后、一妃与内官四、五十人跨马从通济门出,文武百官无一人知者。遗下宫娥女优五、六十人,杂沓于西华门内外;得随一人拉去为幸(``编年''云:`上跨马从聚宝门出狩')。

上至太平府,刘孔昭闭城不纳,傍徨江次;乃奔坂子矶就黄得功营。得功方出兵与左兵战;闻之,即归营,向泣曰:`陛下死守京城,臣等犹可借势作事。奈何听奸人之言轻出,进退将何所据?此陛下自误,非臣等误陛下也!臣营弱薄如此,其何以处陛下哉'!居两日,刘良佐奉大清豫王令追至,且召得功。得功怒,单骑不甲而出,隔河骂之;挥鞭誓死,言`我黄将军志不受屈'!良佐伏弩射中得功喉,得功叹曰:`吾无能为矣'!归营拔剑自刎。良佐即入其营,挟上回南京(一云:马士英撤江北诸军堵左兵,惟刘泽清不行、亦不北拒,大清兵遂直下。五月十一日,宏光出奔。十二日,驻太平府二十里外,阮大铖、朱大典、方国安等来见,欲避入太平;刘孔昭率百姓闭城不纳。十三日,往芜湖;水师总兵官黄斌卿先遁,登中军翁之琪舟。十四日,因就黄得功居两日,将谋往浙。刘良佐追及,得功死之;兵未渡浮桥,铁索忽断,军士望洋而止。上遂蒙尘,翁之琪投水死)。

五月二十五日,宏光以无幔小轿入城,首蒙包头、身衣蓝布衣,以油扇掩面;太后及妃,乘驴随后。夹路百姓唾骂,有投瓦砾者。进南门易马,直至内守备府。见豫王叩头,豫王坐受之。命设酒于灵璧侯府,坐宏光于太子下。赵之龙暨礼部八人侍宴,唤乐户二十八人歌唱,饮酒。席中豫王向宏光问曰:`汝先帝自有子,汝不奉遗诏,擅自称尊,何为'?又曰:`汝既擅立,不遣一兵讨贼,于义何居'?又曰:`先帝遗体,止有太子;逃难远来,汝既不让位,又转辗磨灭之,何为'?宏光总不答。太子曰:`皇伯手札召我来,反不认;又改姓名,极刑加我。奸臣所为,皇伯不知'!豫王又曰:`我兵尚在扬州,汝何为便走?自主之耶,抑人教之耶'?宏光答话支吾,汗出沾背,终席俛首。席散,拘于江宁县,与太后、一妃同处。豫王令旧臣往视,惟安远侯柳祚昌、侍郎何楷视之;宏光嘻笑自若,但问马士英奸臣何在尔。

黄得功既死,得功左协部将田雄负宏光与右协部将马得功降附大清,献于豫王。当雄负宏光时,宏光恨甚,啮其肩,遂成人面疮。时在五月;后每逢夏五月便发,痛不可忍。每日食肉三斤,以一脔覆其上,痛稍止。顷之,复痛;又易新肉覆之,痛乃缓。已而复痛,反覆不得休息。如是者十八年。至康熙二年五月二十日,终以此疮死。雄字明宇,宣府左卫人。得功两目赤,临阵大声呼疾,故众号为``马叫唤''。得功字小山,辽东广宁人。

附记:五月初一日,有书联于东、西长安门柱云:`福人沉醉未醒,全凭马上胡诌;幕府凯歌已休,犹听阮中曲变'。又云:`福建告终,只看卢前、马后;崇基尽毁,何劳东捷、西沾。先是,三月下旬夜半,书马士英堂中云;`闯贼无门,匹马横行天下;元凶有耳,一兀直捣中原'。求其人不得。``福人'',指宏光;本福王也。阮大铖喜作歌曲,时为兵部报捷,故``幕府''云云。``卢'',太监卢九德也。``西沾'',李沾也。``闯贼无门'',骂士英马贼也。``元凶有耳'',阮字也。

豫王渡江入城

五月初八日,大清兵驻瓜州,排列江岸,沿江窥渡;惟总兵官郑鸿逵、郑彩帅水师御之。京口兵船,则有时到江中;而黄斌卿、杨文骢兵列南岸,隔江互发炮声相应如戏赛者已三日矣。

初九日晨,大清兵开闸放行,蔽江而南。二郑兵见之,各扬帆东遁。江南之师,一时皆溃,武弁各卸甲鼠窜。巡抚霍达方整导出衙,未至江边,即狼狈返;易服杂下役中,窃逃附小舟,潜入苏州。郑鸿逵复入丹阳,烧劫南走,鸡犬一空。黔兵之从杨文骢者,存二百五十人,奔还南京。传言大清兵已下江,京口无备;都人大震。豫王谋渡江,夜半乘西北风大顺,令军中每人具案二张、火十把;如违,笞四十棍。众兵掠民间台几及扫帚,将帚系缚台足上,沃油燃火,昏夜乘风放入江中,顺流而下,火光彻天;南兵见之,谓大清师济江,遂大发炮击之。然风顺水急,愈击愈下;久之炮几尽。王乃从七里港渡江(``遗闻''云:`初八日夜,大清兵编筏张灯由镇江,而别由老鹳河渡。初九日,尽抵南岸。老鹳河,即俗称七里港')。

十四日,豫王兵到都城,忻城伯赵之龙率礼部尚书管绍宁、总宪李乔各遣二官缒城出迎,跪道旁,高声报名。将近豫王前,喝起,众人仓皇入报。此时大雨淋漓,无一骑、一卒敢跕檐下者。二大僚匐匍进,行四拜礼。豫王驻师天坛中。

附记:豫王到城下,遣四十人入城,询问降情真否;众以实对。北使乃出,王令兵退四十里驻营(或云:即紫金山下是也)。初,豫王驻师城外,赵之龙欲迎入,百姓不愿,罗拜于地。之龙下马,谕众曰:`扬州已破,若不迎之、又不能守,徒杀百姓耳!惟竖了降旗,方可保全'。众不得已,从之。

赵之龙,号易庵;河南仪封籍,南直虹县人。太子太保、忻城伯。

十五日,大开洪武门。二大僚统百官献册,行四拜礼。赵之龙叩首,请豫王进城。保国公朱国弼、镇远侯顾鸣郊、驸马齐赞元咸至,豫王问勋戚为太祖、为成祖?之龙一一具答。豫王喜,加之龙位兴国公,命立朱国弼上,赐金镫银鞍马、貂裘八宝帽。令军中设牛酒,席地共坐。豫王问太子何在?之龙以``王之明''对。豫王曰:`逃难之人,自然改姓名;若说姓朱,你们早杀过了'。朱国弼曰:`太子原不认,是马士英所易'。豫王大笑曰:`奸臣!奸臣'!晓间,赵之龙捧太子出城至营,豫王离席迎之,坐于己右,相处不离丈许。李乔进城齎告示二道,一为大清摄政王晓谕江南文武官民、一为钦命定国大将军豫王晓谕南京官民;大约言`福王僭称尊号,沉湎酒色,信任佥壬,民生日瘁;文武弄权,只知作恶纳贿,惟思假威跋扈。上下离心,远近仇恨'。时以为实录。

十六日晨,豫王受百官朝贺,递职名到营参谒如蚁。赵之龙令百姓家设香案,黄纸书``大清皇帝万万岁'';又大书``顺民''二字粘门。王铎诣营投到;以其弟王\{金磨\}在营,甚礼之。查不朝参者,妻子为俘;差假,本堂报知注册。每日点名,大僚俱四更进而午复归。工部尚书何瑞征先于十一日自缢不死,损左足,卧家不朝;王令缚之。瑞征索剑自刎,其子持之。赂官以揭进,禁官为之请;乃准调理。

附记:是日,郑鸿逵兵过石幢。予往东观之,水陆拥挤疾行,自北而南,凡三昼夜。或云六万人。呜呼!虽多亦奚以为!

十七日,礼部引大清官二员从五百骑由洪武门入,骑谓城上人曰:`勿放炮'。礼部向帝阙四拜,因泪下。大兵问故?礼部曰:`我痛惜高皇帝三百年之王业,一朝废坠;受国厚恩,岂不痛心'!大兵为之叹息。候正阳门开,索匙不得;礼部引进东长安门。盘九库见银九万两,即命此官驻皇城内守之。总宪李乔独先薙头易服,豫王骂之。

附记:常州知府郭佳胤遁入太湖。郭字如仲,号夔一;河南归德府宁陵县人。崇祯丙子举人,丁丑进士;初为无锡知县,后即升常州太守者也。时大清已遣使至常州索册,府无正官,留张守备坐堂。是日,无锡放监铺。

先是,南京居民自相禁止,途次寸步难行。至是,以豫王晓谕,百姓居行如故。

十八日,文武官与坊保进牲醴、米面、菜果于营,络绎塞路。赵之龙唤优人十五班进营开宴,逐套点演;正酣悦间,忽报各镇兵至,之龙跪呈豫王,王不为意。戏毕散席,发兵三百,遣将将之,即行。有顷,擒刘良佐至。良佐叩首,请以擒宏光赎罪。豫王允而遣之,随拨三百人同行(或云大清将招刘良佐曰:`尔等豪杰,不知天命乎'?良佐遂请降。又内官进鲥鱼二大箩,极其卑礼;豫王不受)。

十九日,赵之龙同大清兵并骑入城。分通济门起,以大中桥北河为界,东为兵房、西为民舍;通济、洪武、太平、神策、金川凡六门,居大清兵。自是东北城民居日夜搬移,提男抱女,哀号满路;西南民房一椽,日值一金。豫王斩兵抢物者八人;并示`前日入内抢掠诸物,自行交还江宁县;藏匿者枭示'。

附记:``无锡日记''云:`是日下午,常州推官伺家驹在无锡杀二人于大市桥。二人俱姓华,兴道乡人;兄弟五人在乡间抢掳,族叔呈之,立刻枭首。所抢不过西瓜及酒二坛而已,族叔止欲笞之;以时乱,借以警众,遂杀之。族叔亦悔而泣焉。

二十日,内院学士洪承畴牌谕:`翰林大小官每日入内办事,仰掌院陈于鼎造册送进。每日清晨,点兵'。午后,令文武将印信、札付尽数交纳武英殿换给,御史王懩、大理丞刘光斗、鸿胪丞黄家鼒等往各府取降顺册。

二十一日,大放三日,妇女出城者万计。赵之龙先薙头,魏国、安远、永康、灵璧、临淮诸爵以渐剃讫;文官惟李乔、姚孙矩自薙。

二十二日,豫王令建史可法祠,优恤其家。

二十三日,豫王进城,衣红锦箭衣,乘马入洪武门。白棍一对前导;文武班列道旁,无一不至者。

二十四日,刘良佐以宏光到,暂停天界寺。

二十六日,豫王各城门帖示云:`薙头一事,本国相沿成俗。今大兵所到,薙武不薙文、薙兵不薙民;尔等毋得不遵法度,自行薙之。前有无耻官员先薙求见,本国已经唾骂。特示'。黄营兵数万人随大清官进城,向豫王求用;豫王收其衣甲,散遣之。

二十七日,豫王谒明太祖陵,行四拜礼。四顾嗟叹,唤灵谷寺住持速行修理。黄家鼒至苏州,抚臣霍达复归郡(一云家鼒至苏州招抚,被害)。
二十八日,豫王出南门报恩寺行香,观者如堵。黄端伯向豫王愤惫大恸,赵之龙欲杀之,豫王不许;之龙乃执送狱。豫王令确报殉节诸臣及民间妇女,各坊共报男女二十八人。

二十九日,豫王令调兵八万下苏、杭。

三十日,豫王以宏光所选淑女配太子。数月后,北行;太子及宏光随之,后俱凶问。

附记:刘孔昭自太平掠舟顺流而东,江行入常熟,诡言起义;佥都御史霍达招之入郡,不应。停攻一县,白粮满载入海。

附记:``无锡日记''云:`五月二十七日,刘光斗至无锡讨册,舟泊西门桥。光斗,武进人;天启乙丑进士。崇祯朝,为河南道御史;因贪黜罚。大清入南京,遂降附为官。安抚常、镇士民,讨州县户口、粮役册,旗盖炫耀,邑中乡绅拜之者如市。望亭巡检来见,光斗曰:``汝好,该升一级''。即升主簿,掌县印。将粮船俱提常州去。先有示云:``安抚刘批:该县速备船只,士民不必惊慌''。常镇道张健批:``本道发令箭一枝,仰无锡百姓各安生理;大兵到处,秋毫无犯'''。又`六月初一日,苏州巡抚霍达将粮散于百姓;常州竖``顺民''旗,至丹徒迎大清兵'。`初二日,无锡选贡士王玉汝等具肉一百担、面一百担、羊三头,以迎大清兵。传闻大清兵恶门神,城中各家洗去;粘``大清万岁''于门上。按玉汝字元琳,庶吉士王表裔孙;崇祯甲申,选贡。大清兵南下,时刘光斗与玉汝善,移札曰:``师至而抗者屠,弃城而乏所供应者火。当为桑梓图万全''!玉汝乃与邑民具牛酒公迎。已而同邑顾杲拥众鹅湖,玉汝单舸往谕,遂遇害。杲掠沙山,亦为土人所杀'。

南都殉节诸臣

乙酉五月十二日,钦天监挈壶官陈于阶自经。此死节之最早者。豫王入京,刑部尚书高倬、户部郎中刘成治署中自经死。十八日,国子监生吴可箕鸡鸣山关帝庙中缢死。其死而不知日者,中书舍人陈爊及子举人伯俞、户部主事吴嘉胤也;不知名者,投秦淮河中冯小珰与百川桥下乞儿也。小珰以色幸,卒以身殉。乞儿题诗桥上有云:`三百年来养士朝,如何文武尽皆逃?纲常留在卑田院,乞丐羞存命一条'!又礼部郎中刘万春、主事黄端伯以不朝,被杀。端伯字元之,江西南昌人;深明禅学。其绝命词云:`问我安身处,刀山是道场'!又``补遗''云:`南京之变,以死闻者,尚书何瑞征、光禄卿葛征奇、户部郎刘光弼也'。

附记:刘万春,扬之泰州人。大清兵入南部,万春降。时豫王禁女出城,万春有妾在城内,缒之而出,为守者所执;入见豫王,万春大骂而死。此与前载稍异;乃泰州邵廷辅口述。邵又云:`吴甡,扬之兴化人;崇祯朝大学士。大清兵至,祝发居师姑潭;自题句云:``宰相出家,师姑潭里吴和尚''。久之,无有续其联者。

宏光时,有古史不经见者二事:其始立也,革工常应俊封伯;及其失也,乞儿死难。一勋臣与一忠臣异矣!然封伯,遇也,为应俊易;死难,义也,为乞儿难。予思乞儿非常人,盖隐君子也;欲以一死愧当时大臣之不如乞儿者。

龚廷祥沈水死
龚廷祥,字伯兴,号佩潜;无锡人。幼时,乡达陈幼学一见称异。为诸生,游马文忠世奇门。崇祯已卯举人,癸未进士;有``不愿为良臣、愿为忠臣''之语。甲申思宗殉社稷,世奇殉难;廷祥设师位,为文祭且哭,如谢翱祭文信国状。乙酉,补中书。

居无何,南都陷。廷祥具衣冠,别文庙;登武定桥,睹秦淮河叹曰:`大丈夫当洁白光明,置身天壤;勿泛若水中凫,与波上下'。迺发愤自誓曰:`敢贪生以全躯者,有如此河'!遂沈水死。

前一夕,手书寄子书成,付家人;越日乃逝,实五月二十三也。书曰:`节义之士,何代无之。只是吾节不成节、义不成义,愧赧在心!愿吾诸儿守父训诫,做好人、行好事;吾虽在地下,有余荣矣。但目前事,不得不细言之:自吾正月出门,与吾母执手相别,欲得一诰命以荣父母。四月十八日,果命下,准诰封;吾事济矣。吾又讨差,可归定省矣。不意五月十一日,天子播迁。吾是时艰苦万状,有欲强吾奉迎一事者;吾此心何心,忍背国恩乎?唯有捐躯见志而已。但思吾一见老母而不得,肝肠寸剖、血泪满襟。气数既如是,汝辈要小心谨慎,奉事祖母;切不可预外事,切不可得罪于人,至惹灾祸:此吾之孝子也。吾因生平愚拙,事事要学古人,故至于此。然不忠、不孝,何以见先人于地下!念之怆然,思之快然'。

附记:公幼颖敏,其父令作破题,时有烛在案,即以为题。公作一破云:`丹心照国,身尽而心完矣'。父大赏之,知非凡儿也;后竟以为谶。公家贫,杭济之先生尝云:`公作文迅疾,有中才'。一日,应童子试不利,共走常州;晨饮白酒于市即大吐,俱粉团也。盖贫不举火,买团坊问,因饿劳作呕耳。时会严寒,与先君子同卧外舅氏。及晨,先君子起,闻公在帷中作衣被声,良久不起。先君子问之,公应曰:`汝不解妙法'。及揭帷,公语先君曰:`吾服尚无棉,颇觉背冷。今以裤下一层反拊背上束之,岂非妙法乎'?相与一笑。其贫苦若此。
华允诚不跪死

华允诚,字汝立,号凤超;常州无锡人。天启壬戌进士;癸亥,选工部都水司主事。会魏奄用事,诸名贤皆放逐;公假归。崇祯己巳,起补营缮司主事;寻升员外郎。其冬,大清兵入塞,都城戒严。诸曹郎分守城门,多以守御不备,杖阙下,有死者;而公守德胜门,独完。调兵部职方员外;乞休,不允。公见当时铨阁比周,举错徇私;上疏言``三大可惜、四大可忧''。可忧一条言:`国家罢设丞相,用人之职,吏部掌之,阁臣不得侵焉。今次辅、冢臣以同邑为朋比,惟异己之驱除;阁臣兼操吏部之权,吏部惟阿阁臣之意。线索呼吸,机关首尾:庇同乡,则逆党可公然保荐;排正类,则讲官可借题逼逐'。又言:`丧师误国之王化贞宜正罪,洁己爱民之余大成有可矜'。疏入,奉旨切责回话。公再疏直纠次辅温体仁、冢臣闵洪学罪状,言尤切直。体仁、洪学疏辨,幸上明察,颇得其情,公仅得罚俸。未几,以终养归。上寻释余大成于狱,置王化贞于法,逐唐世济而罢闵洪学;皆用公之言。

公里居十余年,而有京师之变。南京立,起补吏部验封司员外郎,署选司事。公见时事日非,叹曰:`内无李、赵,外无韩、岳,欲为建炎、绍兴,亦何可得'!遂谢归。南京陷,公惟饰巾待尽,杜门者三年。

戊子,潜居乡间,偶过其婿家,会有告其婿未薙者,下逮,并执公。见巡抚土国宝,国宝劝公薙发,不从。解至南京,见巴帅,不跪。时巴着快靴,踢折公膝;复拔公发几尽。公曰:`吾不爱身'!遂见杀。从孙尚濂字静观,平日举动皆效公;同日遇害,年仅十九耳。

公登第,出贺文忠逢圣之门;而师事高忠宪攀龙。尝师程子静坐,终日如泥塑人。忠宪临难,特书一帖授公曰:`心如公虚,本无生死'。公遂豁然于生死之际矣。诗文不多;盖得力在理学,文章其余技也。最著者有``渡江''一律云:`视死如归不可招,孤魂从此赴先朝。数茎白发应难没,一片丹心岂易消!世杰有灵依海岸,天祥无计挽江潮。山河漠漠长留恨,惟有群鸥伴寂寥'。人共传之。

徐汧沈虎邱后溪死

徐汧字九一,号勿斋;长洲人。崇祯元年(戊辰)进士,改庶吉士。授简讨,累迁右春坊右庶子。庚辰,分考礼闱。辛巳,奉差南归;寻丁忧。南京建国,起詹事府少詹事、翰林院侍读学士。公知事不可为,不之官。乙酉闰六月,大清兵至,下令薙发;公誓不屈辱,曰:`以此不屈膝、不薙发之身见先帝于地下'!遂自沈于虎邱后溪死。

公自己巳之难,从都中寄书故人曰:`明天子在上,知万万无虞;然事势危急,即有不可知,惟以一死报君父'。甲申之变,公方里居,号恸欲绝。是年烈皇圣诞,感激赋诗四章,言之血泪。自题画像曰:`汧乎!而忘甲申三月十九日事耶?而受先皇厚恩,待以师臣之礼;而子枋、柯以稚子,一登贤书、一食廪饩,尺寸皆先皇赐也。而不能断脰纳肝以殉国难、复不能请缨枕戈以雪国耻,而偃息在床,何为者耶?义当寝苫,罪当席席稿。存此寝苫、席稿之心以教诲尔子,庶几其勉于大义,毋若厥父偷惰负恩也'。盖公忠义出于天性,捐躯报国,其志然也。公少就学于兄养淳,养淳为陈文庄妹婿;因得公文,奇之曰:`吾里中乃有汤若士'!每向人述文庄言,有知己之感。

公长子孝廉枋,自公没后,杜门不入城市。

附``圣诞哀感''云:`洒泪先皇似向隅,吞声岂忍忆嵩呼!衣冠此日趋南阙,玉帛何年会冀都?圣主哀思应避殿,微臣隐忍尚全躯。亦知佐命悲欢异,还记今朝令节无'?又``挽许琰''云:`祸缠霄极帝星微,龙驭苍黄去不归!汉殿衣冠浑欲扫,燕京钟虡已全非。人轻李萼师谁借?邑□王生义庶几。赡拜鼎湖因北首,朝朝应见素魂飞'。

孙源文哭死
孙源文字南公,无锡人;万历甲戌状元。孙继皋季子。性孝友,博学工诗文;凡河漕、军屯、钱赋、历律、山川、星纬之书,悉窥其奥。

甲申三月思宗殉社稷,源文昼夜哭。鬻产得金,仿宋任元受故事,集缁流刺血为文,恭荐帝右,躄踊几绝;观者皆泣下。遂咯血声喑。赋诗曰:`少小江南住,不闻鸿雁哀;今宵清枕泪,和尔旧京来'。悲吟不辍,疾益甚。友人询以后事,唯曰:`家受朝廷特恩,死吾分也'!余不及。遂卒。论者谓:源文一草莽臣耳,至悲其君以死,岂特屈原之于怀王哉!

严绍贤同妾缢死

严绍贤字与扬,无锡人;为吴诸生,从叔司寇严一鹏籍也。生而正气岳岳,周文简炳谟深器之;每以正谊相砥。崇祯末,流寇蠢动;绍贤侍司寇,辄云`烽火照天,当坐卧临池一小楼。势亟,有蹈水死耳'!其蓄志如此。

甲申思宗殉社稷,绍贤每慷慨流涕;痛不若都城一菜佣,犹得望梓宫奠杯水也。自此憧憧,若失所依。乙酉,新令下,知国祚改。忽题壁曰:`此何乾坤时,读圣贤书,当守义全归'。与妾张氏相对就经。一女呱呱,亦死。韦布尽节,方知全躯保妻子者,不啻霄壤云。

马纯仁囊石沉河死

马纯仁字朴公,号范二;南京六合县人。曾祖在田,钜富而善;祖字衷一,邑庠生。父之骥,字德符,邑太学生;选县佐二、县丞一。母唐氏;纯仁,仲子也。幼颖慧。崇祯八年,督学金兰补弟子员,许以大成。

乙酉,薙发令下,纯仁方巾,两大袖囊石,不告妻子,竟赴龙津浮桥自沈于河,而尸僵立不移;时七月六日也,年甫二十岁。襟间大书曰:`与死乃心,宁死厥身;一时迂事,千古完人'。先是,同笔研生汪汇子百谷,才名并噪;国变,约同赴水,而汇竟负约。是岁,即举孝廉。己丑,登进士,选湖广承天府景陵知县。未几,纯仁显异,遂卒。纯仁妻侯氏无所出,家人欲令改适,屡欲自经;遂不敢逼。纯仁生平多著作,赴水前一夕,尽取文章诗稿焚之;益不欲以文传世云。康熙间,予在六合,邑人称之。访至其家,弟友仁,亦庠生,出见;述其事如此。纯仁既效屈平之节,生员袁逢盛等具呈在县以表其事。

附记:六合人语予曰:`当汪汇在湖广作令时,一日白昼间适坐公堂,忽见纯仁舁至,以大义责之曰:`汝不能死,已负约矣!复登新朝进士为官,何也'?汇大惊骇,逊谢不能出一语。遂得疾,未几死。妻年少,或萌他志;纯仁辄报梦,令守节,后以贞志闻。纯仁已为水神,凡舟子赛福,祷辄灵应;一时异之。

王域大骂不屈死
王域字元寿,号两瞻;松江华亭人。天启元年举人,以孝友闻。除宿州学正;流贼犯州,公固守以全。甲申十月,积官升建昌知府,加衔江西按察司副使。

大清兵陷抚州,公誓众固守。而城中有内应者,遂陷;益王出走。公被执至南昌,大骂不屈;送武昌,杀之。时八月二十日。同死者,江西布政夏万亨、分巡湖东道副使王养正、推官刘光浩等;与公六人传首江西,弃其尸城下。武昌人收而葬之沌砦河,题曰``六君子之墓''。公第三子钥走福京请恤,未覆;闽中陷,不果。

夏允彝赴池水死

夏允彝字彝仲,号缓公;松江华亭人。嘉善籍。通``尚书''。万历四十五年戊午举人,崇祯十年丁丑进士。宏光立,为吏部主事。

大清兵下松江,允彝避匿。其兄强之谒官,允彝潜赴池中死。同年陈子龙挽诗云:`志在``春秋''真不愧,行成忠孝更何疑'!

眭明永不屈死

睦公讳明永,字嵩年;镇江丹阳人。曾大父烨,官给事;父石,官太史。崇祯壬午举于乡,年六十矣;选华亭教谕。

乙酉八月三日城破,公书明伦堂曰:`明命其永,嵩祝何年?生黍祖父,死依圣贤'。遂自经,不死;出投泮水被执,以不屈而死。

公子本字允立,诸生。甲午春,坐同邑贺太仆王盛事,株连被系一夕死。

李待问、章简被杀

松江原任中书李待问、博罗知县章简,城破被杀。

顾所受投泮池死

顾所受字性之,号东吴;长洲人。六世祖巽、巽子曜、曜子余庆,相继举永乐甲辰、正统丙辰、成化〔口辰〕进士;故表其坊曰``三辰''云。公生而颖异,邑令江盈科称为国士。十一岁,补子弟员。崇祯十五年,流贼破袁州,犯吉安。时龙泉令刘汝谔请公为幕宾,画战守具甚悉,贼因去。

十七年,贼陷北京,公绝饮食。已而闻许琰死,曰:`吾今且可以无死,为琰传'。又一年,南京不守;公夜寝,微闻嗟叹声。明日,言笑如平常;谓子善曰:`吾以老诸生出入文庙者,五十余年矣。时事至此,恐委礼器于草莽也,将往观焉'。遂与其孙珩俱往。既至,作``卷堂''文以辞宣圣,且拜且泣。出庙门,令珩先归;遂投泮池死,尸直立不仆。士民吊者千余人,邑令遂宁李实为文祭之;言两日前君儒服手一状,阅之则言死节事也。闻者矜其志云。

徐怿自缢死

徐怿字瞻淇,常熟诸生;家徐市。闻县城陷,叹曰:`吾家世科第,竟无一义士耶'!薙发令至,服布袍别亲族;题壁曰:`不欲立名垂后代,但求靖节答先朝'。夜半,自缢。

诸侄守质,亦诸生;家南郭。母病不能迁;兵至,母与妹投井。守质曰:`吾不辱身'!与兵格斗死。

项志宁扼吭死

项志宁,常熟诸生;遁于野。方食饼,闻薙发令,饼半堕地,扼吭不食死。

董元哲痛哭死

常州诸生董元哲,痛哭死。崇祯末,元哲岁试,名居第一;盖文行兼优士也。

石生及卖扇欧姓投池死

常州石生及卖扇欧姓者,投西庙池中死。

卖柴乡民跃入河死

乡民卖柴入城;闻安抚使至,弃柴船,跃入文城坝南游河死。

蓄鸟叟缢死

五牧有蓄鸦鸟薛叟,以薙发自缢死。

卖面人自经死

玄妙观前卖面人,夫妇对经死。

许烈妇支解死

烈妇许氏,常熟诸生萧某妻、诸生许重光女。为兵所掠;至蠡口,见同掠有受污者,许氏大骂曰:`人何得狗彘偶'!兵怒,缚之桅,支解之;食其心。群视曰:`此烈妇也'!潜瘗其一股。

初乱时,女子义不受辱者,不能详记;此其最也。

张氏赋诗投江死

扬州既陷,一部将掠张氏至金陵,以珠玉、锦绣罗饰于前;张氏弗顾,悲泣不已。既而部将随豫王北上,张从之。出观音门将渡江,密以白绫二方可二尺许,楷书绝命词五首于上;乘隙投江。尸浮于高子港,为守汛者所获。其诗跋云:`广陵张氏题。有黄金二两,作葬身之费'。遍体索之,无有也;已而于鞋内得之,盖密缝于中者。众以此金易银,葬焉。康熙四年(乙巳)六月七日,予在六合,得阅其书并其事如此。其诗曰:`深闺日日绣鸾凰,忽被干戈出画堂;弱质难禁罹虎口,祗余魂梦绕家乡'。`绣鞋脱却换\{革翁\}靴,女扮男妆实可嗟;跨上玉鞍愁不稳,泪痕多似马蹄沙'!`江山更局听苍天,粉黛无辜实可怜!薄命红颜千载恨,一身何惜误芳年'!`翠翘惊跌久尘埋,车骑辚辚野堑来;离却故乡身死后,花枝移向对园栽'。`吩咐河神仔细收,碎环祝发付东流;已将薄命拚流水,身伴狼豺不自由'!

遁迹诸臣

``补遗''云:南京之变,遯而不与迎降者:尚书张有誉、陈盟、侍郎王心一、少卿张元始、光禄丞葛含声、给事蒋鸣玉、吴适、部属周之玙、黄衷赤、主簿陈济生等二十余人。

有誉号静涵,江阴人;素有品望。潜居青旸,不入城市。南京遘变,五月十八日抵家;有问之者,摇首涕泣而已。尚书印重六十两,挈归。陈明号雪滩,蜀人;道远不能归,潜居浙之台、处间。后寓迹嘉秀,僧服自晦。

起义诸臣

国家一统,自成直破京师,可谓强矣;大清兵战败之,其势为何如者。区区江左,为君为相,必如勾践、蠡、种卧薪尝胆,或可稍支岁月;即不然,大清师之下,御淮、救扬,死守金陵,诸镇犄角,亦庶幸延旦夕。乃大清兵未至,而君相各遁、将士逃降;大清之一统,指日可睹矣。至是而一、二士子率乡愚以抗方张之势,是以羊肉投虎、螳臂当车,虽乌合百万亦安用乎!然其志则可矜矣,勿以成败论可也。

阎、陈二公守江阴城

江阴以乙酉六月方知县至,下薙发令。闰六月朔,诸生许用大言于明伦堂曰:`头可断,发不可薙'!下午,北门乡兵奋袂而起,拘县官于宾馆;四城内外应者数万人,求发旧藏火药、器械,典吏陈明遇许之。随执守备陈端之,搜获在城奸细。以徽商邵康公娴武事,众拜为将;邵亦招兵自卫。旧都司周瑞龙船驻江口,约邵兵出东门,己从北门协剿。遇战,军竟无功。大清兵势日炽,各乡尽力攻杀,每献一级,城上给银四两。徽商程璧入城,尽出所储钱与明遇充饷;而自往田抚及吴总兵志葵乞援。田、吴不至;程亦不返,遂祝发为僧。

是时叛奴乘衅四起,救死不暇。大兵首掠西城,移至南关;邵康公往御,不克。大兵烧东城,乡兵死战,有兄弟杀骑将一人者。乡兵高瑞为大兵所缚,不屈死。周瑞龙船逃去,明遇遣人请旧典史阎应元为将;乡兵拥之入城,率众协守。大兵四散攻剿,乡兵远窜,无复来援者。大兵专意攻城,城中严御。外兵箭如雨,民以锅盖为蔽;以手接箭,日得三、四百枝。一人驾云梯独上,内用长枪拒之;将以口纳枪,奋身跃上。一童子力提而起,旁一人斩首,尸堕城下。又一将周身缚利刃,以大钉插城而上;内用锤击,毙之。大清骑日益依君山为营,瞰城虚实,为炮所中;乃移营去。居民黄明江素善弩,火镞发弩,中人面目,号叫而毙。陈端之子在狱,制木铳;铳类银鞘,从城上投下,火发铳裂,内藏铁鸟菱,触人立毙。应元复制铁挝,用绵绳系掷,着人即吊进城;又制火球、火箭之类,大兵畏之。

刘良佐降大清为上将,设牛皮帐攻城东北角;众索巨石投下,数百人皆死。良佐移营十方庵,令僧望城跪泣,陈说利害;众不听。良佐策马近城谕降,应元骂曰:`我一典史卑官,死何足惜!汝受朝廷封爵,今日反来侵逼,汝心何心'?良佐惭而去。明遇日坐卧城上,与民共甘苦,战则当先;明遇平心经理,民濒死无恨。一夕,风雨怒号,满城灯火不燃。忽有神光四起,大兵时见三绯衣在城指挥,其实无之;又见女将执旗指挥,亦实无之。

大兵破松江,贝勒率马、步二十余万尽来江上。缚吴志葵、黄蜚于十方庵,命作书招降;蜚曰:`我与城中无相识,何书为'!临城下,志葵劝众早降;蜚默然。应元叱曰:`汝不能斩将立功,一朝为所缚,自应速死'!志葵大泣拜谢。城下大炮日增,间五、六尺地一;其弹飞如雹。一人立城上,头随弹去而僵立不仆;又一人胸背洞穿,而直立如故。有将坐十方庵后,城上发炮忽转向营,立毙。

八月望,应元给钱与民赏月,携酒登城啸歌;许用作``五更曲'',命善讴歌唱。城下人悲怒相半,有激烈感慨者。二十一日午时,祥符寺后城倾,大兵从烟雨溷中潜渡,遂入城,民犹巷战;有韩姓格杀三人,乃自焚。男妇死者井中处处填满,孙郎中池及泮池叠尸数层。陈明遇合门投水死,阎应元投水被缚大骂死。明遇浙人,故长厚循史。应元北通州人,多胆略、有治才。甲申海寇顾三麻子直抵黄田港,应元率乡兵拒战,手射三人,应弦而倒。以功加都司衔,升广东簿;道阻未去。义民陆元同殉。训导冯厚培,金坛人;自经于明伦堂。中书戚勋字伯平,家青旸。入城协守;知力不支,大书于壁曰:`戚勋死此。勋之妻若女子、若媳死此'。阖室自焚。许用亦阖室自焚。黄明江,故善弹唱;城陷后,抱胡琴出城,人莫识为弩师也。

续记(难民口述)

崇祯二年(己巳),江阴城鸣;时吴鼎泰作令。及崇祯十五、六年,有阿□鸟在城中哀鸣一月,声如小儿啼;邑令闻之,叹曰:`此城将有兵难'!十七年(甲申)冬,五里亭出一虎,大如犊,而势猛捷。千人持械鸣金,逐至百丈地方欲过河,跳陷水中,不得跃起;适近渔舟,渔妇颇有胆,急持小刀乱斫,杀之。或谓虎属阴,兵兆也。乙酉五月,江阴知县林之骥,福建莆田人;不解江南语,众号``林木瓜''。时有红罗头兵千人过邑卖盐,百姓归启,盖银与爵也;争市之而兵不知。盖小盐包乃掠人者,兵欲劫城,而帅与林同乡,林出谒,宾主燕语,遂敛兵去。五月二十五日,林挂冠归。六月二十日,大清知县方亨到任。方令犹纱帽蓝袍,未改明服;年颇少,不携家属,止有家丁二十人。已而耆老八人入见,方令曰:`各县献册,江阴何以独无'?耆老出,令各图造献册于府;府献南京,已归顺矣。不数日,常州太守宗灏差四兵至,居于察院;方知县供奉甚虔。闰六月朔,方行香,诸生、耆老等从至文庙。众问曰:`今江阴已顺,想无事矣'?方曰:`止有薙发耳。所差四兵,为押薙故也'。众曰:`发何可发耶'!方曰:`此大清律,不可违'!遂回衙。适府中诏下;开读,有`留头不留发、留发不留头'一语。使吏役书示;至此,即投笔于地曰:`就死也罢'!方令欲笞之,共哗而出。北门少年素好拳勇,闻之遂起。乡兵各服册纸,以锦袍蒙外;应者万人,俱扬兵。行至县前,三铳一呐喊;至县后,亦如之。方令见事急,闭衙不出;移书宗太守云:`江阴已反,速下大兵来剿'。时城门已诘奸细;获书,众大怒,将使者脔之。遂入县,以夏手巾系方之颈,拽之曰:`汝欲生乎?死乎'?方曰:`一凭若等'。众使人守视;因曰`既已动手,今察院中有兵四人乃押薙头者,不如杀之'。于是千余人持枪进院;四兵发矢,连伤数人。众惧欲退;有壮者持刀拥进,四兵返走,一堕厕中、一匿厕上、一走夹墙、一跃屋上,悉被擒。四兵初至时,伪作满状、满语;至是,作苏语曰:`吾本苏人,非北人;乞饶性命'!众磔之。入县,携方令与木县丞出;木请曰:`愿降为明官'!遂囚于狱。此闰六月初二日事。

有守备陈端之,居江阴;众欲推为主,端之不遽从。甫出,众以枪刺之;端之跃屋上,趋出城,伏于豆内。次日上午,乡兵缚送城内,杀之;食其心。有一妻、二子、一女、一仆,欲尽杀之;其子叩首谢曰:`吾能制军器,幸贳我'!乃系狱。凡木炮、火球、火砖,俱陈子手造。木炮长二尺五寸、广数寸,置药于中,状如银鞘;攻城,即投下烧之。火砖广二、三寸许。有黄明江善作弩,弓长四尺、箭长一尺;以足踏上弦,百发百中。初,明末兵备曾化龙闻流寇亟,造见血封喉弩,藏三间屋。又张调鼎字太素,福建欧宁人;亦为兵备铸大炮及火药等。至是,发之。徽人程璧字昆玉,开当城中;出金为饷。又徽客邵康公年三十余,力敌五十人;推为将。宗太守得报,遣王良率兵三百人(大半居民)行至湖桥,遇江阴乡兵,被围;俱跪云`献刀',悉杀之。投尸河中积如木筏,南流数十里;经石幢,臭味难闻,撑出高桥外。王良,本江阴大盗而降者也。已而大兵至西门,江民出战,被杀五十人而兵不伤,遂退入城;大清兵又陆续至北门等处。时借靖江沙兵二千,每人犒千钱;与大兵战,杀伤五百人,沙兵扬帆去。程璧当靖江沙兵败归,恨之;劫掠一空。方令在狱,使作书退兵。及兵日进,夜半众拥入,赤身擒出,杀于堂上。

旧典史阎应元,善捕盗。大兵至,见林令归,挈家出城,寓祝塘。六月十五日,典史陈明遇遣邑人迎入城为主。应元曰:`若等能听我则可;不然,不能为若主也'。众从之。祝塘少年六百送应元入城,四门俱以张睢阳、城隍神坐月台上,舁之巡城,仪容甚盛;大清兵遥望,惊疑焉。将四门分堡而守;如南门堡内人,即守南门也。城门用大木塞断,一人守一堞;如战,则两人守之,昼夜轮换。十人,一面小旗、一铳;百人一面大旗、一红衣炮。初间夜两堞一灯,继而五堞一灯。初用烛照,继用油;又以饭和油,则风不动、油不拨。每堞上瓦四块,砖石一堆。大兵攻城,或以船及棺木与牛皮蔽体而进;城内以炮石、箭弩杂发,无不立碎。大兵乘城内食时,架云梯数十而上;凡城堞凹进而两对直守者见兵至,即发铳毙之。或城下攻,将长街沿石掷下,或以旗杆截段列钉于上投之,或以木炮掷出。兵见而异之,咸争夺;忽内机发,反射,皆死。故兵攻一城,无不流涕。阎应元昼夜不寝;夜巡城,见有睡者,以箭穿耳,军令肃然。城堞被炮击堕,即时修葺;外以铁门固蔽,内以棺木泥筑于中,又塞以木石。城下十堞一厂,日夕轮换居内安息、烧煮。公屋无用,则使瞽者毁拆砖瓦,传运不停。攻城日急,城中百计御之;用油与粪各半和煎,俟沸浇之,无不烧着。

闰六月二十四日,降将刘良佐在东城外,射进箭书劝降;其言曰:`传谕乡绅士庶人等知悉:照得本府原为安抚地方,况南北两直、山陕、河南、山东等处俱已剃发,惟尔江阴等处敢抗国令,何不顾身家性命耶?今本府奉旨平定江阴,大兵一、二日即到。尔等速薙发投顺,保全身家。本府访得该县程昆玉若系好人,尔等百姓即便具保,本府题叙管尔县;如有武职官员亦具保状,仍前题叙,照旧管事。本府不忍杀尔百姓,尔等系大清朝赤子,钱粮犹小,薙发为大。今秋成之时,尔等在乡者即便务农,在城者即便贸易;尔等及早投顺,本府断不动尔一丝一粒也。特谕'。二十五日,江阴通邑公议回书;其略曰:`江阴礼乐之邦,忠义素着。止以变革大故,随时从俗。方谓虽经易代,尚不改衣冠、文物之旧;岂意薙发一令,大拂人心。是以城乡老幼誓死不从,坚持不二。屡次兵临境上,胜败相持,皆系各乡镇勤王义师闻风赴斗;若城中大众齐心固守,并未尝轻敌也。今天下大势所争,不在一邑;苏、杭一带俱无定局,何必恋此一方,称兵不解!况既为义举,便当爱养百姓,收拾人心;何故屡杀烧毁,使天怒人怨,惨目痛心?为今之计,当速收兵,静听苏、杭大郡行止;苏、杭若行,何有江阴一邑!不然,纵百万临城,江阴死守之志已决,断不苟且求生也。谨与诸公约,总以苏、杭为率。从否唯命,余无所言'(或传诸生王华作)。

八月初六日,大清将服重甲逼身,系双刀、双斧及箭手执枪登城;毁雉堞,势甚勇猛。守者以棺木捍御,用枪刺之,俱折不能伤。或曰:`止有面可刺耳'。遂群刺其面,旁一人用钩枪挑其甲,乃仆棺中;又一人斩之,首重十八斤。持以示城下兵,皆跪求首级;将尸掷下,取去缝合,挂孝三日,设醮城下招魂。有六人服红箭衣跪拜,城上炮发,悉被伤害。刘良佐百计劝降,城中遣四人出议;良佐厚待之,约曰:`竖了``顺民''旗,薙头数十周行城上,即退兵矣'。一人先还报。三人后去,各送十金;及还白应元竟匿馈银事。次日,四城立``顺民''旗;忽城下呼曰:`昨先回一相公,尚未有银,特送至此'。城中闻之,疑三人为间,即杀之。且内有不愿降者,于是拔``顺民''旗,复竖``大明''旗,守之如故。攻城日急,内外杀伤相当;然江民昼夜拒战,亦甚疲矣。平日攻城,城坏,夜半修讫;城外惊以为神。是时城中益急,人人有必死之志。中秋,家家畅饮,如生祭然。

十九日,贝勒统兵至,巡城下者三;复登君山望之,谓左右曰:`此城舟形也,南首北尾;若攻南北,必不破。惟攻其中,则破矣'。收沿城民家锅铁,铸弹子重二十斤纳大炮中,用长竹笼盛炮。二十日,鼓吹前导,炮手披红;限三日内破城。在南门侧发炮,石泥俱碎;城崩,遂不可修。众困惫已甚,计无所出,待死而已。陈明遇不由阶级,从泥堆走上城;燃火发炮,击死大兵亦众。东、西、南三门坚守,而北门一堡人独少;贝勒舁大炮君山下,八月二十日二更后以大炮连击,城堕。复雨,遂左右两路发炮不止,多置铁石。惟中路一炮止有狼烟,不纳铁石,干响而不伤人;烟漫障天,咫尺不辨。守者谓炮声霹雳,兵难遽入;不知竟由中路黑烟内突入,跃马城上大射,守者溃散,城遂陷。须臾,大兵俱集;恐有伏,立视半日。至午后,城中大沸,遂下。有少年五百人相谓曰:`总是一死'!搏战于安利桥,杀伤甚众,力尽而败;河长三十余丈,积尸与桥齐。杀至夜,始收兵;尸骸满道,家无虚井。凡三日止。

十二、三岁童子不杀。有一四眼井,死者如市。一人趋下,后有壮者提起谓之曰:`让我先下'。壮者死而提起者反生,亦数也。观音寺后华严庵(即毛公祠),有三人避于韦驮头上天花板内;兵以枪刺之而去,得免。有一人趋佛殿隐处,已有一人在内;已而复一人至,三人同匿。至第三日,饿不可忍,一人有生米一掬,出而均分。时大雨,伸手受檐水和米而饮,得不死。有兄弟二人持枪隐衖中曲处,对立;兵不知直入,兄刺仆之,弟拽去。后兵继至,复如前法,凡杀十六人。适一兵继进,望见前兵被杀,走出;引十余人并进。遂走屋上,被执杀之。

阎应元在南门顾振东家,自刎。有黄尔锡与之善,见其佩刀一,右手刺心,仰死庭中。黄欲殓之,适兵至,弃而走;后稍定,觅其尸,失所在矣。邑人义之,为立庙祠焉。大清兵入,肃然起敬。

戚勋字石屏,青旸乡人。合门自焚;题壁曰:`大明中书舍人戚勋合门殉节处'。大清兵趋进,见纱帻红袍仰卧于地,盖灰影也。觉阴风凛烈,惧而返走。

程璧见势急,假乞师出城,故免。

有一家母子二人,城破,其子避于观音寺大钟内;上以绳悬系,下踏一横板。及夜走归,与母寝;未明,直趋入钟内。如此两日夜矣;至第三夜归,对母大哭:`吾今日死矣'!母问故?子曰:`前两夜,神至寺内点死者姓名,不及我;昨夕已呼我名在内矣,故知必死'。是夜,同母宿于家酣寝;及觉,已天明矣。踉跄欲趋寺,适遇大清兵,果被杀。

有一书吏,与孔县丞善。孔升湖州为知县,携吏为主文。在署中,梦神谓之曰:`汝是六万七千数内人,何不速归'!既觉,不解所谓。请归,孔留之;复梦亡祖语,亦然。会孔物故,星驰归。时江阴适起兵,将闭城矣,意欲出城;其父骂曰:`不孝子,去我而之外耶'?复欲送母出城,亦不听。吏以父母家口在城,不得已而止。后合门遇难,果符前梦。

``江阴野史''云:`有明之季,士林无羞恶之心:居高官、享重名,以蒙面乞怜为得意;而封疆大帅,无不反戈内向。独陈、阎二典史,乃于一城见义;向使守京口如是,则江南不至拱手献人矣。时为之语曰:``八十日戴发效忠,表太祖十七朝人物;六万人同心死义,存大明三百里江山'''。

次年正月朔,合城百姓无一人不披麻者,惨甚!及十一月十一日,江阴复纠众,不克而走。抚臣土国宝欲屠之,赖刘知县不从,指名擒获;一邑遂安。

当攻城急时,乡民为奴仆者勾结数百千人,问本主索文书;稍迟则杀之,焚其室庐。凡祝塘、琉璜、旸祁等处,莫不皆然;人人畏惧。旸祁徐亮工,崇祯庚辰钦赐进士;被奴杀死。妻与三子诸生,俱遇害;独季子汝聪遁免。未几事平,为主者亦多擒仆甘心焉。故令冯士仁,蜀人,寓居琉璜;乡兵起,有张姓以旧时被笞十五板,至持斧杀之。
侯峒曾守嘉定城
侯峒曾号广成,嘉定人;天启乙丑进士。历官至顺天府丞,未赴而京师陷。宏光立,召为左通政。峒曾见朝事乖谬,遂不赴。

闰六月,邑人起义,推为盟主;与子元演、元洁大治兵食。李成栋降大清为将,二十二日来争邑城;峒曾约进士黄淳耀共为死守,百计御之。攻城者多死,解而复围者再;死守十二日。七月初四日天忽大雨,平地积数尺;城一隅崩,成栋薄东门上。峒曾与二子犹指挥巷战,乡民争欲扶之去:峒曾曰:`吾既与城守,城亡与亡;去何之'?趋拜家庙,赴池死。元演、元洁相抱入水。成栋恨之,斩其首;题曰:`元凶以徇于城中'。有金生者,夜窃其首藏箧。峒曾之叔入收其尸,方殓,有哭声自外来者,则金生负箧至也。

举人张锡眉、龚用圆、龚用广、夏云蛟、唐全昌皆死。北门有贾朱某者,悉以家财佐军;城破,诱家人尽入一舟自沉。

峒曾弟岐曾坐藏陈子龙,执至官;大骂死。二仆亦骂不绝死。

岐曾大骂难,二仆亦骂更难;非烈丈夫而能如是乎!峒曾父子、兄弟、主仆之际,诚盛事矣!

黄淳耀、渊耀同守御嘉定城

黄淳耀字蕴生,号陶庵;崇祯壬午举人,癸未进士。弟渊耀,字伟恭。淳耀素与僧性如善,性亦非淳耀不交。

乙酉闰六月,大清兵围嘉定。淳耀居城中寺内;渊耀宿城堞,昼夜拒战。七月,势益急,淳耀语渊耀曰:`城破,驰信于我'。渊耀素文弱,城未破三日,两目忽突出,青铁色,状如睢阳;筋悉隆起。堞堕,实泥大袋中重数百斤,用长木肩之,登城修讫;众异焉。癸丑城破,趋报淳耀曰:`吾了纱帽事耳!子若何'?渊耀曰:`吾亦完秀才事,复何言'!淳耀整袍服、渊耀亦儒冠,同缢寺中。淳耀题壁曰:`宏光元年六月初四日,遗臣黄淳耀自裁于西城僧舍。呜呼!进不能宣力王朝,退不能洁身自隐;读书寡益,学道无成!耿耿不昧,此心而已'。

时避难者悉趋寺中,大清兵入寺,俱杀之;次及性如,性如曰:`吾闭关二十年矣'!兵问何人?性如告之;默然去。兵继至,问答如前。兵索宝,性如答以无;有兵曰:`大施主供养,岂无宝乎'?性如指地曰:`若此尸横满地,假有宝,亦逝矣;奈何坐守于此'!兵曰:`无宝,杀矣'。性如曰`杀则杀耳,宝终于无有;此亦前世孽,奈之何哉'!兵问惧否?性如曰:`亦安避之'!兵曰:`遍城皆尸,汝畏乎'?性如曰:`杀尚不畏,何况尸耶'。兵曰:`倒好!吾给一箭于汝,以悬寺门;自此,无有入之者矣'。乃去。兵果不入。及初七日,买二棺殓淳耀、渊耀,俱僵尸,绝无恶气。众尸秽腐难闻,裹以芦席焚之。

``编年''、``遗闻''诸书俱载渊耀为淳耀之兄。

朱集璜起兵守昆山

朱集璜字以发,昆山岁贡生。素有学行,为乡井所推。

南京既亡,邑人议拒守;而县丞阎茂才已遣使投诚,用为知县。乙酉六月,士民起义兵斩茂才,推旧将王佐才为兵主,迎旧令杨永言入城拒守。永言,河南人;善骑射。

抗御若干日,集璜协守甚力。七且初五日,大清兵至城下;初六日,炮击西城,溃而入。集璜被执,大骂不屈,见杀。

故将王公扬年七十,奋勇力战死。陶琰者字圭稚,诸生;以理学称。居鸡鸣塘,去城二十余里;方率乡兵三百人赴援,途中闻城破而溃。彷徨久之,乃曰:`以发其死矣,后之哉'?是夕拒户自缢(而他书则云自刎也)。原任狼山总兵王佐才为乱兵杀死,一家老幼俱殉。

金声、江天一起兵守绩溪

金声字正希,休宁人;崇祯戊辰进士,授编修。南京陷,与其门人江天一纠练义勇。以闰六月奉太祖高皇帝像,率士民拜哭,谋起兵。天一曰:`徽州诸邑皆有阻隘,独绩溪一面当孔道,其地平迤;宜筑关隘,重兵据之'。遂筑丛山关,屯军其中。于是宁国邱祖德、泾县尹民兴、徽州温璜、贵池吴应箕多应之,遂拔宁国、旌、泾诸县。

已而大兵攻绩溪,天一登陴守御,出迎战:杀伤相当,相持累月。降将张天禄于间道从新岭入,守岭者先溃。九月二十日,徽故御史黄澍诈称援兵入绩溪,声见其发未薙、衣冠如故,信之;城遂陷。声呼曰:`徽民之守,吾使之;第执吾去,勿残民'!天一追及之,同系至南京。洪承畴以有年谊,劝之曰:`多少臣子,今俱亡殁;公宜应天顺天,毋徒自苦'!声默然。诸生江天一大言曰:`流芳百世、遗臭万年,千古之下,在此一时;不可错过'!且骂承畴曰:`汝为天朝大臣,不能死而反诱人耶'!承畴命左右断其舌,天一骂不绝曰;遂杀之。声亦骂口:`崇祯是汝君,今何在?父在泉州,今何有?汝无君、无父,与禽兽何异'!承畴曰:`骂我极是,奈时不得已耳'。豫王亦欲留之;声大骂。承畴曰:`使为僧可乎'?声曰:`何以称忠臣'!复戟手大骂。承畴曰:`成彼之名'。遂杀之;仅截其喉而不断其颈,以示全尸。一僧收葬,木客出棺。举族殉义。学者称为正希先生。

天一,字文石。天一外同死者,有陈际遇、吴国祯、余元英。同起兵者,歙县诸生项远、洪士魁、副将罗腾蛟、闵士英、都司汪以玉;先后被执,不屈死。
附记:黟县、休宁,俱属徽州府。乙酉四月大清兵犹未至,邑之奴仆结十二寨,索家主文书;稍拂其意,遂焚杀之。皆云:`皇帝已换,家主亦应作仆事我辈矣'!主仆俱兄弟相称。时有嫁娶者,新人皆步行,竟无一人为僮仆。大约与江阴之变略同,而黟县更甚。延及休宁,休宁良家子闻之大惧,遂立七十二社;富贵者写粮银,保护地方。知县欧阳铉,江西人;邀邑绅饮,痛哭起义。金声、黄赓等亦举兵,而僮仆于是不敢动。

附张天禄袭休宁:天禄字桂吾,陕西榆林人;明将,降大清为总戎。乙酉九月二十二日,引兵下徽州,距休宁六十里。邑人闻之,一夕走空。十月朔,天禄至休宁,下令薙发。知县欧阳铉遁去,邑绅金声曰:`吾不出,恐百姓被害'!乃见天禄,执解南京。时隆武相黄道周遣王总戎率兵千六百人至徽州,义旅从者复数千;与天禄战,互有胜负。后王兵渐伤,乃去。盖王系杭州丝客,诣闽中,用贿得官,本非将材;部下有风、云、雷、雨副将四人,而乡兵多市人,又不习战,故败。十二日,又有郑兵驻休宁;而天禄驻徽州,于三十日午后率总戎贺某、卞某五人引兵万人疾行七十里,至休宁高堰驻营。时一鼓矣,寂然不扰;天禄设虎皮椅而坐。漏下四鼓,起马疾驰百里;晨至黄源小河地方。时为丙戌正月朔,郑兵以除夕醉卧,方起贺岁,竟不及备,而大兵已入营矣;遂大败。天禄追出岭,至徽州及常山等处,俱降之。初,天启、崇祯之际,徽州方一藻为陕西抚臣;时天禄为旗牌官。至是,天禄谒方夫人,遣兵护其家;军令肃然。

卢象观谋攻宜兴城

卢象观字幼哲,宜兴人;故山西宣大总督象升之弟也。崇祯癸未进士,授金溪知县;未仕,改中书舍人。

大兵南下,象观与宗室子遇西湖,相与痛哭;入于忠肃祠,誓同起兵。至茅山,推督部故将陈坦公为将。时大兵已踞宜兴城,而乡镇拥众悉归象观,遂得乌合数万,谋破城。自率前队先行,坦公大军继后。行三十里,至一镇;象观遣使觇城中,还报无兵,信之;竟不俟坦公,身率三十骑疾趋入城。不知大兵驻营城外平野,盖利于驰突也。守卒见象观至,登城射矢;外营大兵驰入,象观遇于曲巷,被围。坦公引兵半道,问留兵曰:`卢公安在'?兵曰:`适报城中无兵,轻骑先入矣'。坦公大惊曰:`书生不晓兵事,身为大帅,轻至此乎'!即选精骑三百赴援。见象观颊中二矢,危甚;杀退大兵,以己马授象观驰出城,自为拒后。初,乡兵甚盛,缘此失势。大兵遂长驱下乡,至中途过镇,坦公驻桥上;大兵骑至,坦公连杀七人,不得过桥;乃由他道填河而渡,乡兵不能御,悉溃。坦公立桥上,四面皆大兵,力战死;大兵脔之。象观之昆季子侄死者,凡四十五人。

大兵将捣卢氏故居,族人谋献象观以灭祸;闻之,遂率三百人入湖。时旧绅王其升、荆本彻拥众湖中,象观述前事;且云宜兴不足为,不如取湖州。于是王、荆率兵陆行,象观由水道。忽遇大兵,与战,众寡不敌;左右欲退,已扬帆矣。象观持刀断索曰:`誓死于此不去'!遂被杀。

卢象晋,象观弟也;不薙发,佯狂。已丑七月,捕置狱中;盖一门忠义云。

此宜兴人口述,而象晋则别闻也。 吴应箕起兵池州

吴应箕字次尾,号楼山;贵池人。父某隐者,家故习儒。少则猎冶诗、古文词,意气横厉,为复社领袖。崇祯壬午乡试副榜。时国事日棘,应箕好奇计划策,门杂进武夫介士,不复经生自处。

会世变,南土陆沈,忠义者起恢复;次尾曰:`吾有以自见矣'!署诗于壁曰:`韩亡子房奋,秦帝鲁连耻'!帅义儿门徒,纠合拳勇,与其曹攻郡城;不克,同事者遁。已独募士治众,以计复东流、建德。时歙州金声首倡义,奉隆武朔,擢都御史,得承制专拜牒;署应箕池州推官,监纪军事,势始彰。而声先败,失援;身练卒深山,飞檄郡治,语皆丑诋。怨家侦间百出,大兵逼战,溃;匿婺源祁门界。被获不屈;与官兵偕,辄踞上坐,亦敬重之、不加害。一卒以刀刃之,叱曰:`吾头岂汝可断耶'?乃伸颈,谓总兵黄某曰:`以此烦公!然毋去吾冠,将以见先朝于地下矣'。其就刑处,血迹洒之不去。头入国门如生,历三日不变;人咸异之。

黄毓祺起兵行塘

黄毓祺字介子,江阴人;贡生。好学,有盛名。其门人徐趋字佩玉,亦以气节着。江阴城守,毓祺与趋起兵行塘以应;鲁监国遥授兵部尚书,赐敕印。城破,亡命淮南。

明年冬,侦城中无备,率王春等十四人来袭;不克。趋被获,丁亥正月八日杀之。捕同谋者,毓祺远逸;收其二子大湛、大红,兄弟争为死。毓祺在泰州,寄书所善江纯一,犹用故时印;纯一之客持之。纯一惧,遂告变。毓祺见执,入江宁。狱成将刑,门人邓大临告之期;命取袭衣自敛,趺坐而逝,戮其尸。大临号泣,赎之归葬;变服为黄冠去。大临字起西。

毓祺在狱,自注``小游仙诗'';注毕,付邓起西云。其诗云:`大梦谁分丑与妍,白杨风起总茫然,瓠缘无用从人剖,膏为能明苦自煎。桂折兰摧诚短景,萧敷艾菀岂长年!归途不向虚无觅,朽骨徒为蔓草缠'。`为愁草盛稻苗稀,日暮徐看荷锸归。何处先生多好好,此中居士故非非。肥鱼不肯怜蛟瘦,饱鷃偏能笑鹤饥。请读蒙庄``齐物论'',横空白月冷侵衣(非非居士,王姓;予尝赠诗曰:``坐中上客有王生,问讯居然字子明;节度声名同豹变,相公事业与槐阴。出奇制胜三军服,守正推诚万物平;文武只今谁得似,因君遥见古人情''。朱梁王铁枪彦章、赵宋王文正旦皆字子明,故云)'。`散发人间汗漫游,风吹白日忽西流。淘沙惯吓斜飞燕,孔雀偏逢抵触牛。乡里小儿朝拜相,江湖暴客夜封侯。神仙赤舌如飞电,开口舒光笑不休(拜相者无救时之手、封侯者有洗村之军,皆小儿暴客也。淘沙之于飞燕、牯牛之于孔雀,有何相及而吓之、触之,真可付之一笑!吸风饮露之神人,岂争烟火食;采薇行歌之义士,岂争钜桥粟哉)'!`腹中书任他人晒,犊鼻裈从甚处悬?惟有丹心坚自爱,忍能凿破化为图(此立秋前一日七夕作也)'!`最无根蒂是人群,会合真成偶尔文;沙际惊鸥常泛泛,风前落叶自纷纷。掉头东海随烟雾,屈指西园散雨云。况复炎凉堪绝倒,灞陵愁杀故将军(宗门云:``如虫啮木,偶尔成文;人生无根蒂,会合亦如是''。杜工部诗:``巢父掉头不肯住,东将入海际烟雾''。风云散流,一别如雨;此五官中郎将所以有西园宾客之感也)'!`百年世事奕棋枰,冷眼常观局屡更;乌喙只堪同患难,龙颜难与共升平。遥空自有饥鹰击,古路曾无蛟兔横?为报韩卢并宋鹊,只今公等因当烹(渡江后,诗皆为守弁取去;止存``小游仙''数章。海陵狱中,多索书者。友人罗学制请予每章下作一小注;注毕,付门人邓起西。嗟乎!``游仙诗'',寓言也;即注,亦非的解。后世知有黄介子,庶几不昧我心)'!

附记:介子居江阴月成桥,素有文学;与常熟武举许彦达善。彦达与南通州监生薛继周第四子称莫逆。薛子亦诸生,居乡间湖荡桥。家赀三万,受隆武制,佩浙直军门印,得私署官属;伪为卜者。游通州,与彦达主于薛;薛生改称周相公。时江阴有徐摩者,字尔参;亦寄食焉。毓祺居久之,凡游击、参将自海上来见者虽满装,及入谒,俱青衣垂手,众莫之知。既而毓祺作一联,人颇疑之。将起义,遣徐摩往常熟钱谦益处提银五千,用巡抚印。摩又与徽州江某善。江嗜赌而贪利,素与大清兵往还。知毓祺事,谓返必挟重赀;发之,可得厚利。及至常熟,钱谦益心知事不密必败,遂却之。摩持空函还,江某诣营告变;遂执毓祺及薜生一门,解于南京部院,悉杀之。钱谦益以答书左袒,得免;然已用贿三十万矣。

王谋驱市人起义死

王谋字献之,号春台;无锡人。本杭姓,济之先生异母弟也。父讳州牧,高才博学;齎志以没。公居三,幼嗣南门王氏;遂因王姓。崇祯己卯、庚辰之际,训蒙洛社,移家居焉。每日晡,辄至先生斋中,清谈片晌而去。性敏而嗜饮,先生每以为狂。

丙戌仲冬,公将起义。时先生居江阴,又以平日性谨,故不敢告。公索精管辂术,卜之不吉;再卜兆益凶,大怒,掷课筒于地。次日,遂行;率乡兵万人,夜薄郡城,积苇焚门,将破。萧太守闻报,登城望之,俱白布裹首;乃曰:`贼夜至,必非明兵'。亲率师,启门出战。有家丁温台者,擒一人斩之,将首级飞掷空中。乡兵本乌合,俱卖菜儿,素不知兵;猝见首级飞堕,皆惊,悉溃走。公皮靴步行,道复滑;萧守驰骑突追,遂被获。廷见不跪,萧太守问何人?公曰:`先锋王谋也'。严刑拷问,公犹自侈其众!大骂不屈。萧守亦异之,因下狱。此十一月十一日事。久之,众囚越狱,公独不走;遂见杀。嗟乎!韦布之中,非无义士。惜乎!其子单寒,不克传之于世也。

予思当日驱市人围郡城,犹似螳臂当车、羊肉投虎;其迂戆固不足道!所难,濒死不屈、狱开不脱;虽古之烈士,何以加焉!

吴易起兵屯长白荡
吴易字日生,号朔清;吴江人,崇祯丁丑进士。祖邦祯,嘉靖癸丑进士;官太仆。

宏光立,见史可法于扬州;奇其才,题授职方主事,留之监军。乙酉,奉檄征饷未还而扬州失。六月,大兵徇吴江,县丞朱国佐以城降。诸生吴鉴欲起兵诛之,徒手入县庭,骂国佐;国佐执送苏州,杀于胥门学士街。易闻而哀之,率众擒国佐授鉴父汝延,令杀以祭鉴。遂起兵,仅得三十人;七日,众至三百并三十艘,居长白荡,出没五湖、三泖间。会松江盗首沈潘有徒千四百人,劫掠不常。诸绅患之,移书于易;易起兵往战,以计擒之。沈潘降,并其众,获艘七十。

居无何,易拜众曰:`镇江谍报:大清兵二千某时过此,愿邀之'!遂伪作农船,每里伏兵于湖滨,凡三十里。大清兵夜至,不疑;过半伏发,以长戈击之,应手而堕。其地左河、右湖,中岸颇高。大清兵止短刀,无舟不得近;大发矢,众以平基蔽之河侧;复以火器夹击,遂败。
丙戌元夕,入吴江;杀令及新科举人,库藏一空。镇将吴胜兆兵至,易已入湖,民尽走;大掠二日而还。

四月,胜兆复率众七千入吴江肆掠,舟重难行;胜兆令军中曰:`敢挈妇人者斩'!有一舟百五十人,悉沈诸湖。甫行,见岸上白衣四人,擒之使挽舟;问曰:`见罗头贼否'?曰:`见之'。问几何?曰:`三十号'。大清兵恃众不戒,呼曰:`蛮子速进'!俄,四人拔刀将舟中兵杀尽之。后兵见而疾追,遥望湖中泊舟,兵至即散,复返之;忽炮发,飞舸四集,矢炮突至,烟火迷天,咫尺莫辨。胜兆势急,弃舟走,兵亦委辎重而溃,凡斩将数人。胜兆大沮;谓`渡江以来,未有此败'!及还苏,惭忿不言;恨吴江民不救,屠之。已而率三千人复至吴江,经长桥,易用草人装兵,大清兵射之;易度箭尽,乃战,大败之。

抚臣土国宝忿易久为湖患,密遣苏人伪降易,推城以待。忽反兵相向,易急换舟;舟皆连系,乃入小舟;舟重,三十人尽覆。易泅水半里,其侄见水面红快鞋,谓易已死;以追兵急,不得遽挈,即系舟后。复行半里,始举视之,尚未死;倒倾血水,酌酒数大觥。乃曰:`今追者已退,吾兵尚有几何'?左右曰:`百人耳'。易曰:`速返追击!此去必大胜'。果败之,夺其辎重而还。

易有腹心某,居嘉善;六月,亲访之。其家仇人密白县令,令遣入猝取之,解于杭州杀焉。

附记:崇桢末有知一禅师,道行高迈;游燕都,士夫悉尊礼之。时易候选在京,闻而往谒,赠银二十两;师慨受不辞,晨夕辨议,相得甚欢。及甲申三月十七、八两日,贼攻城甚急,易叩吉凶;师曰:`止一路,无二路;公试自思。功名是分内带来,便可草草;若是朝廷所赐,则``忠孝''二字正在此际分明'。易闻言大悟,即欲祝发。师曰:`公向以贫衲削发被缁,蒙施多金;今日理应回敬'。遂取前银归赵,原封未启。心异之,乃下拜;师曰;`不须如此;去!去。我与汝从东便门走,送汝还乡;异日汝必尽忠王国。但闯贼非汝对头,今决无患'。次日出城,师送易归,竟不知所之。至丙戌六月被杀,果应``尽忠''之语云。

文秉通吴易不辨

文秉,长洲诸生;相国文震孟仲子。隐居山中,有告其与吴易通者。执至官,秉不辨;曰:`不敢辱吾父,愿速死'!遂见杀。

刘曙就义

刘曙字公旦,号稚圭;长洲人。崇祯癸未进士,授南昌知县。未赴而苏州破,避地邓尉山,未尝一至城市。南海诸生钦浩通款舟山,疏吴中忠义士二十三人,曙为首;游骑获其书上之,乃逮曙。曙膝不屈;诘曰:`反乎'?曰:`诚有之;愧事未成耳'。然曙实不识钦也。下狱八旬,与顾咸正、夏完淳从容就义死。

鲁之玙、韦武韬战死

苏州原任游击鲁之玙、韦武韬,以起兵俱战死。

麻三衡七家军

麻三衡字孟璇,宣城人;布政使溶之孙。生有异相,长好习武事;以诗酒自豪。既起兵,与旁近诸生吴太平、阮恒、阮善长、刘鼎甲、胡天球、冯百家号称七家军,皆诸生也。三衡驻兵稽亭,每战当先,舞大刀陷阵;人多畏之。后以众寡不敌,被获,杀于江宁;七家军皆死。

吴福之、徐安远起兵死

常州诸生吴福之起兵,约任源邃同就李总兵军,与之合;屡与大兵战。越三月,兵溃,投湖死。福之,闽中礼部尚书吴钟峦子。徐安远字世珍,武进人;入太湖从黄蜚兵,兵败被杀。

张龙文乡兵薄城

常州诸生张龙文率乡兵薄郡城,杀死。

钱柄破家起义

钱柄字仲驭,浙江嘉善人;相国士升之仲子也。崇祯丁丑进士,为吏部郎中。破家集义旅拒战,蹑于震泽兵,返战;旋溃,被杀。

徐石麒主盟

徐石麒字宝摩,号虞求;浙江嘉善人。天启壬戌进士,除工部营缮司主事。为权奄所恶,以新城侯王升坟价事,矫旨夺职。崇祯改元,补原官;历升通政司、刑部侍郎、尚书。公奏兵部尚书陈新甲陷边城四、陷腹城七十二、陷亲藩七,当斩;奏上,新甲弃市。时周延儒救解甚力,上不许;新甲之党皆大恨。而公复谳光禄少卿监军张若麒临敌先逃,总兵许定国失误军机、擒杀人民及兵部尚书丁启睿兵败鼠逃、弃去敕印;俱当斩。会礼科姜采、行人熊开元以言事忤旨,上震怒,下二臣狱;而刘宗周争之,并夺职。及二臣发西曹,公疏薄其罪;上怒,罢官。宏光立,起公左都御史;未至,转吏部尚书。公出户科陆朗、御史黄耳鼎为藩臬;有旨:`特留用'。朗与耳鼎遂疏讦公为吴昌时报复,又言公杀新甲以败款局。公乃历陈自有东事以来主款之误;且言`先帝之诛新甲也,曰陷我七亲藩。夫七藩之中,恭皇帝居一焉;皇上忘之乎'?因引疾乞休;命驰驿去。

明年南京陷,公遁于乡。镇将陈梧起义,迎之主盟。三塔之败,城将不守,自经死。其仆祖敏、李谨,皆从公自缢。公有二子;长尔谷,以松江事见杀。

``甲乙''诸书俱载徐锦,非李谨也。未知孰是?

徐尔谷被执无挠词

徐尔谷字似之,官生;石麒长子。受隆武令。松将吴胜兆反,长洲诸生戴务公实说之;远近响应。钱柄从兄旃,字彦林;夏允彝子完淳,字存古:与尔谷皆以胜兆事,被执。尔谷慷慨无挠词;审官曰:`汝父为忠臣,汝定为孝子'!三人同日受刑。旃妻徐氏、谷妻孙氏,各自沈殉其夫。
顾咸正坐吴胜兆事死

顾咸正字端木,号觙庵;昆山人,文康公之曾孙、咸建兄也。崇祯癸酉举人;十三年庚辰,以副榜除延安府推官。延安荒乱,咸正招抚有法。又奉檄追贼李明才等三百人,歼之;又招降\{犭回\}贼张成儒、丁世蕃等三百余人、庆阳土贼潘自安等千余人。于是延中稍宁。会孙传廷将出关,咸正上书,谓`出关安危系全秦,全秦安危系天下。军志曰:``兵无选锋曰北''。万一蹉跌,将不止三秦之忧'。不听。贼陷西安,咸正率三百人登陴,并弃甲去。贼执咸正,欲降之,不屈;乃拘之营中。吴三桂兵入秦,人多应之;韩城人推咸正为主,斩伪令王业昌。已而知为大清兵,遂入山中。

明年,以全发归南。会云间吴胜兆、陈子龙事败,录其党姓名,首及咸正;乃与同事四十余人并死。子天逵字大鸿,贡生;天遴字仲熊,诸生。皆以藏子龙故,亦死。当咸正解南京时,审官洪承畴问曰:`汝知史可法在乎'?咸正亦答曰:`汝知洪承畴死乎、不死乎'?承畴默然。
是时大清兵所过州县,从风而靡,长吏罕有殉城者。独公弟咸建字海石,号如心;崇祯癸未进士,除钱塘知县。以焚册故,被擒不屈,杀之。时盛暑,悬首镇海楼三日,无集蝇。杭人收而殡之,祀之土谷祠中。咸正季弟咸受,天启甲子举人;城破,亦死。仅存一孙晋谷,年五岁;得免。大鸿兄弟自谓`世受国恩;虽书生,义不苟活';故一门父子、五人同死国事。吴中人士,莫不悲之。

陈子龙誓众称监军

陈子龙字卧子,号海士;松江华亭人。幼颖异,工举业,兼治诗、古文词,以经世自在。立几社,与江右艾南英争名。登崇祯丁丑进士,授惠州推官。以招抚许都功,擢兵科给事中。南都立,以原官召用;疏请亲征,又上防守要害及备边三害,皆当时至计而莫之能用也。明年二月,以时事不可为,乞终养去。

南都不守,闰六月十日,松江兵起。子龙设太祖像誓众,称监军;邀致水师吴志葵等为城守计。闽中授兵部左侍郎。八月三日,李成栋破松江,子龙以祖母去,匿深山。无何,吴胜兆之事起,狱词连子龙;子龙亡命奔嘉定,匿侯岐曾仆刘驯家,已迁昆山顾天逵所。当事迹至嘉定,执岐曾;别遣兵围天逵家,遂获子龙,锁舟中。乘间,跃水中死,是月二十四日也;犹戮其尸。
杨廷枢坐门人戴之隽事死

杨廷枢字维斗,吴县人;诸生。以气质自任。崇祯庚午,举应天乡试第一。幼与同里徐汧交最善;乙酉夏,闻其殉难,即隐居邓尉山中。丁亥四月,松江总兵官吴胜兆叛;为之运筹者,乃廷枢门人戴之隽也。事败,词连廷枢;遂被执,系狱中。慨然曰:`予自幼读书,慕文信国之为人;今日之事,素志也'。五月朔,大帅会鞫于吴江泗洲寺。巡抚重其名,欲生之,命之薙头;廷枢曰:`砍头事小,薙头事大'!乃推出斩之。临刑,大声曰:`生为大明人\ldots{}\ldots{}'。刑者急挥刀,首堕地;复曰:`\ldots{}\ldots{}死为大明鬼'。监刑者为之咋舌,礼而殡之。

公在舟中,题书血衣并诗十二首以遗其孤曰:`苏州有明遗士杨廷枢,幼读圣贤之书,长怀忠孝之志;立身行己事,不愧于古人。积学高文,名常满乎宇内。为孝廉者一十五载、生世间者五十三年,作士林乡党之规模,庶几东京郭有道;负纲常名教之重任,愿为宋室文文山。惜时命之不犹,未登朝而食禄;值中原之多难,遂蒙祸以捐生。其年则丁亥之年、其月则孟夏之月,才隐遁于山阿,忽罹陷于罗网。时遭其变,命赋于天。虽云突如其来,吾已知之久矣。有妻费氏,吴江人,归予二十余载;有女观慧,适张氏,亦二十余春。大骂全真,不愧丈夫之气概;舍生就死,殊胜男子之须眉。一家视死如归,轰轰烈烈;举室成仁无愧,炳炳烺烺。生平所学,至此方为快然;千古为心,到底终须不殁。但因报国无能,怀忠未展;终是人臣未竟之志,尚辜累朝所受之恩!魂炯炯而升天,当为厉鬼;气英英而坠地,期待来生!舟中书此,不能尽言;留此血衣,以俟异日。愿我知己,面付遗孤。如痛父母,即思忠孝。垂殁之言,以此为诀。四月二十八日,舟中血书'。又云:`余自幼读书,慕文信国先生之为人;今日之事,乃其志也。四月二十四日被缚,饿五日未死、骂未杀,未知尚有几日未死!遍体受伤,十指俱损,而胸中浩然之气,正与信国燕市时无异;俯仰快然,可以无憾!觉人生读书,至此甚是得力;留此遗墨,以俟后人知之'。因舟中漫就一十二首;诗曰:`人生自古谁无死,留取丹心照汗青'';正气千秋应不散,于今重复有斯人'!`浩气凌空死不难,千年血泪未曾干;夜来星斗终天灿,一点忠魂在此间'。`社稷倾颓已二年,偷生视息又何颜!祗今浩气还天地,方信平生不苟然'。`叹息常山有舌锋,日星炯炯贯空中!子规啼血归来后,夜半声闻远寺钟'。`有妻慷慨死同归,有女坚贞志不移;不是一番同患难,谁知闺阁有奇儿'!`近来卖国尽须眉,断送河山更可悲!幸有一家如母女,纲常犹自赖维持'。六首佚。

黄赓为僧

黄赓,徽人;明季武状元也,与黄澍同族。有膂力,能运铁鞭二十四斤。大清兵至,率众固守徽州;身为前锋,所获甚众。后败,赓走闽。闽复陷,大清帅招之不从,乃削发为僧。宣城人语我曰:`黄赓率乡兵数十,十九战俱捷。后自宣城水东镇统众御大兵于港河,为徽宁界也。大清骑日益被围,赓举鞭忽折,重十二斤;乃易样鞭,重三十四斤。赓马见大清马,即跪;赓怒,鞭杀之,步战。举鞭一击,大清将以刀捍之;连击三鞭,捍之如前。赓乃走,取箭搭射,正中大清将左目;趋上,一鞭击死。然大兵甚盛,赓以众寡不敌,乃走;乡兵被杀遍野,惨不可言'。
许生伪试事败死

许某,武进诸生。顺治三年八月乡试近期,舟车云集;部院洪承畴疑之,每寓密令兵共居以侦察。时有一人,昼则闭户、夜半始出,佯云出恭;兵疑有奸,触之,其人怒而讧。兵握其首,乃未薙发者;解于承畴严讯,遂招多人。遣兵各寓搜获,有册藏金山下。无锡诸生华时亨,字仲通;亦有名。苏抚土国宝逮至,见时亨双瞽,释之。武进许生为首事人,亦见国宝;毅然曰:`老大人三年前,亦与生员一样。生员无他意,只是不忘大明耳。今生员含笑而去,不望含泪而归'。人咸壮之。解南京,杀焉。是案几杀千人,乡试因改期。至十月初七日下午,无锡始报新解元范龙。龙本王姓,字云生。

附记:邹来甫,无锡泰伯乡人;庠生。不薙发,隐居教授。至康熙初年,族绅邹式金被仇家讼陷藏来甫于家,遂逮来甫。郡守赵琪欲并究十年前总甲及馆主不举报罪;某费千金,家几破,解于兵备胡亶。亶本浙之仁和翰林,廉明仁恕,众号神君;呼来甫案前熟视,谒赵守押差曰:`此是薙不全,不是全不薙'。遂申文南京都院郎廷佐,乃免。夫以诸生全身二十载,亦异矣。

总论江南诸臣

东村老人曰:苏代有言:为人妻,则欲其许我也;为我妻,则欲其詈人也。每一王兴,有附而至荣者,即有拒而死烈者;生易而死实难!高帝斩丁公、艺祖褒韩通,所重固自有在;诸君子毋乃能所得乃重乎!

\hypertarget{header-n62}{%
\subsection{卷十·浙纪}\label{header-n62}}

潞王出降

大清顺治二年(乙酉)五月,豫王既定南都,分兵入浙;大帅,贝勒博洛也。时潞藩避杭州;六月,杭人拥戴之。贝勒以书招王,王度力不能拒,又不忍残民,遂身诣营,请勿杀害人民;贝勒许之,遂按兵入杭,市不易肆。后潞王北行,与宏光、王之明俱凶问。

附记:``编年''云:`兵至杭州,原任行人陆培缢死,某县知县梁于涘亦死'(一载自死)。

祁彪佳赴池水

贝勒既驻杭,遂散布官吏至浙东招抚,且令薙发;召乡绅谒见。原任苏松巡抚祁彪佳赴池水死。

祁公讳彪佳,字幼文,号世培;绍兴山阴人。父承\{火业\},知长洲县,有惠政。公年十七,举于乡。天启二年壬戌进士;授兴化府推官。郡兵以稽饷哗于藩司,公挺身往谕,刻期给饷;皆敛手不敢动。复令自推为首者,缚送藩司治之;众皆帖服。

崇祯四年,考满福建道御史。五年冬,上疏言:`凡大小文武、内外诸臣,皆使之各安其位,而后有以各尽其心;若越俎而问庖,即旷官而怠事。迩来六卿、九列之长,诘责时闻、引罪日见,因而有急遽周章,救过不遑之象。窃恐当事诸臣怵于严旨,冀以迎合揣摩,善保名位;则未得振励之效,反滋悠忽之图。臣所虑于大臣者,此也。人才有限,中下参半;非藉上感发其忠义,则无以鼓舞其功名。今司道有司,或``钦案''之累由人、或钱谷之输未至,降级住俸十居二、三。臣子精神、才具,必其稍有余地而后可以展布;若追于功令,必至苟且支吾,急切赴名之心,不胜其掩罪匿瑕之念。臣所虑于群臣者,此也。皇上闻鼙,而思将帅之臣;倘得真英雄,即推毂设坛,夫岂为过。但肮脏负俗,决不肯俯仰司马之门;若必依序循资,则虽冒滥之窦可清,似亦奖拔之术未尽。臣所虑于武臣者,此也。皇上深惩惰窳,特遣内臣;然必搜剔出于不意,奸弊乃可无遗。若抚、按之事,多令监视会同;则恐同□同功,反使互蒙互蔽。开水火之端,其患显;启交纳之渐,其患深。臣所虑于内臣者,此也'。时以为谠论。

寻巡抚苏、松诸府,所至省驺从,延问父老尽得其利病。豪右兼并,细民皆得控陈;一时权贵为之侧目。吴中无赖自署天罡党,凌轹小民;官治以法,则摊赃无辜,人益畏之。公至,捕其尤者四人,立磔于市。由是,群奸股栗。他若征解法、捐赎锾,为长洲置广役田,清吴县隐租以备荒、无锡役米以惠鲜,借华亭义米置上海役田。时粟贵,率二石得一亩;计三年子粒,即偿华亭之数。平漕兑,岁省四郡耗羡十余万金。吴人至今德之。

十五年,大清兵深入逼淮,道路阻绝。起公掌河南道;微服冒险,间行达京师。明年,佐大计;一主虚公,无敢以一钱及门者。会上命台省迁转,必历藩臬以考其才;面折选郎于朝,因疏列其事。于是御史蒋拱宸等群起攻之,事遂已;而公竟改南京畿道。

十七年甲申五月,公与史可法等决计定策;以公旧有威德于吴,命奉敕安抚。寻晋大理寺丞,即留为巡抚。首募技勇,设标营五营,各五百人;缘江要害,增置屯堡。公受事六阅月,开馆礼士、设笥受言,日夕拮据。又上疏,请除诏狱、缉事、廷杖诸弊政;为朝廷所忌,遂谢病。

乙酉夏,大清兵入浙,檄诸绅投揭;公闻,语夫人商氏曰:`此非辞命所能却。若身至杭,辞以疾,或得归耳'。阳为治装将行者;家人信之,不为意。闰六月六日(丙戌)夜分,潜出寓园外放生碣下,自投池中。书于几云:`某月日,已治棺寄蕺山戒珠寺,可即殓我'。其从容就义如此。后谥``忠敏''。

公生二子:长理孙,字奕庆;次班孙,字奕喜:皆有文誉。女德茞,字湘君;年十三、四,即韶慧绝人。其哭父诗有句云:`国耻臣心在,亲恩子报难'。时盛称之。

宏图不食死

原任大学士高宏图流寓绍兴城中。逃至野寺,不食死。

刘宗周绝粒死

刘宗周字启东,绍兴山阴人;学者称为念台先生。万历二十九年进士;三十三年,授行人。先后以母丧及养祖里居者十余年,始补原职。寻充册封益藩副使,归陈宗藩六议。四十一年,疏请修正学。明年,复谢病去。天启元年,起为礼部仪制司主事。劾魏中贤、客氏,坐夺俸半年。二年,迁光禄寺丞。三年,迁尚宝司少卿。寻改太仆寺,告归。四年,补右通政,力辞;上怒其矫情厌世,革为民。崇祯元年,召为应天府尹。二年、三年,以疾在告;复上言除诏狱、蠲新饷为``祈天永命''之本。上方忧旱斋居,责其不修实政,徒托空奏。公遂坚求去,许之。八月,召为工部左侍郎,日上言时政云云。上曰:`宗周素有清名,亦多直言。但大臣论事,宜体国度时,不当效小臣归过朝廷为名高'。会体仁捐俸市马,公言不敢怀利事君;得旨切责,遂引病求罢。既就道,闻大清兵自昌平深入,极论体仁大奸似忠、大佞似信,并及刑政舛谬数事。上怒,以为比私乱政,革为民。十四年,起吏部左侍郎;陈圣学三篇以切劘上躬,多见采纳。寻迁左都御史;请申饬宪纲、复书院社学、罢诏狱,从之。会当大计,发中书某为人行贿事,置之法;一时风纪肃然。已而京师复被围,行人熊开元劾奸辅误国;触上怒,下狱廷杖。公力争于朝,坐免官。

十七年南京再造,起原官。公力诋时政,马士英、刘泽清等欲杀之;遂力请致仕。

明年,大清兵至杭州;公与同郡祁彪幸约举事,不果。彪佳先死;公绝粒二旬,以六月八日(戊子)卒。有绝命诗曰:`留此旬日生,少存匡济志;决此一朝死,了我平生事。慷慨与从容,何难亦何易'!又示婿秦嗣瞻诗云:`信国不可为,偷生岂能久;止水与叠山,只争死先后。若云袁夏甫,时地皆非偶!得政而毙矣,庶几全所受'。

公以宿儒重望,为海内清流领袖;尝以出处卜国家治乱,而终以节见。悲夫!其论学也,以为`学者学为人而已;将学为人,必证其所以为人'。又作``纪过格''以相纠考。立古小学,每日生徒会讲其中。尝与高忠宪攀龙往复辨论,忠宪以为畏友。祁彪佳曰:`公之奏疏出,可废名臣奏议'。人以为知言。子名汋,遵遗命不以诗示人。

王毓蓍赴柳桥河死

王毓蓍字元祉。绍兴卫人,甫婚而父邻卒,经年不就内寝。为郡诸生,师事刘宗周。

乙酉六月,大清兵破杭州。时诸生无赖者群议犒师,毓蓍愤甚,榜其门曰:`不降者,会稽王毓蓍也'。众惧祸,阴去其榜。闻刘宗周举义,毓蓍喜;越数日事不就,乃为书告曰:`门生毓蓍已得死所;愿先生早自决,毋为王炎午所吊'!又作``愤时致命''篇,授其子复榜于孔庙。将赴泮池,池水浅,乃赴柳桥河死。时六月二十二日也。

潘集袖石沉河死

潘集字子翔,会稽布衣。性嗜酒;家贫不数得,时从友人索饮。既醉,或歌、或泣;人皆以狂少年目之。闻大清兵至,自誓必死;家人诧曰:`江南甚大,无死者;一布耳衣,何死为'?集曰:`苏州之役,吾父母俱死;于是吾三奔丧,竟不得一骸骨归。今腼颜为民,苟偷视息;死何以见先人于地下'!已闻毓蓍死,为文以哭之。出东门半里许,袖二石渡东桥下自沈死(或曰:此其意将以击当事之倡降者不得间,故死)。

周卜年跃入海死

周卜年字定夫,山阴人;周文节公族子也。家贫力学,年三十犹为布衣;滨海而居。闻王毓蓍、潘集死,曰:`二子死不先,卜年死不后也'。及传城中已薙发,逻骑四出;卜年仰天大呼曰:`天乎、天乎!余尚何以生乎'!遂肃衣冠趋出,自矶上跃入海中死。时闰六月初六日也。越三日,其妻溯流而号,求之不得;忽见一尸逆流东上,复于矶上兀然而止。就视之,则颜面如生;众嗟异之。

是日,越中师起,承制赠毓蓍翰林待诏,集与卜年教授、训导。而越人感三子之节,私谥毓蓍曰``正义先生''、潘集``成义先生''、卜年``全义先生''。

王思任请斩马士英疏

时马士英潜率所部奉宏光母后突至绍兴,绍兴士大夫犹未知宏光所在。原任九江佥事王思任因上疏太后,请斩马士英;曰:`战斗之气,必发于忠愤之心;忠愤之心,又发于廉耻之念。事至今日,人人无耻、在在不愤矣;所以然者,南都定位以来,从不曾真实讲求报雪也。主上宽仁有余,而刚断不足;心惑奸相马士英援立之功,将天下大计尽行交付。而士英公窃太阿,肆无忌惮,窥上之微而有以中之:上嗜饮,则进醁\{酉灵\};上悦色,则献淫妖;上喜音,则贡优鲍,上好玩,则奉古董。以为君逸臣劳,而以疆场担子尽推于史可法;又心忌其成功,绝不照应。每一出朝,招集亡赖,卖官鬻爵,攫尽金珠。而四方狐狗辈愿出其门下者,得一望见,费至百金;得一登簿,费一千金。以至文选、职方,乘机打劫;巡抚、总督,现在即题。其余编头修脚、服锦横行者,又不足数矣。所以然者,士英独掌朝纲、手握枢柄,知利而不知害、知存而不知亡,朝廷笃信之以至于斯也!兹事急矣,政本阁臣可以走乎?兵部尚书可以逃乎?不战不守而身拥重兵,口称护太后之驾;则圣驾独不当护耶?一味欺蒙,满口谎说:英雄所以解体,豪杰所以灰心也。及今犹可呼号泣召之际,太后宜速趣上照临出政,断绝酒色,卧薪尝胆;立斩士英之头,传示各省,以为误国欺君之戒。仍下哀痛罪己之诏,以昭悔悟:则四方之人心士气犹可复振,而战鼓可励、苞桑可固也'。

又上士英书

`阁下文采风流、才情义侠,职素钦慕。即当国破众疑之际,援立今上,以定时局;以为古之郭汾阳、今之于少保也。然而一立之后,阁下气骄腹满,政本自由、兵权独握,从不讲战守之事,只知贪黩之谋:酒色逢君,门墙固党。以致人心解体、士气不扬,叛兵至则束手无策,强敌来而先期已走;致令乘舆播迁,社稷邱墟。阁下谋国至此,即喙长三尺,亦何以自解?以职上计,莫若明水一盂,自刎以谢天下;则忠愤气节之士,尚尔相谅无他。若但求全首领,亦当立解枢机,授之才能清正大臣以召英雄豪杰,呼号惕励,犹可望幸中兴。如或逍遥湖上,潦倒烟霞,仍似贾似道之故辙;千古笑齿,已经冷绝。再不然,如伯嚭渡江,吾越乃报仇雪耻之国、非藏垢纳污之区也;职请先赴胥涛,乞素车白马以拒阁下:上干洪怒,死不赎辜。阁下以国法处之,则当束身以候缇骑;私法处之,则当引领以待锄麑'。士英愧愤,不能答。

以伯嚭比士英,最为酷肖。一疏、一书,痛快绝伦,足褫奸魄。王公以文采风流擅名当时,岂知其当大事而侃侃若此;可与黄、左两疏,鼎足千古。

鲁王监国

鲁王讳以海,高帝十世孙。父寿镛,世封于鲁;崇祯十五年,大清兵入兖州,城破自缢。以海年幼被执,三刃不中;乃舍去。十七年,嗣鲁王位。

大清顺治二年(南都称弘光元年),张国维为戎政尚书;会与马士英意见不合,遂请归里。五月,南都陷,国维在家闻变,收集义勇以待。六月,杭州拥戴潞王;潞王寻以城降,贝勒散布官吏于浙。招抚使至钱塘江上,原任山西佥事郑之尹子郑遵谦忿杀之。闻鲁王避难台州,而熊汝霖、孙嘉绩各起义余姚,遵谦遂与共谋立迎鲁王于台;适朱大典亦遣孙珏劝进。时张国维至台州,与陈函辉、宋之普、柯夏卿及陈遵谦、熊汝霖、孙嘉绩等合谋定议,拥戴鲁王监国;乙酉六月二十七日(戊寅)也。即日移绍兴,以国维为大学士。是时马士英逡巡浙东,闻鲁王监国,亦率所部至赤城,欲入朝。国维知之,首参其误国十大罪;士英惧,遂不敢入。起旧大学士方逢年入阁,之普、大典俱为大学士,函辉为兵部侍郎。而国维督师江上,调方国安守严州、张鹏翼守衢州。补御史陈潜夫原官,加太仆寺少卿;命监各藩镇马兵。

赐张国维尚方剑

七月,张国维复富阳。时兵马云集,各治一军,不相统一,部曲骚然。国维疏请于王,谓`克期会战,则彼出此入,我有休番之逸;而攻坚捣虚,人无接应之暇:此为胜算。必连诸帅之心化为一心,然后使人人之功罪视为一人之功罪'。鲁王加国维太傅,赐尚方剑以统诸军。

浙、闽水火

闽中隆武立,颁诏至越;越中求富贵者,争欲应之。鲁王不悦,下令欲返台州,士民惶惶。国维闻之,星驰至绍兴;上启监国曰:`国当大变,凡为高皇子孙臣庶,所当同心并力。成功之后,入关者王;监国退居藩服,体谊昭然。若以伦序、叔侄定分,在今日原未假易。且监国当人心奔散之日,鸠集为劳。一旦南拜正朔,恐鞭长不及;猝然有变,唇亡齿寒,悔莫可返!攀龙附凤,谁不欲之;此在他臣则可,在老臣则不可。臣老臣也,岂若朝秦暮楚之客哉'!疏出,于是文武诸臣议始定。然浙、闽遂成水火矣。

``遗闻''及诸书俱云`上疏隆武';独``甲乙史''云`启监国'。

封诸臣

十一月,进方国安为荆国公、张鹏翼为永丰伯、王之仁为武宁伯、郑遵谦为义兴伯、国维子世凤为平敌将军。

王之仁请战

浙东将士与大清兵跨江相距,自丙戌春屡战不胜,各营皆西望心碎。王之仁上疏鲁王曰:`事起日,人人有直取黄龙之志;乃一败后,遽欲以钱塘为鸿沟,天下事何忍言!臣为今日计,惟有前死一尺;愿以所隶沈船一战。今日死,犹战而死;他日即死,恐不能战也'!

申、酉间,武臣未建寸功,辄封侯伯,竭天下之饷以奉之。平日骄横,卑视朝廷;一闻警至,莫不逃降。``战''之一字,虽上趣之不能,而况自请乎!今读王公疏,凛凛有生气;洵推当时武将第一。视国安诸人,真奴隶之不如矣!

王之仁江心袭战

三月朔(戊申),大清兵驱船开堰入江。张国维敕各营守汛,命王之仁率水师从江心袭战。是日,东南风大起,之仁扬帆奋击之,碎舟无数;郑遵谦捞铁甲八百余副。国维督诸军渡浙江,大清兵为之少却。会隆武使陆清源齎诏至江犒师,时马士英依栖方国安,因唆国安斩之;且出檄数隆武罪。国维闻之,叹曰:`祸在此矣'!

士英既断送南朝,复离间闽、浙。小人之败坏国家事,可恨如此。然三月士英唆斩闽使、六月钱邦芑疏斩鲁使,两国相残,俱小人为之。武宁奋击之功,能不付之东流乎!

方国安夜走绍兴

五月,大清贝勒侦知浙东虚实,遂益兵北岸,以江涸可试马,用大炮击南营;适碎方兵内厨锅灶,国安叹曰:`此天夺我食也!我自归唐王耳'。谓大清兵势重莫可支,又私念隆武曾以手敕相招,入闽必大用;即不济,可便道入滇、黔。遂于五月二十七日(丙戌)夜,拔营至绍兴;率马、阮兵,以威劫鲁王而南行。

国安拥众十万,未战而逃;真可斩也。

浙师溃散
五月二十八日(丁酉),江上诸师闻方国安走,郑遵谦携赀入海,余皆溃散。惟王之仁一军尚在,将由江入海;国维与之仁议抽兵五千分守各营,之仁泣曰:`我两人心血,今日尽付东流。坏天下事者非他人,方荆国也。清兵数十万屯北岸,倏然而渡,孤军何以迎敌?吾兵有舟,可以入海;公无舟,速自为计'!国维不得已,乃振旅追扈鲁王。

大清兵渡钱塘江 六月初一日(丙子),大清兵渡钱塘江。

附记:五月中,贝勒闻报方兵诟詈,谕其下曰;`勿听!若有福,人自能过去;如无福,自然过去不得'。二十三、四日间,日夜炮声不绝。二十八、九日,潮不至;贝乃率兵拔船过三坝,坝大鸣。初一日,贝勒登坝渡江,勇甚;身被重甲,负矢三百,长戈、短刀俱备。及已渡,浙兵弃辎重无算;贝勒令诸军无掠,俟回时自有也。旧有谶云:`火烧六和塔,沙涨钱塘江'。崇祯九年,六和塔灾,中心悉烧去;止余四围不动,有若烟楼然。至是,而钱塘江又沙涨矣:前数可知。

鲁王遁入舟山

张国维追王至丰桥,方、马、阮兵断送过桥,桥石下旧刻大字二行云:``方马至此止,敌兵往前行''。国安、士英南行,决计执鲁王投降,为入关进身地。乃遣官守王;守者忽病,鲁王得脱。比及兵追至,王已登海船矣。后王遁入舟山。

余煌赴水

礼部尚书余煌大张朱示,尽启九门,放兵民出走;毕,遂正衣冠赴水死。

余煌,浙人。天启五年乙丑状元;以魏党,崇祯初罢,科名几秽。而其末节如此,亦可称也。

附记:余公微时,祈梦于于忠肃公庙。梦演剧,金鼓竞震,止一丑出场,以头撞公而觉,竟不解。及乙丑及第,有司送匾至,颜曰``乙丑状头'';始恍然前梦。

张国维赴园池死

鲁王既登海船,闻国维至黄石岩,因传命国维遏防四邑。国维至台州,无舟不能从王,遂回东阳治兵再举;时六月十八日也。二十五日,大清兵破义乌,亲众劝国维入山以图后举;国维叹曰:`误天下事者,文山、叠山也;一死而已'。二十六日,大清兵至七里寺;国维具衣冠,南向再拜曰:`臣力竭矣'!作绝命词三章。``自述''云:`艰难百战戴吾君,拒敌辞唐气励云;时去仍为朱氏鬼,精灵当傍孝陵坟'。``念母''里:`一瞑纤尘不挂胸,惟哀耋母暮途穷!仁人锡类能无意,存殁衔恩结草同'。``训子''里:`夙训诗书暂鼓钲,而今绝口莫谈兵!苍苍若肯施存恤,秉耒全身答所生'。

公字正庵,号玉笥;金华东阳人。天启二年壬戌进士,除番禺知县;以卓异,荐擢刑科给事中。历吏科,升礼科都给事中、太常寺少卿。崇祯七年,以都察院右佥都御史巡抚应天等处地方。先是,巡抚驻苏州,行文各属;间一出巡,驻句容。及公时,江北多事,往往出镇皖口。贼破庐、围桐,骎骎有南窥之势。而安庆素无兵(国初有军五千三百余,宣德中徙二千人于河间、怀来诸卫,后又以二千人运粮、三百人入南都,余丁不足待战),乃调吴淞戍卒及徽宁兵往。而海上复告警,公请益募兵千人,比楚、黔故事,留新饷给之;报可。复议增马、步二千人。于是,皖为重镇。上采科臣言,申饬江防;公请募卒千二百人,半戍浦口、半戍镇江,修繁昌、太湖、建平、六合、高淳诸城,建敌楼于芜湖。十二年,海寇焚崇明之东三沙,犯福山及陆座港口;公设伏擒其魁袁四、吴通州等。明年,升兵、工二部侍郎兼佥都御史,总督河道。会大盗李青山起,山左骚动;公擒之,东方遂宁。十五年,升兵部尚书。公视事,时则大清已入边七日矣;乃奏大调天下援师。大清兵深入至山东、淮北;癸未春,载护车牛、人口竟去。周延儒出视师,不能一有所创。公乃请告归;为言官所纠,缇骑逮下刑部狱。甲申春,特旨赦公,以前官督饷直、浙出都,而闻先帝之变。宏光立,授戎政尚书,加衔太子太傅;请建四辅以藩南京,未果行。复告归;而南京失国。会陈遵谦等迎立鲁王,召公直东阁;而以长子世凤代总军事,支撑江上者一年。丙戌六月,大清兵至绍兴,公急走归东阳。赴池中死。

附记:当缇骑逮公过苏州,苏人感公旧德,万众拥之,罗拜恸哭;宰牛羊生祭,且拜且哭,献酒。公从容语众曰:`予何德于汝!今兹行无伤也,有周相公手书在,非我不御罪也'。遂受而饮之。及北上,出书呈上;故得免。此苏人口述,以为公之快事。东阳、义乌,属金华府。

王之仁见杀

兴国公王之仁,载其妻妾并两子、幼女、诸孙等尽沈于蛟门下;捧敕印北面再拜,投之水。独至松江,峨冠登陆;百姓骇愕聚观。之仁从容入见内院洪承畴,自称`仁系前朝大帅,不肯身泛洪涛;愿来投见,死于明处'。承畴优接以礼;命薙发,不从。八月二十四日(丁酉),见杀。

闻之仁骂承畴曰:`昔先帝设三坛祭汝,殆祭狗乎'!

陈函辉死节

公讳函辉,字木叔,号寒山;台州临海人。崇祯甲戌进士,除靖江知县。先帝留心吏治,许科道官以风闻上言。而御史左光先按浙过吴,因劾浒墅钞关主事朱术\{土旬\}及公;公坐罢,里居。浙东监国,授公礼部侍郎。越州之亡,公赴水死。

公少年时,落笔妙天下,笑骂皆成文章;人争诵之。其交游,亦遍吴、越间。及为令,尝以县奉客,遂挂弹文以免。及公一死,海内翕然称其大节焉。今读其文,殆类有道者。其绝命词云:`余以五月晦日晚,从主上出亡。值乱兵间,道相失。还自僻路,徒步重茧;八阅月,始得抵台。城闭,痛哭入云峰山中。有池,可从灵均大夫之后。是夜,宿湛明大师禅房。漏下五鼓,作六言绝句十章。其一云:``生为大明之人,死作大明之鬼;笑指白云深处,萧然一无所累''!其二云:``子房始终为韩,木叔死生为鲁;赤松千古威名,黄蘗寸心独苦''!其三云:``父母恩无可报,妻儿面不能亲;落日樵夫河上,应怜故国孤臣''!其四云:``臣年五十有七,回头万事已毕;徒惭赤手擎天,惟见白虹贯日''。其五云:``去夏六月廿七,今岁六月初八;但严心内春秋,莫问人间花甲''!其六(阙)。其七云:``手着遗文千卷,尚存副在名山;正学焚书亦出,所南心史难删''。其八云:``慧业降生文人,此去不留只字;惟将子孝臣忠,贻与世间同志'':``今日为方正学,前身是寒山子;徒死尚多抱惭,请与同人证此'''!又遗友人书云:`辉死矣!季札之剑、孝标之书,皆诸先生心事也。或念辉生平忠悃,得存其遗孤、藏其遗骨、收其遗文,所谓埋我三年而化碧地下,必有以报诸公矣'!又自作祭文一、``埋骨记''一,从容笑语,扃户自经死。

``编年''载自经;而``启祯录''载赴水。故并志俟核。

陈潜夫阖室沉河

太仆寺少卿陈潜夫偕妻孟氏、妾孟氏夫妻姊妹联臂共沉河死。

陆培与潜夫皆杭人。诸生时,同盟相善;后有违言,遂相仇。寻俱入仕。大清兵至,潜夫死;培居家闻城降,即自缢。两人卒同殉国,人咸称之。

朱大典阖门焚死

朱大典,号未孩;浙江金华人。万历四十四年丙辰进士,历任巡抚凤阳、户部侍郎。及鲁王监国,加大学士衔。大清兵至金华,大典固守;攻月余不下。用红衣炮击破之,大典阖门纵火焚死。其子师郑邠,武进人;亦死。

张鹏翼见杀

总兵张鹏翼守衢州,标下副将秦应科等为大清内应;城破,鹏翼及乐安王、楚王、晋平王皆被杀。督学御史王景亮被执,不顺;遇害。

王瑞旃自缢

王瑞旃字圣木,温州永嘉县人。天启五年乙丑进士,原官兵部职方郎。大清兵陷温州,贝勒下擢用之令;乃集先世遗像,亲为题志,且拜且泣曰:`死见先帝,即归膝下耳'!遂与姻友会酌,悲歌尽欢。已而,入户缢殉。

邹钦尧赴江

邹钦尧字维则,永嘉人;郡庠生。大清朝总镇范某下令髡发,钦尧即赴江;流尸不可得。

叶尚高饮药痛骂死

叶尚高字向桠,永嘉人;邑庠生。大清兵入城,尚高被发佯狂;儒巾帛衣,截神祠本台为铎状,摇布狂言,惟``洪武圣训''四字朗朗彻人耳。上丁释奠,尚高冠进贤冠,倚庙柱肆詈当事;庭鞫,不跪。鞭箠血遍体,略无一语;惟呼`太祖高皇帝'而已。被创后,吟咏自若;和``正气歌'',有`未吞蒲酒心先醉,不浴兰汤骨已香'之句。饮药痛骂死。狱吏欲倒出尚高于窦,诸士拥圜扉,枕尺环哭,几噪;事闻,有司乃坏棘墙,舆尚高至宅殓焉。

钱肃乐入海

钱肃乐字希声,号虞孙;浙江宁波鄞人。崇祯丁丑进士;授太仓知州,尝兼摄崇明、昆山两邑事。年饥,薙山贫民相聚掠富家;公捕倡乱者,杖杀之,邑赖以安。壬午,入为刑部员外;寻丁艰。乙酉,南京破,因遂与郑遵谦、孙嘉绩、陈函辉等会师江干。浙直历授公佥院、副院、少司马,皆辞;戮力军中者一年。

丙戌,钱塘失守,公携家入海。闽中复授公副院;公至,则延平已破,复遁迹海岛中。

丁亥,郑彩治兵海上,福建起兵;公复以掌邦政召,乃与熊汝霖、马思理、沈宸荃、林垐、吴钟峦等协力任事。戊子,加阁衔。公见国势日蹙、藩镇骄悍,忧愤成疾,卒于海外之琅琦山。遗命以先朝员外冠服殓,故仍称员外云。

张名振题诗金山

大清顺治十一年(甲午)正月,海船数百溯流而上;十三日,抵镇江,泊金山:大帅张名振、刘孔昭及史某也。二十日,名振等白衣方巾登山,从者五百人。寺僧募化;名振曰:`大兵到此,秋毫不扰,得福多矣;尚思化乎'?僧曰:`此名山也'。名振助米十石、盐十担;且书簿云:`张某到此,大兵不得侵扰'。徘徊半日,乃下。次日,纱帻、青袍、角带复登山,向东南遥祭孝陵,泣下沾襟。设醮三日;题诗金山云:`十年横海一孤臣,佳气钟山望里真。鹑首义旗方出楚,燕云羽檄已通闽。王师枹鼓心肝噎,父老壶桨涕泪亲。南望孝陵兵缟素,会看大纛祃龙津'!前云:`予以接济秦藩,师泊金山,遥拜孝陵有感'。后云:`甲午年孟春月,定西侯张名振同诚意伯题并书'。越二日,掠辎重东下。二十三日,旌旗蔽江而下,炮声霹雳,人人有惧色。

四月初五日,海艘千数复上镇江,焚小闸。至仪真,索盐商金;弗与,遂焚六百艘而去。名振还师海岛。

是年病,遗言令以所部归监军张煌言,悉以后事畀之。论者谓陶谦之在豫州,不是过也。殁后,煌言为之葬于芦花澳。

附记:张名振舟师至泰兴,有李公仁者被掠,击柝二日,谓卒曰:`吾秀才也,不堪此役'!卒引入见名振;及辕门,有金字牌一面,上书`军令十条:一、劫掠子女者,立刻处斩。一、杀无辜百姓者,斩。一、见敌兵不杀而故纵之者,斩。\ldots{}\ldots{}'云云。进见名振,名振绿袍、戴丞相冠,年六十余;与刘孔昭同居一大舟。知李为庠士,命立语;问南都、镇江等处兵势若何?李迎其意曰:`大清师虽众,能战者少'。名振曰:`取天下当何如'?李曰:`老台台胸中盖已定矣,书生何知;且国家失已十载,何不直抵中原'!名振曰:`极有此意,但兵微将寡,不敢轻试其锋。虽不能恢复中原,而海中明朝依然如故'。语毕,泣下。名振问故;李曰:`思父母耳'。名振曰:`父子乃一生伦理、君臣实万世纲常,何必如此'!遂赠银五两、绢二匹,遣归。

大清部院郎廷佐致明帅张煌言书

钦命南京部院郎,致书于元老大君子阁下:

仆素性愚鲁,谬膺特简。自莅任以来,事无巨细,惟在安民。上天好生恶杀,则人何敢不畏鬼神,而忘自纵横、搅乱百姓也。尝有海上诸公归来如顾镇忠、王镇有才者,日久抵掌,因备悉大君子忠孝至性,出自天成。本标总兵黄鼎,亦津津道之不置。方知至人举动,别有苦心;与寻常山海辈借口起义者如较天壤,语难同日。景仰之私,非今伊始。目今新奉恩诏,为山海诸君子大开宏造:凡投诚文武官员,照原官题职;地方官,即为起文赴部推补实缺。天语煌煌,遐迩昭布,非敢谬言。倘邀天幸,大君子幡然改悟,不终有莘;自膺圣天子特达之知,轰轰烈烈,际会非常,开国奇勋,共襄大业:此其上也。如曰志癖孤忠、愿甘恬退,优然山中宰相,祖茔坟墓朝夕相依,骨肉至亲欢然团聚;出处既成,忠孝两全:此其次也。其或不然,即于归来之日,祝发陈词;仆代请作盛世散人,一瓢、一笠逍遥物外,遍选名胜以娱天年:又其次也。亦强日坐危舟,魂惊恶浪,处不成处、出不成出,既已非孝、亦难名忠;况且震听海岸,未免惊扰百姓,窃为大君子难闻者。仆率愚直之性、行简淡之词,屏去一切繁文套语。如逆闯之害,何以当仇?本朝之恩,何以当报?当仇者,不审天时,自甘扑灭;当报者,妄行恃险,自取沦亡。邪正之至理、兴衰之大数,有识者燎若观火;又何必烦词,取厌大君子之清听哉!

昔人有言:局内人明者自暗、局外人暗者自明。某以局外之观,略陈鄙意;不避嫌疑,倾心万里。终不敢效轻薄者以笔舌争长,不敢蹈骄矜者以高抗取罪;至诚之心,望祈同乐!其采听与否,惟大君子裁酌已耳。临楮神越。

张煌言复书

钦命赞理恢剿机务、察视浙直水陆兵马兼理粮饷兵部侍郎兼翰林院学士张,复书于辽阳世胄郎君执事前:

夫揣摩利钝、指画兴衰,庸夫听之,或为变色;而贞士则不然。其所持者天经地义,所图者国恨君仇,所期待者豪杰事功、圣贤学问;故每毡雪自甘、胆薪弥厉而卒以成功,古今以来,何可胜计?若仆者,将略原非所长;祗以读书知大义,痛愤国变,左袒一呼,甲盾山立。\{山厄\}\{山厄\}此志,济则显君之灵;不济,则全臣之节。遂不惜凭履风涛,纵横锋镝之下,迄今余一纪矣。同仇渐广,晚节弥坚;练兵海宇,祗为乘时。此何时也,两越失守,三楚露布以及八闽羽书,奚啻雷霆飞翰!仆因起而匡扶帝室,克复神州;此忠臣、义士得志之秋也。即不然,谢良、平竹帛,舍黄绮衣冠,一死靡他;岂谀词浮说,足以动其心哉!乃执事以书通,视仆仅为庸庸末流,可以利钝、兴衰夺者。譬诸虎仆戒途、雁奴守夜,既受其役而忘其衰;在执事固无足怪,仆闻之怒发冲冠!

虽然,执事固我明勋旧之裔、辽左死士之孤也。念祖宗之恩泽,当何如怨愤;思父母之患难,当何如动念!稍一转移,不失为中兴人物。执事谅非情薄者,敢附数行以闻。

张煌言临难赋绝命词

张煌言字元箸,号苍水;鄞县人。崇祯壬午,举于乡。鲁王监国,授翰林院编修。丙戌师溃,范海。己丑,从鲁王居健跳。庚寅,闽师溃,诸将以监国退舟山;张名振当国,召以所部入卫。时郑成功纵横海上,遥奉隆武为号;于监国则修寓公之敬而已。惟煌言以名振军为卫,成功因之加礼。煌言极推其忠,尝曰:`招讨始终为唐,真纯臣也'。成功亦言:`公始终为鲁,与吾岂异趋哉'!

迨后势孤力竭,与心腹十余人,将至普陀落伽山祝发为僧。内一人欲降大清,遂私见浙江赵部院。赵曰:`汝欲为官,必先建功为进身地'!其人以某日张煌言至普陀告,遂率师擒获。公方巾见赵,略叙寒温;盖赵曾入海与公会者,只论海中事,降公之意绝不谈。久之,赵始曰:`公若肯降,富贵功各可致'!公正色曰:`此等事讲他恁的,在小弟惟求速死而已'!赵知公意不回,遂馆公。疏闻,廷议有谓宜解京斩之者、有谓宜拘留本处者,又有谓优待以招后来者;久不决。部覆云:`解北,恐途中不测;拘留,虑祸根不除。不如杀之'。临刑时,挺立俟死;乃曰:`陈上交锋被获,死亦甘心;今如此,于心不服'!作绝命诗四章,众竞传之。方杀时,刀折为两,咸大异焉。

其诗曰:`义帜纵横二十年,岂知闽统属于阗!湘江只系严光鼎,震泽难回范蠡船!生比鸿毛犹买国,死留碧血欲支天。忠贞自是人臣事,何必千秋青史传'!`何事孤臣竟息机,暮戈不复挽斜晖?到来晚节惭松柏,此去清风笑蕨薇。双鬓难堪五岳往,一帆犹向十洲归。叠山迟死、文山早,青史他年任是非'!`揶揄一息尚图存,吞炭、吞毡可共论?复望臣靡兴夏祀,祗凭帝眷答商孙。衣冠犹带云霞色,旌旆仍留日月痕。赢得孤臣同硕果,也留正气在乾坤'。`不堪百折播孤臣,一望苍茫九死身;独挽龙髯空问鼎,姑留螳臂强当轮。谋同曹社非无鬼,哭向秦廷那有人!可是红羊刚换劫,黄云白草未曾春'?

附记:当郑成功趋围南京,张煌言一军抵芜湖,令甚严。一兵买面,价直四分,止与十钱。店主哄起白张,张问兵;曰:`诚有之;时无钱耳'。张曰:`汝食大粮,何云无钱'!将蓝旗投下,曰:`拿下去'!左右缚兵,兵问故;曰:`张爷令斩汝'!兵大惊曰:`吾罪岂至此乎,容吾回禀'!张曰:`吾有谕:在外即一钱亦斩,况四分乎'!遂斩之。诸军肃然,狄毫无犯。商舟数百随张,张俱给一小旗,白心、元色镶边,竖舟前。军士望见,即呼曰:`此船板张爷船也'!贾舟虽出入兵间,无不获全者。凡舟坏,俱禀张总管;故兵呼``船板''云。

沈廷扬殉节

沈廷扬字季明,崇明人。为人多智,好谈经济。崇祯中,以海运策干时见用,加光禄寺少卿。宏光立,命以原官督饷,馈江北诸军。疏请海运百艘,可改充水师;沿江招集简练,愿统之以成一军,为长江之卫。不报;但命运米十万饷吴三桂军。大兵下江南,廷扬航海入浙。鲁王监国,加户部侍郎兼右佥都御史,总督浙、直,令由海道以窥三吴。时田仰为相,忌之;乃至舟山依黄斌卿。

丁亥,松江吴胜兆将举事,送款舟山;廷扬曰:`事机不可坐失'!定西侯张名振慨然请行,邀为导;乃谓之曰:`兵至,必以崇明为驻札地'。至崇明,舟泊鹿港;五更,飓风大作,军士溺死过半,大兵岸上呼`薙发者不死'。名振与张煌言、冯京第杂降卒中逸去。廷扬叹曰:`风波如此,其天意邪!吾当以死报国。然死必有名'。乃呼游骑曰:`吾都御史,可解吾之南京'。时经略洪承畴与有旧,使人说之薙发。问谁使汝来?曰:`经略'。廷扬曰:`经略死松山之难久矣,安得有其人'!承畴知不可屈,遂与部下十二人同日被刑死。其亲兵六百人,斩于苏之娄门,无一降者;时比诸田横之士云。

\hypertarget{header-n67}{%
\subsection{卷十一·闽纪}\label{header-n67}}

唐王始末

大清顺治二年(乙酉)五月,大兵渡江,南都失守;镇江总兵官郑鸿逵、郑彩知势不可为,因撤师回闽。会唐王从河南来。王讳聿键,太祖九世孙。性率直,喜文翰,洒洒千言。初封南阳,以父夭,失爱于祖端王;两叔谋夺嫡,未得请名。及祖端王薨,守道陈奇瑜、知府王之柱始为请嗣,遂袭位。后以统兵勤王,擅离南阳;锢高墙。宏光立,赦出;避乱适浙。鸿逵因奉之南至福州,与福建巡抚张肯堂、巡按御史吴春枝、尚书黄道周、南安伯郑芝龙等会议,立王监国。时拥入者艳翊戴功,咸请正位。诸大臣多言`监国名正;出关尺寸,建号未迟'。芝龙意别有在,亦固争,以为不可;而侍郎李长倩有`急出关、缓正位,示监国无富天下心'之疏。惟郑鸿逵请正位;曰:`不正位,无以厌众心以杜后起'。遂定议:于闰六月十五日(乙未)奉王即皇帝位于福州。是日郊天,大风震起,拔木扬沙;及驾回宫,尚宝司卿坐马忽惊跃起,玉玺坠地,损其一角。人咸异之。改福州为天兴府,以布政司为大内。大赦,改元隆武;命颁诏于两浙、两粤。遥上福王尊号曰``圣安皇帝''。

附记:华廷献``闽事纪略''云:`闰六月,邑簿陈王道自京口来任,始知有五月十一日之事。时传郑鸿逵数乘过岭,中有真主。俄而,百官郊迎,闽中有大郑、二郑之目。鸿逵守金山,遇大兵而溃。会唐藩以恩诏出中都,闻变渡江,邂逅于京口;尘埃物色,引与俱东。王雅好图书,喜翰墨,有河间献王风。传檄手书,先及世系、后及时艰;一称张鲸渊先生、一称吴梅谷先生,千言洒洒。即监国位于省城;越旬而登极议起'。

文武诸臣

郑芝龙、郑鸿逵晋爵为侯,封郑芝豹为澄济伯、郑彩为永胜伯。设六部九卿,以张肯堂为吏部尚书、李长倩为户部尚书、曹学佺为礼部尚书、吴春枝为兵部尚书、周应期为刑部尚书、郑瑄为工部尚书、马思理为通政使、郑广英为锦衣卫都督。以天、建、延、兴四府为上游,汀、邵、漳、泉四府为下游,各设抚按。县升府、府升道、道转内卿;一命以上,咸与宠锡。于是敷求耆硕,起蒋德璟、黄景昉、黄道周、苏观生、何楷、陈洪谧、林欲楫、朱继祚、黄鸣俊皆为大学士,而苏观生最信任。又起曾樱、何吾驺、郭维经、叶廷桂、路振飞,以次至,皆入阁办事。其远不能至者,如王应熊、杨廷麟等,仅列其名:阁臣至三十余人。然不令票旨,俱闲无事;凡有批答,皆上亲为之。德璟、景昉、欲楫皆力疏辞,行人以死请,乃至。德璟陛见,以清屯、练军上请;上然之而不能行。改庶吉士为庶萃士,命苏观生主之,以招选贤才。

郑芝龙议战守

时内外文武济济,然兵饷、战守机宜俱郑芝龙为政;鸿逵、芝豹皆其弟也。故八闽,以郑氏为长城。芝龙开府于福州,坐见九乡,入不揖、出不送。集廷臣议战守,兵定二十万。自仙霞关而外,宜守者一百七十处;每处守兵多寡不等,约计十万。余十万今冬精练、明春出关,一出浙东、一出江西。统二十万,合八闽、两浙、两粤之饷计之,尚虞不足。

杀靖江王

粤西有靖江王者,名亨嘉;太祖甥朱文正裔。八月,称监国。隆武诏至,不受;举兵将东。广西巡抚瞿式耜知之,移书两广总制丁魁楚为备;又檄思恩参将陈邦傅防梧,再星檄调狼兵勿听调。靖江遣桂平道井济促式耜入,式耜不允;未几,靖江提兵至梧,命式耜易朝服朝,式耜不从;且以兵胁之,卒不可夺。靖江兵寻为丁兵战败,还桂。

时宣国公焦琏为粤西总镇杨国威旗鼓,式耜因密授计于琏,而邦傅亦应檄统兵并受令,遂擒靖江王及国威与吏科给事中顾奕等;械至福州,奉旨斩于市。以擒靖江功,封魁楚为伯,式耜兵部侍郎衔兼副都御史。

是时,浙东亦奉鲁藩监国。

郑芝龙议助饷

十月,闽饷不足,郑芝龙遣给事梁应奇入广督饷。应奇往督,因参迟误者数十人,俱奉旨提问;然有迟疑未提至者。潮州知府杨球欲入朝,闻旨,遂止粤界不敢入。芝龙又令抚按以下皆捐俸助饷;官助之外有绅助,绅助之外有大户助。又借征次年钱粮;又括府、县库贮存积银未解者,厘毫皆解。不足,又鬻官爵:部司银五百两,后减三百两;武札仅数十两或银数两。而之倡优厮隶,尽列衣冠;然无俸、无衙门,空衔而已。其黠者,倩轩盖、雇仆役,拜谒官府、鞭挞里邻。晋江令金允治莅讼,两造皆称官职,立而语;不服,则互殴于庭而不可制。谣曰:`敌兵如蟹,迟迟其来'!识者已知其必败也。

国家新造,当内抚百姓、外御疆场,或可稍延。乃助饷卖官,较士英当国为更甚焉;安得不偾乎?``易''言`负乘';信矣!
曾后入闽

郑芝龙所招关门兵,不过疲癃数百人耳。廷臣请出关者,章满公车;隆武每欲躬履行间,而芝龙但以缺饷为辞。会十月,曾后至,迎入宫。时胤嗣未育,或劝珍摄以俟来春;乃暂止。

初,隆武孤身南来,鸿逵以所掠美人十二献,随居官衙。至是,曾后至,遂大兴工作,扩构宫殿,卮匜之属皆用黄金。开织造府,造龙袍;后下体衣,皆织龙凤。然后性儆敏,颇知书,有贤能声。隆武召对奏事,后辄于屏后听之,共决进止;隆武颇严惮之。

郑森入侍

隆武尚未有嗣,郑芝龙乃令子郑森入侍;隆武赐国姓,改名成功。隆武每意有所向,成功辄先得之,以告芝龙;由是,廷臣无敢异同者,宰相半出门下。何楷与芝龙争朝班不合,乞归;中途盗截其耳,诏追贼不得。兵科给事中刘中藻亦以忤郑氏去。

有密告芝龙揽权者,隆武辄责芝龙。芝龙怒,佯欲谢事。隆武心知芝龙不可恃,无以制之;因复固留曰:`此非朕意,乃某人言也'。芝龙潜中伤之。于是左右无一同心,皆郑人矣。

筑坛遣将

廷臣屡请,命郑芝龙出关。芝龙亦知不出关无以压众心,因分兵为二,声言万人、实不满千;以郑鸿逵为大元帅出浙东,郑彩为副元帅出江西。隆武仿淮阴故事,筑坛郊拜而送之。二将既出关,疏称候饷,不行;逗遛月余。内催二将檄如雨;隆武下诏责曰:`倘畏缩不前,自有国法'!乃不得已,逾关行四、五百里,仍疏言饷绝,留住如故。

隆武之遣二将,犹思庙之遣李建泰也;二将军之逗遛关外,犹建泰之逡巡圻内也。

刑罚用舍

丙戌正月朔(乙酉)早朝,郑芝龙以手板掷蒋德璟,几伤。

邵武知府吴炆归;推官朱健以南安王入境,疑敌兵,移眷他驻:坐倡逃。建宁府建阳知县施◆为奸胥摘发,坐贪酷。俱骈斩市曹。漳州府龙溪知县谢泰宗以贪参,罚入千金。

杭严道龚可楷航海至闽,不用;有``呼尔蹴尔''之疏,终被贼杀死。而南来无赖之徒,争上疏谈兵,即得召对;片言合旨,赉宝锭、赐官爵。久之渐多,部曹几及千人;所赏,芝龙亦不应。

张肯堂请袭金陵

吏部尚书张肯堂,与吏部郎中赵玉成同籍江南。疏言:`臣等生长海滨,请以水师千人,从海道直抵君山,袭取金陵以迎;陛下陆行,期会于金陵'。隆武大喜,亟催芝龙大造艘;芝龙笑诺。会有上疏言水师诸臣宜留其家眷以防逃归者,事遂不果。

隆武驻建宁

隆武决意亲征;二月,驻建宁。楚抚何腾蛟、江右杨廷麟皆有疏迎隆武,隆武意欲往江右,犹豫未决。而芝龙以关门单弱,固请回省。省中人数万呼拥请还;不还,则绝天下望。因驻跸剑津。任兵部尚书吴春枝留守,晋大学士;辞不受,留驻浦城。

皇子诞生

六月,皇子诞生。群臣贺表,有``日月为明、止戈为武''语。隆武嗟异。大赦。覃恩,郑氏厮养俱得三代封诰;撰敕者、织轴者,日不暇给。

当是之时,兵羸饷绝,行止犹豫;召对会议,欠伸而已。当事无谈及兵事,举朝如梦如醉;不待识者而知其败坏矣。

杀鲁王使陈谦

都督陈谦奉鲁王使,与行人林垐至闽,趑趄不敢入。谦与芝龙有旧,先遣人问之;芝龙以书招之曰:`我在,无妨也'。乃与林垐入见,启函称``皇叔父''而不称``陛下'';隆武大怒,下廷议,二人俱下狱。郑芝龙上疏救之,不听。

陈谦者,武进人,旧镇金、衢。乙酉春,齎宏光诏,封郑芝龙为南安伯。比启读券,乃误书``南安''为``安南'';谦谓芝龙曰:`安南则兼两广,若南安仅一邑耳。请留券而易诏,更晋伯为侯'。郑芝龙大喜,厚赠而别;及半途而南京变。芝龙素德之,故有是救。时有钱邦芑者自请召对,言天下事,语未竟,中旨即擢为监察御史;实出芝龙门下。而与隆武亲,最蒙信任;密启隆武:`陈谦为鲁藩心腹,且与郑至交;不急除,恐有内患'!隆武信之。或以告芝龙;芝龙谓:`刑人必经其门,临期救之更便'。不意至夜半,内传片纸,别移谦他所斩之。芝龙急救,则已授首矣;伏尸而哭极哀。以千金、百布葬谦,为文以祭;中有`我虽不杀伯仁,伯仁为我而死'之句。

天下之势,当论其轻重、大小。七国时,势莫强于秦,苏季子合六国以拒之,得安者十五年;后秦日夜攻韩、魏而齐不救,及韩、魏亡而齐、楚亦遂随之矣。大清势重若泰山,即昔日之秦,不足以喻;而鲁之弱,犹未及韩、魏。隆武虽不悦,而``同舟''、``唇齿''之言,不可不思。姑大度优容,连兵共拒;俟势稍定,大小自分。不此之计而自相寻仇,则鲁必折入于大清,而闽之亡可立待矣。昔晋灭虢而虞亡,秦灭韩、魏而齐、楚亡,晋灭蜀汉而吴亡,八王自残而刘石强,元灭金而宋亡益速;古今之势,大可见矣。

华廷献论浙、闽事

华廷献云:`时东南民望,渐属鲁藩。画钱塘为界,烽火相望。自两都破竹,至此始阻;相距于七里滩者十余月。五月午日,至延平之顺昌县,遍访乡音,微闻有黄兵之说。会中州侯若孩携家往赣,询及世事,摇手蹙额谓:``此时宜枕戈待旦,戮力一心。乃处累卵之危,而修笔舌之怨;忘敷天之愤,而操同室之戈:吾其济乎''!时浙、闽瓯脱,自分彼此;宦两地者各不相安。朱大典以一旅处两大间,左右瞻顾。九江关外,声援既绝;钱塘兵力不支,时事难言之矣'!

进讲

命儒臣赖垓、陈燕翼进讲``易''之``元亨利贞''、``书''之``圣神文武''。圜桥肃穆,圣德诞敷;群臣表贺。

开科

六月,吴炳来自江西,单骑入关;命以布政提调棘闱,以编修刘以修为主考。是月,即开科,题``大学之道''三句;取中举人叶瓒等百余名,犹雍雍太平象也(``编年''载:万瓒解元)。时湖广武昌省郡多陷于大清,遂于衡州府乡试。表题拟上,视学行释奠礼成。

郑芝龙拜表即行

时六月,大清兵渡江,钱塘不守。郑芝龙微闻之,因疏称`海寇狎至;今三关饷取之臣、臣取之海,无海则无家,非往征不可'。拜表即行。隆武手敕留之曰:`先生稍迟,朕与先生同行'!中使奉敕至河,而芝龙飞帆已过延平矣。芝龙既去,守关将施福声言缺饷,尽撤兵还安平。
大清兵从容过岭

是时旧抚田兵及方兵、郑〔兵〕号三家兵,或离、或合,逶迤而南;或手不持铁,所至劫掠,或挟妇女。至山头,呼卢、浮白,漫衍岭界者四、五日。后关门无一守兵、亦无一敌兵,寂如也;如是者三日,始有大清骑二、三千从容过岭,分驰郡邑。然大清兵入闽,或由汀、或由福宁,俱走山谷间;道出不意,不必定走仙霞岭也。

马、阮、方、苏降

马士英、阮大铖等犹拥残兵数千,请入关;隆武以其罪大,不许。士英计穷,遁至台州山寺为僧;寻为大清将搜获。阮大铖迎降,贝勒令随内院办事。方逢年、方国安及刑部尚书苏壮,俱薙发投诚。

大清杀马、阮、方四人

八月二十四日,大清兵至顺昌,获隆武之龙扛;搜之,得马士英、阮大铖、方国安父子及方逢年连名``请驾出关为内应疏''在已降后。大铖方游山,闻信知不免,自投崖死;仍命戮尸。士英骈斩延平城下,家眷百余口悉给赐兵丁。时以周、马作联云:`周延儒字玉绳,先赐玉、后赐绳,绳系延儒颈,一同狐狗之毙;马士英号瑶草,家藏瑶、腹藏草,草贯士英皮,遂作犬羊之鞟'。

附记:华廷献云:`见剑浦城边一堆白骨,云是马士英、阮大铖、方国安父子。所惜者,方书田耳;比匪之伤,悔何及乎!时延、顺间以搜索龙扛,有破家陨命者。迨启扛中,得五人连名``请驾出关''一疏,查在降后;恶其反覆,故杀。嗟乎!卖国者何所逃乎!死一而已,有履刀锯而骨犹香、伏斧钺而血犹污者,岂非处死者异哉'(书田,逢年字也)!

隆武奔赣

隆武自芝龙去后,闻大清兵信念,遂决计幸赣。八月二十一日(甲午),启行;监军钱邦芑先期请清路,犹赫赫颐指属县。二十二日(乙未),驾至行宫,戎服金蟒。而上好书,虽崎岖军旅,犹载书数十车以从。二十四日(丁酉),抵顺昌;未发,巳刻警至,大清兵已及剑津毁关,且踵至。顷之,行宫数骑突出,云驾在内,从行者惟何吾驺、郭维经、朱继祚、黄鸣俊数人已;而何与郭亦散去。曾后肩舆河干,顾从官曰:`刘宫人有怀,好护持就道'。妃媵狂奔,有一舸而数人者、有一骑而三人者。

隆武遇害

大清兵过延平而东,独陈谦之子帅数百骑追驾,为父报仇。及汀州,时隆武将入赣,停一日晒龙凤衣;陈谦子适追至,遂及于难,并执曾后及从驾官朱继祚、黄鸣俊。械至福州,隆武、曾后遂遇害。朱继祚勒令致仕,旋为乱兵所杀;黄鸣俊许授五品官,以老疾辞免。

蒋德璟绝食死

大清别遣李成栋、韩固山略定兴、泉、汀、邵、漳州等处。九月初八日,大兵入泉州,德化知县陈光晋迎降。

先是,大学士蒋德璟见郑师逗遛,因自请行关确察情形,相机督战;隆武许之。比至,则疲兵弱卒、朽甲钝戈,一无可为。因叹息,告病去。户部尚书李长倩亦以饷不继,忧愤而死。提学御史毛协恭亦愤卒。迨泉州既降,德璟遂绝食死。

曹学佺、马思理自缢

十五日,大兵至汀州。十月十九日,入漳州。漳州道傅从龙、知府金丽泽以城降,皆仍旧职任事;不三日,乡兵起,杀从龙、丽泽。礼部尚书曹学佺、通政司马思理俱自经。

黄道周不屈

大学士黄道周愤师不前,因请以师相募兵江西;曰:`江西多臣子弟,愿招之效死军前'!隆武命芝龙助资,芝龙不与;隆武给空札百函而已。道周以札号召门下,得百人;居吉安,与杨廷麟、万元吉为呼应。出兵徽州,竟为大清所执。械送南京,道周绝粒,积十四日不死。大清内院洪承畴怜欲生之,道周不屈;承畴疏救曰:`道周清节夙学,负有重望。今罪在不赦;而臣察江南人情,无不怜悯、痛惜道周者。望皇上赦其重罪,待以不死'!不允。寻同中书赖雍、蔡继谨等俱杀于市。

郑为虹、黄大鹏喷血大骂
闽溃兵先奔者,于路间焚掠为食。至建宁,科臣黄大鹏、郑为虹闭城,发仓米、库银以犒赏,俱欢呼而去;一郡独全。八月十七日,大清兵将至浦,百姓议请出降,郑为虹不可;再请,为虹执不可。大清兵拥见贝勒,众迫跪;为虹不屈。贝勒嘉其节,不忍遽杀,且劝薙发;为虹曰:`负国不忠、辱先不孝,忠孝俱亏,我生何用!宁求速死,发不可薙也'!明日,复召见,责输饷;为虹曰:`清白吏,何处得金来'!百姓争欲代输赎其死;为虹曰:`民穷财尽,乌乎可'!因喷血大骂。贝勒怒,下令斩之。为虹大喊奋跃,夺刀自刺胸,不死;遂见杀。百姓为之立祀。黄大鹏同日殉难。为虹义仆陈龙与标下中军游击原任浦城千户张万明及子都司张翘鸾、都督洪祖烈,俱从死。

黄大鹏,福建建宁府建阳县人。崇祯丁丑进士;甲申,授衢州龙游令。乙酉,授金衢道。大清兵至,杭严道与按察司及建宁、浦城知县三人俱降。大清招抚衢州,谓大鹏:`鼎革之事,自古皆然。天下归大清今已八、九,岂衢之一郡能抗乎?不如早降'!大鹏不从。是时,衢州陆知府与各属县令悉望风投诚,龙游与浦城接壤,贝勒驻兵于此,诸降臣俱入跪见;独大鹏红袍纱帻,挺立众中。贝勒异之,问立者为谁?对曰:`前任龙游知县,今为金衢道黄大鹏是也'。贝勒曰:`汝何不跪'?大鹏肆骂,不拜。贝勒大怒,命割其舌;大鹏喷血连骂,触阶而死。浦城人立庙祀之。

傅冠不屈

公讳冠,字元父,号季庵;江西南昌府进贤人。天启壬戌进士,廷试第二人;授编修,纂修神、光两朝``实录''。丁忧归。起复,升侍读;历左春坊左中允充经筵日讲官、左谕德右庶子。戊辰、甲戌,分考礼闱;历国子监祭酒、詹事府少詹、礼部右侍郎兼侍读学士,掌詹事府事、仍掌翰林院事。上疏,言`欲施政令,必治精神;欲致功能,必集才力。欲精神之四周,当明体要;欲才力之毕出,当别流品'。因奏保元气、辨人才、正纪纲、信诏令四事;上嘉纳之。丁丑,升礼部尚书兼东阁大学士,入阁辨事。戊寅,以疾乞休;赐驰驿、金币归里。公在阁一季,多所献替;以``温室''之义不着之于书,世莫得而详焉。

南京登极,特旨存问。其明年,闯贼部将王体中犯江南,破进贤,杀公孙鼎干,掘公先墓;公奔入闽。闽中起公原官,已而辞任。及大清兵至,公走泰宁门人江亨龙家,为江之仇执之,以献于大清帅;不屈,杀之。公被执时,于石牛羊中作书,以骸骨托汀州士民,并述其奔窜、囚执之状甚详。所著有``宝纶楼集''若干卷。

郑芝豹闭城索饷

初,大清兵未至泉州,郑芝豹先至,闭城门大索饷;皆计乡绅家财勒取,不应即枭首。至缚亲家母于庭,抵暮得数万。又具火手五百,将尽焚一城中宫室;以饷未足,迟至明日。俄报固山兵将至,乃奔安平。

郑芝龙降大清

大清朝招抚福建者,为御史黄熙胤;福建晋江人,与郑芝龙同里。初,芝龙密遣使,微行通款。既而汀、漳皆降,惟芝龙尚保安平,军容烜赫;战舰齐备,炮声不绝,响震天地。以前遣之信未通,犹豫未敢迎入。自恃先撤关兵,无一矢相加,有大功;而两广属部下,若招,两广总督可得。贝勒知泉州乡绅郭必昌与芝龙最厚,因遣招之;芝龙曰:`我非不欲忠于大清,恐以立王为罪耳'。会大清韩固山逼安平,芝龙怒曰:`既招我,何相逼也'!贝勒闻之,乃责固山,令离安平三十里驻军;而遣内院二人持书至安平,书略曰:`吾所以重将军者,以将军能立唐藩也。人臣事主,苟有可为,必竭其力;力尽不胜天,则投明而事,乘时建不世之功,此豪杰事也。若将军不辅立,吾何用将军哉!且两粤未平,铸``闽广总督''印以相待。所以欲将军来见者,欲商地方人才故也'。芝龙得书,大悦。其子弟皆劝芝龙入海,曰:`鱼不可脱于渊'!不愿降。而芝龙田园遍闽、广,秉政以来,增置庄仓五百余所;驽马恋栈,不听子弟谏,遂进降表。过泉州,大张播告,夸投诚之勋;犹持贝勒书招摇,得官者就议价。十一月十五日,至福州谒见贝勒,握手甚欢,折箭为誓;命酒痛饮。饮三日,夜半,忽拔营起,遂挟之而北矣。从者五百人,皆别营不得见,亦不许通家信。芝龙对面作家书数封,皆嘱``无忘大清朝大恩''语。而谓贝勒曰:`北上面君,乃龙本愿。但子弟多不肖,今拥兵海上,倘有不测,奈何'?贝勒曰:`此与尔无与,亦非吾所虑也'。芝龙既行,郑彩、郑鸿逵、郑成功皆率所部入海,张肯堂、沈犹龙等亦往舟山依鲁王;芝豹独奉母居安平。芝龙至京陛见,奉朝请。郑彩、郑成功复入,杀掠漳、泉诸县,皆破之、汀、邵并乱;据建宁,闽邮为之阻。戊子夏,大清兵再入闽,破建宁,直抵漳、泉;郑兵皆遁入海,后为边患。

郑成功入镇江

郑成功原名森,芝龙第四子也;隆武养以为嗣,赐国姓,改名成功。大清顺治丙戌,芝龙降,羁置北京。成功率众入海,驻四明州。及闻芝龙被杀,遂引舟师抵浙,袭温、台四郡,马信等降。江南大震,将沿江数百里港门填塞,以通马路。成功驻台数月,忽去。戊戌,谋入南;启行发炮,飓风大作,坏舟千计,乃还。

十六年(己亥)五月十三日,成功率兵十万入寇;被甲能战者三万而已,余俱火兵。有一甲卒,即有五火卒随之。俱以布裹首,赤足;刀长六尺,或长枪、团牌。二十九日,经江阴;六月初一至初三日,蔽江而上。初八日,至丹徒。十三日,泊金山祭天,诸舟环集,旗盖、袍服俱用红,望之如火;十四日,祭地及山河江海诸神,色俱黑,望之如墨。十五日,先以吉服祭太祖、次以缟服祭先帝,俱用白色,望之如雪。祭毕,大呼高皇者三,将士及诸军俱泣下。

镇江至瓜洲江面十里,大清朝守臣用巨木筑长坝;截断江流;广三丈,覆以泥,可驰马。左右木栅,有穴可射。炮石盘铳,星列江心;用围尺大索牵接木坝两端,以拒海舟:凡费金钱百万。坝始成,被潮水涌涨,立刻冲断;南京部院郎廷佐亲出祭江,坝复成,设兵严守。操江蒋国柱、总兵管效忠、副总高谦协守镇江,又于谈家洲伏兵二千,列炮于上。新操江朱衣助六月十三曰到任,守瓜洲。十五日,海舟二千三百泊焦山,先遣四舟,外蒙白絮、内载乌泥,操舵数人扬帆而上。大清兵望见,大发炮石。海舟近坝,从容复下。大清兵注射,炮声昼夜不绝;有如轰雷,可闻三百里。凡发炮五日,不伤一艘。海舟既上复下,循环数次,一以诱大清炮矢;二以水兵藏内,近坝即入水砍断。十六日,度炮将尽,悉舟过镇江,莫有遏者。十七日,上瓜洲,从后寨杀入,大清兵出御;盖东门外有高岸,骑布列,郑兵立两旁水田中,斫马足,大败之。朱衣助坐北门察院,发令旗,求援淮抚亢得时。忽左右报曰:`海贼至矣'!语未毕,两人趋至,挟朱去。见成功,抚以善言;已而脱之。郑将刘某乘东门之胜,直追入瓜洲城,大杀;将沿江炮移向谈家洲击之,兵立札不定。有海兵二千,忽自江中浮上,持长刀乱斫;洲上兵走,海舟以千人追杀,复移洲炮击镇江。

镇江告急于南京。南京发洪承畴麾下罗将军铁骑千人赴援;其兵铁甲如雪,大言曰:`这些海贼,不彀吾杀'!欲入江剿绝。时苏、常四郡兵方畏敌如虎,见京军欲居前队,甚喜。常州王总镇、无锡守备张科、江阴守备施某、罗将军、管提督等兵共九队,凡万五千人,而马居半。罗兵第一队,管第二队;苏、常四府拈阄,常州士兵第八队,无锡、江阴两营从之。京军憍躁,急欲与战。而海舟忽上、忽下,大兵驻南则泊于北、驻北则泊于南,佯为畏避以诱之。大兵随走三日夜不息,露立江边甚疲。时既酷暑,又连日多雨;雨后蒸热,甲内尤不可忍。且大暑聚立如林,不敢出声,马亦张口喘息。城中百姓送饭江边,兵谢曰:`吾辈不下咽矣'!继送炒米,亦不能食。兵曰:`吾为兵已久,昔日曾作流贼,凡临阵时必先吃牛粉;盖用小牛炙干研末,佩于身间,临阵吃少许即不饿。今为将者不知此;且雨热劳饿,不食已两日矣'!时郑兵前队长枪,次团牌;第二阵倭铳。第一队五十人,前有五色旗一面领之。有滚被二人;滚被者,用一大棉被厚二寸,一人执之,双手有刀。如箭至,即张被遮候;箭过,即卷被持刀滚进,斫人马足。又有团牌二人。五十人内,此四人俱吃双粮。更有挨牌遮箭。前一队五色旗,第二队蜈蚣旗,第三队狼烟,第四队铳,第五队大刀末后。又另用一人敲鼓,头上插一旗。如鼓声缓,则兵行亦缓;鼓声急,则兵行亦急。然多步卒,大兵甚轻之。凡骑兵遇步卒,反退数丈,加鞭突前;敌阵稍动,即乘势杀入,步卒自相残陷,骑兵因而蹂躏:以此常胜。至是遇郑兵,亦用此法。大兵驰骑突前,郑兵严阵当之,屹然不动;俱以团牌自蔽,望之如堵。大兵三却三进,郑阵如山;遥见背后黑烟冉冉而起,欲却马再冲,而郑兵疾走如飞,突至马前杀人。其兵三人一伍,一兵执团牌蔽两人、一兵斫马、一兵砍人;甚锐,一刀挥铁甲、军马为两段。盖铸刀时,用铁匠百人挨递打,成此一刀;故锐特甚。然是时郑兵虽勇而大兵不遽退者,以管效忠立于次队,欲斩返却者耳。战良久,郑阵中一将举白旗一挥,兵即两开,如退避状;有走不及者,即伏于地。大兵望见,谓其将遁,可以乘势冲击,遂驰马突前;不虞郑阵中忽发一大炮,击死千余,余军惊溃。郑兵驰上,截前五队骑兵围之,大杀;罗部下白先锋、郎部下王先锋殁于阵。管效忠多备战马,刀斫至,急避之。马头落,效忠跃上他马;须臾,马头三落,效忠三跃以避。郑将见其勇健绝伦,欲生擒之,故免;败走银山,追兵至,乃走山上。久之冲下,郑兵不动。俱铁甲胄、铁面头子,止露两足;用长刀砍骑,锐不可当。射中其足,则拔箭更战;大兵遂败。二十二日,效忠遥语郑曰:`从来止有马上皇帝,岂有水中皇帝乎?上来决战'!顷之,有两舟渡兵二千,结营于杨篷山之菜园;效忠麾下勇将王大听率兵出战。郑将周都督立阵前,高声问曰:`汝得非管效忠乎?何不速降'!王不应,即发一矢中其趾。周方拔矢,已连中其趾者三;周怒,持刀直前砍杀王,冲入阵。时郑将列一阵;效忠望见,谓麾下曰:`此八卦阵也。生门向江,宜从此攻入,开门而出'。及入,即变为长蛇阵,击首尾应、击尾首应,遂围。效忠见不利,向执旗官手中取旗,自负而返;兵见之,俱退走。郑兵追杀,效忠部下仅存三百人。效忠驰至城濠,郑兵飞走随至,诸军皆散;效忠出兵四千,止存百四十人。叹曰:`吾自满洲入中国,身经十七战,未有此一阵死战者'!常州王镇兵三百,存三十七人;高谦五百兵,存八十骑:入镇江,登城闭守。效忠走南京。而蒋国柱走丹阳,百姓恐追至,闭门不纳;馁其腹,驰至常州,已夜半矣。呼城,门者报太守赵琪;琪不信,曰:`斯时宁有都爷至耶'?令王总镇登城望之,始启入。国柱疲甚,不俟卧具,即寝于门。

镇江守将高谦与太守戴可立列炮城上,郑将马信驰城下大呼曰:`速速献城,迟则屠矣!今外兵已杀尽;汝等不信,请观杨篷山'!守者大惧;有民棍郝十应曰:`候议定出降,明午会'。信乃去。时高谦、戴可立与乡宦笪重光、杨鼎、陈干、王鼎纪俱在城上,商之;纪齿爵俱尊,对可立曰:`老公祖,亦随机可也'!当议降时,笪重光与张九微恸哭力争不得,乃遁去)。可立泣一夜,撤守城兵。次日,率二十人及百姓五十人出城,行至桥上,各将满帽投河中,截辫发;入见。成功问曰:`汝是戴太守乎'?曰:`然'。仍命为太守。又谓百姓曰:`若辈苦十六年矣'!郝十曰:`镇江须有守兵方好;不然,恐后日兵去,百姓不好'。成功怒,叱缚之;已而得释。镇江城共三千七百垛,成功发兵三千七百人登守;旗帜五色,纷耀夺目。成功服葛布箭衣,有暗龙二条;边帽、红靴。从者二人,织锦暗龙纱衣;一人须发皓白,张紫盖。有兵五百,拥卫前后。成功封福建延平王,军中称``王爷''。二十四日,舟中送纱帽三顶入城:高谦挂将军印,银五百两;戴可立三百两,知县任体坤一百两。二十五日,诸官入见,俱去辫;兵民解发,戴网巾、棕帽。下午,市肆大开。二十六日,赏赉从征将士。二十七日,促装;二十八日,启行往南京,留兵四千守镇江城。

附记:六月二十七日,常州释囚。七月初一日,无锡知县王之蔚宰豕,做馒头三百斤、半斤一枚,分赉衙役。扬州盐运使,十六日遁,百姓俱走;城遂空。

泰州人刘坤率党千六百人、粮万石降成功,口称扶助先帝,审系假冒;立唤该地里总杨芳、许秀等供称,实系盐盗假冒明主,作耗地方。立时绑出,号令江口;又差铁甲兵一千前去该地抄没,不容假冒。又发谕该地方,张挂告示一道:`尔百姓等各安生理,毋听讹言煽惑。如再仍前,许地方童叟百姓据实呈报,剿拿扫除。再敢计诱耸动,只字虚报,罪当反坐!特谕'。

郎廷佐大败郑成功

郎廷佐闻郑兵将至,将城外屋悉行烧拆。近城十里居民,俱令入城。大开水西、旱西两门,使百姓置柴;限五日,如城外不卖及卖不完者,俱火之。郑兵至,结营白土山,距南京仪凤门七里。廷佐敛兵闭守;满将哈哈木疑民有异志,郎保无他。令民闭户,鸡犬无声。郑兵围困不攻,城中米七两一石;百姓不敢街上行籴,有饿死室中者。惟柴不甚贵,以烧台凳故也。七月,南京被围,廷佐檄松江总兵马进宝(进宝于顺治十四年改名逢知)及崇明提督梁化凤入援;进宝不奉檄,化凤以四千人至。初,成功入南,进宝递书降。进宝初为闯将,号马铁扛;及居松江,残忍好杀,奇富。化凤字卿天,陕西西安府长安县人;顺治丙戌武进士。成功南来,化凤亦伪降;与马信拜结兄弟,祭誓天地。至是入京,信独不疑。以兵逼城,城内寂然不动,郑兵益懈;谓功在旦夕,甚轻之。

七月十二日,总兵余士信约诸将二十三日计议,二十五、六日破城。有福建林某,入海已十六年,为管甲吏,知郑虚实;从破瓜州时,于仪真淫掠,郑笞二十,以是衔恨。至此闻密计,遂走;夜缒入城,见廷佐曰:`逾三日,城必破矣。明日二十三日为成功生日,诸将卸甲饮酒;乘其不备,可破也'!某处假营、某处实营,一一详报。廷佐令守城军十人留一,余俱下城归营。南京有神策门,向久砌塞;是夜掘开,止留外砖一层。沿城荻深数尺,马信等竟不知内有突门;忽炮大发,梁化凤、哈哈木、管效忠各引精骑乘炮势冲出,信兵大乱,大清将分路袭杀。余士信与先锋甘辉方演戏;得报,被甲而出。战良久,哈兵稍却;廷佐登城见之,惊曰:`如何退了'!复发一队从小东门出,掩郑之后;猝不及备,遂大伤。然军令严,主将不奔,军皆死战。既而甘辉身中三十余矢,力不能支,乃走;兵始走。时海舟泊江边,距城二十余里;廷佐先令军士诡装百装,载柴、酒、米、肉,日与海舟贸易,以观动静。初犹远舟,后渐相昵不疑;遂知火药所在,密以硝黄实酒瓶中,近舟发之,焚其四艘,火药俱尽。成功大惊,谓有奸谋,乃放舟南下。岸兵败走二十里至白土山,欲觅舟,舟已开矣;群趋山上。大兵追至,郑兵杀下。久,哈哈木密从山顶上驰下,廷佐登高远望旗色,喜曰:`吾家兵上山,胜矣'!须臾,郑兵大败,走江边,以无舟,勇锐多投江死者;举其甲,重四十斤。检尸,得四千五百人,长发者千五百余人:时七月二十四日也。

时城中居民已闭门一月,忽闻街上大呼曰:`海贼俱已杀去,汝等百姓俱开了门罢'!门始开。廷佐出示各郡县云:本部院同征黔凯旋大将军噶罗统满汉兵哈、管及水师提督梁,于七月二十三日杀贼,生擒伪都统余士信,箭射死马信,杀贼几许、箭死几许及渰死不计其数云。初,六月下旬,郑兵至天长,知县出迎。七月初八日,张煌言率舟师抵九江等处,所过郡邑多附之。蒋国柱奏曰:`凡献册者十九县,请旨定夺'!上批云:`此非百姓之罪,乃汝失守封疆之罪也;来京听审'。廷佐奏捷;上批云:`海寇犯边,尔乃一城之隔,致之死地;此乃郎卿第一功臣也。沿江所失封疆,俱免屠戮;府、县官,更加培植'。百姓闻之,始安。

亢得时号佑五,山西崞县人;淮安总漕。率兵救镇江,中途遇兵而败,投河死:此七月二十一也。众疑遁去,忽于砚下得书知之;以绳系手于船,故得尸焉。朱衣助既脱,奏曰:`臣履任四日,而贼即至。求救于蒋国柱,国柱不发一兵;求于管效忠,效忠逗遛不进。贼既至,不战而逃;贼既退,不抚而抢'。遂于九月十三日,逮国柱、效忠。而高谦从郑入海,籍其家。

附记:马进宝字惟善;辽东籍,山西隰州人。镇守松江,贪淫酷虐,士民无不被其毒者。有妾八十人,每在阄签而卧。其母劝减妾,进宝语妾:`愿去者拈签'。拈者颇众。进宝佯谓母曰:`今将嫁之矣'。悉斩之;母大骇。一妾有疾,召医视脉;医者曰:`此孕也'。进宝以妾众,亦竟忘之矣;怒曰:`宁有此事!汝止此,如有孕,不杀汝;若非孕,当斩汝矣'!顷之,内托一儿出,乃剖妾腹而得者。医者惊悸,进宝赏五十金。其不仁如此。及降海事发,解北京磔之。

武某,河南人;管效忠部将也。膂力数千斤。年三十余,长可五尺;身兼四人,马不能胜,止一黑马乘。当效忠被围于镇江也,武以大铁棍击杀千余人,众皆披靡。既出重围,问军士曰:`管爷出来否'?对曰:`未也'。复杀入;如此者五出入。万人之中,莫敢有当之者。效忠被鞭落马,武下马负之而趋,效忠因出重围而免。武入南京,以失机事,解上司;笞五十。及南京之战,余士信舞大刀至;武以铁扛拨开,即擒于马上。士信美仪容,布甲跣足;入京时,犹携刀乘马,武押之而行。解哈哈木,哈问曰:`汝将乎'?士信曰:`然'。哈又问:`愿为官否'?士信曰:`不愿也;止求速死足矣'!遂杀之。而武之救管、擒余之功,不录。又有先锋甘耀(辉?)者,短黑而勇;以醉而败,匿于民家。追者至,趋出搏战,杀数人;不虞背后马某擒之。入见哈哈木,哈问如前,甘对曰:`吾为将,杀数千百人矣;宜可以死'。亦杀之。

郑入镇江,大清将彭某引兵五百还六合,闭不纳。已而阮春雷服布衣、戴棕帽乘轿至,称兵部职方司;六合武生王寅生、文生夏志宏、徐三峰率众执香迎之,大清兵五百乃去。有湖贼刘青海率百二人归阮;问何能?曰:`团牌'。阮试之;刘舞毕,阮云:`去得,但不全'。阮置纱帽于几上,自起舞,牌影如花,不见其身。刘年二十余,拜阮为父;阮使为副总。阮审事明速,批答如流;真文武才。七月十八日,出兵滁州;戴小帽,罩甲,赤足出。团牌手十人、大刀手二十人、新降二百人随之;前一旗,写``三军司命''。甫出察院,执旗者仆,欲斩之;众告免,笞二十。至盱眙口,时滁州有凤泗道发炮,击死执旗者;众失色。阮怒曰:`若等无用'!遂持大刀直前,手杀五十余人;大兵退走。阮追杀至滁,大兵入城闭门。近城半里街上有大城墙,阮以两大钉钉壁间,执牌登;兵惊走。阮启门引入,至前街,门闭;复如前法。城内兵惧,开别门走。滁民出迎,遂得滁州。及八月初五日,始还六合;从者百人,拏盐舟一西下,遇大清舟三百,被围;发一炮,碎其二舟,余舟乃退,始扬帆去。王寅生有力,宏光时参将欲杀之,遂依高杰;至是归阮。取示与两人,驰天长;时已暮,寅生城下呼曰:`兵至矣!速开门。否则,鸡犬不留'。守者白王令,令醉不意;百姓曰:`他不降,吾等受屠乎'?遂启门;寅生三骑登堂,令走乡寺自缢。后南京韦巡按奏曰:`六合拒兵献城,天长杀官献城,仪真逐官献城'。众闻之,惧。后上批郎廷佐本:`免屠';乃安。寅生走入乡庄,与妻子酣饮怒歌;以数岁儿投之河,继杀一女,与妻短甲、草履持枪驰骑遁。韦巡按遣人拘逮,已不及矣。擒其两兄至,各笞二十,下狱;后释归。兄名起生,亦文生。

方郑兵入南,杭州有马龙来降事。马龙,或云崇祯时总兵也。顺治己亥,率海兵四十人至海盐县降,不受;因至杭州。守臣李部院、巡抚佟思远及总戎田雄疑有诈,马龙曰:`吾舟漏失机,故来归公。如杀我,恐绝后来者'!李等受其降。此六月初二也。小帽网巾,居城外二日;薙头入城。越三日,李设宴;酒半酣,问曰:`汝有何所能'?马曰:`无所能'。又问:`能箭否'?马曰:`不善也。然吾立于此,可令射之'。重甲端坐;射之,马以鞭拨,箭纷纷下;或击断、或接手中,九矢不伤。至第十矢,马平指如钱者;箭至,当之中钱上,箭激去。军士报射毕,田大惊。马曰:`公部下岂无避箭者?亦发十矢'。佟令一军士前立,马发一矢曰:`吾欲于汝头上过'。果然;二矢亦如前。第三矢,则曰:`此一箭欲于汝胸前穿过'。大惧,号求乃罢。双鞭重八十斤,轮舞如飞;久之,向空一掷,呼随身兵空中接去。众失色。脱甲与众砍,刀折而甲不损。田等窃作满语;马觉之曰:`吾七岁至满洲,岂不解耶'?田愧谢。闻南京败,复下海去。未审确否?

郑自南京败走,次年(庚子)引数百艘至广州府。舟用大木钉成,高与城齐;桅内藏兵五百,并堞攻击。战时三人一队,一人执团牌、二人持枪刀。一人居中击鼓;鼓一震,每队疏列成阵。大兵驰骑突前,其队又各让去,即分两边趋后截。其藤牌,桐油浸透,刀箭不入;大兵患之。有副将进计曰:`惟铁箍头棍可破也'。遂复战,用棍击破其牌,箭无所蔽;乃败。复走入海,居南澳。
台湾复启

久聆智略,芳名流播;虚心仰慕,悒悒何已!顷荷惠书,教以不逮;又遣贵介刘、马二君备述委曲,幸甚、幸甚!然窃怪麾下之未能谅不佞之心,而犹从流俗之未议也。

曩者思明之役,自以粮尽而退,非战之失也。况风帆所指,南极高、琼,北至高、辽,何地不可屯扎、何地不可聚兵?不佞所以横绝大海、移国东宁者,诚伤士女之仳离、干戈之日滋也。是以区区鄙怀,曾见于前札。往岁得贵藩院之书,而贵朝犹未深察,尚严边界之禁;遂使百姓流离、四省邱墟,坐捐数千里之赋税、岁糜亿万之钱粮,斯非贵朝之失策哉?今麾下计法处远,欲为朝廷久远之谋、万民之命;而贵介所传,又述前日之套语、``削发''之虚谈,欲以八闽及沿海各岛二说相饵,尚为知识者之论乎?自昔贵朝议和者屡矣,从先王以至不佞,止缘争此二字。况今东宁远在海外,非属版图之中;东连日本、南蹙吕宋,人民辐辏、商贾交通。王侯之贵,固吾之所自有;衣冠之盛,不输于中土。即未敢遽比太王之迁岐,而生聚教训,足以树万世之基业:此贵介所亲睹者也。不佞有何慕于爵号、亦何贪于疆土,而为此``削发''之举哉?而麾下以海滨为虑、苍生为念,则息兵安农,复归故业;使男女老幼皆得遂其生育,而举朝可以岁获数百万之赋:此仁人之心,不佞亦有同心也。缕缕膈言,麾下亮之!

附记:孔文举本姓王,江阴王鹳嘴人。幼为青旸吴焕如家家僮;少长,祝发于岭圣关帝者。既而至苏州,募化得数百金;走浙,居江边,时与海通。适有孔将军者悦之,与同舟;未几孔死,因蒙其姓,得其副将印。引舟兵至镇江固山刘之源营,请降。刘上闻,召对,赐坐;官以将军,还居镇江。及康熙初年,复召对,赐坐如前。时郑成功已死,其子经犹拥众,居海外台湾;遣文举往招之。经不出见,止答此书;文举乃还。后文举坐巨舟,拥兵至青旸祭墓。见家主称叔,演剧诣宴,俱有馈遗;次请绅衿张有誉等。乃去,北上复命。闾里荣之。

郑鸣骏倾心投诚

大清康熙二年八月,投诚郑鸣骏等疏曰:`臣兄建平侯臣郑泰集众海上,屈指廿年。初缘王化泰沾,维识大命有在;拳拳归款,无由遽达。去年六月内,蒙靖藩臣耿、总督臣李遣官宣布朝廷德威,招抚臣等投诚;臣兄泰遂欢然倾心归顺。初未知朝廷规矩,以为当先请旨,然后削发;随差臣都督杨来嘉赴阙待罪。恭蒙朝廷宽恩含盖,以为无奏章、又未薙发,恐无归诚之实;欲臣等先薙发登岸,乃准投诚。及杨来嘉称述朝廷德意宽宏浩荡,臣兄泰不胜感激,深自悔罪。谓海上诸众,原系臣郑家之火统驭;今郑成功已故,臣兄泰欲举全众归顺,俾诸岛清宁,永免朝廷南顾之忧。已于本年五月初五日,复差臣都督杨来嘉、杨淇先布诚意于靖藩耿、总督臣李、提督臣马、户部臣黄、兵部臣金世德,愿登岸削发。正在调师进发,出泊金门,将趋泉港;不意又有逆党冯澄世、陈永华、洪旭、周全斌等恨臣兄欲统文武水陆全师携眷登岸投诚,谋之逆侄郑锦,诈称大家愿听臣兄调度归顺,请臣兄赴鹭门计事。臣兄正欲乘此等齐集鹭门,说以共事投诚;随轻身而往,误堕奸计,于六月初八日赴席被羁。初九日,臣兄泰手书密嘱臣鸣骏同兄男缵绪速图归正,成其未了之志;身虽死而不恤。初十日,探知臣挈全部舟师进入泉州港,随投缳自缢,以坚臣等归顺之心。臣兄忠于朝廷而不顾其身,虽当危难之际,犹致嘱其后一意朝廷,初终无二;其心迹,并揭日月之昭明矣。今臣统胞侄臣缵绪统所部文武各官四百余员、水陆官兵七千三百余名各带家眷,驾战舰一百八十余号,直抵泉州港口;仍虑各处地方汛守船只及贩运船只恐未知叛逆谋臣兄之消息,一面点拨大小战船四十八号配精兵二千余名,管押文武官四十多员前往台湾、海坛、三都、南澳、铜山等处护接应援。计实在港内文武官四百三十一员、船共一百三十七号、兵五千二百余,蒙靖藩臣耿、总督李会同提督臣马、户部臣黄、兵部臣金安辑家眷泉城,随于二十六日文武官袍帽筵宴,仍计口给粮停妥。其余差出各员及兵丁船只,当俟续到续报。计逆侄郑锦自其父成功已故之后,实赖臣兄协佐拥护;今误听奸人之谋,自坏其长城,孤立无辅,将行趋于瓦解矣。臣仗天朝之威灵、合子弟之痛愤,即当躬率所部还捣鹭门,尽歼群奸,收朝廷之效,藉以雪亡兄九泉之恨。臣心区区,止于如此。若夫臣家三百余口、所部官头兵目为类颇多,不杀之外,另有浩荡异恩;此出上断,非臣等所敢知也'!
擒斩海贼

大清康熙二年十月,福建总督李率泰疏曰:`准水师提督总兵官施琅塘报前事。该臣看为逆孽郑锦,乃郑成功之遗种也。其父已伏冥诛,其子尚不悔祸,犹敢集余党复肆鸱张。当其骨肉相残、丑类溃乱之际,臣与靖藩提师驻扎漳、泉,一面广布皇仁,用示招徕;一面装造快艇,必图进取。业已再三申严水陆将领整顿舟师,以防叵测。兹准水师提臣施琅报称:``郑锦遣发水陆贼将林维等领带船只,抛泊海门,谋欲乘风顺潮直入海澄,烧我新造船只,诈亦狡矣。幸探得实,提臣密遣前营中军守备汪明等统水兵配驾快艇,夜半直抵海门;贼踪前来向敌,我兵奋勇,阵杀伪副将林维、吴习等,活擒贼官、贼兵一百二十五名,夺回船只、器械、伪印、牌札等件累累''。足见官兵用命,可谓破敌之先声矣。是役也,提臣施琅有法纵之功。但获伪官、伪兵名数繁多,恐涂次起解未便;臣会同靖藩,即今就近正法。伪印、牌札,立行焚毁'。

施琅字尊侯,福建晋江人。

厦门大捷

大清康熙二年十二月,靖南王耿继茂疏曰:`臣等奉旨料理水陆大兵进剿海寇,原议臣继茂统陆兵由浔尾而进,臣李率泰统陆兵由嵩屿而进。其水师各标官兵船只,在泉州者提臣马得功统领调度,由围头取齐进发;在漳、澄者提臣施琅统领调度,由海门取齐进发:俱经具题在案。臣等于九月二十九日合疏恭报后,随于十月初二日先发臣继茂旗下都统王大用、纛章京夏功等带领马步官兵,自泉州起行;另调原防漳州右翼总兵官左都督徐成功、右副都统汪元勋等各带官兵,自漳州起行:俱于本月初五日抵浔尾,与厦门高崎紧对下营设备。对岸贼船四十余只见我兵一到,各驾船游奕,朝营内打炮;我兵随发大炮、行营炮攻打。初六日午时,郑锦、周全斌领大\{舟周\}贼艘往来转戗,炮打我营;我兵亦发炮攻打,昼夜防范。初六晚,贼率精锐千余乘夜潜渡,直逼游击史定国营盘,火炮、弓箭齐发,与我兵垛台上相对砍刺;游击史定国带甲兵杀出营门,都督王大用等督兵策应,堵杀力砍,箭伤、炮伤贼兵及落水淹死者甚多。因潮水泛涨,难计其数。夺获盔甲十一副、火箭喷筒二十七件并弓箭、枪刀等项。贼奔败船,其大\{舟周\}随回;余有汛守贼船及陆路贼兵一片营盘,仍札高崎。塘报前来,臣等催督泉州同安水师仍发藩旗甲兵六百名,令都书佥事陈一明等带领,及郑鸣骏、陈辉、总兵杨富等官兵船只收拾开驾,听马提督带领马步官兵,沿岸调度,至围头与夹板船合\{舟周\}去后;臣继茂于本月十二日驰至浔尾,臣率泰在漳遣发漳、澄水师并统陆路官兵亦于十二日同海澄公臣黄梧、中路总兵官王之鼎等驰至嵩屿,兵威大振。所有把守高崎、嵩屿各海口贼船,有撤回厦门者、亦有在海上游奕者。臣继茂一面布置兵马,收拾快船并同安水师总兵杜永和大小战船;臣率泰一面发兵即攻打沿海贼船,夺获古浪屿,屯札架炮:各整顿渡海事务齐备,会议妥当。密令水陆两路官兵俱于二十一日一齐进取厦门,随飞檄围头水师各船于二十一日子时开驾,至厦门白石头会合。惟时夹板船先于十八日开驾八只出泊金门外港,马提督遂于十九日亲率各标大小战船四百余只、夹板船七只经至金门乌沙头;有伪提督黄廷等领巨舰百余号前来迎敌,我兵协力攻打,夹板船首尾击应,贼兵大败。据郑鸣骏、陈辉、藩旗都司佥书陈一明等报称:获贼大船二只、犁船四只,阵斩贼兵三百余名、生擒贼四十余名,炮伤溺死者甚多;据杨富报称:击折贼伪总兵谢福船桅,过船杀贼二百余名、生擒贼十八名并伪防牌、军器等项;据同安水师游击郑洪报称:同中军守备孔应贤等率领兵船与贼打仗,获鸟船一只、赶\{舟曾\}船一只、水底\{舟贡\}二只,杀贼七、八十名,余俱跳水,并获火药、器械等项。至二十日早,郑锦、周全斌亲督巨舰精锐之师,蜂拥挑战;夹板船扬篷出御。重叠锐击,打破贼船数只。我师齐起竭力应援,交锋死战,自辰至酉,贼始退回厦门港。时臣继茂在浔尾营盘了望,贼船遍海;惟虑我兵渡海登岸,贼船尾我之后突犯浔尾,宜计万全。该臣继茂一面令游击范维杰带官兵一千员名扼守高崎海口、副将李之珍等官兵一千员名扼守浔尾海口,各于两岸安设大炮堵御;其浔尾及嵩屿营盘仍札不动,以杜窥伺。一面遣发臣继茂部下都统王大用、总兵官徐成功、纛章京夏有功、副都统江元勋、摆牙喇参领徐文耀、王蟒汉参领马九玉、阿达哈哈番张元德、田养民、拜塔喇布勒哈番王梅、噶卜什章京朱怀德、吴效忠、副将马化麒、游击郭奇、史定国、牛虎等统领摆牙喇蒙古马步官兵四千余名,并同安总兵黄翼及杨富、副将辛球等步兵,衔枚夜渡;臣率泰遣发标下游击谢泗等带领官兵,并海澄公标副将吴淑、游击戴亮等官兵及总兵蔡禄、郭义等各标官兵,俱乘夜渡海。一时两路兵马抢岸同登,奋勇攻打;贼势不支。于二十一日寅时直抵厦门草安山会合,遂长驱至厦门城。时值荷兰出海王领夹板船已抵大担,控扼海面。臣率泰发水师提督施琅领该标官兵船只与督标参将徐登第等、公标总兵沈茂等及总兵蔡禄、郭义、柯义、林祖等各官兵船只,臣继茂发同安水师总兵官杜永和领官兵船只,共出古琅屿;臣率泰传令快捷炮船五十只为先锋,合\{舟周\}前进,发炮攻打。时贼兵屯聚教场前,正在张皇抵敌;忽见马步大兵蔽岭而下追奔欲杀,贼众惊溃,各乱奔上船,亦有奔船不及散伏山洞者。我师将兵马撤开,环札海岸,策应舟师击贼;贼兵披靡,退泊浯屿,计贼大船七、八百只(小者不计),蜂屯蚁聚,狡谋复逞。臣等一面发藩旗督标及各路官兵同提督游击陈天玉、守备华尚兰等带领官兵,并延、建、邵三府官兵焚剿山洞贼寇;一面商议将漳、泉两路舟师挑选精锐,令施琅统领,协同郑鸣骏、陈辉及总兵杨富、都司陈一明等会合夹板船于二十四日直抵浯屿扑剿,炮火夹击,贼船抵敌不住,退泊浯屿之外。本日黑夜潜逃,势必奔往铜山、南澳二巢;臣等已飞檄漳浦总兵王进功严加侦备,并移会广东平南王及将军、督提诸臣共图堵剿,期绝根株。其金门后浦尚多余逆,急须乘胜廓清;于二十六日遣发官兵船只前去攻取,剿杀无遗。所有厦门、金门各岛贼巢既经攻获,本应请旨定夺。因自闽至京往返计三阅月,内地重兵不宜久驻海岛;且士卒云集,挽运为艰。臣等再三公议,合将贼垣房屋尽行拆卸焚毁,免致贼船飘忽,复肆凭陵。惟浯屿小岛居垣无多,出海王暂留修船;俟其工完,即行拆毁。盖缘各岛越在界外,四面皆海;乃从古以来,寇盗窟宅。逆贼郑成功父子盘踞二十年,久逋天讨,皆坐于此。今仗皇上威福,犁庭扫穴。臣等职任封疆,何敢言功。惟是新旧将士航海用命以及荷兰出海王助顺宣劳,均应叙录。所有功次、伤亡并得获大小船只、铳炮、器械等项,容臣等查明另题。谨以克捷情形飞章入告,仰慰宵旰于万一也。臣继茂、臣率泰谨会同海澄公臣黄、水师提臣施合疏具题,伏祈睿鉴,敕部施行'。

徐成功,字凌图;辽东海州人。史定国,字明宇;陕西同州人。陈一明,字光宗;辽东人。中营中军都司陈辉、左营中军守备王之鼎,辽东人。范维杰,字子俊;江南休宁人。李之珍,字岐山;榆林人。马化麒,陕西人。郭奇,河南人。牛虎,字龙泉;山西人。黄翼,字辅卿;福建平和人。孔应贤,字伯柱;江西金溪人。守备徐登第,字云程;辽东人。王进功,字敏齐;辽东辽阳人。陈天玉,字明宇;辽东锦州人。华尚兰,大同左卫人。

十二月,靖南王耿疏曰:`臣等于十月二十七日攻克厦门、金门后,贼势无穴可归,必遁云霄一带;臣等飞檄右路总兵整顿兵马提备去后。十一月初四日,据右路总兵王进功塘报前来等因到臣;据此,该臣等看逆贼郑锦等向以厦门、金门扼险负固,跳梁为患;今仗天威荡剿,穷兽无依。臣等料其败遁,必至突犯云霄,函檄提备。而该镇设奇制胜,诱贼深入平川,断其归路;伏兵夹击,遂使数千贼众,一时擒斩无遗,可称勇略兼优者矣。生擒伪总兵纪凤,行令解赴军前审问正法;阵获船只、器械留备征剿,伤亡兵丁分别恤赏可也。目今余逆鼠窜,出没于铜山、南澳之间。臣等复令海澄公黄梧、总兵蔡禄、郭义并督标左营副将党守全、总兵杨学皋、副将杨乔统领官兵协同右路相机追剿,务期境内廓清而岩疆从此巩固矣。臣继茂、率泰合疏'。

党守全,字巽之;满洲人。 曾樱自缢

曾樱字仲含,号二云;江西临江府峡江县人。年十三,补弟子员。万历壬子举人,丙戌进士。给假归,玩诵``王文成先生集'';谒吉水邹南皋先生,师事之。家居三年,砥志不与外事;惟地方利弊、生民休戚,悉力竭心,以佐邑宰。

己未,授工部营缮司主事,监琉璃黑厂,兴造三殿。督神庙、光庙陵工,与中珰共事;公严厘核,珰敬惮。又奉望檄佑城工,节省以万计。

辛酉,典广东乡试,尽革士相见贽仪、犒从。壬戌,迁正郎。

癸亥,擢知常州府;地冲烦、赋役重而不均、科第显官甲天下、俗好讼,夙号难治。台使者自抚、按外,有巡仓、巡盐、巡江、巡漕、督学诸差,皆出巡操举核,竭地方之供应、掣有司之精神,民受其困。公申文御史台曰:`江南赋重民贫,上台朘剥日至。请一切戒饬,革钩访、取赎诸陋习,苏民困'!时御史同乡熊坛石先生,初骇愕,然卒为移檄饬行。公为政,镇以清静,出之岂弟;持以公平,风以廉俭。于利弊无不兴革、于权豪不少假借,于小民事事优恤、于财用事事节省。高明之家,一裁以法;一切受献、侵占、鱼肉小民之风,敛手屏息。赎锾减之又减,以至于无;即笞罪,亦不轻拟。夫马滥觞,江南孔道往来如织;公收驿册亲掌之,于勘合、廪给事件裁之、革之。有现任政府朱某之用夫太多,裁之;其仆汹汹,锁驿卒、击驿吏,公擒其人杖责之,遣去。文试清严,得士有吴贞启、陆月岩、刘光斗、龚可楷、高世泰、胡时亨、曹荃、吴之鸿、王孙兰、王孙蕙、邹左金、马瑞、王期升、史调元等公,皆成名进士;武闱规巡方以正,宿弊尽革。甲子以后,魏珰炽,党祸作。公独立不惧,护持诸公;调剂周助,曲尽心力。如武进孙公之免于就戍、宜兴毛公之逃戍而家属无恙、无锡高公闻信自沈而缇骑敛戢,上台调护,皆公之力也。其余江阴缪公、李公就逮之时,亦尽心竭力为之扶持;既竭俸囊,复设处以赠其行。诸公家属,无不铭感入骨。丁卯冬,入觐;士民耆老罢市祀祠,送至京口者千万人。

崇祯戊辰,迁福建布政司右参政、兵巡漳南道。有九连山亘闽、广,洞寇盘踞猖蹶,出没无常;自王文成、谭襄敏剿灭以来,种类复炽。公密约惠潮道谢琏刻期会剿,以十二月望启行,声言团练乡勇;偃旗息鼓,月夜扳藤扪萝入其穴。獠贼方睡,歼灭几尽;谢琏拒之于广,胁从就抚,洞寇平。督、抚某某攘其功,公不言也。

己巳,丁内艰;庐墓三年。辛未,起兵巡兴泉道。时海上多事,红夷与海贼刘香冲突。闽、浙、广三省海寇郑师芝龙已就抚,驻札于泉,然闽抚按猜防之甚;郑亦疑畏,每入谒两台拥兵胯刀,格格不浃。两台起杀心,郑亦盟叛志;方虑地方受其害,无复得其力矣。公一见郑,爱其才略;语曰:`君不用忧疑,某愿百口保君;君一心办贼'!郑感泣曰:`上台宪如公,某敢爱顶踵乎'!公乃力言之,两台释其猜疑。值红夷寇漳、泉,用郑为先锋,红夷创去。盖香系郑密戚,非公主持,欲其``心义灭乱''难矣。泉俗贫约,公以治毗陵者治之;豪右敛迹,小民安堵。属官馈送,丝毫不收;一应交际,务从省约。丙子,移福宁守道,加衔按察使。自戊辰为监司,十年不改官;以无一字入长安故也。郑师不平,因遣人携金入都,为公谋迁官。事发,逮公就讯;既事不由公,怡然就道。闽民数百诣阙击登闻鼓言枉;两台及闽绅合疏申雪;会郑帅``任罪疏''亦至,京御史叶初春等连章代白。于公未到之先,事得释;仍以原级,补福建巡海道。

内阁杨嗣昌以临蓝土寇纵横,特疏改任湖广湖南道,驻永州府。公念讨贼事重,因具疏请晏日曙为太守、葛元吉为司理;得旨施行。以戊寅冬月至永,佐偏沅巡抚陈公;剿抚兼施,寇息兵戢。永有祁阳王恣横,公以祖训绳之;王敛威守法,吏饬民安。

庚辰,升山东右布政,分守登莱海防道。五月,抵任。海右风俗,豪强尤横;公仍以治毗陵、治闽楚者治之,不特窒息其凶,抚按亦为敛戢,屡饬其下无犯曾公,亦如在吴、楚、闽时。辛巳,升巡抚登莱副都御史。时山东大饥,人相食;登莱与榆关相对,设法赈荒,应关门之需无缺。平青、济间土寇。时大清入山东,公所辖青、登、莱三府特全。论功,擢南少司空;公不赴任,仍请告归。

未几,京师陷,福王立;旋又南都失守。唐王称号闽中,郑芝龙专柄,因荐樱;起工部尚书兼东阁大学士。时张肯堂吏部移都察院,令樱掌吏部事;樱当铨政,持法不挠。寻荐揭重熙、傅鼎铨等擢用之,后皆以节着;人谓其知贤。以覃恩,晋太子太保、吏部尚书、文渊阁大学士。比隆武幸建宁、驻延平,命与定远侯邓文昌留守福京。大清兵入景宁关,势不支,文昌死之;樱乃挈家避海外,依郑成功于中左所。越五年,其地被兵;叹曰:`吾之不能死者,死有待也;今而已矣'!遂自缢。时辛卯三月朔也。

\hypertarget{header-n72}{%
\subsection{卷十二·粤纪}\label{header-n72}}

永明王始末

大清顺治三年(丙戌)八月,福京既陷,两广总督丁魁楚与广西巡抚瞿式耜、广东巡抚王化澄、广西巡按郑封、肇庆知府朱治憪、广东总镇严从云、旧锦衣卫佥事马吉翔、采买翠羽太监庞天寿等会议监国,而阁学兵部尚书吕大器自闽至,原任兵部尚书李永茂以守制并至;式耜首言监国永明王贤,且为神宗嫡孙,应立。永明王讳由榔,桂恭王常瀛少子。恭王初封湖广衡州府衡阳县,以流寇乱,徙寓粤西梧州府;时恭王已薨,永明王犹在衰绖中也。昔者唐王尝语群臣曰:`永明,神宗嫡孙,统系最疏。朕无子,后当属诸'。时恭王太妃王氏曰:`诸臣何患乎无君!吾儿仁柔,非拨乱才;愿更择可者'!魁楚等请之坚,以十月初十日监国,十四日(丙戌)即皇帝位;仍称隆武二年,以明年为永历元年。改肇庆府署为行宫。推置僚署有差:魁楚、大器俱为大学士,式耜以吏部右侍郎兼阁学,掌铨;魁楚兼戎政、大器兼中枢。永茂请终制。

``粤事记''云:`永历立,晋王化澄宪副、郑封通参;朱治憪右副兼兵侍,提督两广,承丁魁楚后。内外局惟魁楚主裁。肇庆府去广州府仅四百里,拥立时无一函商及;三司各属既立后,复不颁新天子诏。元勋大老,惟鬻爵择腴是务;至于军国重事,如峡以外设守广州、防御梅岭,不暇顾及,且暂为目前计而已'。

丁魁楚,河南永城人;进士。王化澄,江西人;崇祯甲戌进士。郑封,河南人;甲戌进士。朱治憪,浙江举人。严从云,江西人;封靖江伯。马吉翔,顺天人。庞天寿,北直人;司礼。李永茂,丁丑进士。

广州立绍武

福建旧辅苏观生、何吾驺遁回广东,观生尝贻书魁楚,欲预拥戴功,遣陈邦彦来劝进;魁楚与观生素不协,拒之。观生乃自南、韶还师,适唐王弟聿\{金粤\}浮海至广州。十一月癸卯朔,观生与布政使顾元镜、总兵林察等拥聿\{金粤\}入广州城,立为帝,年号绍武;以都司署为行官。招海上郑、石、马、徐四姓盗,授总兵官;以与肇庆相拒。

``粤事记''云:`绍武立,一月内,幸学、大阅、郊天、祭地等钜典,按日举行。二、三文官连膺覃恩数次,举朝无三品以下官'。

``遗闻''载:绍武监国,然改元、行郊礼;是帝矣。

苏观生字宇霖,广东东莞县人;保举选贡。无极知县,隆、绍两朝大学士。何吾驺,广州香山县人;崇祯辛未进士。顾元镜,浙江湖州人;万历丙辰进士,绍武朝大学士。

万元吉固守赣州

丙戌九月,万元吉率义勇固守赣州,杨廷麟等附之。赣地虽高,三面俱水;城中望外,浩淼无际;惟南门无水。元吉于砖城外四面,筑木城一座。大清兵攻围急,元吉出旧库元宝数十万,陈列几案,谓众曰:`能杀一人者,赏元宝一'。众遂奋勇出战;大兵畏之,不敢薄城。坚守一年,众亦惫甚。及丁亥八月,城中忽失火,南门复有两人内应;大清朝抚臣科某率兵乘之,自南门杀入,众犹巷战,杀伤颇众。力竭城陷,万元吉、杨廷麟与杨玉宸等俱投清水塘死。

杨廷麟字机部,崇祯辛未进士。杨玉宸,宁都贡生;与廷麟起义者。

永历移梧州

大清兵破赣州,丁魁楚闻报,与太监王坤趣永历移梧避之。瞿式耜曰:`今日之立,为祖宗雪仇耻;正宜奋大勇,以号远近。东人复不靖,苟自懦,外弃门户、内衅萧墙,国何以立'!争之不得,遂移梧州。寻还肇庆,故大学士陈子壮书达式耜,请力馘苏而趣兵东;永历遣兵科给事彭燿往谕之。燿,粤东人,旧为秦令,有能声;譬晓伦序、监国先后、国家仇仇利害。观生等杀耀于市,日集兵向肇庆。右司马林佳鼎督兵靖东郊,东将诈降,陷佳鼎没于水。

又``粤事记''则云:`绍武立,学臣林佳鼎位总宪,行大司马事;提兵西向,上三水县,欲侵肇庆。式耜奉命出东峡,设炮御焉。十一月十五日,对阵一炮,歼佳鼎;侦者讹传式耜败。肇庆新创,朝廷逃复一空。永历随众奔遁,直达广西梧州府五百里,溯流两日夜,并程而至。太后马氏通史书,本不欲世子称帝;呼省臣李用楫、台臣程源等面呵无固志,且诘责弃逃状。适式耜手报至,知前讹,诸臣皆伏地引罪。后奉永历再下肇庆,别遣靖江伯严从云等护三宫预驻广西桂林府'。

``遗闻''载林佳鼎为永历臣;而``粤事记''则云绍武臣。且胜负各异,并志之以侯考。但``粤记''一书,乃宦广者所寄予也。

陈子壮字集玉,号秋涛;广东南海人。万历己未探花;崇祯朝春坊,后封南海公。李用楫,宜兴人;癸未进士,礼科都给事中。程源,癸未进士。

王坤擅改诸臣

瞿式耜疏言:`草昧之初,惟养圣德、修纪纲、慎政教、挽人心、布威武、起用人望、招徕贤俊为首务'。王坤者,固北阉;自南都失而入闽,隆武遣出。兹用司礼秉笔。有户部郎中周鼎瀚,内批改给事中;式耜力言不可,不听。以粤巡使王化澄升粤督,寻代佳鼎,晋少司马,掌中枢;大器先以病去矣,内批:升化澄为大司马。式耜疏言:`化澄诚贤,有廷论;斜封墨敕,何可为例!请补部疏,尚得体'。盖汲汲为阉预虑也。晋李永茂大学士;茂守制,佥称专知经筵,不入直。茂疏荐十五人,为十五省人望;疏上,王坤启视,殊不悦。既而,山西道御史刘湘客斥罢。永茂怫然曰:`朝廷方以经筵责茂,茂以十五省人进,非私也;斥湘客者,斥茂也'。即日解舟去。式耜疏言:`大臣论荐,新朝盛事;司礼辄去取其间,无以服御史,何以安大臣'?坤复疏荐海内硕望数十人;式耜又言:`司礼抑人不可,荐人更不可'!吏科都给事中刘鼒等疏论坤内臣,不得荐人;永历怒,斥逐鼒等。式耜力持之,得复用。御史童琳参都御史周光夏越资序题差用,私乱台规非法;命廷杖琳。式耜力救,得免。升翰林院检讨方以智为中允,改御史刘湘客充经筵讲官;坤不悦。湘客且疑刘鼒疏出以智手,以智放舟去。

刘湘客,陕西西安人;布衣。

辜朝荐献策下广

先是,两广在籍乡绅多与两院、三司通关节;己未以后,何吾驺主之。辛未以来,潮阳辜朝荐每事与何吾驺角;然吾驺势大,朝荐弗胜也,愤甚。及丙戌八月,清兵取闽,尚无入广之令。潮阳县距闽省止四日程,朝荐亲往福州府献策下广,极言三月内可直达西粤桂林;思得首功,以压吾驺耳。清固山李成栋遂发兵,先选精骑三百宵夜东行,由老龙而下广;过广州增城县,俱潜入花山。十二月十一日上午,止命前锋十人以青白布裹头,扮作洋船舟子状,直至广城布政司前紫薇牌坊。下午,人丛中悉去头上布,现出辫发,露刃大呼;止杀一人,满城崩溃。十人分守六门,闭城,昼夜巡视。第五日,三百骑始至;成栋大军月终乃至。

时苏观生匿酒肆,有于箧中见文渊阁阁臣印,索其一醉,弗与报知;巡缉被执,观生慨然曰:`吾以一布衣登两朝相位,死亦何憾'!质问时,一语弗答;遂杀之。何吾驺、顾元镜率士绅投诚,优礼而去。吾驺乞修``明史'',门署``纂修明史何'';广州有``吾驺修史,真堪羞死''之谣。绍武一身扮买旧衣人,欲出城,未识乡路;貌复寝怪,识者无敢藏匿。为内阁中书所持,卖银十两;副将杜永和擒至,并周王、益王、辽王等俱杀于广州府布政司前双门下。绍武在位,二月而已。百姓俱薙发归顺,市不易肆、人不知兵;但传檄各郡县耳。时有石、马、徐、郑四姓联舶海上,花山杨光林亦拥众数万;水陆交讧,民不聊生。成栋相机剿抚(花山在增城县)。

李绮参丁魁楚

十二月十八日,侍御李绮参丁魁楚十大罪:欺君、误国、玩兵、害民、败群、乱常、罔神、蔑誓并且丧身、辱祖;若不改□,覆亡立俟。面帝朗诵;魁楚亦引罪,上慰谕。明日奉旨:`李绮降三级,调外用'。绮宵夜入廉州府,以家眷在此也(绮,松江华亭人。崇祯庚辰进士,广东提学副使)。

永历奔西峡

李成栋既杀绍武,于十二月二十三日发兵往南韶而亲下肇庆。二十五日闻报,瞿式耜请视师,督战士驻峡口。王坤复请永历西避之,式耜争之,不听;遂驾小艇上西峡。

``粤事记''云:`丁魁楚用旧旗鼓苏文聘升内阁办事中书,昼夜出入,计值,百□,分给文武凭札,绝不示人羽报;以事干者百金减半,诱人多就。有琼崖参将白斌托李用楫弟李来营札总兵,来新受中书,见时事日非,闻广州信,因为观望。斌在海外犹未之知,恐李来为之不力,具廪揭魁楚促早就,云具名礼在中书李来处;来尚不知,忽二十四日夜半遣二十人索斌贽,示禀揭,勒逼如数,来反填入酒器五十金。二十五日黎明,朝见谢恩者犹趋跄殿陛。忽肩舆出城,掠小艇驾上西峡,喧传为帝。于是文武纷逐,各不相顾;帝因知有凶耗,随奔者亦揣广州事必败,不可瞬息留。侍御汪光宝与李来同舟,万众竞进;不知邻舟为帝座,几为刺没。惟魁楚舂容雅度,绝不惶;遽渐移赀入舟,瞪目而视,反若局外观者。魁楚丁拥戴后,即自为计;今则有他意矣'。

苏文聘,广东人。李来,宜兴人。白斌,浙江人。汪光实,淮安清河人;举人。

丁魁楚

``粤事记''云:`丁魁楚,河南永城人;晋抚失机,遣戍五年。崇祯戊寅,奉旨纳饷三千两许本军,准回原籍;魁楚援例得归。永城有旧总兵刘超,壬午十月以私仇杀丁艰侍御魏景琦。豫扶王汉奉旨往勘,超又一箭毙之;率家丁劫众乡绅,勒魁楚上疏讼冤。魁楚计款之,阴遣子弟兵四面布置。至癸未二月朔五鼓,伏兵四起,用铁网遏超,擒解京师献俘。魁楚叙功复职,冢宰李日宣量加本省屯田巡抚衔。甲申南都立,马士英擅权,会推两广总制。十一月,受职。乙酉七月,闻南都陷,即潜通靖江王,约期定计下广举事;王果以桂林推官顾奕为相、临桂知县史其文为大司马。八月初七日直抵肇庆,魁楚已于初六日拜隆武登极诏矣;遂发大炮击碎王舟,袭其所载。执王,并擒顾、史二人解闽京,供论叛逆,诛之。魁楚遂封靖粤伯'。

魏景琦,崇祯庚辰进士;永城人。顾奕,苏州人;举人。史其文,溧阳人;天启辛酉举人。

沐天波激变土司

沐天波号玉液;沐英之裔。袭封黔国公,世守云南。丙戌秋,南都陷,三司、两院请增兵守滇南境,□防客兵流入;增兵必措饷,求助之。天波蹙容曰:`极是紧事,第迩年多费,不能助一缗;奈何?还须从长酌处'。然增兵刻不容缓,而滇田硗瘠,赋复难加;天波谓`各土司用盐颇多,再增本府一票,饷百出矣'。众然之。乃令盐务计会官结连,使盐票再置沐府饷票,准于本年九月始。初行时,土司亦有遵法纳沐票饷银者。初九日,楚雄府土司吾必魁抗令于盐场中,不独弃沐票,并夺商盐。鸣之县,殴差;鸣之府,殴府差。言`己无朱皇帝,何有沐国公'!遂率众入城,执楚雄府文武数众而杀之,据其城。

沐天波调沙亭州

天波欲复楚雄城而力不逮,想调土司强有力者克之。素闻沙亭州骁勇,令符调之,失时。

崇祯初年,滇南有普民升之变。民升非自能为乱也,其妻范氏美而艳,有奇力、且多智,而不好静;自引民升振旗鼓、掠力壮以为乐。朝廷为之耗饷者,凡二千万;两院司道夺职镌级者不可计,恇怯溃弁殒命革逐者数十百人。后民升将败,范氏忽与言别曰:`尔勿以我为妻,我亦不以尔为夫;我去矣'!即往鹤庆府执土司沙亭州者,曰:`惟我与尔,可为夫妇'。亭州曰:`我自有妻'。范氏曰:`请出,我与语之,妯娌称呼'。三言未毕,举刀刃之;即携亭州袖曰:`今不可为百年之好乎'?于是亭州悉遵范氏约束,严号令、明赏罚,生聚教训,遂为滇南土司中富强第一。兹闻天波召,遂欣然倾洞而出(他书载``沙定洲'')。

沙亭州袭破沐府

十一月初旬,沙亭州困楚雄府。十五日,已解吾必魁首级,扫清楚雄;禀请再设文武各属为守土抚治计。天波喜甚,将金帛重赏之。亭州又禀曰:`臣夫妇欲来面恩'。至二十九日,天波升座,两榭设仪仗、鼓乐、旗帜,殊为炫耀,受拜、受贺。亭州与范氏两人三叩未毕,急趋上殿急视之,出刃于靴,四刀飞舞;已格杀左右数人,侍卫人等如风草偃仆。天波速奔入内,亭州、范氏尾后疾追;随见随杀,沐府男妇侍卫约五百人,须臾尸横遍地,天波逾墙走。范氏遂稽核府内藏蓄,绝□未死内寺与姬妾,俨称中阃。亭州整容升座,袭天波冠裳,称沐府新主;已有趋谒拜贺、供其调遣者矣。又遣亲党与守城关,盘诘出入。盖亭州破楚雄后献功时,各兵已伏城垣;至是刻期并起。亭州踞坐沐府,守令仍许照旧。时沐府富厚敌国,石青、珍珠、名宝、落红、琥珀、马\{足臣\}、紫金装以细筏箧五十筋,藏于高板库;每库五十箧,共二百五十库。他物称是,八宝黄龙伞一百四十执。亭州将沐府数十世蓄积,日夕辇运洞底;彼蓄志已久,时乘间突发耳。夫天波蓄养数百年,竟致妻子不保、祖业丧亡,良可嗤也!

语云:`多藏厚亡';于乱世每见之。楚藩不肯养兵三名,及城破,献贼三十万;嘉定不肯助饷二万,迨京师陷,闻籍五十二万。此辈真守库子耳!不知千古来珍宝,多是空费人搬运者;殊可猛省!

孙可望入滇

丙戌,张献忠死成都,孙可望驰入贵州,据定番州,休息士马;意欲入滇南,取沐府三百年厚藏耳。至是,闻为沙亭州所取,大惊;击案曰:`此吾几上肉也;亭州小寇,何得袭我囊中物乎'!遂宵夜启行,疾入云南,为七月初二日;亭州已于前三日遁回本洞。可望止取沐府空署一所,并戮亭州所署官属。天波来自大理府,可望许之复仇,即用天波为报门官。十一月,选二千精锐围亭州土穴。至明年戊子二月,擒亭州、范氏及亲戚四人;天波府藏与亭州素积,乃悉辇入沐府丙宅。可望将所擒六贼,于天波坐前活剥其皮;天波亦叩首称谢。此可望入滇之始末也。

永历至梧州

大清顺治四年(丁亥)正月朔(癸卯。永明王称永历元年),至梧州;盖以腊月二十五日闻报李成栋亲下肇庆,故避至此。时丁魁楚惑于奸弁苏聘,从梧走岑溪;王化澄走浔州。随行者,止式耜一人。是月十六日,成栋克定肇庆,随发副将杨文甫、张月领兵克取高、廉、雷三郡;即于二十九日一鼓而入梧州,广西巡抚曹晔出降,梧属俱遍令纳印。及南雄、韶州二府报捷,别遣副将阎可义等前赴琼州。

``粤事记''云:`正月朔,帝在梧江舟次,免朝贺。知府陆梧廉取库银五十南为雇觅挽夫费,将北进桂林府。所召次辅李永茂、晏日曙同卿田芳,未及朝见;闻帝奔,率银台郑封退潜博白县深山。从行者,惟总宪王化澄、大司农吴炳、宫詹方以智、文选吴贞毓、省臣唐諴、台臣程源、中翰吴其雷、洪土彭、大金吾马吉翔、司礼监庞天寿等而已。式耜犹留肇庆,同朱治憪为守御计。元宵后,溯流上府江;在途,拜方以智、吴炳典枢务'。

前言式耜一人从行,化澄走浔州:是由肇庆至梧州时也。``粤事记''言从行化澄等,而式耜犹留肇庆:是由梧州将奔桂林时也。此皆肇庆未失之先。及肇庆失,而式耜始抵桂林。

陆世廉,苏州人;恩贡生,后为光禄寺正卿。晏日曙,江西人;饶州举人。田芳,河南人;丁丑进士。吴炳,宜兴人;己未进士。后为大学士,死节;先任湖西兵巡道副使。方以智,字密之;桐城人,庚辰进士。唐諴,湖广人;癸未进士,副都御史。吴其諴,宜兴人;庠士,兵科给事中。洪士彭,宁国人;庠士,礼科。曹晔,进士;兵部尚书。广州、韶州、南雄、肇庆、高州、廉州、雷州、琼州八府俱属广东,而桂林、梧州、浔州、平乐四府俱属广西。岑溪、博白二县俱属梧州。

永历抵桂林

三月,上抵桂林,改桂林府署为行宫。式耜肃殿陛,敕守谕诞告楚、蜀各镇:粤西居山川上游,桂诚可都,疏请道里之可达桂林者。王锡衮、文安之为相,周堪赓、郭都贤、刘远生为六卿。时给事中丁时魁疏论新政,烺烺硕画;召掌礼科给事中。金堡素有清直声;终制,敕召还。何腾蛟晋阁学督师。

金堡,浙江杭州人;庚辰进士,兵科。丁时魁,湖广人;庚辰进士,吏科。何腾蛟,黎平人;辛酉举人(或云桐城人)。

李成栋斩丁魁楚

``粤事记''云:`先是,十二月十五日省城之变,丁魁楚知之最早,即密遣亲干齎黄金三千两、珍宝称是,重贿李成栋。至二十五日上奔时,彼日有密报,亲干已投入李成栋帐下为家丁,刻望回音;故众虽□□□□独安闲也。魁楚有大哨船四十,将三年宦囊悉载入,仍在肇庆度岁。丁亥正月初旬,方移舟西向,入岑溪;并于城中修葺茆芦,以候广城信,实不欲登岸。亲干于二月初始以金宝入,达魁楚意;成栋曰:`何不早言,正欲邀尔主仍为两广军门;急齎书去'!二十六日,魁楚投岑溪舟中得成栋手书,大悦;即移舟顺流东下。时成栋驻梧州,先上五里迎之;握手道故,相见恨晚。知魁楚三子入广已夭其二,止存长子;通名先叩,情谊甚笃。临晚,邀魁楚父子饮,隆重如常礼。把臂间,指画岭表,审度当朝;谓`东南半壁,惟某与老先生撑持'!因订云:`明日吉期,敢烦再摄两广篆。拜表即真,亦在明晨'。将旗牌、符纛、制台旧敕即悉手付之。魁楚喜甚,乃别。夜半,成栋戎服升帐,列炬;交战将令旗,请魁楚父子有机密语。魁楚茫然不知所以,即过舟,见成栋正位危坐,知事已变;遂跪请曰:`魁楚止一子,或不及妻拏'!成栋曰:`汝欲饶子乎?令先斫下'!左顾,而首级至矣;即将魁楚斩之。成栋立舟首,火光烛天,照同白日;将魁楚家丁每营分一人细查,家属一妻、四妾、三媳、二女及婢仆妇净身搜检,携入成栋舟中;惟一妾于过船时,投入江中。四十年厚橐,悉归成栋;闻舟中精金八十四万,皆三年中横取者。呜呼!罔民、虐民,可以鉴矣'!

瞿式耜留守桂林
四月,大兵渡海,克定琼州。方警报之叠至也,王坤又趣永历往楚。时有自湘南来堵者,盛言湖广长、衡、永、宝四郡未有所属,宜亟取以为中兴之本;方以智、吴炳奏以为可。式耜上疏,言胜败存亡、山川要害甚激切;略曰:`驾不幸楚,楚师得以展布,自有出楚之期。兹者一年之内,三、四播迁;民心、兵心,狐疑局促,如飞瓦翻手散而覆手合'。又曰:`在粤而粤在,去粤而粤危。我进一步,则人亦进一步;我去速一日,则人来亦速一日'。又曰:`楚不可遽往,粤不可轻弃。今日勿遽往,则往也易;今日若轻弃,则更入也难'。又曰:`海内幅\{巾员\},止此一隅。以全盛规西粤,则一隅似小;而就粤西恢中原,则一隅甚大。若弃而不守,愚者亦知其拱手送矣'!擎跪涕泣,不可挽。无已,请身留桂;乃命式耜留守桂林,各路悉秉节制。式耜仍疏请暂驻全州,以扼楚、粤之中。

当平乐之不守也,大清兵直薄桂林。三月十一日,冲入文昌门,城中大恐。时焦琏自合浦归,从数人推弦提兵,与大清兵接战,稍却之,屯阳朔。遍野俱薙发,式耜与琏孤守危城;疏请征安国公刘承胤兵。承胤初从武岗入护,犹持正守法:逐王坤为弄权,而叱周鼎瀚为奄寺鼻息;故见重式耜,发兵数千援桂。未几,承胤请金吾郭承贤、马吉翔、严云从封伯,御史毛寿登驳参`金吾无矢石功,何得援边镇例晋五等'!吉翔等疑疏出刘湘客,鼎瀚遂造蜚语为``董卓、漼、汜之议''激承胤怒,逼永历立命廷杖,而缚筹登、湘客及御史吴德藻、给事中万六吉于午门外。会诸臣申救,得免;寿登等俱落职。承胤益横,胁劫永历幸武岗。式耜疏留全、阳,曰:`闻郊社礼成,即图移驾,不知移驾将回桂林耶?抑幸武冈、辰、沅耶?今日原以恢复两粤为心。则不徒西粤未恢,不可移动;即东粤尚未恢,亦且当驻全也'。故承胤专嗾杖湘客等,以湘客主还跸桂林之议也。承胤所部至桂,挟饷不出兵。式耜搜括库藏而外,捐囊万金;夫人邵氏,亦捐簪珥数百。兵卒不肯出,与焦兵主客不和,哗变;击门掠布而去,为五月十四日。永历竟驻武岗。

五月二十五日,大清兵侦兵变,积雨城坏;环攻桂城,吏士皆无人色。琏负创奋臂,呼督师、抚、按分门婴守,用西洋铳击中马骑;寻出城战,奋勇击杀。自辰抵午,不及餐;式耜括署米蒸饭分哺之,士卒俱乐用命。明日,复出战,大清兵旋去。式耜先令路将马之骥伏于隔江,犄角相应,固圉倍慎。是三月之内危于乱兵,式耜一手指挥,琏乃得底定。琏久将桂,得桂人心;式耜国士遇之,故独得琏死力。
以保桂功,晋式耜兼太子太师、临桂伯;式耜辞不拜。疏上,不允。复请告自劾:`自二月十五日以迄五月二十九日,此百六日中遇变者三,皆极危险;变故当前,总辨一``死''字,亦遂不生恐怖、不起愁烦。惟是臣之病不独在身而在心,不徒在形而在神;身与形之病可疗也,心与神之病不可医也'。又疏再请返跸全、阳,卒不听。乃督琏恢朔、宾、柳、浔等郡,并复□。

至八月,具疏上言粤西全定,请还桂林,昭告兴陵。

昔刘锜守顺昌,兀术曰:`刘锜何敢与吾战'!则宋之不兢可知。王师南下,将相闻风迎避;惟钱唐江两战,差强人意。入闽、入广,势如破竹;其能鏖战以却兵者,惟瞿、焦二公,真人杰也哉!琏,山西人。

永历驻武岗

``粤事记''云:`五月中,永历自粤西至武岗州(武岗属楚宝庆府)。时方以智、吴炳随驾,马吉翔、庞天寿护三宫移跸荆南。路出衡、永,巡道严起恒郊迎;面广身伟,纵谈时务,遂拜相。相度武岗州,可暂驻驾;遂以州署为行宫。王化澄后至,亦协理阁务。百日间,先朝流寇如湖南曹志建、河南王朝俊等逼入河南者,俱称提兵十万、五万来归宇下;悉赐五等爵。又进何腾蛟为总制、加宫保,建节衡州。李自成余党高必正等声言百万流入长沙,腾蛟具奏,遣堵胤锡统制之,号``忠贞营'';分为十大营,防守长沙。文臣、武将位置星列,兵势称振'。

张家玉沉江

先是,二月朔,张家玉与陈子壮竖义起兵。于是上而苍梧、下而潮阳,所在伏莽淫掠小民、烧毁村堡。家玉六月兵败,自沈于江。子壮潜身高明,复拥一村妓,因而被擒;解至省城,李成栋会齐三司曰:`若依国法,子壮应剐三千六百刀;今抵下十倍,三百六十刀罢'!降臣袁彭年跪禀曰:`国法所在,还应三千六百刀为是'。成栋曰:`我尚恨其不先死来解也,何必如是'。

羊城上下仍不克靖。潮阳界于闽漳,山海蒙箐,盗贼益炽;百姓追原乱始,皆由辜朝荐与何吾驺争权,引大兵入广所致;恨入骨髓。及成栋归明后,永历驻跸端溪,朝荐因吾驺在朝,不敢出山;虽门生李用楫三为荐剡,恐事败露,终未见朝也。

家玉,广东广州府番禺县人;崇祯癸未庶吉,后封番禺公。子壮,广州府南海县人;万历己未探花,后封南海公。

苍梧属广西梧州府,潮阳属广东潮州府,高明属肇庆府。

永历入粤西

大清平定长沙,而衡州相继尽失;总兵黄朝选、杨国栋等被执,尸几断流。八月二十四日,武岗复败;永历又播迁入粤,次柳州。式耜屡疏,极言不可他移一步:`滇、黔地荒势隔,忠义心涣;三百年之土地仅存粤西一线,且山川形势、兵马糗粮俱有可恃'。时督师何腾蛟、新辅严起恒及刘湘客咸至桂、南安侯郝永忠率兵骤至,宜章伯卢鼎亦至自楚;式耜复疏,极言`柳州猺獞杂处,地瘠民贫,不可久驻;庆远壤邻黔、粤,南宁地逼交夷,不可远幸'。

时腾蛟与永忠、鼎、涟等俱分防任汛。会土司覃裕春子鸣坷与道臣龙文明构兵,永历复次象州。式耜与腾蛟、起恒、湘客等筹画调和主客,集永忠、琏誓于神,刻期出师。鼎与滇镇总兵赵印选遂各分路驻全,全州战胜诸帅连营而军;大清兵因次楚。

十一月,永历自象州抵桂,式耜与严起恒并相。司礼庞天寿请催兵下梧,久在粤旧司礼王坤被承胤逐者复入自武岗;至柳、至象,票拟皆金吾吉翔手也。式耜疏请永历搅大权、明赏罚、严好恶、亲正人、闻正言,威德并行,以服远近。时谓名言。

``粤事记''曰:`八月二十四日下午,大兵忽至武岗州南,守兵皆在城北,迅不及支,一战而败;阖城上下斫北关,弃釜飧而走。除帝驾、三宫,无不跣足奔者。皇子甫两匝月,竟委泥沙;中宫嫡妹年亦及笄,与母同舆出城,俱迷失无踪。阁臣吴炳整衣冠北拜君亲,奉敕诰自缢。永历恐乱兵自全州、灌阳由大路抢桂林,乃与臣工从间道踉跄至庆远府。仅觅二小舟,三宫并载;随路逗遛,行行且止。至十一月十五日,始抵象州。意欲进南宁府为久避计,又为新兴伯焦琏乱兵所阻。从行文武官皆以青布囊头,胼手胝足,面无生气,几致散去。马吉翔左右帝舟,力挽众;乃分遣王化澄、吴贞毓、庞天寿护三宫上南宁,永历仍溯十八陡逆流北上。十二月初三日,再达桂林,得延残喘;君臣皆键户遛兵。人无土著、街无独行,薪米、百物价腾五倍。军丁居货,贸易无善颜;众皆度日如年'。

天子流离播迁,子委泥沙、眷戚不保,亦可悲矣!吴公自缢,是``主辱臣死''之义也。庆远、南宁二府属广西,灌阳属桂林府。

三宫至南宁府

十二月初十日,三宫至南宁府,议商安集处。时守道赵台犹据府署不肯让,锦衣马吉翔责台慢视,当坐``大不敬'';台始退入分司署。三宫以南宁府为行宫,供设帐具草率不堪。移入时,恶少逼视。有流寓贡生王者友之弟王者臣语出无状,中宫怒,执送有司;仍以``讹传''告免。

赵台,北京人;官生。以府判升监司,后为巡抚。王者友,南直人;后为御史。

张献忠乱蜀本末
甲申春,献贼大掠湖南。遇左良玉兵战败,遂尽掳湖南船只、居民,自夷陵挽舟入川。时流贼所掳百姓数十万逆流而上,日行一、二十里;舟中乏粮,饥死大半。使川中能扼险而守,夔门三峡之险,虽百万之众不能逆溯而上也。时巡抚陈士奇在重庆,有饷数十万;议者请发饷征兵,守夔关一带。士奇曰:`縻费朝廷之饷,异日难以消算;我虽卖□,不能偿也'。由是,坐视献忠入川。由夔州历忠、万,所在军民望风奔逃,并无一矢相加遗者。

甲申六月,献忠兵至重庆;城中乡绅大家具先以家口逃出城外。瑞王时自汉中避贼来,亦在城中;知贼信紧急,亦欲出城,士奇执不可。及贼至城下,士奇茫然无策。贼围城之第一日,命一人至城下说降;城中守者不应。第三日,贼命两妇人裸体在城下秽骂,城上亦不解何故。重庆城三面临江,皆石壁;至西南,有砖城数十丈。贼就其处挖掘,入火药数石轰之,城崩十余丈,砖石皆飞入云际;贼乘势破城。城三面临江,贼从一面来,城中数百万生灵无一逃者。巡抚陈士奇、知府王行俭、巴县知县王锡俱被执,张献忠欲降之,俱不屈;而王锡尤激烈,愤骂不绝口;俱被害。重庆卫指挥顾景闻城破,急入瑞王府中,以己所乘马扶瑞王乘之疾走;遇贼,为所执。见献忠,□曰:`宁杀我,毋犯亲王'!献忠叱杀瑞王;景大骂,亦被杀。献忠遂屠重庆,砍手三十余万人,流血有声。

七月,献忠率兵向成都,沿途州县或降或逃。八月,围成都;乡绅、有司请蜀王发帑金募兵守城,王真守财虏,吝不与。及城破,王及有司俱被害。巡抚刘之渤,陕□人也;在任有声望。献忠欲降之,而之渤骂不已;献忠怒,杀之。凡成都所属州县,悉降于贼。献忠乃称帝,国号``大西'',称``大顺''元年。以桐城江某为宰相、成都所属乡绅严某为吏部尚书、江某为礼部尚书,以其养子孙可望为平东将军、李定国为安西将军、艾能奇为定北将军、刘文秀为抚南将军(四人皆冒姓张),以其党王尚礼为中军府都督、白文选为前军府都督、王自奇为后军府都督。时本年三月北都陷于李自成,宏光新立于南都,中原多事,不暇问及西川;故献忠得窃据成都。然献忠暴狠嗜杀,鞭挞无虚刻;即左右至宠至爱信者,少失其意,即斩艾如草芥。故百姓惴惴不服,远近州县无不起义兵杀贼。献忠乃大肆屠杀,稍有犯者,即全邑尽屠。然贼兵一过,义兵随起;凡献忠所选府州县官,有到任两、三日即被杀者,甚至有一县三、四月内连杀十余县官者。虽重兵威之,不能止也。故献忠拥兵数十万,□自称制,而其威令所摄伏者,不过成都前后十余县耳。

乙酉春,夺取井研县。内阁大学士陈演女为皇后,问左右以封皇后之礼;伪礼部具仪注进。献忠见其礼数繁多,怒曰:`皇后何必仪注!只要咱老子\{毛来\}头哽\{口养\}得他快活,便是一块皇后矣。要许多仪注何用'?是时,摇黄贼自汉中流入川北;川中乱,且恐为献忠所屠,悉附之,其众日盛。摇黄原名姚黄,原系汉中士贼姚、黄二姓者为首;后其众既多,分十三枝,讹为``摇黄''。以袁韬为首,拥众十万;其余如呼九思、王昌、陈林、景果重、王友进、王兴、杨正荣等,各领数万。川北保宁、顺庆一带,悉为残破。

居民有力者聚众入山,负险结寨自守;其屠者悉据入营。张献忠亦不能问。献忠日肆攻战;川西州县□成都最近,又无兵,不胜其残暴,逃散殆尽。游击曾英,福建莆田人;隶抚院标下。其人通文墨,好交游。先剿摇黄有功,题授游击,守白帝城,总统十三隘,为抚院所制;兵不满一千。见献忠破夔门、陷成都,英料众寡不敌,退守涪州,募义兵于武隆、彭水。适有官解饷二万余过江津县,曾英谓其众曰:`此饷前去,必为乱兵所劫掠;不如取其饷以招募'。一月之间,得众十余万。曾英率其众,即恢复重庆、泸州、洪都、长寿各州县,军声大振。都司王祥有兵数千,亦附之。时宏光正位南都,敕东阁大学士王应熊为督师,赐尚方剑,率兵讨贼。应熊驻兵遵义,以曾英恢复重庆城,兵多樵采不禁;应熊乃重庆人,深怒之,欲加责让。而曾英抚大兵以御贼,应熊怒亦渐消。

乙酉四月,献忠命张定国、张文秀、王复臣等大合兵攻曾英;英率部将余仲、李定、王祥、李占春、余大海等分兵四击之,众贼俱败。捷音至遵义,应熊乃题奏曾英为总兵,王祥为参将,余仲、李占春、余大海、李定等为游击。而曾英兵日强,附之者益众。时巡抚马干率兵三万人驻内江县,参将杨展驻嘉定州,总督樊一衡亦领副将侯天锡、参将马应试等驻札泸州卫,副总兵屠龙率通、巴五营李正门等札纳溪县。八月,献忠命张可旺率兵攻乐用寨罗从义。乐用寨,本古蔺州奢崇明故地。天启初年,调奢崇明兵援辽;至重庆,举兵反,杀巡抚邵用春。朝廷兴兵灭之,改其地属永宁卫;而乐用寨有山最高,日经崖囤土,可屯万人,险峻不可攻。罗从义率五千精兵札其上,可旺兵至,□数月不能解;乃遣人往说之,从义举众降。可旺诱至成都,尽□之。

时献忠开科取士,会试进士得一百二十人。状元张大受,成都华阳县人,年未三十;身长七尺,颇善弓马。群臣谄献忠,咸进表疏称贺;谓`皇上龙飞,首科得天下奇才为鼎元,此实天降大贤助陛下,不日四海一统,即此可卜也'。献忠大喜。召大受;其人果仪表丰伟、气象轩昂,兼之年齿少壮、服饰华美;献忠一见大悦。左右见献忠欣悦,又从旁交口称誉;自顶至踵,色色详赞,以为奇士古今所未有。献忠喜不胜,赏赐金币、刀马至十余种。次日,大受入朝谢恩,面见献忠;左右文武复从旁誉其聪明学问及诗文字画一切技艺。献忠喜甚,召入宫赐宴,诸臣陪宴,欢乐竟日。临散,遂以席间金银器皿尽赐之。次早,大受复入朝谢恩;叩首毕,诸臣复再拜曰:`陛下龙飞之始,天赐贤人辅佐圣明,此国运昌明、万年丕休之象。陛下当图其形像,传播远方;使知我国得人如此奇异,则敌人可不战而服矣'。献忠大悦,遂召画工图其形像;又大宴群臣尽欢,群臣席间又极口称誉,献忠复赏赐美女四人及甲策□□□□二十人。次日,献忠坐朝,文武两班方集,鸿胪□□□新状元午门外谢恩毕,将入朝面谢圣恩;献忠忽颦蹙曰:`这驴养的,咱老子爱得他紧。但一见他,就心上爱得过不的,咱老子有些怕看见他。你们快些与我收拾了,不可叫他再来见咱老子'(凡流贼谓杀人,为``打发'';如尽杀其众,则谓之``收拾''也)!诸臣承命,即刻便将张大受\{扌邦\}去杀之;并传令将大受全家并所赐美女、家丁尽数斩杀,不留一人。是年冬,传令各府州、县考试生童,秀才三等以下童生不入学者尽杀之。丙戌春,复开科取士,生员不到者五家连坐,老幼俱斩;所属州县,无有不到者。至期,典试分考监临及各职事官并生员、供役人等俱入闱,闭门封锁。献忠即发兵万人,围贡院;不问官员、秀才及供役人军丁一齐诛杀,不留一人。

时贼党刘进忠驻兵遂宁县,与汉中相拒。汉中守将,乃马科也;马科原系李自成部将,陕西战败,投顺大清,领兵万余守汉中,将窥西川。进忠恃勇,颇轻科;而进忠不受约,私发部下兵袭汉中,与马科再战再败,折兵大半,仍归驻遂宁。献忠闻进忠败回,大怒,命伪翰林写敕让进忠。献忠一字不识,凡平日发敕书与群下必口述过;不论鄙恶,悉照其口语书之。如差一字,便杀代书者。是时进忠在遂宁,忽传朝廷有敕书至,即传合邑有司、乡绅、士民郊外迎敕至公所拜辞毕,命生员登坛开读,官民跪听;但闻其上高声读云:`奉天承运皇帝诏曰:咱老子叫你不要往汉中去,你强要往汉中去;如今果然折了许多兵马驴球子,入你妈妈\{毛皮\}的!钦哉'。文武士民俱向上叩首,呼万岁谢恩而退。进忠知献忠怒甚,料不能免;于是带兵连夜入汉中求马科投大清矣。

献忠此时方命张可旺、张文秀、王尚礼、狄三品、王复臣等领兵攻州南、嘉定等处,王应熊命杨展、顾存志、张登贵、侯天锡、屠龙、马应试连营犍为、叙州一带,可旺等连战不胜。五月,曾英、王祥、余仲等方整兵向成都,献忠侦知,急撤可旺等回川西。献忠见四面兵马渐逼、刘进忠又投大清,知成都不能守,乃分遣诸将带兵屠杀附近所属州县百姓,不论在城、在乡男女老幼务期尽杀,不许私留;人虽藏匿深山穷谷、悬崖险洞,务必千方百计取而杀之。一月之间,诸将悉回报功,各州县剿除尽绝,更无一人留遗者;然后尽搜成都城外乡间百姓杀之,次乃尽屠城中不余一人。然后拆毁城垣,放火烧尽房屋。

七月,乃拔营尽起,相率走川北,驻札西充山中,列四大营。每日清晨带数人登高埠,逼视诸营;或队伍不整、或旗帜参差、或器具不备,即并一营尽屠杀之。又恐诸将为变,辄以小册揣藏怀中,时取视之;喃喃自语曰:`若此,我事尚未得了,奈何、奈何'!又或时向天自语曰:`天教我杀,我敢不杀'!如是左右愚人偕信以为乃天使杀戮,不敢背叛。及兵马屠杀过半,其左右腹心如张可旺、张能奇等密问:`今上等好汉斩杀将尽,后将何以御敌'?献忠默然久之,曰:`皇帝极是难做,咱老子断做不来。今老子金银甚多,想来做皇帝不如做绒货客人快活!我今藏有银数万两、绒货数十挑、好驴马百余头,将此众人杀尽,我等心腹数十人搬驮金银、绒货前往南京做绒货客人,受享富贵,图下半世快活,有何不可'!众曰:`此事无论未必妥;即欲如此,便将众兵解散亦可,何必定杀尽'?献忠曰:`我面上有刀痕,军中谁不识我;异日撞见,定然漏泄。且数十万人相随,一时岂能脱去'。可旺等见其谋之拙如此,知事必不济;然畏其凶恶,不敢争。

至十月初,可旺与能奇、定国等将谋杀献忠,待日举事。忽刘进忠引马科由汉中出保宁袭献忠营,卒然而至。拨兵报有敌兵,献忠怒杀报者;次报又至,言敌兵将压阵,复斩之;第三次报至,献忠犹不信,自持枪上马出营观之。适进忠与马科冲至,进忠面迎献忠,指谓科曰:`此即张献忠也'!于是齐放箭射之,献忠喉中一箭,坠马死。大清兵直冲入营,诸将犹未知,一时惊溃。张可旺、王尚礼等率残兵五、六万人,由顺庆走重庆。

时曾英全军札营江上,数月前间献忠烧成都等处,率兵走川北,遂以为无事;王应熊等但知遣将收拾成都,侈言恢复之功,竟不防张可旺等败溃之兵从川北突至江上。且重庆附近各府县士绅商民避贼者,皆□□□英以自固,江上因而成市;水陆数十里,兵民相杂。卒闻贼至,未免惊扰;有望风先避者,人情恟恟不定。曾英命李定、余仲、李占春等率兵迎战;可旺等皆穷寇,恐大清兵后追,料无退步,乃奋力死战,李定等失利而归。曾英方欲整顿再战,余仲即入后营放火,劫本营马匹、辎重;各营见本营火起,以为贼至,遂大乱。曾英急率家眷登舟,舟重不可行;后军卒至争舟,曾英坠水死。余仲、李定、王祥等溃走綦江,散入南州县、真安州山中;李占春、余大海等浮舟下夔州。可旺连夜夺船渡江,破綦江县。督师王应熊驻兵遵义,巡按瞿\{日永\}亦按临,同住城中。丁亥正月初七日,瞿\{日永\}走真安州,王应熊亦率诸部将遁入毕节卫山中。正月二十三日,贼入遵义城。

献忠既亡,可旺等乃奉伪皇后陈演女为主,驻遵义桃源洞;可旺等诸贼每早必往朝贼后,凡事奉请而行,伪宰相汪某辅之。汪性□刻过于献忠,平日□以暴□媚献忠,□所欲杀汪必□□,故献忠最信之。诸贼衔之已久,然畏之而不敢发。至是,每公会议事,汪犹傲据诸贼上。一日,张能奇怒曰:`汝今尚敢如是耶'?拔佩剑斩之。二月初,大兵至重庆,诸贼将贼后焚死,拔营渡营乌江,走贵州。

时贵州守将定番伯皮熊闻贼至,走都匀;巡抚米寿图走偏桥,按察使唐勋、副使曾益走定番州,城中百姓逃窜一空。可旺等入城,出示招抚百姓;十日后,百姓悉回。可旺兵进取定番州;定番城中唐勋、曾益调土兵守城,贼至辄败去。张能奇自率众来攻,中药箭几□。贼乃书字射城内云:`数日杀吾兵将甚多;与我斗酒,当即退去'。乃退二十里。城中以为贼怯,守稍□。贼忽涌至,城遂破;唐勋、曾益自缢死。江津进士程三成时在定番,亦被杀。

二月终,大兵至遵义;可旺诸贼尽屠贵州,遂渡盘江走滇。可旺复姓孙,自称平东王;能奇复姓艾,称定北王;定国复姓李,称安西王;文秀复姓刘,称抚南王。时滇中方值土司沙定洲之乱,黔国公沐天波弃省城走滇中,竟无兵防;而诸贼逆得乘乱据滇矣。

附记:甲申正月,献忠自岳阳渡江,虚设伪官于江南;大队俱北,由湖南入川,陷之,瑞王阖宫被难,旧抚陈士奇死之。献忠取丁壮万余,□耳鼻、断一手,驱徇各州县:`兵至不下,以此为令。但能杀王府官吏、封府库以□,则秋毫无犯'。由是,所至官民自乱,无不破竹解甲投降者。献忠陷涪州、再陪泸州,顺流下重庆,进陷成都;蜀主阖宫遇害,巡抚龙文光暨道府各官皆死之。重庆推官王行俭死;总兵赵光远降,马士英犹请降敕奖之。献忠大索全蜀绅士至成都,皆杀之。既而县榜试士,诸生远近争赴;献忠以兵围之,击杀数千人,咸挟笔握策以死,蜀中士类俱尽。时中原多故,诸将无暇西顾;献忠遂奄有两川,据险设官。□入□,改元``义武''。置左右丞相,以徐以□为右丞相、潘独鳌为中书舍人。筑宫室台观,置酒自娱。及闻李自成败,逡巡不复出。其后,献忠被创死。

有自楚蜀□者云:`献忠谓蜀中绅衿反覆,尽行诛灭。既而考试童生,诡云选用汝等;诸童亦谓绅者既绝,必用吾。诣□□者万计,府县署前不能容,使往校场□试。午刻,□少年先完卷上呈,已而交卷者络绎而上。须臾,炮声轰烈,伏兵四起,突入场中,将童生尽杀之。维时或握管濡墨而死、或碎首断肩而死,又或折肱破腹而死,以至横竖倒侧,种种不一,惨不忍言'。夫献忠残恶因不足道,而士子争试亦自取其祸也!
献贼入蜀,蜀人拒战;献恨之,大肆杀戮。饮酒,将小儿抛掷枪上;儿啼,以为笑乐。有童稚杀不尽,则以大锄刈之。其残忍如此,蜀人大惧。有大在洞,内可以容人二十万匿于中;因不得入,置薪洞口焚之。吹烟入洞,众俱糜烂而死。

江阴沙尚宾在扬州与一兵会饮,熟视之;兵曰:`汝何视我'?沙曰:`吾闻食人者,其目必赤。今子目赤,毋乃食人乎'?兵曰:`吾曩年食五人矣。昔从献忠入蜀,蜀人畏之,□避匿深山,无所得食;遂掠人食之。惟女子纤足趾束最佳,如豕蹄然。时献忠每日发粮银一钱,而蜀中米每升值银八钱;若持得米二升,则粜一升,便食不尽矣'。

郭献珂起兵

甲申七月,张献忠遣伪将马科至四川招安、保宁一带,原任兵部主事郭献珂起兵战于桃园,贼兵溃;追获伪将宋朝臣,斩之。

\hypertarget{header-n77}{%
\subsection{卷十三·粤纪(续)}\label{header-n77}}

永历在桂林

戊子正月朔(丁酉),永历在广西桂林府;以朝臣星落,免朝贺。

永历走平乐

何腾蛟再督师出全州,兵益不睦;焦琏下平乐、郝永忠壁兴安。未几,永忠营被袭,疾至关,欲撤兵;左右禁近欲永历迁。式耜持不可;言`督师警报未至,营夜警无火,恐二百外风尘而遽使主露处、播迁无宁日,国势愈弱、兵气愈难振;民心皇皇复何依?潮回波逝,能逆挽戙\{木戈\}哉'!左右禁近周章不能止;式耜又请曰:`无已,侯督师归;果急,亲督甲士背城借一,胜败未知。若以走为策,桂城危,柳益危;若今日可到桂,明日亦可到南、太'。反覆数百言,泪下沾衣。严起恒曰:`迟至五鼓'!夜半,而永历已行矣。是戊子二月二十二日也。

``粤事记''云:`二月二十三日夜三更,有霍允中者(河南归德人,封永城伯)忽抢入大内,劫帝于寝被中,舁出城外;将文武百官捆吊勒索,尽逼所有,妻孥不保。瞿式耜亦被劫掠。帝虽裸体,幸无伤;只身走平乐府。桂林残敝,不可驻足;思三宫已在南宁,马吉翔备布袍、竹轿,掖帝而行。遇水濡足、遇岭披枝,可谓行路难矣'。

此与``遗闻''差一日,而被劫一事更详。嗟乎!天子裸体。时事至此,难言哉!

南、太,南宁府、太平府也。

瞿式耜复守桂林
当永历夜行时,肆掠蹂躏,公署职官无一得免;式耜被逼登舟。黎明,刑部侍郎刘远生、给事中丁时魁、万六吉及刘湘客俱至;盖湘客奉命安抚乱亡及劝饷糈,而远生、时魁以召将入也。遇式耜于樟木港;式耜集远生等入民屋,立草檄,分路四发。暂驻阳朔,催焦琏兵上援。楚镇周金汤、熊兆佐亦入桂,又檄翰林简讨蔡之俊、大理寺评事朱盛\{氵隶\}先入桂宣式耜令,檄按察司佥事邵之骅部琏兵,定人心。式耜于初一日复入桂署,督师何腾蛟自永宁至、滇镇胡一青统兵至、焦琏自平乐统兵至。大清兵疑桂城空虚,直抵北门。三月二十二日,腾蛟督兵三面御之;大兵渡甘棠去,督师列营榕江。永历诏旌式耜银、币又赐``精忠贯日''金图书一枚。式耜念南宁蛮乡不可久居,日为永历清道。

前日所忧在内者,今更在外;督勋镇将士直取全州,保巡抚鲁可藻下梧。会东人有反归信,令可藻缮兵以待。会可藻衔自署``两广''(旧例:东抚称制兼粤西,西抚称抚);式耜曰:`方今武人多自署抚军;予代疏请衔,曷不可'?周鼎瀚以阁部擅假,式耜亦疏正之。当武冈之乱,言官弹鼎瀚以附承胤入直;式耜司票,曰:`独不闻王沂公曰:``进贤、退不肖,皆有体''?瀚系大臣,应听自谢免、不谢免而复擅假,毋乃不可乎'!陈邦傅称``粤西世守'',牒四飞;式耜疏驳之:`今日功晋五等,尚未裂土。海宇剥削,止粤西一隅为驻跸之地;楚、滇数万之师日需食。辄曰``世守'',岂老成忧国所隐料'?式耜身虽在外,在廷大纪纲极言力请;疏曰:`臣与陛下患难相随、休戚与共,原自不同于诸臣;一切大政,自得与问。庙议可否,众指所关。本乱而求末治,未之有也'!

永历驻南宁

``粤事记''云:`三月初十日,永历入南宁府;加守道赵台巡抚衔,令专直大内食馔。随至者,内阁严起恒、吏科许兆进、兵科吴其雷、户科尹三聘、礼科洪玉鼎、礼科洪士彭、兵部尚书萧琦、大金吾马吉翔七人耳。帝欲进土州,萧琦上``十便、十不便疏''止之。君臣资斧,空乏实甚。起恒以首相兼吏部尚书下车之明日,亟收人心,悬示通衢:`民间俊秀愿立本朝者,悉陈履历姓氏'。即于三月十五日,广为开选;邕城通二十四土州,槟榔、盐布诸贾及土乐户皆注仕籍、列鸳班,借府学明伦堂为公座莅任地。自晨至暮,日以百计。鼓乐旗帜、轩舆扇盖,前人按套、伛偻罄折尚未遽毕,后官多叠趾以俟:如吴城、沙市等处极闹。礼生一时骄贵,以公座遽多礼,荣归展亲、谒祠、拜长更多礼也'。

萧琦,江西人;崇祯丁丑进士。

邕江,南宁府。吴城,江西巨镇。沙市,荆州巨镇。

皇子生

四月初一日(丙寅),世子生。王化澄请册为太子,赦天下;诏曰``万喜'',行在文武加一级。

土官升授

``粤事记''云:`时田州、果化州等土官来朝,行在文武各曲意款徇,冀得其欢心,以为异日东道主。土巡司皆升为邑宰、土邑宰皆升为知府,竟有道衔与土知府者。盖土司旧规,原加一等行事;以道衔与之,俨然开府矣。此三百年不破之格也'。

李成栋归明

``粤事记''云云:`四月初十日(乙亥),大清广州统兵固山李成栋将所辖广东、广西兵马、钱粮、户籍、土地悉归永历,遣帐下投诚进士洪天擢、潘曾纬、李绮等三人齎奏称臣,并请圣驾东跸肇庆为逾岭策应地。满朝惊喜,犹恐兵贵神谋,中藏诡秘;擢等极言李固山忠诚恳挚,跂足注望意。详询其归明之故,亦未甚悉;云``于四月初一日寅刻,悬挂奉朔改妆,示谕广州文武立刻更袄参见;时乌纱吉服、腰金象简,满堂为之改观''。初,成栋于丁亥二月收缴两广文武印信凡五十余颗,于中独取``总督''印藏之。有爱妾某(松江妓也;独携闽、粤)揣知其意,因故夕怂恿,成栋置不问。及今年三月三十日晚侍酒,复挑之;成栋抚几曰:``怜此云间眷属也''。盖成栋北来,家眷悉驻松江府城,故云此。妾曰:``我敢独享富贵乎!先死尊前,以成君子之志''。遂引刀自刎。成栋抱尸大哭曰:``女子乎是矣''!即服梨园袍带、冠进贤冠,四拜而殓之。将``两广总督''印具疏迎永历于广西南宁府;选洪天擢三人,令兼程昼夜行'。

``遗闻''云:`六月,粤东李成栋有反向明朝报至'。此载四月,盖志事之始也。

洪天擢,徽州人;崇祯丁丑进士,吏部侍郎。潘曾纬,应天溧水人;进士,大理寺正卿。李绮,松江华亭人;崇祯庚辰进士,广东督学。

附记:自吴胜兆败后,成栋镇松江。戊子春,率兵万人征广,余老弱二千及家眷居松。成栋归明,苏抚土国宝籍其家,得酒器三屋、妻属六十人;解南京马内院,拘徐国公旧府。每人日给米一升,所有兵以次散去;犹存兵妻二千不肯散,亦日给米一升。每成队而出,放马砍草,横行于松。

群臣复出仕

``粤事记''云:`四月十五日(庚辰),又有沈原渭者,再齎成栋速驾之奏至。知事果真,于是弹冠者遍地。王化澄杜门半载,忽入直矣;朱天麟变姓名隐太平府,走别窦邀拜相矣;晏清自田州出,为冢宰矣;张凤翔,兵科兼翰林院修撰矣;张佐辰与扶纲自贵州至,司文选、考功司事矣;顾之俊制中亦出随驾,上天地人三策、水火药三用矣;张起、王者友、朱士焜等各造一名邑,营考选矣;吴江县书役王正国,为吏部司务矣;董云骧,为大行人矣;潘观骏进兵曹、王渚户曹矣;庞天寿,掌司礼监矣;吴贞毓吏侍兼副宪,下广答谕李成栋矣。沈原渭,当即赐宴殿前,敕加都察院右副都御史。一时人情,咸以出仕为荣、不仕为辱矣'。

沈原渭,苏州吴江人;生员。朱天麟,吴江人;崇祯庚辰进士。晏清,湖广黄冈人;进士。张凤翼,庚辰进士。张佐辰,进士。扶纲,贵州人;崇祯甲戌进士。顾之俊,苏州人;癸未进士。张起,苏州举人;户科给事中。朱士焜,靖江人;贡生,吏科都给事中。董云骧,松江生员。潘观骏,湖州生员,武选主事。王渚,池州布衣;户部主事。

章服错乱

``粤事记''云:`四月二十日,又下考贡之旨。村师、巫童以及缁衣、黄冠凡能握管书字者,悉投一呈曰山东、山西某府县生员,必取;极远以为无证。拽裾就道,弥漫如蚁:曾经出仕,佥曰``迎銮'';游手白丁,诡称``原任''。六曹、两侍,旬日之间,驻列济济。然相遇道左,各不举手;为有一、二科甲在内,故凌气质以自尊。余如菜佣、屠夫、倡优、书役,虽冠进贤冠,行行队队,若羞见人。维时无故,或吉服、或衣锦、或卑末而用大蓝翦绒靴。至于章服补带多未合式:或补鹤而带银、或带金而补雀、或带黑而角四云大红石蓝;且有官不如其带、品不如其服者。凡站立位次、称呼礼貌,俱未之娴;文武错佩、大小倒置,满朝皆无等威:攘臂脱肩、半襟马裾。新创朝廷,遂成墟市;严起恒不得不任其咎。喧嚷两月,左、右二江人不称官者少矣'。

地少官多、朝小官大;自古如此,于今为甚!

两粤复全

``遗闻''云:`瞿式耜念无讲官,经筵不御、石室尘封,何由闻得失;手书``八箴''于扇进之。何腾蛟复全阳;是五月二十七日事也。腾蛟报功疏,不肯自为功;有曰:``为陛下以信臣、用臣者,式耜一人也''。时李成栋具疏迎驾;又江右金声桓据南昌,藏表疏于佛经部面中遣使齎奏亦至:两粤俱称全土。式耜疏请永历往桂,又请勿遽东;又言``事权宜专、号令宜一。兹军功爵赏、文武署置,决于成栋。若归之朝廷则中扰,阃外不能专制;听之,朝廷徒虚拱。且楚、黔雄师百万,腾蛟翘首威灵,如望云霓。驾既东,军中将帅谓朝廷乐新复之土,成栋亦有邀驾之嫌;号令既远,则人心涣散。请一见东诸侯,俾共瞻至尊音容,面为慰劳指属,然后责其尽意于东,刻期出战,咸决于外不中扰也''。又令简讨蔡之俊入迎;再疏,令给事中蒙正发迎。永历竟由梧入肇,先后诸疏俱不报;式耜闻,泪簌簌下。再疏谓:``前月粤东未复,宜住桂以视楚;今日江、广反正,则宜住桂以图出楚。事机所在,毫厘千里''。吏部侍郎吴贞毓疏请永历住广城,式耜乃促远生入阻永历。适成栋自岭还师,修行宫,且迓驾。永历命远生诣广劳师,远生谓成栋曰:`今驾至此,爵赏、征伐,人疑有私;不可不嫌'!成栋然之,遂罢修行宫、止迓驾。成栋具疏,言`式耜拥戴元臣,粤西扼御定;毋容久于外,应亟召还纶扉'。永历专命遣官三、四召;式耜曰:``前日在南宁,桂林危;桂林危,则天下去矣。其机在外不在内也。今江、广悉定,何公督师下星沙,朝臣且辐至;予不敢忘危而即安''。具疏乞骸再上,不允'。

蒙正发,湖广举人;户科都给事中。

朱天麟邀相

``粤事记''云:`朱天麟,昆山人;出自羽衣,庚辰进士。是时以知推行取高等,竟入翰林。思庙阅司李访单,拗取无圈点者为上选,天麟遂入翰林。丙戌九月,由闽入广,独携家属舟过肇庆。会永历登极,诸臣适欲觅一老词臣为朝端重,共迎□之。天麟绝维而去,变姓名,隐居广西太平府之云山。至是,知成栋归明;适太监王保入山置市苏木,天麟故为款纳,礼极恭,使具奏。王保回朝,极口荐之;且详述其留待意。内阁严起恒、王化澄忌有三人,以``该部知道''还之。天麟又求庆国公陈邦傅特疏荐曰:``三朝元老,中兴柱石;今潜修太半,此天心欲留伊、吕再造股肱以佐陛下。主政者终不欲密勿之地权分异己,量拟以宗伯召''。时为前六月朔;越三日,天麟见朝。宗伯篆王化澄兼摄;初五日,化澄以策送天麟,天麟固不受;往返推辞,自旦至暮终不允。科道两衙门传揭曰:`天麟以阁臣荐,岂宜授之宗伯;今当合词以阁臣请'。明日疏上,仍以`该部知道'四字还之。于是,天麟勉受宗伯。不三日,营长子日生为侍御,令掌河南道事;次子月生为中书。其弟天凤为大行人,雇乡兵四、五十人执戈揭旗以从;云为将来出将地。凡会客叙谈,日昃夜分葛藤不了。一门可笑如此'。

王保,应天人。朱天凤,乙酉福建举人。

陈邦傅留永历浔州

闰六月初十日(甲辰),永历与三宫邕江登舟出南宁,历横州、永淳。时以急流,两日夜即抵浔州府,有旧总兵庆国公陈邦傅挽留诉功。初,永历驻南宁三月,邦傅不敢入觐;以与朝臣不协,又与赵台有隙。台本北京人,任子也;擢南宁知府。甲申京变,只身宦南。时邦傅总镇粤西,依之为亲,为后日身家计;曾经面订,未行吉礼。邦傅亦屡疏荐扬,台之得擢,颇由于此。后邦傅见驳于式耜、不理于朝臣,台欲绝之,至形之章奏以博众欢。邦傅因宣言圣驾下广,台必随扈;路出浔江,吾必掠其妻女、杀其父子。台闻之,遂不敢从,留子南宁旧署。至是,永历过浔,邦傅留之;面称`帝忘大恩,听两衙门交构,于本爵无少加恩。倘丁亥二月,梧、浔等处无臣父子血战三昼夜扼南下之兵,长驱直捣,皇上焉有今日!至赵台赖婚负义,法所当诛,皇上反加优容;彼不敢经臣河下,何得任为留守!且南、太等府原系臣镇辖下,何必再设巡抚!明系两衙门受贿,计设蒙蔽。望皇上大奋乾纲,毋为文武作奴仆,饱彼私橐;后日将士解体,身受实祸'。永历惭愧,面赤无答;但云`你补本来'。又于随驾诸臣,略不为东道主;反向户部主事王渚索饷,拳殴而死;顾之俊愤极而死。严起恒、王化澄两相以恶草食进,相见无和颜温语。又面□兵部尚书萧琦不遣兵护驾,率家丁乱石碎其舟,舟半沈;复不容琦登岸,蹲踞水舱,遣村儿、野妇环坐而辱詈之。琦愤恚,蒸闷三日,卒于舟次;邦傅取其舟中所有,复不为之殡。

二十日(甲寅),邦傅逼永历上浔州府,驻府署为行宫。邦傅挟朱天麟同严起恒、王化澄知机密,因广为给发札付;始而庆国自札,继而部札,后贵钦札。钦札者,玉玺札官,知县、知府、科道、翰林以至侍郎、尚书,武则正副总兵、游击、参将,使之执札到部照受实在衙门;故陈乞纷纷。尚书正总钦札,亦可易百金;部札半之:下广路费需之此也。

是年永历闰六月,而大清朝闰四月。

横州、永淳县,俱属南宁府。 张立光受贿换敕

七月,永历驻浔州,允陈邦傅居守浔州府如桂林瞿式耜,设官征赋。敕下中书誊黄,中书舍人张立光受贿二十金,遂以``世''字易``居''字,用玺助卿亦不及察;惟永历觉之,曾微言于严起恒,令行在诸臣发其事,欲追前敕更正``居''字,并提立光拟``擅改敕书''罪,竟不果。

立光,苏州昆山人;生员。

晏日曙四臣殒身蛇庙

广州、肇、梧州千余里间注望圣驾,杳不可得。晏日曙、李永茂、田芳、郑封等俱中土产,性恶湿热;又潜伏深山一载余,岚瘴之气浸入肝膈。至是,各舣舟以待于苍梧城下府江、藤港合流处三角嘴挽泊;西望眼穿,满腔欲控。闲游蛇庙避暑纵谈,四人各喘急暴病,相继而亡,竟弗获面圣略倾积悃。亦因热中瘴发,炎蒸不耐;俗云``等人躁急''故耳。

叶子眉朝歌逆旅题壁

`马足飞尘到鬓边,伤心羞整旧花钿;回头难忆宫中事,衰柳空垂起暮烟'。

妾广陵人,从事西宫;曾不二年,马上琵琶,□尘远去。和泪濡毫,语不成声。怆怀赋此,幸梓里同人见之,知浮萍之所归耳。广陵叶子眉题。戊子七夕前二日也。
永历再入肇庆
七月二十五日(戊子),陈邦傅为李成栋促请圣驾,辱詈不堪;不敢挽留。是日,离浔州。二十九日(壬辰),至肇庆城下。成栋先整督学道船长三十三舱者为龙舟,百里外迎接;上慰劳备至。八月初一日(癸巳)辰刻,成栋率文武百官迎帝,手扶銮舆入肇庆行宫。宫中储银一万两以备赏赉,供帐器饰复约万金;象魏殿陛亦粗可观。朝贺后,加成栋卫公(?)爵极品;赐御袍靴带、尚方剑等。成栋再拜,谢。时首相严起恒,次王化澄、次朱天麟;凡政之大小行止,必呈成栋而后奏。

李成栋出师

李成栋,陕西人。为人朴讷刚忍,无衿意、无喜容,不脂韦、不多言;内外文武悉敬畏之。永历命筑坛城东,效汉淮阴故事,令督师南下。坛半就,成栋曰:`事在人之为耳,岂必坛之登与否乎'!盖刎颈爱妾刻不去怀,必欲得当以答其意也。八月十二日(甲辰),面奏永历曰:`南雄以下事,诸臣任之;庾关以外事,臣独有之'。一言竟,去;提兵二十万上南雄。时江西金声桓据南昌,日通成栋约期南下矣。

朝臣媚李元胤

自八月十二日李成栋去后,朝局大变。都察院左都御史袁彭年向为周延儒腹心,延儒议处,首揭延儒;后降大清,授广东学道,示云`金钱鼠尾,乃新朝之雅制;峨冠博带,实亡国之陋规'。及返明朝,又复诋之,矜反正第一功臣。成栋有养子李元胤,本贾姓;河南人。以庸蠢,不携行间,留肇庆守家。彭年特隆其体,以内外权属之。元胤为傀儡,笑骂无情;彭年为线索,机权刺骨:一时政事人心,乖离殊甚。吏部侍郎洪天擢、大理寺正卿潘曾纬、广东学道李绮、兵部尚书曹晔、工部尚书耿献忠、通政使毛毓祥为成栋所亲爱,皆自五羊来,为一种;严、王、朱三相国、吏部尚书晏清、吏部侍郎吴贞毓并杂项出身六科吴其雷、洪士彭、雷得复、尹三聘、许兆进、张起等皆南宁随驾西来,为一种;又有从各路至者都察院副都御史刘湘客、礼部尚书吴璟、吏科都给事中丁时魁、兵科都给事中金堡、户科给事中蒙正发、礼科给事中李用楫、文选司郎中施召征、光禄寺正卿陆世廉、太仆寺正卿马光、仪制司郎中徐世仪等,又为一种;其广东本土人陈世杰词林、吴以连验封司郎、李贞省垣、高赉明西台、杨邦翰□卿、唐元楫方郎等,亦为一种:种种望风,归入一党。彭年招引同志,驱除异己,于是元胤之门如市。登其堂者,不啻龙门;拜盟认宗,李氏、贾姓莫不矜喜。每当朝期,东班趋入西班与元胤交接,东班为之一空。元胤为人暴戾自用,狂率不情;客至不揖、去不送。喜遗仆卒与客赌博,诸臣倾囊奉之,谑浪骂座,弗忌人讳;皆彭年胁制内外曲徇以成之者。中郎无子,洵不诬矣!十月初九夜,元胤奉成栋密计,题请兵部尚书佟养甲往涪州府祭告兴陵。二鼓时,于德庆道上杀养甲;地方官以盗杀报。由是,威权益震。本月十五日,元胤诞辰,在朝文武公分之外,有私分;私分之外,有私公分;私公分之外,又有私私分;馈遗昼夜络绎不绝。自八月至冬杪,莫不奔竞于元胤左右:可叹也!

彭年,湖广公安人;崇祯甲戌进士。毛毓祥,武进人;丁丑进士。吴璟,原名文瀛,松江人;进士。施召征,无锡人;癸未进士。马光,苏州人;保举湖南总督。徐世仪,江南人;升文选司郎中。陈世杰,进士。吴以连,丁丑进士,验封司郎中。李贞,进士,吏科给事中。高赉明,进士。杨邦翰,进士,太仆寺少詹。唐元楫,丁丑进士,职方司郎中。佟养甲,辽东人;都督同知。

李成栋庾关初败

十月二十日(辛亥),李成栋过庾岭。二十五日(丙辰),于江西赣州府城外结营,闻内外已通。二十六日(丁巳)五更,闻城上呼``董大哥''者三。成栋于梦中惊醒曰:`董大成是我中军;彼呼之,我军已为彼有矣'!亟披蓝布短马衣,跨一骡疾走,竟不发一言;庾关至梅岭六百里,两昼夜奔蹶大雨中。初出关兵二十万,分为十大营;每营立一大总镇;成栋弃军走,十总戎亦尾之而行。及进南安府城门,成栋如梦初觉,顾谓十人曰:`尔等何得随来'?十人对曰:`大爷既走,吾辈不得不来'。成栋怒,以为谬;即手刃爱将杨大用。二十万士卒、器械悉弃赣州府城外,止与百人南来;亦羞入肇庆面君,遂顺流直下广州府为再举计。朝中得报,亦不甚异;仍封诰升转,仕籍纷纷,竟不计及外御、内修者。

董大成,河南人。杨大用,陕西人。

赣州府,属江西省;南安府,江西进广处。

以二十万众大帅,岂无故独走?即十将,亦岂竟不知所以,随行两昼夜,独不得成栋一语而奔乎?此必有说。姑志以俟考。
文选给空札

九月、十月,通政司疏陈乞职者日以千计。阁臣票拟刻版,定``着议具奏''四字。吏部堂司两庑,拥簇挨挤;文选司虽掌铨选之权,无出选之地。广东一省,非奉成栋咨,大小有司不得擅为除授;桂林、平乐,则留守阃臣瞿式耜为政;庆远、柳州,则焦琏为政(琏者,陈邦傅出京时京债主);浔、南、恩、太四府,则庆国公陈邦傅为政。又选所副乞陈之望,第给一空札为后日到部凭据而已。

陈邦傅围南宁

十月,邦傅率兵至南宁府,围城二月,斗米一两;活剥媒人钱廷晔。赵台莫能支,开城降;涕泣出女,与邦傅为媳。邦傅父子遂驻南宁府。

钱廷晔,无锡马侨人。

贾士奇辱施召征

十一月朔(辛酉),文选施召征谢恩时,同班舞蹈者二人:一为本兵曹晔、一为銮仪司贾士奇(□为现通谱李元胤)。召征未揖士奇,士奇大怒,指名辱骂。召征初未晓,及恶声出始觉;诿云短视,当诣门请罪也。睡亦解纷,士奇稍舒。至初四日,遇于道,攘臂欲殴。召征避让,细问其故;士奇见召征他处刺名颇小,而请荆字独大,以为欺之也。召征杜门三日,几费调停,代款四两始息议。无非假元胤而恣肆耳。

士奇,湖南人;初为守备。

吴其雷宵遁

十一月初六日,兵科吴其雷具疏``清文武之职掌,以肃朝纲;励新旧之廉耻,以别人品等事'':`内阁六部、四衙门,总兵以下移会用手揭,此三百年之旧规。现文武诸臣有初朝、二朝、三朝、四朝、五朝、六朝之分别,各宜建立为功,以昭靖其自献之本心'。袁彭年、李元胤知之,恨入骨。初十日,其雷宵遁,上桂林府与瞿式耜共事。疏亦留中。

朱容藩僭乱本末

朱容藩,本楚藩通城王派下一庶人;居家无赖,不齿于王府。逃入左良玉军,假称郡王;引兵害人,营中诸将恶之。甲申春北都既陷,容藩至南京,贿马士英请以``镇国将军''监督楚营。行至九江,以横恣激变,军民惧罪逃奔。时李自成溃于陕,余贼流入楚。容藩复入贼中称``楚王宗子'',贼大喜,欲立为王。后见举动乖异、语言虚诈,因疑之。

丙戌十月永历即位端州,容藩走赴行在,言贼中情形甚悉。内阁丁魁楚素庸陋,信其舌辩,遂荐之朝,掌宗人府事;得参大政。而兵科程源喜谈兵,与之结纳甚欢。程源,四川人也;一日与容藩谈川事曰:`川中贼势虽盛,而诸将兵不下数十万;吾两人各请总督之职,公督东北、我督西南,贼不足平也'。容藩喜,具疏请之。朝议以程源方改兵科未及两月,不庆升迁;而源狂躁,意在必得。乃加源太常寺少卿,经理三省;而改容藩为兵部右侍郎兼右佥都御史,总督川东兵马。十二月,大清兵破广东,永历移跸桂林;内阁吕大器回四川,丁魁楚率子女、辎重由浔州走横江小路,内阁随驾者止瞿式耜一人。丁亥正月驾至广西桂林府,朱容藩觊觎入阁办事,逗遛不行;嘱给事中唐諴等连名具疏,参丁魁楚私逃,上护从单弱,如容藩、程源皆系拥戴重臣,不宜轻出外。上素恶容藩,因怒曰:`尔等又欲拥戴容藩耶'?遂命削容藩职,夺其总督敕印,欲斩之。容藩惧,披剃为僧;贿买内监庞天寿求太后谓上曰:`变乱以来,宗室凋零;容藩罪不至死,毋过求'!上素孝,承太后懿旨,即赦容藩;仍复其官,还与敕印。容藩由楚入川。程源由楚入黔,假称三省总督、兵部右侍郎,沿途卖官送札,赃私巨万。四月,四川巡按钱邦芑具疏参之。时上以三月刘承胤迎驾跸武冈;疏入,上震怒,削程源职,逮问。

容藩由辰州入永顺司至施州卫,得王光兴兵马二万人。时光兴为大清兵所击败于郧阳,走入施州卫,无所归;容藩假称``楚王世子、天下兵马副元帅'',光兴诸将不知其伪,遂附之。时川中曾英为张可旺(即孙可望)所败,部将李占春、余大海率舟师东走夔州;容藩既得光兴兵,即移檄占春、大海,两将亦归之。时大清兵既破成都等处,由重庆泛舟而下;容藩命占春、大海截之。二将以舟师溯流而上,七月十一日相遇于忠州之湖滩;占春出不意,以轻舟直入其营,发火炮乱击之,大清兵弃舟登岸,走川北。容藩得三营兵马,益复恣肆妄行;遂称监国,铸副元帅金印佩之。改忠州为木定府,号府门为承运门;称所居为行宫。设祭酒、科道、鸿胪寺等官,擅封拜:王光兴、李占春、余大海、杨朝佐、谭弘、谭文、谭诣、杨展、马应试等为侯、伯;以张京为兵部尚书、程正典为四川总督、朱运久为湖广巡抚。

八月十三日,钱邦芑率王祥复遵义;九月,檄袁韬复重庆,川北总督李干德同袁韬兵驻重庆。十一月,容藩率李占春□营至重庆,会干德;容藩讽干德,欲其拥戴己。干德若不解者,而礼复不相下。适冬至行朝贺礼,袁韬本摇黄贼,初受抚,素不知礼,乃与容藩同班拜舞。容藩怒,命李占春暗袭袁韬并害李干德。是晚,李干德登舟中,忽觉烦闷,遂登岸于高埠,设帐安息;及占春兵至,掳干德舟中,止得一妾、一女,不见干德,乃大惊。及袭袁韬营,复不能胜。次日,袁韬集兵与占春相雠杀,数战不解,互相胜负;容藩走涪州。时钱邦芑在武隆、川南总督杨乔然在彭水,容藩移书邦芑、乔然,请至涪州为两营解释。邦芑复书,深责容藩僭逾之罪;容藩不从,乃私铸锦江侯印送王祥,求其以兵应占春,同战袁韬。戊子正月,王祥以兵出綦江,与袁韬三战不胜,退札南岸。祥独乘轻舟见容藩,少顷占春来相会,王祥力大,遂擒占春过其舟,同至綦江县;命部下王朝兴守之,不听归营,欲并其众。朝兴,陕西人;与占春同里。占春慰以好言;时苦寒,占春解貂衣赠之。朝兴悦,守为之懈;占春乃夜逾城出,得小舟逸归营。王祥既失占春,战袁韬复不胜;兵无粮,杀马而食。二月,遂回遵义;袁韬亦走顺庆,占春退札涪州之平西坝。

夔州临江有天字城,甚险,可守;容藩乃改为``天子城''以为己谶,领部众数千居之。铸印,给诸部下。石砫、酉阳两土官,俱封为伯,挂将军印;厮养蛮獠,俱授监军、总兵之铸职。川中屡经张献忠、摇黄之乱,地方新复。丁亥武冈之变,上由靖州幸柳州,干戈阻道,朝廷文告久不通川中;容藩乘机煽惑川东一带,诸将士多为容藩所动,竞往归之;求官授职、无虚日。钱邦芑乃列其罪,疏劾云:`为奸宗谋逆,请正天讨事。臣察得逆宗朱容藩自元年正月在广西得罪,皇上欲置之死;幸天恩赦宥,还其原官,命料理湖南一带。彼时寇逼湖南,容藩即由施州卫走入川东。五、六月寇陷涪州,臣方至彭水界上。川东夔府一带与朝廷消息不通,文武无主;容藩亦以川中消息不通之故,遂乃假朝廷之威灵,收拾兵将。至八、九月间,川中各镇如王祥、侯天锡、李占春、余大海、赵荣贵、曹勋、马应试、袁韬等各出兵剿寇,四路捷报。维时皇上幸广西,川中不知圣驾所在;容藩即自为吏、兵两尚书,铸刻印信,选授文武,笼络军民,隐有称王之意。今岁六月臣巡川南,忽军□传来朱容藩刊谕建置文武榜文,其自称则曰``予一人''、``予小子'';如此而欲其终守臣节,其可得乎?今皇上远在百粤,四川僻在极西,沿途兵寇阻道,凡诏谕、敕旨经岁余后通,其浮沉不达者尚多。且四川之地,四围蛮夷土司,易生反覆;又迭经寇祸,三年之间四易年号,人情惶惑,莫知适从。故容藩欲乘此摇动人心,谋为变乱。自去岁秋、冬川地渐复,臣不惮艰苦往来深山大箐、荒城破垒之中,驱除豺虎、翦披荆棘,招集残黎、抚慰土司,宣达皇上威德,四川之地始知正统所属。今声教渐着、法纪方行,而容藩包藏祸心,谋窥神器;阳尊朝廷,阴行僭伪。假皇上之威福,布党乱之爪牙;意待羽翼既成,便欲盘据四川,以为公孙子阳、王建、孟知祥之事。臣已早窥其隐,先致书告以大义;随即传檄楚督何腾蛟、堵胤锡、〔杨〕乔然、李干德及各大镇,俾共尊朝廷,勿为叛臣所惑(语不具载)'。邦芑乃封疏稿、檄文达书于堵胤锡,期合兵共讨。胤锡率马进忠驻施州卫;胤锡得书,即乘舟入川会容藩,正色责之。容藩曰:`圣驾播迁,川中不知顺逆;聊假名号弹压之耳'。胤锡呵之曰:`公身自为逆,何能服叛逆乎?钱代巡有檄会兵;若再不悛,钱公率兵下,吾截其后。川将皆朝廷臣子,谁为公作贼者'?时文臣附容藩者,止张京、程正典、朱运久三人;胤锡一一陈大义切责之。川东文武始知容藩名号之伪,各各解散。

八月,督师吕大器至涪州,李占春迎见。适容藩有牌至,期诸将会师,上列``楚王世子监国天下兵马副元师''之衔;大器笑曰:`副元帅,非亲王、太子不敢称;且天子在上,何国可监?此人反叛明矣'!占春曰:`昨堵督师面叱其非,某等已知其伪。然朱千岁犹铸印封拜,奈何'?大器曰:`容藩专擅如此,朝廷即檄兵会讨。尔等受其官,必不免'。占春曰:`误为所惑,今已悔之。讨叛以赎罪,若何'?大器唯唯。占春即整舟师,连夜至天字城攻容藩;容藩以兵相拒不胜,走入夔州山中。占春率部下穷追两日,容藩匿草舍中,为土人擒献;斩之。川东之难悉平。
武冈播迁始末

刘承胤,本南京一市棍,投兵部为家丁。有膂力,酗酒无赖,自号刘铁棍。后随营至楚以征蛮獠,累功至副总兵。甲申北都既变,何腾蛟总督楚中兵马,题承胤总兵官,镇守武冈;招集兵二万人,大半南京市棍。承胤实刚愎不知兵,以``铁棍''之名哄动远近;腾蛟亦误信之,与联姻。丙戌七月,隆武封为定蛮伯,愈恣肆。兵科龚善选以册封李赤心过武冈,承胤令兵辱之。

丁亥正月,永历驾跸桂林,承胤具疏迎驾。二月,兵科给事中刘尧珍过武冈。时太仆寺卿郑逢元以催粤饷驻武冈,见承胤兵盛,与承胤联姻。先是,沅州有妖僧查显仁假称宏光,常煽动地方;逢元亦具表称贺。至是,刘尧珍语讥之。逢元惭怒,以告承胤;次日,对众拳殴尧珍。锦衣卫指挥张同敞至,与御史傅作霖责承胤曰:`尔具疏迎驾,而得罪朝绅,何也'?承胤不自安,乃具酒请罪。后上驾跸全州,见驾,倨侮无人臣礼;御史瞿鸣丰疏劾之。次日,朝退至门外,承胤指都察院御史杨乔然曰:`汝任风宪之长,近日言官混杂妄言,汝不能表率,要汝何为'?因以拳挥之;乔然与相争,至裂冠毁裳,众为劝息。乔然、鸣丰具疏请罢斥,杜门不出;承胤回武冈。

三月,车驾幸武冈,以岷府为行宫。进封承胤为武冈侯。以工部尚书吴炳为东阁大学士,入直;以贵州总督李若星为吏部尚书、黄太玄为太常寺卿、侯伟时为文选司主事;改吏科唐諴为右春坊右谕德,以御史傅作霖为兵部左侍郎、管部事,加吴贞毓太常寺卿、仍管文选司事,加严起恒户部右侍郎、管布政司事。锦衣卫指挥张同敞,江陵故相张居正之曾孙也;威宗以任子官中书。北都陷,同敞怀牙牌间关入闽,隆武命袭锦衣世职。至是,举朝荐其才可大用;上素闻其贤,改翰林院侍读学士。太仆寺卿郑逢元以承胤姻亲,升兵部右侍郎,总督川、楚军务。以刘达生为太仆寺卿,升翰林院简讨刘湘客为右春坊右谕德。改礼部主事刘鼒为翰林院庶吉士;寻复改御史,加给事中。以庶萃士刘鲁生为编修。刘远生、刘湘客、刘鼒、刘鲁生四人皆以刘姓,与承胤认同宗兄弟,故皆得美官。远生原任江西巡抚,丙戌年为清兵所执,逃回失官;至是,得复用。湘客,即其弟也。刘鼒,四川人,颇能诗文;喜交游。刘鲁生,楚人;丙戌五月,以贡生入闽廷试。隆武见翰林诸官皆不知书,每问故事,瞠然无以应;及命撰文,又浅俚不通。因选贡生二十人,命为萃士,附翰林院读书,准同庶吉士服。三年后,再加考试,如文理果通,方淮实授庶吉士;倘文理仍前不通,即与选州县等官。刘鲁生时亦与萃士之选;及福建之变,鲁生走回楚中,遂自称庶吉士也。又因六月间隆武生太子推恩,群臣各加一级,鲁为庶吉士加一级,乃简讨也;遂自称翰林院简讨。朝廷当变乱之后,无能核其来历;兼鲁生谄交承胤认为同宗兄弟,即有知其非者,莫敢与辨,而鲁生又居然编修矣。四月,加巡湖广堵胤锡兵部右侍郎,总督恢剿军务。先是,胤锡招抚李赤心、高必正等十营兵颇强,驻札楚界,声言欲入湖南就粮;承胤惧为所并,以为非胤锡莫能制。于是加胤锡衔,敕其督兵出江西。

五月,承胤骄横日甚,动辄以兵挟朝廷;群臣畏其刚暴,争谄之以自固,交疏颂功德。遂进封兴国公、上柱国,赐尚方剑、蟒玉,便宜行事。承胤止二子,五月间以功荫锦衣卫世袭指挥者七。承胤亦侈然自以功盖古今,莫之与京矣。六月,督师大学士何腾蛟闻承胤专擅,颇不平,愤欲制之。承胤原系腾蛟荐为总兵,遂称门生,稍倨肆;后联为姻亲,竟不受约束。及上幸武冈,遂挟天子作威福。既得上柱国之衔、赐尚方剑,意欲与腾蛟抗衡并驾。忌其权出己上,乃上疏请改腾蛟户部尚书,专理粮饷;上不允。腾蛟辞朝,归镇长沙;上赐纻丝、金币,敕廷臣郊饯。腾蛟托病驻城外荒寺,不言行期,盖防承胤也;承胤果伏千骑于道中,欲暗害腾蛟。时赵印选领滇兵五百随腾蛟,皆悍卒也;行至中道,伏兵起,印选率部下尽歼之,承胤讳之而不言。时总兵张先璧自江西溃入楚,兵犹数万,欲入朝;承胤素怯先璧,请旨勿许。先璧怒,驻兵武冈城外;承胤闭门,出城与战,屡为先璧兵所败。承胤詈先璧以犯阙〔先璧〕詈承胤以劫驾,相持月余不解。上命兵部主事龙之洙奉敕往解之,先璧奉敕退兵,往札沅州。七月,督师堵胤锡劾承胤专擅截杀腾蛟,因率高、李诸营出江西;承胤见胤锡疏,始知惶惧。上加胤锡兵部尚书,赐尚方剑,总督江、楚军务,专辨恢剿。

八月,大清兵破常德,留守广西大学士瞿式耜请上移跸桂林。上召承胤,茫然无策;但强言`我兵多,他决不敢来'。越数月,警报迭至,人情汹汹;承胤与部下密议投降。上觉之,与辅臣吴炳议由古泥幸柳州。二十五日,上奉两宫太后先发,上及中宫随行;至二渡水,车驾甫过、诸臣渡未半,而浮桥遂断。凡无马者,追随不及,皆被乱兵劫杀;李若星走黔阳、张佐辰走平溪,余多流匿白云诸山。上过木瓜桥,迷城步小路,循大道竟抵靖州。内阁吴炳、吏部主事侯伟时走城步县,大兵追及,二人被执,不屈死之;御史刘鼒疾驰,得免。九月初一日,上次靖州,乃由古泥幸柳州。刘承胤虚声恐吓,及上出城,遂不守、不战,惟议降。兵部侍郎傅作霖勃然大骂曰:`吾始以汝为人,今乃知汝狗彘也!汝迎驾至此,挟天子作威福,惟所欲为,富贵已极。一旦有事,束手无策,致天子蒙尘,罪已不容于死矣。且汝拥兵数万、糜饷十年,平日夸口天下莫当;今不谋战守,先议迎降,真狗彘不如也'!承胤不顾,早命人远迎大清兵;及入城,作霖冠带坐堂上大骂。时偏沅巡抚传上瑞已投顺,与承胤再三婉劝之;作霖唾其面,遂遇害。作霖妾郑氏有殊色,作霖甚宠之;既为大兵所执,求与作霖一面不许,遂从马上跃入桥下水中而死。承胤既降,全营诸将及家口数万人同移至武昌,驻札汉口之后湖。

十二月,承胤部下副总兵陈友龙亦投降,带兵追驾至浔州,忽然反正;报至武昌,大清朝疑承胤与友龙通。至戊子四月,以兵围承胤营,并傅上瑞家口百余,不问老幼、男女尽杀之;五、六万人敛手受害,无一免者。

雷雨风雹

己丑正月庚申,永历在肇庆;大雷雨风雹,群臣免朝贺。

元旦而有雷雹之异,天之警之也深矣!殆何、李败没之兆乎? 李成栋驻军信丰

李成栋于去年十一月返广州府,整顿士马、储备粮械。十一月二十七日(丁巳),具疏遥辞永历,再上南雄府。及今年正月初五日(甲子),于滇阳峡中白日闲坐,忽见所杀爱将杨大用持刀索命。成栋举弓射之,身随弓去,坠入江水;急为救援,神情惨急,英勃之气十减五、六。自是,不敢逾梅关;枉道东旋,驻军信丰县界。

南雄,属广东。滇阳峡,在广东韶州府之英德县界。信丰县,属江西赣州府。

科道击陈邦傅

行在科道两衙门迎合彭年、元胤意,正月、二月以攻陈邦傅为正课;其余国政,无人谈及。陈邦傅,浙东处州府人;崇祯末年,广西总兵。隆武二年春,挂征蛮将军印。成栋素恶之;以其丁亥年二月大清兵未入粤西,先有降表到广州通款故也。后成栋为陈子壮、张家玉乱,西进之兵撤回广城;邦傅得安浔、庆二载,自侈以为功。丙戌之夏,亦预谋靖江王下广,事幸未□。今竟谓浔、庆、南、太未经薙发,勋比汾阳,借以陵人;不亦诬乎!

科道散朝

时攻击陈邦傅科臣中,惟金堡为最劲。邦傅立疏曰:`皇上两、三年几次奔逃,流离颠沛之极;并无一位两衙门官共尝辛苦!何今日?即次稍安,侍御济跄,议论纷纷若是。如以臣为无饷、无兵窃取勋爵,请即遣金堡为臣监军,观臣十年粮草、十万铁骑'!疏入,十一月朱天麟票拟,有``金堡从来,朕亦未悉''之句。时严起恒久欲挤天麟而无隙,即以此票拟密示吏科丁时魁;〔时魁〕忿怒,即夜约两衙门科道十六人,于十三日晨率入丹墀,共言`强臣箝结言官之口,将来唐末节度可虞'!因哄叫而出曰:`吾等不做官矣'!将公服袍带掷弃庭中,小帽叉手、白衣冠联袂去;所恃者,李元胤也。元胤承彭年橐籥,权通大内、势逼至尊。时永历坐穿堂,召太仆马光追叙五年前永州破难、逃入全州前后手书谢马光有`先生衣我、食我,后日岁月皆先生生我、成我'句等事;忽闻外变,两手振索,茶欲倾衣。十四日,特降谕旨,敕李元胤各到十六位科道门,谕令仍入本衙门办事;前本另发票拟,阁臣朱天麟即日放还田里。诸臣以为手裁矫矫,中兴朝政第一美举。

何腾蛟死难

己丑正月,大清兵破湘潭,何腾蛟被执。初,腾蛟檄各处兵马齐集湘潭,而命马进忠等由益阳出长沙下,邀截上下舟船,焚掠湘乡,断绝水道。一只虎率大队复至长沙,络绎攻围。值大清援兵合至,战胜于湘潭;腾蛟被擒,斩之(``遗闻'')。

湘潭、湘乡、益阳三县,俱长沙府。

``何堵事略''云:`癸未冬,何公云从来抚楚。是时左帅三十六营散处江、汉间,凶狞之状不可向迩。群帅故多降贼,桀骜难制;公推毂接待,控御有体。又时以忠义激劝左大帅,以故兵犹戢。尝对人诵``鞠躬尽瘁,死而后已''之句二语;公平生大节,具见之矣。轸残黎、收遗胔,皇皇无虚日。尝出行城市,竟日未得食,属门者购饼饵从舆上啖之;戒勿令有司知,恐为具馔。夜则寝穿堂一门◆上,枕以木石。其自为刻励如此。乙酉三月,左焚劫武汉趋江南,以讨君侧为名,逼何共事;何知左反侧,誓死不从。强舆至舟,即投江;众兵掖之起,委于岸。左舟发,残民万余闻何在,簇拥之;舆至通山,取道河南而去。四月,抵长沙;与堵胤锡合,收集余众,行大募兵,滇兵主将黄朝宣隶之。丙戌、丁亥间,南北尝小小交锋,南取胜;燕子矶一战,黄老将武功为赫。戊子,自洞庭口转至衡阳,胜负相当;章旷当其冲,与三王兵鏖战三昼夜,大将刘承胤卖降,章愤激死,势岌岌且殆。何、堵闻变,亟援之;大清兵退。己丑,堵驻长沙;何麾下诸□喜长沙货物辐辏,夺其居。是时一只虎已署为忠贞营,在辰、常界;堵引兵就之。十月,大清盛兵上湖南,沿湖杀哨拨兵,抵长沙;长沙城下人无知者。何仓皇命出师,众散走;有卒数十人拥何出城,何怒叱之曰:``若属不用命,去将安之?吾今而得死所矣''!以旧时衣冠衣之,独危坐沙屿上。大兵至,自言``何督师'';执送古庵中,不食数日死。公煦煦爱人,爵赏少滥;庖丁厮养,多列旌旄。以故爵不足劝,诸营将渐跳梁不奉检束;务含贷,不遽绳以法。然而忠义慷慨,固其天性;语及疆场事,辄流涕呜咽不胜。盖诚有余而才不足,拊循有余而军旅非其所长也。向之所谓``鞠躬尽瘁,死而后已'';其言不虚矣'。

何腾蛟,字云从;黎平人。追封中湘王,谥``文忠'';庙祀灵州县。章旷,字于野;松江华亭人;崇祯丙子解元、丁巳进士。

李成栋信丰再败

己丑二月二十六日(乙卯),李成栋南下。四更时,先发火器手三百人,责咐曰:`如前遇兵,尽发火炮,我为后应;不尔竟前走'。至黎明,杳无炮声。众皆曰:`火器军往矣,吾当拔营,披甲上马'。言未毕,清兵杀入,满营溃乱。盖先发铳手前遇大兵,适欲举火,忽暴雨突至,炮声不发;三百人杀无遗,故尔寂然。成栋营后即系急流山涧,有见成栋被甲未完,乘一跛马渡涧者;及过后,遍觅无有也。三月初七日(丙寅),成栋与腾蛟凶问同时报至肇庆,君臣大惧;大雨中,昼夜逃徙。门户水火,悉皆冰释,寂静者一月。至九月、十月,先后赠腾蛟中湘王、成栋宁夏王。永历设坛挂帛,皆抆泪亲祭;纸锞与千钧舟并大,以示酬恩(``粤事记'')。

``遗闻''云:`二月,大兵破信丰,李成栋殁于阵'。

金声桓赴水

``遗闻''云:`金声桓踞南昌,大清兵昼夜攻击,破之;王得仁自杀、声桓赴水死'。

赣州信丰县贡生曹兑光,多才智。先是,起义赣州;及赣城破,金声桓擒兑光降,遂居声桓营中。迄戊子年,声桓诣关庙行香,兑光指关神说之曰:`此是何人'?声桓曰:`汉朝忠义人也'。兑光曰:`彼丈夫也,我丈夫也;若能举大事,将军亦与关神同矣'!声桓心然之。兑光知其意,遂移书宁都所善贡士卢南金、廪生赵日觐、庠士曾传灯等八人。南金次子将书示外,知县田某闻之,申文上司,报八人解省中;声桓勘问,尽释还。声桓本约八月合南京诸处起义,以南金等故,恐迟久事泄,遂于四月同副将王得仁邀诸将领既毕,不欲令出;黎明,命左右取优孟衣冠,即于座上服戴,以袍冠递送诸将,俱从之。遂反踞南昌,命得仁提兵上赣州围之。时守赣城者,大清巡抚刘武元、巡道张凤、总兵胡某也;率兵坚守,不出战。围三月,城中粮米五十两一石、盐十六两一斛、糕四两一盘;有宁都人李去白在围城中,将家乡圩田三百亩仅易杨某米二石。百姓止存三百余人,俱挂一腰牌,不许私语;语则拿去。缴上腰牌数十,俱饿死者。围至八月,势益不支。武元欲降,凤不从;曰:`再俟三日,无援兵则降耳'。及三日,而谭固山兵至南昌矣。得仁解围去,赣城得全(赣州人口述)。

日觐,赵某字。曾传灯,号廷问,改名畹中。

姜曰广赋诗殉节

姜公讳曰广,字居之,号燕及;江西南昌新建人。万历己未进士,改庶吉士。邹忠姜公讳曰广,字居之,号燕及;江西南昌新建人。万历己未进士,改庶吉士。邹忠介公荐李三才,为廷论所指;公出揭直之。天启甲子,授翰林院编修。奉使朝鲜,不携中国一物往、不取朝鲜一钱归;奉旨阅视岛帅毛文龙还。乙丑,分考礼闱。权奄用事,令其甥傅应皇纳交于公,拒之;复令其孙魏抚民谒公,不见。坐门户,落职为民。丁卯冬,起原官。崇祯己巳,大清兵深入,上特简马世龙为经略;世龙拥兵不战,公力言于朝,罢之。庚午,补讲官;主应天乡试。壬午,升詹事,掌南京翰林院印。先是,公在讲筵见时事日非,进谏上勿任性、勿用左右小人;上顾谓阁臣:`曰广言词激切,大见不平。朕知其人每优容之'。及甲申年三月,先帝升遐;公与南兵部尚书史可法议立君未定,诸帅受太监卢九德指,奉福藩至江上。于是南京文武大臣并集内官宅,韩赞周出簿令各署名;公言`不可如此草草,贻羞史册。须来日为文祭告奉先殿,乃举行'。迨明日,乃与众同至奉先殿,议监国事;诸勋臣语侵史公,公厉声呵之。于是,内外侧目公。宏光立,以公为礼部尚书兼东阁大学士;公辞,改礼部左侍郎,入直。刘孔昭廷讦吏部尚书张慎言;公因上疏求斥罢,不许。马士英荐阮大铖,得召见;公争之不得,再求罢,不许。公上疏言事,而四镇合疏诋公。会有建安王府镇国中尉朱统\{金类\}侯考吏部,因奏公定策时有异心;公求去益力。以皇太后至京,加公太子太保;寻致仕。明年南京陷,公潜里中。后金声桓归明,迎曰广至南昌;奉为盟主,以资号召。迨己丑正月十九日(戊寅)城溃,声桓自杀;曰广乃作绝命歌,投偰家池死,一家从死者三十余人。

其诗曰:`有君美好且宣通,志轶唐、虞争比隆。智辨惊臣谢莫及,宵旰□□急治功。逢天瘁怒日多故,奸相踵继荧圣聪。因循养乱难救药,贼气直逼大明宫。臣甘婢媵死贼手,君死社稷独正终。慷慨乘龙归帝所,亘天长气化为虹。龙髯难攀弓剑□,楸松万树泣忠风。一盂麦饭无人荐,孤臣永念泣无穷'!其二曰:`哀哀吾父性方格,道遵先民事垂则;严性正气泱其仪,寒冬孤岭松百尺。施济恳恳同吉凶,磨错粲粲傅清白。惨矜偶影惊鬼神,诚达体质贯金石。攻苦积学名不成,闇修备德福弗获。终身勤动日悯惶,遭遇坎坷含辛蘗。发愤于子望眼穿,大志竟齎入窀穸。窀穸之中恨不磨,贻禄不待亦何益!忆昔提携绕膝时,惓惓忠孝是鞭策;国破踌躇且苟延,永念教言当促刺。至今百尔计无之,惟有一死少逭责'!其三曰:`哀哀吾母称至善,淑慎其身如战战;通晓大义本性生,发言闇自合经传。初能孝养被□□,病亦女红至瞑眩;代父教子备苦辛,苦辛伤心强自咽。恩斯勤斯育子劳,怠则谯诃蹇则唁。恨昔因人凭远游,南陔养缺情恋。晚达徒含风未悲,朝朝泣对大官膳。我今一笑入黄泉,喜得慈颜永相见'。其四曰:`哀哀四兄圣贤徒,心行直与先君符。端凝皓皓自洗濯,俯视流俗如负涂。忆昔提携绕膝时,篝灯子夜共咿唔;明发常怀过庭训,日征月迈争步趋。兄德则丰命则塞,拂乱烦冤志不输;精心戮力无已时,□□□□□□□。无先无后俱已矣,天乎与养何弗渝!人谁不死弟亦死,弟有余憾兄则无'。其五曰:`有子、有子方襁褓,见我争向泣呜呜。我年十岁便无父,汝今失怙早过吾。穷民遍产德门里,茹荼未了又茹荼。夜来抱汝看汝笑,我心隐痛欲欷歔!母言尔幼全赖我,国破君亡我更孤。夜夜钟声疑禁里,朝朝泪眼看飞乌。匪我忍情恩不甚,名义千秋自凛如!我念作人全末路,汝思生子在厥初。良田有收无晚岁,过河枉泣是枯鱼!无父之子人易议,勖哉早克读父书'!其六曰:`有孙、有孙在童蒙;读罢依依就阿翁。汝方一岁便无父,小叔零丁与汝同。汝今谨识趋庭训,异日记取共磨砻!崇崖其阿多坠石,茂枝之下鲜芳丛。炎炎者灭隆隆绝,宦裔谁家克有终!祖泽一经传九世,罔因骤发坠素风!伤心阿翁教汝父,呕肝敝舌苦归空!汝父临危终爱汝,眩目顾汝血泪红。析薪是荷能克念,祖父瞑目幽冥中'。又绝句二首:`自古谁人不死亡,要知遗臭与流芳;读书九世才今日,莫谓偷生是吉昌'!`要知喜死原非死,况复衰年岂记年;杯酒从容微笑去,此心朗畅亦何言'!

此得自围城中,传寄于文登嵓处。公孙姜□振志。\\
假山图、五虎号

两衙门谓李元胤不足恃,思抑袁彭年;遂同具``重朝廷以力抑奔竞''一疏,申请以事权归永历,隐弹彭年把持。彭年奋怒,答疏中有`倘臣旧年以三千铁骑鼓励西行,今日君臣安在'等语;永历持其章疏,泣诉臣下,举朝失色。行在因有假山图、五虎号。假山图者,贾也;绘假山一座,下绘朝官数百。有以首□之者、有以肩负之者、有以手托之者,有仰望者、有远听者、有指点话言者,有惊恐退避、两手掩耳而走者。山岭黑气一道,直上冲天。此皆郊市童叟胸中不平,为此图以诙谐之也。五虎号者,吏都丁时魁、户都蒙正发,□彭年同乡,楚产也;一为虎尾、一为虎脚。兵都金堡,浙人也;昔为福建延平知府,疏激隆武赏罚不明,连杀同乡吴文炜、施◆二人,人畏之,为虎牙。副宪刘湘客,关中一布衣;来自留守阁臣瞿式耜,又为成栋同乡故,为虎皮。虎头,则袁彭年也;日将``祖制''二字说迂谈讲空话,因之获厚利。言非虎党不发,事非虎党不成;星岩道上,遂成虎市(``粤事记'')。

吴文炜,浙江湖州人;进士,邵武知府。施◆,字火然,号嘉峪,湖州人;崇祯癸未进士,建阳知县。

贺全业出狱

贺全业,镇江丹阳诸生。崇祯末,随父懋让之任上林。永历在藩邸,受教讲读。登极日,除主客司主事;后相失二载。戊子五月,于南宁府考改授;以囊涩不得入清华,量移精膳司郎中。七月十五日,邕江解维,前追帝驾。忽有试御史傅宏烈修旧隙,舟为夺去;声言觅全业,理前不了事。丕业罄身,手持两诰轴而已;借友人华姓银二十两,亟驰去。八月初□日,抵肇庆;拜贺后,即具□诉冤。□温纶复召对,有`先生与他人不同,后将大用'。因加太仆寺少卿衔,暂为受职。本司无事,见在内袁彭年等议论乖方、在外陈邦傅等跋扈无上,慨陈``四维三纲、人心朝廷''一疏,言词激切,传诵当时,共推中兴第一疏。彭年等恨之,暗指李元胤于朝班驾退后,欲殴杀之;友人急告,潜入高明、四会等处。十一月初四日临晚,道路喧传外县解来假官,肘缚下狱;全业极口呼冤,莫为之理。次晨,友人入告司礼夏国祥;国祥亦念系永历旧师,告狱主留活全业。至今年己丑三月初四日,部覆为彭年、元胤所持,奉``永不叙用''旨,保出狱。友人再赠资斧,令上桂林谒瞿式耜。三月十九日,寄一函于蜀僧,三达梧州,寓水井寺。

后竟不知所之,或曰卒于桂矣(``粤事记'')。

贺懋让,举人;广西田州府上林县知县。傅宏烈,上林人;布衣。御史夏国祥,池州人。

瞿式耜兼督各省

自何、李败后,永历专命瞿式耜留守督师兼〔督〕江、楚各省兵马;式耜疏请兵科给事中吴其雷监各营军。永历驻肇庆,疏奏谆谆,以`岁月稍暇、财赋优裕,用心尽力修内治以自固、严外备以自强;且一材一艺之士,靡不收罗。幕府每慨人才易尽,凡趼足而至者,非怀忠抱义之人,亦乱世取功名之士。人之岁月、精神不用之于正,则用之于邪;安可驱为他人用哉'!人咸以桂林为稷下。

瞿公爱惜人才,真将相之器;宜保危疆数载也。使镇抚诸臣尽如公意,则一线可延。

\hypertarget{header-n82}{%
\subsection{卷十四·粤纪(续)}\label{header-n82}}

堵胤锡始末

堵公讳胤锡,字锡君,更字仲缄,号牧游;宜兴人。万历二十九年(辛丑)十二月初八日酉时,母陶氏诞生公于武进之夹山村。父维尝,号冲宇;邑庠生。公六岁丧母,十一岁丧父;十二岁,依岳丈陈娱济家。二十岁,七月游于无锡,依四兄知白读书;因通籍于无锡。知白讳大建,抚公如子;后公抚知白之子正明亦如子,而正明后随公殉节于楚焉。天启元年,受业于马文忠。三月,婚于妇家。壬戌,二十二岁,补郡诸生。十二月,徙居中桥庄。甲子,复徙归于妇氏。己丑,岁试五等。丙寅,科试一等。癸酉,三十三岁,举乡试十六名。丁丑,三十七岁,登进士一百零八名。九月,庐墓虞山(山在宜兴岂山之南)。戊寅六月,辞墓。己卯四月,授南京户部主事。庚辰二月,莅北新钞关分司。辛巳,解任;三月,归里。六月,升湖广长沙知府。壬午冬入觐,离任。癸未四月至京,举廉卓,赐宴礼部。五月,辞朝复任。九月,加升二级。十月,行至兰溪,闻寇陷长沙而还。十二月,复任,路阻;岁除,守墓。甲申正月,复至长沙。二月,升巡道。九月,授楚督学副使。

乙酉四月,督师何腾蛟抵长沙相见,抱头大哭;徐挥泪进策曰:`楚囚泣,无为也!不措饷,能辑兵乎;不招降,能张楚乎'?何公然之。自是,公措置转运,资何军不绝。

五月,南都失守报至;檄告天下勤王,设三科以募士。六月二十七日,寄子书云:`世界至此,国破家亡,吾再有何言!鞠躬尽瘁,危疆待命;急则身受于刲,暇则梁悬一索。苟无愧为祖宗、父母之身,斯已矣。迟速吉凶,吾已置之度外,吾儿可勿虑也。为尔虑者,只有``逃''与``匿''两字;逃要相机、匿要深晦。念三百年宗族邱墓一旦沦亡,尚何忍言!吾儿若一蓑卧月、终身农竖,春秋不乏祀火,传之子孙,永不出仕二姓,便吾不失忠、尔不失孝矣!言至此,泪如雨下;尔母劝令勿戚;四十仳离,差不恶也。一切家务,吾悉置不言;只尔自强为善,守身弗辱,便瞑吾目。后音难继,书此绝词,儿可牢识'!

公自起兵,即谢学政,日夜为守御计。隆武中,以何腾蛟荐,七月升公左佥都御史,巡抚湖广。驻节辰溪,悬异格以罗奇才;诸生赵振芳上谒,公曰:`国家多难,主辱臣死;本院倡率天下义士,即日东下勤王。以何道而必济,君试言之'!振芳曰:`使相东下勤王,未必不济。岂不闻川、陕为天下肩背,荆、襄为天下要枢?控制上游,实为恢复根本。若舍此东下,使敌骑长驱,荆、襄以南不复为国家有。是公以现在之封疆委之于人,虽赴死金陵,亦何益'!公为首肯者久之。于是留致幕下,决意经理荆、楚以窥中原。寻题振芳荆州府推官。振芳别字胥山,浙江上虞人;在松江从师十年学``易'',刻``易讲''甚多。后大清兵至,降;为福建延平府同知。

八月,时李自成败,而遗孽李锦(号一只虎)性凶暴,与母舅高一功尚合十营约三十万众。自川、陕由当阳转战而来,剽掠荆南间;声言约我军会猎湖南。公为兵粮百不能支,乃集麾下谓曰:`覆亡无日,吾愿赤身往,为国家抚集其众。事成,则宗社之灵;否则,某授命之日也'。先遣监纪陈某、副将某将命曰:`他人恐不达命,今本院即日亲临,约以某日相见'。遂以匹马率帐下执事数人,由武陵、澧水以达草坪;积骨纵横,闾舍荡然。贼徒初见公驰驱,且沮且诘;未至三百里,拒不达前,止空城中。逾刻,望见戈矛蔽天、烟尘塞野,从者皆泣;公顾谓曰:`若等当生还无恙,某死,乞收吾骸以返'!顷之,贼徒猝至,拥入营;以大义谕李锦等曰:`将军辈有大用材,而陷于不义,亦当事者之过。但念国家三百年来若祖宗世食其德,卒以乌合之众覆灭宗社,止博贼名,为此何利?今若能悔祸改行,协力同心以建立功业,某当与将军等共之!昔五代马殷抚据湖南,延祚四十余年。宋之南渡,杨公作乱,其党王佐、杨钦等亦于此地为岳武穆所招,表授官爵;后随武穆协力中原,遂为中兴名将。湖南一片地,正英雄出世展略扬声之藉也。且赤眉当年思``为贼''二字名号不顺,其尊刘盆子为王。今福京新建,主圣臣贤;以此号召天下,何难比美南阳,光复旧宇!以天道、人事卜之,中兴无疑。将军千古得失之机,正视举足间尔;安可执迷自误哉'!锦等见公开诚布示,意欲定盟;忽李锦母高氏屏后出呼锦跪堂下,诫之曰:`使尔辈赎前罪、洗贼名者,堵公之赐也。尔其始终勉之,勿生二心'!锦素敬惮其母,即叩首曰:`愿遵慈命,誓当死报'!因与公酾酒誓,复留宴奏伎;所用女乐,犹是帝宫掠来者。公方素服腰绖,乃却坐不观;痛哭而言曰:`两京未复,万姓倒悬;某求死无所,尚何心听乐耶'!锦即撤去,并令散西秦、燕、晋、豫、楚等处百姓不下数万。公返之日,民皆抱携,呼拜道虏遍数百里。于是李锦、高一功等三十余万皆就抚,听公节制,楚中号``十家兵''云。公自收高、李后,兵强且众,得专力守御,屡有战功。公乃为李锦等上疏,请封伯爵;遣武弁举人傅作霖齎赴行在。乙酉九月,备车战火攻。先是,公遣人四出募兵。及是,所集二万人。特题蜀人杨国栋为大帅,令统之为亲兵,使与各营相表里;以车制骑,以整暇制野战,以火器制弓矢之长,以更番制坚久之战。国栋后封武陵伯。十一月,请封疏达行在。内阁蒋德璟、路振飞、林增志持议,谓`李贼破北京,罪在不赦;其党安得封拜'?御史钱邦芑具疏,言`方今国家新造,兵势单弱;高、李诸贼拥三十万众于楚中,若不以高爵招之,彼必不肯为我用,全楚非我有也。今出空爵于朝廷之上,一月而得三十万之兵,免全楚生灵之涂炭;孰得孰失?即昔汉高王韩信于齐,岂得已哉!今当权宜,假以封号'。诏从之。乃赐高、李诸营名``忠贞营'',改李锦名``赤心''、高一功名``必正'',诸将封侯伯有差;特命兵科给事中龚善选往楚封之,而别降敕奖公忠勤,授傅作霖兵部主事。

丙戌正月下旬,公渡澧水,督忠贞营往恢荆州以上,荆南副使燕如、荆州推官赵振芳监其军。措粮储、运火器,自二月朔渡江攻城;凡六昼夜,大小神器如轰雷不绝,远近闻数十里,云车炮石百道齐攻。大清帅郑四维监守,城崩数十处,皆随方拒战。有献策者谓:`荆城夹蜀、汉二江之门,水高于城者数丈;昔人筑堤为长围,使水入江,安流赴海。若决二堤,则两江之水建瓴而下,荆、襄一带望风归附,恢复之机在此一举矣'。公曰:`我为朝廷复疆土,首以民人为本;若此,则生民胥溺,我得空城何益'?忠贞营诸将闻,亦遣人会商;一面下令营兵各备锹锄以待。往返商榷,稽迟二月;而省兵救至,与战失利,诸将遂溃而还。公坠马,伤臂几死;过新化驿,题诗云:`不眠灯火暗孤村,风雨萧萧杂夜魂!鬼定有知号汉阙,家于何处吊荒原?三更鸟化千年血,万里人悲一豆恩!南望诸陵迷野渡,钟山肠尽可怜猿'!又云:`短策回瞻城曲阴,剑声犹吼不平心。孤军雨里鸟声碎,乱水渡边马影深。南北试看谁世界?死生此刻辨人禽。倒番``廿一''听前史,``正气''千秋歌到今'。三月,公归常、武,勉励文武;于是公安、松滋、枝江沿江一带各设兵将为守御计。

四月,擒沅州妖僧付有司,请旨诛之。有僧自称宏光帝者,自思州历平溪来,据沅道署为行宫,张官设卫。辰沅道副使徐伟驰文报公,公闻即曰:`此必假托者。奈何令诪张如是'!乃命监纪官某往辨之;谕以果伪,即擒付有司。监纪至沅见兵道徐伟及李若星,皆以为无可拟议。时若星监军郑逢元统兵二十余,又为之抚卫,俨然皇帝制。适有米寿图以勤王道出沅州,故侍御旧臣也;监纪遂约逢元叩之,因相与谋诘朝见。预戒甲士环集门外,止携亲随数人以进马为名,裹甲藏刃而入。其僧朱衣幅巾,仅出半面;寿图拜阶下毕,即启曰:`侍卫退,臣有密语,请得上殿面奏'。左右卫侍环呼`无旨,不得上殿'!寿图等疾趋直上;出不意,手揭幅巾,大呼`非是'!亲随即掣刀上殿传呼门外甲士,各露刃弯弓,夺刀争前;捽妖僧衣领而下。缚送辰州司李戴□刑鞫,具吐妖状:即沅州人,姓名查显仁。先是,饷部喻思恂等误以为真,飞章奏闻,廷臣莫决其真伪,议差官探问;未及行。至是,公乃疏陈本末,请斩之;并言诸文武官为所惑者,宜置不问。六月中,疏达行在;从之。

六月,督师何腾蛟约期会议,公因往长沙。七月,驻湘潭。九月,永明王立粤中;丁亥四月,升公兵部左侍郎,总督恢剿军务。车驾自桂林幸武冈,刘承胤掌兵专政;而忠贞营十家兵颇强,声言入湘南就粮。承胤惧为所并,以为非公莫制;于是加公衔,敕督兵出江、楚。七月,上疏请诛承胤。公在湖北,藉督师何腾蛟为表里。刘承胤者,故何公部将;及荐为总兵,遂称门生,稍倨肆。后联为姻亲,不复受约束。迨驾幸武冈,承胤以定蛮伯进武冈侯,辄挟兵权擅作威福。群臣畏其刚暴,争谄之以自固,交疏颂功德;竟进封兴国公、上柱国。承胤止有二子,五月间以冒功荫锦衣指挥者也。何公闻而欲制之。而承胤骄日甚,思欲与何公抗礼;忌其权出己上,请改为户部尚书,专理粮饷。不听。何公辞朝归镇长沙,廷臣奉敕郊饯。何公托病驻城外荒寺,不言行期,盖防承胤也;果伏千骑于道中,欲暗害何公。时赵印选领滇兵五百随何公,皆悍卒;中道伏起,印选率部下尽谶之。承胤讳之,人莫敢言。公因上疏劾承胤专恣不法,截杀督师重臣,谋危社稷;请正典刑。承胤见公疏,始知惶惧。

八月,常德陷;公乃率马进忠、王进才、牛万才、张光萃等驻札永顺、保靖二土司界上。九月,大清兵逼武冈,车驾将幸柳州;方出城,承胤即遣人迎降,兵部侍郎傅作霖死之。报至,公与诸将谋请荣王监国,冀禀号令,以镇抚人心;乃于舟次启王,王固让不肯。既而知车驾无恙,遂中止;惟传令各营协力防守。十二月,率忠武营兵复常德,进复辰州。自退保土司以来,采薇茹蕨,淹及半载;公愤不能恢复,乃刺血书二祖、列宗之牌位,恸哭欲自杀。诸将感动,三军皆哭,哭声震山岳。马进忠、王进才等辄夺兵而出,公亲甲胄督阵,率侄正明血战三日,遂复辰、常,俘获甚多。事闻,诸将各升叙有差。于是乘胜图下江、汉,军声复振。戊子正月,``过天门山''诗云:`终朝马背随风雨,尽日刀尖度死生。全副骨峰贫已赤,一双眼角老难青。才淹骚、赋非伤主,学窃``春秋''未解兵。四十八年心事左,只因多难独精神'。三月,檄忠武营诸将与忠贞诸将合营,同驻常德。公虑马进〔忠、王进〕才等孤军难支,更调高、李等诸部为犄角,为乘胜东下计;诸将皆赴命。有奸人郑古爱者,东西唆构;诸将惑之,始各怀疑忌矣。上``绝口勿谈款和''疏。大清帅驻楚,会使公卿以书招公,公峻拒之。时有倡和议者,公上疏痛陈其不可,时论壮之。

七月,度彝陵至夔州,诘责楚宗朱容藩不当僭监国之号,遂散其党。

十月,还至湘南,督忠贞营诸将复湘潭。公还,马进忠与李赤心不合;恐其相图,遂掠常德,移营湖南矣。公乃调护诸将,鼓励士卒,躬率高、李兵出征;先复湘潭县,次复衡州、郴州。既而进兵江右,所过郡邑多下。师次吉安,得故将归明之报;乃回茶陵。金声桓、王得仁皆左良玉部下旧将,先降大清;忽归明,江右响应。公闻报,即回茶陵;欲与何督帅定谋,合兵江、汉图中原。十一月,升公兵部尚书,赐尚方剑,便宜行事;专督诸营恢剿。十二月,督兵援江西,至袁州。先是,何腾蛟自粤西还楚,因听细言,致书于公曰:`曩附八行奉侯台端,不卜得达记室否?腾蛟与大清战于严关日月桥,三王却走;进围零陵,指日可下,各郡邑尽入掌中。闻忠贞诸营驻节中湘,分取衡阳;则功又有所属矣。近王、马诸勋举动,甚是乖张;腾蛟已有檄谕之矣,谅此辈必不负腾蛟也'。公得书,语枢贰毛寿登曰:`我等封疆之臣,罪且难赎;何公尚欲言功耶'!至是,金、王诸将为大清兵所困,何公调忠贞营往援,公即率诸将赴之。己丑正月,湘潭复失,大学士何腾蛟死之。二月,公闻变,师还至衡州。时忠贞营与何标下不协,远驻辰、常界上。公乃率滇将胡一清等扎营衡阳,悉力拒守。二月,大清兵破南昌,金、王诸将俱殁。公``过安仁道次''诗云:`乱里看花试一临,廿年零落又春深!柳桃尽入兵戈眼,溪涧争鸣风雨心。野鸟向时三月丽,峡猿枯绝暮山阴。天涯即事浑伤旧,马背须眉自感吟'。公聚军中所赋诗,名曰``马革集'';今逸。公在军中五年,着``春秋说义''五卷,凡万五千余言。

四月朔,大战于草桥,败绩;退札耒阳。阵于衡之草桥,自辰至酉,斩伐相当;大清兵以轻师截出阵后,兵遂败。公乃弃衡州,退札耒阳河上;而永营驻于永兴,相去百五十里。初五日,永兴陷,从子正明死之,诸眷属亦皆遇害。公自耒阳以数千骑退入龙虎关,暂依保昌侯曹志建营。志建素骄横,纵兵掠永、郴界上,又坐视不救援,屡被公诃责,方惭不自安。至是,见公兵败,遂欲乘机害公。夜坑杀公从兵千余人;及旦而公觉,乃入猺峒之何家寨。志建追至,何生等率众力拒之;志建怒,悉屠其寨。公得走粤之贺县,沿途招集散亡,从兵甚众。

六月,时上在肇庆;十五日,公至肇庆。十六日,朝于行在,奉命入阁办事。朝廷先遣官迎公富川之野,比十五晚至肇庆。晤阁臣严起恒,叙故旧谊;明旦,相引陛见。朝廷亲劳之,面命入阁;赐宴,礼意有加。然旁观已有侧目者矣。十九日,奉命安插诸营。时高、李十营兵尚十余万,分道从柳、郴入梧州;既而移屯德庆州。客兵猝至,粤西震恐;又有言其将统兵入卫、清君侧者,众益危之。朝廷乃命公度地方安插之,人心始安。加升总督直省军务兼理粮饷,特赐龙旗、尚方剑便宜行事、少傅兼太子太师、文渊阁大学士、吏部尚书兼兵部尚书,敕忠贞、忠武、忠开诸营悉听节制;忠贞即高、李十家,忠武即马进忠、王进才、张光萃、牛万才等,忠开为于大海、李占春、袁韬、武大定、王光兴、王友进、王昌、王祥等。上``急措兵饷以求招集实功''疏,凡五上;议于学道李锜衙门支拨事例三千两以给之。已舁至寓,忽为李成栋养子元胤攫去,仅领布绘龙旆二面以壮军容而已。二十四日陛辞,奉敕出师至江、楚。先是,颁敕书、旗牌、关防;是日,公含泪辞朝,遂同新设湖南抚臣马光整旅启行。七月初三日,师次梧州。朝廷念公勤劳,因降敕封为光化伯,给诰券;公以廷臣立门户,师旅齮龁无成功,惟当任罪,何敢冒功;上疏力辞。遂赐公四代诰命以奖之。

疏论孙可望封爵。时可望求实封,朝议难之;会朝臣遣使劳军以问公,公上疏曰:`臣窃谓孙可望父子久已割据西川,今滇、黔尽为所有,固能自立,曷能禁其不自王;今可望尚知请命,其意犹可取。我不能禁其不王而欲制之,势将偾决;当即降敕封之,使恩出朝廷,乃可得其用。令彼缚胡执恭归朝,正法诛之;则是赏罚之权,庶不倒置。不然,是驱之为变也'。首辅严起恒、户部尚书吴贞毓、兵部侍郎杨鼎和、给事刘尧珍、吴霖、张载述等坚持不可;公又密疏曰:`廷臣谓异姓封王非祖制,不当自可望变乱始。持论良正,然不为今日言。可望固逆献养子,凡逆献滔天之恶,与有力焉。今姑取其归正一念,冀收其将来之用;安可泥颁爵之常法哉!且可望已自称平东王,一旦封以公爵,彼必不乐受。因而为逆,谓天下威灵何、谓天下事势何!若欲收其用而反损国体,非良策也。臣窃有一说于此:臣谨按开国功臣徐达、常遇春等侑食太庙称六王,皆进封也。伏乞皇上干断,量封可望为二字王;即于敕书中详载旧制,明示破格沛恩,而勉之以中山、开平之功。如此可望必能感激用命,揆之祖制亦不为背谬。国家今日于可望善收之,则复有滇、黔;不善收之,则增一敌国。利害无两立、得失不再图,不可不熟虑也'!制曰``可'';命铸印封可望平辽王,差赵昱齎往。

十一月,师次浔州,公有疾。时李元胤用事,每有奏请,辄为掣肘;遂发愤成疾,乃驻兵长生寺。刘湘客五人附元胤为丑虎,其余依附者甚众,总谓之东人;公甚疾之。二十五日拜遗疏,二十六日五时公卒。疏略曰:`臣受命以来,罪大孽重;不复自谅,拟再合余烬,少收桑榆。不料请兵则一营不发,若曰``堵阁臣而有兵则丰其羽翼也'';索饷则一毫不与,若曰``堵阁臣而有饷则资其号召也''。致臣如穷山独夫,坐视疆场孔亟。昨西上横邑,感疠大重;一病不起,遂快群腹。臣但恨以万死不死之身,不能为皇上毕命疆场,而死于枕席:是为恨也。臣死之后,愿为厉鬼以杀贼。伏乞皇上简任老成,用图恢复;如国家大事有李元胤、刘湘客、袁彭年、金堡、丁时魁、蒙正发五人作皇上心腹股肱,成败可虞!祖宗有灵,实鉴临之!臣死矣,不胜余憾'云。拜疏讫,又南向拜父母曰:`儿死,不获更还邱陇矣'!复悬``在三图'',拜君、亲、师讫,遂自题十语云:`有明堵子,生而精敏。遭家不造,诚身事亲;遭时多艰,诚身事君。四十九年,孤儿、逋臣;而今而后,浩然苍旻'!遂卒。前一夕,亲吏欧阳和梦公骑牛升空去;次日语人,左右皆同。呜呼!公生以辛丑、捷以丁丑、卒以己丑;公之生卒,夫岂偶然哉!朝廷闻□,涕泣减膳,辍朝五日;赠上柱国、中极殿大学士、太傅兼太子太师、镇国公,谥``文襄'',荫一子锦衣卫指挥同知世袭,予祭九坛,遣礼部官致祭,赐茔浔州之西山。公所著,有``十四朝史纲''。

傅作霖,字润生;历官兵部左侍郎,管部事。死于武冈之变,谥曰``忠烈''。

``何、堵事略''云:`堵公以甲申九月受督学事,十一月,试汉阳。左营将自总戎下至守、把,有所请,辄报可;众议以为怯。月杪起行,往湖南。乙酉四月,何腾蛟抵长沙。时闯逆余众号一只虎者约二十万,屯聚常德之间,谋割地自王。何欲往招抚,曰:``若就抚,不惟得劲助,且除内蝥''。公毅然请行,齎牛酒、金币往。先遣员通意,一只虎大张兵卫,沿途迎候,将士夹道露刃立;公不慑,安行至中军。命设香案,各俯伏听宣旨毕,即出敕印以次给之,徐为譬晓忠义、陈说祸福,慷慨激烈,声泪俱下。三军之士,无不耸然,听行大阅;器仗精整、旌旗鲜明,各以艺试,终事无哗者。有一阵乱于次,讣七十人,俱命斩以殉;三军股栗。徐出金帛,厚犒赏;大喜过望。己丑冬,何腾蛟死;公伤左臂已断,郁郁成疾。越数日,亦死。公性喜奕,每临阵,奕不少休;哨者报敌且近,曰:``尚堪一局''!赌墅与东山同,不知处分何如耳'。

堵牧游与侄书

两接吾侄手札,恻然忠爱,溢于楮端。江左应有夷吾,屈指当以吾侄为一座。但时事至今,已全坏矣!江北四镇鹰视虎步、汉江一带拥兵踞流,秦庭无可泣之处也。愚叔妄拟川蜀全盛,且据形胜,西蹙秦腋、南压楚头,假一、二岁之饷便宜倡义,尚有可为;而今又为张、李所摧残矣。寇焰已炽,加以强藩;闻湘中复有蠢蠢思逞者。翘首九州,无地用武,宁有固志!天下事至此,有不忍言;当事君子尚燕怡不畏、蹈辙不顾,尚日式臧,抑又甚焉。嗟嗟!吾辈一、二血性男子,从何处跕脚?惟有俯首摽心,中夜陨号而已。老侄之身,尚是可进可退之身,且有母在;括囊善刀,养晦待用正今日事。量先入后,勿以愚叔之言为妄也!

若愚叔已身许君国,览镜峨然冠佩者,皆先帝之要领,而星沙脱弃之余也已矣。一腔血、七尺躯,时事朝来,大命夕逝而已。兹虽有学政之移,不与地方事;然一旦不谨,断断不作逋亡客。文庙哀魂,是愚叔一生归宿地;潸然不禁!他日老侄当不弃予一孤;事后之托,止此而已。至王雪老死事于粤,言念悼叹。彼之孤,愚之责。宦橐清凉,吾辈本色;安足复计!但恨家国祸深,无暇旁及儿女耳;如何!堂上二嫂安吉?井木侄与澍生侄近况何如?惫甚握管,不及作书;惟叱及之!薄俸聊以示念,溯流不尽。骥儿试事,幸教率之!寅叔在锡,宜训以义行。吾宗无多正人,故睠睠及之。
九月二十日,愚叔锡顿首。

``粤西实录''云:`公纳浙绍叶氏女,公卒,有三月遗孕;因嘱部将常,竟负托。及可望至粤迎驾,执而数之曰:``堵制台何人?佣奴敢为此态耶''!鞭之至百,而遗孕得不死。今闻尚在滇中云'。

康熙九年冬,往南门偕张子秋绍登一小楼,见堵氏祖祠有木像数寸侍立,即先生昔年亲制己像,以识不离左右之意;此世所未见者。瞻揖之下,仰其孝思。明年(辛亥)四月二十二日,复阅先生手札;用竹纸三幅,信笔草书。凡五百言,无非忧时殉国之志;真忠、孝两全,为吾邑奇男子也。

缝甲泣

甲申备兵黄州,媿臣面之犹生、痛国仇之未复。爰命匠氏,制我甲裳。衽起中夜,不能成寐;作缝甲泣。

臣官兵马监,枕戈不旦中夜天。臣逢四七期,二百八十年数齐。臣备古黄邱,磷火接地天风愁。四野蛇斗龙失窟,一旦君亡臣尚活;臣活何为肝脑裂。臣冠泣作囚,臣活长掩羞;掩羞本掩泣,恻恻衣衫血。噫嘻吁!泣血缝甲翦落声,着肉着甲先着心;好向原头裹处寻。
莫缝甲!缝甲贼识我,劲镞长矛不得躲。莫缝甲!身逐贼。生有骨,骨如铁;生耐金革尖头霜雪寒,不耐绮罗着身儿女热。泣复泣兮缝复缝,夜半长歌起北风。

孙可望请封王

己丑四月初六日,云南张献忠养子孙可望遣龚彝之弟龚鼎、杨可仕等六人诣肇庆,献南金二十两、琥珀四块、马四匹,移书求封秦王;书曰:`先秦王荡平中土,扫除贪官污吏;十年来,未尝忘忠君爱国心。不谓李自成犯顺,王步旋移。孤守滇南,恪遵先志,合移知照。王绳父爵、国继先秦,乞敕重臣会观诏土。谨书。己丑正月十五日,孙可望拜书'。以方幅黄纸书之;不奉朔,亦不建朔。一时群臣怂恿以秦王封者十之五;独兵部金堡固诤,以为祖制无有。李元胤、袁彭年因龚鼎、杨可仕等自陈邦傅来,亦执不可;阻挠者两月而未定。盖广南西宁府与云南广南府错趾,中止间一田州,两日可达;时邦傅驻南宁,因通可望,可望所遣之人邦傅引进。可望遣使行时,有`不允封号,即提兵杀出南宁'等语,邦傅恐先受兵,惧甚;知行在刻印、刓\textbackslash{}印,喙长计短,又为金堡所持,必不能得,乃先假敕封孙可望为秦王。可望肃然就臣礼,先五拜叩头,舞蹈称臣,受封秦王;后率义兄弟三人并三军士卒各呼万岁。后又升座,受义兄弟三人及三军士卒庆贺礼毕。正欲撰表奏覆,适龚鼎等齎金堡所议荆郡王敕至,可望毁裂弃地不问、亦不改前封,谢表亦遂止。时已十月初矣。

龚鼎,云南人,癸未进士;彝之胞弟。杨可仕,淮安人;举人,云南右布政使。

孙可望胁封谋禅本末

孙可望,陕西米脂人;一无赖子,流落为贼。张献忠有养子四人,长即可旺、次李定国、次艾能奇、次刘文秀。丙戌秋,大清兵入蜀,献忠箭死;可旺率众四万人冲散曾英营,由遵义渡乌江,屯贵州。丁亥二月,大清兵至遵义,可旺遂率众走滇,攻下曲靖、云南据之。始可旺等四人俱冒姓张,至是各复本姓。可旺自以名不雅,改名可望,称平东王;李定国称安西王,艾能奇称定北王,刘文秀称抚南王。四人同称王,议推可旺为主,凡事听其号令。

先是,云南土司沙定洲反,逐黔国公沐天波,据云南省城;天波走避永昌。及可望入滇,沙寇战不胜,逃回土司;可望遣定国往灭之。又命文秀往永昌擒沐天波并兵道杨畏知,天波畏之,俱降。云南十八府悉归可望,兵势颇盛。

丁亥秋,四川巡按钱邦芑率总兵王祥复遵义。至戊子春,金川俱复。总兵侯天锡见可望强,甚欲招之,乃商之王祥;祥曰:`可望乃献忠余孽,狼子野心,恐不为我用'!邦芑曰:`闻可望行兵有纪律,不轻杀人,似非献忠故态,不可逆料'。因修书草檄,差推官王显往招之。至滇,可望大喜过望,谓显曰:`从来朝廷文官与我辈为雠,绝不相通。今遣使通问,何敢自外。但我辈称王已久,求钱巡按具疏封我为王,我当举全滇归朝廷矣'。邦芑复之曰:`本朝祖制,无异姓封王者'。因具疏,称可望归顺,请封公爵。上敕部议。适庆国公陈邦傅驻札广西,兵势甚弱;日张边情,假要封赏,至厮役皆冒侯伯。而高以正、李来亨又率兵入粤,邦傅欲自固,闻可望归命、私求王爵,朝议未决,邦傅乃遣心腹人胡执恭私铸``秦王之宝''金印一颗重百两,伪造敕书封可望为秦王,以为外援;并封李定国、艾能奇、刘文秀为国公,俱伪造敕印。执恭,京师游棍,惯造私印、假札,屡犯大辟,逃入军中者;遂主其议。己丑秋七月,齎假敕宝入滇见可望,拜舞称臣,述皇上系眷之意;可望大悦,受封秦王。既而可望闻朝议未决,疑其伪;因私诘执恭,执恭语塞,因诳曰:`此敕印俱系太后与皇上在宫中密商私铸者,外廷诸臣实不知也'。可望虽探知``秦王''之封为伪,然亦但假其名以威众;定国与文秀卒不受,托言未与朝廷立功,不敢受爵。行在知执恭假封事,朝议哄然,知邦傅所为,交章参劾;邦傅只推不知。时执恭子钦华任宾州知州,因执赴行在,众请诛之;上曰:`其父作逆,其子何与'!赦之。是月,适督师堵胤锡入朝,奏上曰:`可望盘据滇中,若不封,恐生他变'。首辅严起恒力持不允;胤锡乃铸``平辽王''印,密奏上,差都察院右佥都御史赵昱齎往。昱入滇界,先遣报;可望谓已称``一字''王,今反降``二字'',欲拒昱使不入。定国等劝曰:`天使既来,何可绝之'!乃令入。昱知可望不悦,一见叩首称臣,私归诚于可望;可望予昱十金,其``平辽王''印受而藏之,仍称``秦王''。朝中知昱辱国,欲处昱;昱不敢回朝。滇中臣民皆知``秦王''之封为伪,多有窃议者。可望亦以为耻,因遣御史瞿鸣丰入朝,必欲实求``秦王''之封;请即用原宝,但求上加敕一道。而内阁严起恒、户部尚书吴贞毓、兵部侍郎杨鼎和、兵科给事中刘尧珍、吴霖、张载述持议更坚,可望遣私人杨惺先入朝通贿,诸公怒不受。可望愤甚,乃遣贺九义带兵五千至南宁,假称护驾,刺严起恒及吴霖、刘尧珍、张〔载〕述。时鼎和已加大学士,奉命督师川、黔;行至昆仑关,九义遂遣将追杀之。独贞毓以差出,得免。此辛卯二月事。是时,朝廷震动,失上下体。严起恒既被害,上特简吴贞毓入阁办事。时可望必欲得``秦王''实封,再遣龚彝、杨畏知入朝。畏知,陕西解元,为人抗直;既见上,密奏可望奸诡难测,宜预防之。上信之,拜畏知东阁大学士;贞毓等与订交,同心辅政。龚彝乃可望心腹,见畏知与朝臣深交,又得拜相,心私恨之;归谗于可望曰:`畏知之得拜相,盖卖国求荣也'。可望怒,乃杀畏知。贞毓议曰:`秦王即欲``一字''王,亦当另议国号;若封秦王,是陈邦傅、胡执恭为天子矣'!于是定可望为征王,差翰林院编修刘\{氵茞\}往黔册封。\{氵茞\}至黔,可望怒曰:`吾久为秦王,安得屡更'?可望礼部尚书任僎曰:`大丈夫当自王,何必朝廷乎'!可望是其言,遂竟称``秦王'',不奉朝命。

时上驾驻广西南宁府,大清兵破浔州,陈邦傅父子俱投顺,大清兵渐逼南宁,驾移濑湍。可望命提塘总兵曹延生、胡正国各带兵三百,紧随左右以备不虞。上与群臣议,欲入黔暂避;吴贞毓曰:`可望跋扈无礼,若一入黔,上下俱为所制,国事危矣'!时马吉翔已暗通款可望,请上急入黔;私与太监庞天寿曰:`今日天下大势已归秦王,吾辈须早与结纳,以为退步。今提塘曹延生、胡正国乃秦王心腹,托二人为我辈输诚,异日庶有照应'。天寿曰:`若此,则吾辈须与两人结为兄弟,乃可行事'。曹延生,大竹人;胡正国,淮安人:两人虽为可望用,其实乃心王室。吉翔、天寿不知两人心事,请结兄弟之盟;盟毕,吉翔曰:`秦王功德隆盛,天下钦仰,今日天命在秦。天之所命,人不能违;我辈意欲劝皇上禅位秦王,烦两公为我先达此意'!延生、正国愕然曰:`此事何可轻易!且吾辈一提塘耳,止可传报军情;国家大事,非我辈所敢与'!吉翔、天寿辞去,私具启以达之知可望。可望恐中外人心不服,未敢轻举,意欲迎驾入黔,挟天子以令诸侯,乃便行事;故姑止不行。而延生、正国素与吴贞毓善,暗以此意告之,请上暂止广西境上,系属人心、号召远近,以阻吉翔之谋。吉翔遂密告可望,谓事将成,为吴贞毓所阻;可望遂遣总兵高天贵、耿三品、黑邦俊带兵迎驾幸黔,改安隆所为安龙所,请上居之:时壬午二月也。六月,李定国复广西,擒陈邦傅并子曾鲁,解至黔;可望召执恭视之曰:`使汝与邦傅一处,久已投顺大清矣'!遂将邦傅父子剥皮支解,兼命执恭监视以儆之。执恭惊悸恍惚,因以成疾;数月而卒。

时上在安龙,夹于万山之中,群蛮杂处,荒陋鄙俗,百物俱无;茅茨土库,随扈者止五十人,仪制草率之甚。而可望自居贵州省城,大造宫殿,设立文武百官。凡四川、云南、贵州文武大臣数百余员,俱挟以威令,刻期朝见,授以伪衔;有不从者,即诛之。以吏部侍郎雷跃龙为宰相、贵州总督范矿为吏部尚书、御史任僎为礼部尚书、四川总督任源为兵部尚书、御史张重任为六科都给事、礼部主事方于宣为翰林院编修。又铸伪印为八叠文,尽换明朝旧印。方于宣极其谄谀,为可望拟``国史''。称张献忠为太祖,作``太祖本纪'';比献忠为汤、武,崇祯为桀、纣。进可望览之;可望曰:`亦不必如此之甚'!于宣曰:`古来史书皆如此;否则,无以纪开创之勋'。于宣又为制天子卤簿、九奏万舞之乐,作为诗歌,纪功颂德;与鸣胪寺薛宫商订朝仪,可望苦甚。癸巳秋,于宣屡上表劝进;可望曰:`我何难即登九五,但恐人心未附'!于宣曰:`朝内相左者,止吴贞毓、徐极等数人;川、黔两省,止钱邦芑、陈起相数人。除此数人,其余不足虑矣'。可望曰:`吴贞毓等易为处分;但邦芑在外,系川、黔人望所归,杀之恐士民解体'。乃发令旨与余庆知县邹秉浩,令催邦芑入朝,待以不次之位。时邦芑已退隐余庆之浦村,秉浩逼勒百端;邦芑恐不免,遂祝发为僧。其祝发偈云:`一杖横担日月行,山奔海立问前程;任他霹雳眉边过,谈笑依然不转睛'。可望闻邦芑为僧,外虽怒骂而中惭愤;命任僎等以书婉劝之。邦芑答以诗曰:`破衲蒲团伴此身,相逢谁不讯孤臣;也知官爵多荣显,只恐田横笑杀人'!方于宣录其诗呈可望;可望怒,命邹秉浩解执贵州。将杀之,适有安龙十八忠臣之变,人情汹汹,遂释邦芑不问而禅受之谋亦遂阻矣。

四川巡按钱邦芑招孙可望书

前差官至滇,所以不敢即致书奉候者,盖未知老先生尊意何如耳!昨差官回,备道老先生优礼之殷;兼述老先生雅意翊戴天王,至真至切,更无他念。虽一时同事诸公犹未深信,而芑所以独信之不疑者,盖观老先生之为人,乃当今之豪杰也。从来无欺人之豪杰、无负心之豪杰、无面是背非之豪杰、无朝三暮四之豪杰,芑是以不顾议论是非,敢为具疏,竟请封爵。然老先生便当从此改弦易辙,拜表称臣;奉正朔、归版册,文武之升降一禀于天子、征伐之行止必请乎朝命。如是,乃不愧祖宗、不负朝廷、不负芑之荐举,乃成千古真豪杰矣。

芑先始祖吴越王讳钱俶者,以江南之地归宋,而太宗赐铁券金书,子孙世世与国同休。芑恐皇上不允封爵,故疏中即引先始祖为例。然当日先始祖与老先生今日不同者有四,请为老先生言之。先始祖立国吴越,传三世四王,保有江南之地将近百年,与五代相终始;而宋始兴,与宋朝未尝有君臣之分也。而老先生,大明之旧臣也。其不同者一也。先始祖王爵传自先人,历梁、唐、晋、汉、周俱受册封。而老先生之王号,则自己之僭称也。其不同者二也。先始祖土地授自祖宗,始于唐末;并非取之宋朝。而老先生之云南,则天朝之封疆也。其不同者三也。先始祖保有江南,世世奉贡,未尝与中国有一矢之加。而老先生二十年来残破数省,屠戮朝廷之人民、糜费朝廷之金钱,何止数千百万;甚至杀亲王、辱大臣,于朝廷不得为无罪。其不同者四也。芑之所以引始祖为例,明知事势不合;不过委曲以成老先生之美。倘蒙明旨俞允,是圣天子破格之洪恩,芑不敢居功;即或朝议不从,另议封号,老先生亦当拜受,以俟再请加封。老先生如此谦让不遑,恪守臣节,则是功名之路正长、子孙之福无量;青史扬名、姓氏俱香,非芑所能测也。

芑生平心事光明磊落,不肯自欺欺人;一遇当行之事,即举世非之而不顾。即今日为老先生请封一事,其阻而且忌者正自不少;而芑反衷无愧,竟行不疑。即老先生异日相信、相负,总不问矣。然老先生身为男子,顶天立地;不乘此时立万世不朽之功名,而徒据一隅以自雄,非所称大丈夫也。且今日之劲敌,非直我明朝之患也;令先人曾被大难,是亦老先生不共之耻也。芑辈戮力于外,日夜图维;而老先生拥强兵安坐海内地,恐不免贻笑于海内英雄矣!齐襄公复九世之仇,``春秋''大之;老先生能无意乎?伫望之切,言出不伦,伏惟原亮!

逼袁彭年守制

己丑年,袁彭年生母死,自谓丁艰不守制,喧言于众曰:`吾家受国恩深重,奕世科名;更受天地之恩洪大,代产异才。吾今享年远过先人,天正不欲置我于无用地;何得苦守三年,虚度岁月'!同党以为国尔忘家,中兴可望;宜晋世爵。马太后甚恶之,宣查丁艰不守制,是何朝祖制?彭年腼颜月余,挟重赀而去;拥富寡为妾,寓于佛山。寡妇,生员李戌妻;拥产数万。彭年督学广州时,掖其嗣;戊子年反正后,招彭年主其家。

佛山,广州巨镇。

永历骑射

永历宫禁湫隘,供奉清简,不逾千金子家。侍女寥寥,俱幼蠢荆布。内侍夏国祥以六十金于广城觅一歌舞青娥,发方覆额;不一月,失所在。遍索内外,越三月于东池水面浮起红蒂,已殒命于中;想亦有所不得已也。盖府署与高要县学并峙,中隔一池。于是覆土填其半,日于下午偕庞天寿等骑射其中,帝亦多命中;三宫从侧楼阅视以为乐。三宫者,太后马氏,桂王原配;圣后王氏,帝之生母也;中宫王氏,正宫也。每日三宫同帝供膳止限二十四金,内寺包值;凡有赏赉,亦在其中。帝复不节省,报捷面恩奏毕,必左顾曰:`赏银十两与他'!司礼吴国泰、夏国祥等深以值日为苦。至大司礼庞天寿,自养御营兵十营;每营正总兵一〔人〕、副总兵二人、参将四人、参将以下官头二人、官头以下小卒一人耳:一营十人,十营百人。此皆天寿出自己钞以为永历视朝日仪卫拥护,亦竭力苦支矣。

桂林民力穷竭

滇营自永、全还,与焦琏兵猜疑生隙。忠贞营自蜀转战,由楚至梧休息甲士。大清遣使贻书招式耜,式耜不从。永宁州再报失利,兴宁侯胡一青还榕江;式耜复办粮械,趋出兵屯于全。民力穷竭,诛割无术,槁悴万状。永历间为废食,召廷臣议于慈宁宫,发东饷一万。

胡一青,滇之镇臣也。全与永宁州,俱属桂林。

福建尽失

大清兵围困曾庆于平和,寻出降;杀之。而诏安等处,一时俱归大清。郑芝鹏踞石榴城,大兵至,随遁去。刘中藻在福宁,势穷自缢:福建尽失。惟延、漳、汀三府界连江右,而延平所属皆在万山中;大兵既回,遂立德化王朱慈晔踞将军寨。先陷大田,继破龙溪,次顺昌、将乐。至十一月,大兵讨平之。王被执,兵部尚书罗南生等降。

\hypertarget{header-n87}{%
\subsection{卷十五·粤纪(续)}\label{header-n87}}

永历至梧州

庚寅正月朔(乙卯),永历在广东肇庆府,群臣朝贺。

前除夕夜(甲寅),大清兵过梅岭。初三日(丁巳),克南雄府;而宝丰伯罗成耀弃韶州。初七日(辛酉)报至,永历震恐,戒舟西上。戎政〔刘〕远生奏自请行清道;给事中金堡特奏请留,争之不得。时上下崩溃,武弁家丁大肆抢杀。先劫囊之厚者如冢宰晏清等、宦之显者吏部丁时魁等;凡文臣所有,悉为之掠。初九日(癸亥),永历登舟;十三日(丁卯),解维。随路劫夺,文职俱无完肤。二月初一日(甲申),永历至广西梧州府。自前至是,凡三至矣;皆以舟为家。瞿式耜疏曰:`粤东水多于山,虽良骑不能野合;自成栋归顺,始有宁宇。赋财繁盛,廿倍粤西,内强而外可备;韶州去肇庆数百里,强弩乘城、竖营固守,亦可待勤王兵四至。何乃朝闻警而夕登舟'?疏再上,而永历移德庆、抵梧州境矣。盖自成栋首疏文武各还事权、言官正气宜奖,失权者意;故急欲永历移舟,弃东如屣。

永历移武冈则有疏、前往肇庆则有疏,勿东;今移梧州则有疏,勿西。瞿公非自违也,盖以新造小邦,宜以镇定;若轻转徙,则人心易涣而叛将溃兵得以乘机劫掠,敌人闻声而至矣。至永历之易于奔迁,亦自有说:一以知文武诸臣不足恃,战不胜、守不固也;一以鉴崇祯以下数主奔避不早,悉罹亡灭,故亟亟以登舟为逃命计耳。吁!国势至此,有不土崩瓦解者乎?

由前``遗闻''观之,则以丁、蒙等诏狱为非;由后``粤记''观之,则以五虎等严刑为快。姑并存之,以俟笔之史者。

瞿式耜谏勿滥刑

时词谏诸臣疏请正纲纪、慎名器,多失人意。而御史程源辈以攫官不得,伺权者指,攻其所必去,荧惑永历听;下给事中丁时魁、金堡、蒙正发及侍郎刘湘客于狱。式耜闻报,上疏申救;谓`中兴之初,宜保元气,勿滥刑'。再疏争之,曰:`诏狱追赃,乃熹庙魏忠贤弄权,鍜炼杨、左事;何可祖而行之'?上颁敕布四人罪状,敕出忌者之手,式耜封还;谓`法者,天下之至公也;不可以蜚语饮章,横加考案,开天下之疑。且四人得罪,各有本末。臣在政府,若不言,恐失远近人望,其何辞于后世'!凡七疏。遣孙昌文入见,陈说粤西民贫食尽。时昌文孑身由海上来;阁试,授昌文翰林院简讨。

``粤事记``云:`李元胤久与陈邦傅相轧,不敢西上,挽舟崧台。丁时魁等失势,仇家尽发其结党贪纵;独袁彭年以艰先去。将金堡、丁时魁、蒙正发、刘湘客四人奉旨逮问,照北京厂卫故事,全副刑具轮番更用;以有马吉翔主事,彼固北金吾起家,纵送乘落尽其法也。招赃俱十五、六万,云为受刑不过所致;拷问时,金堡呼二祖、列宗,丁与蒙、刘则有`老爷饶命,万代公侯'等语,不计叩头而已。向之附五虎得志者大惧,倾家掩盖。永历登极三年,恭默简静,言笑无间;至是,始见声色'。

永历中秋坐水殿

庚寅五、六月间,广州固守弗下;两广总制杜允和时有报捷至梧州江渚,李元胤又于肇庆以计杀叛将罗守诚,局势稍缓。再行考选,略似人形者,无不绣衣铁简,末忝铨席。然得之非其分,即有以败之:如董云骧以台中谢恩,即叩头不起,殒于帝舟;朱士萧吏科归省,全家歼于贼手。潘骏观改铨部,见朝尚无官帽,以便服行礼,时有``方巾片片潘双鹤''口号;亦遂夺职。如此之类,不一而足。严起恒与二、三同官濯缨唱和,萧索兴味。

八月十五日,无以为金镜之献,亲书``水殿''二字置一牌坊,鼓吹送入帝舟;再令群臣上表称贺:情实孤舟嫠妇,形同画船箫鼓。

杜允和固守羊城

杜允和,河南人;李成栋之爱友。成栋没时,``两广印''允和佩之,得不亡失。次传之李栖鹏,栖鹏陷梅岭;再传之阎可义,病卒于韶州府;又传之李五老(五老者,元胤之兄,亦成栋养子),军士鼓噪而罢;又传罗守诚(守诚,浙江人;成栋之中军),亦以不协众望而罢:此皆己丑秋间事。至九月,允和摄两广篆,专守羊城。庚寅正月初七日大清兵过岭,允和与三司江槱等于十四日出城登舟,仍泊海珠寺侧;俟烽火照影,即挂帆虎头门。不意候至月终,杳无音耗;允和后率三司官属入城,各派汛地为固守计。至二月初四日,大清兵始至,驻营城北,仰攻甚难。盖羊城东、南二面距珠江,北城濠外有二里许污田,人马不得跕立;惟西门一带为山麓,允和为石重城守之。珠江以南五大县钱粮,输贡不懈。二月至十月三大战,允和晋封豫国公。

羊城,广州府,又曰五羊城。珠江,在南门外,中有海珠寺;虎头门从此去。

羊城崩陷

十月初十日(庚寅),永历圣诞;杜允和会齐文武官于五层楼拜祝,时有守西门外城主将范承恩亦在焉。承恩本淮安府皂役,从成栋入广者;目不识丁,故绰号``草包''。时允和直呼之,承恩谓辱之于众也,恨甚;遂潜通平南、靖南二王。十月二十八日(戊申),大清兵竟攻西外城;承恩退入里城,而外城失矣。连攻三日,十一月初二日(辛未)未刻,羊城崩陷;允和仍率三司官属携``两广总督印''航海而去。后二年,俱归顺,南海悉平。

瞿式耜殉节

己丑年六月,大清再发师征广;遣平南王尚可喜、靖南王耿继茂出鄱阳、逾梅岭入广东,而定南王孔有德则渡洞庭湖、牂牁(江名)入广西。时称三王征广,南京提兵索饷甚急。舟约万五千,兵俱带妻随征。

先是,永历阁臣瞿式耜留守广西桂林府,已阅三载。自戊子二月二十三日夜,乱兵劫掠,式耜下平乐、帝往南宁;君臣从此永诀。行在诸臣各私功名、各徇门户,畏避老成先达,外托留守以尊其体;实疏远之,以便己之所为耳。庚寅年,大清兵再薄全州,卫国公胡一青之兵已撤守榕江;是时勋帅咸进公、次者侯伯,桂林衙门相望,号令纷出。十一月初五日(甲寅)辰刻,报大清兵大举入严关。赵印选、胡一青、王永祚俱以分饷入桂,榕江空壁;武陵侯杨国栋、宁武侯马养麟方驰出小路军榕江,兵未战而溃。发使赵印选,印选已出城;城中大乱,沿途驱掠,式耜令戢不得。城外溃兵云飞鸟散,一青、永祚从城外去。式耜衣冠危坐署中,适总督张同敞自灵川回,过江东不入寓;过式耜署曰:`事迫矣!公将奈何'?式耜曰:`封疆之臣,知有封疆;封疆既失,身将安往'!同敞曰:`公言是矣!君恩师义,敞当共之'。遂哭,与式耜饮。家人泣,请身出危城,号召诸勋再图恢复;式耜挥去,不从。厥明,被执;见大清朝定南王孔有德,式耜以死自誓,不复一言。命幽式耜、同敞于别所,式耜赋诗,日与同敞相赓和。至闰十一月十七日,杀之。其绝命词有云:`从容待死与城亡,千古忠臣自主张;三百年来恩泽久,头丝犹带满天香'。死之日,雷电大发,远近皆为称异。时给事中金堡已削发为僧,乃上书定南王孔有德,请葬式耜、同敞;而吴江义士杨艺(字硕文)为具衣冠棺殓,并同敞瘗于北门之园。

公在狱赋诗,名``浩气吟''。自序云:`庚寅十一月初五日闻警,诸将弃城而去;城亡与亡,余誓必死。别山张司马自江东来城,与余同死;被刑不屈,累月幽囚。漫赋数章,以明厥志;别山从而和之'。其一曰:`籍草为茵枕□眠,更长寂寂夜如年;苏卿绛节惟思汉,信国丹心止告天。九死如饴遑惜苦,三生有石只随缘。残灯一室群魔绕,宁识孤臣梦坦然'!其二曰:`已拚薄命付危疆,生死关头岂待商!二祖江山人尽掷,四年精血我偏伤!羞将颜面寻吾主,剩取忠魂落异乡。不有江陵真铁汉,腐儒谁为剖心肠'!其三曰:`正襟危坐待天光,两鬓依然劲似霜。愿仰须臾阶下鬼,何愁慷慨殿中狂!须知榜辱神无变,旋与衣冠语益庄。莫笑老夫轻一死,汗青留取姓名香'!其四曰:`年年索赋养边臣,曾见登陴有一人?上爵满门皆紫绶,荒村无处不青磷!仅存皮骨民堪畏,乐尔妻孥国已贫。试问怡堂今在否,孤存留守自捐身'?其五曰:`边臣死节亦寻常,恨死犹衔负国伤!拥主竟成千古罪,留京翻失一隅疆。骂名此日知难免,厉鬼他年讵敢忘!幸有颠毛留旦夕,魂兮早赴祖宗旁'。其六曰:`拘幽土室岂偷生,求死无门虑转清;劝勉烦君多苦语,痴愚叹我太无情!高歌每羡``骑箕''句,洒泪偏为滴雨声。四大久拚同泡影,英魂到底护皇明'。其七曰:`严疆数载尽臣心,坐看神州已陆沈!天命岂同人事改,孙谋争及祖功深。二陵风雨时来绕,历代衣冠何处寻!衰病余生刀俎寄,还欣短鬓尚萧森'。其八曰:`年逾六十复奚求?多难频经浑不愁。劫运千年弹指去,纲常万古一身留。欲坚道力凭魔力,何事俘囚学楚囚!了却人间生死事,黄冠莫拟故乡游'!

临难遗表

罪臣瞿式耜谨奏:

臣本书生,未知军旅。自永历元年谬膺``留守''之寄,拮据四载,力尽心枯。无如将悍兵骄,勋镇诸臣惟以家室为念。言战、言守,多属虚文;逼饷、逼粮,刻无宁晷!臣望不能弹压、才不能驾驭,请督师而不应,求允放而不从。驯至今秋,灼知事不可为;呼吁益力,章凡数上,而朝延漠然置之。近于十月十三日集众会议,搜括悬赏;方谓即不能战,尚可以守。忽于十一月初五之辰,开国公赵印选传到安塘报一纸,知严关诸塘尽已失去;当即飞催印选等星赴危急,而印选踌躇不前,臣窃讶之!讵意其精神全注老营,止办移营一着。午后臣遣人再侦之,则已丛室而行;并在城卫国公胡一青、宁远伯王永祚、绥宁伯蒲缨、武陵侯杨国栋、宁武伯马养麟各家老营俱去,城中竟为一空矣。臣抚膺顿足曰:`朝廷以高爵饵此辈、百姓以膏血养此辈,今遂作如此散场乎'?至酉刻,督臣张同敞从江东遥讯城中光景,知城中已虚无人,止留守一人尚在;遂泅水过江,直入臣寓。臣告之曰:`城亡与亡。自丁亥三月已拚一死,吾今日得死所矣!子非``留守'',可以无死;盍去诸'!同敞毅然正色曰:`死则俱死;古人耻独为君子,君独不容我同殉乎'?即于是夜,明灯正襟而坐;时臣之童仆散尽,止一老成尚在身旁。夜雨涔涔,遥见城外火光烛天,满城中寂无声响。迨坐至鸡唱,有守门兵入告臣曰:`大清已围守各门矣'!天渐明,臣与同敞曰:`吾二人死期近矣'!辰刻,噪声始至靖江府前;再一刻,直至臣寓。臣与同敞危坐中堂,屹不为动;忽数骑持弓腰矢,突至臣前,执臣与同敞而去。臣语之曰:`吾等坐待一夕矣,毋庸执'!遂与偕行。时大雨如注,臣与同敞从泥淖中踔跚数时,始至靖江府之后门。时大清定南王孔有德已坐王府矣;靖江父子亦以守国未尝出城,业已移置别室,不加害。惟见甲仗如云,武士如林;少之,引见定南。臣等以必死之身不拜,定南亦不强;臣与同敞立而语曰:`城已陷矣,惟求速死,夫复何言'!定南霁色温慰曰:`吾在湖南,已知有``留守''在城中;吾至此,即知有两公不怕死而不去。吾断不杀忠臣,何必求死!甲申闯贼之变,大清国为先帝复仇,且葬祭成礼;固人人所当感激者。今人事如此,天意可知'!臣与同敞复定南:`吾两人昨已办一死。其不死于兵未至之前,正以死于一室,诚不若死于大廷耳'。定南随遣人安置一所,臣不薙发亦不强。只今大清兵已克平乐、阳朔等处,取梧祗旦晚间。臣涕下沾襟,仰天长号曰:`吾君遂至此极乎'!当年拥戴,一片初心,惟以国统绝维之关系乎一线;不揣力绵,妄举大事。四载以来,虽未竖有寸功,庶几保全尺土。岂知天意难窥、人谋舛错,岁复一岁,竟至于斯!即寸磔臣身,何足以蔽负君误国之罪。然累累诸勋,躬受国恩,敌未临城,望风逃遁;大厦倾圮,固非一木所能支也!臣洒泪握笔,具述初五至十四十日内情形,仰渎圣听;心痛如割,血与泪俱。惟愿皇上勿生短见,暂宽圣虑,保护宸躬;以全万姓之命、以留一丝之绪!至于臣等罪戾,自知青史难逃;窃计惟有坚求一死,以报皇上之隆恩、以尽臣子之职分。天地鬼神,实鉴临之!临表,不胜呜咽瞻仰之至。

张同敞殉节

张同敞,湖广江陵人;曾祖居正,相神宗有声。崇祯间,同敞以荫补中书舍人。至十七年,闯贼李自成陷北京,怀宗殉难,贼索朝官甚急,文武逼降者多;同敞藏匿民间,潜出城,徒步南归。时宏光嗣位,同敞痛怀宗之死,服丧三年,誓不仕;往来吴、浙山水间。

及南都复陷,同敞入闽。适隆武新立,博求先朝旧臣,时宰言同敞,亟召见;上悲喜甚,命之官。力辞,上曰:`尔祖有功,先朝曾荫锦衣卫指挥。使今尔不受职,数年后此爵湮矣;尔纵欲报先帝,奈祖爵何?强为朕袭锦衣官。尔文人不当授武职,然朕文武兼任,尔慎毋过辞'!同敞感泣;不得已,改授锦衣卫指挥使。时隆武二年二月也。未几,堵〔胤〕锡督师楚中,收降余贼李赤心等;表至行在,上谓同敞曰:`楚尔父母邦也,尔家世有名于楚,素为楚人所信服。今降贼在楚地,可往为朕抚之;俾戮力报效,毋扰赤子'。同敞受命,行至楚,谕胤锡抚赤心等,宣布上威信,群贼稽颡归化,无不感服。同敞遂即复命还朝;行至粤界,闻八闽不守,同敞仰天大哭,如穷人无所归。

及永历即位端州,粤东已陷,上留大学士瞿式耜守粤西;驾幸武岗,起同敞入朝。同敞见上,号哭不已;上曰:`尔文人也,复有大节;何可以武职屈'!因改授翰林院右春坊侍读学士。丁亥八月,寇陷武岗;上狩粤西,同敞为乱兵所掠,避入黔地。时黔、粤隔绝,人情汹扰,数月不闻行在消息。川、黔士绅,议立荣、韩二藩;同敞与钱邦芑及郑逢元、杨乔然力争不可,众议乃沮。戊子,同敞从间道赴行在,升詹事府正詹事;留守瞿式耜疏荐同敞知兵、得士心,上命以兵部侍郎经略楚、粤兵马。时兵弱饷匮,同敞身在行间分甘苦,以忠义激劝,将士人人自奋;每接战,同敞即以死誓。

及庚寅冬,同敞督开国公赵印选、卫国公胡一青连营于桂林之小榕江;十一月初五日,大清兵至,两营战败。同敞率数骑入桂林城,时军民俱散,留守大学士朝服坐堂上,誓与城亡;及见同敞至喜,曰:`我守臣,不容他适;子军中总督,自宜行。天下事尚可为乎,子勉之'!同敞笑曰:`公能为朝廷死,同敞独不能乎?何相待之薄也'!连取酒共饮,坐而待之。次日,大清兵入城;同敞与式耜见孔有德,两人不跪,同敞尤大骂。有德部下捶辱之,同敞骂愈厉。有德命拘二人于城北一小室,中命左右说之降;劝谕百端,式耜但大哭、同敞则毒骂,暇则两人赋诗。有德愤甚,命折同敞右臂,仍谈笑赋诗不绝。同敞右臂既损,诗成,式耜代书之。两人幽囚唱和者四十余日,诗各数十章。有德见两人困愈久、苦愈甚而志愈坚烈,知终不可辱;至闰十一月十八日,杀之。金堡时已为僧,致书于孔有德,乃收殓瞿、张两公尸葬于白鹤山下。上闻同敞死,深为痛悼,累日不食,望而祭之;赠陵江伯。无子。所著诗文四十余卷,以兵燹亡失。止临难时绝命词数十章传达行在,上读而悲焉;命工部刻传之,赐名``御览伤心吟''。

金堡上孔定南王书

茅坪衲僧性因和尚,谨致书于定南王殿下:

山僧,梧水之罪人也。承乏掖垣,奉职无状;系锦衣狱,几死杖下。今夏编戍清浪,以路道之梗,养痾招提;皈命三宝,四阅月于兹矣。车骑至桂,咫尺阶前而不欲通;盖以罪人自处,亦以废人自弃,又以世外之人自恕也。

今且有不得不一言于左右者:故督师大学士瞿公、总督学士张公皆山僧之友,为王所杀,可谓得死所矣。敌国之人,势不两存;忠臣义士杀之而后成名,两公岂有遗憾于王!即山僧亦岂有所私痛惜于两公哉!然闻遗骸未殡,心窃惑之。古之成大业者,杀其身而敬且爱其人,若唐高祖之于尧君素、周世宗之于刘仁瞻是也。我明太祖之下金陵,于元御史大夫福寿既葬之矣,复立祠以祀之;其子犯法当死,又曲法以赦之:盛德美名,于今为烈。至如元世祖祭文天祥、伯颜恤汪立信之家,岂非褒扬忠义、扶植彝伦者耶?山僧闻尝论之:衰国之忠臣与开国之功臣,皆受命于天,同分砥柱乾坤之任。天下无功臣,则世道不平;天下无忠臣,则人心不正。事虽殊轨,道实同源。两公一死之重,岂轻于百战之勋哉!王既已杀之,则忠臣之忠见、功臣之功亦见矣。此又王见德之时也,请具衣冠为两公殓。瞿公幼子,尤宜存恤;张公无子,益可矜哀。并当择付亲知,归葬故里;则仁义之举,王且播于无穷矣!如其不尔,亦许山僧领尸,随缘稿葬;揆之情理,亦未相妨。岂可视忠义之士如盗贼寇仇然,必灭其家、狼籍其肢体而后快于心耶!夫杀两公于生者,王所以自为功也;礼两公于死者,天下万世所共以王为德也:惟王图之!

物外闲人,不辞多口。既为生死之交情,不忍默默;然于我佛``冤亲平等''之心、王者泽及枯骨之政、圣人维护纲常之教,一举而三善备矣。山僧跛不能履,敢遣侍者以书献,敬候斧钺;惟王图之!

郑之珖传

郑之珖,四川广安州人;崇祯庚午举人。庚辰,授广东高州府推官。粤东素富饶,而高州又濒海、去京师远,官其地者多贪墨不检。之珖独清约,不妄取,于刑狱尤甚;以故士民戴之。考绩以最闻,以之珖为通州知府;未赴,值闯贼陷北都,之珖为粤士民攀留不得去。及隆武嗣位闽中,召之珖入,授工部主事,升员外。丁亥秋,八闽皆陷,士绅半降;之珖削发为僧,卖药于广之新会县。有司及土人逼胁万端,之珖终不易志。

戊子,李成栋归明;之珖乃蓄发赴行在,改授户部员外。庚寅二月,升礼部祠祭司郎中;典试贵州。五月,至贵阳。适流寇孙可望入黔,以兵胁朝廷索封``秦王'',大学士严起恒、杨鼎和、兵科给事中刘尧珍、吴霖、张载述建议不从,可望遂命部下杀五人,投其尸于水,乃自称秦王。上惊悼不已;赐可望名朝宗,遣官抚谕之。可望遂尽胁诸文武,授以官爵,改铸印章,更立制度;有不从者,辄诛之。一时士绅怵其威,无不屈从者。之珖乃弃官携妻孥隐于湄水之阳,自号蛾眉道者。之珖素贫,居官绝苞苴,行李萧然;躬耕自给,或至并食,恬如也。

时钱邦芑弃官隐于余庆之蒲村,相去三舍。寒暑朝昏得村酒一壶,必相招共饮;醉则悲歌不辍。及甲午春,邦芑迫于可望之征逼,祝发为僧,号大错和尚;之珖闻之,大哭,走唁邦芑曰:`昔吾遇闽难为僧,今公遇贼亦为僧,天厄我辈,固如是乎'!自是,放情诗酒,不复以人世为意。至丙申九月,之珖忽病;谓妻汤氏曰:`我若不起,大错和尚必来,后事惟彼可托'。至十月初五日,卒。邦芑闻讣,奔往哭之。时其友山阴胡钦华、门人西川陶五柳、湄水龚惟达、吴开元、赵时达俱来会哭,因私谥之曰``贞确先生'';卜葬于湄水桥西,为立碑表其墓。

之珖娶吴氏,继杨氏。子三:长先卒,次方三岁,次方一岁;江津程源为抚其两孤。所著有``明书''二十卷、``罏史''八卷、``椟庵文集''六卷、``诗集''七卷、``纪难''二卷,行于世。其所杂着尚多,俱散失不可考矣。

钱邦芑曰:`士之犯难不辱,激于一时、义形于色,易易也;至屡遭大变,百折不挫,几几乎难哉!三十年来国难频兴,所见抗节自全者固不乏人;若夫张、郑二子,文章、事业已龙变鸿翥,光昭天壤矣。而矫矫志节,复风被百世;振起懦顽,自非祖宗布德之深、养士之善,曷臻此哉!野史议曰:``同敞始不污贼、终不屈于大清,之珖始不屈于大清、终也不污于贼;二子之死,慷慨从容虽有异,要其清白一节、始终不辱,一也''。宣圣曰:``不降其志,不辱其身'';二子之谓欤?以备国史述采择云'。

乡城异岁

庚寅年大统历,两广、云、贵地方永历于己丑年十月朔颁发,闰月在十一月;广东广州府省城与广西桂林府省城俱前十一月内失陷,下而肇庆、高、雷、浔、梧、平庆等府一切道府州县大小官属则于十一月下旬陆续抵任,所奉者大清时宪历也。时宪闰月不在庚寅而在辛卯二月,一时城中官府军丁自北来者,悉以十二月朔为辛卯元旦,行拜贺礼;各乡镇居民仍永历旧历,则以辛卯年二月朔日为元旦。守除、拜岁,有乡城之别;至交四月,岁时始同:亦一异也。

袁彭年献金

十一月初五日,佛山至羊城最近,袁彭年首先投诚,献犒军银八百两;哭诉当年迫于李成栋之逆贼,后则着着仍为大清朝,此心可表天日。因求降级,实授通判,亦逼于富妾之命也。平南王、靖南王挥出之。

黄士俊薙发

先,朝辅黄士俊、何吾驺及乡绅杨邦翰、李贞、吴以连俱投诚恐后,当时打油腔嘲士俊有`君王若问臣年纪,为道今年方薙头'之句。盖崇祯末年,士俊曾膺存问也。士俊,广东广州府顺德县人;字亮垣,号振宇。万历癸卯年举人;丁未状元,时年二十五岁。至是,年已八十二;状元、宰相俱遭此几代升沈,所生不辰矣!他人不寿为不幸,而士俊又以多寿为不幸也。噫!

永历梧江酉奔

十月初七日(辛巳),永历挽舟梧州城外;闻羊城尽失,俱各奔窜。移舟西上,不五里遂抢杀遍行。上至藤县,分为两股:从永历者上右江,若严起恒、马吉翔等是也;余则入容县港,若王化澄等是也。上右江者,至浔州道上,兵各溃散;永历呼之不应。入容县港者,于北流境上,为土寇劫夺;弃妻失妾、亡子遗仆,比比皆是。茕茕只身,步行足重,乞食羞颜。向为鸳班贵客、今为鹄形丧狗,哀苦万状,生不如死矣。

永历再上南宁

永历再上南宁府,仍以府署为行宫。时陈邦傅为李定国所驱,不知所之;赵只因邦傅强夺其女,遁入土司。所喜阁臣严起桓尚不忍舍去,同大金吾马吉翔、大司礼庞天寿三人班荆对泣而已。

永历在南宁
辛卯正月己酉,永历在南宁,免朝贺。正、二两月,稍觉平安。间有旧臣从别道而至者、又有新臣贪爵衔而就职者,文武两班位列楚楚;然亦薤上露、水中沤,究无所恃也。

孙可望入南宁

二月朔(戊寅),孙可望忽发兵三千直抵邕江,亟取兵科都给事中。当日现任其职者,应天吴晋也。可望腰斩之,犹以为向日之金堡耳;不知官是而人非矣。

吴晋,字叔山;上元县生员也。邕江,即南宁府。 严起恒被难

孙可望既杀吴晋,复取首相严起恒,与之追论不封``秦王''之故;相对舟中,犹然成礼。及别后,则竟挤入邕江。江出交趾,流极迅暴;起恒家人急驾舟往救,直追至横州,始得其尸。呜呼!亦可谓死得其所矣。其余朝臣悉皆奔散,为生、为死,不得而闻之。

何、辜毁灭

两广军民杀戮百万、城郭村墅毁灭万千;致世界两次鼎革者,皆由辜朝荐、何吾驺争事权、分尔我所起。吾驺家赀三百万,所居号大澜、小澜,巍焕壮丽,海内无比;辛卯年,总付之一炬。初,辜朝荐引李成栋入广,自谓首建大功;而潮阳士庶久恨其开祸西粤,于辛卯四月激于义愤,尽毁灭其家云吾驺一炬,多藏厚亡也;朝荐毁减,天道好还也。利徒戎首,可以鉴矣!

贞女绝命诗

女郎,不知何氏,大约湖南闺秀也。顺治十一年(甲午)秋,兵旋被掳;女郎抗志不辱。行至鹦鹉洲,伺间投江死,浮身于黄鹤渚,有司怜而命瘗之。乃于衣裾间得绝命诗云:`征帆已说过双姑,眼泪声声泣夜乌;葬入江鱼波底没,不留青冢在单于'。其二曰:`厌听行间带笑歌,几回肠断已无多;青鸾有意随王母,空费人间设网罗'!其三曰:`遮身只是旧罗衣,梦到湘江恐未归;冥冥风涛又谁伴?声声遥祝两灵妃'。其四曰:`少小伶仃画阁时,诗书曾拜母兄师;涛声夜夜催何急,犹纪挑镫读``楚词'''。其五曰:`影照江干不暇悲,永辞鸾镜敛双眉;朱门曾识谐秦、晋,死后相逢总未知'。其六曰:`生来弱质未簪笄,身没狂澜叹不齐!河伯有灵怜薄命,东流直绕洞庭西'!其七曰:`当时闺阁惜如金,何事牵裾遂水滨!寄语双亲休眷恋,入江犹是女儿身'。其八曰:`国史当年强记亲,杀身自古以成仁;簪缨虽愧奇男子,犹胜王朝共事臣'!

顺治辛丑仲秋十日,予始得此。读前数章,想见贞节女子;读至卒章``杀身犹胜''等语,则非闺秀口角,俨与文山争烈矣。惜乎!失其氏里。

\hypertarget{header-n92}{%
\subsection{卷十六·粤纪(续)}\label{header-n92}}

安龙纪事

壬辰二月初六日,上自广西南宁移跸贵州安龙府(本安笼所)。时云、贵皆为孙可望所据;初阳尊朝廷,要封``秦王'',朝廷内外臣子稍忤其意,则击斩随之,以故□皆胁署伪职。及大清兵陷广西,可望随改安笼所为安龙府,迎上居之;宫室礼仪,一切草简。

时廷臣扈随者,文武止五十余人。中有马吉翔者,本北京市棍也;性便黠,颇识字。初投身内监门下充长班,复为书办;逢迎内监,得其欢心,故内监皆托以心腹。及高起潜入典兵,吉翔窜入锦衣卫籍,冒授都司。居起潜门下,荼毒军民,无所不至;后又贿升广东都司。及乙酉隆武即位福建,吉翔解粤饷赴行在,自陈原系锦衣世职;遂冒升锦衣卫指挥。后奉使楚中,谀诸将;凡报军功,必窜入其名。屡冒边功,渐次升至总兵。及永历即位,又营求宫禁勋戚,得封文安侯。吉翔历事既久,专意结媚宫竖;凡上举一动,无不预知,巧为迎合。于是上及太后皆深信之,以为忠勤;遂命掌戎政事。及至安龙,见国事日非,遂与管勇卫营内监庞天寿谋逼上禅位秦王,以图富贵;独虑内阁吴贞毓及朝中大臣不相附顺,阴嗾其党冷孟銋、吴象铉、方祥亨交章参毓。先是,濑湍移跸时,贞毓欲上暂留,以系中外人民之望,遂与寿、翔忤;至是,两逆交煽,急谋去贞毓。而孟銋等参疏屡上,上素知贞毓忠贞,俱寝不行。寿、翔、〔谓〕銋等曰:`贞毓入阁视事,则我不得参预机密。公等参贞毓,徒费纸笔。今秦王权倾内外,我具一启托张提塘封去求秦王令,谕以内外事委戎政、勇卫两衙门总理,则大权归我两人;我两人作秦王心腹、公等作羽翼,然后徐谋尊上为太上皇,让位于秦王,则我辈富贵无量,贞毓何能为乎'?吉翔遂遣门生郭璘说武选主事胡士瑞云:`今上困处安龙,大势已去。我辈追随至此,无非为爵位利禄耳。揣时观变,当归秦王。况马公甚为倚重,目下欲以中外事属之;若公能达此意于诸当事共相附和,力劝禅位,何愁不富贵。不然,我辈俱不知死所矣'!士瑞即厉声叱璘曰:`汝丧心病狂,欺蔑朝廷;遂谓我辈亦随波逐流乎'?璘惭而退。吉翔复遣璘持白绫一幅,求武选司郎中古其品画``尧舜禅受图'',欲以进秦王;其品愤怒,不画。吉翔阴报秦王,秦王遂将其品锁去,立毙之杖下。

六月,秦王有札谕天寿、吉翔云:`凡朝廷内外机务,惟执事力为仔肩。若有不法臣工,一听戎政、勇卫两衙门参处,以息纷嚣'。札到,中外惶惧;独吏科给事中徐极、兵部武选司员外林青阳、主事胡士瑞、职方司主事张镌、工部营缮司员外蔡縯等相谓曰:`天寿、吉翔曩在楚、粤怙宠弄权,以致楚、粤不戒,銮舆播迁。今不悔祸,且包藏祸心,称臣于可望。一人孤立,百尔寒心。我辈若畏缩不言,不几负国恩、羞鹓列乎'?由是,各疏参二逆罪状;章三上,上始知两人欺君卖国,并发其在安龙时曾偷用御笔,私封龙府土官赵维宗为龙英伯。上怒,即召集廷臣,欲治寿、翔罪;寿、翔惧,急入内廷求救太后,得免。两人奸既露,怨愈甚;欲谋杀极等。于是专意谄附可望;凡可望欲为者,二人辄先意为请。可望愈肆无惮,自设内阁六部科道等官;一切文武,皆署伪衔。复私铸八叠伪印,尽易本朝旧印。而贼臣方于宣谄可望尤甚,为之定仪制。立太庙,享太祖高皇帝主于中、张献忠主于左,而右则可望祖父主也。拟改国号曰``后明'',日夜谋禅位。上仅守府,势甚岌岌;私与内监张福禄、全为国曰:`可望待朕,无复有人臣礼;奸臣马吉翔、庞天寿为之耳目,朕寝食不安。近闻西藩李定国亲领大师直捣楚、粤,俘叛逆陈邦傅父子;报国精忠久播中外,军声丕振。将来出朕于险,必此人也。且定国与可望久有隙,朕欲密撰一敕,差官齎驰行营,召定国来护卫;汝等能为朕密图此事否'?福禄等即奏曰:`前给事徐极、都司林青阳、胡士瑞、张镌、蔡縯于秦王发札宠任天寿、吉翔时,曾抗疏交参,忠愤勃发;实陛下一德一心之臣也。臣将圣意与他密商,自能得当以报'。上允之。福禄与为国诣张镌、蔡縯私寓,适极与青阳、士瑞俱至,福禄等密传意;诸臣叩首云:`此事关系国家安危,首辅吴公老成持重,当密商之'!五人即诣贞毓寓言其事;贞毓曰:`今日朝廷式微至此,正我辈致命之秋也。奈权奸刻刻窥伺,恐机事不密;诸公中谁能充此使者'?林青阳即应曰:`某愿往'。贞毓曰:`固知非公不可;但奸人疑阻,须借告假而行可也'。青阳乃即日请假归葬。贞毓属祠祭司员外蒋干昌密拟敕,属职方司主事朱东旦缮写、福禄等密持入用宝;青阳即日陛辞。时可望沿途有塘拨盘诘,阴藏密敕,从间道驰出。此六年十一月事也。

癸巳六月,上以青阳去久不回,欲差官往催;贞毓即以翰林院孔目周官对。武安伯郑允元云:`此番比前更要慎重。今日吉翔在左右,日夜窥探,凡事必报可望;必须先将马吉翔差出,使他不得窥探,事乃可济。若吉翔在内,则奸党蒲缨、宋德亮、郭璘、蒋御曦等往来奔走,阴伺举动,深为不便'。时因节届霜降,上以兴陵越在广西,例用勋臣一员代祭;遂使吉翔往粤行礼。去后,即命蒋干昌撰敕,复遣周官賷往;官涕泣受命而行。时吉翔奉差在粤,探知青阳賷有密敕至李定国营,私差汪锡玄至营探听。未几,而刘议新途遇吉翔,不知吉翔不与谋,向告;吉翔大惧,遂逼令具启报知秦王西藩接敕之事,又嘱其弟雄飞尽出家赀阴赂提塘王爱秀,求其应援。时吉翔党与布列甚密,日伺探听。上孤立自危,以台省员缺敕部考选。于十二月二十四日临轩亲试,将蒋干昌、李元开选翰林院简讨,张镌选刑科给事中,李颀、胡士瑞选浙江福建两道监察御史;杨钟、徐极、蔡縯、赵赓禹、易士佳、任斗墟、朱东旦等亦以资深俸久,各加秩升职。自是,天寿、雄飞益相危惧;谓蒲缨、宋德亮、郭璘、蒋御曦等曰:`凡我仇敌俱选清华,我辈危矣'!缨等曰:`时闻周官之行,系众人密谋;待马公察访详悉,具报秦王,则此辈死无日矣'。不数日,马吉翔果具密启与秦王,报知此事;天寿、雄飞持启诣王爱秀云:`马公访得朝中有两次差官,賷敕往西藩去,召他带兵迎驾。现有启报秦王,烦公即发拨启闻'!秀闻大惊曰:`果有此事!我系提塘,亦当具启报知'。寿、飞即下拜曰:`公果具启,救我辈性命,诚再生之恩也'!启去,可望大怒;甲午正月,差郑国往南宁马吉翔处探听周官事迹,并看西府兵势。时吉翔疏证青阳、周官甚急,由是吏科都给事徐极、大理寺少卿杨钟、太仆寺少卿赵赓禹、光禄卿蔡縯、刑给事张镌、御史李颀、福建道监察御史胡士瑞交章参吉翔欺君卖国、天寿表里为奸。上见事急,即敕廷臣公议治罪;天寿惧,与雄飞数骑逃出。雄飞遂见秦王,将密敕与谋之人一一报知;而十八人狱成矣。

先是,正月内,林青阳回行在复命。至田州总镇常荣营,荣知密敕事已发,止青阳勿回行在,即密奏上;青阳遂留营中,暗遣心腹刘吉至行在,藏张镌、蔡縯寓,即密奏上。上甚喜,即擢青阳兵科给事。上谓贞毓曰:`仍撰敕与青阳。敕内先要说寿、翔表里为奸,将谋不利于朕,着令翦除;俟朕与将军握手时,即行告庙晋封之典。发金二十两为西藩铸印'。贞毓拟篆``屏藩亲臣''四字发与青阳,差人刘吉领去;青阳接敕与金,常荣发兵护送。至广东广州,得遇周官;同青阳将空敕书写及``屏翰亲臣''四字样铸成,遂至高州西藩李定国营内。讵意可望差标官至常荣营急拿青阳,已去旬日矣;遂将荣撤回。而郑国已于南宁取吉翔回行在,可望亦疑吉翔与谋,令行在各官与吉翔对理密敕之事;各官既集,郑国云:`马吉翔已拿在此;列位要明白说出林青阳、周官齎敕之事,他果与谋否?以便回覆国主'。贞毓云:`学生职司票拟,关防严密,如何晓得'!国云:`既如此,我到朝内请上面对'。诸臣俱造朝候;上御文华殿,召郑国、王爱秀进殿,国与爱秀奏云:`西藩私通朝内奸臣,胁敕要封;国主已发人往拿正法。林青阳、周官不日便到,皇上可知是何臣主持?待臣等好回覆国主'。上云:`密敕一事,朝中臣子必不敢做。数年以来,外面假敕、假宝亦多,尔等还要密访;岂皆是朝里事'!国与爱秀愤愤而出,即同天寿汹汹至朝房云:`我们要回贵州,列位须快说明白'。贞毓云:`皇上虽值播迁,朝廷法度尚在,谁敢妄行!学生们实不晓得'。天寿力证曰:`你如何推避得'!国与爱秀即将贞毓扭出朝房;一任天寿指挥,又将杨钟、郑允元、蒋干昌、蔡縯、赵赓禹、张镌、徐极、李颀、胡士瑞、李元开、朱东旦、朱议、周允吉、许绍亮、胡世寅、陈麟瑞、易士佳、任斗墟等俱收锁王爱秀宅内。随带家丁同天寿进宫,拿内监张福禄、全为国、刘衡,宫中大震。少顷,福禄与为国、刘衡铁索系出,惟胡世寅于是日释放。此甲午三月初六日事也。入朝时天气清明,及诸君子被执,忽烈风霾日,阴云惨黑。安龙士民惊曰:`此天壤间一大变事'!其逆党冷孟銋、朱企鋘、蒲缨、宋德亮且扬扬得意,犹奏:`上速将密敕情由指出是出何人所为,以便处分;不然,危亡在旦夕矣'!上曰:`汝等逼朕认出,朕知是谁'!因悲愤而退。翌日,国具严刑拷究。先将贞毓妾父户部员外裴廷谟提到,国叱谟跪;谟厉声曰:`我是朝廷五品大夫,如何跪你'!国怒,令乱棍交下,几断两臂。复将谟拷夹,问密敕事;谟不应。次将张镌、徐极、周允〔吉〕、赵赓禹、蔡縯、任斗墟、陈麟瑞、张福禄、全为国等一一酷刑拷鞫,惟贞毓以大臣免刑;余皆夹数夹、笞数百,痛苦难禁,惟呼二祖、列宗。时天色晴明,忽风雷震烈。蔡縯厉声曰:`我辈枉取刑辱;取纸笔来,待我拱招'。国即将縯扭解放松,縯持笔告天曰:`皇天后土、二祖列宗!今日蔡縯供招与谋密敕之事,以见臣子报国苦衷'。由是一一写出。国又问曰:`皇上知否'?縯恐有害国家,答曰:`未经奏明'。招罢,仍扭锁收管。越三日,将许绍亮、裴廷谟释放;绍亮流涕不肯出狱,向十八人曰:`今日同事为国,生死与共;安忍独生'!贞毓等曰:`公今得生,是天未尽灭忠臣。尔既生,我辈虽死犹生'。绍亮等挥泪拜辞,十八公挥泪答拜;绍亮同廷谟出狱。天寿、吉翔出家赀厚赂国、秀;吉翔以女送郑国为妾,国留宿三日遣还。即诬十八公以``欺君误国、盗宝矫诏''为辞飞报;可望发令于本月二十日到安龙,以十八为奸、以吉翔为忠,请上裁断!国等请上召对,上忧愤御殿,随发廷臣公议。由是,吏部侍郎张佐辰、绥宁伯蒲缨、太常寺少卿冷孟銋、武选司郎中朱企鋘、总兵宋德亮、刑部主事蒋御曦等俱附耳向郑国云:`这些官,今日都要处死。若留一个,祸根不绝'!国云:`自然;还须列位主持'!维时刑部司官蒋御曦执事、吏部侍郎张佐辰票旨,竟以``盗宝矫诏、欺君误国''八字为案。定张镌、张福禄、全为国凌迟;蒋干昌、徐极、杨钟、赵赓禹、蔡縯、郑允元、周允吉、李颀、胡士瑞、朱议◆、李元开、朱东旦、任斗墟、易士佳等为从,拟弃市;惟贞毓以大臣,赐绞。陈麟瑞与佐辰同乡、同年,力救,得杖一百二十,拟遣戍。刘议新杖一百二十;越五日死。刘衡杖一百,免罪。复以福禄乃中宫近侍,用宝发敕虽皇上自行,中宫俱知其事;天寿、吉翔等将废中宫,嘱仪制司萧尹上疏,引古废后事为例。维时中宫流涕,哭诉上前,始免。遂将诸君子缚赴法场,俱能神色不变,望阙叩头云:`臣子一念,今日尽矣;无以报国,虽死有余责耳'!又云:`天寿、吉翔、雄飞朋胁为奸,欺君卖国。我辈今日为他杀尽,他日必借秦王势挟制天子,为所欲为;中兴大业,从兹已矣'!张福禄曰:`我辈生不能杀此三贼,死当作厉鬼杀之,以除国害'!诸君子临刑绝无戚容,各赋诗见志。吴贞毓诗云:`九世承恩愧未酬,忧时惆怅乏良谋!躬逢多难惟依汉,梦绕高堂亦报刘。忠孝两穷嗟百折,匡扶有愿赖同俦'!蒋干昌诗云:`天道昭然不可欺,此心未许泛常知。奸臣祸国从来惨,志士成仁自古悲!十载辛勤为报国,孤臣百折止忧时。我今从此归天去,化作河山壮帝畿'!李元开诗云,`忧愤呼天洒酒卮,六年辛苦恋王畿;生前只为忠奸辨,死后何知仆立碑!报国痴心容易死,还家春梦不须期!汨罗江上逢人旧,自愧无能续``楚词'''!朱东旦诗云:`邕江昔日五君子,随扈安龙十八人;尽瘁鞠躬今已矣,忠臣千载气犹生'!朱议◆诗,有`精忠贯日吞河岳,劲气凌霜砥浪涛'之句,词极悲壮;余不及详纪。赋诗毕,仍对各官拱手曰:`学生辈行矣,中兴大事交付列位。但列位都要忠于朝廷,切不可附天寿、吉翔卖国!学生辈虽死犹生也'!言罢,引颈受戮。时安龙虽三尺童子,无不垂涕者。郑国仍将诸君子暴尸三日;时天气炎热,颜面如生,各家亲族买棺收殓。

十八忠臣既死,雄飞遂自黔回。吉翔倚藉可望,挟制朝廷,复预机密;引其党张佐辰、扶纲摄相行事,内外大权尽归庞、马。时人以佐辰相貌丑劣,谄事权奸,供庞、马指挥,号拱辰为判官、扶纲为小鬼,而国势日削矣。

吴贞毓,宜兴人;丙辰生,时年九十有七。论者谓寿享百岁、榜登会元、官居宰相、名著忠臣,此人间四难也,而萃于一人。呜呼!盛哉。

钱邦芑祝发记

自庚寅八月孙可望入黔,逼勒王号,迫授余官,拒不受;退隐黔之蒲村,躬耕自给。历辛卯迄癸巳,可望遣官逼召一十有三次,余多方峻拒;甚至封刃行诛,余亦义命自安,不为动也。

甲午二月二十三日,为余初度之辰。山阴胡凫庵、邻水甘羽嘉、富顺杜耳侯、西湖许飞则、渝州倪宁之、遂宁黄玺卿、湄水马仲立、黄月子同集假园,酾酒祝余。适广安郑于斯致书云:`偶以采薪不能来,谨寄一赞为寿'。赞云:`昔与先生同朝,帝尝曰:``直臣矣,汲黯有其骨而学术逊之''。今与先生同隐,人咸曰:``隐者矣,严光有其高而气节逊之''。夫汲黯无学术、严光无气节,吾有以知先生矣'。诸子读是赞,举觞祝曰:`非郑公不知先生,非先生不足当是赞也'。余再拜,谢曰:`芑不敏,敢忘诸君子今日之训,以贻知己羞'!次日,余庆县令邹秉浩复将可望命,趣余就道;威以恐吓,危害万端。余酌酒饮之,谈笑相谢。凫庵知余意,席间私赋诗曰:`酒中寒食雨中天,此日衔杯却惘然!痛哭花前莫相讶,不如往泛五湖船'!是晚,余遂祝发于小年庵;乃说偈云:`一杖横担日月行,山奔海立问前程;任他霹雳眉边过,谈笑依然不转睛'。是时,门下同日祝发者四人,曰古心、古雪、古愚。时古心亦有偈云:`风乱浮云日月昏,书生投体向空门;不须棒喝前因现,慷慨随缘念旧恩'。次日,祝发者五人,曰古德、古义、古拙、古荒、古怀。次日,又二人,曰古圉、古处。时诸人争先披剃,呵禁不得;余委曲阻之,譬晓百端,余乃止。

先后随余出家者,盖十有一人。因改故居为``大错庵'',俾诸弟子居之,共焚修焉。

邦芑被可望扭械至黔,途中口占:`才说求生便害仁,一声长啸出红尘;精忠大节千秋在,桎梏原来是幻身'!其二:`扭械营缠悟夙因,千磨百折为天伦;虚空四大终须坏,忠孝原来是法身'。其三曰:`前劫曾为忍辱仙,百般磨炼是奇缘;红炉□□□□雪,弱水洋中泛铁船'。

孙可望犯阙、败逃本末

甲午三月,孙可望以受禅不遂,深忌宰相吴贞毓等;适上有密旨召李定国入卫,遂以``盗宝假敕''之名诬贞毓等十八人,杀之于安龙府。
乙未十一月,遣总兵张明志、关有才等往广西暗袭定国。时定国在广东为大清兵所攻,败于新会;收集残兵万余,驻扎南宁府,势甚单弱。闻张明志等将近,计无所出,召中书金维新、曹延生计之;二人曰:`明志等兵虽多,皆帅主旧部下,安敢相敌。今明志等从大路来,我从小路迳截其后,彼出不意,定然惊溃;我辈乘胜率兵至安龙迎皇上驾,径至云南,美名、厚实兼收之矣'。定国然之;与靳统武、高文贵等集兵万人,拔寨而起。从小路行五日,抄出明志营后;卒然冲之,明志等不知兵从何来,前后大乱。定国乘势急追,要截残兵,得三千人;遂连夜赴安龙府。时可望闻明志兵败,料定国必至安龙;疾召白文选带兵数百至安龙迎驾,幸贵州。时丙申正月也。先时,可望投顺后,钱邦芑见其跋扈,可望部将有白〔文〕选者忠诚可托,私语之曰:`忠义,美名也;叛逆,恶号也。孺子且辨之,丈夫可陷身不义乎'!文选感其言,遂与私誓。遣至安龙请驾,文选知定国兵将到,托以夫马不足,故缓行期两日;后定国兵果至,定国谓文选曰:`圣驾宜幸云南。我与秦王原系弟兄,彼此和好,同辅国家,何事不可为;然全藉众调停耳'!定国遂护驾径至云南,将可望所造宫殿,请上居之。时丙申三月也。定国命靳统武执马吉翔家眷数人,防其出入;欲请诏治罪。

时文选回贵州,可望大怒,欲举兵与定国决战;文选曰:`天子在彼不便,两和为是'。可望命文选入云南议和。文选入朝,上即封李定国为晋王、刘文秀为蜀王;时艾能奇已死,授其子总兵;封白文选为巩国公、王尚礼保国公、王自奇夔国公、张虎淳化伯;靳统武、高文贵、窦民望皆授总兵。王尚礼、王自寄、张虎皆可望心腹,而张虎奸黠,尤用事;虎自以位在诸人下,甚怏怏。文选密谓定国、文秀曰:`今可望死党王尚礼、王自奇拥重兵在辇毂之下,而张虎尤诡,日伺左右,祸且不测;今欲与可望议和,须奏皇上遣张虎行,乃可无反覆'。定国、文秀于是奏上;召张虎至后殿,上曰:`秦、晋两王义当和好,此须卿一行'!虎拜受命。上从头上拔金簪一枝,赐虎曰:`和议成,卿功不朽,必赐公爵。此簪赐卿为信;此去见簪,如见朕也'。虎临行,私谓王尚礼、王自奇曰:`我此行不半年,必与秦王整顿兵马来取云南;尔二人如何接应'?自奇曰:`尚礼率亲丁在城内为内应,我兵马俱扎楚雄、姚安一带;秦王自黔来,我从楚雄而下夹攻之,尚礼为内应。定国、文秀不满三万人,又皆疲弱;我辈上下精兵二十万,彼能支乎'?张虎辞行。至黔,见秦王曰:`上虽在滇,端拱而已。文武两班唯唯诺诺,内外大权尽归李定国。定国所信,则中书金维新、龚铭,武则靳统武、高文贵。终日升官加赏,兵马不满三万人,无固志;可唾手取也'。可望大悦。虎复上封伯印缴还。可望,曰:`在彼处不受,恐生疑忌;故伪受之。臣受国主厚恩,岂敢背贰哉!白文选受国公之职,已为彼所用矣'。因请屏退左右,取上所赐簪示可望曰:`臣临行时,皇上赐此簪,命臣刺国主以报功,许封臣``二字''王;臣不敢不以上闻'。可望信以为然,愤怒愈甚,而犯阙之意于是决矣。伪翰林方于宣者谄事可望,正在可望宫中;献计曰:`臣有二策;但用其一,不烦一兵而皇上自毙。定国、文秀二人之首,自然致矣'!可望问何策?于宣请屏人密言;左右远窥,但见于宣叩头跪奏,可望点头应之,竟不知所献何策也。于宣出,得意之极;谓家人曰:`今年入滇功成后,国主登九五,我为首相,已亲许我矣'。此时可望欲发兵,以粮草不足,稍缓其期。适上又差白文选来议和,可望拘留之;即差通政司朱运久来议。运久大轿黄盖,径至朝门,无人臣礼;名为讲和,实暗与可望心腹文武相约,俾为内应。此时上以扶纲为东阁大学士、张佐辰为吏部尚书、龚彝为户部左侍郎、孙顺为兵部右侍郎、冷孟銋为刑部左侍郎、王应龙为工部尚书、尹三聘为通政司、杨左为詹事府詹事、张重任为大理寺寺丞、汪蛟为文选司郎中之职。其中惟龚彝奏言在云南受可望十年厚恩,辞不受;举朝大哗,诘龚彝云:`尔在本朝中戊辰进士,屡任显官至于司道。可望入滇,尔首迎降,即得高位;家世受明朝三百年之恩忍忘,而十年之恩独不忍忘也'?龚彝恬不为耻。时适雷跃龙来朝,即命入阁办事。盖跃龙在昔威宗时,曾为吏部左侍郎,仕可望为宰相;至是,仍入阁,人颇笑之。

马吉翔既为靳统武所拘禁,日夜谄媚,统武悦之。时定国最信金维新、龚铭二人,尝至统武宅议事;吉翔一见,曲意逢迎,金维新、龚铭遂信吉翔为佳士。加之统武又极称誉,兼为吉翔颂冤;吉翔乃言`前事皆他人所为,嫁祸于我。但得一见晋王,诉明心事,死不恨矣'!金、龚两人信之,归言于定国,称吉翔之才,兼辨其枉。定国犹不以为然,乃召吉翔见。吉翔一见定国,先叩首,称颂定国千古无两;`从此以后,青史流芳。吉翔今日得际此时望见颜色,死且不朽。其他是非冤苦,俱不足辨'。定国于是大悦,与吉翔握手谈心,惟恨相见之晚。而吉翔得侍左右数日,其谄谀之工无所不至;凡左右、内外,众口同声交谀吉翔。定国本目不识丁,粗戆直率;竟为吉翔所弄,堕其术中。一日,吉翔谓金维新、龚铭曰:`晋王功高,皆两公为之提挈。今晋王既进封,两公亦当不次封赏,安得仍旧职乎!若吉翎得在皇上左右,定当为两公言之'。金、龚两人大悦,于是见定国曰:`吉翔原是朝廷旧人,当仍荐补朝廷要职。彼实归诚殿下,凡事必与我辈相照应也'。定国然其言,即命金维新草疏荐吉翔入阁办事;上不得已,从之。吉翔入,内既挟定国之权以要上、又假上之宠以动晋王,而内外大权,不一月尽归吉翔;不但诸臣屏听,即上亦坐视,无可奈何矣。

丁酉春,方于宣启可望曰:`今皇上在滇,定国辅之,人心渐属于彼;臣意请国主早正大统,封拜文武世爵,则人心自定矣'。可望遂日夜谋犯阙,调练兵马。时钱邦芑为可望拘于大兴寺,心私忧之;可望兵部尚书程源与都察院郑逢元过邦芑寓,亦深以为虑。邦芑知其心事,与之计曰:`今马宝、马进忠、马维兴等三人虽隶可望麾下,皆朝廷旧勋臣,受国恩颇重;彼曾与我言及此事,彼自愤恨,欲图报朝廷而无路也。至于可望标下,惟白文选有心朝廷;我曾与之私誓,决不相负。可望率兵入滇,必用此数人为将;倘得从中用计,图可望如反掌耳。今被幽禁在此,烦二公可为致意之'!程源即商之文选,文选曰:`我矢心不负朝廷,只恐力难济事'。源曰:`马宝兄弟,有心人也'。文选然之。源又与逢元私见马宝,相约既定。至八月初一日,可望誓师发兵,以白文选为大总统、以马宝为先锋,合兵十四万入滇。十八日,兵渡盘江;滇中震动,王尚礼私约龚彝、张重任等为内应。先是,七月间,王自奇在楚雄,醉后误杀定国营将:惧定国袭之,遂引其众渡澜沧江,据永昌府,去云南二千余里。故可望入滇不相闻,遂不相应。九月初,李定国、刘文秀闻可望率十余万众至交水,列三十六营,去曲靖止三十里;相顾失色。文秀曰:`城中有王尚礼诸人为内应,迟二十日,王自奇必知消息,必引兵从永昌而下;云南腹背受敌,不战自溃矣。莫若乘此时走交趾,犹可自全'。定国曰:`交趾兵亦不少,我辈不过二、三万人,且有家口,安能往!不若由沅江、景东取土司以安身为上'。踌躇两日,终不能决。初四日,白文选率所部兵连夜拔营逃至曲靖,单马引数骑定云南。初六日进城,竟入朝,上细言兵事;定国、文秀闻之,大为惊骇,随至朝中相见。文选曰:`此时宜速出兵交战。马宝、马维兴及诸要□诸将已俱有约,稍迟则事机必露,断不可为矣'!定国尚疑文选为反间,犹豫未决;文选曰:`若再迟,则我辈死无地矣!有一字\{言任\}皇上、负国家,当死万箭之下!我当先赴阵前,汝等整兵速进'!言毕,即上马驰去;文秀遂率祁三升及贺九仪、胡一清、赵印选、吴子金、李本高等御之。十五日至交水,相去十里,列三营。初,可望见文选率所部逃去,恐人心多不服,意欲退兵;召诸将议之,诸将未敢应。马宝自思若退回贵州,则我辈之谋必泄,岂能自存乎;因挺身言曰:`文选所部不及万人,今我辈之众十倍于彼;若以文选一人为进退,我辈岂皆非人乎'!张胜亦曰:`只某一人,亦能擒定国以献;文选何足为重轻'!可望大悦曰:`诸将能如是,吾复何忧'!至十八日,可望见定国对列三营,知云南兵马尽出,城内空虚;乃召张胜曰:`尔可率领武大定、马宝选铁骑七千,连夜走小路至云南城下,暗袭之。城中有王尚礼、龚彝等为内应,尔一入城,则定国、文秀等知家口已失,不战而走矣'。张胜领命整点马骑,与马宝约傍晚起行。马宝回营,写密书,差心腹人送入定国营内曰:`张胜等领精兵七千往袭云南;云南若破,则事不可为!必须明日决战,迟则无及矣'。可望、定国约二十一日会战;十八日晚,定国见马宝〔书〕,大惊。是夜,即传各营诸将:十九日天未明拔寨而出,列阵相向。可望亦命各营会战;两阵相交,文秀骁将崇信伯李本高马蹶被杀,文秀退回、定国亦为小却。可望乘高山观战,见定国等锐气既挫,命诸营速进;定国、文秀色惧,相谓曰:`毕竟众寡不敌,不若暂退再议'!文选怒曰:`张胜已往袭云南,我兵若退,彼以精骑蹑吾后,我兵不鸟散,则蹂为肉泥矣!尚能归乎?进而死于阵,不犹愈于走而死于马足乎?况彼阵中马兴、袁韬等皆与我有约,若决志而前,必相应也'。定国、文秀未应,文选乃策马率所部五千铁骑直冲之。见马维兴列阵未动,文选飞奔而来,维兴不放一箭,开阵迎文选入;两人合兵抄出可望阵后,所向披靡,连破数营。可望在高阜望见,大惊曰:`维兴诸营俱叛矣'!诸将见之,遂无斗志。定国、文秀见文选、维兴乘胜截杀,可望旗帜渐乱;因召各营奋勇齐进,可望遂大败而逃。定国与文秀计曰:`今张胜往袭云南府、王自奇又据永昌,圣驾在云南,我当回救;汝可同文选急追可望,必擒之而后已'!于是文秀、文选率诸将追可望,定国率兵回救云南。是时张胜由小路行,道走五日,至云南城下驻扎;正欲攻城,王尚礼即披挂上城,欲为内应。黔国公沐天波探知其情,奏上急召王尚礼、龚彝、张重任入朝;独尚礼三召始至,沐天波率亲兵防守之。先是,定国自交水遣人报捷,上命将飞报大捷旗插金马碧鸡坊下,晓谕军民。及张胜兵至城外,正欲攻城,见飞报大捷旗,问居民曰:`此何处报捷'?居民曰:`李晋王在交水杀败秦王,昨晚差官来报捷耳'。张胜大惊,谓武大定曰:`我大营兵既败,李定国必截吾之后;我辈孤军,安可居此'!于是抄掠居民,拔营而去。王尚礼见张胜兵退,知其情已露,自缢而死。张胜兵回至浑水塘,正遇定国兵回,列阵死战。定国兵自交水力战后,又远行而来,疲敝之极;张胜争归路,拚命死战,定国兵几不支,将次败走。马宝见定国势危,从张胜阵后连放大炮,拥兵杀来;张胜大惊曰:`马宝亦反矣'!遂溃而走。次日,过益州;其部将总兵李承爵驻扎其地,率兵来迎。张胜大喜,方与承爵叙战败之故;忽左右数人,直前擒张胜缚之。胜骂曰:`汝为部将,何敢叛我'?承爵曰:`汝敢叛天子,吾何有干汝乎'?于是解云南,告庙献俘;与其党赵珣伏诛。

十月初一日,可望逃至贵州,命冯双礼带兵守威清要路;约曰:`若文秀□至,可速放三炮'。时文秀追至普安,尚未敢轻进;双礼欲可望速逃、劫其辎重,乃连放三炮。可望逃回贵州时,不过十五、六骑,城中并无一兵;闻双礼号炮,挈妻子连夜出城,其辎重、妇女尽为双礼部兵所劫。初十日,走至偏桥,随行止二十余人。及过镇远、平溪、沅州,各守将俱闭营不纳;至靖州,户书吴逢圣为靖州道,率所部迎之。可望曰:`一路人心俱变,惟有投大清朝可免'。于是遣杨惺先、郑国先往宝庆投降大清朝。三日后,白文选追兵至,可望乃与吴逢圣、程万里数十骑连夜奔逃。至武冈界上,总兵杨武伏兵截杀;□存妻子十余人投归大清,余众走散。可望既逃后,文秀至黔招集旧将;黔中诸文武皆曰:`犯阙之祸,起于张虎、方于宣二人'。数日后,张虎率残兵从滇逃回投文秀,文秀问曰:`皇上赐金簪,原嘱汝议和;何从有``行刺''之说'?虎不能应。文秀乃囚虎解云南,上告庙御门,献俘磔之,无不利决。时方于宣正为提□,考试沅、靖等处,所出表题有``拟秦王出师讨逆大捷''等语;及闻可望兵败,即驰书于邦芑云:`欲纠集义旅,擒可望以献功朝廷'。邦芑鄙之,答以诗曰:`修史当年笔削余,帝皇井度竟成虚;秦宫火后收图籍,犹见君家劝进书'。盖于宣为可望修史者,又尝对人言`帝星明于井度,秦王当有天下'故也。

十一月,李定国率马宝、高文贵等进兵永昌,擒王自奇诛之。可望诸营兵部将俱归诚朝廷,滇、黔之难悉平;乃下诏大赦,封白文选为巩昌王,遣召川、黔大臣程源、郑逢元、万年策、刘泌等。李定国率诸文武上疏,请表彰安龙死难十八忠臣及叙追剿可望诸文武勋劳;于是赠吴毓贞少师兼太子太师中极殿大学士吏部尚书,谥``文忠'',荫一子锦衣卫佥事世袭;赠郑允元武安侯,谥``武简'';张镌、徐极兵部右侍郎,杨钟、蔡縯、赵赓禹大理寺正卿,蒋干昌、李元开、陈麟瑞侍读学士,周允吉、朱议◆、胡士瑞、李颀副都御史,易士佳、任斗墟太常少卿,朱东旦、刘议新太仆少卿,各荫一子入监读书;内监张福禄、全为国弟侄一人锦衣卫指挥佥事。俱遣布司官谕祭,文曰:`卿等乾坤正气、社稷忠臣。早倾捧日之忱,共效旋天之力。讵意叛逆生忌,祸起萧墙;枭獍横行,顿忘君父。安龙之血,终当化碧九原;汗青之书,各自流芳千古。今日移跸滇云,鹓鹭骈列;回思卿等簪履趋跄,杳不可见!夫独何心,能不悲哉!将兹俎豆,慰彼泉台'。后吴贞毓妻裴氏、子谷戬、郑允元夫人邓氏,扶两公柩合葬于城西海源寺。时马吉翔复当国,奸党侧目,不敢通知;在廷诸公知其事者,白衣冠往送之。户郎中吴鼎吊以诗曰:`国运如丝系暴秦,须眉那得有完人!智称``武简''知名重,美谥``文忠''见道真。千古史传双烈士,一山石伴两孤臣!黄冠酾酒临风吊,愁说中兴志未伸'!御史陈起相诗曰:`烬灰冷作一瓶收,送上荒原源海头。天府星残埋三曲,辽东鹤返泣千秋!雨中昏夜催人去,夜里空山付鬼愁。眼底须眉今略尽,更将忠义向谁筹'!廷臣谓可望之不至于篡弑,皆贞毓诸公护持之力也。十一月,上乃复遣通政使尹三聘往安龙立十八忠臣之庙。

是时,周官、裴廷谟、许绍亮、金简维等交章参劾吉翔,而吉翔当权,与金维新朋比;定国听其蛊惑,渐次疏远正人,奸党仍复布列,识者已知国事之不可为矣。

\hypertarget{header-n97}{%
\subsection{卷十七·粤纪(续)}\label{header-n97}}

孙、李构隙本末

张献忠起于陕西,有养子四人。孙可旺、艾能奇、李定国、刘文秀,献忠养以为子,皆冒姓张;然稍违其意,挞之至百余。故四人虽为献忠所亲信,而两腿恒溃烂,更无完者。可旺本名旺儿,米脂人;幼无赖,乡人恶之。与母同居。受直为人赶驴,远出数日返,不见其母;问之邻人,皆云不知。可旺讼之官,官怒曰:`汝出门时,原未尝以母托邻人;今汝母自他适,邻人安所知'!因杖之。可旺愤怒无归,逃而为贼。初入贼营,为主者负锅;雪天行山路六昼夜不息,两足十指俱落,疲困不能行,遂弃所负锅。至晚,主者炊无锅,欲斩之;旁一贼力救,得免。可旺苦甚,逃出营;遇献忠,收为伴当。可旺性狡黠,犹伺献忠意。能奇、定国皆愚蠢无知,故献忠尤喜可望,抚为长子;众贼遂呼可旺为大哥。献忠既得志,以可旺为平东将军、能奇为定北将军、定国为安西将军、文秀为抚南将军;又以王尚礼为中军府都督、白文选为前军都督、王复臣为左军都督、冯双礼为右军都督、王自奇为后军都督。军中于是称可旺〔为东府、能奇〕为北府、定国为西府、文秀为南府。而彼此往来,则皆称为兄弟;属下文武,皆称师主。诸贼中,可旺稍识字;故献忠平日一切密谋,惟可旺独参之。每遇敌,可旺能率部下坚立不动,贼中呼``一堵墙''。

自献忠死于川,丁亥春,可旺、能奇、定国、文秀同王尚礼等由贵州走云南,首攻曲靖府。时隆武差都察院右佥都御史朱寿琳率总兵孔思诚、副总兵孙守约、监纪通判张京元驻札曲靖;三月贼至,寿琳同道府有司坚守,以炮石击伤贼无算。可旺乃率众力攻三日,城陷;执寿琳等。寿琳不屈,可旺劝之至三,骂愈烈,遂遇害;〔思〕诚、守约等俱降。先是,黔国公沐天波以听信家丁,刻害土司激变;沙定洲陷云南,沐天波走大理府。沙定洲据云南,请乡宦大学士王锡衮相见,王不屈。贡生唐泰为沙定洲谋主,劝定洲杀王,并杀诸乡绅;云南大乱。洱海道杨畏知集义兵讨定洲,相拒于楚雄府。及可旺破曲靖,定洲以兵死相袭;可旺一战大败之,定洲溃逃。可旺乘胜破云南,分兵袭楚雄。杨畏知战败被擒,初不屈;可旺以畏知同乡,闻其任云南甚得士民心,故欲降之以收人望,多方劝之。畏知降;可旺待之甚厚,畏知遂为之用,因与定国联姻。畏知既降,沐天波遂走永昌;刘文秀引兵追至永昌,王自奇入城,擒天波回云南。天波请降,可旺乃命天波招降各府;云南三百年知有沐国公,凡各土司闻天波归顺,无不降者。

可旺既据有云南,耻其名不雅,改名可望,因与能奇等各复原姓;可望称平东王、艾能奇称定北王、李定国称安西王、刘文秀称抚南王。是时四人并大,各领一军不相下;而艾能奇、李定国兵尤多。可望意欲并之,而兵独弱,恐不能得;先与王尚礼私议。尚礼曰:`自然应尊大哥为主;但得定北师主无异议,无不从矣'。可望因嘱尚礼往说能奇曰:`我等兵马虽多,号令不一;若不尊一人为主,恐难以约束。众议欲请公与平东议一人为主'。能奇曰:`大哥有学问,我等不及;自当尊之'。尚礼复可望,遂传令四月初一日各营兵将同赴演武场尊可望为主。及是日,李定国先到营中,□放炮,将``帅''字旗扯起;可望与能奇等后至,可望遂问曰:`我尚未至,谁升``帅''字旗'?众答曰:`西府老爷先至,众将不知,照往日例,遂将旗升起'。可望曰:`军中旧制:主将入营,方升帅旗;天下所同也。今日既以我为主,应候我入营,方升旗放炮;若西府入营,何升旗?目中明无我矣,我安能为众人主乎'!刘文秀曰:`此西府一时之误,望大哥姑容'!可望愤不已,尚礼请责旗鼓官赎罪,可望亦不允。定国曰:`我与汝兄弟耳。今日因无主,尊汝为首领,遂欲如是;异日可知矣!汝不做则已,我何必定靠你生活'。众人多方劝解可望登座发落;可望怒曰:`必欲我为主,必杖定国百棍乃可'!定国怒曰:`谁敢打我'?可望曰:`定国不受杖,则军法不能行;异日何以约束诸将'?众力劝不已,定国喧哄愈甚;可望怒,欲上马去。白文选从定国后抱持之,曰:`请老爷勉强受责,以成好事。不然,从此一决裂,则我辈必至各散,皆为人所乘矣'!于是王尚礼、冯双礼等同将定国按倒于地,特杖□之;杖至五十,定国不得已呼曰:`我今服矣'!众乃为求免,遂舍之。是日,可望遂为诸将主,于是军中无敢不服者。是可望之能用其众在此,而定国之嫌隙亦由是成矣。是晚公会既散,可望私入定国室中,再三慰之曰:`不如是,号令不行,众军皆叛;我等何能行'!从此四人虽并肩仍称兄弟,每公事相会四人并坐于上;然各营诸将赏罚,则一禀于可望。

戊子秋,可望得钱邦芑招降书,欲要封王爵;朝议未决。己丑春,广西总兵陈邦傅畏李赤心、高必正势盛,恐为所并,欲借援可望;乃假铸``秦王之宝'',命其私人胡执恭往云南封可望为秦王,能奇、定国、文秀三人为国公。定国等心疑其伪,与能奇、文秀议不受;乃可望欲借王号以压三人,劝三人同受。能奇曰:`我等自为王耳,何必封'!定国曰:`我等无尺寸之功,何敢负朝廷之封'!可望不悦,相持不决。越月余,能奇病死,可望乃独受秦王之封;而定国、文秀卒不受,仍各称帅主。可望既假称王,乃使人讽定国、文秀,欲其拜见叩贺;定国不从,文秀劝曰:`以弟拜兄,亦无不可'。于是定国勉强下拜。此后公会,定国、文秀俱左右列坐;然定国终愤愤,可望亦心衔之。后朝廷知可望受胡执恭伪封,众议□不决。督师堵胤锡请之于上,封为平辽王;差佥都御史赵昱至滇封之,并封定国为康国公、文秀为泰国公。定国知此封出自朝廷,与文秀议欲受封;可望已称秦王,不欲受``二字''王,乃咈然谓定国曰:`汝前不受封,今何为而受乎'?定国乃不敢受。

及庚寅秋,可望出黔,命定国守云南;定国终日操演兵马、制造盔甲,一年练就精兵三十万人。至壬辰三月,乃致书可望,欲出楚立功以报朝廷;可望不能止,乃听之出。四月,至贵州,可望命冯双礼等领兵二万人同行。五月,由镇远下偏桥,一战复沅州;复大战,遂复靖州。六月,至全州;大清定南王孔有德兵出接战,败绩。有德等严守关,以精骑三千大战;定国直前杀数人,纵兵围杀,有德大惊,急传令百姓守城。次日,定国同冯双礼兵至城下。有德乘城,见定国兵马强盛,知不敌;乃回宅运火药于室内,嘱家人曰:`事急,则举火'!次日大雨,城破;有德自回家,杀其爱妾数人自缢,命家人纵火,阖门焚死。余一子七岁,定国收养之;并擒陈邦傅及子曾禹解至贵州,诛之。广西既破,金帛山积;定国贪而愚,凡部下所掳之物,定国必兼取之。冯双礼以是不服,密启可望云:`定国专擅之甚,后恐难制'!八月,定国复衡州;凡永、彬一带望风而降。定国兵至江西吉安,凡招抚所到,定国委选州县官。可望封定国为西宁王、冯双礼兴国侯,差杨惺先往封;至衡州,李定国曰:`封赏出自天子,今以王封王,可乎'?遂不受封。可望虑定国功大权重难制,楚、粤人心归之;因为书召之,不至。十月,可望出兵至沅江,命张虎督兵复辰州;连书催定国至靖州相会,意欲图之。定国心腹人龚铭至沅州见可望,探知其意;密书报定国,令勿来、来必不免。癸巳正月,定国行至武冈州;见书叹曰:`本欲共图恢复;今忌刻如此,安能成大功乎'!因率所部走广西。

四月,可望与大清兵战于两路口,大败;走回贵州。八月,闻李定国驻兵柳州,命冯双礼统兵三万往袭之;定国闻可望兵至,烧粮而走。双礼谓定国怯,率兵追之;定国回兵有击,双礼大败而回。时上在安龙,愤可望陵逼,遣武选司员外林青阳、翰林院孔目周官封定国亲王,命将兵至安龙护驾;后可望知之,甲午三月忿杀宰相吴贞毓等十八人。

至乙未冬,定国败于粤东,回札南宁;可望又遣张明志、关有才引兵潜赴南宁袭定国,复为所败。丙申三月,定国乘胜入安龙迎驾,径赴云南,与可望议和。丁酉八月,可望以白文选为总统、马宝为先锋,统兵十五万入云南,札于交水。文选曾与马宝密商为定国内应,至是文选竟率兵定国合,还击之;可望大败,走回黔。左右皆叛,文秀率兵急追之;可望恐不免,遂入楚降大清。其部下兵将,皆为定国所有。

观此,则知构隙本末,曲在可望、不在定国明矣!

续孙可望踞云、贵事

崇祯甲申张献忠入蜀,僭号成都,残忍不可尽述。岁丁亥,大清肃王统兵至蜀,杀献忠于西充县之凤凰山。其党孙可望、李定国、刘文秀、艾能奇、白文选、冯双礼、王尚礼、王复臣等领溃众夺重庆江,杀隆武所封平寇伯曾英,遂由遵义取贵州。值云南土司沙定洲与妻范氏叛踞省城,黔国公沐天波走楚雄,定洲围其城;可望等诡称援师,由贵州兼程于三月二十八日屠曲靖,定洲解楚雄围,悉众走阿迷州,遇可望等于蛇花口战败,定洲集溃众遁守佴革龙。可望取云南,李定国推可望为平东王,其相雄长如故也。

旋以兵袭天波,有佴海道杨畏知统义旅与可望等战于禄丰县之启明桥,畏知被执随营;天波走永昌。可望至大理,天波自永昌遣其子为质,可望许之;阴令心腹混于沐众至澜沧江,夺铁索桥。比沐众到永昌,可望兵亦到;天波仓卒不备,被执回滇。戊子,可望、李定国、刘文秀领兵围沙定洲于佴革龙,擒定洲、范氏剥皮游示。天波恨既雪,听可望指示,分檄号召各土司出兵认饷。遂巢穴云南,营土木;铸造印敕,设六部、九卿、科道。昆明乡原任御史任僎倡称国主,率众推戴;可望令僎兼礼、兵二部尚书。时惟李定国多所扞格,可望密与文秀商擒定国于教场,责百棍示威:定国之嫌始此。

可望自揣:昔皆比肩其事,思所以压服其心;杨畏知、袭彝同赴广西浔州府永历处请封,有庆国公陈邦傅矫诏遣标官武康伯胡执恭由间道齎敕印往封可望为平辽王,改名朝宗。执恭至滇迎可望意,又私改敕印封可望``秦王'',以悦其心。铸``兴朝通宝''。

庚寅,可望败匡国公皮熊于贵筑,杀忠国公王祥于绥阳(皆隆武所封者);兼定北将军艾能奇病故,可望悉收其部曲,声势益张。永历内阁严起恒、总督杨鼎和及科道官追论陈邦傅矫诏、胡执恭假敕印之罪,可望令裈督贺九仪等往南宁护驾,遂盗杀严起恒等以泄``追论''之忿。杨畏知既脱虎口,不欲回黔,永历留为相;可望怒,差指挥郑国于永历处拏畏知回黔杀之,令贺九仪、张胜、张明志移驻永历于安龙所。改所为府,令范应旭知府事;凡永历及随侍文武支粮,提塘章应科与旭造册,开``皇帝一员、皇后一口'',余可知矣。又令李定国攻广,以冯双礼与陈国能随之。揣可望之心,以定国胜则可以崇功、死则借以除患、败则可以加罪;不意突破广西,子女、玉帛定国无不私厚。双礼、国能归报,可望即调撤定国,定国疑中谗,不赴;封西宁主,定国亦不受。可望以冯双礼为兴国侯,率兵往擒;双礼败归,可望恐迫则生变,仍善养定国家口于云南。壬辰三月,可望以成都、叙府、重庆各要地皆吴三桂同定西将军开服,令刘文秀领兵复四川。三桂同定西将军撤兵回保宁,文秀追至保宁;一战而十余万众立膏锋刃,获都督王复臣杀之。刘文秀止单身走,可望责令投闲:文秀之嫌始此。

李定国避粤攻新会,为大清击败,仅存兵二千。至丙申春,定国将奔回安龙;可望恐定国以永历为奇货,亟遣心腹叶应桢随白文选同往安龙探听定国动静,即逼永历移黔。永历合宫惨哭,白文选亦泣下;遂以定国无他志报可望。及定国见永历,即挟之行。

可望谋夺永历,复遣文秀至曲靖府;文选意在永历,与定国一同护行。刘文秀与可望□都督王尚礼、王自奇、贺九仪等守滇,文秀闻定国奉永历回滇,阳与尚礼等密议勒兵守城,自以数骑会定国,云`我辈将以秦王为董卓;但恐诛卓,又有曹操'!定国指天设誓,同文秀迎归云南,即倡言秦王若尊永历,我辈当尊秦王。未几,封李定国为晋王、刘文秀为蜀王;艾能奇之子承业为镇国将军,管延安王事。以定国办事金维新为行在吏部侍郎兼都察院、龚铭为行在兵部侍郎、白文选为巩国公、王尚礼为保国公、王自奇为夔国公、贺九仪为保康侯。马吉翔工弥缝,仍以文安侯入阁办事。适白文选往黔,令可望赴滇保驾,将钱粮归之永历,兵马交定国、文秀经营川、广;可望以妻子尚在云南,忿衷不露。永历令可望□□□护卫东昌侯张虎送可望妻子赴黔,又赐虎金簪一杖,令从中开导;虎既回黔,详言永历赐簪密令行刺,以媚激可望。时可望妻子已至黔中,无复顾忌,遂大言永历负义,定国、文秀谋反;追文选巩国公敕印:文选之嫌始此。

可望决意攻滇,有马维兴(?)白文选密议乘机反正,言于可望曰:`白文选恩受有年,昨在滇受封,屡辞不允,实出无奈;今重加爵赏,用为总统,必恩感图报'。可望即以冯双礼守贵州;封白文选为征逆招讨郡王,总统兵马。定国、文秀方揣势迟疑,忽文选来归,即请封白文选为巩昌王,遣内阁文安侯马吉翔视师,同定国、文秀、文选等于丁酉九月十四日至三岔,距交水二十里下营。可望因总统之变,欲引兵回黔;马维兴、马宝等绐言:`逃文选,不过一人;有他不多,无他不少。尽这兵马,做个明白'。可望大喜,密议安定侯马宝、临潼侯武大定、汉川侯张胜等率劲旅四千,由寻甸寻道攻袭滇省,可望仍于交水索战,令首尾不能兼顾。马宝、马维兴于十八日夜各差心腹,将可望密议报知定国等,且催速战。定国等即于十九日交锋,直扑马维兴;维兴内应,余悉瓦解。可望逃回贵州,即遣其大理卿杨惺先奔赴经略洪承畴前军降大清。李定国回滇省,于浑水塘收马宝;擒张胜,剥其皮。文选、文秀追可望尚远,冯双礼言追兵已到,促可望携家口前奔,自请断后掩其玉帛,追兵方至。可望自智自尊,一旦被愚、被卖,殊可捧腹。至长沙,承畴疏闻,大清封可望为义王。

李定国以黔、蜀、辰、沅镇将皆可望所设,悉调赴云南。核功罪,封冯双礼为庆阳王、马进忠为汉阳王、马维兴为叙国公、贺九仪为广国公、马宝为淮国公,余爵不赘;可望部下德安侯狄三品、岐山侯王会、荆江王张光翠等降级有差。凡永历左右,皆定国心腹;正睥眤尊大,而大清兵三路入矣。

吴三桂兵取云南

大清封吴三桂为平西王,居秦之汉中府。顺治十五年(戊戌),三桂偕定西将军固山额真侯墨勒根由四川一路,令荆州之宁南靖寇大将军宗室洛托由湖广一路、征南将军固山额真卓布泰由广西一路,定于二月二十五日三路出师,先取贵州;命安远靖寇大将军信郡王铎尼自都门统领大兵入黔,分三路进取云南,换宁南靖寇大将军回荆州弹压。三桂由沔县至朝天驿,顺流击楫。三月初四日,抵蜀之保宁府;具舟舰,载军糈。预揣蜀之重庆府水陆交冲,请以副将程廷俊为重夔总兵,设水陆官兵五千。三月初七日,起营过南部、西充,犹见数家烟火;自顺庆而前,大路枳棘丛生、箐林密布,虽乡导莫知所从。惟描踪伐木,伐一程木、进一程兵。三月十四日,至蜀之合州,俨同鬼域。合州属重庆,永历重庆总兵杜子香以轻舟哨至合州江口。此合州江北,则自阳平合翟汝至合州南,有绵州一江横出于合江南,水势汹涌。三桂偕定西将军挥甲兵跨马渡江,杜子香弃重庆,分水陆奔逃;三桂偕定西将军由铜梁、壁山、来凤、白石进发。铜梁、壁山二县属重庆;凡驻营帐房左右,满地头颅,皆张献忠及摇黄十三家所戕杀。间有庐舍,入视,则残书、坏券与糜烂之躯具在。四月初三日,三桂军至重庆,为明玉珍负固之地,铁壁金城,足称天险。蜀、楚界中如房、竹、归、□、大昌、大宁有塔天保、郝摇旗、李来亨、袁宗第、党守素,贺州、施州卫有王光兴,长寿、奇县有刘体仁、谭诣、谭宏、谭文,达州有杨秉胤、徐邦定等连兵分守;三桂俱不之问。以永宁总兵严自明合镇兵马留重庆,与新设重夔总兵程廷俊合防,固根本;调陕西炮火,裕城守。十三日,搭浮桥,渡黄葛江;溽暑薰蒸,心迷目眩。翌日,渡綦江,历东溪、松坎、新站、夜郎。其中如滴溜、三坡、红关、石壶关,上摩九天、下坠重渊,人皆履涩、马皆钉掌,节节陡险,一夫可守;晋王李定国、蜀王刘文秀预遣将军刘正国率兵众据险设伏。二十五日,三桂偕定西将军抵三坡,刘正国由水西逃奔云南。自铜梓至四渡站,明将军郭李受、刘董才、王明池、朱守合、王刘仓、总兵王友臣等以家口并五千兵众降大清;三桂偕定西将军收服遵义。五月初三日,自遵义由新站、乌江、养龙、息烽、礼佐会宁南靖寇大将军于贵州。十一日,回息烽,袭明将军杨武大营于开州之倒流水。回遵义,有水西宣慰使安坤、酉阳宣慰使冉奇镳、蔺州宣慰使奢保受等降大清。兴宁伯王兴受李定国指授,回绥阳;子友臣首先归降,遂亲诣军前缴敕印,三桂与以盔甲、名马、金币。七月初二日,新津侯谭宏等率众攻童庆,败回。

广西一路,征南将军卓布泰与提督线国安抵独山州时,大清使日传上谕:`克取贵州,如云南机有可乘,大兵马匹行得,即乘势进取,不必候旨;如兵马疲弱,侯安远靖寇大将军信郡王到日,三路进取云南。宁南靖寇大将军驻贵州,侯开服云南回荆州'。三路承旨,屯□养锐。三桂始终以重庆为忧,调四川巡抚高民赡于重庆弹压,又调建昌总兵王明德赴重庆协防;檄永宁总兵严自明俟王明德至重庆,即领所部官兵赴遵义:厚重、遵两镇之防,固川、黔一线之脉。值安远靖寇大将军信郡王铎尼统大兵入黔境,先约三桂会商;三桂自遵义六百里至平越府之杨老堡,同信郡王等与经略洪承畴会订师期。晋王李定国受黄钺,同王公侯伯将军冯双礼等悉众扼盘江河,踞鸡公背,谋攻贵州,相违咫尺;巩昌王白文选同窦名望等四万余众守七星关,嗣抵生界扎营,离遵义一日之程示攻遵势,牵制应援,以助定国复黔之举。三桂兼程回遵义。前此数月,三桂驻遵义、征南将军卓布泰驻独山州、信郡王在武陵,惟宁南靖寇大将军驻贵州;当大众未合之际,定国观望逡巡。及杨老堡订期进兵,定国始秉钺而出,事机已失矣。十一月初十日,三桂统藩下四镇及援剿左路镇总兵沈应时、右路镇总兵马宁等自遵义出师,白文选于二十日五更自生界遁回七星关守险。此关四山壁立,水势涌汹;山上树木参天,名曰天生桥,其实未尝有桥。三桂先在遵义厚养乡导,朝夕垂问,默识于心;十二月初二日,于水西苗猓地方安营,次晨忽由天生桥进乌撒军民府,扼七里关大路。文选侦三桂从别路越险进兵,弃七星关,走可渡桥;即焚桥走沾益州,思奔云南顾家口。李定国见信郡王中路兵前进,即退回盘江河;又报征南将军广西一路甚急,自领部众堵御。定国连败于安龙之罗炎河、凉水井,撤寨踉跄奔回;奉永历并宫眷大营,于十五日弃云南走永昌府。白文选中道飞奔大营,定国留文选驻守玉龙关;盖永昌之要道也。三桂至乌撒剿白文选余众,收降之;设官安抚毕,涉可渡河、出交水大道,晤信郡王征南将军于板桥。己亥正月初三日,三桂等收服云南,明公侯伯将军镇将胡一青等,士司总兵龙世荣等降。

是时大兵云集,镇静为难;益以逃降之众、逃窜之兵掠入口资粮,无所不至,滇民水深火热。定国犹在永昌,三路议信郡王驻镇省城,以多罗贝勒尚善领中路兵马,计定师期。三桂于初八日移营罗次县。十二月初二日,谭宏等悉众再犯;三桂设备严□□□□□自相猜忌,宏、诣杀谭文降大清;封谭宏为慕义侯、谭诣为向化侯。又闻冯双礼、狄三品等与白文选下自乌撒,追散之。将军王安等持白文选金印、金章过金沙江,逃往四川建昌卫。十五日,三桂发檄招抚,密授狄三品方略,并谕川南诸镇将归诚。二月初五日,三桂自罗次出师,征南将军多罗贝勒同于一日自云南出师。初九日,三桂出镇南州,征南将军合兵杀明总兵王国勋于普湖,又追败白文选等于玉龙关之西,获巩昌王金印。追至沧澜江,溃兵烧毁铁索桥;大兵扎筏过江,马玉同杨筠白上流觅渡,一叶扁舟,几罹不测。十五日,李定国自永昌奉永历并宫眷大营奔腾越州。三桂扎筏渡江,江不甚宽,水势甚恶。其地每自清明至霜降,有青草瘴;凡往来,虽士人亦恶之。过江二十里,有磨盘山;所入之路,坎陡箐深,屈曲仅容单马。定国度大兵累胜穷追,必不戒;设栅数重其间:窦名望初伏、高文贵二伏、王玺三伏,每伏兵二千,约俟大兵至山巅,号炮起,首尾横突截攻,必无一骑返。我军筏渡澜沧江、潞江逐北数百里,无一夫守拒;谓定国窜远,队伍散乱,上山已万有□人。而降官卢桂生来泄其计,则前驱已入二伏;诸帅急退,传令舍骑而步,以炮发其伏。伏兵死林箐中者三之一,伏起而鏖斗死者三之一。定国坐山巅闻信炮失序,惊骇;忽飞炮落其前,击土满面,乃奔窦名望、王玺皆战死。穷追至腾越州西百二十里,为云南迤西尽界,即三宣六慰、缅甸。三十日,振旅班师。闰三月十一日,三桂抵姚安府,永历东阁大学士张佐辰、户部尚书孙顺、侍郎万年策、都察院钱邦芑、少卿刘泌、兵科胡显等一百五十九人先后降。德安侯狄三品等受三桂密指,以庆阳王冯双礼并勘定大将军金印及金册赴军前。二十三日,三桂等旋师昆明,景东土知府陶斗、蒙化土知府左星海、发江土知府木懿等暨各土州县降。延长伯朱养恩、总兵龙海阳、吴宗秀自四川嘉定走雪山至云南,巩昌王部下将军王安等自川建昌卫至云南,缴白文选荡平大将军金印。心膂藩臣金章、将军郝承裔、广平伯陈建杀咸宁侯高承恩自雅州至云南,宁国侯王友进、总兵杜子香、陈希贤等、乌撒土知府安重金、东川土知府禄万兆、乌蒙土知府禄世孝、镇雄土知府陇宏勋等俱自川来降。四月二十四日,三桂以冯双礼请旨,待以不死,解京安置;续封狄三品为杼城侯,余各差等授级。其为李定国率引出边者,亦先后归降;如大学士扶纲、兵部侍郎尹三品、翰林刘\{氵茞\}、贵州布政宋企鋘等、淮国公马宝、叙国公马维兴、武靖侯王国玺、怀仁侯吴子金、宜川伯高启隆、公安伯李如碧、阳武伯廖鱼、都督王朝钦。总兵单泰征缴故汉阳王马进忠敕印,将军杨武缴永历母皇太后金宝一颗。维时滇民离散,斗米三两;发帑金十五万两赈给。

其边外情形,缅甸留永历与宫眷及黔国公沐天波等于境内,布兵众拒晋王李定国、白文选于境外。定国无永历可恃,无根本可凭;暂驻遐荒,用永历敕印,将各土司概加勋爵,令其内应。元江土知府那嵩受总督衔,为定国密传敕印;各土司有听命者、有两可观望者、有不从而自出首于大清帅者。维时三桂奉旨驻镇云南,又总统满、汉大兵。明延长伯朱养恩、将军高应凤、总兵许名臣、土司总兵龙赞阳等前皆归降,至是复与元江合,许内应定国。九月二十一日,三桂自云南出师至石屏州,土司总兵龙荣率赘婿黔国公之子沐忠显赴军前。那嵩等负固元江,十月初六日三桂率满、汉兵围其城;十一月初六日破元江,那嵩合室自焚。十二月初六日,信郡王遵旨赴京。二十三日,三桂还军云南。

十七年(庚子),永历在缅甸;朝廷度外置之,议撤兵节饷。而三桂擅□权,必欲俘获永历为功;遂有``渠魁不翦,三患二难''之疏。乃命内大臣爱星阿为定西将军,赴滇会剿;颁敕印于各土司,并购缅擒献。十八年九月,满、汉土司及降卒七万五千并炊伋余丁共十万,由大理、腾越出边。三桂、爱星阿将五万人出南甸、陇川、猛卯;分二万余出姚关,总兵马宁、王辅臣、马宝将之。十一月,会师木邦。闻白文选方扼锡箔江,遣前锋疾驰三百余里及江滨,白文选毁桥走茶山;令马宁等分道追文选,俾不得窥木邦后路。而大军筏渡趋缅,以降人为乡导。十二月,抵兰鸠江,缅人遂执永历及其母太后等并从官家口献军前。文选为马宁等追及,亦以兵万余、象马数千降;留提督张勇以万人守普洱,备定国。

未几,定国死于景线,云南悉平。

\hypertarget{header-n3068}{%
\subsection{卷十八·余纪}\label{header-n3068}}

投诚安插

广东道御史范疏曰:`从来治理莫大于疆圉,绸缪莫切于寇孽。国家振旅以来,廓清四海;么么窃发,歼灭何难。但朝廷好生为心,凡有伪众投诚,即准归顺,且予官爵;所以开自新之路,而施覆育之仁也。今投诚之倾心报效者固多,然亦有一、二鹰眼未化、狼性犹存。大则如四川之郝承裔、山东之于七等,次则如江、浙之湖寇张守智等,已投复叛。揆厥所由,盖由来投诚之后,仍安插本地;其伙众团聚一隅,逞臂一呼,党羽立集:此叛谋之所易起也。臣以为今后投诚,如功多首事者不妨优其爵赏,或宠入亲班、或另推别地。即在外安插者,亦必分其部曲散置诸营,另拨兵丁畀之统领;官仍不失为官、兵仍不失为兵,稍分其势以防不测。是亦弭乱之先机,不可不为区画者也'。

楚、蜀会剿

大清康熙元年七月,兵部疏曰:`台臣顾条陈:夔门、郧、襄,界处腹心,与边隅不同。袁宗第、贺珍等诸有名巨寇,各拥众屯于大昌王山寨中;湖广荆州之界如兴山县水筒、梁材等处,无非盗贼盘踞。长江阻塞,商贾弗通。郧、襄之贼强盛者,如郝永忠约有数万。以臣愚见,会剿诚为不易之定算;仍敕川、陕、湖广三省招抚,如负固不降,发大兵剿灭可也'。

楚师全胜

二年四月,湖广总督张疏曰:`臣看得西山诸逆,逋诛有年;叠蒙皇上招抚,而始终怙恶不悛,致烦天讨。幸仗皇上威灵,楚兵自出师以来,于本年正月初五日李家店一战,即获全胜。今臣由省赴夷陵巡视,于正月十六、十七两日途次接到塘报,言巨寇李来亨、马腾霄、党守素自败回老巢之后,即以多贼把守两关。一名双龙观、一名三白亚,最为险恶;以为天堑不拔之处长坪地方屯扎。于初九、初十即迎锋交战,又大获全胜;当阵生擒及杀死伪总兵、副将、参游、都守与贼兵甚多,已将两关攻夺。惟逆贼李来亨败遁逃回七连坪老巢,现今分兵追剿;直奏凯荡平,在指顾间矣'。

楚师堵剿

钦命挂印提督湖广全省驻札武昌总兵官左都督董学礼,二年八月疏曰:`李逆自被困之后,百计千谋图所以豕突者而不可得;乃乘连雨重雾之夜,率逆党千人分头扒屋,希图偷越塘汛。孰知我兵防站戒严,当贼众连上悬岩陡涧之时,已为各处堵截;号炮一响,各汛官兵齐出堵御,臣即发各镇营官兵四路策剿。惟时亲领官兵迎头堵剿者,郧阳镇臣穆生辉也;亲领官兵出奇截杀者,辰常镇臣高守贵也;发兵防汛又夹剿者,襄阳镇臣于大海也;分发官兵出汛协同援剿者,夷陵镇兵金万镒也。至于副将、参游、都守及千把等官并臣标随征各兵,无不奋勇用命,觅路追杀。除滚崖跌死逆贼不问外,总计阵擒逆贼共七十三名、阵斩七百六十五名;又斩伪总兵三名首级及阵擒逆贼将王福,俱系贼中枭雄。臣因新奉特旨,不敢令官兵深入穷追'。

穆生辉,字荣之;天城卫人,总兵都督同知。高守贵,字健侯;陕西延安人,总兵都督同知。于大海,字昆山;江南项城人,总兵都督。金万镒,字宝山;辽东广宁人,总兵都督佥事。

房、保荡平

二年八月,陕西提督王一正疏曰:`郝逆率众暗逃,臣兵追杀,得奏肤功。又思逆贼披靡之余,势必潜伏深山密箐之内,万难遍行搜戮;惟招安一着,庶可得净根株。随遣各将给免死牌入山招抚,并(?)六月内投出伪副将夏启明前去招抚。又活擒六百二十四名,又招获伪总副、参游、都守等兵六十二名、贼兵七百五十名、伪官贼丁家口妇女幼小共五百五十一名,听候发落。又招抚贼民男妇幼小家口共二千二百一十一名,俱发房县署事知县金殿臣收领安插,以为归正之良民。查郝逆盘踞竹溪、竹山、房县、保康四县地方久历年所,侵占我土地、蹂躏我民人。臣恨不立枭其首,以泄十余年之公愤。据报带几百残兵,窜入刘二虎营内;不过釜中之鱼耳。然此在山游魂,俱已剿抚尽净;从贼人民,俱已报安信妥。而房陵□□之妖氛,尽扫无遗矣。前追官兵越古坪而南时,值霪雨连绵,经月不止。且各兵粮已断,脚腿被雨浸淖,竟无完肤;而染患时疫、痢泻者,十之四、五。困苦乞哀之声,接踵而至。臣思我兵追渡古坪,即为刘逆塘拨地方。元凶报已至刘逆营盘,追固无及;而缺粮病卒,又未便驱之深入。遂檄令各该官兵将贼内三座庵、邓川峪、白玉坪、红花垛、紫竹、上项、东河、不挡沟等处贼巢及各山隘口闸路挡木栅栏一齐焚毁,以杜日后啸聚之源。臣又另拨官兵五千,遣参将周元、游击李登相、张四直、守备韩国祚、韩宏胤等统领前去上项、白玉坪、邓川峪及阎王寺一带,不特侦探贼情,暂将前遣官兵撤回大营休养'。

周元,字孟祥;江西广信人,为灵州参将。韩宏胤,北直人。张四直,字次峰,顺天昌平人。

擒获伪王

三年正月,平西亲王吴三桂疏曰:`据三路总兵王会、张鹏程、赵良栋塘报前事到臣。据此,该臣看得妖逆阿仲鼓惑粤蛮僭称年号,聚□□为害封疆;土目贺云等党附阿仲藉妖谋反,煽动土司:虽衅生于粤西,而毒切于黔、滇;普坪之杀戮最惨,安龙之城郭尤危。臣遵上命,发兵分道进征。中路总统总兵王会奋骑长关、破额老塞而入,左路总统总兵张鹏程等取花韦塞、破阿积而入,屡战屡胜。既而会师直捣陇纳,探知遮别为阿仲巢穴;王会、张鹏程不及等待右路,遂即一面分布围攻,于十二月初四日申时攻破,将伪丞相等官以及蛮兵尽行剿杀。右路总统赵良栋三日之内,斩夺三关。至初四日,到遮别会合。查初四日破巢之后,本日抵晚广西思南副将赵直植、广东提标后营游击朱尚文,其领兵六百名亦抵遮别;泗城土官岑维禄,亦带土兵千余续后方至。先是,阿仲原在遮别北门;滇、黔兵马将到,岑维禄初三日三更时分差人暗接阿仲而去,不解何故?见遮别已破,维禄始将阿仲交广西副将赵真植、游击朱尚文于初五日解交总统总兵王会等。虽幸渠魁未脱,就中关节可疑;但既行献出,姑免深究。再查此举,作孽虽由阿仲,而佐妖为祟及号召土蛮侵掠地方、行军指授者,则土目贺良臣、贺云、贺富、贺坤新等辈也。诸贺非阿仲不足动众,阿仲非诸贺不足成谋。今阿仲虽已成擒,贺逆尚稽授首。察贺云等各据一寨,料彼匿山寨之中。臣再檄总兵王会、张鹏程、赵良栋等攻剿各逆所居山寨,另报'。

赵真植,辽东人;副总兵。

擒获郝逆

三年正月,四川总督李疏曰:`渠魁刘二虎、郝摇旗、袁宗第自巫山奔窜以后,犹抗抚负固。蒙敕发西安将军传夸蟾、副都统杜敏领大兵于十一月二十九日从水路进抵巫山,因栈道崎岖,马足困惫,尚未到汛。臣恐有误进剿机宜,会同将军副都统并提诸臣将绿旗马疋那偕满兵骑征,鼓励汉兵荷戈步走;自十二月十八日,臣同将军傅、副都统杜、提督臣郑率满、汉官兵,至二十二日直逼刘逆营巢陈家坡。我兵奋勇争先,夺其要隘、捣其巢穴;贼力不支,奔入后巢天地寨。臣等率兵追剿计老爬空隘口,逼临逆穴;声威震天,一时逆营各伪镇将砍寨投诚。刘逆势穷命蹙,自伏冥诛;妻妾生女,登时缢死。郝、袁二渠犹思走脱,冀延旦夕之生。赖副都督两臣分发总兵等率满、汉兵夤夜力追,于二十六日赶至黄家坪;二渠领贼拒敌,我兵奋勇,阵擒逆渠郝、袁并伪部院洪育勋,而伪朱安王随于寨中捉获,党太监尔(即异□)自缢。是役也,元凶俱无漏网。计进兵未及旬日,而数万巨寇扫荡殆尽'。

傅夸蟾,满洲人;提督陕西,统满兵驻西安。郑蛟麟,字西云;辽东人,提督四川全省总兵都督同知。

洪承畴行状摘略

洪承畴,字彦演,号亨九;福建泉州南安县人。万历四十三年举于乡,丙辰登进士。初授刑部主事,升两浙提学道;历甬东兵备道、三秦参储、巡抚榆林,旋制三边。崇祯十五年(壬午),降大清;太宗文皇帝不令服官,凡大祭祀、宴会,必令亲随。赐房屋、庄田、男女有差,赐上御服、膳无虚日。甲申,□军入北京,入内院办事;赐第、庄田、人口。十月,以〔登〕极,荫一子入监。顺治二年,世祖章皇帝命往江宁绥辑,赐朝帽、玉带、貂蟒、披领、大蟒貂裘外褂及鞋袜、王\{马酉\}、骆驼并蒙古人口、帐房、凉棚、银碗等物;随行员役,令部各给缎袍、靴帽并马六十余匹。承畴驻南京三年,心计日数,手答口授。自办事至夜分不辍,心血为耗。目睛渐花。顺治五年,还朝报命,特召赐宴,赐袍靴等物,又太皇太后赐宴;仍入内院办事。是年,遇加四祖尊号覃恩,封赠三代。己丑,典春闱。顺治八年,世祖亲政,命掌御史大夫事。赐貂皮披领,又赐团蟒、日月肩貂裘;旨云:`你系有功之臣,此袍应赐服用'。八月,以上``昭圣慈寿恭简皇太后''尊号,荫一子入监。九年,命掌大学士印,与修``太宗文皇帝实录'',赐蟒服等。又荐人才十四人。

顺治十年五月,时湖南煽动,滇、黔犹阻;进承畴太保,经略五省。赐内厩马、玲珑鞍辔、袍帽及嵌宝石带、撤袋弓矢、顺刀等物。于陛辞前五日,赐宴,随行官一百二十员,俱引见赐蟒。御五凤楼,目送久之。敕巡抚、提督、总兵以下,悉听节制;文官五品以至武官副将以下有违命者,听以军法从事。一应抚剿事宜,不从中制;事后具疏报闻。文武各官在京、在外,随时择用。所属各省升转补调,一面奏闻;应用钱粮即与解给,吏、兵二部不得掣肘,户部不得稽迟。功成之日,优加爵赏;事定之后,即命还朝。承畴既受命,未出都门,先疏选将召兵,而西北之名将劲旅群集。六月,出都。十一月,抵武昌受事;有``多得贤良,安民劝农,以守为战;简拔将领,练兵制胜,以战为守;联络土司,使不为贼用,以树我之藩篱;计离贼党,使自为解散,以溃彼之腹心''一疏:`念三湖荒残,兵多粮少,发金买牛数十头,分布屯粮,为持久之计。以昆、沙为湖南北总汇之区,身自弹压。初至长沙,城中、城外皆瓦砾荆榛;乃抚降众、招流移,修城浚池、建仓筑坝,而城郭改观、兵民辏聚矣。思削平湖南,必先安湖北。遴举郧抚,设郧镇以堵西山刘、郝诸逆;立武昌城守,设洞庭水师,以壮全楚腹心之备;分一提、三镇驻扎武陵,以固辰、沅之门户;立一抚、两镇驻扎宝庆,以遏武、靖之狡窥。他如永州、祁阳、衡州、湘潭、益阳、常德、彝陵、荆州以迨郧、襄,节节设镇,五千里之长边首动尾顾、此呼彼应,而楚中之气脉贯若连星矣'。又初奉命之始,粤西各入版图,实存一府、三县、一州,余尽贼踞。承畴谓`欲取滇、黔,必先复广西,开入滇之路;乃设抚、提两大镇驻扎桂林、再设一大镇驻扎苍梧,粤西之险要\{山厄\}如长城矣'。十二年五月,常德告警;偕长沙大兵同至衡州堵防宝庆,以分贼势;而荆、澧大兵直趋常德,乘夜出奇,大破贼众。捷闻,赐盔甲、弓矢、刀带、裘帽等物。子士铭,登进士。十四年三月,以太祖高皇帝、太宗文皇帝并配覃恩,封赠三代;赐羊酒,荫一子入监。

承畴在楚五年,休养兵马,欲待敌敝。适有疾,予告;未行,而孙可望降,封义王。承畴上方略,言滇、黔可取状。十二月,命调度五省事,三道会师。道里险隘,糗粮为艰;承畴廉知积谷处,辄因粮焉。其无积谷处,则发价土司,倍其直;携负鳞次而至,悉偿之:不惟省转输之苦,而招徕之计亦寓焉。十五年,三路兵进滇;上增发广西兵,从间道出。其后李定国逆战大败,三路之师齐克云南。上发帑金三十万赡兵赈民;复命三大臣到滇时,察民间夫妇不相保者,俾复完聚,凡数万人。承畴初偕大将军入黔,舍骑徒步;每身居前锋,而先士卒。及至滇,调和兵民,尤殚苦心。

大事既竣,精力渐惫,双睛俱损;遂乞休,命还京调理。舟次淮安,闻世祖章皇帝宾天,号恸欲绝;疾趋抵京,不入私舍,哭于景山。既而陈疏请老,特予武功世职。承畴以覃恩封赠三代者三、荫子入监者五、予三等阿达哈哈番准世袭者四;有请必允、有奏必行:可谓异遇矣。又爱惜三军,食在兵后、苦在兵前;故人乐为之用。自天启丁卯与父宦别,癸未卒。至顺治丁亥,迎母至江宁,养四载;母以老思故乡,乃归,数载卒。教子士铭,几屏之间每录先圣贤格言示诫。每遇令节,家庭设香案,朝服望阙叩祝。乃于康熙四年(乙巳)二月十七日卒。卒之次日,遣多里机昂邦问丧,赐茶;谕士铭曰:`朝廷闻你父病故,不胜痛惜'!遍赐合家及有服亲友并仆人等,尤称异数云。

\end{document}
