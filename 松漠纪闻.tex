\PassOptionsToPackage{unicode=true}{hyperref} % options for packages loaded elsewhere
\PassOptionsToPackage{hyphens}{url}
%
\documentclass[]{article}
\usepackage{lmodern}
\usepackage{amssymb,amsmath}
\usepackage{ifxetex,ifluatex}
\usepackage{fixltx2e} % provides \textsubscript
\ifnum 0\ifxetex 1\fi\ifluatex 1\fi=0 % if pdftex
  \usepackage[T1]{fontenc}
  \usepackage[utf8]{inputenc}
  \usepackage{textcomp} % provides euro and other symbols
\else % if luatex or xelatex
  \usepackage{unicode-math}
  \defaultfontfeatures{Ligatures=TeX,Scale=MatchLowercase}
\fi
% use upquote if available, for straight quotes in verbatim environments
\IfFileExists{upquote.sty}{\usepackage{upquote}}{}
% use microtype if available
\IfFileExists{microtype.sty}{%
\usepackage[]{microtype}
\UseMicrotypeSet[protrusion]{basicmath} % disable protrusion for tt fonts
}{}
\IfFileExists{parskip.sty}{%
\usepackage{parskip}
}{% else
\setlength{\parindent}{0pt}
\setlength{\parskip}{6pt plus 2pt minus 1pt}
}
\usepackage{hyperref}
\hypersetup{
            pdfborder={0 0 0},
            breaklinks=true}
\urlstyle{same}  % don't use monospace font for urls
\setlength{\emergencystretch}{3em}  % prevent overfull lines
\providecommand{\tightlist}{%
  \setlength{\itemsep}{0pt}\setlength{\parskip}{0pt}}
\setcounter{secnumdepth}{0}
% Redefines (sub)paragraphs to behave more like sections
\ifx\paragraph\undefined\else
\let\oldparagraph\paragraph
\renewcommand{\paragraph}[1]{\oldparagraph{#1}\mbox{}}
\fi
\ifx\subparagraph\undefined\else
\let\oldsubparagraph\subparagraph
\renewcommand{\subparagraph}[1]{\oldsubparagraph{#1}\mbox{}}
\fi

% set default figure placement to htbp
\makeatletter
\def\fps@figure{htbp}
\makeatother


\date{}

\begin{document}

\hypertarget{header-n0}{%
\section{松漠纪闻}\label{header-n0}}

\begin{center}\rule{0.5\linewidth}{\linethickness}\end{center}

\tableofcontents

\begin{center}\rule{0.5\linewidth}{\linethickness}\end{center}

\hypertarget{header-n10}{%
\subsection{松漠纪闻}\label{header-n10}}

女真即古肃慎国也,东汉谓之挹娄,元魏谓之勿吉,隋唐谓之靺鞨。开皇中,遣使贡献,文帝因宴劳之。使者及其徒起舞于前,曲折皆为战斗之状。上谓侍臣曰:``天地间乃有此物,常作用兵意。''其属分六部,有黑水部,即今之女真。其水掬之则色微黑,契丹目为混同江。其江甚深,狭处可六七十步,阔处百余步。唐太宗征高丽,靺鞨佐之,战甚力。驻跸之败,高延寿、高惠真以众及靺鞨兵十余万来降,太宗悉纵之,独坑靺鞨三千人。开元中,其酋来朝,拜为勃利州刺史,遂置黑水府,以部长为都督、刺史,朝廷为置长史监之。赐府都督姓李氏,讫唐世朝献不绝。五代时始称女真。后唐明宗时,尝寇登州渤海,击走之。其后避契丹讳,更为女直,俗讹为女质。居混同江之南者谓之熟女真,以其服属契丹也,江之北为生女真,亦臣于契丹。后有酋豪受其宣命为首领者,号``太师''。契丹自宾州混同江北八十余里建寨以守,予尝自宾州涉江过其寨,守御已废,所存者数十家耳。

女真酋长乃新罗人,号完颜氏。完颜犹汉言``王''也。女真以其练事,后随以首领让之。兄弟三人,一为熟女真酋长,号万户。其一适他国。完颜年六十余,女真妻之以女亦六十余。生二子,其长即胡来也。自此传三人,至杨哥太师无子,以其侄阿骨打之弟谥曰文烈者为子。其后杨哥生子闼辣,乃令文烈归宗。

金主九代祖名龛福,追谥景元皇帝,号始祖,配曰明懿皇后。八代祖名讹鲁,追谥德皇帝,配曰思皇后。七代祖名佯海,追谥安皇帝,配曰节皇后。六代祖名随阔,追谥定昭皇帝,号献祖,配曰恭靖皇后。五代祖孛堇名实鲁,追谥成襄皇帝,号昭祖,配曰威顺皇后。高祖太师名胡来,追谥惠桓皇帝,号景祖,配曰昭肃皇后。曾祖太师名核里颇,追谥圣肃皇帝,号世祖,配曰翼简皇后。曾叔祖太师名蒲刺束,追谥穆宪皇帝,号肃宗,配曰静宣皇后。曾季祖太师名杨哥,追谥孝平皇帝,号穆宗,配曰贞惠皇后。伯祖太师名吴刺束,追谥恭简皇帝,号康宗,配曰敬僖皇后。祖名旻,世祖第二子,咸雍四年岁在戊申生,即阿骨打也。灭契丹,谥大圣武元皇帝,号太祖。同母弟二人,长曰吴乞买,次曰撒也。阿骨打卒,吴乞买立,更名晟,谥文烈皇帝,号太宗,配曰明德皇后。今主名亶,阿骨打之孙,绳果之子。绳果追谥景宣皇帝,亶之配曰屠姑坦氏。

阿骨打八子,正室生绳果,于次为第五,又生第七子,乃燕京留守易王之父。正室卒,其继室立,亦生二子,长曰二太子,为东元帅,封许王,南归至燕而卒。次生第六子曰蒲路虎,为兖王、太傅、领尚书省事。长子固碖力本切,侧室所生,为太师,凉国王,领尚书省事。第三曰三太子,为左元帅,与四太子同母。四太子即兀术,为越王,行台尚书令。第八子曰邢王,为燕京留守;打球坠马死。自固碖以下皆为奴婢。绳果死,其妻为固碖所收,故今主养于固碖家。及吴乞买卒,其子宋国王与固碖、粘罕争立,以今主为嫡,遂立之。

吴乞买,乙卯年卒。长子曰宗磐,为宋王、太傅,领尚书省事,与滕王、虞王皆为悟室所诛。次曰贤,为沂王,燕京留守。次曰滕王、虞王。袁王撒也,称揞邬感切板揞板,彼云大也孛极烈,吴乞买时为储君,尝谋尽诛南人。

闼辣封鲁王,为都元帅,后被诛。其子太拽马亦被囚,因赦得出。庶子乌拽马名勖,字勉道,今为平章。

粘罕者,吴乞买三从兄弟,名宗干,小名乌家奴,本曰粘汉,言其貌类汉儿也,其父即阿卢里移赍。粘罕为西元帅,后虽贵,亦袭父官,称曰阿卢里移赍孛极烈都元帅。``孛极烈'',彼云``大官人''也。其庶弟名宗宪,字吉甫,好读书,甚贤。

悟室者,女真人。``悟''作``邬''音,或云悟失,名希尹,封陈王,为左相。诛宋、兖,滕、虞凡七十二王,后为兀术族诛。

回鹘自唐末浸微,本朝盛时,有入居秦川为熟户者。女真破陕,悉徙之燕山、甘、凉、瓜、沙。旧皆有族帐,后悉羁縻于西夏,唯居四郡外地者,颇自为国,有君长。其人卷发深目,眉脩而浓,自眼睫而下多虬髯。士多瑟瑟珠玉,帛有兜罗绵、毛鬤、狨锦、注丝、熟绫、斜褐。药有腽肭脐、硇砂。香有乳香、安息、笃耨。善造宾铁刀剑、乌金银器。多为商贾于燕,载以橐驼过夏地,夏人率十而指一,必得其最上品者,贾人苦之。后以物美恶杂贮毛连中,毛连以羊毛缉之,单其中,两头为袋,以毛绳或线封之。有甚粗者,有间以杂色毛者则轻细。然所征亦不赀。其来浸熟,始厚赂税吏,密识其中下品,俾指之。尤能别珍宝,蕃、汉为市者,非其人为侩则不能售价。奉释氏最甚,共为一堂,塑佛像其中,每斋必刲羊,或酒酣以指染血涂佛口,或捧其足而鸣之,谓为亲敬。诵经则衣袈裟,作西竺语,燕人或俾之祈祷,多验。妇人类男人,白晢,着青衣,如中国道服。然以薄青纱幂首而见其面。其居秦川时,女未嫁者先与汉人通,有生数子年近三十始能配其种类。媒妁来议者,父母则曰,吾女尝与某人某人昵,以多为胜,风俗皆然。其在燕者皆久居业成,能以金相瑟瑟为首饰,如钗头形而曲一二寸,如古之笄状。又善结金线相瑟瑟为珥及巾环,织熟锦、熟绫、注丝、线罗等物。又以五色线织成袍,名曰``克丝'',甚华丽。又善捻金线别作一等,背织花树,用粉缴,经岁则不佳,唯以打换达靼。辛酉岁,金国肆眚,皆许西归,多留不反。今亦有目微深而髯不虬者,盖与汉儿通而生也。

嗢熟者,国最小,不知其始所居,后为契丹徙置黄龙府南百余里,曰宾州。州近混同江,即古之粟末河黑水也。部落杂处,以其族类之长为千户统之。契丹、女真贵游子弟及富家儿月夕被酒,则相率携尊,驰马戏饮。其地妇女闻其至,多聚观之。闲令侍坐,与之酒则饮,亦有起舞歌讴以侑觞者,邂逅相契,调谑往反,即载以归。不为所顾者,至追逐马足不远数里。其携去者父母皆不问,留数岁,有子,始具茶食、酒数车归宁,谓之拜门,因执子贌之礼。其俗谓男女自媒,胜于纳币而昏者。饮食皆以木器,好置蛊,他人欲其不验者,乃三弹指于器上,则其毒自解,亦间有遇毒而毙者。族多李姓,予顷与其千户李靖相知。靖二子亦习进士举,其侄女嫁为悟室子妇。靖之妹曰金哥,为金主之伯固碖侧室。其嫡无子,而金哥所生今年约二十余,颇好延接儒士,亦读儒书,以光禄大夫为吏部尚书。其父死,托宇文虚中、高士谈、赵伯璘为志,高、宇以赵贫,命赵为之,而二人书、篆其文、额,所濡甚厚。曾在燕识之,亦学弈、象戏、点茶。靖以光禄知同州,冒墨有素,今亡矣。其论议亦可听,衣制皆如汉儿。

渤海国,去燕京、女真所都皆千五百里,以石累城足,东并海。其王旧以大为姓,右姓曰高、张、杨、窦、乌、李,不过数种。部曲、奴婢无姓者皆从其主。妇人皆悍妒,大氐与他姓相结为十姊妹,迭稽察其夫,不容侧室及他游,闻则必谋置毒死其所爱。一夫有所犯而妻不之觉者,九人则群聚而诟之。争以忌嫉相夸,故契丹、女真诸国皆有女倡,而其良人皆有小妇、侍婢,唯渤海无之。男子多智谋,骁勇出他国右,至有``三人渤海当一虎''之语。契丹阿保机灭其王大諲撰,徙其各帐千余户于燕,给以田畴,捐其赋入,往来贸易,关市皆不征,有战则用为前驱。天祚之乱,其聚族立姓大者于旧国为王,金人讨之,军未至,其贵族高氏弃家来降,言其虚实,城后陷。契丹所迁民益蕃,至五千余户,胜兵可三万。金人虑其难制,频年转戍山东,每徙不过数百家,至辛酉岁尽驱以行。其人大多富室,安居逾二百年,往往为围池,植牡丹多至三二百本,有数十干丛生者,皆燕地所无,才以十数千或五千贱贸而去。其居故地者令归契丹,旧为东京,置留守,有苏、扶等州。苏与中国登州青州相直,每大风顺,隐隐闻鸡犬声。阿保机长子东丹王赞华封于此,谓之人皇。王不得立,鞅鞅,尝赋诗曰:``小山压大山,大山全无力,羞见当乡人,从此投外国。''遂自苏乘筏浮海归唐明宗。善画马,好经籍,犹以筏载行。其国初仿唐置官司,国少浮图氏,有赵崇德者为燕都运,未六十余,休致为僧,自为大院,请燕竹林寺慧日师住持,约供众僧三年费。竹林乃四明人,赵与予相识颇久。

古肃慎城,四面约五里余,遗堞尚在,在渤海国都外三十里,亦以石累城脚。

黄头女真者皆山居,号合苏馆女真。合苏馆,河西亦有之,有八馆在黄河东,今皆属金人,与金粟城、五花城隔河相近。三城八馆旧属契丹,今属夏人。金人约以兵取关中,以三城八馆报之,后背约,再取八馆,而三城在河西,屡争不得。其一城忘其名。其人戆朴勇騺,不能别死生,金人每出战,皆被以重札,令前驱,谓之硬军。后役之益苛,廪给既少,遇卤掠所得复夺之,不胜忿,天会十一年遂叛。兴师讨之,但守遏山下,不敢登其巢穴。经二年,出斗而败,复降,疑即黄头室韦也。金国谓之黄头生女真,髭发皆黄,目精多绿亦黄而白多,因避契丹讳,遂称黄头女真。

盲骨子,《契丹事迹》谓之朦骨国,即《唐书》所谓蒙兀部。大辽道宗朝,有汉人讲《论语》至``北辰居所而众星拱之'',道宗曰:``吾闻北极之下为中国,此岂其地邪?''至``夷狄之有君'',疾读不敢讲,则又曰:``上世獯鬻猃狁荡无礼法,故谓之夷,吾修文物,彬彬不异中华,何嫌之有?''卒令讲之。道宗末年,阿骨打来朝,以悟室从。与辽贵人双陆,贵人投琼不胜,妄行马。阿骨打愤甚,拔小佩刀欲剚之,悟室急以手握鞘,阿骨打止得其柄,杙其胸,不死。道宗怒,侍臣以其强悍,咸劝诛之。道宗曰:``吾方示信以待远人,不可杀。''或以王衍纵石勒、张守珪赦安禄山终致后害为言,亦不听,卒归之。至叛辽,用悟室为谋主。阿骨打且死,属其子固碖善待之。

大辽盛时,银牌天使至女真,每夕必欲荐枕者。其国旧轮中、下户作止宿处,以未出适女待之。后求海东青使者络绎,恃大国使命,惟择美好妇人,不问其有夫及阀阅高者,女真浸忿,遂叛。初,女真有戎器而无甲,辽之近亲有以众叛,间入其境上,为女真一酋说而擒之,得甲首五百。女真赏其酋为阿卢甲移赍。彼云第三个官人,亦呼为相公。既起师,才有千骑,用其五百甲攻破宁江州。辽众五万御之,不胜,复倍遣之,亦折北,遂益至二十万。女真以众寡不敌,谋降。大酋粘罕、悟室、娄宿等曰:``我杀辽人已多,降必见剿,不若以死拒之。''时胜兵至三千,既连败辽师,器甲益备,与战,复克。天祚乃发蕃、汉五十万亲征。大将余都姑谋废之,立其庶长子赵王,谋泄,以前军十万降。辽军大震。天祚怒国人叛己,命汉儿遇契丹则杀之。初,辽制:契丹人杀汉儿者皆不加刑。至是摅其宿愤,见者必死,国中骇乱,皆莫为用。女真乘胜入黄龙府五十余州,浸逼中京。中京,古白霫城。天祚惧,遣使立阿骨打为国王。阿骨打留之,遣人邀请十事,欲册帝,为兄弟国及尚主。使数往反,天祚不得已,欲帝之,而他请益坚。天祚怒曰:``小夷乃欲偶吾女邪?''囚其使不报。已而中京被围,逃至上京。过燕,遂投西夏。夏人虽舅甥国,畏女真之强,不果纳。初,大观中,本朝遣林摅使辽,辽人命习仪,摅恶其屑屑,以``蕃狗''诋伴使。天祚曰:``大宋兄弟之邦,臣吾臣也,今辱吾左右,与辱我同。''欲致之死,在廷恐兆衅,皆泣谏,止枚半百而释之。时天祚穷,将来归,以是故恐不加礼,乃走小勃律,复不纳,乃夜回,欲之云中。未明,遇谍者言娄宿军且至,天祚大惊。时从骑尚千余,有精金铸佛,长丈有六尺者,他宝货称是,皆委之而遁。值天微雪,车马皆有辙迹,为敌所及。先遣近贵谕降,未复。娄宿下马,跽于天祚前曰:``奴婢不佞,乃以介胄犯皇帝天威,死有余罪。''因捧觞而进,遂俘以还。封海滨王,处之东海上。其初走河西也,国人立其季父于燕,俄死,以其妻代。后与郭药师来降,所谓萧太后者。

宁江州去冷山百七十里,地苦寒,多草木,如桃李之类,皆成园。至八月则倒置地中,封土数尺,覆其枝干。季春出之,厚培其根,否则冻死。每春水始泮,辽王必至其地,凿冰钓鱼,放弋为乐。女真率来献方物,若貂鼠之属,各以所产量轻重而打博,谓之``打女真''。后多强取,女真始怨。暨阿骨打起兵,首破此州,驯至亡国。辽亡,大实林牙亦降。大实,小名。林牙,犹翰林学士。虏俗大概以小名居官上。后与粘罕双陆争道,粘罕心欲杀之而口不言。大实惧,及既归帐,即弃其妻携五子宵遁。诘旦,粘罕怪其日高而不来,使召之。其妻曰:``昨夕以酒忤大人,大音柁。畏罪而窜。''询其所之,不以告。粘罕大怒,以配部落之最贱者,妻不肯屈。强之,极口嫚骂,遂射杀之。大实深入沙子,立天祚之子梁王为帝而相之。女真遣故辽将余都姑帅兵经略屯田于合董城。城去上京三千里。大实游骑数千,出入军前。余都姑遣使打话,遂退。沙子者,盖不毛之地,皆平沙广漠,风起扬尘至不能辨色,或平地顷刻高数丈。绝无水泉,人多渴死。大实之走,凡三昼夜始得度,故女真不敢穷追。辽御马数十万牧于碛外,女真以绝远未之取,皆为大实所得。今梁王、大实皆亡,余党犹居其地。

合董之役,令山西、河北运粮给军。予过河阴,县令以病解,独簿出迎,以线系槐枝垂绿袍上。命之坐,恳辞。叩其故,以实言曰:``县馈饷失期,令被挞柳条百,惭不敢出。某亦罹此罚,痛楚特甚,故不可坐。创未愈,惧为腋气所侵,故带槐以辟之。''

余都姑之降,金人以为西军大监军。久不迁,常鞅鞅。其军合董也,失其金牌。金人疑其与林牙暗合,遂质其妻子。余都姑有叛心。明年九月,约燕京统军反。统军之兵皆契丹人。余都姑谋诛西军之在云中者,尽约云中、河东、河北、燕京郡守之契丹汉儿,令诛女真之在官在军者。天德知军伪许之,遣其妻来告。时悟室为西监军,自云中来燕,微闻其事而未信。与通事汉儿那也回,行数百里,那也见二骑驰甚遽,问之曰:``曾见监军否?''以不识对。问为谁,曰:``余都姑下人。''那也追及悟室曰:``适两契丹云`余都姑下人',既在西京,何故不识监军?北人称云中为西京。恐有奸谋。''遂回马追获之,搜其靴中,得余都姑书曰:``事已泄,宜便下手。''复驰告悟室,即回燕,统军来谒,缚而诛之。又二日,至云中。余都姑微觉,父子以游猎为名,遁入夏国。夏人问:``有兵几何?''云:``亲兵三二百。''遂不纳。投达靼,达靼先受悟室之命,其首领诈出迎,具食帐中,潜以兵围之。达靼善射,无衣甲,余都姑出敌不胜,父子皆死。凡预谋者悉诛,契丹之黠、汉儿之有声者皆不免。

金国旧俗多指腹为昏姻,既长,虽贵贱殊隔亦不可渝。婿纳币皆先期拜门,戚属偕行,以酒馔往。少者十余车,多至十倍。饮客佳酒则以金银杯贮之,其次以瓦杯,列于前以百数。宾退则分饷焉,男女异行而坐,先以乌金银杯酌饮,贫者以木。酒三行,进大软脂、小软脂、如中国寒具。蜜袴,以松实、胡桃肉渍蜜和糯粉为之,形或方或圆或为柿蒂花,大略类浙中宝阶袴。人一盘,曰``茶食''。宴罢,富者瀹建茗,留上客数人啜之,或以粗者煎乳酪。妇家无大小皆坐炕上,婿党罗拜其下,谓之``男下女''。礼毕,婿牵马百匹,少者十匹,陈其前。妇翁选子姓之别马者视之,``塞痕''则留,好也。``辣辣''则退。不好也。留者不过什二三,或皆不中选,虽婿所乘亦以充数,大氐以留马少为耻。女家亦视其数而厚薄之,一马则报衣一袭。婿皆亲迎。既成昏,留妇氏执仆隶役,虽行酒进食,皆躬亲之。三年,然后以妇归。妇氏用奴婢数十户,奴曰``亚海'',婢曰``亚海轸''。牛马十数群,每群九牸一牡,以资遣之。夫谓妻为``萨那罕'',妻谓夫为``爱根''。契丹男女拜皆同,其一足跪,一足着地,以手动为节,数止于三。彼言``捏骨地''者,即跪也。

女真旧绝小,正朔所不及。其民皆不知纪年,问之,则曰:``我见草青几度矣。''盖以草一青为一岁也。自兴兵以后,浸染华风。酋长生朝皆自择佳辰,粘罕以正旦,悟室以元夕,乌拽马以上巳。其他如重午、七夕、重九、中秋、中下元、四月八日皆然。亦有用十一月旦者,谓之``周正''。金主生于七月七日,以国忌用次日。今朝廷遣贺使以正月至彼,盖循契丹故事,不欲使人两至也。

金国治盗甚严,每捕获,论罪外,皆七倍责偿。唯正月十六日则纵偷一日以为戏。妻女、宝货、车马为人所窃,皆不加刑。是日,人皆严备,遇偷至,则笑遣之。既无所获,虽畚微物亦携去。妇人至显入人家,伺主者出接客,则纵其婢妾盗饮器。他日知其主名,或偷者自言,大则具茶食以赎,谓羊、酒、肴馔之类。次则携壶,小亦打袴取之。亦有先与室女私约,至期而窃去者,女愿留则听之。自契丹以来皆然,今燕亦如此。

女真旧不知岁月,如灯夕,皆不晓。己酉岁,有中华僧被掠至其阙,遇上元,以长竿引灯球,表而出之以为戏。女真主吴乞买见之,大骇,问左右曰:``得非星邪?''左右以实对。时有南人谋变,事泄而诛。故乞买疑之曰:``是人欲啸聚为乱,克日时立此以为信耳。''命杀之。后数年至燕颇识之,至今遂盛。

胡俗奉佛尤谨,帝、后见像设皆梵拜。公卿诣寺,则僧坐上坐。燕京兰若相望,大者三十有六,然皆建院。自南僧至,始立四禅,曰``太平''、``招提''、``竹林''、``瑞像''。贵游之家多为僧,衣盂衣钵也甚厚。延寿院主有质坊二十八所。僧职有正、副判录,或呼``司空'',辽代僧有累官至检校司空者,故名称尚存。出则乘马佩印,街司、五伯各二人前导。凡僧事无所不统,有罪者则挞之,其徒以为荣。出家者无买牒之费。金主以生子肆赦,令燕、云、汴三台普度,凡有师者皆落发。奴婢欲脱隶役者,才以数千属请即得之,得度者亡虑三十万。旧俗奸者不禁,近法益严,立赏三百千,它人得以告捕。尝有家室则许之归俗,通平民者杖背流递,僧尼自相通及犯品官家者皆死。

蒲路虎性爱民,所居官必复租薄征,得蕃、汉间心,但时有酒过。后除东京留守,治渤海城。来令止饮。行未抵治所,有一僧以㮦柃瘿盂遮道而献,㮦柃,木名,有文缕可爱,多用为碗。曰:``可以酌酒。''蒲路虎曰:``皇帝临遣时宣戒我勿得饮,尔何人,乃欲以此器导我邪?''顾左右令洼勃辣骇,彼云敲杀也。即引去。行刑者哀其亡辜,击其脑不力,欲令宵遁而以死告。未毕,复呼使前,僧被血淋漓。蒲路虎曰:``所以献我者意安在?''对曰:``大王仁慈正直,百姓喜幸,故敢奉此为寿,无它志也。''蒲路虎意解,欲释之,询其乡,以渤海对。蒲路虎笑曰:``汝闻我来,用此相鹘突耳,岂可赦也!''卒杀之。又于道遇僧尼五辈共辇而载,召而责之曰:``汝曹群游已冒法,而乃敢显行吾前邪!''皆射杀之。

金国之法,夷人官汉地者皆置通事。即译语官也,或以有官人为之。上下重轻皆出其手,得以舞文招贿,三二年皆致富,民俗苦之。有银珠哥大王者,银珠者,行第六十也。以战多贵显,而不熟民事。尝留守燕京,有民数十家负富僧金六七万缗,不肯偿,僧诵言欲申诉。逋者大恐,相率赂通事,祈缓之。通事曰:``汝辈所负不赀,今虽稍迁延,终不能免,苟能厚谢我,为汝致其死。''皆欣然许诺。僧既陈牒,跪听命。通事潜易它纸,译言曰:``久旱不雨,僧欲焚身动天以苏百姓。''银珠笑,即书牒尾,称``塞痕''者再。庭下已有牵拢官二十辈驱之出。僧莫测所以,扣之,则曰:``塞痕,好也,状行矣。''须臾出郛,则逋者已先期积薪,拥僧于上,四面举火。号呼称冤,不能脱,竟以焚死。

胡俗旧无仪法,君民同川而浴,肩相摩于道。民虽杀鸡,亦召其君同食,炙股烹䔕,音蒲,肉也。以余肉和藄菜捣臼中,糜烂而进,率以为常。吴乞买称帝亦循故态,今主方革之。

金国新制,大氐依仿中朝法律。至皇统三年颁行其法。有创立者率皆自便,如殴妻至死,非用器物者不加刑,以其侧室多,恐正室妒忌。汉儿妇莫不唾骂,以为古无此法,曾臧获不若也。

北人重赦,无郊霈。予衔命十五年,才见两赦:一为余都姑叛,一为皇子生。

盲骨子,其人长七八尺,捕生麋鹿食之。金人尝获数辈至燕。其目能视数十里,秋豪皆见。盖不食烟火,故眼明。与金人隔一江,常渡江之南为寇,御之则返,无如之何。

金国天会十四年四月,中京小雨,大雷震,群犬数十争赴土河而死,所可救者才二三尔。

\hypertarget{header-n17}{%
\subsection{松漠纪闻续}\label{header-n17}}

冷山去燕山三千里,去金国所都二百余里,皆不毛之地。乙卯岁,有二龙,不辨名色,身高丈余,相去数步而死。冷气腥焰袭人,不可近。一已无角,如截去。一额有窍,大若当三钱,如斧凿痕。悟室欲遣人截其角,或以为不祥,乃止。

戊午夏,熙州野外渭水有龙见三日。初于水面见苍龙一条,良久即没。次日,见金龙以爪托一婴儿,儿虽为龙所戏弄,略无惧色。三日金龙如故,见一帝者乘白马,红衫玉带,如少年中官状,马前有六蟾蜍,凡三时方没。郡人竞往观之,相去甚近而无风涛之害。熙州尝以图示刘豫,刘不悦。赵伯璘曾见之。

是年五月,汴都大康县一夕大雷雨,下冰龟亘数十里,龟大小不等,首足卦文皆具。

阿保机居西楼,宿毡帐中。晨起,见黑龙长十余丈,蜿蜒其上。引弓射之,即腾空夭矫而逝,坠于黄龙府之西,相去已千五百里,才长数尺。其骸尚在金国内库。悟室长子源尝见之,尾鬣支体皆全,双角已为人所截。与予所藏董羽画出水龙绝相似,盖其背上鬣不作鱼鬣也。

悟室第三子挞挞,劲勇有智,力兼百人,悟室常与之谋国。蒲路虎之死,挞挞承诏召入,自后执其手而杀之。为明威将军。正月十六挟奴仆入寡婶家烝焉。悟室在阙下,虏都也。其长子以告,命械击于家。悟室至,问其故。曰:``放偷敢尔。''悟室命缚,杖其背百余,释之,体无伤。虏法,缚者必死,挞挞始谓必杖,闻缚而惊,遂失心,归室不能坐,呼曰:``我将去。''人问之,曰:``适蒲路虎去。''后旬日死。悟室哭之恸,曰:``折我左手。''是年九月,悟室亦坐诛。

己未年五月,客星守鲁。悟室占之,太史曰:``不在我分野,外方小灾无伤。''至七月,鲁、兖、宋、滕、虞诸王同日诛。庚申年,星守陈。太史以告宇文,宇文语悟室,悟室时为陈王。悟室不以为怪。至九月而诛。虏亦应天道如此。

金人科举,先于诸州分县赴试。诗赋者兼论策作一日,经义者兼论策作三日,号为``乡试'',悉以本县令为试官。预试之士,唯杂犯者黜。榜首曰``乡元'',亦曰``解元''。次年春,分三路类试,自河以北至女真皆就燕,关西及河东就云中,河以南就汴,谓之``府试''。试诗赋、论时务策。经义,则试五道、三策、一论、一律义。凡二人取一,榜首曰``府元''。至秋,尽集诸路举人于燕,名曰``会试''。凡六人取一。榜首曰``来头'',亦曰``状元''。分三甲,曰上甲、中甲、下甲。来头补承德郎,视中朝之承议。上甲皆赐绯,七年即至奉直大夫,谓之``正郎''。第二、第三人八年或九年。中甲十二年,下甲十三年,不以所居官高卑,皆迁大夫。中、下甲服绿,例赐银带。府试差官取旨,尚书省降札。知举一人,同知二人,又有弥封、誊录、监门之类。试闱用四柱,揭彩其上,目曰``至公楼''。主文登之,以观试。或有私者,停官不叙,仍决沙袋。亲戚不回避。尤重书法,凡作字,有点画偏旁微误者,皆曰``杂犯''。先是考校毕,知举即唱名。近岁,上、中、下甲杂取十名,纳之国中,下翰林院重考,实欲私取权贵也。考校时,不合格者日榜其名,试院欲开,余人方知中选。后又置御试,已会试中选者皆当至其国都,不复试文,只以会试榜殿廷唱第而已。士人颇以为苦,多不愿往,则就燕径官之,御试之制遂绝。又有明经、明法、童子科,然不擢用,止于簿尉。明经至于为直省官,事宰执,持笔研。童子科止有赵宪甫位至三品。

省部有令史,以进士及第者为之。又有译史,或以练事,或以关节。凡递来或除州太守,告令史、译史送之,大州三数百千,帅府千缗。若兀术诸贵人除授,则令宰执子弟送之,获数万缗。

北方苦寒,故多衣皮,虽得一鼠,亦褫皮藏去。妇人以羔皮帽为饰,至值十数千,敌三大羊之价。不贵貂鼠,以其见日及火则剥落无色也。

初,汉儿至曲阜,方发宣圣陵,粘罕闻之,问高庆绪渤海人曰:``孔子何人?''对曰:``古之大圣人。''曰:``大圣人墓岂可发?''皆杀之,故阙里得全。

燕京茶肆设双陆局,或五或六,多至十。博者蹴局,如南人茶肆中置棋具也。

女真多白芍药花,皆野生,绝无红者。好事之家采其芽为菜,以面煎之,凡待宾、斋素则用。其味脆美,可以久留。无生姜,至燕方有之,每两价至千二百。金人珍甚,不肯妄设。遇大宾至,缕切数丝置楪中,以为异品,不以杂之饮食中也。

西瓜形如匾蒲而圆,色极青翠,经岁则变黄。其瓞类甜瓜,味甘脆,中有汁,尤冷。《五代史。四夷附录》云:``以牛粪覆棚种之。''予携以归,今禁圃乡囿皆有。亦可留数月,但不能经岁,仍不变黄色。鄱阳有久苦目疾者,曝干服之而愈,盖其性冷故也。

长白山在冷山东南千余里,盖白衣观音所居。其山禽兽皆白,人不敢入,恐秽其间,以致蛇虺之害。黑水发源于此,旧云粟末河。契丹德光破晋,改为混同江。其俗刳木为舟,长可八尺,形如梭,曰``梭船'',上施一桨,止以捕鱼。至渡车,则方舟或三舟。后悟室得南人,始造船,如中国运粮者,多自国都往五国城载鱼。

西楼有蒲,濒水丛生,一干,叶如柳,长不盈寻丈,用以作箭,不矫揉而坚。左氏所谓``董泽之蒲''是也。

关西羊出同州沙苑,大角虬上盘至耳,最佳者为卧沙细肋。北羊皆长面多髯,有角者百无二三,大仅如指长,不过四寸。皆目为``白羊'',其实亦多浑黑。亦有肋细如箸者,味极珍,性畏怯,不抵触,不越沟堑。善牧者每群必置羖䍽羊数头,羖䍽音古力,北人讹呼``羖''为``骨''。仗其勇狠,行必居前,遇水则先涉,群羊皆随其后,以羖䍽发风,故不食。生达靼者大如驴,尾巨而厚,类扇,自脊至尾或重五斤,皆膋脂,以为假熊白,食饼饵。诸国人以它物易之。羊顺风而行,每大风起,至举群万计皆失亡,牧者驰马寻逐,有至数百里外方得者。三月、八月两翦毛。当翦时,如欲落絮。不翦,则为草绊落。可捻为线。春毛不直钱,为毡则蠹。唯秋毛最佳,皮皆用为裘。凡宰羊,但食其肉。贵人享重客,间兼皮以进,必指而夸曰:``此潜羊也。''

回鹘豆高二尺许,直干有叶,无旁枝。角长二寸,每角止两豆,一根才六七角,色黄,味如栗。

渤海螃蟹红色,大如碗,螯巨而厚,其跪如中国蟹螯。石举、鮀鱼之属皆有之。

自上京至燕二千七百五十里。上京即西楼也。三十里至会宁头铺,四十五里至第二铺,三十五里至阿萨铺,四十里至来流河,四十里至报打孛堇铺,七十里至宾州。渡混同江七十里至北易州,五十里至济州东铺,二十里至济州。四十里至胜州铺,五十里至小寺铺,五十里至威州。四十里至信州北,五十里至木阿铺,五十里至没瓦铺,五十里至奚营西,四十五里至杨相店,四十五里至夹道店,五十里至安州南铺,四十里至宿州北铺,四十里至咸州南铺,四十里至铜州南铺,四十里至银州南铺,五十里至兴州。四十里至蒲河,四十里至沈州,六十里至广州。七十里至大口,六十里至梁渔务,三十五里至兔儿埚,五十里至沙河,五十里至显州,五十里至军官寨,四十里至惕隐寨,四十里至茂州,四十里至新城,四十里至麻吉步落,四十里至胡家务,四十里至童家庄,四十里至桃花岛,四十里至杨家馆,五十里至隰州,四十里至石家店,四十里至来州,四十里至南新寨,四十里至千州,四十里至润州,三十里至旧榆关,三十里至新安,四十里至双望店,四十里至平州,四十里至赤峰口,四十里至七个岭,四十里至榛子店,四十里至永济务,四十里至沙流河,四十里至玉田县,四十里至罗山铺,三十里至蓟州,三十里至邦军店,三十五里至下店,四十里至三河县,三十里至潞县,三十里至交亭,三十里至燕。自燕至东京一千三百十五里,自东京至泗州一千三十四里。自云中至燕山数百里皆下坡,其地形极高,去天甚近。

虏之待中朝使者、使副,日给细酒二十量罐,羊肉八斤,果子钱五百,杂使钱五百,白面三斤,油半斤,醋二斤,盐半斤,粉一斤,细白米三升,面酱半斤,大柴三束。上节细酒六量罐,羊肉五斤,面三斤,杂使钱二百,白米二斤,中节常供酒五量罐,羊肉三斤,面二斤,杂使钱一百,白米一升半。下节常供酒三量罐,羊肉二斤,面一斤,杂使钱一百,白米一升半。

天眷二年,奏《请定官制》札子:``窃以设官分职、创制立法者,乃帝王之能事而不可阙者也。在昔致治之主,靡不皆然。及世之衰也,侵冒放纷,官无常守,事与言戾,实由名丧,至于不可复振。逮圣人之作也,刬弊救失,乘时变通,致治之具,然后焕然一新,`九变复贯,知言之选',其此之谓矣。太祖皇帝圣武经略,文物度数,曾不遑暇。太宗皇帝嗣位之十二载也,威德畅洽,万里同风,聪明自民,不凝于物。始下明诏,建官正名,欲垂范于将来,以为民极。圣谟弘远,可举而行,克成厥终,正在今日。伏惟皇帝陛下,天性孝德,钦奉先猷,奚命有司,用精详订。臣等谨按:当唐之治朝,品位爵秩,考覈选举,其法号为精密。尚虑拘牵,故远自开元所记,降及辽宋之传,参用讲求。有便于今者,不必泥古,取正于法者,亦无徇习。今先定到官号品次职守,上进御府,以尘乙览。恭俟圣断,曲加是正。言顺事成,名宾实举,兴化阜民,于是乎在。凡新书未载,并乞姑仍旧贯。徐用讨论,继此奏请。臣等顾惟虚薄,讲究不能及远,以塞明命是惧。倘涓埃有取,伏乞先赐颁降施行。''答诏曰:``朕闻可则循,否则革,事不惮于改为;言之易,行之难,政或讥于欲速。审以后举,示将不刊。爰自先皇,已颁明命;顺考古道,作新斯人。欲端本于朝廷,首建官于台省。岂止百司之职守,必也正名;是将一代之典章,无乎不在。能事未毕,眇躬嗣承。惧坠先猷,惕增夕厉,勉图继述,申命讲求。虽曰法唐,宜后先之一揆;至于因夏,固损益之殊途。务折衷以适时,肆于今而累岁。庶同乃绎,仅至有成,掇所先行,用敷众听。作室肯构,第遵底法之良;若网在纲,庶弭有条之紊。自余款备,继此施陈。已革乃孚,行取四时之信;所由适治,揭为万世之常。凡在见闻,共思遵守。''翰林学士韩昉撰诏书曰:``皇祖有训,非继体者所敢忘;圣人无心,每立事于不得已。朕丕承洪绪,一纪于兹;祗遹先猷,百为不越。故在朝廷之上,其犹草昧之初。比以大臣力陈恳奏,谓纲纪之未举,在国家以何观!且名可言,而言可行,所由集事;盖变则通,而通则久,故用裕民。宜法古官,以开政府。正号以责实效,着仪而辨等威。天有雷风,辞命安得不作;人皆颜闵,印符然后可捐。凡此数条,皆今急务。礼乐之备,源流在兹,祈以必行,断宜有定。仰惟先帝,亦鉴微衷。神岂可诬,方在天而对越;时由偶异,若易地则皆然。是用载惟,殆非相反。何必改作,盖尝三复于斯言;皆曰可行,庶将一变而至道。乃从所议,用创新规。维兹故土之风,颇尚先民之质。性成于习,遽易为难;政有所因,姑宜仍旧。渐祈胥效,翕致大同。凡在迩遐,当体朕意。其所改创事件,宜令尚书省就便从宜施行。''

宋、兖诸王之诛,韩昉作诏曰:``周行管叔之诛,汉致燕王之辟,兹维无赦,古不为非。岂亲亲之道有所未敦?以恶恶之心是不可忍。朕自惟冲昧,猥嗣统临。盖由文烈之公,欲大武元之后。德虽为否,义亦当然。不图骨肉之闲,有怀蜂虿之毒。皇伯太师、宋国王宗磐,族联诸父,位冠三师。始朕承祧,乃繄协力,肆登极品,兼绾剧权,何为失图,以底不类?谓为先帝之元子,当蓄无君之祸心,昵信宵人,煽为奸党,坐图问鼎,行将弄兵。皇叔太傅、领三省事。兖国王宗隽为国至亲,与朕同体,内怀悖德,外纵虚骄。肆己之怒,专杀以取威;擅公之财,市恩而惑众。力摈勋旧,欲孤朝廷。即其所誱,济以同恶。皇叔虞王宗英、滕王宗伟、殿前左副点检浑睹、会宁少尹胡实刺、郎君石家奴、千户述离、古楚等,竞为祸始,举好乱从。逞躁欲以无厌,助逆谋之妄作。意所非冀,获其必成。先将贼其大臣,次欲危其宗庙。造端累岁,举事有期。早露端倪,每存含覆;第严禁卫,载肃礼文。庶见君亲之威,少安臣子之分。蔑然不顾,狂甚自如。尚赖神明之灵,克开社稷之福。日者叛人吴十稔心称乱,授首底亡。爰致克奔之徒,乃穷相与之党,得厥情状,孚于见闻。皆由左验以质成,莫敢诡辞而抵谰。欲申三宥,公议岂容;不烦一兵,群凶悉殄。于今月三日,已各伏辜,并令有司除属籍讫。自余诖误,更不蹑寻;庶示宽容,用安反侧。民画衣而有犯,古犹钦哉;予素服以如丧,情可知也。''

陈王悟室《加恩制》词曰:``贵贵尊贤,式重仪刑之望;亲亲尚齿,亦优宗族之恩。朕俯迫群情,祗膺显号。爰第景风之赏,孰居台曜之先。凡尔在廷,听予作命。具官属为诸父,身相累朝。蹈五常九德之规,为四辅三公之冠。当艰难创业之际,藉左右宅师之勤。如献兆之信蓍龟,如济川之待舟楫。迪我高后,格于皇天。属正统之有归,赖嘉谋之先定。缉熙百度,董正六官。雍容以折肘腋之奸,指顾以定朔南之地。德业并茂,古今罕伦。迨兹庆赐之颁,询及佥谐之论。谓上公之加命有九,而天下之达尊者三。既已兼全,无可增益。乃敷求于载籍,仍自断于朕心。杖以造朝,前已加于异数;坐于论道,今复举于旧章。萧相国赐诏不名,安平王肩舆升殿。并兹优渥,以奖耆英。于戏!建无穷之基,则必享无穷之福;锡非常之礼,所以报非常之功。钦承体貌之隆,并对邦家之祉。''

皇后裴摩申氏《谢表》曰:``龙衮珠旒,端临云陛;玉书金玺,荣畀椒房。恭受以还,凌竞罔措。恭惟道兼天覆,明并日升。诚意正心,基周王之风化;制礼作乐,焕尧帝之文章。俯矜奉事之劳,饬遣光华之使。温言奖饰,美号重仍。顾拜命之甚优,惭省躬而莫称。谨当恪遵睿训,益励肃心。庶几妇道之修,仰助人文之化。''后父小名胡搭。

渤海《贺正表》曰:``三阳应律,载肇于岁华;万寿称觞,欣逢于元会。恭惟受天之祜,如日之升。布治惟新,顺夏时而谨始;卜年方永,迈周历以垂休。臣幸际明昌,良深抃颂。远驰信币,用申祝圣之诚;仰冀清躬,茂集履端之庆。''

夏国《贺正表》曰:``斗柄建寅,当帝历更新之旦;葭灰飞管,属皇图正始之辰。四序推先,一人履庆。恭惟化流中外,德被迩遐。方熙律之载阳,应令候而布惠。克凝神于窔奥,务行政于要荒。四表无虞,群黎至洽。爰凤阙届春之早,协龙廷展贺之初。百辟称觞,用尽输诚之意;万邦荐祉,克坚献岁之心。臣无任云云。大使武功郎没细好德、副使宣德郎李膺等齎表诣阙以闻。''

高丽《贺正表》曰:``帝出乎震,方当遂三阳之主;王次于春,所以大一统之始。覆帱之内,欢庆皆均。恭惟中孚应天,大有得位。所过者化,阅众甫以常新;不怒而威,观庶邦以率服。茂对佳辰之复,备膺诸福之休。臣幸遘昌期,远居外服。上千万岁寿,曾莫预于胪传;同亿兆人心,但窃深于善祝云云。使朝散大夫卫尉、少卿轻车都尉、赐紫金鱼袋李仲衍奉表称贺以闻。''

右《松漠纪闻》二卷。先君衔使十五年,深阸穷漠,耳目所接,随笔纂录。闻孟公庾发箧汴都,危变归计,创艾而火其书,秃节来归。因语言得罪柄臣,诸子佩三缄之戒,循陔侍膝,不敢以北方事置齿牙间。及南徙炎荒,视膳余日,稍亦谈及远事。凡不涉今日强弱利害者,因操牍记其一二。未几复有私史之禁,先君亦枕末疾,遂废不录。及柄臣盖棺,弛语言之律,而先君已齎恨泉下。鸠拾残稿,仅得数十事,反袂拭面,着为一编。绍兴丙子夏长男适谨书。

\hypertarget{header-n24}{%
\subsection{松漠纪闻补遗}\label{header-n24}}

虏中庙讳尤严,不许人犯。尝有一武弁经西元帅投牒,误斥其讳,杖背流递。武元初,只讳``旻'',后有申请云:旻,闵也。遂并``闵''讳之。

虏中中丞唯掌讼牒,若断狱会法。或春山秋水,谓去国数百里,逐水草而居处。从驾在外,卫兵物故,则掌其骸骼,至国则归其家。谏官并以他官兼之,与台官皆备员,不弹击。外道虽有漕使,亦不刺举,故官吏赃秽,略无所惮。

虏法:文武官不以高下,凡丁家难未满百日,皆差监关税、州商税院、盐铁场,一年为任,谓之``优饶''。其税课倍增者谓之``得筹''。每一筹转一官,有岁中八、九迁者。近有止法,不得过三官。富者择课额少处受之,或以家财贴纳,只图迁转。其不欲迁者于课利多处,除岁额外,公然分之。

虏中有负犯者,不责降,只差监盐场。课额虽登,出卖甚迟,虽任满去官,非卖尽不得仕,至有十年不调者。无磨勘之法,每一任转一官,以二十五月为任,将满即改除,并不待阙。

北地汉儿张献甫作太原都军,都监也。其姊夫刘思与侍郎高庆裔为十友之数。张有一犀带,国初钱王所献者,号``镇国宝带'',是正透,中间龙形。

契丹重骨咄犀,犀不大,万株犀无一不曾作带纹,如象牙带黄色,止是作刀把,已为无价。天祚以此作兔鹘,中国谓之腰条皮。插垂头者。

鹿顶合,燕以北者方可车,须是未解角之前。才解角,血脉通,冬至方解。顶之上为``合正须'',亦作``合''。好者有人字,不好者成八字,有髓眼,不实。北人谓角为鹿角合,顶为鹿顶合。南中止有鹿角合。南鹿不实,定有髓眼,不可车。北地角未老,不至秋时不中。

糜角与鹿角不同,糜角如驼骨,通身可车,却无纹,生枝不比。鹿皆小鹿,顶骨有纹,上下无之,亦可熏成纹。

犀有三种:重透,外黑有一晕白,中又黑,世艰得之。正透,又曰通犀。例透,亦曰花犀或班犀,有游鱼形诸犀中。水犀最贵。秀州周通直家有正透犀带,其中一点白,以纸镫近之即时灭,有湿气,疑是水犀。

耀段褐色,泾段白色。生丝为经,羊毛为纬,好而不耐。丰段有白有褐最佳,驼毛段出河西,有褐有白。

秋毛最佳,不蛀。冬间毛落,去毛上之粗者,取其茸毛。皆关西羊为之,蕃语谓之``髊劝''。北羊止作粗毛。

先忠宣《松漠纪闻》,伯兄镂板歙越。遵来守建业又刻之。暇日,搜阅故牍,得北方十有一事,皆曩岁侍旁亲闻之者,目曰``补遗'',附载于此。干道九年六月二日,第二男资政殿大学士、左中大夫知建康府江南东路安抚使兼行宫留守遵谨书。

\end{document}
