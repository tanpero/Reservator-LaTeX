\PassOptionsToPackage{unicode=true}{hyperref} % options for packages loaded elsewhere
\PassOptionsToPackage{hyphens}{url}
%
\documentclass[]{article}
\usepackage{lmodern}
\usepackage{amssymb,amsmath}
\usepackage{ifxetex,ifluatex}
\usepackage{fixltx2e} % provides \textsubscript
\ifnum 0\ifxetex 1\fi\ifluatex 1\fi=0 % if pdftex
  \usepackage[T1]{fontenc}
  \usepackage[utf8]{inputenc}
  \usepackage{textcomp} % provides euro and other symbols
\else % if luatex or xelatex
  \usepackage{unicode-math}
  \defaultfontfeatures{Ligatures=TeX,Scale=MatchLowercase}
\fi
% use upquote if available, for straight quotes in verbatim environments
\IfFileExists{upquote.sty}{\usepackage{upquote}}{}
% use microtype if available
\IfFileExists{microtype.sty}{%
\usepackage[]{microtype}
\UseMicrotypeSet[protrusion]{basicmath} % disable protrusion for tt fonts
}{}
\IfFileExists{parskip.sty}{%
\usepackage{parskip}
}{% else
\setlength{\parindent}{0pt}
\setlength{\parskip}{6pt plus 2pt minus 1pt}
}
\usepackage{hyperref}
\hypersetup{
            pdfborder={0 0 0},
            breaklinks=true}
\urlstyle{same}  % don't use monospace font for urls
\setlength{\emergencystretch}{3em}  % prevent overfull lines
\providecommand{\tightlist}{%
  \setlength{\itemsep}{0pt}\setlength{\parskip}{0pt}}
\setcounter{secnumdepth}{0}
% Redefines (sub)paragraphs to behave more like sections
\ifx\paragraph\undefined\else
\let\oldparagraph\paragraph
\renewcommand{\paragraph}[1]{\oldparagraph{#1}\mbox{}}
\fi
\ifx\subparagraph\undefined\else
\let\oldsubparagraph\subparagraph
\renewcommand{\subparagraph}[1]{\oldsubparagraph{#1}\mbox{}}
\fi

% set default figure placement to htbp
\makeatletter
\def\fps@figure{htbp}
\makeatother


\date{}

\begin{document}

\hypertarget{header-n0}{%
\section{棋经十三篇}\label{header-n0}}

\begin{center}\rule{0.5\linewidth}{\linethickness}\end{center}

\tableofcontents

\begin{center}\rule{0.5\linewidth}{\linethickness}\end{center}

\hypertarget{header-n10}{%
\subsection{论局篇}\label{header-n10}}

夫万物之数,从一而起。局之路,三百六十有一。一者,生数之主,据其极而运四方也。三百六十,以象周天之数。分而为四,以象四时。隅各九十路,以象其日。外周七二路,以象其候。枯棋三百六十,白黑相半,以法阴阳。局之线道,谓之枰。线道之间,谓之罫。局方而静,棋圆而动。自古及今,弈者无同局。《传》曰:``日日新。''故宜用意深而存虑精,以求其胜负之由,则至其所未至矣。

\hypertarget{header-n151}{%
\subsection{得算篇}\label{header-n151}}

棋者,以正合其势,以权制其敌。故计定于内而势成于外。战未合而算胜者,得算多也。算不胜者,得算少也。战已合而不知胜负者,无算也。兵法曰﹕``多算胜,少算不胜,而况于无算乎?由此观之,胜负见矣。''

\hypertarget{header-n155}{%
\subsection{权舆篇}\label{header-n155}}

权舆者,弈棋布置,务守纲格。先于四隅分定势子,然后拆二斜飞,下势子一等。立二可以拆三,立三可以拆四,与势子相望可以拆五。近不必比,远不必乖。此皆古人之论,后学之规,舍此改作,未之或知。诗曰﹕``靡不有初,鲜克有终。''

\hypertarget{header-n159}{%
\subsection{合战篇}\label{header-n159}}

博弈之道,贵乎谨严。高者在腹,下者在边,中者占角,此棋家之常然。法曰﹕宁输数子,勿失一先。有先而后,有后而先。击左则视右,攻后则瞻前。两生勿断,皆活勿连。阔不可太疏,密不可太促。与其恋子以求生,不若弃子而取势,与其无事而强行,不若因之而自补。彼众我寡,先谋其生。我众彼寡,务张其势。善胜者不争,善阵者不战。善战者不败,善败者不乱。夫棋始以正合,终以奇胜。必也,四顾其地,牢不可破,方可出人不意,掩人不备。凡敌无事而自补者,有侵袭之意也。弃小而不救者,有图大之心也。随手而下者,无谋之人也。不思而应者,取败之道也。诗云﹕``惴惴小心,如临于谷。''

\hypertarget{header-n163}{%
\subsection{虚实篇}\label{header-n163}}

夫弈棋,绪多则势分,势分则难救。投棋勿逼,逼则使彼实而我虚。虚则易攻,实则难破。临时变通,宜勿执一。《传》曰﹕``见可而进,知难而退。''

\hypertarget{header-n167}{%
\subsection{自知篇}\label{header-n167}}

夫智者见于未萌,愚者暗于成事。故知己之害而图彼之利者,胜。知可以战不可以战者,胜。识众寡之用者,胜。以虞待不虞者,胜。以逸待劳者,胜。不战而屈人者,胜。《老子》曰﹕``自知者明。''

\hypertarget{header-n171}{%
\subsection{审局篇}\label{header-n171}}

夫弈棋布势,务相接连。自始至终,着着求先。临局离争,雌雄未决,毫厘不可以差焉。局势已赢,专精求生。局势已弱,锐意侵绰。沿边而走,虽得其生者,败。弱而不伏者,愈屈。躁而求胜者,多败。两势相违,先蹙其外。势孤援寡,则勿走。机危阵溃,则勿下。是故棋有不走之走,不下之下。误人者多方,成功者一路而已。能审局者多胜。《易》曰﹕``穷则变,变则通,通则久。''

\hypertarget{header-n175}{%
\subsection{度情篇}\label{header-n175}}

人生而静,其情难见;感物而动,然后可辨。推之于棋,胜败可得而先验。持重而廉者多得,轻易而贪者多丧。不争而自保者多胜,务杀而不顾者多败。因败而思者,其势进;战胜而骄者,其势退。求己弊不求人之弊者,益;攻其敌而不知敌之攻己者,损。目凝一局者,其思周;心役他事者,其虑散。行远而正者吉,机浅而诈者凶。能畏敌者强,谓人莫己若者亡。意旁通者高,心执一者卑。语默有常,使敌难量。动静无度,招人所恶。《诗》云﹕``他人之心,予时度之。''

\hypertarget{header-n179}{%
\subsection{斜正篇}\label{header-n179}}

或曰﹕``棋以变诈为务,劫杀为名,岂非诡道耶?''予曰﹕``不然。''《易》云﹕``师出以律,否藏凶。''兵本不尚诈,谋言诡行者,乃战国纵横之说。棋虽小道,实与兵合。故棋之品甚繁,而弈之者不一。得品之下者,举无思虑,动则变诈。或用手以影其势,或发言以泄其机。得品之上者,则异于是。皆沉思而远虑,因形而用权。神游局内,意在子先。图胜于无朕,灭行于未然。岂假言辞喋喋,手势翩翩者哉?《传》曰﹕``正而不谲。''其是之谓欤?

\hypertarget{header-n183}{%
\subsection{洞微篇}\label{header-n183}}

凡棋有益之而损者,有损之而益者。有侵而利者,有侵而害者。有宜左投者,有宜右投者。有先著者,有后著者。有紧避者,有慢行者。粘子勿前,弃子思后。有始近而终远者,有始少而终多者。欲强外先攻内,欲实东先击西。路虚而无眼,则先觑。无害于他棋,则做劫。饶路则宜疏,受路则勿战。择地而侵,无碍而进。此皆棋家之幽微也,不可不知也。《易》曰﹕``非天下之至精,其孰能与于此。''

\hypertarget{header-n187}{%
\subsection{名数篇}\label{header-n187}}

夫弈棋者,凡下一子,皆有定名。棋之形势、死生、存亡,因名而可见。有冲,有斡,有绰,有约,有飞,有关,有札,有粘,有顶,有尖,有觑,有门,有打,有断,有行,有捺,有立,有点,有聚,有跷,有夹,有拶,有避,有刺,有勒,有扑,有征,有劫,有持,有杀,有松,有盘。围棋之名,三十有二,围棋之人,意在可周。临局变化,远近纵横,吾不得而知也。用幸取胜,难逃此名。《传》曰﹕``必也,正名乎棋!''

\hypertarget{header-n191}{%
\subsection{品格篇}\label{header-n191}}

夫围棋之品有九。一曰入神,二曰坐照,三曰具体,四曰通幽,五曰用智,六曰小巧,七曰斗力,八曰若愚,九曰守拙。九品之外不可胜计,未能入格,今不复云。《传》曰﹕``生而知之者,上也;学而知之者,次也;困而学之又其次也。''

\hypertarget{header-n195}{%
\subsection{杂说篇}\label{header-n195}}

夫棋边不如角,角不如腹。约轻于捺,捺轻于避。夹有虚实,打有情伪。逢绰多约,遇拶多粘。大眼可赢小眼,斜行不如正行。两关对直则先觑,前途有碍则勿征。施行未成,不可先动。角盘曲四,局终乃亡。直四扳六,皆是活棋,花聚透点,多无生路。十字不可先纽,势子在心,勿打角图。弈不欲数,数则怠,怠则不精。弈不欲疏,疏则忘,忘则多失。胜不言,败不语。振廉让之风者,君子也;起忿怒之色者,小人也。高者无亢,卑者无怯。气和而韵舒者,喜其将胜也。心动而色变者,忧其将败也。赧莫赧于易,耻莫耻于盗。妙莫妙于用松,昏莫昏于复劫。凡棋直行三则改,方聚四则非。胜而路多,名曰赢局;败而无路,名曰输筹。皆筹为溢,停路为芇。打筹不得过三,淘子不限其数。劫有金井、辘轳,有无休之势,有交递之图。弈棋者不可不知也。凡棋有敌手,有半先,有先两,有桃花五,有北斗七。夫棋者有无之相生,远近之相成,强弱之相形,利害之相倾,不可不察也。是以安而不泰,存而不骄。安而泰则危,存而骄则亡。《易》曰﹕``君子安而不忘危,存而不忘亡。''

\end{document}
