\PassOptionsToPackage{unicode=true}{hyperref} % options for packages loaded elsewhere
\PassOptionsToPackage{hyphens}{url}
%
\documentclass[]{article}
\usepackage{lmodern}
\usepackage{amssymb,amsmath}
\usepackage{ifxetex,ifluatex}
\usepackage{fixltx2e} % provides \textsubscript
\ifnum 0\ifxetex 1\fi\ifluatex 1\fi=0 % if pdftex
  \usepackage[T1]{fontenc}
  \usepackage[utf8]{inputenc}
  \usepackage{textcomp} % provides euro and other symbols
\else % if luatex or xelatex
  \usepackage{unicode-math}
  \defaultfontfeatures{Ligatures=TeX,Scale=MatchLowercase}
\fi
% use upquote if available, for straight quotes in verbatim environments
\IfFileExists{upquote.sty}{\usepackage{upquote}}{}
% use microtype if available
\IfFileExists{microtype.sty}{%
\usepackage[]{microtype}
\UseMicrotypeSet[protrusion]{basicmath} % disable protrusion for tt fonts
}{}
\IfFileExists{parskip.sty}{%
\usepackage{parskip}
}{% else
\setlength{\parindent}{0pt}
\setlength{\parskip}{6pt plus 2pt minus 1pt}
}
\usepackage{hyperref}
\hypersetup{
            pdfborder={0 0 0},
            breaklinks=true}
\urlstyle{same}  % don't use monospace font for urls
\setlength{\emergencystretch}{3em}  % prevent overfull lines
\providecommand{\tightlist}{%
  \setlength{\itemsep}{0pt}\setlength{\parskip}{0pt}}
\setcounter{secnumdepth}{0}
% Redefines (sub)paragraphs to behave more like sections
\ifx\paragraph\undefined\else
\let\oldparagraph\paragraph
\renewcommand{\paragraph}[1]{\oldparagraph{#1}\mbox{}}
\fi
\ifx\subparagraph\undefined\else
\let\oldsubparagraph\subparagraph
\renewcommand{\subparagraph}[1]{\oldsubparagraph{#1}\mbox{}}
\fi

% set default figure placement to htbp
\makeatletter
\def\fps@figure{htbp}
\makeatother


\date{}

\begin{document}

\hypertarget{header-n12}{%
\section{浮生六记}\label{header-n12}}

\begin{center}\rule{0.5\linewidth}{\linethickness}\end{center}

\tableofcontents

\begin{center}\rule{0.5\linewidth}{\linethickness}\end{center}

\hypertarget{header-n18}{%
\subsection{闺房记乐}\label{header-n18}}

余生乾隆癸未冬卜一月二十有二日,正值太平盛世,且在衣冠之家,后苏州沧浪亭畔,天之厚我可谓至矣。东坡云:``事如春梦了无痕'',苟不记之笔墨,未免有辜彼苍之厚。因思《关雎》冠三百篇之首,被列夫妇于首卷,余以次递及焉。所愧少年失学,稍识之无,不过记其实情实事而已,若必考订其文法,是责明于垢鉴矣。

余幼聘金沙于氏,八龄而夭。娶陈氏。陈名芸,字淑珍,舅氏心余先生女也,生而颖慧,学语时,口授《琵琶行》,即能成诵。四龄失怙,母金氏,弟克昌,家徒壁立。芸既长,娴女红,三口仰其十指供给,克昌从师,修脯无缺。一日,于书簏中得《琵琶行》,挨字而认,始识字。刺绣之暇,渐通吟咏,有``秋侵人影瘦,霜染菊花肥''之句。余年---十三,随母归宁,两小无嫌,得见所作,虽叹其才思隽秀,窃恐其福泽不深,然心注不能释,告母曰:``若为儿择妇,非淑姊不娶。''母亦爱其柔和,即脱金约指缔姻焉。此乾隆乙末七月十六日也。

是中冬,值其堂姊出阁,余又随母往。芸与余同齿而长余十月,自幼姊弟相呼,故仍呼之曰淑姊。时但见满室鲜衣,萎独通体素淡,仅新其鞋而已。见其绣制精巧,询为己作,始知其慧心不仅在笔墨也。其形削肩长项,瘦不露骨,眉弯目秀,顾盼神飞,唯两齿微露;似非佳相。一种缠绵之态,令人之意也消。索观诗稿,有仅一联,或三四句,多未成篇者,询其故,笑曰:``无师之作,愿得知己堪师者敲成之耳。''余戏题其签曰``锦囊佳句''。不知夭寿之机此已伏矣。是夜送亲城外,返已漏三下,腹饥索饵,婢妪以枣脯进,余嫌其甜。芸暗牵余袖,随至其室,见藏有暖粥并小菜焉,余欣然举箸。忽闻芸堂兄玉衡呼曰:``淑妹速来!''芸急闭门曰:``已疲乏,将卧矣。''玉衡挤身而入,见余将吃粥,乃笑睨芸曰:``顷我索粥,汝曰`尽矣',乃藏此专待汝婿耶?''芸大窘避去,上下哗笑之。余亦负气,挈老仆先归。自吃粥被嘲,再往,芸即避匿,余知其恐贻人笑也。

至乾隆庚子正月二十二日花烛之夕,见瘦怯身材依然如昔,头巾既揭,相视嫣然。合卺后,并肩夜膳,余暗于案下握其腕,暖尖滑腻,胸中不觉抨抨作跳。让之食,适逢斋期,已数年矣。暗计吃斋之初,正余出痘之期,因笑调曰:``今我光鲜无恙,姊可从此开戒否?''芸笑之以目,点之以首。

廿四日为余姊于归,廿三国忌不能作乐,故廿二之夜即为余婉款嫁。芸出堂陷宴,余在洞房与伴娘对酌,拇战辄北,大醉而卧,醒则芸正晓妆未竟也。是日亲朋络绎,上灯后始作乐。廿四子正,余作新舅送嫁,丑末归来,业已灯残人静,悄然入室,伴妪盹于床下,芸卸妆尚未卧,高烧银烛,低垂粉颈,不知观何书而出神若此,因抚其肩曰:``姊连日辛苦,何犹孜孜不倦耶?''芸忙回首起立曰:``顷正欲卧,开橱得此书,不觉阅之忘倦。《西厢》之名闻之熟矣,今始得见,莫不傀才子之名,但未免形容尖薄耳。''余笑曰:``唯其才子,笔墨方能尖薄。''伴妪在旁促卧,令其闭门先去。遂与比肩调笑,恍同密友重逢。戏探其怀,亦怦怦作跳,因俯其耳曰:``姊何心舂乃尔耶?''芸回眸微笑。便觉一缕情丝摇人魂魄,拥之入帐,不知东方之既白。

芸作新妇,初甚缄默,终日无怒容,与之言,微笑而已。事上以敬,处下以和,井井然未尝稍失。每见朝暾上窗,即披衣急起,如有人呼促者然。余笑曰:``今非吃粥比矣,何尚畏人嘲耶?''芸曰:``曩之藏粥待君,传为话柄,今非畏嘲,恐堂上道新娘懒惰耳。''余虽恋其卧而德其正,因亦随之早起。自此耳鬓相磨,亲同形影,爱恋之情有不可以言语形容者。

而欢娱易过,转睫弥月。时吾父稼夫公在会稽幕府,专役相迓,受业于武林赵省斋先生门下。先生循循善诱,余今日之尚能握管,先生力也。归来完姻时,原订随侍到馆。闻信之徐,心甚怅然,恐芸之对人堕泪。而芸反强颜劝勉,代整行装,是晚但觉神色稍异面已。临行,向余小语曰:``无人调护,自去经心!''及登舟解缆,正当桃李争研之候,而余则恍同林鸟失群,天地异色。到馆后,吾父即渡江东去。

居三月,如十年之隔。芸虽时有书来,必两问一答,中多勉励词,余皆浮套语,心殊怏怏。每当风生竹院,月上蕉窗,对景怀人,梦魂颠倒。先生知其情,即致书吾父,出十题而遣余暂归。喜同戍人得赦,登舟后,反觉一刻如年。及抵家,吾母处问安毕,入房,芸起相迎,握手未通片语,而两人魂魄恍恍然化烟成雾,觉耳中惺然一响,不知更有此身矣。

时当六月,内室炎蒸,幸居沧浪亭爱莲居西间壁,板桥内一轩临流,名曰``我取'',取``清斯濯缨,浊斯濯足''意也。榴前老树一株,浓阴覆窗,人画俱绿。隔岸游人往来不绝。此吾父稼夫公垂帘宴客处也。禀命吾母,携芸消夏于此。因暑罢绣,终日伴余课书论古,品月评花而已。芸不善饮,强之可三杯,教以射覆为令。自以为人间之乐,无过于此矣。

一日,芸问曰:``各种古文,宗何为是?''余曰:``《国策》、《南华》取其灵快,匡衡、刘向取其雅健,史迁、班固取其博大,昌黎取其浑,柳州取其峭,庐陵取其宕,三苏取其辩,他若贾、董策对,庾、徐骈体,陆贽奏议,取资者不能尽举,在人之慧心领会耳。''芸曰:``古文全在识高气雄,女子学之恐难入彀,唯诗之一道,妾稍有领悟耳。''余曰:``唐以诗取士,而诗之宗匠必推李、杜,卿爱宗何人?''芸发议曰:``杜诗锤炼精纯,李诗激洒落拓.与其学杜之森严,不如学李之活泼。''余曰:``工部为诗家之大成,学者多宗之,卿独取李,何也?''芸曰:``格律谨严,词旨老当,诚杜所独擅。但李诗宛如姑射仙子,有一种落花流水之趣,令人可爱。非杜亚于李,不过妾之私心宗杜心浅,爱李心深。''余笑日:``初不料陈淑珍乃李青莲知已。''芸笑曰:``妄尚有启蒙师自乐天先生,时感于怀,未尝稍露。''余曰:``何谓也?''芸曰:``彼非作《琵琶行》者耶?''余笑曰:``异哉!李太白是知己,自乐天是启蒙师,余适字三白,为卿婿,卿与`白'宇何其有缘耶?''差笑曰:``白字有缘,将来恐白字连篇耳(吴音呼别字为白字)。''相与大笑。余曰:``卿既知诗,亦当知赋之弃取。''芸曰:``《楚辞》为赋之祖,妾学浅费解。就汉、晋人中调高语炼,似觉相如为最。''余戏曰:``当日文君之从长卿,或不在琴而在此乎?''复相与大笑而罢。

余性爽直,落拓不羁;芸若腐儒,迂拘多礼。偶为之整袖,必连声道``得罪'';或递巾授扇,必起身来接。余始厌之,曰:``卿欲以礼缚我耶?《语》曰:`礼多必诈'。''芸两颊发赤,曰:``恭而有礼,何反言诈?''余曰:``恭敬在心,不在虚文。''芸曰:``至亲莫如父母,可内敬在心而外肆狂放耶?''余曰:``前言戏之耳。''芸曰:``世间反目多由戏起,后勿冤妾,令人郁死!''余乃挽之入怀,抚慰之,始解颜为笑。自此``岂敢''、``得罪''竟成语助词矣。鸿案相庄廿有三年,年愈久而情愈密。家庭之内,或暗室相逢,窄途邂逅,必握手问曰:``何处去?''私心忒忒,如恐旁人见之者。实则同行并坐,初犹避人,久则不以为意。芸或与人坐谈,见余至,必起立偏挪其身,余就而并焉。彼此皆不觉其所以然者,始以为惭,继成不期然而然。独怪老年夫妇相视如仇者,不知何意?或日:``非如是,焉得白头偕老哉?''斯言诚然钦?

是年七夕,芸设香烛瓜果,同拜天孙干我取轩中。余镌``愿生生世世为夫妇''图章二方,余执朱文,芸执白文,以为往来书信之用。是夜月色颇佳,俯视河中,波光如练,轻罗小扇,并坐水窗,仰见---飞云过天,变态万状。芸曰:``宇宙之大,同此一月,不知今日世间,亦有如我两人之情兴否?''余曰:``纳凉玩月,到处有之。若品论云霞,或求之幽闺绣闼,慧心默证者固亦不少。若夫妇同观,所品论着恐不在此云霞耳。''未几,烛烬月沉,撤果归卧。

七月望,俗谓鬼节,芸备小酌,拟邀月畅饮。夜忽阴云如晦,芸愀然曰:``妾能与君白头偕老,月轮当出。''余亦索然。但见隔岸萤光,明灭万点,梳织于柳堤蓼渚间。余与芸联句以遣闷怀,而两韵之后,逾联逾纵,想入非夷,随口乱道。芸已漱涎涕泪,笑倒余怀,不能成声矣。觉其鬃边茉莉浓香扑鼻,因拍其背,以他词解之曰:``想古人以茉莉形色如珠,故供助妆压鬓,不知此花必沾油头粉面之气,其香更可爱,所供佛手当退三舍矣。''芸乃止笑曰:``佛手乃香中君子,只在有意无意间;莱莉是香中小人,故须借人之势,其香也如胁肩谄笑。''余曰:``卿何远君子而近小人?''芸曰:``我笑君子爱小人耳。''正话间,漏已三滴,渐见风扫云开,一轮涌出,乃大喜,倚窗对酌。酒未三杯,忽闻桥下哄然一声,如有人堕。就窗细瞩,波明如镜,不见一物,惟闻河滩有只鸭急奔声.余知沧浪亭畔素有溺鬼,恐芸胆怯,未敢即言,芸曰:``噫!此声也,胡为乎来哉?''不禁毛骨皆栗。急闭窗,携酒归房.一灯如豆,罗帐低垂,弓影杯蛇,惊神未定。剔灯入帐,芸已寒热大作。余亦继之,困顿两旬。真所谓乐极灾生,亦是白头不终之兆。

中秋日,余病初愈。以芸半年新妇,未尝一至间壁之沧浪亭,先令老仆约守者勿放闲人,于将晚时,偕芸及余幼妹,一妪一婢扶焉,老仆前导,过石桥,进门折东,曲径而入。叠石成山,林木葱翠,亭在土山之巅。循级至亭心,周望极目可数里,炊烟四起,晚霞灿然。隔岸名``近山林'';为大宪行台宴集之地,时正谊书院犹未启也。携一毯设亭中,席地环坐,守着烹茶以进。少焉,一轮明月已上林梢,渐觉风生袖底,月到被心,俗虑尘怀,爽然顿释。芸曰:``今日之游乐矣!若驾一叶扁舟,往来亭下,不更快哉!''时已上灯,亿及七月十五夜之惊,相扶下亭而归。吴俗,妇女是晚不拘大家小户皆出,结队而游,名曰``走月亮''。沧浪亭幽雅清旷,反无一人至者。

吾父稼夫公喜认义子,以故余异姓弟兄有二十六人。吾母亦有义女九人,九人中王二姑、俞六姑与芸最和好。王痴憨善饮,俞豪爽善谈。每集,必逐余居外,而得三女同锡,此俞六姑一人计也。余笑曰:``俟妹于归后,我当邀妹丈来,一住必十日。''俞曰:``我亦来此,与嫂同榻,不大妙耶?''芸与王微笑而已。

时为吾弟启堂娶妇,迁居钦马桥之米仓巷,屋虽宏畅,非复沧浪亭之幽雅矣。吾母诞辰演剧,芸初以为奇观。吾父素无忌讳,点演《惨别》等剧,老伶刻画,见者情动,余窥帘见芸忽起去,良久不出,入内探之,俞与王亦继至。见芸一人支颐独坐镜窗之侧,余曰:``何不快乃尔?''劳曰:``观剧原以陶情,今日之戏徒令人断肠耳。''俞与王皆笑之。系曰:``此深于情者也。''俞曰:``嫂将竟日独坐于此耶?''莹曰:``候有可观者再往耳。''王闻言先出,请吾母点《刺梁》《后索》等剧,劝芸出观,始称快。

余堂伯父素存公早亡,无后,吾父以余嗣焉。墓在西跨塘福寿山祖茔之侧,每年春日,必挈芸拜扫。王二姑闻其地有戈园之胜,请同往。芸见地下小乱石有苔纹,斑驳可观,指示余曰:``以此叠盆山,较宣州白石为古致。''余曰:``若此者恐难多得。''王曰:``嫂果爱此,我为拾之。''即向守坟者借麻袋一,鹤步而拾之.每得一块,余曰``善'',即收之;余曰``否'',即去之。未几,粉汗盈盈,拽袋返曰:``再拾则力不胜矣。''芸且拣且言曰:``我闻山果收获,必借猴力,果然。''王愤撮十指作哈痒状,余横阻之,责芸曰:``人劳汝逸,犹作此语,无怪妹之动愤也。''归途游戈园,稚绿娇红,争妍竞媚。王素憨,逢花必折,芸叱曰:``既无瓶养:又不簪戴,多折何为?!''王曰:``不知痛痒者,何害?''余笑曰:``将来罚嫁麻面多须郎,为花泄忿。''王怒余以目,掷花于地,以莲钩拨入池中,曰,``何欺侮我之甚也!''芸笑解之而罢。

芸初缄默,喜听余议论。余调其言,如蟋蟀之用纤草,渐能发议。其每日饭必用茶泡,喜食芥卤乳腐,吴俗呼为臭乳腐,又喜食虾卤瓜。此二物余生平所最恶者,因戏之曰:``狗无胃而食粪,以其不知臭秽;蜣螂团粪而化蝉,以其欲修高举也。卿其狗耶?蝉耶?''芸曰:``腐取其价廉而可粥可饭,幼时食惯,今至君家已如蜣螂化蝉,犹喜食之者,不忘本出;至卤瓜之味,到此初尝耳。''余曰;``然则我家系狗窦耶?''芸窘而强解日:``夫粪,人家皆有之,要在食与不食之别耳。然君喜食蒜,妾亦强映之。腐不敢强,瓜可扼鼻略尝,入咽当知其美,此犹无益貌丑而德美也。''余笑曰:``卿陷我作狗耶?''芸曰:``妾作狗久矣,屈君试尝之。''以箸强塞余口。余掩鼻咀嚼之,似觉脆美,开鼻再嚼,竟成异味,从此亦喜食。芸以麻油加白糖少许拌卤腐,亦鲜美;以卤瓜捣烂拌卤腐,名之曰双鲜酱,有异昧。余曰:``始恶而终好之,理之不可解也。''芸曰:``情之所钟,虽丑不嫌。''

余启堂弟妇,王虚舟先生孙女也,催妆时偶缺珠花,芸出其纳采所受者呈吾母,婢妪旁惜之,芸日:``凡为妇人,已属纯阴,珠乃纯阴之精,用为首饰,阳气全克矣,何贵焉?''而于破书残画反极珍惜:书之残缺不全者,必搜集分门,汇订成帙,统名之曰``继简残编'';字画之破损者,必觅故纸粘补成幅,有破缺处,倩予全好而卷之,名门``弃余集赏''。于女红、中馈之暇,终日琐琐,不惮烦倦。芸于破笥烂卷中,偶获片纸可观者,如得异宝.旧邻冯妪每收乱卷卖之。

其癖好与余同,且能察眼意,锤眉语,一举一动,示之以色,无不头头是道。余尝曰:``惜卿雌而伏,苟能化女为男,相与访名山,搜胜迹,遨游天下,不亦快哉!''芸曰:``此何难,俟妾鬃斑之后,虽不能远游五岳,而近地之虎阜、灵岩,南至西湖,北至平山,尽可偕游。''余曰:``恐卿鬓斑之日,步履已艰。''芸曰,``今世不能,期以来世。''余曰:``来世卿当作男,我为女子相从。''芸曰:``必得不昧今生,方觉有情趣。''余笑曰:``幼时一粥犹谈不了,若来世不昧今生,合卺之夕,细谈隔世,更无合眼时矣。''芸曰:``世传月下老人专司人间婚姻事,今生夫妇已承牵合,来世姻缘亦须仰借神力,盍绘一像祀之?''时有苕溪戚柳堤名遵,善写人物。倩绘一像:一手挽红丝,一手携杖悬姻缘簿,童颜鹤发,奔驰于非烟非雾中。此戚君得意笔也。友人石琢堂为题赞语于首,悬之内室,每逢朔望,余夫妇必焚香拜祷。后因家庭多故,此画竟失所在,不知落在谁家矣。``他生未卜此生休'',两人痴情,果邀神鉴耶?

迁仓米巷,余颜其卧楼曰``宾香阁'',盖以芸名而取如宾意也。院窄墙高,一无可取。后有厢谈,通藏书处,开窗对陆氏废园,但有荒凉之象。沧浪风景,时切芸怀。有老妪居金母桥之东、埂巷之北,绕屋皆菜圃,编篱为门,门外有池约亩许,花光树影,错杂篱边,其地即元末张士诚王府废基也。屋西数武,瓦砾堆成土山,登其巅可远眺,地旷人稀,颇饶野趣。妪偶言及,芸神往不置,谓余曰:``自自别沧浪,梦魂常绕,每不得已而思其次,其老妪之居乎?''余曰:``连朝秋暑灼人,正思得一清凉地以消长昼,卿若愿往,我先观其家可居,即袱被而往,作一月盘桓何如?''劳曰:``恐堂上不许。''余曰:``我自请之。''越日至其地,屋仅二间,前后隔而为四,纸窗竹榻,颇有幽趣。老妪知余意,欣然出其卧室为赁,四壁糊以白纸,顿觉改观。于是禀知吾母,挈芸居焉。邻仅老夫妇二人,灌园为业,知余夫妇避暑于此,先来通殷勤,并钓池鱼、摘园蔬为馈,偿其价,不受,芸作鞋报之,始谢而受。时方七月,绿树阴浓,水面风来,蝉鸣聒耳。邻老又为制鱼竿,与芸垂钓于柳阴深处。日落时登土山观晚霞夕照,随意联吟,有``兽云吞落日,弓月弹流星''之句。少焉月印池中,虫声四起,设竹榻于篱下,老妪报酒温饭熟,遂就月光对酌,微醺而饭。浴罢则凉鞋蕉扇,或坐或卧,听邻老谈因果报应事。三鼓归卧,周体清凉,几不知身居城市矣。篱边倩邻老购菊,遍植之。九月花开,又与芸居十日。吾母亦欣然来观,持螯对菊,赏玩竟日。芸喜曰:``他年当与君卜筑于此,买绕屋菜园十亩,课仆妪,植瓜蔬,以供薪水。君画我绣,以为持酒之需。布衣菜饭,可乐终身,不必作远游计也。''余深然之。今即得有境地,预知己沦亡,可胜浩叹!

离余家中里许,醋库巷有洞庭君祠,俗呼水仙庙。回廊曲折,小有园亭.每逢神诞,众姓各认一落,密悬一式之玻璃灯,中设宝座,旁列瓶几,插花陈设,以较胜负。日惟演戏,夜则参差高下,插烛于瓶花间,名曰``花照''。花光好影,宝鼎香浮,若龙宫夜宴。司事者或笙箫歌唱,或煮茗清谈,观者如蚁集,檐下皆设栏为限。余为众友邀去插花布置,因得躬逢其盛。归家向芸艳称之,芸曰:``惜妾非男子,不能往。''余曰:``冠我冠,衣我衣,亦化女为男之法也。''于是易鬓为辫,添扫蛾眉;加余冠,微露两鬃,尚可掩饰;服余衣,长一寸又半;于腰间折而缝之,外加马褂。芸曰:``脚下将奈何?''余曰:``坊间有蝴蝶履,大小由之,购亦极易,且早晚可代撤鞋之用,不亦善乎?''芸欣然。及晚餐后,装束既毕,效男子拱手阔步者良久,忽变卦曰:``妾不去矣,为人识出既不便,堂上闻之又不可。''余怂恿曰:``庙中司事者谁不知我,即识出亦不过付之一笑耳。吾母现在九妹丈家,密去密来,焉得知之。''芸揽镜自照,狂笑不已。余强挽之,悄然径去,遍游庙中,无识出为女子者。或问何人,以表弟对,拱手而已。最后至一处,有少妇幼女坐于所设宝座后,乃杨姓司事者之眷属也。芸忽趋彼通款曲,身一侧,而不觉一按少妇之肩,旁有婢媪怒而起曰:``何物狂生,不法乃尔!''余试为措词掩饰,芸见势恶,即脱帽翘足示之曰:``我亦女子耳。''相与愕然,转怒为欢,留茶点,唤肩舆送归。

吴江钱师竹病放,吾父信归,命余往吊。芸私调余曰:``吴江必经太湖,妾欲偕往,一宽跟界。''余曰:``正虑独行踽踽,得卿同行,固妙,但无可托词耳。''芸曰,``托言归宁。君先登舟,妾当继至。''余曰:``若然,归途当泊舟万年桥下,与卿待月乘凉,以续沧浪韵事。''时六月十八日也。是日早凉,携一仆先至胥江渡口,登舟而待,芸果肩舆至。解维出虎啸桥,渐见风帆沙鸟,水天一色。芸曰:``此即所谓太湖耶?今得见天地之宽,不虚此生矣!想闺中人有终身中能见此者!''闲话未几,风摇岸柳,已抵江城。

余登岸拜奠毕,归视舟中洞然,急询舟子。舟子指曰:``不见长桥柳阴下,观鱼鹰捕鱼者乎?''盖芸已与船家女登岸矣。余至其后,芸犹粉汗盈盈,倚女而出神焉。余拍其肩口:``罗衫汗透矣!''芜回首曰:``恐钱家有人到舟,故暂避之。君何回来之速也?''余笑曰:``欲捕逃耳。''于是相挽登舟,返棹至万年桥下,阳乌犹末落山。舟窗尽落,清风徐来,绒扇罗衫,剖瓜解暑。少焉霞映桥红,烟笼柳暗,银瞻欲上,渔火满江矣。命仆至船梢与舟子同饮。船家女名素云,与余有杯酒交,人颇不俗,招之与芸同坐。船头不张灯火,待月快酌,射覆为令。素云双目闪闪,听良久,曰:``觞政侬颇娴习,从未闻有斯令,愿受教。''芸即譬其言而开导之,终茫然。余笑曰:``女先生且罢论,我有一言作譬,即了然矣。''芸曰:``君若何譬之?''余曰:``鹤善舞而不能耕,牛善耕而不能舞,物性然也,先生欲反而教之,无乃劳乎?''素云笑捶余肩曰:``汝骂我耶!''芸出令曰;``只许动口,不许动手。违者罚大觥。''素云量豪,满斟一觥,一吸而尽。余曰:``动手但准摸索,不准捶人。''芸笑挽素云置余怀,曰:``请君摸索畅怀。''余笑曰:``卿非解人,摸索在有意无意间耳,拥而狂探,田舍郎之所为也。''时四鬃所簪莱莉,为酒气所蒸,杂以粉汗油香,芳馨透鼻,余戏曰:``小人臭味充满船头,令人作恶。''素云不禁握拳连捶曰:``谁教汝狂嗅耶?''芸呼曰:``违令,罚两大觥!''素云曰:``彼又以小人骂我,不应捶耶?''芸曰:``彼之所谓小人,益有故也。请干此,当告汝。''素云乃连尽两觥,芸乃告以沧浪旧居乘凉事。素云曰:``若然,真错怪矣,当再罚。''又干一觥。芸曰:``久闻素娘善歌,可一聆妙音否?''素即以象箸击小碟而歌。芸欣然畅饮,不觉酩酊,乃乘舆先归。余又与素云茶话片刻,步月而回。时余寄居友人鲁半舫家萧爽楼中,越数日,鲁夫人误有所闻,私告芸曰:``前日闻若婿挟两妓饮于万年桥舟中,子知之否?''姜口:``有之,其一即我也。''因以偕游始末详告之,鲁大笑,释然而去。

乾隆甲寅七月,亲自粤东归。有同伴携妾回者,曰徐秀峰,余之表妹婿也。艳称新人之美,邀芸往观。芸他日谓秀峰曰:``美则美矣,韵犹未也。''秀峰口:``然则若郎纳妾,必美而韵者?''芸口:``然。''从此痴心物色,而短于资。时有浙妓温冷香者,寓于吴,有咏柳絮四律,沸传吴下,好事者多和之。余友吴江张闲憨素赏冷香,携柳絮诗索和。芸微其人而置之,余技痒而和其韵,中有``触我春愁偏婉转,撩他离绪更缠绵''之句,芸甚击节。

明年乙卯秋八月五日,吾母将挈芸游虎丘,闲憨忽至曰:``余亦有虎丘之游,今日特邀君作探花使者。''因请吾母先行,期于虎丘半塘相晤,拉余至冷香寓。见冷香已半老;有女名憨园,瓜期未破,亭亭玉立,真``一泓秋水照人寒''者也,款接间,颇知文墨;有妹文园,尚雏。余此时初无痴想,且念一杯之叙,非寒士所能酬,而既入个中,私心忐忑,强为酬答。因私谓闲憨曰:``余贫士也,子以尤物玩我乎?''闲憨笑曰:``非也,今日有友人邀憨园答我,席主为尊客拉去,我代客转邀客,毋烦倾他虑也。''余始释然。

至半塘,两舟相遇,令憨园过舟叩见吾母。芸、憨相见,欢同旧识,携手登山,备览名胜。菩独爱千顷云高旷,坐赏良久。返至野芳滨,畅饮甚欢,并舟而泊。及解维,劳谓众出:``子陪张君,留憨陪妾可乎?''余诺之。返棹至都中桥,始过船分袂。归家已三鼓,芸曰:``今日得见美丽韵者矣,顷已约憨园明日过我,当为于图之。''余骇曰:``此非金屋不能贮,穷措大岂敢生此妄想哉?况我两人伉俪正笃,何必外求?''芸笑曰:``我自爱之,子姑待之。''

明午,憨果至。芸殷勤款接,缝中以猜枚赢吟输饮为令,终席无一罗致语。及憨园归,芸曰:``顷又与密约,十八日来此结为姊妹,子宜备牲牢以待。''笑指臂上翡翠钏曰:``若见此铡属于憨,事必谐矣,顷已吐意,未深结其心也。''余姑听之。十八日大雨,憨竟冒雨至。入室良久,始挽手出,见余有羞色,盖翡翠铡已在憨臂矣。焚香结盟后,拟再续前饮,适憨有石湖之游,即别去。芸欣然告余曰:``丽人已得,君何以谢媒耶?''余询其详,芸曰:``向之秘言,恐憨意另有所属也,顷探之无他,语之曰:`妹知今日之意否?'憨曰:`蒙夫人抬举,真蓬篙倚玉树也,但吾母望我奢,恐难自主耳,愿彼此缓图之。'脱钏上臂时,又语之曰:`玉取其坚,且有团园不断之意,妹试笼之以为先兆。'憨曰:`聚合之权总在夫人也。'即此观之,憨心已得,所难必者冷香耳,当再图之。''余笑曰:``卿将效笠翁之《怜香伴》耶?''芸曰:``然。''自此无日不谈憨园矣。

后憨为有力者夺去,不果。芸竟以之死。

\hypertarget{header-n19}{%
\subsection{闲情记趣}\label{header-n19}}

余忆童稚时,能张目对日,明察秋毫。见藐小微物,必细察其纹理,故时有物外之趣。夏蚊成雷,私拟作群鹤舞空,心之所向,则或千或百果然鹤也。昂首观之,项为之强。又留蚊于素帐中,徐喷以烟,使其冲烟飞鸣,作青云白鹤观,果如鹤唳云端,怡然称快。于土墙凹凸处、花台小草丛杂处,常蹲其身,使与台齐,定神细视,以丛草为林,以虫蚁为兽,以土砾凸者为丘,凹者为堑,神游其中,怡然自得。一日,见二虫斗草间,观之正浓,忽有庞然大物拔山倒树而来,盖一癞蛤蟆也,舌一吐而二虫尽为所吞。余年幼方出神,不觉呀然惊恐,神定,捉蛤蟆,鞭数数十,驱之别院。年长思之,二虫之斗,盖图奸不从也,古语云``奸近杀'',虫亦然耶?贪此生涯,卵为蚯蚓所哈(吴俗称阳曰卵),肿不能便,捉鸭开口哈之,婢妪偶释手,鸭颠其颈作吞噬状,惊而大哭,传为语柄。此皆幼时闲情也。

及长,爱花成癣,喜剪盆树。识张兰坡,始精剪枝养节之法,继悟接花叠石之法。花以兰为最,取其幽香韵致也,而瓣品之稍堪入谱者不可多得。兰坡临终时,赠余荷瓣素心春兰一盆,皆肩平心阔,茎细瓣净,可以入谱者,余珍如拱壁,值余幕游于外,芸能亲为灌溉,花叶颇茂,不二年,一旦忽萎死,起根视之,皆白如玉,且兰芽勃然,初不可解,以为无福消受,浩叹而已,事后始悉有人欲分不允,故用滚汤灌杀也。从此誓不植兰。次取杜鹃,虽无香而色可久玩,且易剪裁。以芸惜枝怜叶,不忍畅剪,故难成树。其他盆玩皆然。

惟每年篱东菊绽,积兴成癖。喜摘插瓶,不爱盆玩。非盆玩不足观,以家无园圃,不能自植,货于市者,俱丛杂无致,故不取耳。其插花朵,数宜单,不宜双,每瓶取一种不取二色,瓶口取阔大不取窄小,阔大者舒展不拘。自五、七花至三、四十花,必于瓶口中一丛怒起,以不散漫、不挤轧、不靠瓶口为妙,所谓``起把宜紧''也。或亭亭玉立,或飞舞横斜。花取参差,间以花蕊,以免飞钹耍盘之病;况取不乱;梗取不强;用针宜藏,针长宁断之,毋令针针露粳,所谓``瓶口宜清''也。视桌之大小,一桌三瓶至七瓶而止,多则眉目不分,即同市井之菊屏矣。几之高低*自三四寸至二尺五六寸而止,必须参差高下互相照应,以气势联络为上,若中高两低,后高前低,成排对列,又犯俗所谓``锦灰堆''矣。或密或疏,或进或出,全在会心者得画意乃可。

若盆碗盘洗,用漂青松香榆皮面和油,先熬以稻灰,收成胶,以铜片按钉向上,将膏火化,粘铜片于盘碗盆洗中。俟冷,将花用铁丝扎把,插于钉上,宜偏斜取势不可居中,更宜枝疏叶清,不可拥挤。然后加水,用碗沙少许掩铜片,使观者疑丛花生于碗底方妙。

若以木本花果插瓶,剪裁之法(不能色色自觅,倩人攀折者每不合意),必先执在手中,横斜以观*势,反侧以取其态;相定之后,剪去杂技,以疏瘦古怪为佳;再思其梗如何入瓶,或折或曲,插入瓶口,方免背叶侧花之患。若一枝到手,先拘定其梗之直者插瓶中,势必枝乱梗强,花侧叶背,既难取态,更无韵致矣。折梗打曲之法,锯其梗之半而嵌以砖石。则直者曲矣,如患梗倒,敲一二钉以菀之。即枫叶竹枝,乱草荆棘,均堪入选。或绿竹一竿配以枸杞数粒,几茎细草伴以荆棘两枝,苟位置得宜,另有世外之趣。若新栽花木,不妨歪斜取势,听其叶侧,一年后枝叶自能向上,如树树直栽,即难取势矣。

至剪裁盆树,先取根露鸡爪者,左右剪成三节,然后起枝。---枝一节,七枝到顶,或九枝到顶。枝忌对节如肩臂,节忌臃肿如鹤膝;须盘旋出枝,不可光留左右,以避赤胸露背之病;又不可前后直出.有名双起三起者,一根而起两三树也。如根无爪形,便成插树,故不取。然一树剪成,至少得三四十年。余生平仅见吾乡万翁名彩章者,一生剪成数树。又在扬州商家见有虞山游客携送黄杨翠柏各一盆,惜乎明珠暗投,余未见其可也。若留枝盘如宝塔,扎枝曲如蚯蚓者,便成匠气矣。

点缀盆中花石,小景可以入画,大景可以入神。一瓯清茗,神能趋入其中,方可供幽斋之玩。种水仙无灵壁石,余尝以炭之有石意者代之。黄芽菜心其白如玉,取大小五七枝,用沙土植长方盘内,以炭代石,黑白分明,颇有意思。以此类推,幽趣无穷,难以枚举。如石葛蒲结子,用冷米汤同嚼喷炭上,置阴湿地,能长细菖蒲,随意移养盆碗中,茸茸可爱。以老蓬子磨薄两头,入蛋壳使鸡翼之,俟雏成取出,用久中燕巢泥加天门冬十分之二,搞烂拌匀,植于小器中,灌以河水,晒以朝阳,花发大如酒杯,缩缩如碗口,亭亭可爱。

若夫园亭楼阁,套室回廊,叠石成山,栽花取势,又在大中见小,小中见大,虚中有实,实中有虚,或藏或露,或浅或深。不仅在``周回曲折''四宇,又不在地广石多徒烦工费。或掘地堆土成山,间以块石,杂以花草,篱用梅编,墙以藤引,则无山而成山矣。大中见小者,散漫处植易长之竹,编易茂之梅以屏之。小中见大者,窄院之墙宜凹凸其形,饰以绿色,引以藤蔓;嵌大石,凿字作碑记形;推窗如临石壁,便觉峻峭无穷。虚中有实者,或山穷水尽处,一折而豁然开朗;或轩阁设厨处,一开而通别院。实中有虚者,开门于不通之院,映以竹石,如有实无也;设矮栏于墙头,如上有月台而实虚也。贫士屋少人多,当仿吾乡太平船后梢之位置,再加转移。其间台级为床,前后借凑,可作三塌,间以板而裱以纸,则前后上下皆越绝,譬之如行长路,即不觉其窄矣。余夫妇乔寓扬州时,曾仿此法,屋仅两椽,上下卧室、厨灶、客座皆越绝而绰然有余。芸曾笑曰:``位置虽精,终非富贵家气象也。''是诚然欤?

余扫墓山中,检有峦纹可观之石,归与芸商曰:``用油灰叠宣州石于白石盆,取色匀也。本山黄石虽古朴,亦用油灰,则黄白相阅,凿痕毕露,将奈何?''芸曰:``择石之顽劣者,捣末于灰痕处,乘湿糁之,干或色同也。''乃如其言,用宜兴窑长方盆叠起一峰:偏于左而凸于右,背作横方纹,如云林石法,廛岩凹凸,若临江石砚状;虚一角,用河泥种千瓣白萍;石上植茑萝,俗呼云松。经营数日乃成。至深秋,茑萝蔓延满山,如藤萝之悬石壁,花开正红色,白萍亦透水大放,红白相间。神游其中,如登蓬岛。置之檐下与芸品题:此处宜设水阁,此处宜立茅亭,此处宜凿六字曰``落花流水之间'',此可以居,此可以钓,此可以眺。胸中丘壑,若将移居者然。一夕,猫奴争食,自檐而堕,连盆与架顷刻碎之。余叹曰:``即此小经营,尚干造物忌耶!''两人不禁泪落。

静室焚香,闲中雅趣。芸尝以沉速等香,于饭镢蒸透,在炉上设一铜丝架,离火中寸许,徐徐烘之,其香幽韵而无烟。佛手忌醉鼻嗅,嗅则易烂;木瓜忌出汗,汗出,用水洗之;惟香圆无忌。佛手、木瓜亦有供法,不能笔宣。每有入将供妥者随手取嗅,随手置之,即不知供法者也。

余闲居,案头瓶花不绝。芸曰:``子之插花能备风晴雨露,可谓精妙入神。而画中有草虫一法,盍仿而效之。''余曰;``虫踯躅不受制,焉能仿效?''芸曰:``有一法,恐作俑罪过耳。''余曰:``试言之。''曰:``虫死色不变,觅螳螂蝉蝶之属,以针刺死,用细丝扣虫项系花草间,整其足,或抱梗,或踏叶,宛然如生,不亦善乎?''余喜,如其法行之,见者无不称绝。求之闺中,今恐未必有此会心者矣。

余与芸寄届锡山华氏,时华夫人以两女从芸识字。乡居院旷,夏日逼人,劳教其家,作活花屏法甚妙。每屏---扇,用木梢二枝约长四五寸作矮条凳式,虚其中,横四挡,宽一尺许,四角凿圆眼,插竹编方眼,屏约高六七尺,用砂盆种扁豆置屏中,盘延屏上,两人可移动。多编数屏,随意遮拦,恍如绿阴满窗,透风蔽日,纡回曲折,随时可更,故曰活花屏,有此一法,即一切藤本香草随地可用。此真乡居之良法也。

友人鲁半舫名璋,字春山,善写松拍及梅菊,工隶书,兼工铁笔。余寄居其家之萧爽楼一年有半。楼共五椽,东向,余后其三.晦明风雨,可以远眺。庭中有木犀一株,清香撩人。有廓有厢,地极幽静。移居时,有一仆一妪,并挈其小女来。仆能成衣,妪能纺绩,于是芸绣、妪绩、仆则成衣,以供薪水.余素爱客,小酌必行令。芸善不费之烹庖,瓜蔬鱼虾,一经芸手,便有意外昧.同人知余贫,每出杖头钱,作竟日叙。余又好洁,地无纤尘,且无拘束,不嫌放纵。时有杨补凡名昌绪,善人物写真;袁少迂名沛,工山水;王星澜名岩,工花卉翎毛,爱萧爽楼幽雅,皆携画具来。余则从之学画,写草篆,镌图章,加以润笔,交芸备茶酒供客,终日品诗论画而已。更有夏淡安、揖山两昆季,并缪山音、知白两昆季,及蒋韵香、陆橘香、周啸霞、郭小愚,华杏帆、张闲酣诸君子,如梁上之燕,自去自来。芸则拔钗沽酒,不动声色,良辰美景,不放轻越。今则天各一方,风流云散,兼之玉碎香埋,不堪回首矣!非所谓``当日浑闲事,而今尽可怜''者乎!

萧爽楼有四忌:谈官宦升迁、公廨时事、八股时文、看牌掷色,有犯必罚酒五厅。有四取:慷慨豪爽、风流蕴藉、落拓不羁、澄静缄默。长夏无事,考对为会,每会八人,每人各携青蚨二百.先拈阄,得第一者为主者,关防别座,第二者为誊录,亦就座,余作举子,各于誊录处取纸一条,盖用印章。主考出五七言各一句,刻香为限,行立构思,不准交头私语,对就后投入一匣,方许就座。各人交卷毕,誊录启匣,并录一册,转呈主考,以杜徇私。十六对中取七言三联,五言三联。六联中取第一者即为后任主考,第二者为誊录,每人有两联不取者罚钱二十文,取一联者免罚十文,过限者倍罚。一场,主考得香钱百文。一日可十场,积钱千文,酒资大畅矣。惟芸议为官卷,准坐而构思。

杨补凡为余夫妇写载花小影,神情确肖。是夜月色颇佳,兰影上粉墙,别有幽致,星澜醉后兴发曰:``补凡能为君写真,我能为花图影。''余笑曰:``花影能如入影否?''星澜取素纸铺于墙,即就兰影,用墨浓淡图之。日间取视,虽不成画,而花叶萧疏,自有月下之趣。芸宝之,各有题咏。

苏城有南园、北园三处,菜花黄时,苦无酒家小饮。携盒而往,对花冷饮,殊无意昧。或议就近觅饮者,或议看花归饮者,终不如对花热饮为快。众议末定。芸笑曰:``明日但各出杖头钱,我自担炉火来。''众笑曰:``诺。''众去,余问曰:``卿果自往乎?''芸曰:``非也,妾见市中卖馄饨者,其担锅、灶无不备,盍雇之而往?妾先烹调端整,到彼处再一下锅,茶酒两便。''余曰:``酒菜固便矣,茶乏烹具。''芸曰:``携一砂罐去,以铁叉串串罐柄,去其锅,悬于行灶中,加柴火煎茶,不亦便乎?''余鼓掌称善。街头有鲍姓者,卖馄饨为业,以百钱雇其担,约以明日午后,鲍欣然允议。明日看花者至,余告以故,众咸叹服。饭后同往,并带席垫至南园,择柳阴下团坐。先烹茗,饮毕,然后暖酒烹肴。是时风和日丽,遍地黄金,青衫红袖,越阡度陌,蝶蜂乱飞,令人不饮自醉。既而酒肴俱熟,坐地大嚼,担者颇不俗,拉与同饮。游人见之莫不羡为奇想。杯盘狼籍,各已陶然,或坐或卧,或歌或啸。红日将颓,余思粥,但者即为买米煮之,果腹而归。芸曰:``今日之游乐乎?''众曰:``非夫人之力不及此。''大笑而散。贫士起居服食以及器皿房舍,宜省俭而雅洁,省俭之法曰``就事论事''。余爱小饮,不喜多菜.芸为置一梅花盒:用二寸白磁

六只,中置一只,外置五只,用灰漆就,其形如梅花,底盖均起凹楞,盖之上有柄如花蒂。置之案头,如一朵墨梅覆桌;启盖视之,如菜装与花瓣中,一盒六色,二三知己可以随意取食,食完再添。另做矮边圆盘一只,以便放杯、箸、酒壶之类,随处可摆,移掇亦便。即食物省俭之一端也。余之小帽、领、袜,皆芸自做。衣之破者,移东补西,必整必洁,色取暗淡,以免垢迹,既可出客,又可家常。此又服饰省俭之一端也。

初至萧爽楼中,嫌其暗,以白纸糊壁,遂亮。夏月楼下去窗,无阑干,觉空洞无遮拦。芸曰:``有旧竹帘在,何不以帘代栏?''余曰:``如何?''芸曰:``用竹数根,黝黑色,一横一竖,留出走路,截半帘搭在横竹上,垂至地,高与桌齐。中竖短竹四根,用麻线扎定,然后于横竹搭帘处,寻旧黑布条,连横竹裹缝之。既可遮栏视观,又不费钱。''此``就事论事''之一法也。以此推之,古人所谓竹头木屑皆有用,良有以也。夏月荷花初开时,晚含而晓放,芸用小纱囊撮茶叶少许,置花心,明早取出,烹天泉水泡之,香韵尤绝。

\hypertarget{header-n21}{%
\subsection{坎坷记愁}\label{header-n21}}

人生坎坷何为乎来哉?往往皆自作孽耳,余则非也,多情重诺,爽直不羁,转因之为累。况吾父稼夫公慷慨豪侠,急人之难、成人之事、嫁人之女、抚人之儿,指不胜屈,挥金如土,多为他人。余夫妇居家,偶有需用,不免典质。始则移东补西,继则左支右决绌。谚云:``处家人情,非钱不行。''先起小人之议,渐招同室之讥。``女子无才便是德'',真千古至言也!余虽居长而行三,故上下呼芸为``三娘''。后忽呼为``三太太'',始而戏呼,继成习惯,甚至尊卑长幼,皆以``三太太''呼之,此家庭之变机欤?

乾隆乙巳,随侍吾父于海宁官舍。芸于吾家书中附寄小函,吾父曰:``媳妇既能笔墨,汝母家信付彼司之。''后家庭偶有闲言,吾母疑其述事不当,仍不令代笔。吾父见信非芸手笔,询余曰:``汝妇病耶?''余即作札问之,亦不答。久之,吾父怒曰:``想汝妇不屑代笔耳!''迨余归,探知委曲,欲为婉剖,芸急止之曰:``宁受责于翁,勿失欢于姑也。''竟不自白。

庚成之春,予又随侍吾父于邗江幕中,有同事俞孚亭者挈眷居焉。吾父谓孚亭曰:``一生辛苦,常在客中,欲觅一起居服役之人而不可得。儿辈果能仰体亲意,当于家乡觅一人来,庶语音相合。''罕亭转述于余,密札致芸,倩媒物色,得姚氏女.芸以成否未定,未即禀知吾母。其来也,托言邻女为嬉游者,及吾父命余接取至署,芸又听旁人意见,托言吾父素所合意者。吾母见之曰:``此邻女之嬉游者也,何娶之乎?''芸遂并失爱于姑矣。

壬子容,余馆真州。吾父病于邗江,余往省,亦病焉。余弟启堂时亦随待。芸来书曰:``启堂弟曾向邻妇借贷,倩芸作保,现追索甚急。''余询启堂,启堂转以嫂氏为多事,余遂批纸尾曰:``父子皆病,无钱可偿,俟启弟归时,自行打算可也。''未几病皆愈,余仍往真州。芸覆书来,吾父拆视之,中述启弟邻项事,且云:``令堂以老人之病留由姚姬而起,翁病稍痊,宜密瞩姚托言思家,妾当令其家父母到扬接取。实彼此卸责之计也。''吾父见书怒甚,询启堂以邻项事,答言不知,遂札饬余曰:``汝妇背夫借债,谗谤小叔,且称姑曰令堂,翁曰老人,悖谬之甚!我已专人持札回苏斥逐,汝若稍有人心,亦当知过!''余接此札,如闻青天霹雳,即肃书认罪,觅骑遄归,恐芸之短见也。到家述其本末,而家人乃持逐书至,历斥多过,言甚决绝。芸泣曰:``妾固不合妄言,但阿翁当恕妇女无知耳。''越数日,吾父又有手谕至,曰:``我不为已甚,汝携妇别居,勿使我见,免我生气足矣。''乃寄芸于外家,而芸以母亡弟出,不愿往依族中,幸友人鲁半舫闻而怜之,招余夫妇往居其家萧爽楼。

越两载,吾父渐知始未,适余自岭南归,吾父自至萧爽楼谓芸曰:``前事我已尽知,汝盍归乎?''余夫妇欣然,仍归故宅,骨肉重圆。岂料又有憨园之孽障耶!

芸素有血疾,以其弟克昌出亡不返。母金氏复念子病没,悲伤过甚所致,自识憨园,年余未发,余方幸其得良药。而憨为有力者夺去,以千金作聘,且许养其母。佳人已属沙叱利矣!余知之而未敢言也,及芸往探始知之,归而呜咽,谓余口:``初不料憨之薄情乃尔也!''余曰:``卿自情痴耳,此中人何情之有哉?况锦衣玉食者,未必能安于荆钗布裙也,雨其后悔,莫若无成。''因抚慰之再三。而芸终以受愚为恨,血疾大发,床席支离,刀圭无效,时发时止,骨瘦形销。不数年而逋负曰增,物议日起,老亲又以盟妓一端,憎恶日甚,余则调停中立。已非生人之境矣。

芸生一女名青君,时年十四,颇知书,且极贤能,质钗典服,幸赖辛劳。子名逢森,时年十二,从师读书。余连年无馆,设一书画铺于家门之内,三日所进,不敷一日所出,焦劳困苦,竭蹶时形。隆冬无裘,挺身而过,青君亦衣中股栗,犹强曰``不寒''。因是芸誓不医药。偶能起床,适余有友人周春煦自福郡王幕中归,倩人绣《心经》一部,芸念绣经可以消灾降福,且利其绣价之丰,竟绣焉。而春煦行色匆匆,不能久待,十日告成,弱者骤劳,致增腰酸头晕之疾。岂知命薄者,佛亦不能发慈悲也!

绣经之后,芸病转增,唤水索汤,上下厌之。有西人赁屋于余画铺之左,放利债为业,时倩余作画,因识之。友人某间渠借五十金,乞余作保,余以情有难却,允焉,而某竟挟资远遁。西人惟保是问,时来饶舌,初以笔墨为抵,渐至无物可偿。岁底吾父家居,西人索债,咆哮于门。吾父闻之,召余诃责曰:``我辈衣冠之家,何得负此小人之债!''正剖诉间,适芸有自幼同盟姊锡山华氏,知其病,遣人问讯。堂上误以为憨园之使,因愈怒曰:``汝妇不守闺训,结盟娼妓;汝亦不思习上,滥伍小人。若置汝死地,情有不忍.姑宽三日限,速自为计,退必首汝逆矣!''

芸闻而泣曰:``亲怒如此,皆我罪孽。妾死君行,君必不忍;妾留君去,君必不舍。姑密唤华家人来,我强起问之。''因令青君扶至房外,呼华使问曰:``汝主母特遗来耶?抑便道来耶?''曰:``主母久闻夫人卧病,本欲亲来探望,因从未登门,不敢造次,临行嘱咐:``倘夫人不嫌乡居简亵,不妨到乡调养,践幼时灯下之言。''盖芸与同绣日,曾有疾病相扶之誓也。因嘱之曰:``烦汝速归,禀知主母,于两日后放舟密来。''

其人既退,谓余曰:``华家盟姊情逾骨肉,君若肯至其家,不妨同行,但儿女携之同往既不便,留之累亲又不可,必于两日内安顿之。''时余有表兄王荩臣一子名韫石,愿得青君为媳妇。芸曰:``闻王郎懦弱无能,不过守成之子,而王又无成可守。幸诗礼之家,且又独子,许之可也。''余谓荩臣曰:``吾父与君有渭阳之谊,欲媳青君,谅无不允。但待长而嫁,势所不能。余夫妇往锡山后,君即禀知堂上,先为童熄;何如?''荩臣喜曰:``谨如命''。逢森亦托友人夏揖山转荐学贸易。

安顿已定,华舟适至,时庚申之腊二十五日也。芸曰:``孑然出门,不惟招邻里笑,且西人之项无着,恐亦不放,必于明日五鼓悄然而去。''余曰:``卿病中能冒晓寒耶?''芸曰;``死生有命,无多虑也。''密禀吾父,办以为然。是夜先将半肩行李挑下船,令逢森先卧。青君泣于母侧,芸嘱曰:``汝母命苦,兼亦情痴,故遭此颠沛,幸汝父待我厚,此去可无他虑。两三年内,必当布置重圆。汝至汝家须尽妇道,勿似汝母。汝之翁姑以得汝为幸,必善视汝。所留箱笼什物,尽付汝带去。汝弟年幼,故未令知,临行时托言就医,数日即归,俟我去远告知其故,禀闻祖父可也。''旁有旧妪,即前卷中曾赁其家消暑者,愿送至乡,故是时陪傍在侧,拭泪不已。将交五鼓,暖粥共啜之。芸强颜笑曰:``昔一粥而聚,今一粥而散,若作传奇,可名《吃粥记》矣。''逢森闻声亦起,呻曰:``母何为?''芸曰:``将出门就医耳。''逢森曰:``起何早?''曰:``路远耳。汝与姊相安在家,毋讨祖母嫌。我与汝父同往,数日即归。''鸡声三唱,芸含泪扶妪,启后门将出,逢森忽大哭曰:``噫,我母不归矣!''青君恐惊人,急掩其口而慰之.当是时,余两人寸肠已断,不能复作一语,但止以``匆哭''而已。青君闭们后,芸出巷十数步,已疲不能行,使妪提灯,余背负之而行。将至舟次,几为逻者所执,幸老妪认芸为病女,余为婿,且得舟子皆华氏工人,闻声接应,相扶下船。解维后,芸始放声痛哭。是行也,其母子已成永诀矣!

华名大成,居无锡之东高山,面山而居,躬耕为业,人极朴诚,其妻夏氏,即芸之盟姊也。是日午未之交,始抵其家。华夫人已倚门而侍,率两笑女至舟,相见甚欢,扶芸登岸,款待殷勤。四邻妇人孺子哄然入室,将芸环视,有相问讯者,有相怜惜者,交头接耳,满室啾啾。芸谓华夫人曰:``今日真如渔父入桃源矣。''华曰:``妹莫笑,乡人少所见多所怪耳。''自此相安度岁。

至元宵,仅隔两旬而芸渐能起步,是夜观龙灯于打麦场中,神情态度渐可复元。余乃心安,与之私议曰:``我居此非计,欲他适而短于资,奈何?''芸曰:``妾亦筹之矣。君姊丈范惠来现于靖江盐公堂司会计,十年前曾借君十金,适数不敷,妾典钗凑之,君忆之耶?''余曰:``忘之矣。''芸曰:``闻靖江去此不远,君盍一往?''余如其言。

时天颇暖,织绒袍哗叽短褂犹觉其热,此辛酉正月十六日也。是夜宿锡山客旅,赁被而卧。晨起趁江阴航船,一路逆风,继以微雨。夜至江阴江口,春寒彻骨,沽酒御寒,囊为之罄。踌躇终夜,拟卸衬衣质钱而渡。十九日北风更烈,雪势犹浓,不禁惨然泪落,暗计房资渡费,不敢再饮。正心寒股栗间,忽见一老翁草鞋毡笠负黄包,入店,以目视余,似相识者。余曰:``翁非泰州曹姓耶?''答曰:``然。我非公,死填沟壑矣!今小女无恙,时诵公德。不意今日相逢,何逗留于此?''盖余幕泰州时有曹姓,本微贱,一女有姿色,已许婿家,有势力者放债谋其女,致涉讼,余从中调护,仍归所许,曹即投入公们为隶,叩首作谢,故识之。余告以投亲遇雪之由,曹曰:``明日天晴,我当顺途相送。''出钱沽酒,备极款洽。二十日晓钟初动,即闻江口唤渡声,余惊起,呼曹同济。曹曰:``勿急,宜饱食登舟。''乃代偿房饭钱,拉余出沽。余以连日逗留,急欲赶渡,食不下咽,强啖麻饼两枚。及登舟,江风如箭,四肢发战。曹曰:``闻江阴有人缢于靖,其妻雇是舟而往,必俟雇者来始渡耳。''枵腹忍寒,午始解缆。至靖,暮烟四合矣。曹曰:``靖有公堂两处,所访者城内耶?城外耶?''余踉跄随其后,且行且对曰:``实不知其内外也。''曹曰:``然则且止宿,明日往访耳。''进旅店,鞋袜已为泥淤湿透,索火烘之,草草饮食,疲极酣睡。晨起,袜烧其半,曹又代偿房饭钱。访至城中,惠来尚未起,闻余至,披衣出,见余状惊曰:``舅何狼狈至此?''余曰:``姑勿问,有银乞借二金,先遣送我者。''惠来以香饼二圆授余,即以赠曹。曹力却,受一圆而去。余乃历述所遭,并言来意。惠来曰:``郎舅至戚,即无宿逋,亦应竭尽绵力,无如航海盐船新被盗,正当盘帐之时,不能挪移丰赠,当勉描番银二十圆以偿旧欠,何如?''余本无奢望,遂诺之.

留住两日,天已晴暖,即作归计。二十五日仍回华宅。芸曰:``君遇雪乎?''余告以所苦。因惨然曰:``雪时,妾以君为抵靖,乃尚逗留江口。幸遇曹老,绝处逢生,亦可谓吉人天相矣。''越数日,得青君信,知逢森已为揖山荐引入店,荩臣请命于吾父,择正月二十四日将伊接去。儿女之事粗能了了,但分离至此,令人终觉惨伤耳。

二月初,日暖风和,以靖江之项薄备行装,访故人胡肯堂于邗江盐署,有贡局众司事公延入局,代司笔墨,身心稍定。至明年壬戌八月,接芸书曰:``病体全廖,惟寄食于非亲非友之家,终觉非久长之策了,愿亦来邗,一睹平山之胜。''余乃赁屋于邗江先春门外,临河两椽,自至华氏接芸同行。华夫人赠一小奚奴曰阿双,帮司炊爨,并订他年结邻之约。

时已十月,平山凄冷,期以春游。满望散心调摄,徐图骨肉重圆。不满月,而贡局司事忽裁十有五人,余系友中之友,遂亦散闲。芸始犹百计代余筹画,强颜慰藉,未尝稍涉怨尤。至癸亥仲春,血疾大发。余欲再至靖江作将伯之呼,芸曰:``求亲不如求友。''余曰:``此言虽是,亲友虽关切,现皆闲处,自顾不遑。''芸曰:``幸天时已暖,前途可无阻雪之虑,愿君速去速回,勿以病人为念。君或体有不安,妾罪更重矣。''时已薪水不继,余佯为雇骡以安其心,实则囊饼徒步,且食且行。向东南,两渡叉河,约八九十里,四望无村落。至更许,但见黄沙漠漠,明星闪闪,得一土地祠,高约五尺许,环以短墙,植以双柏,因向神叩首,祝曰:``苏州沈某投亲失路至此,欲假神祠一宿,幸神怜佑。''于是移小石香炉于旁,以身探之,仅容半体。以风帽反戴掩面,坐半身于中,出膝于外,闭目静听,微风萧萧而已。足疲神倦,昏然睡去。及醒,东方已白,短墙外忽有步语声,急出探视,盖土人赶集经此也。问以途,曰;``南行十里即泰兴县城,穿城向东南十里一土墩,过八墩即靖江,皆康庄也。''余乃反身,移炉于原位,叩首作谢而行。过泰兴,即有小车可附。申刻抵靖。投刺焉。良久,司阍者曰:``范爷因公往常州去矣。''察其辞色,似有推托,余诘之曰:``何日可归?''曰:``不知也。''余曰:``虽一年亦将待之。''阍者会余意,私问曰:``公与范爷嫡郎舅耶?''余曰:``苟非嫡者,不待其归矣。''阍者曰:``公姑待之。''越三日,乃以回靖告,共挪二十五金。

雇骡急返,芸正形容惨变,咻咻涕泣。见余归,卒然曰:``君知昨午阿双卷逃乎?倩人大索,今犹不得。失物小事,人系伊母临行再三交托,今若逃归,中有大江之阻,已觉堪虞,倘其父母匿子图诈,将奈之何?且有何颜见我盟姊?''余曰:``请勿急,卿虑过深矣。匿子图诈,诈其富有也,我夫妇两肩担一口耳,况携来半载,授衣分食,从未稍加扑责,邻里咸知。此实小奴丧良,乘危窃逃。华家盟姊赠以匪人,彼无颜见卿,卿何反谓无颜见彼耶?今当一面呈县立案,以杜后患可也。''芸闻余言,意似稍释。然自此梦中呓语,时呼``阿双逃矣'',或呼``憨何负我'',病势日以增矣。

余欲延医诊治,芸阻曰;``妾病始因弟亡母丧,悲痛过甚,继为情感,后由忿激,而平素又多过虑,满望努力做一好媳妇,而不能得,以至头眩、怔忡诸症毕备,所谓病人膏盲,良医束手,请勿为无益之费。忆妾唱随二十三中,蒙君错爱,百凡体恤,不以顽劣见弃,知己如君,得婿如此,妾已此生无憾!若布衣暖,菜饭饱,一室雍雍,优游泉石,如沧浪亭、萧爽楼之处境,真成烟火神仙矣。神仙几世才能修到,我辈何人,敢望神仙耶?强而求之,致干造物之忌,即有情魔之扰。总因君太多情,妾生薄命耳!''因又呜咽而言曰:``人生百年,终归一死。今中道相离,忽焉长别,不能终奉箕帚、目睹逢森娶妇,此心实觉耿耿。''言已,泪落如豆。余勉强慰之曰:``卿病八年,恹恹欲绝者屡矣,今何忽作断肠语耶?''芸曰:``连日梦我父母放舟来接,闭目即飘然上下,如行云雾中,殆魂离而躯壳存乎?''余曰:``此神不收舍,服以补剂,静心调养,自能安痊。''芸又唏嘘曰:``妾若稍有生机---线,断不敢惊君听闻。今冥路已近,苟再不言,言无日矣.君之不得亲心,流离颠沛,皆由妾故,妾死则亲心自可挽回,君亦可免牵挂。堂上春秋高矣,妾死,君宜早归。如无力携妾骸骨归,不妨暂居于此,待君将来可耳。愿君另续德容兼备者,以奉双亲,抚我遗子,妾亦瞑目矣。''言至此,痛肠欲裂,不觉惨然大恸。余曰:``卿果中道相舍,断无再续之理,况`曾经沧海难为水,除却巫山不是云'耳。''芸乃执余手而更欲有言,仅断续叠言``来世''二宇,忽发喘口噤,两目瞪视,千呼万唤已不能言。痛泪两行,涔涔流溢.既而喘沥微,泪渐干,一灵缥缈,竟尔长逝!时嘉庆癸亥三月三十日也。当是时,孤灯一盏,举目无亲,两手空拳,寸心欲碎。绵绵此恨,曷其有极!

承吾友胡省堂以十金为助,余尽室中所有,变卖一空,亲为成殓。呜呼!芸一女流,具男子之襟怀才识。归吾门后,余日奔走衣食,中馈缺乏,芸能纤悉不介意。及余家居,惟以文字相辩析而已。卒之疾病颠连,赍恨以没,谁致之耶?余有负闺中良友,又何可胜道哉?!奉劝世间夫妇,固不可彼此相仇,亦不可过于情笃。话云``恩爱夫妻不到头'',如余者,可作前车之鉴也。

回煞之期,俗传是日魂必随煞而归,故居中铺设一如生前,且须铺生前旧衣于床上,置旧鞋于床下,以待魂归瞻顾,吴下相传谓之``收眼光''。延羽士作法,先召于床而后遣之,谓之``接眚''。邗江俗例,设酒肴于死者之室。一家尽出,调之``避眚''。以故有因避被窃者。芸娘眚期,房东因同居而出避,邻家嘱余亦设肴远避。众冀魄归一见,姑漫应之。同乡张禹门谏余曰:``因邪入邪,宜信其有,勿尝试也。''余曰:``所以不避而待之者,正信其有也。''张曰:``回煞犯煞不利生人,夫人即或魂归,业已阴阳有间,窃恐欲见者无形可接,应避者反犯其锋耳。''时余痴心不昧,强对曰:``死生有命。君果关切,伴我何如?''张口:``我当于门外守之,君有异见,一呼即入可也。''余乃张灯入室,见铺设宛然而音容已杳,不禁心伤泪涌。又恐泪眼模糊失所欲见,忍泪睁目,坐床而待。抚其所遗旧服,香泽犹存,不觉柔肠寸断,冥然昏去。转念待魂而来,何去遽睡耶?开目四现,见席上双烛青焰荧荧,缩光如豆,毛骨悚然,通体寒栗。因摩两手擦额,细瞩之,双焰渐起,高至尺许,纸裱顶格几被所焚。余正得借光四顾间,光忽又缩如前。此时心舂股栗,欲呼守者进观,而转念柔魂弱魄,恐为盛阳所逼,悄呼芸名而祝之,满室寂然,一无所见,既而烛焰复明,不复腾起矣。出告禹门,服余胆壮,不知余实一时情痴耳。

芸没后,忆和靖``妻梅子鹤''语,自号梅逸。权葬芸于扬州西门外之金桂山,俗呼郝家宝塔。买一棺之地,从遗言寄于此。携木主还乡,吾母亦为悲悼,青君、逢森归来,痛哭成服。启堂进言曰:``严君怒犹未息,兄宜仍往扬州,俟严君归里,婉言劝解,再当专札相招。''余遂拜母别子女,痛哭一场,复至扬州,卖画度日。因得常哭于芸娘之墓,影单形只,备极凄凉,且偶经故居,伤心惨目。重阳日,邻冢皆黄,芸墓独青,守坟者曰:``此好穴场,故地气旺也。''余暗祝曰:``秋风已紧,身尚衣单,卿若有灵,佑我图得一馆,度此残年,以持家乡信息。''未几,江都幕客章驭庵先生欲回浙江葬亲,倩余代庖三月,得备御寒之具。封篆出署,张禹门招寓其家。张亦失馆,度岁艰难,商于余,即以余资二十金倾囊借之,且告曰:``此本留为亡荆扶柩之费,一俟得有乡音,偿我可也。''是年即寓张度岁,晨占夕卜,乡音殊杳。

至甲子三月,接青君信,知吾父有病。即欲归苏,又恐触旧忿。正趑趄观望间,复接青君信,始痛悉吾父业已辞世。刺骨痛心,呼天莫及。无暇他计,即星夜驰归,触首灵前,哀号流血。呜呼!吾父一生辛苦,奔走于外。生余不肖,既少承欢膝下,又未侍药床前,不孝之罪何可逭哉!吾母见余哭,曰:``汝何此日始归耶?''余曰:``儿之归,幸得青君孙女信也。''吾母目余弟妇,遂默然。余入幕守灵至七,终无一人以家事告,以丧事商者。余自问人子之道已缺,故亦无颜询问。

一日,忽有向余索逋者登门饶舌,余出应曰,``欠债不还,固应催索,然吾父骨肉未寒,乘凶追呼,未免太甚。''中有一人私谓余曰:``我等皆有人招之使来,公且避出,当向招我者索偿也。''余曰:``我欠我偿,公等速退!''皆唯唯而去。余因呼启堂谕之曰:``兄虽不肖,并未作恶不端,若言出嗣降服,从未得过纤毫嗣产,此次奔丧归来,本人子之道,岂为产争故耶?大丈夫贵乎自立,我既一身归,仍以一身去耳!''言已,返身入幕,不觉大恸。叩辞吾母,走告青君,行将出走深山,求赤松子于世外矣。

青君正劝阻间,友人夏南熏字淡安、夏逢泰字揖山两昆季寻踪而至,抗声谏余曰:``家庭若此,固堪动忿,但足下父死而母尚存,妻丧而子未立,乃竟飘然出世,于心安乎。''余曰:``然则如之何?''淡安曰:``奉屈暂居寒舍,闻石琢堂殿撰有告假回籍之信,盍俟其归而往谒之?其必有以位置君也。''余曰:``凶丧未满百日,兄等有老亲在堂,恐多未便。''揖山曰:``愚兄弟之相邀,亦家君意也。足下如执以为不便,四邻有禅寺,方丈僧与余交最善,足下设榻于寺中,何如?''余诺之。青君曰:``祖父所遗房产,不下三四千金,既已分毫不取。岂自己行囊亦舍去耶?我往取之,径送禅寺父亲处可也。''因是于行囊之外,转得吾父所遗图书、砚台、笔筒数件。

寺僧安置予于大悲阁。阁南向,向东设神像,隔西首一间,设月窗,紧对佛龛,中为作佛事者斋食之地。余即设榻其中,临门有关圣提刀立像,极威武。院中有银杏一株,大三抱,荫覆满阁,夜静风声如吼。揖山常携酒果来对酌,曰:``足下一人独处,夜深不寐,得无畏怖耶?''余口:``仆一生坦直,胸无秽念,何怖之有?''居未几,大雨倾盆,连宵达旦三十条天,时虑银杏折枝,压梁倾屋。赖神默佑,竟得无恙。而外之墙坍屋倒者不可胜计,近处田禾俱被漂没。余则日与僧人作画,不见不闻。七月初,天始霁,揖山尊人号几莼芗有交易赴崇明,偕余往,代笔书券得二十金。归,值吾父将安葬,启堂命逢森向余曰:``叔因葬事乏用,欲助一二十金。''余拟倾囊与之,揖山不允,分帮其半。余即携青君先至墓所,葬既毕,仍返大悲阁。九月杪,揖山有田在东海永寨沙,又偕余往收其息。盘桓两月,归已残冬,移寓其家雪鸿草堂度岁。真异姓骨肉也。

乙丑七月,琢堂始自都门回籍。琢堂名韫玉,字执如,琢堂其号也,与余为总角交。乾隆庚戌殿元,出为四川重庆守。白莲教之乱,三年戎马,极著劳绩。及归,相见甚欢,旋于重九日挈眷重赴四川重庆之任,邀余同往。余即四别吾母于九妹倩陆尚吾家,盖先君故居已属他人矣。吾母嘱曰``汝弟不足恃,汝行须努力。重振家声,全望汝也!''逢森送余至半途,忽泪落不已,因嘱勿送而返。舟出京口,琢堂有旧交王惕夫孝廉在淮扬盐署,绕道往晤,余与偕往,又得一顾芸娘之墓。返舟由长江溯流而上,一路游览名胜。至湖北之荆州,得升潼关观察之信,遂留余雨其嗣君敦夫眷属等,暂寓荆州,琢堂轻骑减从至重庆度岁,遂由成都历栈道之任。丙寅二月,川眷始由水路往,至樊城登陆。途长费短,车重人多,毙马折轮,备尝辛苦。抵潼关甫三月,琢堂又升山左廉访,清风两袖。眷属不能偕行,暂借潼川书院作寓。十月杪,始支山左廉俸,专人接眷。附有青君之书,骇悉逢森于四月间夭亡。始忆前之送余堕泪者,盖父子永诀也。呜呼!芸仅一子,不得延其嗣续耶!琢堂闻之,亦为之浩叹,赠余一妾,重入春梦。从此扰扰攘攘,又不知梦醒何时耳。

\hypertarget{header-n23}{%
\subsection{浪游记快}\label{header-n23}}

余游幕三十年来,天下所未到者,蜀中、黔中与滇南耳。惜乎轮蹄征逐,处处随人,山水怡情,云烟过眼,不道领略其大概,不能探僻寻幽也。余凡事喜独出己见,不屑随人是非,即论诗品画,莫不存人珍我弃、人弃我取之意,故名胜所在,贵乎心得,有名胜而不觉其佳者,有非名胜面自以为妙者,聊以平生历历者记之。

余年十五时,吾父稼夫公馆于山阴赵明府幕中。有赵省斋先生名传者,杭之宿儒也,赵明府延教其子,吾父命余亦拜投门下。暇日出游,得至吼山,离城约十余里。不通陆路。近山见一石洞,上有片石横裂欲堕,即从其下荡舟入。豁然空其中,四面皆峭壁,俗名之曰``水园''。临流建石阁五椽,对面石壁有``观鱼跃''三宇,水深不测,相传有巨鳞潜伏,余投饵试之,仅见不盈尺者出而唼食焉。阁后有道通旱园,拳石乱矗,有横阔如掌者,有柱石平其顶而上加大石者,凿痕犹在,一无可取。游览既毕,宴于水阁,命从者放爆竹,轰然一响,万山齐应,如闻霹雳生。此幼时快游之始。惜乎兰亭、禹陵未能一到,至今以为憾。

至山阴之明年,先生以亲老不远游,设帐于家,余遂从至杭,西湖之胜因得畅游。结构之妙,予以龙井为最,小有天园次之。石取天竺之飞来峰,城隍山之瑞石古洞。水取玉泉,以水清多鱼,有活泼趣也。大约至不堪者,葛岭之玛瑙寺。其余湖心亭,六一泉诸景,各有妙处,不能尽述,然皆不脱脂粉气,反不如小静室之幽僻,雅近天然。

苏小墓在西泠桥侧。土人指示,初仅半丘黄土而已,乾隆庚子圣驾南巡,曾一询及,甲辰春复举南巡盛典,则苏小墓已石筑其坟,作八角形,上立一碑,大书曰:``钱塘苏小小之墓''。从此吊古骚人不须徘徊探访矣。余思古来烈魄忠魂堙没不传者,固不可胜数,即传而不久者亦不为少,小小一名妓耳,自南齐至今。尽人而知之,此殆灵气所钟,为湖山点缀耶?

桥北数武有祟文书院,余曾与同学赵缉之投考其中。时值长夏,起极早,出钱塘门,过昭庆寺,上断桥,坐石阑上。旭日将升,朝霞映于柳外,尽态极妍;白莲香里,清风徐来,令人心骨皆清。步至书院,题犹未出也。午后交卷。

偕缉之纳凉于紫云洞,大可容数十人,石窍上透日光。有入设短几矮凳,卖酒于此。解衣小酌,尝鹿脯甚妙,佐以鲜菱雪藕,微酣出洞。缉之曰:``上有朝阳台,颇高旷,盍往一游?''余亦兴发,奋勇登其巅,觉西湖如镜,杭城如丸,钱塘江如带,极目可数百里。此生平第一大观也。坐良久,阳乌将落,相携下山,南屏晚钟动矣。韬光、云栖路远未到,其红门局之梅花,姑姑庙之铁树,不过尔尔。紫阳洞予以为必可观,而访寻得之,洞口仅容---指,涓涓流水而已,相传中有洞天,恨不能抉门而入。

清明日,先生春祭扫墓,挈余同游。墓在东岳,是乡多竹,坟丁掘未出土之毛笋,形如梨而尖,作羹供客。余甘之,尽其两碗。先生曰:``噫!是虽味美而克心血,宜多食肉以解之。''余素不贪屠门之嚼,至是饭量且因笋而减,归途觉烦躁,唇舌几裂。过石屋洞,不甚可观。水乐洞峭壁多藤萝,入洞如斗室,有泉流甚急,其声琅琅。池广仅三尺,深五寸许,不溢亦不竭。余俯流就饮,烦躁顿解。洞外二小亭,坐其中可听泉声。衲子请观万年缸。缸在香积厨,形甚巨,以竹引泉灌其内,听其满溢,年久结苔厚尺许,冬日不冰,故不损也。

辛丑秋八月吾父病疟返里,寒索火,热索冰,余谏不听,竟转伤寒,病势日重。余侍奉汤药,昼夜不交睫者几一月。吾妇芸娘亦大病,恹恹在床。心境恶劣,莫可名状。吾父呼余嘱之曰:``我病恐不起,汝守数本书,终非糊口计,我托汝于盟弟蒋思斋,仍继吾业可耳。''越日思斋来,即于榻前命拜为师。未几,得名医徐观莲先生诊治,父病渐痊。芸亦得徐力起床。而余则从此习幕矣。此非快事,何记于此?曰:此抛书浪游之始,故记之。

思斋先生名襄,是年冬,即相随习幕于奉贤宫舍。有同习幕者,顾姓名金鉴,宇鸿干,号紫霞,亦苏州人也。为人慷慨刚毅,直谅不阿,长余一岁,呼之为兄。鸿干即毅然呼余为弟,倾心相交。此余第一知己交也,惜以二十二岁卒,余即落落寡交,今年且四十有六矣,茫茫沧海,不知此生再遇知己如鸿干者否?

忆与鸿干订交,襟怀高旷,时兴山居之想。重九日,余与鸿干俱在苏,有前辈王小侠与吾父稼夫公唤女伶演剧,宴客吾家,余患其扰,先一日约鸿干赴寒山登高,借访他日结庐之地。芸为整理小酒(木盍)。

越日天将晓,鸿干已登门相邀。遂携(木盍)出胥门,入面肆,各饱食。渡胥江,步至横塘枣市桥,雇一叶扁舟,到山日犹未午。舟子颇循良,令其籴米煮饭。余两人上岸,先至中峰寺。寺在支硎古刹之南,循道而上,寺藏深树,山门寂静,地僻僧闲,见余两人不衫不履,不甚接待,余等志不在此,未深入。归舟,饭已熟。饭毕,舟子携(木盍)相随,瞩其子守船,由寒山至高义园之自云精舍。轩临峭壁,飞凿小池,围以石栏,一泓秋水,崖悬薜荔,墙积莓苔。坐轩下,惟闻落叶萧萧,悄无人迹。出门有一亭,嘱舟子坐此相候。余两人从石罅中入,名``一线天'',循级盘旋,直造其巅,曰``上白云'',有庵已坍颓,存一危栈,仅可远眺。小憩片刻,即相扶而下,舟子曰:``登高忘携酒(木盍)矣。''鸿干曰:``我等之游,欲觅偕隐地耳,非专为登高也。''舟子曰:``离此南行二三里,有上沙村,多人家,有隙地,我有表戚范姓居是村,盍往一游?''余喜曰:``此明末徐俟斋先生隐居处也,有园闻极幽雅,从未一游。''于是舟子导往。村在两山夹道中。园依山而无石,老树多极纡回盘郁之势,亭榭窗栏尽从朴素,竹篱茆舍,不愧隐者之居。中有皂荚亭,树大可两抱。余所历园亭,此为第一。园左有山,俗呼鸡笼山,山峰直竖,上加大石,如杭城之瑞石古洞,而不及其玲珑。旁一青石加榻,鸿干卧其上曰:``此处仰观峰岭,俯视园亭,既旷且幽,可以开樽矣。''因拉舟子同饮,或歌或啸,大畅胸怀。土人知余等觅地而来,误以为堪舆,以某处有好风水相告。鸿干曰:``但期合意,不论风水。''(岂意竟成谶语!)酒瓶既罄,各采野菊插满两鬓。

归舟,日已将没。更许抵家,客犹未散。芸私告余曰:``女伶中有兰官者,端庄可取。''余假传母命呼之入内,握其腕而睨之,果丰颐白腻。余顾芸曰:``美则美矣,终嫌名不称实。''芸曰:``肥者有福相。''余曰:``马亏嵬之祸,玉环之福安在?''芸以他辞遣之出。谓余曰:``今日君又大醉耶?''余乃历述所游,芸亦神往者久之。

癸卯春,余从思斋先生就维扬之聘,始见金、焦面目。金山宜远观,焦山宜近视,惜余往来其间未尝登眺。渡江而北,渔洋所谓``绿杨城郭是扬州''一语已活现矣!平山堂离城约三四里,行其途有八九里,虽全是人工,而奇思幻想,点缀天然,即阆苑瑶池、琼楼玉宇,谅不过此。其妙处在十余家之园亭合而为一,联络至山,气势俱贯。其最难位置处,出城入景,有一里许紧沿城郭。夫城缀于旷远重山间,方可入画,园林有此,蠢笨绝伦。而观其或亭或台、或墙或石、或竹或树,半隐半露间,使游人不觉其触目,此非胸有丘壑者断难下手。城尽,以虹园为首折面向北,有石梁曰``虹桥'',不知园以桥名乎?桥以园名乎?荡舟过,曰``长堤春柳'',此景不缀城脚而缀于此,更见布置之妙。再折而西,垒土立庙,曰``小金山'',有此一挡便觉气势紧凑,亦非俗笔。闻此地本沙土,屡筑不成,用木排若干,层叠加土,费数万金乃成,若非商家,乌能如是。过此有胜概楼,年年观竞渡于此。河面较宽,南北跨一莲花桥,桥门通八面,桥面设五亭,扬人呼为``四盘一暖锅'',此思穷力竭之为,不甚可取。桥南有莲心寺,寺中突起喇嘛白塔,金顶缨络,商矗云霄,殿角红墙松柏掩映,钟磬时闻,此天下园亭所未有者。过桥见三层高阁,画栋飞檐,五采绚烂,叠以太湖石,围以白石栏,名目``五云多处'',如作文中间之大结构也。过此名``蜀冈朝阳'',平坦无奇,且属附会。将及山,河面渐束,堆土植竹树,作四五曲。似已山穷水尽,而忽豁然开朗,平山之万松林已列于前矣。``平山堂''为欧阳文忠公所书。所谓淮东第五泉,真者在假山石洞中,不过一井耳,味与天泉同;其荷亭中之六孔铁井栏者,乃系假设,水不堪饮。九峰园另在南门幽静处,别饶天趣,余以为诸园之冠。康山未到,不识如何。此皆言其大概,其工巧处、精美处,不能尽述,大约宜以艳妆美人目之,不可作浣纱溪上观也。余适恭逢南巡盛典,各工告竣,敬演接驾点缀,因得畅其大观,亦人生难遇者也。

甲辰之春,余随待吾父于吴江明府幕中,与山阴章苹江、武林章映牧、苕溪颐蔼泉诸公同事,恭办南斗圩行宫,得第二次瞻仰天颜。一日,天将晚矣,忽动归兴。有办差小快船,双舻两浆,于太湖飞棹疾驰,吴俗呼为``出水辔头'',转瞬已至吴门桥。即跨鹤腾空,无此神爽。抵家,晚餐未熟也。吾乡素尚繁华,至此日之争奇夺胜,较昔尤奢。灯彩眩眸,笙歌聒耳,古人所谓``画栋雕甍''、``珠帘绣幕''、``玉栏干''、``锦步障'',不啻过之。余为友人东拉西扯,助其插花结彩,闲则呼朋引类,剧饮狂歌,畅怀游览,少年豪兴,不倦不疲。苟生于盛世而仍居僻壤,安得此游观哉?

是年,何明府因事被议,吾父即就海宁王明府之聘。嘉兴有刘蕙阶者,长斋佞佛,来拜吾父。其家在烟雨楼侧,一阁临河,曰``水月居'',其涌经处也,洁静如僧舍。烟雨楼在镜湖之中,四岸皆绿杨,惜无多竹。有平台可远眺,渔舟星列,漠漠平波,似宜月夜。衲子备素斋甚佳。至海宁,与白门史心月、山阴俞午桥同事。心月一子名烛衡,澄静缄默,彬彬儒雅,与余莫逆,此生平第二知心交也。惜萍水相逢,聚首无多日耳。游陈氏安澜园,地占百亩,重楼复阁,夹道回廊;池甚广,桥作六曲形;石满藤萝,凿痕全掩;古木千章,皆有参天之势;鸟啼花落,如人深山。此人工而归于天然者。余所历平地之假石园亭,此为第一。曾于桂花楼中张宴,诸味尽为花气所夺,惟酱姜味不变。姜接之性老而愈辣,以喻忠节之臣,洵不虚也。出南门即大海,一日两潮,如万丈银堤破海而过。船有迎潮者,潮至,反棹相向,于船头设一木招,状如长柄大刀,招一捺,潮即分破,船即随招而入,俄顷始浮起,拨转船头随潮而去,顷刻百里。塘上有塔院,中秋夜曾随吾父观潮于此。循塘东约三十里,名尖山,一峰突起,扑入海中,山顶有阁,匾曰``海阔天空'',一望无际,但见怒涛接天而已。

余年二十有五,应徽州绩溪克明府之召,由武林下``江山船'',过富春山,登子陵钓台。台在山腰,一峰突起,离水十余丈。岂汉时之水竞与峰齐耶?月夜泊界口,有巡检署,``山高月小,水落石出'',此景宛然。黄山仅见其脚,惜未一瞻面目。绩溪城处于万山之中,弹丸小邑,民情淳朴。近城有石镜山,由山弯中曲折中里许,悬崖急湍,湿翠欲滴;渐高至山腰,有一方石亭,四面皆陡壁;亭左石削如屏,青色光润,可鉴人形,俗传能照前生。黄巢至此,照为猿猴形,纵火焚之,故不复现。离域十里有火云洞天,石纹盘结,凹凸廛岩,如黄鹤山樵笔意,而杂乱无章,洞石皆深绛色。旁有一庵甚幽静,盐商程虚谷曾招游设宴于此。席中有肉馒头,小沙弥眈眈旁视,授以四枚,临行以番银二圆为酬,山僧不识,推不受。告以一枚可易青钱七百余文,僧以近无易处,仍不受。乃攒凑青蚨六百文付之,始欣然作谢。他日余邀同人携(木盍)再往,老僧嘱曰:``曩者小徒不知食何物而腹泻,今勿再与。''可知藜藿之腹不受肉味,良可叹也。余谓同人曰:``作和尚者,必用此等僻地,终身不见不闻,或可修真养静。若吾乡之虎丘山,终日目所见者妖童艳妓,耳所听者弦索笙歌,鼻所闻者佳肴美酒,安得身如枯木、心如死灰哉?''

又去城三十里,名曰仁里,有花果会,十二年一举,每举各出盆花为赛。余在绩溪适逢其会,欣然欲往,苦无轿马,乃教以断竹为杠,缚椅为轿,雇人肩之而去,同游者惟同事许策廷,见者无不讶笑。至其地,有庙,不知供何神。庙前旷处高搭戏台,画梁方柱极其巍焕,近视则纸扎彩画,抹以油漆者。锣声忽至,四人抬对烛大如断柱,八人抬一猪大若牯牛,盖公养十二年始宰以献神。策廷笑曰:``猪固寿长,神亦齿利。我若为神,乌能享此。''余曰:``亦足见其愚诚也。''入庙,殿廊轩院所设花果盆玩,并不剪枝拗节,尽以苍老古怪为佳,大半皆黄山松。既而开场演剧,人如潮涌而至,余与策廷遂避去。未两载,余与同事不合,拂衣归里。

余自绩溪之游,见热闹场中卑鄙之状不堪入目,因易儒为贾。余有姑丈袁万九,在盘溪之仙人塘作酿酒生涯,余与施心耕附资合伙。袁酒本海贩,不一载,值台湾林爽文之乱,海道阻隔,货积本折,不得已仍为冯妇。馆江北四年,一无快游可记。迨居萧爽楼,正作烟火神仙,有表妹倩徐秀峰自粤东归,见余阅居,慨然曰:``足下待露而爨,笔耕而炊,终非久计,盍偕我作岭南游?当不仅获蝇头利也。''芸亦劝余曰:``乘此老亲尚健,子尚壮年,与其商柴计米而寻欢,不如一劳永逸。''余乃商诸交游者,集资作本。芸会亦自办绣货及岭南所无之苏酒醉蟹等物。禀知堂上,于小春十日,偕秀峰由东坝出芜湖口。

长江初历,大畅襟怀。每晚舟泊后,必小酌船头。见捕鱼者罾幂不满三尺,孔大约有四寸,铁箍四角,似取易沉。余笑曰:``圣人之教虽曰`罟不用数',而如此之大孔小罾,焉能有获?''秀峰曰;``此专为网(鱼便)鱼设也。''见其系以长绠,忽起忽落,似探鱼之有无。末几,急挽出水,已有(鱼便)鱼枷罾孔而起矣。余始喟然曰:``可知一己之见,未可测其奥妙。''一日,见江心中一峰突起,四无依倚。秀峰曰:``此小孤山也。''霜林中,殿阁参差。乘风径过,惜未一游。至滕王阁,犹吾苏府学之尊经阁移于胥门之大马头,王子安序中所云不足信也。即于阁下换高尾昂首船,名``三板子'',由赣关至南安登陆。值余三十诞辰,秀峰备面为寿。越日过大庾岭,出巅一亭,匾曰``举头日近'',言其高也。山头分为二,两边峭壁,中留一道如石巷。口列两碑,一曰``急流勇退'',一曰``得意不可再往''。山顶有梅将军祠,未考为何朝人。所谓岭上梅花,并无一树,意者以梅将军得名梅岭耶?余所带送礼盆梅,至此将交腊月,已花落而叶黄矣。过岭出口,山川风物便觉顿殊。岭西一山,石窍玲珑,已忘其名,舆夫曰:``中有仙人床榻。''匆匆竟过,以未得游为怅。至南雄,雇老龙船,过佛山镇,见人家墙顶多列盆花,叶如冬青,花如牡丹,有大红、粉白、粉红三种,盖山茶花也。

腊月望,始抵省城,寓靖海门内,赁王姓临街楼屋三椽。秀峰货物皆销与当道,余亦随其开单拜客,即有配礼者络绎取货,不旬日而余物已尽。除夕蚊声如雷。岁朝贺节,有棉袍纱套者。不惟气候迥别,即土著人物,同一五官而神情迥异。

正月既望,有署中园乡三友拉余游河观妓,名曰``打水围'',妓名``老举''。于是同出靖海门,下小艇(如剖分之半蛋而加篷焉),先至沙面。妓船名``花艇'',皆对头分排,中留水巷以通小艇往来。每帮约一二十号,横木绑定,以防海风。两船之间钉以木桩,套以藤圈,以便随潮长落。鸨儿呼为``梳头婆'',头用银丝为架,高约四寸许,空其中而蟠发于外,以长耳挖插一朵花于鬓,身披元青短袄,著元青长裤,管拖脚背,腰束汗巾,或红或绿,赤足撒鞋,式如梨园旦脚。登其艇,即躬身笑迎,搴帏入舱。旁列椅杌,中设大炕,一门通艄后。妇呼有客,即闻履声杂沓而出,有挽髻者,有盘辫者,傅粉如粉墙,搽脂如榴火,或红袄绿裤,或绿袄红裤,有著短袜而撮绣花蝴蝶履者,有赤足而套银脚镯者,或蹲于炕,或倚于门,双瞳闪闪,一言不发。余顾秀峰曰:``此何为者也?''秀峰曰:``目成之后,招之始相就耳。''余试招之,果即欢容至前,袖出槟榔为敬。入口大嚼,涩不可耐,急吐之,以纸擦唇,其吐如血。合艇留大笑。又至军工厂,妆束亦相等,惟长幼皆能琵琶而已。与之言,对曰``(口迷)'',``(口迷)''者,``何''也。余曰:```少不入广'者,以其销魂耳,若此野妆蛮语,谁为动心哉?''一友曰:``潮帮妆束如仙,可往一游。''至其帮,排舟亦如沙面。有著名鸨儿素娘者,妆束如花鼓妇。其粉头衣皆长领,颈套项锁,前发齐眉,后发垂肩,中挽一鬏似丫髻,裹足者著裙,不裹足者短袜,亦著蝴蝶履,长拖裤管,语音可辩。而余终嫌为异服,兴趣索然。秀峰曰:``靖海门对渡有扬帮,留吴妆,君往,必有合意者。''一友曰:``所谓扬帮者,仅一鸨儿,呼曰邵寡妇,携一媳日大姑,系来自扬州,余皆湖广江西人也。''因至扬帮。对面两排仅十余艇,其中人物皆云鬟雾鬓,脂粉薄施,阔袖长裙,语音了了,所谓邵寡妇者殷勤相接。遂有一友另唤酒船,大者曰``恒(舟娄)'',小者曰``沙姑艇'',作东道相邀,请余择妓。余择一雏年者,身材状貌有类余妇芸娘,而足极尖细,名喜儿。秀峰唤一统名翠姑。余皆各有旧交。放艇中流,开怀畅饮。至更许,余恐不能自持,坚欲回寓,而城已下钥久矣。盖海疆之城,日落即闭,余不知也。及终席,有卧吃鸦片烟者,有拥妓而调笑者,使头各送衾枕至,行将连床开铺。余暗询喜儿:``汝本艇可卧否?''对曰:``有寮可居,未知有客否也。''(寮者,船顶之楼。)余曰:``姑往探之。''招小艇渡至邵船,但见合帮灯火相对如长廊,寮适无客。鸨儿笑迎曰:``我知今日贵客来,故留寮以相待也。''余笑曰:``姥真荷叶下仙人哉!''遂有使头移烛相引,由舱后梯而登。宛如斗室,旁一长榻,几案俱备。揭帘再进,即在头舱之顶,床亦旁设,中间方窗嵌以玻璃,不火而光满一室,盖对船之灯光也。衾帐镜奁,颇极华美。喜儿曰:``从台可以望月。''即在梯门之上叠开一窗,蛇行而出,即后梢之顶也。三面皆设短栏,一轮明月,水阔天空。纵横如乱叶浮水者,酒船也;闪烁如繁星列天者,酒船之灯也;更有小艇梳织往来,笙歌弦索之声杂以长潮之沸,令人情为之移。余曰:```少不入广',当在斯矣!''惜余妇芸娘不能偕游至此,回顾喜儿,月下依稀相似,因挽之下台,息烛而卧。天将晓,秀峰等已哄然至,余披衣起迎,皆责以昨晚之逃。余曰:``无他,恐公等掀衾揭帐耳!''遂同归寓。

越数日,偕秀峰游海珠寺。寺在水中,围墙若城四周。离水五尺许有洞,设大炮以防海寇,潮长潮落,随水浮沉,不觉炮门之或高或下,亦物理之不可测者。十三洋行在幽兰门之西,结构与洋画同。对渡名花地,花木甚繁,广州卖花处也。余自以为无花不识,至此仅识十之六七,询其名有《群芳谱》所未载者,或土音之不同钦?海珠寺规模极大,山门内植榕树,大可十余抱,阴浓如盖,秋冬不凋。柱槛窗栏皆以铁梨木为之。有菩提树,其叶似柿,浸水去皮,肉筋细如蝉翼纱,可裱小册写经。

归途访喜儿于花艇,适翠、喜二妓俱无客。茶罢欲行,挽留再三。余所属意在寮,而其媳大姑已有酒客在上,因渭邵鸨儿曰:``若可同往寓中,则不妨一叙。''邵曰:``可。''秀峰先归,嘱从者整理酒肴。余携翠、喜至寓。正谈笑间,适郡署王懋老不期来,挽之同饮。酒将沾唇,忽闻楼下人声嘈杂,似有上楼之势,盖房东一侄素无赖,知余招妓,故引人图诈耳。秀蜂怨曰:``此皆三白一时高兴,不合我亦从之。''余曰:``事已至此,应速思退兵之计,非斗口时也。''懋老曰:``我当先下说之。''余即唤仆速雇两轿,先脱两妓,再图出城之策。闻懋老说之不退,亦不上楼。两轿已备,余仆手足颇捷,令其向前开路,秀挽翠姑继之,余挽喜儿于后,一哄而下。秀峰、翠姑得仆力已出门去,喜儿为横手所拿,余急起腿,中其臂,手一松面喜儿脱去,余亦乘势脱身出。余仆犹守于门,以防追抢。急问之曰:``见喜儿否?''仆曰:``翠姑已乘轿去,喜娘但见其出,未见其乘轿也。''余急燃炬,见空轿犹在路旁。急追至靖海门,见秀峰侍翠轿而立,又问之,对曰:``或应投东,而反奔西矣。''急反身,过寓十余家,闻暗处有唤余者,烛之,喜儿也,遂纳之轿,肩而行。秀峰亦奔至,曰:``幽兰门有水窦可出,已托人贿之启钥,翠姑去矣,喜儿速往!''余曰:``君速回寓退兵,翠、喜交我!''至水窦边,果已肩钥,翠先在。余遂左掖喜,右挽翠,折腰鹤步,踉跄出窦。天适微雨,路滑如油,至河干沙面,笙歌正盛。小艇有识翠姑者,招呼登舟。始见喜儿首如飞蓬,钗环俱无有。余曰:``被抢去耶?''喜儿笑曰:``闻此皆赤金,阿母物也,妾于下楼时已除去,藏于囊中。若被抢去,累君赔偿耶。''余闻言,心甚德之,令其重整钗环,勿舍阿母,托言寓所人杂,故仍归舟耳。翠姑如言告母,并曰:``酒菜已饱,备粥可也。''时寮上酒客已去,邵鸨儿命翠亦陪余登寮。见两对绣鞋泥污已透。三人共粥,聊以充饥。剪烛絮谈,始悉翠籍湖南,喜亦豫产,本姓欧阳,父亡母醮,为恶叔所卖。翠姑告以迎新送旧之苦,心不欢必强笑,酒不胜必强饮,身不快必强陪,喉不爽必强歌。更有乖张其性者,稍不合意,即掷酒翻案,大声辱骂,假母不察,反言接待不周,又有恶客彻夜蹂躏,不堪其扰。喜儿年轻初到,母犹惜之。不觉泪随言落。喜儿亦嘿然涕泣。余乃挽喜入杯,抚慰之。瞩翠姑卧于外榻,盖因秀峰交也。

自此或十日或五日,必遣人来招,喜或自放小艇,亲至河干迎接。余每去必邀秀峰,不邀他客,不另放艇。一夕之欢,番银四圆而已。秀峰今翠明红,俗谓之跳槽,甚至一招两妓;余则惟喜儿一人,偶独往,或小酌于平台,或清谈于寮内,不令唱歌,不强多钦,温存体恤,一艇怡然,邻妓皆羡之。有空闲无客者,知余在寮,必来相访。合帮之妓无一不识,每上其艇,呼余声不绝,余亦左顾右盼,应接不暇,此虽挥霍万金所不能致者。余四月在彼处,共费百余金,得尝荔枝鲜果,亦生平快事。后鸨儿欲索五百金强余纳喜,余患其扰,遂图归计。秀峰迷恋于此,因劝其购一妾,仍由原路返吴。明年,秀峰再往,吾父不准偕游,遂就青浦杨明府之聘。及秀峰归,述及喜儿因余不往,几寻短见。噫!``半年一觉扬帮梦,赢得花船薄幸名''矣!

余自粤东归来,馆青浦两载,无快游可述。未几,芸、憨相遇,物议沸腾,芸以激愤致病。余与程墨安设一书画铺于家门之侧,聊佐汤药之需。

中秋后二日,有吴云客偕毛忆香、王屋灿邀余游西山小静室,余适腕底无闲,嘱其先往。吴曰:``子能出城,明午当在山前水踏桥之来鹤庵相候。''余诺之。

越日,留程守铺,余独步出阊门,至山前过水踏桥,循田塍而西。见一庵南向,门带清流,剥琢问之,应曰:``客何来?''余告之。笑曰:``此`得云'也,客不见匾额乎?`来鹤'已过矣!''余曰:``自桥至此,未见有庵。''其人回指曰:``客不见土墙中森森多竹者,即是也。''余乃返至墙下。小门深闭,门隙窥之,短篱曲径,绿竹猗猗,寂不闻人语声,叩之亦无应者。一人过,曰:``墙穴有石,敲门具也。''余试连击,果有小沙弥出应。余即循径入,过小石桥,向西一折,始见山门,悬黑漆额,粉书``来鹤''二字,后有长跋,不暇细观。入门经韦陀殿,上下光洁,纤尘不染,知为好静室。忽见左廊又一小沙弥奉壶出,余大声呼问,即闻室内星灿笑曰:``何如?我谓三白决不失信也!''旋见云客出迎,日:``候君早膳,何来之迟?''一僧继其后,向余稽首,问知为竹逸和尚。入其室,仅小屋三椽,额曰``桂轩'',庭中双桂盛开。星灿、忆香群起嚷曰:``来迟罚三杯!''席上荤素精洁,酒则黄白俱备。余问曰:``公等游几处矣?''云客曰:``昨来已晚,今晨仅到得云、河亭耳。''欢饮良久。饭毕,仍自得云、河亭共游八九处,至华山而止。各有佳处,不能尽述。华山之顶有莲花峰,以时欲暮,期以后游。桂花之盛至此为最,就花下饮清茗---瓯,即乘山舆,径回来鹤。

桂轩之东另有临洁小阁,已杯盘罗列。竹逸寡言静坐而好客善饮。始则折桂催花,继则每人一令,二鼓始罢。余曰:``今夜月色甚佳,即此酣卧,未免有负清光,何处得高旷地,一玩月色,庶不虚此良夜也?''竹逸曰:``放鹤亭可登也。''云客曰:``星灿抱得琴来,未闻绝调,到彼一弹何如?''乃偕往.但见木犀香里,一路霜林,月下长空,万籁俱寂。星灿弹《梅花三弄》,飘飘欲仙。忆香亦兴发,袖出铁笛,呜呜而吹之。云客曰:``今夜石湖看月者,谁能如吾辈之乐裁?''盖吾苏八月十八日石湖行春桥下有看串月胜会,游船排挤,彻夜笙歌,名虽看月,实则挟妓哄饮而已。未几,月落霜寒,兴圃归卧。

明晨,云客谓众曰:``此地有无隐庵,极幽僻,君等有到过者否?''咸对曰:``无论未到,并未尝闻也。''竹逸曰:``无隐四面皆山,其地甚僻,僧不能久居。向年曾一至,已坍废,自尺木彭居士重修后,未尝往焉,今犹依稀识之。如欲往游,请为前导。''忆香曰:``枵腹去耶?''竹逸笑曰:``已备素面矣,再令道人携酒盒相从也。''面毕,步行而往。过高义园,云客欲往白云精舍,入门就坐。一僧徐步出,向云客拱手曰:``违教两月,城中有何新闻?抚军在辕否?''忆香忽起曰:``秃!''拂袖径出。余与星灿忍笑随之,云客、竹逸酬答数语,亦辞出。高义园即范文正公墓,白云精舍在其旁。一轩面壁,上悬藤萝,下凿一潭,广丈许,一泓清碧,有金鳞游泳其中,名曰``钵盂泉''。竹炉茶灶,位置极幽。轩后于万绿丛中,可瞰范园之概。惜衲子俗,不堪久坐耳。是时由上沙村过鸡笼山,即余与鸿干登高处也。风物依然,鸿干已死,不胜今昔之感。正惆怅间,忽流泉阻路不得进,有三五村童掘菌子于乱草中,探头而笑,似讶多人之至此者。询以无隐路,对曰:``前途水大不可行,请返数武,南有小径,度岭可达。''从其言。度岭南行里许,渐觉竹树丛杂,四山环绕,径满绿茵,已无人迹。竹逸徘徊四顾曰:``似在斯,而径不可辨,奈何?''余乃蹲身细瞩,于千竿竹中隐隐见乱石墙舍,径拨丛竹间,横穿入觅之,始得一门,曰``无隐禅院,某年月日南园老人彭某重修'',众喜曰:``非君则武陵源矣!''山门紧闭,敲良久,无应者。忽旁开一门,呀然有声,一鹑衣少年出,面有菜色,足无完履,问曰:``客何为者?''竹逸稽首曰:``慕此幽静,特来瞻仰。''少年曰:``如此穷山,僧散无人接待,请觅他游。''言已,闭门欲进。云客急止之,许以启门放游,必当酬谢。少年笑曰:``茶叶俱无,恐慢客耳,岂望酬耶?''山门一启,即见佛面,金光与绿阴相映,庭阶石础苔积如绣,殿后台级如墙,石栏绕之。循台而西,有石形如馒头,高二丈许,细竹环其趾。再西折北,由斜廊蹑级而登,客堂三卷楹紧对大石。石下凿一小月池,清泉一派,荇藻交横。堂东即正殿,殿左西向为僧房厨灶,殿后临峭壁,树杂阴浓,仰不见天。星灿力疲,就池边小憩,余从之。将启盒小酌,忽闻忆香音在树杪,呼曰:``三白速来,此间有妙境!''仰而视之,不见其人,因与星灿循声觅之。由东厢出一小门,折北,有石蹬如梯,约数十级,于竹坞中瞥见一楼。又梯而上,八窗洞然,额曰``飞云阁''。四山抱列如城,缺西南一角,遥见一水浸天,风帆隐隐,即太湖也。倚窗俯视,风动竹梢,如翻麦浪。忆香曰:``何如?''余曰:``此妙境也。''忽又闻云客于楼西呼曰:``忆香速来,此地更有妙境!''因又下楼,折而西,十余级,忽豁然开朗,平坦如台。度其地,已在殿后峭壁之上,残砖缺础尚存,盖亦昔日之殿基也。周望环山,较阁更畅。忆香对太湖长啸一声,则群山齐应。乃席地开樽,忽愁枵腹,少年欲烹焦饭代茶,随令改茶为粥,邀与同啖。询其何以冷落至此,曰:``四无居邻,夜多暴客,积粮时来强窃,即植蔬果,亦半为樵子所有。此为崇宁寺下院,长厨中月送饭干一石、盐菜一坛而已。某为彭姓裔,暂居看守,行将归去,不久当无人迹矣。''云客谢以番银一圆。

返至来鹤,买舟而归。余绘《无隐图》一幅,以赠竹逸,志快游也。

是年冬,余为友人作中保所累,家庭失欢,寄居锡山华氏。明年春,将之维扬而短于资,有故人韩春泉在上洋幕府,因往访焉。衣敝履穿,不堪入署,投札约晤于郡庙园亭中。及出见,知余愁苦,概助十金。园为洋商捐施而成,极为阔大,惜点缀各景,杂乱无章,后叠山石,亦无起伏照应。归途忽思虞山之胜,适有便舟附之。时当春仲,桃李争研,逆旅行踪,苦无伴侣,乃怀青铜三百,信步至虞山书院。墙外仰瞩,见丛树交花,娇红稚绿,傍水依山,极饶幽趣。惜不得其门而入,问途以往,遇设篷瀹茗者,就之,烹碧罗春,饮之极佳。询虞山何处最胜,一游者曰:``从此出西关,近剑门,亦虞山最佳处也,君欲往,请为前导。''余欣然从之。出西门,循山脚,高低约数里,渐见山峰屹立,石作横纹,至则一山中分,两壁凹凸,高数十仞,近而仰视,势将倾堕。其人曰:``相传上有洞府,多仙景,惜无径可登。''余兴发,挽袖卷衣,猿攀而上,直造其巅。所谓洞府者,深仅丈许,上有石罅,洞然见天。俯首下视,腿软欲堕。乃以腹面壁,依藤附蔓而下。其人叹曰:``壮裁!游兴之豪,未见有如君者。''余口渴思饮,邀其人就野店沽饮三杯。阳乌将落,未得遍游,拾赭石十余块,怀之归寓,负笈搭夜航至苏,仍返锡山。此余愁苦中之快游也。

嘉庆甲子春,痛遭先君之变,行将弃家远遁,友人夏揖山挽留其家。秋八月,邀余同往东海永泰沙勘收花息。沙隶崇明。出刘河口,航海百余里。新涨初辟,尚无街市。茫茫芦荻,绝少人烟,仅有同业丁氏仓库数十椽,四面掘沟河,筑堤栽柳绕于外。丁字实初,家于崇,为一沙之首户;司会计者姓王。俱家爽好客,不拘礼节,与余乍见即同故交。宰猪为饷,倾瓮为饮。令则拇战,不知诗文;歌则号呶,不讲音律。酒酣,挥工人舞拳相扑为戏。蓄牯牛百余头,皆露宿堤上。养鹅为号,以防海盗。日则驱鹰犬猎于芦丛沙渚间,所获多飞禽。余亦从之驰逐,倦则卧。引至园田成熟处,每一字号圈筑高堤,以防潮汛。堤中通有水窦,用闸启闭,旱则长潮时启闸灌之,潦则落潮时开闸泄之。佃人皆散处如列星,一呼俱集,称业户曰``产主'',唯唯听命,朴诚可爱。而激之非义,则野横过于狼虎;幸一言公平,率然拜服。风雨晦明,恍同太古。卧床外瞩即睹洪涛,枕畔潮声如鸣金鼓。一夜,忽见数十里外有红灯大如栲栳,浮于海中,又见红光烛天,势同失火,实初日:``此处起现神灯神火,不久又将涨出沙田矣。''揖山兴致素豪,至此益放。余更肆无忌惮,牛背狂歌,沙头醉舞,随其兴之所至,真生平无拘之快游也。事竣,十月始归。

吾苏虎丘之胜,余取后山之千顷云一处,次则剑池而已,余皆半借人工,且为脂粉所污,已失山林本相。即新起之白公祠、塔影桥,不过留雅名耳。其冶坊滨,余戏改为``野芳滨'',更不过脂乡粉队,徒形其妖冶而已。其在城中最著名之狮子林,虽曰云林手笔,且石质玲珑,中多古木,然以大势观之,竟同乱堆煤渣,积以苔藓,穿以蚁灾,全无山林气势。以余管窥所及,不知其抄。灵岩山,为吴王馆娃宫故址,上有西施洞、响屉廊、采香径诸胜,面其势散漫,旷无收束,不及天平支硎之别饶幽趣。

邓尉山一名元墓,西背太湖,东对锦峰,丹崖翠阁,望如图画,居人种梅为业,花开数十里,一望如积雪,故名``香雪海''。山之左有古柏四树,名之曰``清、奇、古、怪'':清者,一株挺直,茂如翠盖;奇者,卧地三曲,形``之''字;古者,秃顶扁阔,半朽如掌;怪者,体似旋螺,枝干皆然。相传汉以前物也。

乙丑孟春,揖山尊人莼芗先生偕其弟介石,率子侄四人,往幞山家祠春祭,兼扫祖墓,招余同往。顺道先至灵岩山,出虎山桥,由费家河进香雪海现梅。幞山祠宇即藏于香雪海中,时花正盛,咳吐俱香,余曾为介石画《幞山风木国》十二册。是年九月,余从石琢堂殿撰赴四川重庆府之任,溯长江而上,舟抵皖城。皖山之麓,有元季忠臣余公之墓,墓侧有堂三楹,名曰``大观亭'',面临南湖,背倚潜山。亭在山脊,眺远颇畅。旁有深廊,北窗洞开,时值霜时初红,烂如桃李。同游者为蒋寿朋、蔡子琴。南城外又有王氏园,其地长于东西,短于南北,盖北紧背城、南则临湖故也。既限于地,颇难位置,而观其结构,作重台叠馆之法。重台者,屋上作月台为庭院,叠石栽花于上,使游人不知脚下有屋。盖上叠石者则下实,上庭院者则下虚,故花木仍得地气而生也。叠馆者,楼上作轩,轩上再作平台。上下盘折,重叠四层,且有小池,水不漏泄,竟莫测其何虚何实。其立脚全用砖石为之,承重处仿照西洋立柱法。幸面对南湖,目无所阻,骋怀游览,胜于平园。真人工之奇绝者也。

武昌黄鹤楼在黄鹄矶上,后拖黄鹄山,俗呼为蛇山。楼有三层,画栋飞檐,倚城屹峙,面临汉江,与汉阳晴川阁相对。余与琢堂冒雪登焉,俯视长空,琼花飞舞,遥指银山玉树,恍如身在瑶台。江中往来小艇,纵横掀播,如浪卷残叶,名利之心至此一冷。壁间题咏甚多,不能记忆,但记楹对有云:``何时黄鹤重来,且共倒金樽,浇洲渚千年芳草;但见白云飞去,更谁吹玉笛,落江城五月梅花。

黄州赤壁在府城汉川门外,屹立江滨,截然如壁。石皆绛色,故名焉。《水经》渭之赤鼻山,东坡游此作二赋,指为吴魏交兵处,则非也。壁下已成陆地,上有二赋亭。

是年仲冬抵荆州。琢堂得升潼关观察之信,留余住荆州,余以未得见蜀中山水为怅。时琢堂入川,而哲嗣敦夫眷属及蔡子琴、席芝堂俱留于荆州,居刘氏废园。余记其厅额曰``紫藤红树山房''。庭阶围以石栏,凿方池一亩;池中建一亭,有石桥通焉;亭后筑土垒石,杂树丛生;余多旷地,楼阁俱倾颓矣。客中无事,或吟或啸,或出游,或聚谈。岁暮虽资斧不继,而上下雍雍,典衣沽酒,且置锣鼓敲之。每夜必酌,每酌必令。窘则四两烧刀,亦必大施觞政。遇同乡蔡姓者,蔡子琴与叙宗系,乃其族子也,倩其导游名胜。至府学前之曲江楼,昔张九龄为长史时,赋诗其上,朱子亦有诗曰:``相思欲回首,但上曲江楼。''城上又有雄楚搂,五代时高氏所建。规模雄峻,极目可数百里。绕城傍水,尽植垂杨,小舟荡浆往来,颇有画意。荆州府署即关壮缪帅府,仪门内有青石断马槽,相传即赤兔马食槽也。访罗含宅于城西小湖上,不遇。又访宋玉故宅于城北。昔庾信遇侯景之乱,遁归江陵,居宋玉故宅,继改为酒家,今则不可复识矣。

是年大除,雪后极寒,献岁发春,无贺年之扰,日惟燃纸炮、放纸鸢、扎纸灯以为乐。既而风传花信,雨濯春尘,琢堂诸姬携其少女幼子顺川流而下,敦夫乃重整行装,合帮而走。由樊城登陆,直赴潼关。

由山南阌乡县西出函谷关,有``紫气东来''四宇,即老子乘青牛所过之地。两山夹道,仅容二马并行。约十里即潼关,左背峭壁,右临黄河,关在山河之间扼喉而起,重楼垒垛,极其雄峻。而车马寂然,人烟亦稀。昌黎诗曰:``日照潼关四扇开'',殆亦言其冷落耶?

城中观察之下,仅一别驾。道署紧靠北城,后有园圃,横长约三亩。东西凿两池,水从西南墙外而入,东流至两池间,支分三道:一向南至大厨房,以供日用;一向东入东池;一向北折西、由石螭口中喷入西池,绕至西北,设闸泄泻,由城脚转北,穿窦而出,直下黄河。日夜环流,殊清人耳。竹树阴浓,仰不见天。西池中有亭,藕花绕左右。东有面南书室三间,庭有葡萄架,下设方石,可弈可饮,以外皆菊畦。西有面东轩屋三间,坐其中可听流水声。轩南有小门可通内室。轩北窗下另凿小池,池之北有小庙,祀花神。园正中筑三层楼一座,紧靠北城,高与城齐,俯视城外即黄河也。河之北,山如屏列,已属山西界。真洋洋大观也!余居园南,屋如舟式,庭有土山,上有小亭,登之可览园中之概,绿阴四合,夏无暑气。琢堂为余颜其斋曰''不系之舟''。此余幕游以来第一好居室也。土山之间,艺菊数十种,惜未及含葩,而琢堂调山左廉访矣。眷属移寓潼川书院,余亦随往院中居焉。

琢堂先赴任,余与子琴、芝堂等无事,辄出游。乘骑至华阴庙。过华封里,即尧时三祝处。庙内多秦槐汉柏,大皆三四抱,有槐中抱拍而生者,柏中抱槐而生者。殿廷古碑甚多,内有陈希夷书``福''、``寿''字。华山之脚有玉泉院,即希夷先生化形骨蜕处。有石洞如斗室,塑先生卧像于石床。其地水净沙明,草多绛色,泉流甚急,修竹绕之。洞外一方亭,额曰``无忧亭''。旁有古树三栋,纹如裂炭,叶似槐而色深,不知其名,土人即呼曰``无忧树''。太华之高不知几千仞,惜未能裹粮往登焉。归途见林柿正黄,就马上摘食之,土人呼止弗听,嚼之涩甚,急吐去,下骑觅泉漱口,始能言,土人大笑。盖柿须摘下煮一沸,始去其涩,余不知也。

十月初,琢堂自山东专人来接眷属,遂出潼关,由河南入鲁。山东济南府城内,西有大明湖,其中有历下亭、水香亭诸胜。夏月柳阴浓处,菡萏香来,载酒泛舟,极有幽趣。余冬日往视,但见衰柳寒烟,一水茫茫而已。趵突泉为济南七十二泉之冠,泉分三眼,从地底怒涌突起,势如腾沸。凡泉皆从上而下,此独从下而上,亦一奇也。池上有楼,供吕祖像,游者多于此品茶焉。明年二月,余就馆莱阳。至丁卯秋,琢堂降官翰林,余亦入都。所谓登州海市,竟无从一见。

\end{document}
