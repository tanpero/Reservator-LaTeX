\PassOptionsToPackage{unicode=true}{hyperref} % options for packages loaded elsewhere
\PassOptionsToPackage{hyphens}{url}
%
\documentclass[]{article}
\usepackage{lmodern}
\usepackage{amssymb,amsmath}
\usepackage{ifxetex,ifluatex}
\usepackage{fixltx2e} % provides \textsubscript
\ifnum 0\ifxetex 1\fi\ifluatex 1\fi=0 % if pdftex
  \usepackage[T1]{fontenc}
  \usepackage[utf8]{inputenc}
  \usepackage{textcomp} % provides euro and other symbols
\else % if luatex or xelatex
  \usepackage{unicode-math}
  \defaultfontfeatures{Ligatures=TeX,Scale=MatchLowercase}
\fi
% use upquote if available, for straight quotes in verbatim environments
\IfFileExists{upquote.sty}{\usepackage{upquote}}{}
% use microtype if available
\IfFileExists{microtype.sty}{%
\usepackage[]{microtype}
\UseMicrotypeSet[protrusion]{basicmath} % disable protrusion for tt fonts
}{}
\IfFileExists{parskip.sty}{%
\usepackage{parskip}
}{% else
\setlength{\parindent}{0pt}
\setlength{\parskip}{6pt plus 2pt minus 1pt}
}
\usepackage{hyperref}
\hypersetup{
            pdfborder={0 0 0},
            breaklinks=true}
\urlstyle{same}  % don't use monospace font for urls
\setlength{\emergencystretch}{3em}  % prevent overfull lines
\providecommand{\tightlist}{%
  \setlength{\itemsep}{0pt}\setlength{\parskip}{0pt}}
\setcounter{secnumdepth}{0}
% Redefines (sub)paragraphs to behave more like sections
\ifx\paragraph\undefined\else
\let\oldparagraph\paragraph
\renewcommand{\paragraph}[1]{\oldparagraph{#1}\mbox{}}
\fi
\ifx\subparagraph\undefined\else
\let\oldsubparagraph\subparagraph
\renewcommand{\subparagraph}[1]{\oldsubparagraph{#1}\mbox{}}
\fi

% set default figure placement to htbp
\makeatletter
\def\fps@figure{htbp}
\makeatother


\date{}

\begin{document}

\hypertarget{header-n0}{%
\section{礼记}\label{header-n0}}

\begin{center}\rule{0.5\linewidth}{\linethickness}\end{center}

\tableofcontents

\begin{center}\rule{0.5\linewidth}{\linethickness}\end{center}

\hypertarget{header-n7}{%
\subsection{曲礼上}\label{header-n7}}

《曲礼》曰:``毋不敬,俨若思,安定辞。''安民哉!

敖不可长,欲不可从,志不可满,乐不可极。

贤者狎而敬之,畏而爱之。爱而知其恶,憎而知其善。积而能散,安安而能迁。临财毋茍得,临难毋茍免。很毋求胜,分毋求多。疑事毋质,直而勿有。

若夫,坐如尸,立如斋。礼从宜,使从俗。夫礼者所以定亲疏,决嫌疑,别同异,明是非也。礼,不妄说人,不辞费。礼,不逾节,不侵侮,不好狎。修身践言,谓之善行。行修言道,礼之质也。礼闻取于人,不闻取人。礼闻来学,不闻往教。

道德仁义,非礼不成,教训正俗,非礼不备。分争辨讼,非礼不决。君臣上下父子兄弟,非礼不定。宦学事师,非礼不亲。班朝治军,莅官行法,非礼威严不行。祷祠祭祀,供给鬼神,非礼不诚不庄。是以君子恭敬撙节退让以明礼。鹦鹉能言,不离飞鸟;猩猩能言,不离禽兽。今人而无礼,虽能言,不亦禽兽之心乎?夫唯禽兽无礼,故父子聚麀。是故圣人作,为礼以教人。使人以有礼,知自别于禽兽。

太上贵德,其次务施报。礼尚往来。往而不来,非礼也;来而不往,亦非礼也。人有礼则安,无礼则危。故曰:礼者不可不学也。夫礼者,自卑而尊人。虽负贩者,必有尊也,而况富贵乎?富贵而知好礼,则不骄不淫;贫贱而知好礼,则志不慑。

人生十年曰幼,学。二十曰弱,冠。三十曰壮,有室。四十曰强,而仕。五十曰艾,服官政。六十曰耆,指使。七十曰老,而传。八十、九十曰耄,七年曰悼,悼与耄虽有罪,不加刑焉。百年曰期,颐。

大夫七十而致事。若不得谢,则必赐之几杖,行役以妇人。适四方,乘安车。自称曰老夫,于其国则称名;越国而问焉,必告之以其制。

谋于长者,必操几杖以从之。长者问,不辞让而对,非礼也。

凡为人子之礼:冬温而夏清,昏定而晨省,在丑夷不争。

夫为人子者,三赐不及车马。故州闾乡党称其孝也,兄弟亲戚称其慈也,僚友称其弟也,执友称其仁也,交游称其信也。见父之执,不谓之进不敢进,不谓之退不敢退,不问不敢对。此孝子之行也。

夫为人子者:出必告,反必面,所游必有常,所习必有业。恒言不称老。年长以倍则父事之,十年以长则兄事之,五年以长则肩随之。群居五人,则长者必异席。

为人子者,居不主奥,坐不中席,行不中道,立不中门。食飨不为概,祭祀不为尸。听于无声,视于无形。不登高,不临深。不茍訾,不茍笑。

孝子不服暗,不登危,惧辱亲也。父母存,不许友以死。不有私财。

为人子者:父母存,冠衣不纯素。孤子当室,冠衣不纯采。

幼子常视毋诳,童子不衣裘裳。立必正方。不倾听。长者与之提携,则两手奉长者之手。负剑辟咡诏之,则掩口而对。

从于先生,不越路而与人言。遭先生于道,趋而进,正立拱手。先生与之言则对;不与之言则趋而退。

从长者而上丘陵,则必乡长者所视。

登城不指,城上不呼。

将适舍,求毋固。将上堂,声必扬。户外有二屦,言闻则入,言不闻则不入。将入户,视必下。入户奉扃,视瞻毋回;户开亦开,户阖亦阖;有后入者,阖而勿遂。毋践屦,毋踖席,抠衣趋隅。必慎唯诺。

大夫士出入君门,由闑右,不践阈。

凡与客入者,每门让于客。客至于寝门,则主人请入为席,然后出迎客。客固辞,主人肃客而入。主人入门而右,客入门而左。主人就东阶,客就西阶,客若降等,则就主人之阶。主人固辞,然后客复就西阶。主人与客让登,主人先登,客从之,拾级聚足,连步以上。上于东阶则先右足,上于西阶则先左足。

帷薄之外不趋,堂上不趋,执玉不趋。堂上接武,堂下布武。室中不翔,并坐不横肱。授立不跪,授坐不立。

凡为长者粪之礼,必加帚于箕上,以袂拘而退;其尘不及长者,以箕自乡而扱之。奉席如桥衡,请席何乡,请衽何趾。席:南乡北乡,以西方为上;东乡西乡,以南方为上。

若非饮食之客,则布席,席间函丈。主人跪正席,客跪抚席而辞。客彻重席,主人固辞。客践席,乃坐。主人不问,客不先举。将即席,容毋怍。两手抠衣去齐尺。衣毋拨,足毋蹶。

先生书策琴瑟在前,坐而迁之,戒勿越。虚坐尽后,食坐尽前。坐必安,执尔颜。长者不及,毋儳言。正尔容,听必恭。毋剿说,毋雷同。必则古昔,称先王。侍坐于先生:先生问焉,终则对。请业则起,请益则起。父召无诺,先生召无诺,唯而起。侍坐于所尊敬,毋余席。见同等不起。烛至起,食至起,上客起。烛不见跋。尊客之前不叱狗。让食不唾。

侍坐于君子,君子欠伸,撰杖屦,视日蚤莫,侍坐者请出矣。侍坐于君子,君子问更端,则起而对。侍坐于君子,若有告者曰:「少间」,愿有复也;则左右屏而待。毋侧听,毋噭应,毋淫视,毋怠荒。游毋倨,立毋跛,坐毋箕,寝毋伏。敛发毋髢,冠毋免,劳毋袒,暑毋褰裳。

侍坐于长者,屦不上于堂,解屦不敢当阶。就屦,跪而举之,屏于侧。乡长者而屦;跪而迁屦,俯而纳屦。

离坐离立,毋往参焉;离立者,不出中间。

男女不杂坐,不同椸枷,不同巾栉,不亲授。嫂叔不通问,诸母不漱裳。外言不入于捆,内言不出于捆。

女子许嫁,缨;非有大故,不入其门。姑姊妹女子子,已嫁而反,兄弟弗与同席而坐,弗与同器而食。父子不同席。

男女非有行媒,不相知名;非受币,不交不亲。故日月以告君,齐戒以告鬼神,为酒食以召乡党僚友,以厚其别也。

取妻不取同姓;故买妾不知其姓则卜之。寡妇之子,非有见焉,弗与为友。

贺取妻者,曰:「某子使某闻子有客,使某羞。」

贫者不以货财为礼,老者不以筋力为礼。

名子者不以国,不以日月,不以隐疾,不以山川。

男女异长。男子二十,冠而字。父前,子名;君前,臣名。女子许嫁,笄而字。

凡进食之礼,左殽右胾,食居人之左,羹居人之右。脍炙处外,酰酱处内,葱渫处末,酒浆处右。以脯修置者,左朐右末。客若降等执食兴辞,主人兴辞于客,然后客坐。主人延客祭:祭食,祭所先进。殽之序,遍祭之。三饭,主人延客食胾,然后辩殽。主人未辩,客不虚口。

侍食于长者,主人亲馈,则拜而食;主人不亲馈,则不拜而食。

共食不饱,共饭不泽手。毋抟饭,毋放饭,毋流歠,毋咤食,毋啮骨,毋反鱼肉,毋投与狗骨。毋固获,毋扬饭。饭黍毋以箸。毋嚃羹,毋絮羹,毋刺齿,毋歠醢。客絮羹,主人辞不能亨。客歠醢,主人辞以窭。濡肉齿决,干肉不齿决。毋嘬炙。

卒食,客自前跪,彻饭齐以授相者,主人兴辞于客,然后客坐。侍饮于长者,酒进则起,拜受于尊所。长者辞,少者反席而饮。长者举未釂,少者不敢饮。长者赐,少者、贱者不敢辞。赐果于君前,其有核者怀其核。御食于君,君赐余,器之溉者不写,其余皆写。

馂余不祭。父不祭子,夫不祭妻。御同于长者,虽贰不辞,偶坐不辞。羹之有菜者用梜,其无菜者不用梜。

为天子削瓜者副之,巾以絺。为国君者华之,巾以绤。为大夫累之,士疐之,庶人龁之。

父母有疾,冠者不栉,行不翔,言不惰,琴瑟不御,食肉不至变味,饮酒不至变貌,笑不至矧,怒不至詈。疾止复故。

有忧者侧席而坐,有丧者专席而坐。

水潦降,不献鱼鳖,献鸟者拂其首,畜鸟者则勿拂也。献车马者执策绥,献甲者执胄,献杖者执末。献民虏者操右袂。献粟者执右契,献米者操量鼓。献孰食者操酱齐。献田宅者操书致。

凡遗人弓者:张弓尚筋,弛弓尚角。右手执箫,左手承弣。尊卑垂帨。若主人拜,则客还辟,辟拜。主人自受,由客之左接下承弣;乡与客并,然后受。进剑者左首。进戈者前其鐏,后其刃。进矛戟者前其镦。

进几杖者拂之。效马效羊者右牵之;效犬者左牵之。执禽者左首。饰羔雁者以缋。受珠玉者以掬。受弓剑者以袂。饮玉爵者弗挥。凡以弓剑、苞苴、箪笥问人者,操以受命,如使之容。

凡为君使者,已受命,君言不宿于家。君言至,则主人出拜君言之辱;使者归,则必拜送于门外。若使人于君所,则必朝服而命之;使者反,则必下堂而受命。

博闻强识而让,敦善行而不怠,谓之君子。君子不尽人之欢,不竭人之忠,以全交也。

《礼》曰:「君子抱孙不抱子。」此言孙可以为王父尸,子不可以为父尸。为君尸者,大夫士见之,则下之。君知所以为尸者,则自下之,尸必式。乘必以几。

齐者不乐不吊。

居丧之礼,毁瘠不形,视听不衰。升降不由阼阶,出入不当门隧。居丧之礼,头有创则沐,身有疡则浴,有疾则饮酒食肉,疾止复初。不胜丧,乃比于不慈不孝。五十不致毁,六十不毁,七十唯衰麻在身,饮酒食肉,处于内。生与来日,死与往日。知生者吊,知死者伤。知生而不知死,吊而不伤;知死而不知生,伤而不吊。吊丧弗能赙,不问其所费。问疾弗能遗,不问其所欲。见人弗能馆,不问其所舍。赐人者不曰来取。与人者不问其所欲。适墓不登垄,助葬必执绋。临丧不笑。揖人必违其位。望柩不歌。入临不翔。当食不叹。邻有丧,舂不相。里有殡,不巷歌。适墓不歌。哭日不歌。送丧不由径,送葬不辟涂潦。临丧则必有哀色,执绋不笑,临乐不叹;介胄,则有不可犯之色。

故君子戒慎,不失色于人。国君抚式,大夫下之。大夫抚式,士下之。礼不下庶人,刑不上大夫。刑人不在君侧。

兵车不式。武车绥旌,德车结旌。史载笔,士载言。前有水,则载青旌。前有尘埃,则载鸣鸢。前有车骑,则载飞鸿。前有士师,则载虎皮。前有挚兽,则载貔貅。行:前朱鸟而后玄武,左青龙而右白虎。招摇在上,急缮其怒。进退有度,左右有局,各司其局。

父之雠,弗与共戴天。兄弟之雠不反兵。交游之雠不同国。四郊多垒,此卿大夫之辱也。地广大,荒而不治,此亦士之辱也。

临祭不惰。祭服敝则焚之,祭器敝则埋之,龟策敝则埋之,牲死则埋之。凡祭于公者,必自彻其俎。

卒哭乃讳。礼,不讳嫌名。二名不偏讳。逮事父母,则讳王父母;不逮事父母,则不讳王父母。君所无私讳,大夫之所有公讳。《诗》、《书》不讳,临文不讳。庙中不讳。夫人之讳,虽质君之前,臣不讳也;妇讳不出门。大功小功不讳。入竟而问禁,入国而问俗,入门而问讳。

外事以刚日,内事以柔日。

凡卜筮日:旬之外曰远某日,旬之内曰近某日。丧事先远日,吉事先近日。曰:「为日,假尔泰龟有常,假尔泰筮有常。」

卜筮不过三,卜筮不相袭。龟为卜,策为筮,卜筮者,先圣王之所以使民信时日、敬鬼神、畏法令也;所以使民决嫌疑、定犹与也。故曰:「疑而筮之,则弗非也;日而行事,则必践之。」

君车将驾,则仆执策立于马前。已驾,仆展軨、效驾,奋衣由右上取贰绥,跪乘,执策分辔,驱之五步而立。君出就车,则仆并辔授绥。左右攘辟,车驱而驺。至于大门,君抚仆之手而顾,命车右就车;门闾沟渠,必步。

凡仆人之礼,必授人绥。若仆者降等,则受;不然,则否。若仆者降等,则抚仆之手;不然,则自下拘之。客车不入大门。妇人不立乘。犬马不上于堂。故君子式黄髪,下卿位,入国不驰,入里必式。

君命召,虽贱人,大夫士必自御之。介者不拜,为其拜而蓌拜。祥车旷左,乘君之乘车不敢旷左;左必式。仆御、妇人则进左手,后右手;御国君,则进右手、后左手而俯。国君不乘奇车。车上不广咳,不妄指。立视五巂,式视马尾,顾不过毂。国中以策彗恤勿驱。尘不出轨。国君下齐牛,式宗庙。大夫士下公门,式路马。乘路马,必朝服载鞭策,不敢授绥,左必式。步路马,必中道。以足蹙路马刍,有诛。齿路马,有诛。

\hypertarget{header-n74}{%
\subsection{曲礼下}\label{header-n74}}

凡奉者当心,提者当带。

执天子之器则上衡,国君则平衡,大夫则绥之,士则提之。

凡执主器,执轻如不克。执主器,操币圭璧,则尚左手,行不举足,车轮曳踵。立则磬折垂佩。主佩倚,则臣佩垂。主佩垂,则臣佩委。执玉,其有藉者则裼;无藉者则袭。

国君不名卿老世妇,大夫不名世臣侄娣,士不名家相长妾。君大夫之子,不敢自称曰``余小子'';大夫士之子,不敢自称曰``嗣子某'',不敢与世子同名。君使士射,不能,则辞以疾;言曰:``某有负薪之忧。''侍于君子,不顾望而对,非礼也。

君子行礼,不求变俗。祭祀之礼,居丧之服,哭泣之位,皆如其国之故,谨修其法而审行之。去国三世,爵禄有列于朝,出入有诏于国,若兄弟宗族犹存,则反告于宗后;去国三世,爵禄无列于朝,出入无诏于国,唯兴之日,从新国之法。君子已孤不更名。已孤暴贵,不为父作谥。居丧,未葬,读丧礼;既葬,读祭礼;丧复常,读乐章。

居丧不言乐,祭事不言凶,公庭不言妇女。

振书、端书于君前,有诛。倒策侧龟于君前,有诛。龟策、几杖、席盖、重素、袗絺绤,不入公门。苞屦、扱衽、厌冠,不入公门。书方、衰、凶器,不以告,不入公门。

公事不私议。

君子将营宫室:宗庙为先,厩库为次,居室为后。凡家造:祭器为先,牺赋为次,养器为后。无田禄者不设祭器;有田禄者,先为祭服。君子虽贫,不粥祭器;虽寒,不衣祭服;为宫室,不斩于丘木。大夫、士去国,祭器不逾竟。大夫寓祭器于大夫,士寓祭器于士。

大夫、士去国:逾竟,为坛位乡国而哭。素衣,素裳,素冠,彻缘,鞮屦,素幂,乘髦马。不蚤鬋。不祭食,不说人以无罪;妇人不当御。三月而复服。

大夫、士见于国君,君若劳之,则还辟,再拜稽首;君若迎拜,则还辟,不敢答拜。大夫、士相见,虽贵贱不敌,主人敬客,则先拜客;客敬主人,则先拜主人。凡非吊丧、非见国君,无不答拜者。

大夫见于国君,国君拜其辱。士见于大夫,大夫拜其辱。同国始相见,主人拜其辱。君于士,不答拜也;非其臣,则答拜之。大夫于其臣,虽贱,必答拜之。

男女相答拜也。

国君春田不围泽;大夫不掩群,士不取麑卵。

岁凶,年谷不登,君膳不祭肺,马不食谷,驰道不除,祭事不县。大夫不食粱,士饮酒不乐。

君无故,玉不去身;大夫无故不彻县,士无故不彻琴瑟。士有献于国君,他日,君问之曰:``安取彼?''再拜稽首而后对。

大夫私行出疆,必请。反,必有献。士私行出疆,必请;反,必告。君劳之,则拜;问其行,拜而后对。国君去其国,止之曰:``奈何去社稷也!''大夫,曰:``奈何去宗庙也!''士,曰:``奈何去坟墓也!''国君死社稷,大夫死众,士死制。

君天下,曰天子。朝诸侯,分职授政任功,曰予一人。践阼临祭祀:内事曰孝王某,外事曰嗣王某。临诸侯,畛于鬼神,曰有天王某甫。崩,曰天王崩。复,曰天子复矣。告丧,曰天王登假。措之庙,立之主,曰帝。天子未除丧,曰予小子。生名之,死亦名之。

天子有后,有夫人,有世妇,有嫔,有妻,有妾。天子建天官,先六大:曰大宰、大宗、大史、大祝、大士、大卜,典司六典。

天子之五官:曰司徒、司马、司空、司士、司寇,典司五众。

天子之六府:曰司土、司木、司水、司草、司器、司货,典司六职。

天子之六工:曰土工、金工、石工、木工、兽工、草工,典制六材。五官致贡,曰享。

五官之长,曰伯:是职方。其摈于天子也,曰天子之吏。天子同姓,谓之伯父;异姓,谓之伯舅。自称于诸侯,曰天子之老,于外曰公;于其国曰君。

九州岛之长入天子之国,曰牧。天子同姓,谓之叔父;异姓,谓之叔舅;于外曰侯,于其国曰君。其在东夷、北狄、西戎、南蛮,虽大,曰子。于内自称曰不谷,于外自称曰王老。庶方小侯入天子之国,曰某人,于外曰子,自称曰孤。天子当依而立,诸侯北面而见天子,曰觐。天子当宁而立,诸公东面、诸侯西面,曰朝。

诸侯未及期相见曰遇,相见于却地曰会。诸侯使大夫问于诸侯曰聘,约信曰誓,莅牲曰盟。

诸侯见天子曰臣某、侯某;其与民言,自称曰寡人;其在凶服,曰适子孤。临祭祀,内事曰孝子某侯某,外事曰曾孙某侯某。死曰薨,复曰某甫复矣。既葬见天子曰类见。言谥曰类。

诸侯使人使于诸侯,使者自称曰寡君之老。天子穆穆,诸侯皇皇,大夫济济,士跄跄,庶人僬僬。

天子之妃曰后,诸侯曰夫人,大夫曰孺人,士曰妇人,庶人曰妻。

公侯有夫人,有世妇,有妻,有妾。夫人自称于天子,曰老妇;自称于诸侯,曰寡小君;自称于其君,曰小童。自世妇以下,自称曰婢子。子于父母则自名也。

列国之大夫,入天子之国曰某士;自称曰陪臣某。于外曰子,于其国曰寡君之老。使者自称曰某。天子不言出,诸侯不生名。

君子不亲恶。诸侯失地,名;灭同姓,名。

为人臣之礼:不显谏。三谏而不听,则逃之。

子之事亲也:三谏而不听,则号泣而随之。君有疾,饮药,臣先尝之。亲有疾,饮药,子先尝之。

医不三世,不服其药。

儗人必于其伦。

问天子之年,对曰:``闻之:始服衣若干尺矣。''问国君之年:长,曰能从宗庙社稷之事矣;幼,曰未能从宗庙社稷之事也。问大夫之子:长,曰能御矣;幼,曰未能御也。问士之子:长,曰能典谒矣;幼,曰未能典谒也。问庶人之子:长,曰能负薪矣;幼,曰未能负薪也。

问国君之富,数地以对,山泽之所出。问大夫之富,曰有宰食力,祭器衣服不假。问士之富,以车数对。问庶人之富,数畜以对。

天子祭天地,祭四方,祭山川,祭五祀,岁遍。诸侯方祀,祭山川,祭五祀,岁遍。大夫祭五祀,岁遍。士祭其先。

凡祭,有其废之莫敢举也,有其举之莫敢废也。非其所祭而祭之,名曰淫祀。淫祀无福。天子以牺牛,诸侯以肥牛,大夫以索牛,士以羊豕。支子不祭,祭必告于宗子。

凡祭宗庙之礼:牛曰一元大武,豕曰刚鬣,豚曰腯肥,羊曰柔毛,鸡曰翰音,犬曰羹献,雉曰疏趾,兔曰明视,脯曰尹祭,槁鱼曰商祭,鲜鱼曰脡祭,水曰清涤,酒曰清酌,黍曰芗合,粱曰芗萁,稷曰明粢,稻曰嘉蔬,韭曰丰本,盐曰咸鹾,玉曰嘉玉,币曰量币。

天子死曰崩,诸侯曰薨,大夫曰卒,士曰不禄,庶人曰死。在床曰尸,在棺曰柩。羽鸟曰降,四足曰渍。死寇曰兵。

祭王父曰皇祖考,王母曰皇祖妣。父曰皇考,母曰皇妣。夫曰皇辟。生曰父、曰母、曰妻,死曰考、曰妣、曰嫔。

寿考曰卒,短折曰不禄。

天子视不上于袷,不下于带;国君,绥视;大夫,衡视;士视五步。凡视:上于面则敖,下于带则忧,倾则奸。

君命,大夫与士肄。在官言官,在府言府,在库言库,在朝言朝。

朝言不及犬马。辍朝而顾,不有异事,必有异虑。故辍朝而顾,君子谓之固。

在朝言礼,问礼对以礼。大飨不问卜,不饶富。

凡挚,天子鬯,诸侯圭,卿羔,大夫雁,士雉,庶人之挚匹;童子委挚而退。野外军中无挚,以缨,拾,矢,可也。

妇人之挚,椇榛、脯修、枣栗。

纳女于天子,曰备百姓;于国君,曰备酒浆;于大夫,曰备扫洒。

\hypertarget{header-n127}{%
\subsection{檀弓上}\label{header-n127}}

公仪仲子之丧,檀弓免焉。仲子舍其孙而立其子,檀弓曰:``何居?我未之前闻也。''趋而就子服伯子于门右,曰:``仲子舍其孙而立其子,何也?''伯子曰:``仲子亦犹行古之道也。昔者文王舍伯邑考而立武王,微子舍其孙腯而立衍也;夫仲子亦犹行古之道也。''

子游问诸孔子,孔子曰:``否!立孙。''

事亲有隐而无犯,左右就养无方,服勤至死,致丧三年。事君有犯而无隐,左右就养有方,服勤至死,方丧三年。事师无犯无隐,左右就养无方,服勤至死,心丧三年。

季武子成寝,杜氏之葬在西阶之下,请合葬焉,许之。入宫而不敢哭。武子曰:``合葬非古也,自周公以来,未之有改也。吾许其大而不许其细,何居?''命之哭。

子上之母死而不丧。门人问诸子思曰:``昔者子之先君子丧出母乎?''曰:``然''。``子之不使白也丧之。何也?''子思曰:``昔者吾先君子无所失道;道隆则从而隆,道污则从而污。汲则安能?为汲也妻者,是为白也母;不为汲也妻者,是不为白也母。''故孔氏之不丧出母,自子思始也。

孔子曰:``拜而后稽颡,颓乎其顺也;稽颡而后拜,颀乎其至也。三年之丧,吾从其至者。''

孔子既得合葬于防,曰:``吾闻之:古也墓而不坟;今丘也,东西南北人也,不可以弗识也。''于是封之,崇四尺。

孔子先反,门人后,雨甚;至,孔子问焉曰:``尔来何迟也?''曰:``防墓崩。''孔子不应。三,孔子泫然流涕曰:``吾闻之:古不修墓。''

孔子哭子路于中庭。有人吊者,而夫子拜之。既哭,进使者而问故。使者曰:``醢之矣。''遂命覆醢。

曾子曰:``朋友之墓,有宿草而不哭焉。''

子思曰:``丧三日而殡,凡附于身者,必诚必信,勿之有悔焉耳矣。三月而葬,凡附于棺者,必诚必信,勿之有悔焉耳矣。丧三年以为极,亡则弗之忘矣。故君子有终身之忧,而无一朝之患。故忌日不乐。''

孔子少孤,不知其墓。殡于五父之衢。人之见之者,皆以为葬也。其慎也,盖殡也。

问于郰曼父之母,然后得合葬于防。邻有丧,舂不相;里有殡,不巷歌。丧冠不緌。

有虞氏瓦棺,夏后氏墍周,殷人棺椁,周人墙置翣。周人以殷人之棺椁葬长殇,以夏后氏之墍周葬中殇、下殇,以有虞氏之瓦棺葬无服之殇。

夏后氏尚黑;大事敛用昏,戎事乘骊,牲用玄。殷人尚白;大事敛用日中,戎事乘翰,牲用白。周人尚赤;大事敛用日出,戎事乘騵,牲用骍。

穆公之母卒,使人问于曾子曰:``如之何?''对曰:``申也闻诸申之父曰:哭泣之哀、齐斩之情、饘粥之食,自天子达。布幕,卫也;縿幕,鲁也。''

晋献公将杀其世子申生,公子重耳谓之曰:``子盖言子之志于公乎?''世子曰:``不可,君安骊姬,是我伤公之心也。''曰:``然则盖行乎?''世子曰:``不可,君谓我欲弒君也,天下岂有无父之国哉!吾何行如之?''使人辞于狐突曰:``申生有罪,不念伯氏之言也,以至于死,申生不敢爱其死;虽然,吾君老矣,子少,国家多难,伯氏不出而图吾君,伯氏茍出而图吾君,申生受赐而死。''再拜稽首,乃卒。是以为``恭世子''也。

鲁人有朝祥而莫歌者,子路笑之。夫子曰:``由,尔责于人,终无已夫?三年之丧,亦已久矣夫。''子路出,夫子曰:``又多乎哉!逾月则其善也。''

鲁庄公及宋人战于乘丘。县贲父御,卜国为右。马惊,败绩,公队。佐车授绥。公曰:``末之卜也。''县贲父曰:``他日不败绩,而今败绩,是无勇也。''遂死之。圉人浴马,有流矢在白肉。公曰:``非其罪也。''遂诔之。士之有诔,自此始也。

曾子寝疾,病。乐正子春坐于床下,曾元、曾申坐于足,童子隅坐而执烛。童子曰:``华而睆,大夫之箦与?''子春曰:``止!''曾子闻之,瞿然曰:``呼!''曰:``华而睆,大夫之箦与?''曾子曰:``然,斯季孙之赐也,我未之能易也。元,起易箦。''曾元曰:``夫子之病革矣,不可以变,幸而至于旦,请敬易之。''曾子曰:``尔之爱我也不如彼。君子之爱人也以德,细人之爱人也以姑息。吾何求哉?吾得正而毙焉斯已矣。''举扶而易之。反席未安而没。

始死,充充如有穷;既殡,瞿瞿如有求而弗得;既葬,皇皇如有望而弗至。练而慨然,祥而廓然。邾娄复之以矢,盖自战于升陉始也。鲁妇人之髽而吊也,自败于台鲐始也。

南宫绛之妻之姑之丧,夫子诲之髽曰:``尔毋从从尔,尔毋扈扈尔。盖榛以为笄,长尺,而总八寸。''

孟献子禫,县而不乐,比御而不入。夫子曰:``献子加于人一等矣!''

孔子既祥,五日弹琴而不成声,十日而成笙歌。

有子盖既祥而丝屦组缨。

死而不吊者三:畏、厌、溺。

子路有姊之丧,可以除之矣,而弗除也,孔子曰:``何弗除也?''子路曰:``吾寡兄弟而弗忍也。''孔子曰:``先王制礼,行道之人皆弗忍也。''子路闻之,遂除之。

大公封于营丘,比及五世,皆反葬于周。君子曰:``乐乐其所自生,礼不忘其本。古之人有言曰:狐死正丘首。仁也。''

伯鱼之母死,期而犹哭。夫子闻之曰:``谁与哭者?''门人曰:``鲤也。''夫子曰:``嘻!其甚也。''伯鱼闻之,遂除之。

舜葬于苍梧之野,盖三妃未之从也。季武子曰:``周公盖祔。''

曾子之丧,浴于爨室。大功废业。或曰:``大功,诵可也。''

子张病,召申祥而语之曰:``君子曰终,小人曰死;吾今日其庶几乎!''曾子曰:``始死之奠,其余阁也与?''曾子曰:``小功不为位也者,是委巷之礼也。子思之哭嫂也为位,妇人倡踊;申祥之哭言思也亦然。''

古者,冠缩缝,今也,衡缝;故丧冠之反吉,非古也。曾子谓子思曰:``汲!吾执亲之丧也,水浆不入于口者七日。''子思曰:``先王之制礼也,过之者俯而就之,不至焉者,跂而及之。故君子之执亲之丧也,水浆不入于口者三日,杖而后能起。''曾子曰:``小功不税,则是远兄弟终无服也,而可乎?''

伯高之丧,孔氏之使者未至,冉子摄束帛、乘马而将之。孔子曰:``异哉!徒使我不诚于伯高。''

伯高死于卫,赴于孔子,孔子曰:``吾恶乎哭诸?兄弟,吾哭诸庙;父之友,吾哭诸庙门之外;师,吾哭诸寝;朋友,吾哭诸寝门之外;所知,吾哭诸野。于野,则已疏;于寝,则已重。夫由赐也见我,吾哭诸赐氏。''遂命子贡为之主,曰:``为尔哭也来者,拜之;知伯高而来者,勿拜也。''

曾子曰:``丧有疾,食肉饮酒,必有草木之滋焉。以为姜桂之谓也。''

子夏丧其子而丧其明。曾子吊之曰:``吾闻之也:朋友丧明则哭之。''曾子哭,子夏亦哭,曰:``天乎!予之无罪也。''曾子怒曰:``商,女何无罪也?吾与女事夫子于洙泗之间,退而老于西河之上,使西河之民疑女于夫子,尔罪一也;丧尔亲,使民未有闻焉,尔罪二也;丧尔子,丧尔明,尔罪三也。而曰女何无罪与!''子夏投其杖而拜曰:``吾过矣!吾过矣!吾离群而索居,亦已久矣。''

夫昼居于内,问其疾可也;夜居于外,吊之可也。是故君子非有大故,不宿于外;非致齐也、非疾也,不昼夜居于内。

高子皋之执亲之丧也,泣血三年,未尝见齿,君子以为难。

衰,与其不当物也,宁无衰。齐衰不以边坐,大功不以服勤。

孔子之卫,遇旧馆人之丧,入而哭之哀。出,使子贡说骖而赙之。子贡曰:``于门人之丧,未有所说骖,说骖于旧馆,无乃已重乎?''夫子曰:``予乡者入而哭之,遇于一哀而出涕。予恶夫涕之无从也。小子行之。''孔子在卫,有送葬者,而夫子观之,曰:``善哉为丧乎!足以为法矣,小子识之。''子贡曰:``夫子何善尔也?''曰:``其往也如慕,其反也如疑。''子贡曰:``岂若速反而虞乎?''子曰:``小子识之,我未之能行也。''颜渊之丧,馈祥肉,孔子出受之,入,弹琴而后食之。

孔子与门人立,拱而尚右,二三子亦皆尚右。孔子曰:``二三子之嗜学也,我则有姊之丧故也。''二三子皆尚左。

孔子蚤作,负手曳杖,消摇于门,歌曰:``泰山其颓乎?梁木其坏乎?哲人其萎乎?''既歌而入,当户而坐。子贡闻之曰:``泰山其颓,则吾将安仰?梁木其坏、哲人其萎,则吾将安放?夫子殆将病也。''遂趋而入。夫子曰:``赐!尔来何迟也?夏后氏殡于东阶之上,则犹在阼也;殷人殡于两楹之间,则与宾主夹之也;周人殡于西阶之上,则犹宾之也。而丘也殷人也。予畴昔之夜,梦坐奠于两楹之间。夫明王不兴,而天下其孰能宗予?予殆将死也。''盖寝疾七日而没。

孔子之丧,门人疑所服。子贡曰:``昔者夫子之丧颜渊,若丧子而无服;丧子路亦然。请丧夫子,若丧父而无服。''

孔子之丧,公西赤为志焉:饰棺、墙,置翣设披,周也;设崇,殷也;绸练设旐,夏也。

子张之丧,公明仪为志焉;褚幕丹质,蚁结于四隅,殷士也。

子夏问于孔子曰:``居父母之仇如之何?''夫子曰:``寝苫枕干,不仕,弗与共天下也;遇诸市朝,不反兵而斗。''曰:``请问居昆弟之仇如之何?''曰:``仕弗与共国;衔君命而使,虽遇之不斗。''曰:``请问居从父昆弟之仇如之何?''曰:``不为魁,主人能,则执兵而陪其后。''

孔子之丧,二三子皆绖而出。群居则绖,出则否。

易墓,非古也。

子路曰:``吾闻诸夫子:丧礼,与其哀不足而礼有余也,不若礼不足而哀有余也。祭礼,与其敬不足而礼有余也,不若礼不足而敬有余也。''

曾子吊于负夏,主人既祖,填池,推柩而反之,降妇人而后行礼。从者曰:``礼与?''曾子曰:``夫祖者且也;且,胡为其不可以反宿也?''从者又问诸子游曰:``礼与?''子游曰:``饭于牖下,小敛于户内,大敛于阼,殡于客位,祖于庭,葬于墓,所以即远也。故丧事有进而无退。''曾子闻之曰:``多矣乎,予出祖者。''曾子袭裘而吊,子游裼裘而吊。曾子指子游而示人曰:``夫夫也,为习于礼者,如之何其裼裘而吊也?''主人既小敛、袒、括发;子游趋而出,袭裘带绖而入。曾子曰:``我过矣,我过矣,夫夫是也。''

子夏既除丧而见,予之琴,和之不和,弹之而不成声。作而曰:``哀未忘也。先王制礼,而弗敢过也。''

子张既除丧而见,予之琴,和之而和,弹之而成声,作而曰:``先王制礼不敢不至焉。''

司寇惠子之丧,子游为之麻衰牡麻绖,文子辞曰:``子辱与弥牟之弟游,又辱为之服,敢辞。''子游曰:``礼也。''文子退反哭,子游趋而就诸臣之位,文子又辞曰:``子辱与弥牟之弟游,又辱为之服,又辱临其丧,敢辞。''子游曰:``固以请。''文子退,扶适子南面而立曰:``子辱与弥牟之弟游,又辱为之服,又辱临其丧,虎也敢不复位。''子游趋而就客位。将军文子之丧,既除丧,而后越人来吊,主人深衣练冠,待于庙,垂涕洟,子游观之曰:``将军文氏之子其庶几乎!亡于礼者之礼也,其动也中。''

幼名,冠字,五十以伯仲,死谥,周道也。绖也者实也。掘中溜而浴,毁灶以缀足;及葬,毁宗躐行,出于大门-─殷道也。学者行之。

子柳之母死,子硕请具。子柳曰:``何以哉?''子硕曰:``请粥庶弟之母。''子柳曰:``如之何其粥人之母以葬其母也?不可。''既葬,子硕欲以赙布之余具祭器。子柳曰:``不可,吾闻之也:君子不家于丧。请班诸兄弟之贫者。''君子曰:``谋人之军师,败则死之;谋人之邦邑,危则亡之。''公叔文子升于瑕丘,蘧伯玉从。文子曰:``乐哉斯丘也,死则我欲葬焉。''蘧伯玉曰:``吾子乐之,则瑗请前。''

弁人有其母死而孺子泣者,孔子曰:``哀则哀矣,而难为继也。夫礼,为可传也,为可继也。故哭踊有节。''

叔孙武叔之母死,既小敛,举者出户,出户袒,且投其冠括发。子游曰:``知礼。''扶君,卜人师扶右,射人师扶左;君薨以是举。

从母之夫,舅之妻,二夫人相为服,君子未之言也。或曰同爨缌。

丧事,欲其纵纵尔;吉事,欲其折折尔。故丧事虽遽,不陵节;吉事虽止,不怠。故骚骚尔则野,鼎鼎尔则小人。

君子盖犹犹尔。丧具,君子耻具,一日二日而可为也者,君子弗为也。丧服,兄弟之子犹子也,盖引而进之也;嫂叔之无服也,盖推而远之也;姑姊妹之薄也,盖有受我而厚之者也。食于有丧者之侧,未尝饱也。

曾子与客立于门侧,其徒趋而出。曾子曰:``尔将何之?''曰:``吾父死,将出哭于巷。''曰:``反,哭于尔次。''曾子北面而吊焉。

孔子曰:``之死而致死之,不仁而不可为也;之死而致生之,不知而不可为也。是故,竹不成用,瓦不成味,木不成斫,琴瑟张而不平,竽笙备而不和,有钟磬而无簨虡,其曰明器,神明之也。''

有子问于曾子曰:``问丧于夫子乎?''曰:``闻之矣:丧欲速贫,死欲速朽。''有子曰:``是非君子之言也。''曾子曰:``参也闻诸夫子也。''有子又曰:``是非君子之言也。''曾子曰:``参也与子游闻之。''有子曰:``然,然则夫子有为言之也。''曾子以斯言告于子游。子游曰:``甚哉,有子之言似夫子也。昔者夫子居于宋,见桓司马自为石椁,三年而不成。夫子曰:『若是其靡也,死不如速朽之愈也。』死之欲速朽,为桓司马言之也。南宫敬叔反,必载宝而朝。夫子曰:『若是其货也,丧不如速贫之愈也。』丧之欲速贫,为敬叔言之也。''曾子以子游之言告于有子,有子曰:``然,吾固曰:非夫子之言也。''曾子曰:``子何以知之?''有子曰:``夫子制于中都,四寸之棺,五寸之椁,以斯知不欲速朽也。昔者夫子失鲁司寇,将之荆,盖先之以子夏,又申之以冉有,以斯知不欲速贫也。''

陈庄子死,赴于鲁,鲁人欲勿哭,缪公召县子而问焉。县子曰:``古之大夫,束修之问不出竟,虽欲哭之,安得而哭之?今之大夫,交政于中国,虽欲勿哭,焉得而弗哭?且且臣闻之,哭有二道:有爱而哭之,有畏而哭之。''公曰:``然,然则如之何而可?''县子曰:``请哭诸异姓之庙。''于是与哭诸县氏。仲宪言于曾子曰:``夏后氏用明器,示民无知也;殷人用祭器,示民有知也;周人兼用之,示民疑也。''曾子曰:``其不然乎!其不然乎!夫明器,鬼器也;祭器,人器也;夫古之人,胡为而死其亲乎?''公叔木有同母异父之昆弟死,问于子游。子游曰:``其大功乎?''狄仪有同母异父之昆弟死,问于子夏,子夏曰:``我未之前闻也;鲁人则为之齐衰。''狄仪行齐衰。今之齐衰,狄仪之问也。

子思之母死于卫,柳若谓子思曰:``子,圣人之后也,四方于子乎观礼,子盖慎诸。''子思曰:``吾何慎哉?吾闻之:有其礼,无其财,君子弗行也;有其礼,有其财,无其时,君子弗行也。吾何慎哉!''

县子琐曰:``吾闻之:古者不降,上下各以其亲。滕伯文为孟虎齐衰,其叔父也;为孟皮齐衰,其叔父也。''后木曰:``丧,吾闻诸县子曰:夫丧,不可不深长思也,买棺外内易,我死则亦然。''曾子曰:``尸未设饰,故帷堂,小敛而彻帷。''仲梁子曰:``夫妇方乱,故帷堂,小敛而彻帷。''小敛之奠,子游曰:``于东方。''曾子曰:``于西方,敛斯席矣。''小敛之奠在西方,鲁礼之末失也。县子曰:``绤衰繐裳,非古也。''子蒲卒,哭者呼灭。子皋曰:``若是野哉。''哭者改之。

杜桥之母之丧,宫中无相,以为沽也。夫子曰:``始死,羔裘玄冠者,易之而已。''羔裘玄冠,夫子不以吊。

子游问丧具,夫子曰:``称家之有亡。''子游曰:``有亡恶乎齐?''夫子曰:``有,毋过礼;茍亡矣,敛首足形,还葬,县棺而封,人岂有非之者哉!''司士贲告于子游曰:``请袭于床。''子游曰:``诺。''县子闻之曰:``汰哉叔氏!专以礼许人。''

宋襄公葬其夫人,酰醢百瓮。曾子曰:``既曰明器矣,而又实之。''孟献子之丧,司徒旅归四布。夫子曰:``可也。''读赗,曾子曰:``非古也,是再告也。''成子高寝疾,庆遗入,请曰:``子之病革矣,如至乎大病,则如之何?''子高曰:``吾闻之也:生有益于人,死不害于人。吾纵生无益于人,吾可以死害于人乎哉?我死,则择不食之地而葬我焉。''

子夏问诸夫子曰:``居君之母与妻之丧。''``居处、言语、饮食衎尔。''

宾客至,无所馆。夫子曰:``生于我乎馆,死于我乎殡。''国子高曰:``葬也者,藏也;藏也者,欲人之弗得见也。是故,衣足以饰身,棺周于衣,椁周于棺,土周于椁;反壤树之哉。''

孔子之丧,有自燕来观者,舍于子夏氏。子夏曰:``圣人之葬人与?人之葬圣人也。子何观焉?昔者夫子言之曰:『吾见封之若堂者矣,见若坊者矣,见若覆夏屋者矣,见若斧者矣。』从若斧者焉。马鬣封之谓也。今一日而三斩板,而已封,尚行夫子之志乎哉!''妇人不葛带。有荐新,如朔奠。既葬,各以其服除。池视重溜。君即位而为椑,岁一漆之,藏焉。复、楔齿、缀足、饭、设饰、帷堂并作。父兄命赴者。君复于小寝、大寝,小祖、大祖,库门、四郊。丧不剥,奠也与?祭肉也与?既殡,旬而布材与明器。朝奠日出,夕奠逮日。父母之丧,哭无时,使必知其反也。练,练衣黄里、縓缘,葛要绖,绳屦无絇,角瑱,鹿裘衡长袪,袪裼之可也。

有殡,闻远兄弟之丧,虽缌必往;非兄弟,虽邻不往。所识其兄弟不同居者皆吊。天子之棺四重;水兕革棺被之,其厚三寸,杝棺一,梓棺二,四者皆周。棺束缩二衡三,衽每束一。伯椁以端长六尺。

天子之哭诸侯也,爵弁绖缁衣;或曰:使有司哭之,为之不以乐食。天子之殡也,菆涂龙輴以椁,加斧于椁上,毕涂屋,天子之礼也。

唯天子之丧,有别姓而哭。鲁哀公诔孔丘曰:``天不遗耆老,莫相予位焉,呜呼哀哉!尼父!''国亡大县邑,公、卿、大夫、士皆厌冠,哭于大庙,三日,君不举。或曰:君举而哭于后土。孔子恶野哭者。未仕者,不敢税人;如税人,则以父兄之命。士备入而后朝夕踊。祥而缟,是月禫,徙月乐。君于士有赐帟。

\hypertarget{header-n206}{%
\subsection{檀弓下}\label{header-n206}}

君之适长殇,车三乘;公之庶长殇,车一乘;大夫之适长殇,车一乘。

公之丧,诸达官之长,杖。

君于大夫,将葬,吊于宫;及出,命引之,三步则止。如是者三,君退;朝亦如之,哀次亦如之。五十无车者,不越疆而吊人。

季武子寝疾,蟜固不说齐衰而入见,曰:``斯道也,将亡矣;士唯公门说齐衰。''武子曰:``不亦善乎,君子表微。''及其丧也,曾点倚其门而歌。

大夫吊,当事而至,则辞焉。吊于人,是日不乐。妇人不越疆而吊人。行吊之日不饮酒食肉焉。吊于葬者必执引,若从柩及圹,皆执绋。丧,公吊之,必有拜者,虽朋友州里舍人可也。吊曰:``寡君承事。''主人曰:``临。''君遇柩于路,必使人吊之。大夫之丧,庶子不受吊。妻之昆弟为父后者死,哭之适室,子为主,袒免哭踊,夫入门右,使人立于门外告来者,狎则入哭;父在,哭于妻之室;非为父后者。哭诸异室。有殡,闻远兄弟之丧,哭于侧室;无侧室,哭于门内之右;同国,则往哭之。

子张死,曾子有母之丧;齐衰而往哭之。或曰:``齐衰不以吊。''曾子曰:``我吊也与哉?''有若之丧,悼公吊焉,子游摈,由左。齐谷王姬之丧,鲁庄公为之大功。或曰:``由鲁嫁,故为之服姊妹之服。''或曰:``外祖母也,故为之服。''

晋献公之丧,秦穆公使人吊公子重耳,且曰:``寡人闻之:亡国恒于斯,得国恒于斯。虽吾子俨然在忧服之中,丧亦不可久也,时亦不可失也。孺子其图之。''以告舅犯,舅犯曰:``孺子其辞焉;丧人无宝,仁亲以为宝。父死之谓何?又因以为利,而天下其孰能说之?孺子其辞焉。''公子重耳对客曰:``君惠吊亡臣重耳,身丧父死,不得与于哭泣之哀,以为君忧。父死之谓何?或敢有他志,以辱君义。''稽颡而不拜,哭而起,起而不私。子显以致命于穆公。穆公曰:``仁夫公子重耳!夫稽颡而不拜,则未为后也,故不成拜;哭而起,则爱父也;起而不私,则远利也。''

帷殡,非古也,自敬姜之哭穆伯始也。丧礼,哀戚之至也。节哀,顺变也;君子念始之者也。复,尽爱之道也,有祷祠之心焉;望反诸幽,求诸鬼神之道也;北面,求诸幽之义也。拜稽颡,哀戚之至隐也;稽颡,隐之甚也。饭用米贝,弗忍虚也;不以食道,用美焉尔。铭,明旌也,以死者为不可别已,故以其旗识之。爱之,斯录之矣;敬之,斯尽其道焉耳。重,主道也,殷主缀重焉;周主重彻焉。奠以素器,以生者有哀素之心也;唯祭祀之礼,主人自尽焉尔;岂知神之所飨,亦以主人有齐敬之心也。辟踊,哀之至也,有算,为之节文也。袒、括发,变也;愠,哀之变也。去饰,去美也;袒、括发,去饰之甚也。有所袒、有所袭,哀之节也。弁绖葛而葬,与神交之道也,有敬心焉。周人弁而葬,殷人冔而葬。歠主人、主妇室老,为其病也,君命食之也。反哭升堂,反诸其所作也;主妇入于室,反诸其所养也。反哭之吊也,哀之至也-\/-反而亡焉,失之矣,于是为甚。殷既封而吊,周反哭而吊。

孔子曰:``殷已悫,吾从周。''葬于北方北首,三代之达礼也,之幽之故也。既封,主人赠,而祝宿虞尸。既反哭,主人与有司视虞牲,有司以几筵舍奠于墓左,反,日中而虞。葬日虞,弗忍一日离也。是月也,以虞易奠。卒哭曰成事,是日也,以吉祭易丧祭,明日,祔于祖父。其变而之吉祭也,比至于祔,必于是日也接-\/-不忍一日末有所归也。殷练而祔,周卒哭而祔。孔子善殷。君临臣丧,以巫祝桃茢执戈-\/-恶之也;所以异于生也。丧有死之道焉。先王之所难言也。丧之朝也,顺死者之孝心也,其哀离其室也,故至于祖考之庙而后行。殷朝而殡于祖,周朝而遂葬。

孔子谓:为明器者,知丧道矣,备物而不可用也。哀哉!死者而用生者之器也。不殆于用殉乎哉。其曰明器,神明之也。涂车刍灵,自古有之,明器之道也。孔子谓为刍灵者善,谓为俑者不仁-\/-殆于用人乎哉!

穆公问于子思曰:``为旧君反服,古与?''子思曰:``古之君子,进人以礼,退人以礼,故有旧君反服之礼也;今之君子,进人若将加诸膝,退人若将队诸渊,毋为戎首,不亦善乎!又何反服之礼之有?''

悼公之丧,季昭子问于孟敬子曰:``为君何食?''敬子曰:``食粥,天下之达礼也。吾三臣者之不能居公室也,四方莫不闻矣,勉而为瘠则吾能,毋乃使人疑夫不以情居瘠者乎哉?我则食食。''

卫司徒敬子死,子夏吊焉,主人未小敛,绖而往。子游吊焉,主人既小敛,子游出,绖反哭,子夏曰:``闻之也与?''曰:``闻诸夫子,主人未改服,则不绖。''

曾子曰:``晏子可谓知礼也已,恭敬之有焉。''有若曰:``晏子一狐裘三十年,遣车一乘,及墓而反;国君七个,遣车七乘;大夫五个,遣车五乘,晏子焉知礼?''曾子曰:``国无道,君子耻盈礼焉。国奢,则示之以俭;国俭,则示之以礼。''

国昭子之母死,问于子张曰:``葬及墓,男子、妇人安位?''子张曰:``司徒敬子之丧,夫子相,男子西乡,妇人东乡。''曰:``噫!毋。''曰:``我丧也斯沾。尔专之,宾为宾焉,主为主焉-\/-妇人从男子皆西乡。''

穆伯之丧,敬姜昼哭;文伯之丧,昼夜哭。孔子曰:``知礼矣。''

文伯之丧,敬姜据其床而不哭,曰:``昔者吾有斯子也,吾以将为贤人也,吾未尝以就公室;今及其死也,朋友诸臣未有出涕者,而内人皆行哭失声。斯子也,必多旷于礼矣夫!''季康子之母死,陈亵衣。敬姜曰:``妇人不饰,不敢见舅姑,将有四方之宾来,亵衣何为陈于斯?''命彻之。

有子与子游立,见孺子慕者,有子谓子游曰:``予壹不知夫丧之踊也,予欲去之久矣。情在于斯,其是也夫?''子游曰:``礼:有微情者,有以故兴物者;有直情而径行者,戎狄之道也。礼道则不然,人喜则斯陶,陶斯咏,咏斯犹,犹斯舞,舞斯愠,愠斯戚,戚斯叹,叹斯辟,辟斯踊矣。品节斯,斯之谓礼。人死,斯恶之矣,无能也,斯倍之矣。是故制绞衾、设蒌翣,为使人勿恶也。始死,脯醢之奠;将行,遣而行之;既葬而食之,未有见其飨之者也。自上世以来,未之有舍也,为使人勿倍也。故子之所刺于礼者,亦非礼之訾也。''

吴侵陈,斩祀杀厉,师还出竟,陈大宰嚭使于师。夫差谓行人仪曰:``是夫也多言,盍尝问焉;师必有名,人之称斯师也者,则谓之何?''大宰嚭曰:``古之侵伐者,不斩祀、不杀厉、不获二毛;今斯师也,杀厉与?其不谓之杀厉之师与?''曰:``反尔地,归尔子,则谓之何?''曰:``君王讨敝邑之罪,又矜而赦之,师与,有无名乎?''

颜丁善居丧:始死,皇皇焉如有求而弗得;及殡,望望焉如有从而弗及;既葬,慨焉如不及其反而息。

子张问曰:``《书》云:『高宗三年不言,言乃欢。』有诸?''仲尼曰:``胡为其不然也?古者天子崩,王世子听于冢宰三年。''

知悼子卒,未葬;平公饮酒,师旷、李调侍,鼓钟。杜蒉自外来,闻钟声,曰:``安在?''曰:``在寝。''杜蒉入寝,历阶而升,酌,曰:``旷饮斯。''又酌,曰:``调饮斯。''又酌,堂上北面坐饮之。降,趋而出。平公呼而进之曰:``蒉,曩者尔心或开予,是以不与尔言;尔饮旷何也?''曰:``子卯不乐;知悼子在堂,斯其为子卯也大矣。旷也大师也,不以诏,是以饮之也。''``尔饮调何也?''曰:``调也君之亵臣也,为一饮一食,忘君之疾,是以饮之也。''``尔饮何也?''曰:``蒉也宰夫也,非刀匕是共,又敢与知防,是以饮之也。''平公曰:``寡人亦有过焉,酌而饮寡人。''杜蒉洗而扬觯。公谓侍者曰:``如我死,则必无废斯爵也。''至于今,既毕献,斯扬觯,谓之杜举。

公叔文子卒,其子戍请谥于君曰:``日月有时,将葬矣。请所以易其名者。''君曰:``昔者卫国凶饥,夫子为粥与国之饿者,是不亦惠乎?昔者卫国有难,夫子以其死卫寡人,不亦贞乎?夫子听卫国之政,修其班制,以与四邻交,卫国之社稷不辱,不亦文乎?故谓夫子『贞惠文子』。''

石骀仲卒,无适子,有庶子六人,卜所以为后者。曰:``沐浴、佩玉则兆。''五人者皆沐浴、佩玉;石祁子曰:``孰有执亲之丧而沐浴、佩玉者乎?''不沐浴、佩玉。石祁子兆。

卫人以龟为有知也。陈子车死于卫,其妻与其家大夫谋以殉葬,定,而后陈子亢至,以告曰:``夫子疾,莫养于下,请以殉葬。''子亢曰:``以殉葬,非礼也;虽然,则彼疾当养者,孰若妻与宰?得已,则吾欲已;不得已,则吾欲以二子者之为之也。''于是弗果用。

子路曰:``伤哉贫也!生无以为养,死无以为礼也。''孔子曰:``啜菽饮水尽其欢,斯之谓孝;敛首足形,还葬而无椁,称其财,斯之谓礼。''

卫献公出奔,反于卫,及郊,将班邑于从者而后入。柳庄曰:``如皆守社稷,则孰执羁靮而从;如皆从,则孰守社稷?君反其国而有私也,毋乃不可乎?''弗果班。

卫有大史曰柳庄,寝疾。公曰:``若疾革,虽当祭必告。''公再拜稽首,请于尸曰:``有臣柳庄也者,非寡人之臣,社稷之臣也,闻之死,请往。''不释服而往,遂以襚之。与之邑裘氏与县潘氏,书而纳诸棺,曰:``世世万子孙,无变也。''

陈干昔寝疾,属其兄弟,而命其子尊已曰:``如我死,则必大为我棺,使吾二婢子夹我。''陈干昔死,其子曰:``以殉葬,非礼也,况又同棺乎?''弗果杀。

仲遂卒于垂;壬午犹绎,万入去龠。仲尼曰:``非礼也,卿卒不绎。''季康子之母死,公输若方小,敛,般请以机封,将从之,公肩假曰:``不可!夫鲁有初,公室视丰碑,三家视桓楹。般,尔以人之母尝巧,则岂不得以?其母以尝巧者乎?则病者乎?噫!''弗果从。

战于郎,公叔禺人遇负杖入保者息,曰:``使之虽病也,任之虽重也,君子不能为谋也,士弗能死也。不可!我则既言矣。''与其邻童汪踦往,皆死焉。鲁人欲勿殇童汪踦,问于仲尼。仲尼曰:``能执干戈以卫社稷,虽欲勿殇也,不亦可乎!''

子路去鲁,谓颜渊曰:``何以赠我?''曰:``吾闻之也:去国,则哭于墓而后行;反其国,不哭,展墓而入。''谓子路曰:``何以处我?''子路曰:``吾闻之也:过墓则式,过祀则下。''

工尹商阳与陈弃疾追吴师,及之。陈弃疾谓工尹商阳曰:``王事也,子手弓而可。''手弓。``子射诸。''射之,毙一人,韔弓。又及,谓之,又毙二人。每毙一人,掩其目。止其御曰:``朝不坐,燕不与,杀三人,亦足以反命矣。''孔子曰:``杀人之中,又有礼焉。''

诸侯伐秦,曹桓公卒于会。诸侯请含,使之袭。襄公朝于荆,康王卒。荆人曰:``必请袭。''鲁人曰:``非礼也。''荆人强之。巫先拂柩。荆人悔之。滕成公之丧,使子叔、敬叔吊,进书,子服惠伯为介。及郊,为懿伯之忌,不入。惠伯曰:``政也,不可以叔父之私,不将公事。''遂入。哀公使人吊蒉尚,遇诸道。辟于路,画宫而受吊焉。曾子曰:``蒉尚不如杞梁之妻之知礼也。齐庄公袭莒于夺,杞梁死焉,其妻迎其柩于路而哭之哀,庄公使人吊之,对曰:『君之臣不免于罪,则将肆诸市朝,而妻妾执;君之臣免于罪,则有先人之敝庐在。君无所辱命。』''

孺子穔之丧,哀公欲设拨,问于有若,有若曰:``其可也,君之三臣犹设之。''颜柳曰:``天子龙輴而椁帱,诸侯輴而设帱-\/-为榆沈故设拨;三臣者废輴而设拨,窃礼之不中者也,而君何学焉!''

悼公之母死,哀公为之齐衰。有若曰:``为妾齐衰,礼与?''公曰:``吾得已乎哉?鲁人以妻我。''季子皋葬其妻,犯人之禾,申祥以告曰:``请庚之。''子皋曰:``孟氏不以是罪予,朋友不以是弃予,以吾为邑长于斯也。买道而葬,后难继也。''仕而未有禄者:君有馈焉曰献,使焉曰寡君;违而君薨,弗为服也。虞而立尸,有几筵。卒哭而讳,生事毕而鬼事始已。既卒哭,宰夫执木铎以命于宫曰:``舍故而讳新。''自寝门至于库门。二名不偏讳,夫子之母名征在;言在不称征,言征不称在。军有忧,则素服哭于库门之外,赴车不载橐韔。有焚其先人之室,则三日哭。

故曰:``新宫火,亦三日哭。''

孔子过泰山侧,有妇人哭于墓者而哀,夫子式而听之。使子贡问之曰:``子之哭也,壹似重有忧者。''而曰:``然,昔者吾舅死于虎,吾夫又死焉,今吾子又死焉。''夫子曰:``何为不去也?''曰:``无苛政。''夫子曰:``小子识之,苛政猛于虎也。''

鲁人有周丰也者,哀公执挚请见之,而曰不可。公曰:``我其已夫!''使人问焉,曰:``有虞氏未施信于民而民信之,夏后氏未施敬于民而民敬之,何施而得斯于民也?''对曰:``墟墓之间,未施哀于民而民哀;社稷宗庙之中,未施敬于民而民敬。殷人作誓而民始畔,周人作会而民始疑。茍无礼义忠信诚悫之心以莅之,虽固结之,民其不解乎?''

丧不虑居,毁不危身。丧不虑居,为无庙也;毁不危身,为无后也。

延陵季子适齐,于其反也,其长子死,葬于嬴博之间。孔子曰:``延陵季子,吴之习于礼者也。''往而观其葬焉。其坎深不至于泉,其敛以时服。既葬而封,广轮掩坎,其高可隐也。既封,左袒,右还其封且号者三,曰:``骨肉归复于土,命也。若魂气则无不之也,无不之也。''而遂行。孔子曰:``延陵季子之于礼也,其合矣乎!''

邾娄考公之丧,徐君使容居来吊含,曰:``寡君使容居坐含进侯玉,其使容居以含。''有司曰:``诸侯之来辱敝邑者,易则易,于则于,易于杂者未之有也。''容居对曰:``容居闻之:事君不敢忘其君,亦不敢遗其祖。昔我先君驹王西讨济于河,无所不用斯言也。容居,鲁人也,不敢忘其祖。''

子思之母死于卫,赴于子思,子思哭于庙。门人至曰:``庶氏之母死,何为哭于孔氏之庙乎?''子思曰:``吾过矣,吾过矣。''遂哭于他室。

天子崩,三日祝先服,五日官长服,七日国中男女服,三月天下服。虞人致百祀之木,可以为棺椁者斩之;不至者,废其祀,刎其人。

齐大饥,黔敖为食于路,以待饿者而食之。有饿者蒙袂辑屦,贸贸然来。黔敖左奉食,右执饮,曰:``嗟!来食。''扬其目而视之,曰:``予唯不食嗟来之食,以至于斯也。''从而谢焉;终不食而死。曾子闻之曰:``微与?其嗟也可去,其谢也可食。''

邾娄定公之时,有弒其父者。有司以告,公瞿然失席曰:``是寡人之罪也。''曰:``寡人尝学断斯狱矣:臣弒君,凡在官者杀无赦;子弒父,凡在宫者杀无赦。杀其人,坏其室,洿其宫而猪焉。盖君逾月而后举爵。''

晋献文子成室,晋大夫发焉。张老曰:``美哉轮焉!美哉奂焉!歌于斯,哭于斯,聚国族于斯。''文子曰:``武也得歌于斯,哭于斯,聚国族于斯,是全要领以从先大夫于九京也。''北面再拜稽首。君子谓之善颂善祷。

仲尼之畜狗死,使子贡埋之,曰:``吾闻之也:敝帷不弃,为埋马也;敝盖不弃,为埋狗也。丘也贫,无盖;于其封也,亦予之席,毋使其首陷焉。''

路马死,埋之以帷。

季孙之母死,哀公吊焉,曾子与子贡吊焉,阍人为君在,弗内也。曾子与子贡入于其厩而修容焉。子贡先入,阍人曰:``乡者已告矣。''曾子后入,阍人辟之。涉内溜,卿大夫皆辟位,公降一等而揖之。君子言之曰:``尽饰之道,斯其行者远矣。''阳门之介夫死,司城子罕入而哭之哀。晋人之觇宋者,反报于晋侯曰:``阳门之介夫死,而子罕哭之哀,而民说,殆不可伐也。''孔子闻之曰:``善哉觇国乎!《诗》云:『凡民有丧,扶服救之。』虽微晋而已,天下其孰能当之。''

鲁庄公之丧,既葬,而绖不入库门。士、大夫既卒哭,麻不入。

孔子之故人曰原壤,其母死,夫子助之沐椁。原壤登木曰:``久矣予之不托于音也。''歌曰:``狸首之斑然,执女手之卷然。''夫子为弗闻也者而过之,从者曰:``子未可以已乎?''夫子曰:``丘闻之:亲者毋失其为亲也,故者毋失其为故也。''

赵文子与叔誉观乎九原。文子曰:``死者如可作也,吾谁与归?''叔誉曰:``其阳处父乎?''文子曰:``行并植于晋国,不没其身,其知不足称也。''``其舅犯乎?''文子曰:``见利不顾其君,其仁不足称也。我则随武子乎,利其君不忘其身,谋其身不遗其友。''

晋人谓文子知人。文子其中退然如不胜衣,其言吶吶然如不出诸其口;所举于晋国管库之士七十有余家,生不交利,死不属其子焉。叔仲皮学子柳。叔仲皮死,其妻鲁人也,衣衰而缪绖。叔仲衍以告,请繐衰而环绖,曰:``昔者吾丧姑姊妹亦如斯,末吾禁也。''退,使其妻繐衰而环绖。

成人有其兄死而不为衰者,闻子皋将为成宰,遂为衰。成人曰:``蚕则绩而蟹有匡,范则冠而蝉有緌,兄则死而子皋为之衰。''

乐正子春之母死,五日而不食。曰:``吾悔之,自吾母而不得吾情,吾恶乎用吾情!''

岁旱,穆公召县子而问然,曰:``天久不雨,吾欲暴尫而奚若?''曰:``天久不雨,而暴人之疾子,虐,毋乃不可与!''``然则吾欲暴巫而奚若?''曰:``天则不雨,而望之愚妇人,于以求之,毋乃已疏乎!''``徙市则奚若?''曰:``天子崩,巷市七日;诸侯薨,巷市三日。为之徙市,不亦可乎!''孔子曰:``卫人之祔也,离之;鲁人之祔也,合之,善夫!''

\hypertarget{header-n266}{%
\subsection{王制}\label{header-n266}}

王者之制禄爵,公侯伯子男,凡五等。诸侯之上大夫卿,下大夫,上士中士下士,凡五等。

天子之田方千里,公侯田方百里,伯七十里,子男五十里。不能五十里者,不合于天子,附于诸侯曰附庸。天子之三公之田视公侯,天子之卿视伯,天子之大夫视子男,天子之元士视附庸。

制:农田百亩。百亩之分:上农夫食九人,其次食八人,其次食七人,其次食六人;下农夫食五人。庶人在官者,其禄以是为差也。

诸侯之下士视上农夫,禄足以代其耕也。中上倍下士,上士倍中士,下大夫倍上士;卿,四大夫禄;君,十卿禄。次国之卿,三大夫禄;君,十卿禄。小国之卿,倍大夫禄,君十卿禄。

次国之上卿,位当大国之中,中当其下,下当其上大夫。小国之上卿,位当大国之下卿,中当其上大夫,下当其下大夫,其有中士、下士者,数各居其上之三分。凡四海之内九州岛,州方千里。州,建百里之国三十,七十里之国六十,五十里之国百有二十,凡二百一十国;名山大泽不以封,其余以为附庸间田。八州,州二百一十国。天子之县内,方百里之国九,七十里之国二十有一,五十里之国六十有三,凡九十三国;名山大泽不以晳,其余以禄士,以为间田。凡九州岛,千七百七十三国。天子之元士、诸侯之附庸不与。天子百里之内以共官,千里之内以为御。千里之外,设方伯。五国以为属,属有长。十国以为连,连有帅。三十国以为卒,卒有正。二百一十国以为州,州有伯。八州八伯,五十六正,百六十八帅,三百三十六长。八伯各以其属,属于天子之老二人,分天下以为左右,曰二伯。千里之内曰甸,千里之外,曰采、曰流。天子:三公,九卿,二十七大夫,八十一元士。大国:三卿;皆命于天子;下大夫五人,上士二十七人。次国:三卿;二卿命于天子,一卿命于其君;下大夫五人,上士二十七人。小国:二卿;皆命于其君;下大夫五人,上士二十七人。天子使其大夫为三监,监于方伯之国,国三人。天子之县内诸侯,禄也;外诸侯,嗣也。制:三公,一命卷;若有加,则赐也。不过九命。次国之君,不过七命;小国之君,不过五命。大国之卿,不过三命;下卿再命,小国之卿与下大夫一命。凡官民材,必先论之。论辨然后使之,任事然后爵之,位定然后禄之。爵人于朝,与士共之。刑人于市,与众弃之。是故公家不畜刑人,大夫弗养,士遇之涂弗与言也;屏之四方,唯其所之,不及以政,亦弗故生也。诸侯之于天子也,比年一小聘,三年一大聘,五年一朝。

天子五年一巡守:岁二月,东巡守至于岱宗,柴而望祀山川;觐诸侯;问百年者就见之。命大师陈诗以观民风,命市纳贾以观民之所好恶,志淫好辟。命典礼考时月,定日,同律,礼乐制度衣服正之。山川神只,有不举者,为不敬;不敬者,君削以地。宗庙,有不顺者为不孝;不孝者,君绌以爵。变礼易乐者,为不从;不从者,君流。革制度衣服者,为畔;畔者,君讨。有功德于民者,加地进律。五月,南巡守至于南岳,如东巡守之礼。八月,西巡守至于西岳,如南巡守之礼。十有一月,北巡守至于北岳,如西巡守之礼。归,假于祖祢,用特。

天子将出,类乎上帝,宜乎社,造乎祢。诸侯将出,宜乎社,造乎祢。天子无事与诸侯相见曰朝,考礼正刑一德,以尊于天子。天子赐诸侯乐,则以柷将之,赐伯、子、男乐,则以鼗将之。诸侯,赐弓矢然后征,赐鈇钺然后杀,赐圭瓒然后为鬯。未赐圭瓒,则资鬯于天子。

天子命之教然后为学。小学在公宫南之左,大学在郊。天子曰辟痈,诸侯曰頖宫。

天子将出征,类乎上帝,宜乎社,造乎祢,祃于所征之地。受命于祖,受成于学。出征,执有罪;反,释奠于学,以讯馘告。

天子、诸侯无事则岁三田:一为干豆,二为宾客,三为充君之庖。无事而不田,曰不敬;田不以礼,曰暴天物。天子不合围,诸侯不掩群。天子杀则下大绥,诸侯杀则下小绥,大夫杀则止佐车。佐车止,则百姓田猎。獭祭鱼,然后虞人入泽梁。豺祭兽,然后田猎。鸠化为鹰,然后设罻罗。草木零落,然后入山林。昆虫未蛰,不以火田,不麑,不卵,不杀胎,不殀夭,不覆巢。

冢宰制国用,必于岁之杪,五谷皆入然后制国用。用地小大,视年之丰耗。以三十年之通制国用,量入以为出,祭用数之仂。丧,三年不祭,唯祭天地社稷为越绋而行事。丧用三年之仂。丧祭,用不足曰暴,有余曰浩。祭,丰年不奢,凶年不俭。国无九年之蓄曰不足,无六年之蓄曰急,无三年之蓄曰国非其国也。三年耕,必有一年之食;九年耕,必有三年之食。以三十年之通,虽有凶旱水溢,民无菜色,然后天子食,日举以乐。

天子七日而殡,七月而葬。诸侯五日而殡,五月而葬。大夫、士、庶人,三日而殡,三月而葬。三年之丧,自天子达,庶人县封,葬不为雨止,不封不树,丧不贰事,自天子达于庶人。丧从死者,祭从生者。支子不祭。天子七庙,三昭三穆,与太祖之庙而七。诸侯五庙,二昭二穆,与太祖之庙而五。大夫三庙,一昭一穆,与太祖之庙而三。士一庙。庶人祭于寝。

天子、诸侯宗庙之祭:春曰礿,夏曰禘,秋曰尝,冬曰烝。天子祭天地,诸侯祭社稷,大夫祭五祀。天子祭天下名山大川:五岳视三公,四渎视诸侯。诸侯祭名山大川之在其地者。天子诸侯祭因国之在其地而无主后者。天子犆礿,祫禘,祫尝,祫烝。诸侯礿则不禘,禘则不尝,尝则不烝,烝则不礿。诸侯礿,犆;禘,一犆一祫;尝,祫;烝,祫。

天子社稷皆大牢,诸侯社稷皆少牢。大夫、士宗庙之祭,有田则祭,无田则荐。庶人春荐韭,夏荐麦,秋荐黍,冬荐稻。韭以卵,麦以鱼,黍以豚,稻以雁。祭天地之牛,角茧栗;宗庙之牛,角握;宾客之牛,角尺。诸侯无故不杀牛,大夫无故不杀羊,士无故不杀犬豕,庶人无故不食珍。庶羞不逾牲,燕衣不逾祭服,寝不逾庙。

古者:公田,藉而不税。市,廛而不税。关,讥而不征。林麓川泽,以时入而不禁。夫圭田无征。用民之力,岁不过三日。田里不粥,墓地不请。司空执度度地,居民山川沮泽,时四时。量地远近,兴事任力。凡使民:任老者之事,食壮者之食。凡居民材,必因天地寒暖燥湿,广谷大川异制。民生其间者异俗:刚柔轻重迟速异齐,五味异和,器械异制,衣服异宜。修其教,不易其俗;齐其政,不易其宜。中国戎夷,五方之民,皆有其性也,不可推移。东方曰夷,被髪文身,有不火食者矣。南方曰蛮,雕题交趾,有不火食者矣。西方曰戎,被髪衣皮,有不粒食者矣。北方曰狄,衣羽毛穴居,有不粒食者矣。中国、夷、蛮、戎、狄,皆有安居、和味、宜服、利用、备器,五方之民,言语不通,嗜欲不同。达其志,通其欲:东方曰寄,南方曰象,西方曰狄鞮,北方曰译。

凡居民,量地以制邑,度地以居民。地、邑、民、居,必参相得也。无旷土,无游民,食节事时,民咸安其居,乐事劝功,尊君亲上,然后兴学。

司徒修六礼以节民性,明七教以兴民德,齐八政以防淫,一道德以同俗,养耆老以致孝,恤孤独以逮不足,上贤以崇德,简不肖以绌恶。命乡,简不帅教者以告。耆老皆朝于庠,元日,习射上功,习乡上齿,大司徒帅国之俊士与执事焉。不变,命国之右乡,简不帅教者移之左;命国之左乡,简不帅教者移之右,如初礼。不变,移之郊,如初礼。不变,移之遂,如初礼。不变,屏之远方,终身不齿。命乡,论秀士,升之司徒,曰选士。司徒论选士之秀者而升之学,曰俊士。升于司徒者,不征于乡;升于学者,不征于司徒,曰造士。乐正崇四术,立四教,顺先王诗书礼乐以造士。春、秋教以礼乐,冬、夏教以诗书。王大子、王子、群后之大子、卿大夫元士之适子、国之俊选,皆造焉。凡入学以齿。将出学,小胥、大胥、小乐正简不帅教者以告于大乐正。大乐正以告于王。王命三公、九卿、大夫、元士皆入学。不变,王亲视学。不变,王三日不举,屏之远方。西方曰棘,东方曰寄,终身不齿。

大乐正论造士之秀者以告于王,而升诸司马,曰进士。司马辨论官材,论进士之贤者以告于王,而定其论。论定然后官之,任官然后爵之,位定然后禄之。大夫废其事,终身不仕,死以士礼葬之。有发,则命大司徒教士以车甲。凡执技论力,适四方,裸股肱,决射御。凡执技以事上者:祝史、射御、医卜及百工。凡执技以事上者:不贰事,不移官,出乡不与士齿。仕于家者,出乡不与士齿。司寇正刑明辟以听狱讼。必三刺。有旨无简不听。附从轻,赦从重。凡制五刑,必即天论。邮罚丽于事。

凡听五刑之讼,必原父子之亲、立君臣之义以权之。意论轻重之序、慎测浅深之量以别之。悉其聪明、致其忠爱以尽之。疑狱,泛与众共之;众疑,赦之。必察小大之比以成之。成狱辞,史以狱成告于正,正听之。正以狱成告于大司寇,大司寇听之棘木之下。大司寇以狱之成告于王,王命三公参听之。三公以狱之成告于王,王三又,然后制刑。凡作刑罚,轻无赦。刑者侀也,侀者成也,一成而不可变,故君子尽心焉。析言破律,乱名改作,执左道以乱政,杀。作淫声、异服、奇技、奇器以疑众,杀。行伪而坚,言伪而辩,学非而博,顺非而泽,以疑众,杀。假于鬼神、时日、卜筮以疑众,杀。此四诛者,不以听。凡执禁以齐众,不赦过。有圭璧金璋,不粥于市;命服命车,不粥于市;宗庙之器,不粥于市;牺牲不粥于市;戎器不粥于市。用器不中度,不粥于市。兵车不中度,不粥于市。布帛精粗不中数、幅广狭不中量,不粥于市。奸色乱正色,不粥于市。锦文珠玉成器,不粥于市。衣服饮食,不粥于市。五谷不时,果实未熟,不粥于市。木不中伐,不粥于市。禽兽鱼鳖不中杀,不粥于市。关执禁以讥,禁异服,识异言。

大史典礼,执简记,奉讳恶。天子齐戒受谏。司会以岁之成,质于天子,冢宰齐戒受质。大乐正、大司寇、市,三官以其成,从质于天子。大司徒、大司马、大司空齐戒受质;百官各以其成,质于三官。大司徒、大司马、大司空以百官之成,质于天子。百官齐戒受质。然后,休老劳农,成岁事,制国用。

凡养老:有虞氏以燕礼,夏后氏以飨礼,殷人以食礼,周人修而兼用之。五十养于乡,六十养于国,七十养于学,达于诸侯。八十拜君命,一坐再至,瞽亦如之。九十使人受。五十异粻,六十宿肉,七十贰膳,八十常珍;九十,饮食不离寝、膳饮从于游可也。六十岁制,七十时制,八十月制;九十日修,唯绞、衾、冒,死而后制。五十始衰,六十非肉不饱,七十非帛不暖,八十非人不暖;九十,虽得人不暖矣。五十杖于家,六十杖于乡,七十杖于国,八十杖于朝;九十者,天子欲有问焉,则就其室,以珍从。七十不俟朝,八十月告存,九十日有秩五十不从力政,六十不与服戎,七十不与宾客之事,八十齐丧之事弗及也。五十而爵,六十不亲学,七十致政。唯衰麻为丧。

有虞氏养国老于上庠,养庶老于下庠。夏后氏养国老于东序,养庶老于西序。殷人养国老于右学,养庶老于左学。周人养国老于东胶,养庶老于虞庠:虞庠在国之西郊。有虞氏皇而祭,深衣而养老。夏后氏收而祭,燕衣而养老。殷人冔而祭,缟衣而养老。周人冕而祭,玄衣而养老。凡三王养老皆引年。八十者一子不从政,九十者其家不从政,废疾非人不养者一人不从政。父母之丧,三年不从政。齐衰、大功之丧,三月不从政。将徙于诸侯,三月不从政。自诸侯来徙家,期不从政。

少而无父者谓之孤,老而无子者谓之独,老而无妻者谓之矜,老而无夫者谓之寡。此四者,天民之穷而无告者也,皆有常饩。瘖、聋、跛、躃、断者、侏儒、百工,各以其器食之。

道路:男子由右,妇人由左,车从中央。父之齿随行,兄之齿雁行,朋友不相逾。轻任并,重任分,斑白者不提挈。君子耆老不徒行,庶人耆老不徒食。

大夫祭器不假。祭器未成,不造燕器。

方一里者为田九百亩。方十里者,为方一里者百,为田九万亩。方百里者,为方十里者百,为田九十亿亩。方千里者,为方百里者百,为田九万亿亩。

自恒山至于南河,千里而近;自南河至于江,千里而近。自江至于衡山,千里而遥;自东河至于东海,千里而遥。自东河至于西河,千里而近;自西河至于流沙,千里而遥。西不尽流沙,南不尽衡山,东不近东海,北不尽恒山,凡四海之内,断长补短,方三千里,为田八十万亿一万亿亩。方百里者为田九十亿亩:山陵、林麓、川泽、沟渎、城郭、宫室、涂巷,三分去一,其余六十亿亩。

古者以周尺八尺为步,今以周尺六尺四寸为步。古者百亩,当今东田百四十六亩三十步。古者百里,当今百二十一里六十步四尺二寸二分。

方千里者,为方百里者百。封方百里者三十国,其余,方百里者七十。又封方七十里者六十-\/-为方百里者二十九,方十里者四十。其余,方百里者四十,方十里者六十;又封方五十里者二十-\/-为方百里者三十;其余,方百里者十,方十里者六十。名山大泽不以封,其余以为附庸间田。诸侯之有功者,取于间田以禄之;其有削地者,归之间田。

天子之县内:方千里者为方百里者百。封方百里者九,其余方百里者九十一。又封方七十里者二十一-\/-为方百里者十,方十里者二十九;其余,方百里者八十,方十里者七十一。又封方五十里者六十三-\/-为方百里者十五,方十里者七十五;其余方百里者六十四,方十里者九十六。诸侯之下士禄食九人,中士食十八人,上士食三十六人。下大夫食七十二人,卿食二百八十八人。君食二千八百八十人。次国之卿食二百一十六人,君食二千一百六十人。小国之卿食百四十四人,君食千四百四十人。次国之卿,命于其君者,如小国之卿。天子之大夫为三监,监于诸侯之国者,其禄视诸侯之卿,其爵视次国之君,其禄取之于方伯之地。方伯为朝天子,皆有汤沐之邑于天子之县内,视元士。诸侯世子世国,大夫不世爵。使以德,爵以功,未赐爵,视天子之元士,以君其国。诸侯之大夫,不世爵禄。

六礼:冠、昏、丧、祭、乡、相见。七教:父子、兄弟、夫妇、君臣、长幼、朋友、宾客。八政:饮食、衣服、事为、异别、度、量、数、制。

\hypertarget{header-n300}{%
\subsection{月令}\label{header-n300}}

孟春之月,日在营室,昏参中,旦尾中。其日甲乙。其帝大皞,其神句芒。其虫鳞。其音角,律中大蔟。其数八。其味酸,其臭膻。其祀户,祭先脾。

东风解冻,蛰虫始振,鱼上冰,獭祭鱼,鸿雁来。天子居青阳左个。乘鸾路,驾仓龙,载青旗,衣青衣,服仓玉,食麦与羊,其器疏以达。

是月也,以立春。先立春三日,大史谒之天子曰:某日立春,盛德在木。天子乃齐。立春之日,天子亲帅三公、九卿、诸侯、大夫以迎春于东郊。还反,赏公卿、诸侯、大夫于朝。命相布德和令,行庆施惠,下及兆民。庆赐遂行,毋有不当。乃命大史守典奉法,司天日月星辰之行,宿离不贷,毋失经纪,以初为常。

是月也,天子乃以元日祈谷于上帝。乃择元辰,天子亲载耒耜,措之参保介之御间,帅三公、九卿、诸侯、大夫,躬耕帝藉。天子三推,三公五推,卿诸侯九推。反,执爵于大寝,三公、九卿、诸侯、大夫皆御,命曰:劳酒。

是月也,天气下降,地气上腾,天地和同,草木萌动。王命布农事,命田舍东郊,皆修封疆,审端经术。善相丘陵阪险原隰土地所宜,五谷所殖,以教道民,必躬亲之。田事既饬,先定准直,农乃不惑。

是月也,命乐正入学习舞。乃修祭典。命祀山林川泽,牺牲毋用牝。禁止伐木。毋覆巢,毋杀孩虫、胎、夭、飞鸟。毋麑,毋卵。毋聚大众,毋置城郭。掩骼埋胔。

是月也,不可以称兵,称兵必天殃。兵戎不起,不可从我始。毋变天之道,毋绝地之理,毋乱人之纪。

孟春行夏令,则雨水不时,草木蚤落,国时有恐。行秋令则其民大疫,猋风暴雨总至,藜莠蓬蒿并兴。行冬令则水潦为败,雪霜大挚,首种不入。

仲春之月,日在奎,昏弧中,旦建星中。其日甲乙,其帝大皞,其神句芒。其虫鳞。其音角,律中夹钟。其数八。其味酸,其臭膻,其祀户,祭先脾。始雨水,桃始华,仓庚鸣,鹰化为鸠。天子居青阳大庙,乘鸾路,驾仓龙,载青旗,衣青衣,服仓玉,食麦与羊,其器疏以达。

是月也,安萌芽,养幼少,存诸孤。择元日,命民社。命有司省囹圄,去桎梏,毋肆掠,止狱讼。是月也,玄鸟至。至之日,以大牢祠于高禖。天子亲往,后妃帅九嫔御。乃礼天子所御,带以弓韣,授以弓矢,于高禖之前。

是月也,日夜分。雷乃发声,始电,蛰虫咸动,启户始出。先雷三日,奋木铎以令兆民曰:雷将发声,有不戒其容止者,生子不备,必有凶灾。日夜分,则同度量,钧衡石,角斗甬,正权概。是月也,耕者少舍。乃修阖扇,寝庙毕备。毋作大事,以妨农之事。

是月也,毋竭川泽,毋漉陂池,毋焚山林。天子乃鲜羔开冰,先荐寝庙。上丁,命乐正习舞,释菜。天子乃帅三公、九卿、诸侯、大夫亲往视之。仲丁,又命乐正入学习舞。是月也,祀不用牺牲,用圭璧,更皮币。

仲春行秋令,则其国大水,寒气总至,寇戎来征。行冬令,则阳气不胜,麦乃不熟,民多相掠。行夏令,则国乃大旱,暖气早来,虫螟为害。

季春之月,日在胃,昏七星中,旦牵牛中。其日甲乙。其帝大皞,其神句芒。其虫鳞。其音角,律中姑洗。其数八。其味酸,其臭膻。其祀户,祭先脾。桐始华,田鼠化为鴽,虹始见,萍始生。天子居青阳右个,乘鸾路,驾仓龙,载青旗,衣青衣,服仓玉。食麦与羊,其器疏以达。

是月也,天子乃荐鞠衣于先帝。命舟牧覆舟,五覆五反。乃告舟备具于天子焉,天子始乘舟。荐鲔于寝庙,乃为麦祈实。

是月也,生气方盛,阳气发泄,句者毕出,萌者尽达。不可以内。天子布德行惠,命有司发仓廪,赐贫穷,振乏绝,开府库,出币帛,周天下。勉诸侯,聘名士,礼贤者。

是月也,命司空曰:时雨将降,下水上腾,循行国邑,周视原野,修利堤防,道达沟渎,开通道路,毋有障塞。田猎罝罘、罗网、毕翳、餧兽之药,毋出九门。

是月也,命野虞毋伐桑柘。鸣鸠拂其羽,戴胜降于桑。具曲植籧筐。后妃齐戒,亲东乡躬桑。禁妇女毋观,省妇使以劝蚕事。蚕事既登,分茧称丝效功,以共郊庙之服,无有敢惰。

是月也,命工师令百工审五库之量:金铁,皮革筋,角齿,羽箭干,脂胶丹漆,毋或不良。百工咸理,监工日号;毋悖于时,毋或作为淫巧以荡上心。

是月之末,择吉日,大合乐,天子乃率三公、九卿、诸侯、大夫亲往视之。是月也,乃合累牛腾马,游牝于牧。牺牲驹犊,举,书其数。命国难,九门磔攘,以毕春气。

季春行冬令,则寒气时发,草木皆肃,国有大恐。行夏令,则民多疾疫,时雨不降,山林不收。行秋令,则天多沉阴,淫雨蚤降,兵革并起。

孟夏之月,日在毕,昏翼中,旦婺女中。其日丙丁。其帝炎帝,其神祝融。其虫羽。其音征,律中中吕。其数七。其味苦,其臭焦。其祀灶,祭先肺。蝼蝈鸣,蚯螾出,王瓜生,苦菜秀。天子居明堂左个,乘朱路,驾赤骝,载赤旗,衣朱衣,服赤玉。食菽与鸡,其器高以粗。

是月也,以立夏。先立夏三日,大史谒之天子曰:某日立夏,盛德在火。天子乃齐。立夏之日,天子亲帅三公、九卿、大夫以迎夏于南郊。还反,行赏,封诸侯。庆赐遂行,无不欣说。乃命乐师,习合礼乐。命太尉,赞桀俊,遂贤良,举长大,行爵出禄,必当其位。

是月也,继长增高,毋有坏堕,毋起土功,毋发大众,毋伐大树。是月也,天子始絺。命野虞出行田原,为天子劳农劝民,毋或失时。命司徒巡行县鄙,命农勉作,毋休于都。

是月也,驱兽毋害五谷,毋大田猎。农乃登麦,天子乃以彘尝麦,先荐寝庙。是月也,聚畜百药。靡草死,麦秋至。断薄刑,决小罪,出轻系。蚕事毕,后妃献茧。乃收茧税,以桑为均,贵贱长幼如一,以给郊庙之服。是月也,天子饮酎,用礼乐。

孟夏行秋令,则苦雨数来,五谷不滋,四鄙入保。行冬令,则草木蚤枯,后乃大水,败其城郭。行春令,则蝗虫为灾,暴风来格,秀草不实。

仲夏之月,日在东井,昏亢中,旦危中。其日丙丁。其帝炎帝,其神祝融。其虫羽。其音征,律中蕤宾。其数七。其味苦,其臭焦。其祀灶,祭先肺。小暑至,螳蜋生。鵙始鸣,反舌无声。天子居明堂太庙,乘朱路,驾赤骝,载赤旗,衣朱衣,服赤玉,食菽与鸡,其器高以粗。养壮佼。

是月也,命乐师修鼗鞞鼓,均琴瑟管箫,执干戚戈羽,调竽笙篪簧,饬钟磬柷敔。命有司为民祈祀山川百源,大雩帝,用盛乐。乃命百县,雩祀百辟卿士有益于民者,以祈谷实。农乃登黍。

是月也,天子乃以雏尝黍,羞以含桃,先荐寝庙。令民毋艾蓝以染,毋烧灰,毋暴布。门闾毋闭,关市毋索。挺重囚,益其食。游牝别群,则絷腾驹,班马政。

是月也,日长至,阴阳争,死生分。君子齐戒,处必掩身,毋躁。止声色,毋或进。薄滋味,毋致和。节嗜欲,定心气,百官静事毋刑,以定晏阴之所成。鹿角解,蝉始鸣。半夏生,木堇荣。是月也,毋用火南方。可以居高明,可以远眺望,可以升山陵,可以处台榭。

仲夏行冬令,则雹冻伤谷,道路不通,暴兵来至。行春令,则五谷晚熟,百螣时起,其国乃饥。行秋令,则草木零落,果实早成,民殃于疫。

季夏之月,日在柳,昏火中,旦奎中。其日丙丁。其帝炎帝,其神祝融。其虫羽。其音征,律中林钟。其数七。其味苦,其臭焦。其祀灶,祭先肺。温风始至,蟋蟀居壁,鹰乃学习,腐草为萤。天子居明堂右个,乘朱路,驾赤骝,载赤旗,衣朱衣,服赤玉。食菽与鸡,其器高以粗。命渔师伐蛟取鼍,登龟取鼋。命泽人纳材苇。

是月也,命四监大合百县之秩刍,以养牺牲。令民无不咸出其力,以共皇天上帝名山大川四方之神,以祠宗庙社稷之灵,以为民祈福。是月也,命妇官染采,黼黻文章,必以法故,无或差贷。黑黄仓赤,莫不质良,毋敢诈伪,以给郊庙祭祀之服,以为旗章,以别贵贱等给之度。

是月也,树木方盛,乃命虞人入山行木,毋有斩伐。不可以兴土功,不可以合诸侯,不可以起兵动众,毋举大事,以摇养气。毋发令而待,以妨神农之事也。水潦盛昌,神农将持功,举大事则有天殃。是月也,土润溽暑,大雨时行,烧薙行水,利以杀草,如以热汤。可以粪田畴,可以美土强。

季夏行春令,则谷实鲜落,国多风咳,民乃迁徙。行秋令,则丘隰水潦,禾稼不熟,乃多女灾。行冬令,则风寒不时,鹰隼蚤鸷,四鄙入保。

中央土。其日戊己。其帝黄帝,其神后土。其虫裸,其音宫,律中黄钟之宫。其数五。其味甘,其臭香。其祠中溜,祭先心。天子居大庙大室,乘大路,驾黄骝,载黄旗,衣黄衣,服黄玉,食稷与牛,其器圜以闳。

孟秋之月,日在翼,昏建星中,旦毕中。其日庚辛。其帝少皞,其神蓐收。其虫毛。其音商,律中夷则。其数九。其味辛,其臭腥。其祀门,祭先肝。凉风至,白露降,寒蝉鸣。鹰乃祭鸟,用始行戮。天子居总章左个,乘戎路,驾白骆,载白旗,衣白衣,服白玉,食麻与犬,其器廉以深。

是月也,以立秋。先立秋三日,大史谒之天子曰:某日立秋,盛德在金。天子乃齐。立秋之日,天子亲帅三公、九卿、诸侯、大夫,以迎秋于西郊。还反,赏军帅武人于朝。天子乃命将帅,选士厉兵,简练桀俊,专任有功,以征不义。诘诛暴慢,以明好恶,顺彼远方。

是月也,命有司修法制,缮囹圄,具桎梏,禁止奸,慎罪邪,务搏执。命理瞻伤,察创,视折,审断。决狱讼,必端平。戮有罪,严断刑。天地始肃,不可以赢。

是月也,农乃登谷。天子尝新,先荐寝庙。命百官,始收敛。完堤防,谨壅塞,以备水潦。修宫室,坏墙垣,补城郭。是月也,毋以封诸侯、立大官。毋以割地、行大使、出大币。孟秋行冬令,则阴气大胜,介虫败谷,戎兵乃来。行春令,则其国乃旱,阳气复还,五谷无实。行夏令,则国多火灾,寒热不节,民多疟疾。

仲秋之月,日在角,昏牵牛中,旦觜觿中。其日庚辛,其帝少皞,其神蓐收。其虫毛。其音商,律中南吕。其数九。其味辛,其臭腥。其祀门,祭先肝。盲风至,鸿雁来,玄鸟归,群鸟养羞。天子居总章大庙,乘戎路,驾白骆,载白旗,衣白衣,服白玉,食麻与犬,其器廉以深。

是月也,养衰老,授几杖,行糜粥饮食。乃命司服,具饬衣裳,文绣有恒,制有小大,度有长短。衣服有量,必循其故,冠带有常。乃命有司,申严百刑,斩杀必当,毋或枉桡。枉桡不当,反受其殃。

是月也,乃命宰祝,循行牺牲,视全具,案刍豢,瞻肥瘠,察物色。必比类,量小大,视长短,皆中度。五者备当,上帝其飨。天子乃难,以达秋气。以犬尝麻,先荐寝庙。

是月也,可以筑城郭,建都邑,穿窦窖,修囷仓。乃命有司,趣民收敛,务畜菜,多积聚。乃劝种麦,毋或失时。其有失时,行罪无疑。是月也,日夜分,雷始收声。蛰虫坏户,杀气浸盛,阳气日衰,水始涸。日夜分,则同度量,平权衡,正钧石,角斗甬。

是月也,易关市,来商旅,纳货贿,以便民事。四方来集,远乡皆至,则财不匮,上无乏用,百事乃遂。凡举大事,毋逆大数,必顺其时,慎因其类。仲秋行春令,则秋雨不降,草木生荣,国乃有恐。行夏令,则其国乃旱,蛰虫不藏,五谷复生。行冬令,则风灾数起,收雷先行,草木蚤死。

季秋之月,日在房,昏虚中,旦柳中。其日庚辛。其帝少皞,其神蓐收。其虫毛。其音商,律中无射。其数九。其味辛,其臭腥。其祀门,祭先肝。鸿雁来宾,爵入大水为蛤。鞠有黄华,豺乃祭兽戮禽。天子居总章右个,乘戎路,驾白骆,载白旗,衣白衣,服白玉。食麻与犬,其器廉以深。

是月也,申严号令。命百官贵贱无不务内,以会天地之藏,无有宣出。乃命冢宰,农事备收,举五谷之要,藏帝藉之收于神仓,祗敬必饬。

是月也,霜始降,则百工休。乃命有司曰:寒气总至,民力不堪,其皆入室。上丁,命乐正入学习吹。是月也,大飨帝、尝,牺牲告备于天子。合诸侯,制百县,为来岁受朔日,与诸侯所税于民轻重之法,贡职之数,以远近土地所宜为度,以给郊庙之事,无有所私。

是月也,天子乃教于田猎,以习五戎,班马政。命仆及七驺咸驾,载旌旐,授车以级,整设于屏外。司徒搢扑,北面誓之。天子乃厉饰,执弓挟矢以猎,命主祠祭禽于四方。

是月也,草木黄落,乃伐薪为炭。蛰虫咸俯在内,皆墐其户。乃趣狱刑,毋留有罪。收禄秩之不当、供养之不宜者。是月也,天子乃以犬尝稻,先荐寝庙。季秋行夏令,则其国大水,冬藏殃败,民多鼽嚏。行冬令,则国多盗贼,边境不宁,土地分裂。行春令,则暖风来至,民气解惰,师兴不居。

孟冬之月,日在尾,昏危中,旦七星中。其日壬癸。其帝颛顼,其神玄冥。其虫介。其音羽,律中应钟。其数六。其味咸,其臭朽。其祀行,祭先肾。水始冰,地始冻。雉入大水为蜃。虹藏不见。天子居玄堂左个,乘玄路,驾铁骊,载玄旗,衣黑衣,服玄玉,食黍与彘,其器闳以奄。

是月也,以立冬。先立冬三日,太史谒之天子曰:某日立冬,盛德在水。天子乃齐。立冬之日,天子亲帅三公、九卿、大夫以迎冬于北郊,还反,赏死事,恤孤寡。是月也,命大史衅龟策,占兆审卦吉凶,是察阿党,则罪无有掩蔽。

是月也,天子始裘。命有司曰:天气上腾,地气下降,天地不通,闭塞而成冬。命百官谨盖藏。命司徒循行积聚,无有不敛。坏城郭,戒门闾,修键闭,慎管龠,固封疆,备边竟,完要塞,谨关梁,塞徯径。饬丧纪,辨衣裳,审棺椁之薄厚,茔丘垄之大小、高卑、厚薄之度,贵贱之等级。

是月也,命工师效功,陈祭器,按度程,毋或作为淫巧以荡上心。必功致为上。物勒工名,以考其诚。功有不当,必行其罪,以穷其情。是月也,大饮烝。天子乃祈来年于天宗,大割祠于公社及门闾。腊先祖五祀,劳农以休息之。天子乃命将帅讲武,习射御角力。

是月也,乃命水虞渔师,收水泉池泽之赋。毋或敢侵削众庶兆民,以为天子取怨于下。其有若此者,行罪无赦。孟冬行春令,则冻闭不密,地气上泄,民多流亡。行夏令,则国多暴风,方冬不寒,蛰虫复出。行秋令,则雪霜不时,小兵时起,土地侵削。

仲冬之月,日在斗,昏东壁中,旦轸中。其日壬癸。其帝颛顼,其神玄冥。其虫介。其音羽,律中黄锺。其数六。其味咸,其臭朽。其祀行,祭先肾。冰益壮,地始坼。鹖旦不鸣,虎始交。天子居玄堂大庙,乘玄路,驾铁骊,载玄旗,衣黑衣,服玄玉。食黍与彘,其器闳以奄。饬死事。命有司曰:土事毋作,慎毋发盖,毋发室屋,及起大众,以固而闭。地气且泄,是谓发天地之房,诸蛰则死,民必疾疫,又随以丧。命之曰畅月。

是月也,命奄尹,申宫令,审门闾,谨房室,必重闭。省妇事毋得淫,虽有贵戚近习,毋有不禁。乃命大酋,秫稻必齐,曲蘗必时,湛炽必洁,水泉必香,陶器必良,火齐必得,兼用六物。大酋监之,毋有差贷。天子命有司祈祀四海大川名源渊泽井泉。

是月也,农有不收藏积聚者、马牛畜兽有放佚者,取之不诘。山林薮泽,有能取蔬食、田猎禽兽者,野虞教道之;其有相侵夺者,罪之不赦。

是月也,日短至。阴阳争,诸生荡。君子齐戒,处必掩身。身欲宁,去声色,禁耆欲。安形性,事欲静,以待阴阳之所定。芸始生,荔挺出,蚯螾结,麋角解,水泉动。日短至,则伐木,取竹箭。

是月也,可以罢官之无事、去器之无用者。涂阙廷门闾,筑囹圄,此所以助天地之闭藏也。仲冬行夏令,则其国乃旱,氛雾冥冥,雷乃发声。行秋令,则天时雨汁,瓜瓠不成,国有大兵。行春令,则蝗虫为败,水泉咸竭,民多疥疠。

季冬之月,日在婺女,昏娄中,旦氐中。其日壬癸。其帝颛顼,其神玄冥。其虫介。其音羽,律中大吕。其数六。其味咸,其臭朽。其祀行,祭先肾。雁北乡,鹊始巢。雉雊,鸡乳。天子居玄堂右个。乘玄路,驾铁骊,载玄旗,衣黑衣,服玄玉。食黍与彘,其器闳以奄。命有司大难,旁磔,出土牛,以送寒气。征鸟厉疾。乃毕山川之祀,及帝之大臣,天子神只。

是月也,命渔师始渔,天子亲往,乃尝鱼,先荐寝庙。冰方盛,水泽腹坚。命取冰,冰以入。令告民,出五种。命农计耦耕事,修耒耜,具田器。命乐师大合吹而罢。乃命四监收秩薪柴,以共郊庙及百祀之薪燎。

是月也,日穷于次,月穷于纪,星回于天。数将几终,岁且更始。专而农民,毋有所使。天子乃与公、卿、大夫,共饬国典,论时令,以待来岁之宜。乃命太史次诸侯之列,赋之牺牲,以共皇天、上帝、社稷之飨。乃命同姓之邦,共寝庙之刍豢。命宰历卿大夫至于庶民土田之数,而赋牺牲,以共山林名川之祀。凡在天下九州岛之民者,无不咸献其力,以共皇天、上帝、社稷、寝庙、山林、名川之祀。

季冬行秋令,则白露早降,介虫为妖,四鄙入保。行春令,则胎夭多伤,国多固疾,命之曰逆。行夏令,则水潦败国,时雪不降,冰冻消释。

\hypertarget{header-n367}{%
\subsection{曾子问}\label{header-n367}}

曾子问曰:``君薨而世子生,如之何?''孔子曰:``卿、大夫、、士从摄主,北面,于西阶南。大祝裨冕,执束帛,升自西阶尽等,不升堂,命毋哭。祝声三,告曰:『某之子生,敢告。』升,奠币于殡东几上,哭,降。众主人、卿、大夫、士,房中,皆哭不踊。尽一哀,反位。遂朝奠。小宰升举币。三日,众主人、卿、大夫、士,如初位,北面。大宰、大宗、大祝皆裨冕。少师奉子以衰;祝先,子从,宰宗人从。入门,哭者止,子升自西阶。殡前北面。祝立于殡东南隅。祝声三曰:『某之子某,从执事,敢见。』子拜稽颡哭。祝、宰、宗人、众主人、卿、大夫、士,哭踊三者三,降东反位,皆袒,子踊,房中亦踊三者三。袭衰,杖,奠出。大宰命祝史,以名遍告于五祀山川。''曾子问曰:``如已葬而世子生,则如之何?''孔子曰:``大宰、大宗从大祝而告于祢。三月,乃名于祢,以名遍告及社稷宗庙山川。''

孔子曰:``诸侯适天子,必告于祖,奠于祢。冕而出视朝,命祝史告于社稷、宗庙、山川。乃命国家五官而后行,道而出。告者,五日而遍,过是,非礼也。凡告,用牲币。反,亦如之。诸侯相见,必告于祢,朝服而出视朝。命祝史告于五庙所过山川。亦命国家五官,道而出。反,必亲告于祖祢。乃命祝史告至于前所告者,而后听朝而入。''

曾子问曰:``并有丧,如之何?何先何后?''孔子曰:``葬,先轻而后重;其奠也,先重而后轻;礼也。自启及葬,不奠,行葬不哀次;反葬奠,而后辞于殡,逐修葬事。其虞也,先重而后轻,礼也。''孔子曰:``宗子虽七十,无无主妇;非宗子,虽无主妇可也。''

曾子问曰:``将冠子,冠者至,揖让而入,闻齐衰大功之丧,如之何?''孔子曰:``内丧则废,外丧则冠而不醴,彻馔而扫,即位而哭。如冠者未至,则废。如将冠子而未及期日,而有齐衰、大功、小功之丧,则因丧服而冠。''``除丧不改冠乎?''孔子曰:``天子赐诸侯大夫冕弁服于大庙,归设奠,服赐服,于斯乎有冠醮,无冠醴。父没而冠,则已冠扫地而祭于祢;已祭,而见伯父、叔父,而后飨冠者。''

曾子问曰:``祭如之何则不行旅酬之事矣?''孔子曰:``闻之:小祥者,主人练祭而不旅,奠酬于宾,宾弗举,礼也。昔者,鲁昭公练而举酬行旅,非礼也;孝公大祥,奠酬弗举,亦非礼也。''

曾子问曰:``大功之丧,可以与于馈奠之事乎?''孔子曰:``岂大功耳!自斩衰以下皆可,礼也。''曾子曰:``不以轻服而重相为乎?''孔子曰:``非此之谓也。天子、诸侯之丧,斩衰者奠;大夫,齐衰者奠;士则朋友奠;不足,则取于大功以下者;不足,则反之。''曾子问曰:``小功可以与于祭乎?''孔子曰:``何必小功耳!自斩衰以下与祭,礼也。''曾子曰:``不以轻丧而重祭乎?''孔子曰:``天子、诸侯之丧祭也,不斩衰者不与祭;大夫,齐衰者与祭;士,祭不足,则取于兄弟大功以下者。''曾子问曰:``相识,有丧服可以与于祭乎?''孔子曰:``缌不祭,又何助于人。''

曾子问曰:``废丧服,可以与于馈奠之事乎?''孔子曰:``说衰与奠,非礼也;以摈相可也。''

曾子问曰:``昏礼既纳币,有吉日,女之父母死,则如之何?''孔子曰:``婿使人吊。如婿之父母死,则女之家亦使人吊。父丧称父,母丧称母。父母不在,则称伯父世母。婿,已葬,婿之伯父致命女氏曰:『某之子有父母之丧,不得嗣为兄弟,使某致命。』女氏许诺,而弗敢嫁,礼也。婿,免丧,女之父母使人请,婿弗取,而后嫁之,礼也。女之父母死,婿亦如之。''

曾子问曰:``亲迎,女在涂,而婿之父母死,如之何?''孔子曰:``女改服布深衣,缟总以趋丧。女在途,而女之父母死,则女反。''``如婿亲迎,女未至,而有齐衰大功之丧,则如之何?''孔子曰:``男不入,改服于外次;女入,改服于内次;然后即位而哭。''曾子问曰:``除丧则不复昏礼乎?''孔子曰:``祭,过时不祭,礼也;又何反于初?''孔子曰:``嫁女之家,三夜不息烛,思相离也。取妇之家,三日不举乐,思嗣亲也。三月而庙见,称来妇也。择日而祭于祢,成妇之义也。''曾子问曰:``女未庙见而死,则如之何?''孔子曰:``不迁于祖,不祔于皇姑,婿不杖、不菲、不次,归葬于女氏之党,示未成妇也。''

曾子问曰:``取女,有吉日而女死,如之何?''孔子曰:``婿齐衰而吊,既葬而除之。夫死亦如之。''曾子问曰:``丧有二孤,庙有二主,礼与?''孔子曰:``天无二日,土无二王,尝禘郊社,尊无二上。未知其为礼也。昔者齐桓公亟举兵,作伪主以行。及反,藏诸祖庙。庙有二主,自桓公始也。丧之二孤,则昔者卫灵公适鲁,遭季桓子之丧,卫君请吊,哀公辞不得命,公为主,客人吊。康子立于门右,北面;公揖让升自东阶,西乡;客升自西阶吊。公拜,兴,哭;康子拜稽颡于位,有司弗辩也。今之二孤,自季康子之过也。''

曾子问曰:``古者师行,必以迁庙主行乎?''孔子曰:``天子巡守,以迁庙主行,载于齐车,言必有尊也。今也取七庙之主以行,则失之矣。当七庙、五庙无虚主;虚主者,唯天子崩,诸侯薨与去其国,与祫祭于祖,为无主耳。吾闻诸老聃曰:天子崩,国君薨,则祝取群庙之主而藏诸祖庙,礼也。卒哭成事而后,主各反其庙。君去其国,大宰取群庙之主以从,礼也。祫祭于祖,则祝迎四庙之主。主,出庙入庙必跸;老聃云。''曾子问曰:``古者师行,无迁主,则何主?''孔子曰:``主命。''问曰:``何谓也?''孔子曰:``天子、诸侯将出,必以币帛皮圭告于祖祢,遂奉以出,载于齐车以行。每舍,奠焉而后就舍。反必告,设奠卒,敛币玉,藏诸两阶之间,乃出。盖贵命也。''

子游问曰:``丧慈母如母,礼与?''孔子曰:``非礼也。古者,男子外有傅,内有慈母,君命所使教子也,何服之有?昔者,鲁昭公少丧其母,有慈母良,及其死也,公弗忍也,欲丧之,有司以闻,曰:『古之礼,慈母无服,今也君为之服,是逆古之礼而乱国法也;若终行之,则有司将书之以遗后世。无乃不可乎!』公曰:『古者天子练冠以燕居。』公弗忍也,遂练冠以丧慈母。丧慈母,自鲁昭公始也。''

曾子问曰:``诸侯旅见天子,入门,不得终礼,废者几?''孔子曰:``四。''请问之。曰:``大庙火,日食,后之丧,雨沾服失容,则废。如诸侯皆在而日食,则从天子救日,各以其方色与其兵。大庙火,则从天子救火,不以方色与兵。''曾子问曰:``诸侯相见,揖让入门,不得终礼,废者几?''孔子曰:``六。''请问之。曰:``天子崩,大庙火,日食,后夫人之丧,雨沾服失容,则废。''曾子问曰:``天子尝禘郊社五祀之祭,簠簋既陈,天子崩,后之丧,如之何?''孔子曰:``废。''曾子问曰:``当祭而日食,太庙火,其祭也如之何?''孔子曰:``接祭而已矣。如牲至,未杀,则废。天子崩,未殡,五祀之祭不行;既殡而祭,其祭也,尸入,三饭不侑,酳不酢而已矣。自启至于反哭,五祀之祭不行;已葬而祭,祝毕献而已。''曾子问曰:``诸侯之祭社稷,俎豆既陈,闻天子崩、后之丧、君薨、夫人之丧,如之何?''孔子曰:``废。自薨比至于殡,自启至于反哭,奉帅天子。''曾子问曰:``大夫之祭,鼎俎既陈,笾豆既设,不得成礼,废者几?''孔子曰:``九。''请问之。曰:``天子崩、后之丧、君薨、夫人之丧、君之大庙火、日食、三年之丧、齐衰、大功,皆废。外丧自齐衰以下,行也。其齐衰之祭也,尸入,三饭不侑,酳不酢而已矣;大功酢而已矣;小功、缌,室中之事而已矣。士之所以异者,缌不祭,所祭于死者无服则祭。''

曾子问曰:``三年之丧,吊乎?''孔子曰:``三年之丧,练,不群立,不旅行。君子礼以饰情,三年之丧而吊哭,不亦虚乎?''曾子问曰:``大夫、士有私丧,可以除之矣,而有君服焉,其除之也如之何?''孔子曰:``有君丧服于身,不敢私服,又何除焉?于是乎有过时而弗除也。君之丧,服除而后殷祭,礼也。''曾子问曰:``父母之丧,弗除可乎?''孔子曰:``先王制礼,过时弗举,礼也;非弗能勿除也,患其过于制也,故君子过时不祭,礼也。''

曾子问曰:``君薨,既殡,而臣有父母之丧,则如之何?''孔子曰:``归居于家,有殷事,则之君所,朝夕否。''曰:``君既启,而臣有父母之丧,则如之何?''孔子曰:``归哭而反送君。''曰:``君未殡,而臣有父母之丧,则如之何?''孔子曰:``归殡,反于君所,有殷事则归,朝夕否。大夫,室老行事;士,则子孙行事。大夫内子,有殷事,亦之君所,朝夕否。''

贱不诔贵,幼不诔长,礼也。唯天子,称天以诔之。诸侯相诔,非礼也。

曾子问曰:``君出疆以三年之戒,以椑从。君薨,其入如之何?''孔子曰:``共殡服,则子麻,弁绖,疏衰,菲,杖。入自阙,升自西阶。如小敛,则子免而从柩,入自门,升自阼阶。君大夫士一节也。''曾子问曰:``君之丧既引,闻父母之丧,如之何?''孔子曰:``遂。既封而归,不俟子。''曾子问曰:``父母之丧既引,及涂,闻君薨,如之何?''孔子曰:``遂。既封,改服而往。''

曾子问曰:``宗子为士,庶子为大夫,其祭也如之何?''孔子曰:``以上牲祭于宗子之家。祝曰:『孝子某为介子某荐其常事。』若宗子有罪,居于他国,庶子为大夫,其祭也,祝曰:『孝子某使介子某执其常事。』摄主不厌祭,不旅,不假,不绥祭,不配。布奠于宾,宾奠而不举,不归肉。其辞于宾曰:『宗兄、宗弟、宗子在他国,使某辞。』''曾子问曰:``宗子去在他国,庶子无爵而居者,可以祭乎?''孔子曰:``祭哉!''请问:``其祭如之何?''孔子曰:``望墓而为坛,以时祭。若宗子死,告于墓而后祭于家。宗子死,称名不言孝,身没而已。子游之徒,有庶子祭者以此,若义也。今之祭者,不首其义,故诬于祭也。''

曾子问曰:``祭必有尸乎?若厌祭亦可乎?''孔子曰:``祭成丧者必有尸,尸必以孙。孙幼,则使人抱之。无孙,则取于同姓可也。祭殇必厌,盖弗成也。祭成丧而无尸,是殇之也。''孔子曰:``有阴厌,有阳厌。''曾子问曰:``殇不祔祭,何谓阴厌、阳厌?''孔子曰:``宗子为殇而死,庶子弗为后也。其吉祭,特牲。祭殇不举,无肵俎,无玄酒,不告利成,是谓阴厌。凡殇,与无后者,祭于宗子之家,当室之白,尊于东房,是谓阳厌。''

曾子问曰:``葬引至于堩,日有食之,则有变乎?且不乎?''孔子曰:``昔者吾从老聃助葬于巷党,及堩,日有食之,老聃曰:『丘!止柩,就道右,止哭以听变。』既明反而后行。曰:『礼也。』反葬,而丘问之曰:『夫柩不可以反者也,日有食之,不知其已之迟数,则岂如行哉?』老聃曰:『诸侯朝天子,见日而行,逮日而舍奠;大夫使,见日而行,逮日而舍。夫柩不早出,不暮宿。见星而行者,唯罪人与奔父母之丧者乎!日有食之,安知其不见星也?且君子行礼,不以人之亲痁患。』吾闻诸老聃云。''

曾子问曰:``为君使而卒于舍,礼曰:公馆复,私馆不复。凡所使之国,有司所授舍,则公馆已,何谓私馆不复也?''孔子曰:``善乎问之也!自卿、大夫、士之家,曰私馆;公馆与公所为,曰公馆。公馆复,此之谓也。''曾子问曰:``下殇:土周葬于园,遂舆机而往,途迩故也。今墓远,则其葬也如之何?''孔子曰:``吾闻诸老聃曰:昔者史佚有子而死,下殇也。墓远,召公谓之曰:『何以不棺敛于宫中?』史佚曰:『吾敢乎哉?』召公言于周公,周公曰:『岂不可?』史佚行之。下殇用棺衣棺,自史佚始也。''

曾子问曰:``卿、大夫将为尸于公,受宿矣,而有齐衰内丧,则如之何?''孔子曰:``出,舍于公馆以待事,礼也。''孔子曰:``尸弁冕而出,卿、大夫、士皆下之,尸必式,必有前驱。''子夏问曰:``三年之丧卒哭,金革之事无辟也者,礼与?初有司与?''孔子曰:``夏后氏三年之丧,既殡而致事,殷人既葬而致事。《记》曰:『君子不夺人之亲,亦不可夺亲也。』此之谓乎?''子夏曰:``金革之事无辟也者,非与?''孔子曰:``吾闻诸老聃曰:昔者鲁公伯禽有为为之也。今以三年之丧,从其利者,吾弗知也!''

\hypertarget{header-n392}{%
\subsection{文王世子}\label{header-n392}}

文王之为世子,朝于王季,日三。鸡初鸣而衣服,至于寝门外,问内竖之御者曰:``今日安否何如?''内竖曰:``安。''文王乃喜。及日中,又至,亦如之。及莫,又至,亦如之。其有不安节,则内竖以告文王,文王色忧,行不能正履。王季腹膳,然后亦复初。食上,必在,视寒暖之节,食下,问所膳;命膳宰曰:``末有原!''应曰:``诺。''然后退。武王帅而行之,不敢有加焉。文王有疾,武王不脱冠带而养。文王一饭,亦一饭;文王再饭,亦再饭。旬有二日乃间。文王谓武王曰:``女何梦矣?''武王对曰:``梦帝与我九龄。''文王曰:``女以为何也?''武王曰:``西方有九国焉,君王其终抚诸?''文王曰:``非也。古者谓年龄,齿亦龄也。我百尔九十,吾与尔三焉。''文王九十七乃终,武王九十三而终。成王幼,不能莅阼,周公相,践阼而治。抗世子法于伯禽,欲令成王之知父子、君臣、长幼之道也;成王有过,则挞伯禽,所以示成王世子之道也。文王之为世子也。

凡学世子及学士,必时。春夏学干戈,秋冬学羽龠,皆于东序。小乐正学干,大胥赞之。龠师学戈,龠师丞赞之。胥鼓南。春诵夏弦,大师诏之。瞽宗秋学礼,执礼者诏之;冬读书,典书者诏之。礼在瞽宗,书在上庠。凡祭与养老,乞言,合语之礼,皆小乐正诏之于东序。大乐正学舞干戚,语说,命乞言,皆大乐正授数,大司成论说在东序。

凡侍坐于大司成者,远近间三席,可以问。终则负墙,列事未尽,不问。凡学,春官释奠于其先师,秋冬亦如之。凡始立学者,必释奠于先圣先师;及行事,必以币。凡释奠者,必有合也,有国故则否。凡大合乐,必遂养老。凡语于郊者,必取贤敛才焉。或以德进,或以事举,或以言扬。曲艺皆誓之,以待又语。三而一有焉,乃进其等,以其序,谓之郊人,远之。于成均以及取爵于上尊也。始立学者,既兴器用币,然后释菜不舞不授器,乃退。傧于东序,一献,无介语可也。教世子。

凡三王教世子必以礼乐。乐,所以修内也;礼,所以修外也。礼乐交错于中,发形于外,是故其成也怿,恭敬而温文。立大傅、少傅以养之,欲其知父子、君臣之道也。大傅审父子、君臣之道以示之;少傅奉世子,以观大傅之德行而审喻之。大傅在前,少傅在后;入则有保,出则有师,是以教喻而德成也。师也者,教之以事而喻诸德者也;保也者,慎其身以辅翼之而归诸道者也。《记》曰:``虞、夏、商、周,有师保,有疑丞。''设四辅及三公。不必备,唯其人。语使能也。君子曰德,德成而教尊,教尊而官正,官正而国治,君之谓也。仲尼曰:``昔者周公摄政,践阼而治,抗世子法于伯禽,所以善成王也。闻之曰:为人臣者,杀其身有益于君则为之,况于其身以善其君乎?周公优为之!''是故知为人子,然后可以为人父;知为人臣,然后可以为人君;知事人,然后能使人。成王幼,不能莅阼,以为世子,则无为也,是故抗世子法于伯禽,使之与成王居,欲令成王之知父子、君臣、长幼之义也。君之于世子也,亲则父也,尊则君也。有父之亲,有君之尊,然后兼天下而有之。是故,养世子不可不慎也。行一物而三善皆得者,唯世子而已。其齿于学之谓也。故世子齿于学,国人观之曰:``将君我而与我齿让何也?''曰:``有父在则礼然,然而众知父子之道矣。''其二曰:``将君我而与我齿让何也?''曰:``有君在则礼然,然而众着于君臣之义也。''其三曰:``将君我而与我齿让何也?''曰:``长长也,然而众知长幼之节矣。''故父在斯为子,君在斯谓之臣,居子与臣之节,所以尊君亲亲也。故学之为父子焉,学之为君臣焉,学之为长幼焉,父子、君臣、长幼之道得,而国治。语曰:``乐正司业,父师司成,一有元良,万国以贞。''世子之谓也。周公践阼。

庶子之正于公族者,教之以孝弟、睦友、子爱,明父子之义、长幼之序。其朝于公:内朝,则东面北上;臣有贵者,以齿。其在外朝,则以官,司士为之。其在宗庙之中,则如外朝之位。宗人授事,以爵以官。其登馂献受爵,则以上嗣。庶子治之,虽有三命,不逾父兄。其公大事,则以其丧服之精粗为序。虽于公族之丧亦如之,以次主人。若公与族燕,则异姓为宾,膳宰为主人,公与父兄齿。族食,世降一等。其在军,则守于公祢。公若有出疆之政,庶子以公族之无事者守于公宫,正室守大庙,诸父守贵宫贵室,诸子诸孙守下宫下室。五庙之孙,祖庙未毁,虽为庶人,冠,取妻,必告;死,必赴;练祥则告。族之相为也,宜吊不吊,宜免不免,有司罚之。至于赗赙承含,皆有正焉。公族其有死罪,则磬于甸人。其刑罪,则纤剸,亦告于甸人。公族无宫刑。狱成,有司谳于公。其死罪,则曰``某之罪在大辟'';其刑罪,则曰``某之罪在小辟''。公曰:``宥之。''有司又曰:``在辟。''公又曰:``宥之。''有司又曰:``在辟。''及三宥,不对,走出,致刑于于甸人。公又使人追之曰:``虽然,必赦之。''有司对曰:``无及也!''反命于公,公素服不举,为之变,如其伦之丧。无服,亲哭之。公族朝于内朝,内亲也。虽有贵者以齿,明父子也。外朝以官,体异姓也。宗庙之中,以爵为位,崇德也。宗人授事以官,尊贤也。登馂受爵以上嗣,尊祖之道也。丧纪以服之轻重为序,不夺人亲也。公与族燕则以齿,而孝弟之道达矣。其族食世降一等,亲亲之杀也。战则守于公祢,孝爱之深也。正室守大庙,尊宗室,而君臣之道着矣。诸父诸兄守贵室,子弟守下室,而让道达矣。五庙之孙,祖庙未毁,虽及庶人,冠,取妻必告,死必赴,不忘亲也。亲未绝而列于庶人,贱无能也。敬吊临赙赗,睦友之道也。古者,庶子之官治,而邦国有伦;邦国有伦,而众乡方矣。公族之罪,虽亲不以犯有司,正术也,所以体百姓也。刑于隐者,不与国人虑兄弟也。弗吊,弗为服,哭于异姓之庙,为忝祖远之也。素服居外,不听乐,私丧之也,骨肉之亲无绝也。公族无宫刑,不翦其类也。天子视学,大昕鼓征,所以警众也。众至,然后天子至。乃命有司行事。兴秩节,祭先师先圣焉。有司卒事,反命。始之养也:适东序,释奠于先老,遂设三老五更群老之席位焉。适馔省醴,养老之珍,具;遂发咏焉,退修之以孝养也。反,登歌清庙,既歌而语,以成之也。言父子、君臣、长幼之道,合德音之致,礼之大者也。下管《象》,舞《大武》。大合众以事,达有神,兴有德也。正君臣之位、贵贱之等焉,而上下之义行矣。有司告以乐阕,王乃命公侯伯子男及群吏曰:``反!养老幼于东序。''终之以仁也。是故圣人之记事也,虑之以大,爱之以敬,行之以礼,修之以孝养,纪之以义,终之以仁。是故古之人一举事而众皆知其德之备也。古之君子,举大事,必慎其终始,而众安得不喻焉?《兑命》曰:``念终始典于学。''

世子之记曰:朝夕至于大寝之门外,问于内竖曰:``今日安否何如?''内竖曰:``今日安。''世子乃有喜色。其有不安节,则内竖以告世子,世子色忧不满容。内竖言``复初'',然后亦复初。朝夕之食上,世子必在,视寒暖之节。食下,问所膳羞。必知所进,以命膳宰,然后退。若内竖言``疾'',则世子亲齐玄而养。膳宰之馔,必敬视之;疾之药,必亲尝之。尝馔善,则世子亦能食;尝馔寡,世子亦不能饱;以至于复初,然后亦复初。

\hypertarget{header-n401}{%
\subsection{礼运}\label{header-n401}}

昔者仲尼与于蜡宾,事毕,出游于观之上,喟然而叹。仲尼之叹,盖叹鲁也。言偃在侧曰:``君子何叹?''孔子曰:``大道之行也,与三代之英,丘未之逮也,而有志焉。''大道之行也,天下为公。选贤与能,讲信修睦,故人不独亲其亲,不独子其子,使老有所终,壮有所用,幼有所长,矜寡孤独废疾者,皆有所养。男有分,女有归。货恶其弃于地也,不必藏于己;力恶其不出于身也,不必为己。是故谋闭而不兴,盗窃乱贼而不作,故外户而不闭,是谓大同。今大道既隐,天下为家,各亲其亲,各子其子,货力为己,大人世及以为礼。城郭沟池以为固,礼义以为纪;以正君臣,以笃父子,以睦兄弟,以和夫妇,以设制度,以立田里,以贤勇知,以功为己。故谋用是作,而兵由此起。禹、汤、文、武、成王、周公,由此其选也。此六君子者,未有不谨于礼者也。以着其义,以考其信,着有过,刑仁讲让,示民有常。如有不由此者,在势者去,众以为殃,是谓小康。言偃复问曰:``如此乎礼之急也?''孔子曰:``夫礼,先王以承天之道,以治人之情。故失之者死,得之者生。《诗》曰:『相鼠有体,人而无礼;人而无礼,胡不遄死?』是故夫礼,必本于天,殽于地,列于鬼神,达于丧祭、射御、冠昏、朝聘。故圣人以礼示之,故天下国家可得而正也。''言偃复问曰:``夫子之极言礼也,可得而闻与?''孔子曰:``我欲观夏道,是故之杞,而不足征也;吾得夏时焉。我欲观殷道,是故之宋,而不足征也;吾得坤干焉。坤干之义,夏时之等,吾以是观之。''夫礼之初,始诸饮食,其燔黍捭豚,污尊而抔饮,蒉桴而土鼓,犹若可以致其敬于鬼神。及其死也,升屋而号,告曰:``皋!某复。''然后饭腥而苴孰。故天望而地藏也,体魄则降,知气在上,故死者北首,生者南乡,皆从其初。昔者先王,未有宫室,冬则居营窟,夏则居橧巢。未有火化,食草木之实、鸟兽之肉,饮其血,茹其毛。未有麻丝,衣其羽皮。后圣有作,然后修火之利,范金合土,以为台榭、宫室、牖户,以炮以燔,以亨以炙,以为醴酪;治其麻丝,以为布帛,以养生送死,以事鬼神上帝,皆从其朔。故玄酒在室,醴醆在户,粢醍在堂,澄酒在下。陈其牺牲,备其鼎俎,列其琴瑟管磬钟鼓,修其祝嘏,以降上神与其先祖。以正君臣,以笃父子,以睦兄弟,以齐上下,夫妇有所。是谓承天之祜。作其祝号,玄酒以祭,荐其血毛,腥其俎,孰其殽,与其越席,疏布以幂,衣其浣帛,醴醆以献,荐其燔炙,君与夫人交献,以嘉魂魄,是谓合莫。然后退而合亨,体其犬豕牛羊,实其簠簋、笾豆、铏羹。祝以孝告,嘏以慈告,是谓大祥。此礼之大成也。

孔子曰:``于呼哀哉!我观周道,幽、厉伤之,吾舍鲁何适矣!鲁之郊禘,非礼也,周公其衰矣!杞之郊也禹也,宋之郊也契也,是天子之事守也。故天子祭天地,诸侯祭社稷。''祝嘏莫敢易其常古,是谓大假。祝嘏辞说,藏于宗祝巫史,非礼也,是谓幽国。醆斝及尸君,非礼也,是谓僭君。冕弁兵革藏于私家,非礼也,是谓胁君。大夫具官,祭器不假,声乐皆具,非礼也,是谓乱国。故仕于公曰臣,仕于家曰仆。三年之丧,与新有昏者,期不使。以衰裳入朝,与家仆杂居齐齿,非礼也,是谓君与臣同国。故天子有田以处其子孙,诸侯有国以处其子孙,大夫有采以处其子孙,是谓制度。故天子适诸侯,必舍其祖朝,而不以礼籍入,是谓天子坏法乱纪。诸侯非问疾吊丧而入诸臣之家,是谓君臣为谑。是故,礼者君之大柄也,所以别嫌明微,傧鬼神,考制度,别仁义,所以治政安君也。故政不正,则君位危;君位危,则大臣倍,小臣窃。刑肃而俗敝,则法无常;法无常,而礼无列;礼无列,则士不事也。刑肃而俗敝,则民弗归也,是谓疵国。故政者君之所以藏身也。是故夫政必本于天,殽以降命。命降于社之谓殽地,降于祖庙之谓仁义,降于山川之谓兴作,降于五祀之谓制度。此圣人所以藏身之固也。故圣人参于天地,并于鬼神,以治政也。处其所存,礼之序也;玩其所乐,民之治也。故天生时而地生财,人其父生而师教之:四者,君以正用之,故君者立于无过之地也。故君者所明也,非明人者也。君者所养也,非养人者也。君者所事也,非事人者也。故君明人则有过,养人则不足,事人则失位。故百姓则君以自治也,养君以自安也,事君以自显也。故礼达而分定,人皆爱其死而患其生。故用人之知去其诈,用人之勇去其怒,用人之仁去其贪。故国有患,君死社稷谓之义,大夫死宗庙谓之变。故圣人耐以天下为一家,以中国为一人者,非意之也,必知其情,辟于其义,明于其利,达于其患,然后能为之。何谓人情?喜怒哀惧爱恶欲七者,弗学而能。何谓人义?父慈、子孝、兄良、弟弟、夫义、妇听、长惠、幼顺、君仁、臣忠十者,谓之人义。讲信修睦,谓之人利。争夺相杀,谓之人患。故圣人所以治人七情,修十义,讲信修睦,尚辞让,去争夺,舍礼何以治之?饮食男女,人之大欲存焉;死亡贫苦,人之大恶存焉。故欲恶者,心之大端也。人藏其心,不可测度也;美恶皆在其心,不见其色也,欲一以穷之,舍礼何以哉?故人者,其天地之德,阴阳之交,鬼神之会,五行之秀气也。故天秉阳,垂日星;地秉阴,窍于山川。播五行于四时,和而后月生也。是以三五而盈,三五而阙。五行之动,迭相竭也,五行、四时、十二月,还相为本也;五声、六律、、十二管,还相为宫也;五味、六和、、十二食,还相为质也;五色、六章、十二衣,还相为质也。故人者,天地之心也,五行之端也,食味别声被色而生者也。故圣人作则,必以天地为本,以阴阳为端,以四时为柄,以日星为纪,月以为量,鬼神以为徒,五行以为质,礼义以为器,人情以为田,四灵以为畜。以天地为本,故物可举也;以阴阳为端,故情可睹也;以四时为柄,故事可劝也;以日星为纪,故事可列也;月以为量,故功有艺也;鬼神以为徒,故事有守也;五行以为质,故事可复也;礼义以为器,故事行有考也;人情以为田,故人以为奥也;四灵以为畜,故饮食有由也。

何谓四灵?麟凤龟龙,谓之四灵。故龙以为畜,故鱼鲔不淰;凤以为畜,故鸟不獝;麟以为畜,故兽不狘;龟以为畜,故人情不失。故先王秉蓍龟,列祭祀,瘗缯,宣祝嘏辞说,设制度,故国有礼,官有御,事有职,礼有序。故先王患礼之不达于下也,故祭帝于郊,所以定天位也;祀社于国,所以列地利也;祖庙所以本仁也,山川所以傧鬼神也,五祀所以本事也。故宗祝在庙,三公在朝,三老在学。王,前巫而后史,卜筮瞽侑皆在左右,王中心无为也,以守至正。故礼行于郊,而百神受职焉,礼行于社,而百货可极焉,礼行于祖庙而孝慈服焉,礼行于五祀而正法则焉。故自郊社、祖庙、山川、五祀,义之修而礼之藏也。是故夫礼,必本于大一,分而为天地,转而为阴阳,变而为四时,列而为鬼神。其降曰命,其官于天也。夫礼必本于天,动而之地,列而之事,变而从时,协于分艺,其居人也曰养,其行之以货力、辞让:饮食、冠昏、丧祭、射御、朝聘。故礼义也者,人之大端也,所以讲信修睦而固人之肌肤之会、筋骸之束也。所以养生送死事鬼神之大端也。所以达天道顺人情之大窦也。故唯圣人为知礼之不可以已也,故坏国、丧家、亡人,必先去其礼。故礼之于人也,犹酒之有蘗也,君子以厚,小人以薄。故圣王修义之柄、礼之序,以治人情。故人情者,圣王之田也。修礼以耕之,陈义以种之,讲学以耨之,本仁以聚之,播乐以安之。故礼也者,义之实也。协诸义而协,则礼虽先王未之有,可以义起也。义者艺之分、仁之节也,协于艺,讲于仁,得之者强。仁者,义之本也,顺之体也,得之者尊。故治国不以礼,犹无耜而耕也;为礼不本于义,犹耕而弗种也;为义而不讲之以学,犹种而弗耨也;讲之于学而不合之以仁,犹耨而弗获也;合之以仁而不安之以乐,犹获而弗食也;安之以乐而不达于顺,犹食而弗肥也。四体既正,肤革充盈,人之肥也。父子笃,兄弟睦,夫妇和,家之肥也。大臣法,小臣廉,官职相序,君臣相正,国之肥也。天子以德为车、以乐为御,诸侯以礼相与,大夫以法相序,士以信相考,百姓以睦相守,天下之肥也。是谓大顺。大顺者,所以养生送死、事鬼神之常也。故事大积焉而不苑,并行而不缪,细行而不失。深而通,茂而有间。连而不相及也,动而不相害也,此顺之至也。故明于顺,然后能守危也。故礼之不同也,不丰也,不杀也,所以持情而合危也。故圣王所以顺,山者不使居川,不使渚者居中原,而弗敝也。用水火金木,饮食必时。合男女,颁爵位,必当年德。用民必顺。故无水旱昆虫之灾,民无凶饥妖孽之疾。故天不爱其道,地不爱其宝,人不爱其情。故天降膏露,地出醴泉,山出器车,河出马图,凤凰麒麟皆在郊棷,龟龙在宫沼,其余鸟兽之卵胎,皆可俯而窥也。则是无故,先王能修礼以达义,体信以达顺,故此顺之实也。

\hypertarget{header-n407}{%
\subsection{礼器}\label{header-n407}}

礼器是故大备。大备,盛德也。礼释回,增美质;措则正,施则行。其在人也,如竹箭之有筠也;如松柏之有心也。二者居天下之大端矣。故贯四时而不改柯易叶。故君子有礼,则外谐而内无怨,故物无不怀仁,鬼神飨德。先王之立礼也,有本有文。忠信,礼之本也;义理,礼之文也。无本不正,无文不行。礼也者,合于天时,设于地财,顺于鬼神,合于人心,理万物者也。是故天时有生也,地理有宜也,人官有能也,物曲有利也。故天不生,地不养,君子不以为礼,鬼神弗飨也。居山以鱼鳖为礼,居泽以鹿豕为礼,君子谓之不知礼。故必举其定国之数,以为礼之大经,礼之大伦。以地广狭,礼之薄厚,与年之上下。是故年虽大杀,众不匡惧。则上之制礼也节矣。礼,时为大,顺次之,体次之,宜次之,称次之。尧授舜,舜授禹;汤放桀,武王伐纣,时也。《诗》云:``匪革其犹,聿追来孝。''天地之祭,宗庙之事,父子之道,君臣之义,伦也。社稷山川之事,鬼神之祭,体也。丧祭之用,宾客之交,义也。羔豚而祭,百官皆足;大牢而祭,不必有余,此之谓称也。诸侯以龟为宝,以圭为瑞。家不宝龟,不藏圭,不台门,言有称也。礼,有以多为贵者:天子七庙,诸侯五,大夫三,士一。天子之豆二十有六,诸公十有六,诸侯十有二,上大夫八,下大夫六。诸侯七介七牢,大夫五介五牢。天子之席五重,诸侯之席三重,大夫再重。天子崩,七月而葬,五重八翣;诸侯五月而葬,三重六翣;大夫三月而葬,再重四翣。此以多为贵也。有以少为贵者:天子无介;祭天特牲;天子适诸侯,诸侯膳以犊;诸侯相朝,灌用郁鬯,无笾豆之荐;大夫聘礼以脯醢;天子一食,诸侯再,大夫、士三,食力无数;大路繁缨一就,次路繁缨七就;圭璋特,琥璜爵;鬼神之祭单席。诸侯视朝,大夫特,士旅之。此以少为贵也。有以大为贵者:宫室之量,器皿之度,棺椁之厚,丘封之大。此以大为贵也。有以小为贵者:宗庙之祭,贵者献以爵,贱者献以散,尊者举觯,卑者举角;五献之尊,门外缶,门内壶,君尊瓦甒。此以小为贵也。有以高为贵者:天子之堂九尺,诸侯七尺,大夫五尺,士三尺;天子、诸侯台门。此以高为贵也。有以下为贵者:至敬不坛,扫地而祭。天子诸侯之尊废禁,大夫、士棜禁。此以下为贵也。礼有以文为贵者:天子龙衮,诸侯黼,大夫黻,士玄衣纁裳;天子之冕,朱绿藻十有二旒,诸侯九,上大夫七,下大夫五,士三。此以文为贵也。有以素为贵者:至敬无文,父党无容,大圭不琢,大羹不和,大路素而越席,牺尊疏布幂,樿杓。此以素为贵也。孔子曰:``礼,不可不省也。''礼不同,不丰、不杀,此之谓也。盖言称也。礼之以多为贵者,以其外心者也;德发扬,诩万物,大理物博,如此,则得不以多为贵乎?故君子乐其发也。礼之以少为贵者,以其内心者也。德产之致也精微,观天子之物无可以称其德者,如此则得不以少为贵乎?是故君子慎其独也。古之圣人,内之为尊,外之为乐,少之为贵,多之为美。是故先生之制礼也,不可多也,不可寡也,唯其称也。是故,君子大牢而祭,谓之礼;匹士大牢而祭,谓之攘。管仲镂簋朱纮,山节藻棁,君子以为滥矣。晏平仲祀其先人,豚肩不揜豆;浣衣濯冠以朝,君子以为隘矣。是故君子之行礼也,不可不慎也;众之纪也,纪散而众乱。孔子曰:``我战则克,祭则受福。''盖得其道矣。君子曰:祭祀不祈,不麾蚤,不乐葆大,不善嘉事,牲不及肥大,荐不美多品。

孔子曰:``臧文仲安知礼!夏父弗綦逆祀,而弗止也。燔柴于奥,夫奥者,老妇之祭也,盛于盆,尊于瓶。礼也者,犹体也。体不备,君子谓之不成人。设之不当,犹不备也。''

礼有大有小,有显有微。大者不可损,小者不可益,显者不可掩,微者不可大也。故《经礼》三百,《曲礼》三千,其致一也。未有入室而不由户者。君子之于礼也,有所竭情尽慎,致其敬而诚若,有美而文而诚若。君子之于礼也,有直而行也,有曲而杀也,有经而等也,有顺而讨也,有摭而播也,有推而进也,有放而文也,有放而不致也,有顺而摭也。三代之礼一也,民共由之。或素或青,夏造殷因。周坐尸,诏侑武方;其礼亦然,其道一也;夏立尸而卒祭;殷坐尸。周旅酬六尸,曾子曰:``周礼其犹醵与!''

君子曰:礼之近人情者,非其至者也。郊血,大飨腥,三献爓,一献孰。是故君子之于礼也,非作而致其情也,此有由始也。是故七介以相见也,不然则已悫。三辞三让而至,不然则已蹙。故鲁人将有事于上帝,必先有事于頖宫;晋人将有事于河,必先有事于恶池;齐人将有事于泰山,必先有事于配林。三月系,七日戒,三日宿,慎之至也。故礼有摈诏,乐有相步,温之至也。

礼也者,反本修古,不忘其初者也。故凶事不诏,朝事以乐。醴酒之用,玄酒之尚。割刀之用,鸾刀之贵。莞簟之安,而稿鞂之设。是故,先王之制礼也,必有主也,故可述而多学也。

君子曰:无节于内者,观物弗之察矣。欲察物而不由礼,弗之得矣。故作事不以礼,弗之敬矣。出言不以礼,弗之信矣。故曰:``礼也者,物之致也。''是故昔先王之制礼也,因其财物而致其义焉尔。故作大事,必顺天时,为朝夕必放于日月,为高必因丘陵,为下必因川泽。是故天时雨泽,君子达亹亹焉。是故昔先王尚有德、尊有道、任有能;举贤而置之,聚众而誓之。是故因天事天,因地事地,因名山升中于天,因吉土以飨帝于郊。升中于天,而凤凰降、龟龙假;飨帝于郊,而风雨节、寒暑时。是故圣人南面而立,而天下大治。

天道至教,圣人至德。庙堂之上,罍尊在阼,牺尊在西。庙堂之下,县鼓在西,应鼓在东。君在阼,夫人在房。大明生于东,月生于西,此阴阳之分、夫妇之位也。君西酌牺象,夫人东酌罍尊。礼交动乎上,乐交应乎下,和之至也。礼也者,反其所自生;乐也者,乐其所自成。是故先王之制礼也以节事,修乐以道志。故观其礼乐,而治乱可知也。蘧伯玉曰:``君子之人达,故观其器,而知其工之巧;观其发,而知其人之知。''故曰:``君子慎其所以与人者。''

太庙之内敬矣!君亲牵牲,大夫赞币而从。君亲制祭,夫人荐盎。君亲割牲,夫人荐酒。卿、大夫从君,命妇从夫人。洞洞乎其敬也,属属乎其忠也,勿勿乎其欲其飨之也。纳牲诏于庭,血毛诏于室,羹定诏于堂,三诏皆不同位,盖道求而未之得也。设祭于堂,为祊乎外,故曰:``于彼乎?于此乎?''一献质,三献文,五献察,七献神。大飨其王事与!三牲鱼腊,四海九州岛之美味也;笾豆之荐,四时之和气也。内金,示和也。束帛加璧,尊德也。龟为前列,先知也。金次之,见情也。丹漆丝纩竹箭,与众共财也。其余无常货,各以其国之所有,则致远物也。其出也,肆夏而送之,盖重礼也。祀帝于郊,敬之至也。宗庙之祭,仁之至也。丧礼,忠之至也。备服器,仁之至也。宾客之用币,义之至也。故君子欲观仁义之道,礼其本也。

君子曰:甘受和,白受采;忠信之人,可以学礼。茍无忠信之人,则礼不虚道。是以得其人之为贵也。孔子曰:``诵《诗》三百,不足以一献。一献之礼,不足以大飨。大飨之礼,不足以大旅。大旅具矣,不足以飨帝。''毋轻议礼!子路为季氏宰。季氏祭,逮暗而祭,日不足,继之以烛。虽有强力之容、肃敬之心,皆倦怠矣。有司跛倚以临祭,其为不敬大矣。他日祭,子路与,室事交乎户,堂事交乎阶,质明而始行事,晏朝而退。孔子闻之曰:``谁谓由也而不知礼乎?

\hypertarget{header-n419}{%
\subsection{郊特牲}\label{header-n419}}

郊特牲,而社稷大牢。天子适诸侯,诸侯膳用犊;诸侯适天子,天子赐之礼大牢;贵诚之义也。故天子牲孕弗食也,祭帝弗用也。大路繁缨一就,先路三就,次路五就。郊血,大飨腥,三献爓,一献熟;至敬不飨味而贵气臭也。诸侯为宾,灌用郁鬯。灌用臭也,大飨,尚腶修而已矣。大飨,君三重席而酢焉。三献之介,君专席而酢焉。此降尊以就卑也。飨禘有乐,而食尝无乐,阴阳之义也。凡饮,养阳气也;凡食,养阴气也。故春禘而秋尝;春飨孤子,秋食耆老,其义一也。而食尝无乐。饮,养阳气也,故有乐;食,养阴气也,故无声。凡声,阳也。鼎俎奇而笾豆偶,阴阳之义也。笾豆之实,水土之品也。不敢用亵味而贵多品,所以交于旦明之义也。宾入大门而奏《肆夏》,示易以敬也。卒爵而乐阕,孔子屡叹之。奠酬而工升歌,发德也。歌者在上,匏竹在下,贵人声也。乐由阳来者也,礼由阴作者也,阴阳和而万物得。旅币无方,所以别土地之宜而节远迩之期也。龟为前列,先知也,以钟次之,以和居参之也。虎豹之皮,示服猛也。束帛加璧,往德也。庭燎之百,由齐桓公始也。大夫之奏《肆夏》也,由赵文子始也。朝觐,大夫之私觌,非礼也。大夫执圭而使,所以申信也;不敢私觌,所以致敬也;而庭实私觌,何为乎诸侯之庭?为人臣者,无外交,不敢贰君也。大夫而飨君,非礼也。大夫强而君杀之,义也;由三桓始也。天子无客礼,莫敢为主焉。君适其臣,升自阼阶,不敢有其室也。觐礼,天子不下堂而见诸侯。下堂而见诸侯,天子之失礼也,由夷王以下。诸侯之宫县,而祭以白牡,击玉磬,朱干设锡,冕而舞《大武》,乘大路,诸侯之僭礼也。台门而旅树,反坫,绣黼,丹朱中衣,大夫之僭礼也。故天子微,诸侯僭;大夫强,诸侯胁。于此相贵以等,相觌以货,相赂以利,而天下之礼乱矣。诸侯不敢祖天子,大夫不敢祖诸侯。而公庙之设于私家,非礼也,由三桓始也。

天子存二代之后,犹尊贤也,尊贤不过二代。诸侯不臣寓公。故古者寓公不继世。君之南乡,答阳之义也。臣之北面,答君也。大夫之臣不稽首,非尊家臣,以辟君也。大夫有献弗亲,君有赐不面拜,为君之答己也。乡人禓,孔子朝服立于阼,存室神也。孔子曰:``射之以乐也,何以听,何以射?''孔子曰:``士,使之射,不能,则辞以疾。县弧之义也。''孔子曰:``三日齐,一日用之,犹恐不敬;二日伐鼓,何居?''孔子曰:``绎之于库门内,祊之于东方,朝市之于西方,失之矣。''

社祭土而主阴气也。君南乡于北墉下,答阴之义也。日用甲,用日之始也。天子大社必受霜露风雨,以达天地之气也。是故丧国之社屋之,不受天阳也。薄社北牖,使阴明也。社所以神地之道也。地载万物,天垂象。取财于地,取法于天,是以尊天而亲地也,故教民美报焉。家主中溜而国主社,示本也。唯为社事,单出里。唯为社田,国人毕作。唯社,丘乘共粢盛,所以报本反始也。季春出火,为焚也。然后简其车赋,而历其卒伍,而君亲誓社,以习军旅。左之右之,坐之起之,以观其习变也;而流示之禽,而盐诸利,以观其不犯命也。求服其志,不贪其得,故以战则克,以祭则受福。

天子适四方,先柴。郊之祭也,迎长日之至也,大报天而主日也。兆于南郊,就阳位也。扫地而祭,于其质也。器用陶匏,以象天地之性也。于郊,故谓之郊。牲用骍,尚赤也;用犊,贵诚也。郊之用辛也,周之始郊日以至。卜郊,受命于祖庙,作龟于祢宫,尊祖亲考之义也。卜之日,王立于泽,亲听誓命,受教谏之义也。献命库门之内,戒百官也。大庙之命,戒百姓也。祭之日,王皮弁以听祭报,示民严上也。丧者不哭,不敢凶服,汜扫反道,乡为田烛。弗命而民听上。祭之日,王被衮以象天,戴冕,璪十有二旒,则天数也。乘素车,贵其质也。旗十有二旒,龙章而设日月,以象天也。天垂象,圣人则之。郊所以明天道也。帝牛不吉,以为稷牛。帝牛必在涤三月,稷牛唯具。所以别事天神与人鬼也。万物本乎天,人本乎祖,此所以配上帝也。郊之祭也,大报本反始也。

天子大蜡八。伊耆氏始为蜡,蜡也者,索也。岁十二月,合聚万物而索飨之也。蜡之祭也:主先啬,而祭司啬也。祭百种以报啬也。飨农及邮表畷,禽兽,仁之至、义之尽也。古之君子,使之必报之。迎猫,为其食田鼠也;迎虎,为其食田豕也,迎而祭之也。祭坊与水庸,事也。曰``土反其宅'',水归其壑,昆虫毋作,草木归其泽。皮弁素服而祭。素服,以送终也。葛带榛杖,丧杀也。蜡之祭,仁之至、义之尽也。黄衣黄冠而祭,息田夫也。野夫黄冠;黄冠,草服也。大罗氏,天子之掌鸟兽者也,诸侯贡属焉。草笠而至,尊野服也。罗氏致鹿与女,而诏客告也。以戒诸侯曰:``好田好女者亡其国。''天子树瓜华,不敛藏之种也。八蜡以记四方。四方年不顺成,八蜡不通,以谨民财也。顺成之方,其蜡乃通,以移民也。既蜡而收,民息已。故既蜡,君子不兴功。

恒豆之菹,水草之和气也;其醢,陆产之物也。加豆,陆产也;其醢,水物也。笾豆之荐,水土之品也,不敢用常亵味而贵多品,所以交于神明之义也,非食味之道也。先王之荐,可食也而不可耆也。卷冕路车,可陈也而不可好也。武壮,而不可乐也。宗庙之威,而不可安也。宗庙之器,可用也而不可便其利也,所以交于神明者,不可以同于所安乐之义也。酒醴之美,玄酒明水之尚,贵五味之本也。黼黻文绣之美,疏布之尚,反女功之始也。莞簟之安,而蒲越稿鞂之尚,明之也。大羹不和,贵其质也。大圭不琢,美其质也。丹漆雕几之美,素车之乘,尊其朴也,贵其质而已矣。所以交于神明者,不可同于所安亵之甚也。如是而后宜。鼎俎奇而笾豆偶,阴阳之义也。黄目,郁气之上尊也。黄者中也;目者气之清明者也。言酌于中而清明于外也,祭天,扫地而祭焉,于其质而已矣。酰醢之美,而煎盐之尚,贵天产也。割刀之用,而鸾刀之贵,贵其义也。声和而后断也。

冠义:始冠之,缁布之冠也。大古冠布,齐则缁之。其緌也,孔子曰:``吾未之闻也。冠而敝之可也。''适子冠于阼,以着代也。醮于客位,加有成也。三加弥尊,喻其志也。冠而字之,敬其名也。委貌,周道也。章甫,殷道也。毋追,夏后氏之道也。周弁,殷冔,夏收。三王共皮弁素积。无大夫冠礼,而有其昏礼。古者,五十而后爵,何大夫冠礼之有?诸侯之有冠礼,夏之末造也。天子之元子,士也。天下无生而贵者也。继世以立诸侯,象贤也。以官爵人,德之杀也。死而谥,今也;古者生无爵,死无谥。礼之所尊,尊其义也。失其义,陈其数,祝史之事也。故其数可陈也,其义难知也。知其义而敬守之,天子之所以治天下也。

天地合而后万物兴焉。夫昏礼,万世之始也。取于异姓,所以附远厚别也。币必诚,辞无不腆。告之以直信;信,事人也;信,妇德也。壹与之齐,终身不改。故夫死不嫁。男子亲迎,男先于女,刚柔之义也。天先乎地,君先乎臣,其义一也。执挚以相见,敬章别也。男女有别,然后父子亲,父子亲然后义生,义生然后礼作,礼作然后万物安。无别无义,禽兽之道也。婿亲御授绥,亲之也。亲之也者,亲之也。敬而亲之,先王之所以得天下也。出乎大门而先,男帅女,女从男,夫妇之义由此始也。妇人,从人者也;幼从父兄,嫁从夫,夫死从子。夫也者,夫也;夫也者,以知帅人者也。玄冕斋戒,鬼神阴阳也。将以为社稷主,为先祖后,而可以不致敬乎?共牢而食,同尊卑也。故妇人无爵,从夫之爵,坐以夫之齿。器用陶匏,尚礼然也。三王作牢用陶匏。厥明,妇盥馈。舅姑卒食,妇馂余,私之也。舅姑降自西阶,妇降自阼阶,授之室也。昏礼不用乐,幽阴之义也。乐,阳气也。昏礼不贺,人之序也。

有虞氏之祭也,尚用气;血腥爓祭,用气也。殷人尚声,臭味未成,涤荡其声;乐三阕,然后出迎牲。声音之号,所以诏告于天地之间也。周人尚臭,灌用鬯臭,郁合鬯;臭,阴达于渊泉。灌以圭璋,用玉气也。既灌,然后迎牲,致阴气也。萧合黍稷;臭,阳达于墙屋。故既奠,然后焫萧合膻芗。凡祭,慎诸此。魂气归于天,形魄归于地。故祭,求诸阴阳之义也。殷人先求诸阳,周人先求诸阴。诏祝于室,坐尸于堂,用牲于庭,升首于室。直祭,祝于主;索祭,祝于祊。不知神之所在,于彼乎?于此乎?或诸远人乎?祭于祊,尚曰求诸远者与?祊之为言倞也,肵之为言敬也。富也者福也,首也者,直也。相,飨之也。嘏,长也,大也。尸,陈也。毛血,告幽全之物也。告幽全之物者,贵纯之道也。血祭,盛气也。祭肺肝心,贵气主也。祭黍稷加肺,祭齐加明水,报阴也。取膟菺燔燎,升首,报阳也。明水涚齐,贵新也。凡涚,新之也。其谓之明水也,由主人之絜着此水也。君再拜稽首,肉袒亲割,敬之至也。敬之至也,服也。拜,服也;稽首,服之甚也;肉袒,服之尽也。祭称孝孙孝子,以其义称也;称曾孙某,谓国家也。祭祀之相,主人自致其敬,尽其嘉,而无与让也。腥肆爓腍祭,岂知神之所飨也?主人自尽其敬而已矣。举斝角,诏妥尸。古者,尸无事则立,有事而后坐也。尸,神象也。祝,将命也。缩酌用茅,明酌也。醆酒涚于清,汁献涚于醆酒;犹明清与醆酒于旧泽之酒也。祭有祈焉,有报焉,有由辟焉。齐之玄也,以阴幽思也。故君子三日齐,必见其所祭者。

\hypertarget{header-n431}{%
\subsection{内则}\label{header-n431}}

后王命冢宰,降德于众兆民。子事父母,鸡初鸣,咸盥漱,栉縰笄总,拂髦冠緌缨,端韠绅,搢笏。左右佩用,左佩纷帨、刀、砺、小觿、金燧,右佩玦、捍、管、遰、大觿、木燧,偪,屦着綦。妇事舅姑,如事父母。鸡初鸣,咸盥漱,栉縰,笄总,衣绅。左佩纷帨、刀、砺、小觿、金燧,右佩箴、管、线、纩,施縏帙,大觿、木燧、衿缨,綦屦。以适父母舅姑之所,及所,下气怡声,问衣燠寒,疾痛苛痒,而敬抑搔之。出入,则或先或后,而敬扶持之。进盥,少者奉盘,长者奉水,请沃盥,盥卒授巾。问所欲而敬进之,柔色以温之,饘酏、酒醴、芼羹、菽麦、蕡稻、黍粱、秫唯所欲,枣、栗、饴、蜜以甘之,堇、荁、枌、榆免槁薧滫以滑之,脂膏以膏之,父母舅姑必尝之而后退。男女未冠笄者,鸡初鸣,咸盥漱,栉縰,拂髦总角,衿缨,皆佩容臭,昧爽而朝,问何食饮矣。若已食则退,若未食则佐长者视具。凡内外,鸡初鸣,咸盥漱,衣服,敛枕簟,洒扫室堂及庭,布席,各从其事。孺子蚤寝晏起,唯所欲,食无时。由命士以上,父子皆异宫。昧爽而朝,慈以旨甘,日出而退,各从其事,日入而夕,慈以旨甘。父母舅姑将坐,奉席请何乡;将衽,长者奉席请何趾。少者执床与坐,御者举几,敛席与簟,县衾箧枕,敛簟而襡之。父母舅姑之衣衾簟席枕几不传,杖屦只敬之,勿敢近。敦牟卮匜,非馂莫敢用;与恒食饮,非馂,莫之敢饮食。父母在,朝夕恒食,子妇佐馂,既食恒馂,父没母存,冢子御食,群子妇佐馂如初,旨甘柔滑,孺子馂。在父母舅姑之所,有命之,应唯敬对。进退周旋慎齐,升降出入揖游,不敢哕噫、嚏咳、欠伸、跛倚、睇视,不敢唾洟;寒不敢袭,痒不敢搔;不有敬事,不敢袒裼,不涉不撅,亵衣衾不见里。父母唾洟不见,冠带垢,和灰请漱;衣裳垢,和灰请浣;衣裳绽裂,纫箴请补缀。五日,则燂汤请浴,三日具沐,其间面垢,燂潘请靧;足垢,燂汤请洗。少事长,贱事贵,共帅时。男不言内,女不言外。非祭非丧,不相授器。其相授,则女受以篚,其无篚则皆坐奠之而后取之。外内不共井,不共湢浴,不通寝席,不通乞假,男女不通衣裳,内言不出,外言不入。男子入内,不啸不指,夜行以烛,无烛则止。女子出门,必拥蔽其面,夜行以烛,无烛则止。道路:男子由右,女子由左。子妇孝者、敬者,父母舅姑之命,勿逆勿怠。若饮食之,虽不耆,必尝而待;加之衣服,虽不欲,必服而待;加之事,人待之,己虽弗欲,姑与之,而姑使之,而后复之。子妇有勤劳之事,虽甚爱之,姑纵之,而宁数休之。子妇未孝未敬,勿庸疾怨,姑教之;若不可教,而后怒之;不可怒,子放妇出,而不表礼焉。父母有过,下气怡色,柔声以谏。谏若不入,起敬起孝,说则复谏;不说,与其得罪于乡党州闾,宁孰谏。父母怒、不说,而挞之流血,不敢疾怨,起敬起孝。父母有婢子若庶子、庶孙,甚爱之,虽父母没,没身敬之不衰。子有二妾,父母爱一人焉,子爱一人焉,由衣服饮食,由执事,毋敢视父母所爱,虽父母没不衰。子甚宜其妻,父母不说,出;子不宜其妻,父母曰:``是善事我。''子行夫妇之礼焉,没身不衰。

父母虽没,将为善,思贻父母令名,必果;将为不善,思贻父母羞辱,必不果。舅没则姑老,冢妇所祭祀、宾客,每事必请于姑,介妇请于冢妇。舅姑使冢妇,毋怠,不友无礼于介妇。舅姑若使介妇,毋敢敌耦于冢妇,不敢并行,不敢并命,不敢并坐。凡妇,不命适私室,不敢退。妇将有事,大小必请于舅姑。子妇无私货,无私畜,无私器,不敢私假,不敢私与。妇或赐之饮食、衣服、布帛、佩帨、茝兰,则受而献诸舅姑,舅姑受之则喜,如新受赐,若反赐之则辞,不得命,如更受赐,藏以待乏。妇若有私亲兄弟将与之,则必复请其故,赐而后与之。适子庶子只事宗子宗妇,虽贵富,不敢以贵富入宗子之家,虽众车徒舍于外,以寡约入。子弟犹归器衣服裘衾车马,则必献其上,而后敢服用其次也;若非所献,则不敢以入于宗子之门,不敢以贵富加于父兄宗族。若富,则具二牲,献其贤者于宗子,夫妇皆齐而宗敬焉,终事而后敢私祭。

饭:黍,稷,稻,粱,白黍,黄粱,稰,穛。

膳:膷,臐,膮,醢,牛炙。醢,牛胾,醢,牛脍。羊炙,羊胾,醢,豕炙。醢,豕胾,芥酱,鱼脍。雉,兔,鹑,鷃。

饮:重醴,稻醴清糟,黍醴清糟,粱醴清糟,或以酏为醴,黍酏,浆,水,醷,滥。

酒:清、白。

羞:糗,饵,粉,酏。

食:蜗醢而菰食,雉羹;麦食,脯羹,鸡羹;析稌,犬羹,兔羹;和糁不蓼。濡豚,包苦实蓼;濡鸡,醢酱实蓼;濡鱼,卵酱实蓼;濡鳖,醢酱实蓼。腶修,蚳醢,脯羹,兔醢,糜肤,鱼醢,鱼脍,芥酱,麋腥,醢,酱,桃诸,梅诸,卵盐。

凡食齐视春时,羹齐视夏时,酱齐视秋时,饮齐视冬时。凡和,春多酸,夏多苦,秋多辛,冬多咸,调以滑甘。牛宜稌,羊宜黍,豕宜稷,犬宜粱,雁宜麦,鱼宜菰。春宜羔豚膳膏芗,夏宜腒鱐膳膏臊,秋宜犊麑膳膏腥,冬宜鲜羽膳膏膻。牛修,鹿脯,田豕脯,糜脯,麇脯,麋、鹿、田豕、麇,皆有轩,雉兔皆有芼。爵,鷃,蜩,范,芝栭,菱,椇,枣,栗,榛,柿,瓜,桃,李,梅,杏,楂,梨,姜,桂。大夫燕食,有脍无脯,有脯无脍。士不贰羹胾,庶人耆老不徒食。脍:春用葱,秋用芥、豚;春用韭,秋用蓼。脂用葱,膏用薤,三牲用藙,和用酰,兽用梅。鹑羹、鸡羹、鴽,酿之蓼。鲂鱮烝,雏烧,雉,芗无蓼。不食雏鳖,狼去肠,狗去肾,狸去正脊,兔去尻,狐去首,豚去脑,鱼去乙,鳖去丑。肉曰脱之,鱼曰作之,枣曰新之,栗曰撰之,桃曰胆之,柤梨曰攒之。牛夜鸣则庮,羊泠毛而毳、膻,狗赤股而躁、臊,鸟麃色而沙鸣、郁,豕望视而交睫、腥,马黑脊而般臂、漏,雏尾不盈握弗食,舒雁翠,鹄鸮胖,舒凫翠,鸡肝,雁肾,鸨奥,鹿胃。肉腥细者为脍,大者为轩;或曰麋鹿鱼为菹,麇为辟鸡,野豕为轩,兔为宛脾,切葱若薤,实诸酰以柔之。羹食,自诸侯以下至于庶人无等。大夫无秩膳,大夫七十而有阁,天子之阁。左达五,右达五,公侯伯于房中五,大夫于阁三,士于坫一。

凡养老:有虞氏以燕礼,夏后氏以飨礼,殷人以食礼,周人修而兼用之。凡五十养于乡,六十养于国,七十养于学,达于诸侯。八十拜君命,一坐再至,瞽亦如之,九十者使人受。五十异粻,六十宿肉,七十二膳,八十常珍,九十饮食不违寝,膳饮从于游可也。六十岁制,七十时制,八十月制,九十日修,唯绞紟衾冒,死而后制。五十始衰,六十非肉不饱,七十非帛不暖,八十非人不暖,九十虽得人不暖矣。五十杖于家,六十杖于乡,七十杖于国,八十杖于朝,九十者天子欲有问焉,则就其室以珍从。七十不俟朝,八十月告存,九十日有秩。五十不从力政,六十不与服戎,七十不与宾客之事,八十齐丧之事弗及也。五十而爵,六十不亲学,七十致政;凡自七十以上,唯衰麻为丧。凡三王养老皆引年,八十者一子不从政,九十者其家不从政;瞽亦如之。凡父母在,子虽老不坐。有虞氏养国老于上庠,养庶老于下庠;夏后氏养国老于东序,养庶老于西序;殷人养国老于右学,养庶老于左学;周人养国老于东胶,养庶老于虞庠,虞庠在国之西郊。有虞氏皇而祭,深衣而养老;夏后氏收而祭,燕衣而养老;殷人冔而祭,缟衣而养老;周人冕而祭,玄衣而养老。

曾子曰:``孝子之养老也,乐其心不违其志,乐其耳目,安其寝处,以其饮食忠养之孝子之身终,终身也者,非终父母之身,终其身也;是故父母之所爱亦爱之,父母之所敬亦敬之,至于犬马尽然,而况于人乎!''凡养老,五帝宪,三王有乞言。五帝宪,养气体而不乞言,有善则记之为惇史。三王亦宪,既养老而后乞言,亦微其礼,皆有惇史。

淳熬:煎醢,加于陆稻上,沃之以膏曰淳熬。淳毋煎醢,加于黍食上,沃之以膏曰淳毋。

炮:取豚若将,刲之刳之,实枣于其腹中,编萑以苴之,涂之以谨涂,炮之,涂皆干,擘之,濯手以摩之,去其皽,为稻粉糔溲之以为酏,以付豚煎诸膏,膏必灭之,巨镬汤以小鼎芗脯于其中,使其汤毋灭鼎,三日三夜毋绝火,而后调之以酰醢。

捣珍:取牛羊麋鹿麇之肉必脄,每物与牛若一捶,反侧之,去其饵,熟出之,去其饵,柔其肉。

渍:取牛肉必新杀者,薄切之,必绝其理;湛诸美酒,期朝而食之以醢若酰醷。

为熬:捶之,去其皽,编萑布牛肉焉,屑桂与姜以洒诸上而盐之,干而食之。施羊亦如之,施麋、施鹿、施麇皆如牛羊。欲濡肉则释而煎之以醢,欲干肉则捶而食之。

糁:取牛羊豕之肉,三如一小切之,与稻米;稻米二肉一,合以为饵煎之。

肝菺:取狗肝一,幪之,以其菺濡炙之,举焦,其菺不蓼;取稻米举糔溲之,小切狼臅膏,以与稻米为酏。

礼,始于谨夫妇,为宫室,辨外内。男子居外,女子居内,深宫固门,阍寺守之。男不入,女不出。男女不同椸枷,不敢悬于夫之楎椸,不敢藏于夫之箧笥,不敢共湢浴。夫不在,敛枕箧簟席、襡器而藏之。少事长,贱事贵,咸如之。夫妇之礼,唯及七十,同藏无间。故妾虽老,年未满五十,必与五日之御。将御者,齐,漱浣,慎衣服,栉縰笄,总角,拂髦,衿缨綦屦。虽婢妾,衣服饮食必后长者。妻不在,妾御莫敢当夕。

妻将生子,及月辰,居侧室,夫使人日再问之,作而自问之,妻不敢见,使姆衣服而对,至于子生,夫复使人日再问之,夫齐则不入侧室之门。子生,男子设弧于门左,女子设帨于门右。三日,始负子,男射女否。国君世子生,告于君,接以大牢,宰掌具。三日,卜士负之,吉者宿齐朝服寝门外,诗负之,射人以桑弧蓬矢六。射天地四方,保受乃负之,宰醴负子,赐之束帛,卜士之妻、大夫之妾,使食子。凡接子,择日,冢子则大牢,庶人特豚,士特豕,大夫少牢,国君世子大牢,其非冢子,则皆降一等。异为孺子室于宫中,择于诸母与可者,必求其宽裕慈惠、温良恭敬、慎而寡言者,使为子师,其次为慈母,其次为保母,皆居子室,他人无事不往。三月之末,择日剪发为鬌,男角女羁,否则男左女右。是日也,妻以子见于父,贵人则为衣服,由命士以下,皆漱浣,男女夙兴,沐浴衣服,具视朔食,夫入门,升自阼阶。立于阼西乡,妻抱子出自房,当楣立东面。姆先,相曰:``母某敢用时日只见孺子。''夫对曰:``钦有帅。''父执子之右手,咳而名之。妻对曰:``记有成。''遂左还,授师,子师辩告诸妇诸母名,妻遂适寝。夫告宰名,宰辩告诸男名,书曰:``某年某月某日某生。''而藏之,宰告闾史,闾史书为二,其一藏诸闾府,其一献诸州史;州史献诸州伯,州伯命藏诸州府。夫入食如养礼。世子生,则君沐浴朝服,夫人亦如之,皆立于阼阶西乡,世妇抱子升自西阶,君名之,乃降。适子庶子见于外寝,抚其首咳而名之,礼帅初,无辞。凡名子,不以日月,不以国,不以隐疾;大夫、士之子,不敢与世子同名。妾将生子,及月辰,夫使人日一问之。子生三月之末,漱浣夙齐,见于内寝,礼之如始入室;君已食,彻焉,使之特馂,遂入御。公庶子生,就侧室。三月之末,其母沐浴朝服见于君,摈者以其子见,君所有赐,君名之。众子,则使有司名之。庶人无侧室者,及月辰,夫出居群室,其问之也,与子见父之礼,无以异也。凡父在,孙见于祖,祖亦名之,礼如子见父,无辞。食子者,三年而出,见于公宫则劬。大夫之子有食母,士之妻自养其子。由命士以上及大夫之子,旬而见。冢子未食而见,必执其右手,适子庶子已食而见,必循其首。子能食食,教以右手。能言,男唯女俞。男鞶革,女鞶丝。六年教之数与方名。七年男女不同席,不共食。八年出入门户及即席饮食,必后长者,始教之让。九年教之数日。十年出就外傅,居宿于外,学书计,衣不帛襦裤,礼帅初,朝夕学幼仪,请肄简谅。十有三年学乐,诵《诗》,舞《勺》,成童舞《象》,学射御。二十而冠,始学礼,可以衣裘帛,舞《大夏》,惇行孝弟,博学不教,内而不出。三十而有室,始理男事,博学无方,孙友视志。四十始仕,方物出谋发虑,道合则服从,不可则去。五十命为大夫,服官政。七十致事。凡男拜尚左手。女子十年不出,姆教婉娩听从,执麻枲,治丝茧,织纴组紃,学女事以共衣服,观于祭祀,纳酒浆、、笾豆、菹醢,礼相助奠。十有五年而笄,二十而嫁;有故,二十三年而嫁。聘则为妻,奔则为妾。凡女拜尚右手。

\hypertarget{header-n454}{%
\subsection{玉藻}\label{header-n454}}

天子玉藻,十有二旒,前后邃延,龙卷以祭。玄端而朝日于东门之外,听朔于南门之外,闰月则阖门左扉,立于其中。皮弁以日视朝,遂以食,日中而馂,奏而食。日少牢,朔月大牢;五饮:上水、浆、酒、醴、酏。卒食,玄端而居。动则左史书之,言则右史书之,御瞽几声之上下。年不顺成,则天子素服,乘素车,食无乐。诸侯玄端以祭,裨冕以朝,皮弁以听朔于大庙,朝服以日视朝于内朝。朝,辨色始入。君日出而视之,退适路寝,听政,使人视大夫,大夫退,然后适小寝寝,释服。又朝服以食,特牲三俎祭肺,夕深衣,祭牢肉,朔月少牢,五俎四簋,子卯稷食菜羹,夫人与君同庖。

君无故不杀牛,大夫无故不杀羊,士无故不杀犬、豕。君子远庖厨,凡有血气之类,弗身践也。至于八月不雨,君不举。年不顺成,君衣布搢本,关梁不租,山泽列而不赋,土功不兴,大夫不得造车马。卜人定龟,史定墨,君定体。君羔幦虎犆;大夫齐车,鹿幦豹犆,朝车;士齐车,鹿幦豹犆。君子之居恒当户,寝恒东首。若有疾风迅雷甚雨,则必变,虽夜必兴,衣服冠而坐。日五盥,沐稷而靧粱,栉用椫栉,发曦用象栉,进禨进羞,工乃升歌。浴用二巾,上絺下绤,出杅,履蒯席,连用汤,履蒲席,衣布曦身,乃屦进饮。将适公所,宿齐戒,居外寝,沐浴,史进象笏,书思对命;既服,习容观玉声,乃出,揖私朝,辉如也,登车则有光矣。天子搢挺,方正于天下也,诸侯荼,前诎后直,让于天子也,大夫前诎后诎,无所不让也。

侍坐,则必退席;不退,则必引而去君之党。登席不由前,为躐席。徒坐不尽席尺,读书,食,则齐,豆去席尺。若赐之食而君客之,则命之祭,然后祭;先饭辩尝羞,饮而俟。若有尝羞者,则俟君之食,然后食,饭,饮而俟。君命之羞,羞近者,命之品尝之,然后唯所欲。凡尝远食,必顺近食。君未覆手,不敢飧;君既食,又饭飧,饭飧者,三饭也。君既彻,执饭与酱,乃出,授从者。凡侑食,不尽食;食于人不饱。唯水浆不祭,若祭为已侪卑。君若赐之爵,则越席再拜稽首受,登席祭之,饮卒爵而俟君卒爵,然后授虚爵。君子之饮酒也,受一爵而色洒如也,二爵而言言斯,礼已三爵而油油以退,退则坐取屦,隐辟而后屦,坐左纳右,坐右纳左。凡尊必上玄酒,唯君面尊,唯飨野人皆酒,大夫侧尊用棜,士侧尊用禁。

始冠,缁布冠,自诸侯下达,冠而敝之可也。玄冠朱组缨,天子之冠也。缁布冠缋緌,诸侯之冠也。玄冠丹组缨,诸侯之齐冠也。玄冠綦组缨,士之齐冠也。缟冠玄武,子姓之冠也。缟冠素纰,既祥之冠也。垂緌五寸,惰游之士也,玄冠缟武,不齿之服也。居冠属武,自天子下达,有事然后緌。五十不散送,亲没不髦,大帛不緌。衣冠紫緌,自鲁桓公始也。

朝玄端,夕深衣。深衣三袪,缝齐倍要,衽当旁,袂可以回肘。长中继掩尺。袷二寸,祛尺二寸,缘广寸半。以帛裹布,非礼也。士不衣织,无君者不贰采。衣正色,裳间色。非列采不入公门,振絺绤不入公门,表裘不入公门,袭裘不入公门。纩为茧,缊为袍,褝为絅,帛为褶。朝服之以缟也,自季康子始也。孔子曰:``朝服而朝,卒朔然后服之。''曰:``国家未道,则不充其服焉。''唯君有黼裘以誓省,大裘非古也。君衣狐白裘,锦衣以裼之。君之右虎裘,厥左狼裘。士不衣狐白。君子狐青裘豹褎,玄绡衣以裼之;麑裘青豻褎,绞衣以裼之;羔裘豹饰,缁衣以裼之;狐裘,黄衣以裼之。锦衣狐裘,诸侯之服也。犬羊之裘不裼,不文饰也不裼。裘之裼也,见美也。吊则袭,不尽饰也;君在则裼,尽饰也。服之袭也,充美也,是故尸袭,执玉龟袭,无事则裼,弗敢充也。

笏:天子以球玉;诸侯以象;大夫以鱼须文竹;士竹本,象可也。见于天子与射,无说笏,入大庙说笏,非古也。小功不说笏,当事免则说之。既搢必盥,虽有执于朝,弗有盥矣。凡有指画于君前,用笏造,受命于君前,则书于笏,笏毕用也,因饰焉。笏度二尺有六寸,其中博三寸,其杀六分而去一。

韠:君朱,大夫素,士爵韦。圜杀直,天子直,公侯前后方,大夫前方后挫角,士前后正。韠下广二尺,上广一尺,长三尺,其颈五寸,肩革带博二寸。一命缊韨幽衡,再命赤韨幽衡,三命赤韨葱衡。天子素带朱里终辟,而素带终辟,大夫素带辟垂,士练带率下辟,居士锦带,弟子缟带。并纽约,用组、三寸,长齐于带,绅长制,士三尺,有司二尺有五寸。子游曰:``参分带下,绅居二焉,绅韨结三齐。''大夫大带四寸。杂带,君朱绿;大夫玄华,士缁辟,二寸,再缭四寸。凡带,有率无箴功,肆束及带勤者,有事则收之,走则拥之。王后袆衣,夫人揄狄;君命屈狄,再命袆衣,一命襢衣,士褖衣。唯世妇命于奠茧,其它则皆从男子。

凡侍于君,绅垂,足如履齐,颐溜垂拱,视下而听上,视带以及袷,听乡任左。凡君召,以三节:二节以走,一节以趋。在官不俟屦,在外不俟车。士于大夫,不敢拜迎而拜送;士于尊者,先拜进面,答之拜则走。士于君所言,大夫没矣,则称谥若字,名士。与大夫言,名士字大夫。于大夫所,有公讳无私讳。凡祭不讳,庙中不讳,教学临文不讳。古之君子必佩玉,右征角,左宫羽。趋以《采齐》,行以《肆夏》,周还中规,折还中矩,进则揖之,退则扬之,然后玉锵鸣也。故君子在车,则闻鸾和之声,行则鸣佩玉,是以非辟之心,无自入也。

君在不佩玉,左结佩,右设佩,居则设佩,朝则结佩,齐则綪结佩而爵韨。凡带必有佩玉,唯丧否。佩玉有冲牙;君子无故,玉不去身,君子于玉比德焉。天子佩白玉而玄组绶,公侯佩山玄玉而朱组绶,大夫佩水苍玉而纯组绶,世子佩瑜玉而綦组绶,士佩瓀玟而缊组绶。孔子佩象环五寸,而綦组绶。

童子之节也,缁布衣锦缘,锦绅,并纽锦,束发皆朱锦也。童子不裘不帛,不屦絇,无缌服。听事不麻,无事则立主人之北面,见先生从人而入。侍食于先生异爵者,后祭先饭。客祭,主人辞曰:``不足祭也。''客飧,主人辞以疏。主人自置其酱,则客自彻之。一室之人,非宾客,一人彻。壹食之人,一人彻。凡燕食,妇人不彻。食枣桃李,弗致于核,瓜祭上环,食中弃所操。凡食果实者后君子,火孰者先君子。有庆,非君赐不贺。孔子食于季氏,不辞,不食肉而飧。

君赐车马,乘以拜赐;衣服,服以拜赐;君未有命,弗敢即乘服也。君赐,稽首,据掌致诸地;酒肉之赐,弗再拜。凡赐,君子与小人不同日。凡献于君,大夫使宰,士亲,皆再拜稽首送之。膳于君,有荤桃茢,于大夫去茢,于士去荤,皆造于膳宰。大夫不亲拜,为君之答己也。大夫拜赐而退,士待诺而退,又拜,弗答拜。大夫亲赐士,士拜受,又拜于其室。衣服,弗服以拜。敌者不在,拜于其室。凡于尊者有献,而弗敢以闻。士于大夫不承贺,下大夫于上大夫承贺。亲在,行礼于人称父,人或赐之,则称父拜之。礼不盛,服不充,故大裘不裼,乘路车不式。

父命呼,唯而不诺,手执业则投之,食在口则吐之,走而不趋。亲老,出不易方,复不过时。亲癠色容不盛,此孝子之疏节也。父殁而不能读父之书,手泽存焉尔;母殁而杯圈不能饮焉,口泽之气存焉尔。

君入门,介拂闑,大夫中枨与闑之间,士介拂枨。宾入不中门,不履阈,公事自闑西,私事自闑东。君与尸行接武,大夫继武,士中武,徐趋皆用是。疾趋则欲发而手足毋移,圈豚行不举足,齐如流,席上亦然。端行,颐溜如矢,弁行,剡剡起屦,执龟玉,举前曳踵,蹜蹜如也。凡行容愓愓,庙中齐齐,朝庭济济翔翔。君子之容舒迟,见所尊者齐遬。足容重,手容恭,目容端,口容止,声容静,头容直,气容肃,立容德,色容庄,坐如尸,燕居告温温。凡祭,容貌颜色,如见所祭者。丧容累累,色容颠颠,视容瞿瞿梅梅,言容茧茧,戎容暨暨,言容詻詻,色容厉肃,视容清明。立容辨,卑毋谄,头颈必中,山立时行,盛气颠实,扬休玉色。凡自称:天子曰予一人,伯曰天子之力臣。诸侯之于天子曰某土之守臣某,其在边邑,曰某屏之臣某。其于敌以下曰寡人,小国之君曰孤,摈者亦曰孤。上大夫曰下臣,摈者曰寡君之老,下大夫自名,摈者曰寡大夫。世子自名,摈者曰寡君之适,公子曰臣孽。士曰传遽之臣,于大夫曰外私。大夫私事使,私人摈则称名,公士摈则曰寡大夫、寡君之老。大夫有所往,必与公士为宾也。

\hypertarget{header-n470}{%
\subsection{明室位}\label{header-n470}}

昔者周公朝诸侯于明堂之位:天子负斧依南乡而立;三公,中阶之前,北面东上。诸侯之位,阼阶之东,西面北上。诸伯之国,西阶之西,东面北上。诸子之国,门东,北面东上。诸男之国,门西,北面东上。九夷之国,东门之外,西面北上。八蛮之国,南门之外,北面东上。六戎之国,西门之外,东面南上。五狄之国,北门之外,南面东上。九采之国,应门之外,北面东上。四塞,世告至。此周公明堂之位也。明堂也者,明诸侯之尊卑也。

昔殷纣乱天下,脯鬼侯以飨诸侯。是以周公相武王以伐纣。武王崩,成王幼弱,周公践天子之位以治天下;六年,朝诸侯于明堂,制礼作乐,颁度量,而天下大服;七年,致政于成王;成王以周公为有勋劳于天下,是以封周公于曲阜,地方七百里,革车千乘,命鲁公世世祀周公天以子之礼乐。

是以鲁君,孟春乘大路,载弧韣;旗十有二旒,日月之章;祀帝于郊,配以后稷。天子之礼也。季夏六月,以禘礼祀周公于大庙,牲用白牡;尊用牺象山罍;郁尊用黄目;灌用玉瓒大圭;荐用玉豆雕篹;爵用玉琖,仍雕,加以璧散璧角;俎用梡嶡;升歌《清庙》,下管《象》;朱干玉戚,冕而舞《大武》;皮弁素积,裼而舞《大夏》。昧,东夷之乐也;《任》,南蛮之乐也。纳夷蛮之乐于大庙,言广鲁于天下也。

君卷冕立于阼,夫人副袆立于房中。君肉袒迎牲于门;夫人荐豆笾。卿、大夫赞君,命妇赞夫人:各扬其职。百官废职服大刑,而天下大服。是故,夏礿、秋尝、冬烝,春社、秋省而遂大蜡,天子之祭也。

大庙,天子明堂。库门,天子皋门。雉门,天子应门。振木铎于朝,天子之政也。山节藻棁,复庙重檐,刮楹达乡,反坫出尊,崇坫康圭,疏屏;天子之庙饰也。

鸾车,有虞氏之路也。钩车,夏后氏之路也。大路,殷路也。乘路,周路也。有虞氏之旗,夏后氏之绥,殷之大白,周之大赤。夏后氏骆马,黑鬣。殷人白马,黑首。周人黄马,蕃鬣。夏后氏,牲尚黑,殷白牡,周骍刚。

泰,有虞氏之尊也。山罍,夏后氏之尊也。着,殷尊也。牺象,周尊也。爵,夏后氏以琖,殷以斝,周以爵。灌尊,夏后氏以鸡夷。殷以斝,周以黄目。其勺,夏后氏以龙勺,殷以疏勺,周以蒲勺。土鼓蒉桴苇龠,伊耆氏之乐也。拊搏玉磬揩击,大琴大瑟,中琴小瑟,四代之乐器也。

鲁公之庙,文世室也。武公之庙,武世室也。米廪,有虞氏之庠也;序,夏后氏之序也;瞽宗,殷学也;頖宫,周学也。

崇鼎,贯鼎,大璜,封父龟,天子之器也。越棘,大弓,天子之戎器也。夏后氏之鼓,足。殷,楹鼓;周,县鼓。垂之和钟,叔之离磬,女娲之笙簧。夏后氏之龙簨虡,殷之崇牙,周之璧翣。

有虞氏之两敦,夏后氏之四连,殷之六瑚,周之八簋。俎,有虞氏以梡,夏后氏以嶡,殷以椇,周以房俎。夏后氏以楬豆,殷玉豆,周献豆。有虞氏服韨,夏后氏山,殷火,周龙章。有虞氏祭首,夏后氏祭心,殷祭肝,周祭肺。夏后氏尚明水,殷尚醴,周尚酒。有虞氏官五十,夏后氏官百,殷二百,周三百。有虞氏之绥,夏后氏之绸练,殷之崇牙,周之璧翣。

凡四代之服、器、官,鲁兼用之。是故,鲁,王礼也,天下传之久矣。君臣,未尝相弒也;礼乐刑法政俗,未尝相变也,天下以为有道之国。是故,天下资礼乐焉。

\hypertarget{header-n484}{%
\subsection{丧服小记}\label{header-n484}}

斩衰,括发以麻;为母,括发以麻,免而以布。齐衰,恶笄以终丧。男子冠而妇人笄,男子免而妇人髽。其义:为男子则免,为妇人则髽。苴杖,竹也;削杖,桐也。祖父卒,而后为祖母后者三年。为父母,长子稽颡。大夫吊之,虽缌必稽颡。妇人为夫与长子稽颡,其余则否。男主必使同姓,妇主必使异姓。为父后者为出母无服。亲亲,以三为五,以五为九。上杀,下杀,旁杀,而亲毕矣。王者禘其祖之所自出,以其祖配之,而立四庙。庶子王,亦如之。别子为祖,继别为宗,继祢者为小宗。有五世而迁之宗,其继高祖者也。是故,祖迁于上,宗易于下。尊祖故敬宗,敬宗所以尊祖祢也。庶子不祭祖者,明其宗也。庶子不为长子斩,不继祖与祢故也。庶子不祭殇与无后者,殇与无后者从祖祔食。庶子不祭祢者,明其宗也。亲亲尊尊长长,男女之有别,人道之大者也。

从服者,所从亡则已。属从者,所从虽没也服。妾从女君而出,则不为女君之子服。礼不王不禘。世子不降妻之父母;其为妻也,与大夫之适子同。父为士,子为天子诸侯,则祭以天子诸侯,其尸服以士服。父为天子诸侯,子为士,祭以士,其尸服以士服。妇当丧而出,则除之。为父母丧,未练而出,则三年。既练而出,则已。未练而反,则期;既练而反,则遂之。

再期之丧,三年也;期之丧,二年也。九月七月之丧,三时也;五月之丧,二时也;三月之丧,一时也。故期而祭,礼也;期而除丧,道也。祭不为除丧也。三年而后葬者必再祭,其祭之间不同时而除丧。大功者主人之丧,有三年者,则必为之再祭。朋友,虞祔而已。士妾有子,而为之缌,无子则已。生不及祖父母诸父昆弟,而父税丧,己则否。降而在缌小功者,则税之。为君之父母、妻、长子,君已除丧而后闻丧,则不税。近臣,君服斯服矣;其余,从而服,不从而税。君虽未知丧,臣服已。

虞,杖不入于室;祔,杖不升于堂。为君母后者,君母卒,则不为君母之党服。绖杀五分而去一,杖大如绖。妾为君之长子与女君同。除丧者,先重者;易服者,易轻者。无事不辟庙门。哭皆于其次。复与书铭,自天子达于士,其辞一也。男子称名,妇人书姓与伯仲,如不知姓则书氏。

斩衰之葛与齐衰之麻同。齐衰之葛与大功之麻同。麻同,皆兼服之。报葬者报虞,三月而后卒哭。父母之丧偕,先葬者不虞祔,待后事。其葬,服斩衰。\_

大夫降其庶子,其孙不降其父。大夫不主士之丧。为慈母之父母无服。夫为人后者,其妻为舅姑大功。士祔于大夫则易牲。继父不同居也者;必尝同居。皆无主后。同财而祭其祖祢为同居;有主后者为异居。哭朋友者于门外之右南面。祔葬者不筮宅。士大夫不得祔于诸侯,祔于诸祖父之为士大夫者,其妻祔于诸祖姑,妾祔于妾祖姑;亡则中一以上而祔。祔必以其昭穆。诸侯不得祔于天子,天子、诸侯、大夫可以祔于士。

为母之君母,母卒则不服。宗子,母在为妻禫。为慈母后者,为庶母可也,为祖庶母可也。为父母、妻、长子禫。慈母与妾母,不世祭也。丈夫冠而不为殇,妇人笄而不为殇。为殇后者,以其服服之。久而不葬者,唯主丧者不除;其余以麻终月数者,除丧则已。箭笄终丧三年。齐衰三月与大功同者,绳屦。练,筮日筮尸,视濯,皆要绖杖绳屦。有司告具,而后去杖。筮日筮尸,有司告事毕而后杖,拜送宾。大祥,吉服而筮尸。庶子在父之室,则为其母不禫。庶子不以杖即位。父不主庶子之丧,则孙以杖即位可也。父在,庶子为妻以杖即位可也。诸侯吊于异国之臣,则其君为主。诸侯吊,必皮弁锡衰。所吊虽已葬,主人必免。主人未丧服,则君于不锡衰。养有疾者不丧服,遂以主其丧。非养者入主人之丧,则不易己之丧服。养尊者必易服,养卑者否。妾无妾祖姑者,易牲而祔于女君可也。妇之丧、虞、卒哭,其夫若子主之。祔,则舅主之。士不摄大夫。士摄大夫,唯宗子。主人未除丧,有兄弟自他国至,则主人不免而为主。

陈器之道,多陈之而省纳之可也;省陈之而尽纳之可也。奔兄弟之丧,先之墓而后之家,为位而哭。所知之丧,则哭于宫而后之墓。父不为众子次于外。与诸侯为兄弟者服斩。下殇小功,带,澡麻不绝本,诎而反以报之。

妇祔于祖姑,祖姑有三人,则祔于亲者。其妻为大夫而卒,而后其夫不为大夫,而祔于其妻则不易牲;妻卒而后夫为大夫,而祔于其妻,则以大夫牲。为父后者,为出母无服。无服也者,丧者不祭故也。妇人不为主而杖者:姑在为夫杖,母为长子削杖。女子子在室为父母,其主丧者不杖,则子一人杖。

缌小功,虞卒哭则免。既葬而不报虞,则虽主人皆冠,及虞则皆免。为兄弟既除丧已。及其葬也,反服其服。报虞卒哭则免。如不报虞则除之。远葬者比反哭者皆冠,及郊而后免反哭。君吊,虽不当免时也,主人必免,不散麻。虽异国之君,免也。亲者皆免。除殇之丧者,其祭也必玄。除成丧者,其祭也朝服缟冠。

奔父之丧,括发于堂上,袒降踊,袭绖于东方。奔母之丧,不括发,袒于堂上,降踊,袭免于东方。绖即位成踊,出门哭止。三日而五哭三袒。适妇不为舅后者,则姑为之小功。

\hypertarget{header-n498}{%
\subsection{大传}\label{header-n498}}

礼:不王不禘。王者禘其祖之所自出,以其祖配之。诸侯及其大祖,大夫士有大事,省于其君,干祫,及其高祖。

牧之野,武王之大事也。既事而退,柴于上帝,祈于社,设奠于牧室。遂率天下诸侯,执豆笾,逡奔走;追王大王亶父、王季历、文王昌;不以卑临尊也。上治祖祢,尊尊也;下治子孙,亲亲也;旁治昆弟,合族以食,序以昭缪,别之以礼义,人道竭矣。

圣人南面而听天下,所且先者五,民不与焉。一曰治亲,二曰报功,三曰举贤,四曰使能,五曰存爱。五者一得于天下,民无不足、无不赡者。五者,一物纰缪,民莫得其死。圣人南面而治天下,必自人道始矣。

立权度量,考文章,改正朔,易服色,殊徽号,异器械,别衣服,此其所得与民变革者也。其不可得变革者则有矣:亲亲也,尊尊也,长长也,男女有别,此其不可得与民变革者也。

同姓从宗,合族属;异姓主名,治际会。名著,而男女有别。其夫属乎父道者,妻皆母道也;其夫属乎子道者,妻皆妇道也。谓弟之妻``妇''者,是嫂亦可谓之``母''乎?名者人治之大者也,可无慎乎?四世而缌,服之穷也;五世袒免,杀同姓也。六世,亲属竭矣。其庶姓别于上,而戚单于下,昏姻可以通乎?系之以姓而弗别,缀之以食而弗殊,虽百世而昏姻不通者,周道然也。

服术有六:一曰亲亲,二曰尊尊,三曰名,四曰出入,五曰长幼,六曰从服。从服有六:有属从,有徒从,有从有服而无服,有从无服而有服,有从重而轻,有从轻而重。

自仁率亲,等而上之,至于祖,名曰轻。自义率祖,顺而下之,至于祢,名曰重。一轻一重,其义然也。

君有合族之道,族人不得以其戚戚君,位也。

庶子不祭,明其宗也。庶子不得为长子三年,不继祖也。别子为祖,继别为宗,继祢者为小宗。有百世不迁之宗,有五世则迁之宗。百世不迁者,别子之后也;宗其继别子者,百世不迁者也。宗其继高祖者,五世则迁者也。尊祖故敬宗。敬宗,尊祖之义也。有小宗而无大宗者,有大宗而无小宗者,有无宗亦莫之宗者,公子是也。公子有宗道:公子之公,为其士大夫之庶者,宗其士大夫之适者,公子之宗道也。绝族无移服,亲者属也。

自仁率亲,等而上之,至于祖;自义率祖,顺而下之,至于祢。是故,人道亲亲也。亲亲故尊祖,尊祖故敬宗,敬宗故收族,收族故宗庙严,宗庙严故重社稷,重社稷故爱百姓,爱百姓故刑罚中,刑罚中故庶民安,庶民安故财用足,财用足故百志成,百志成故礼俗刑,礼俗刑然后乐。《诗》云:``不显不承,无斁于人斯'',此之谓也。

\hypertarget{header-n511}{%
\subsection{少仪}\label{header-n511}}

闻始见君子者,辞曰:``某固愿闻名于将命者。''不得阶主。敌者曰:``某固愿见。''罕见曰:``闻名''。亟见曰:``朝夕''。瞽曰:``闻名''。适有丧者曰:``比''。童子曰:``听事''。适公卿之丧,则曰:``听役于司徒''。君将适他,臣如致金玉货贝于君,则曰:``致马资于有司'';敌者曰:``赠从者''。臣致禭于君,则曰:``致废衣于贾人'';敌者曰:``襚''。亲者兄弟不,以襚进。臣为君丧,纳货贝于君,则曰:``纳甸于有司''。赗马入庙门;赙马与其币,大白兵车,不入庙门。赙者既致命,坐委之,摈者举之。

主人无亲受也。受立,授立不坐。性之直者则有之矣。始入而辞,曰:``辞矣''。即席,曰:``可矣''。排阖说屦于户内者,一人而已矣。有尊长在则否。问品味曰:``子亟食于某乎?''问道艺曰:``子习于某乎?''、``子善于某乎?''不疑在躬,不度民械,不愿于大家,不訾重器。泛扫曰扫,扫席前曰拚;拚席不以鬣。执箕膺鬛。不贰问。问卜筮曰:``义与?志与?''义则可问,志则否。

尊长于己逾等,不敢问其年。燕见不将命。遇于道,见则面,不请所之。丧俟事不特吊。侍坐弗使,不执琴瑟,不画地,手无容,不翣也。寝则坐而将命。侍射则约矢,侍投则拥矢。胜则洗而以请,客亦如之。不角,不擢马。执君之乘车则坐。仆者右带剑,负良绥,申之面,拖诸幦,以散绥升,执辔然后步。请见不请退。朝廷曰退,燕游曰归,师役曰罢。侍坐于君子,君子欠伸,运笏,泽剑首,还屦,问日之蚤莫,虽请退可也。事君者量而后入,不入而后量;凡乞假于人,为人从事者亦然。然,故上无怨,而下远罪也。不窥密,不旁狎,不道旧故,不戏色。为人臣下者,有谏而无讪,有亡而无疾;颂而无谄,谏而无骄;怠则张而相之,废则扫而更之;谓之社稷之役。

毋拔来,毋报往,毋渎神,毋循枉,毋测未至。士依于德,游于艺;工依于法,游于说。毋訾衣服成器,毋身质言语。言语之美,穆穆皇皇;朝廷之美,济济翔翔;祭祀之美,齐齐皇皇;车马之美,匪匪翼翼;鸾和之美,肃肃雍雍。问国君之子长幼,长,则曰:``能从社稷之事矣'';幼,则曰:``能御'',``未能御''。问大夫之子长幼,长,则曰:``能从乐人之事矣'';幼,则曰:``能正于乐人'',未能正于乐人''。问士之子长幼,长,则曰:``能耕矣'';幼,则曰:``能负薪''、``未能负薪''。执玉执龟策不趋,堂上不趋,城上不趋。武车不式;介者不拜。

妇人吉事,虽有君赐,肃拜。为尸坐,则不手拜,肃拜;为丧主则不手拜。葛绖而麻带。取俎进俎不坐。执虚如执盈,入虚如有人。凡祭于室中堂上无跣,燕则有之。未尝不食新。

仆于君子,君子升下则授绥;始乘则式;君子下行,然后还立。乘贰车则式,佐车则否。贰车者,诸侯七乘,上大夫五乘,下大夫三乘。有贰车者之乘马服车不齿。观君子之衣服,服剑,乘马,弗贾。

其以乘壶酒,束修,一犬赐人,若献人,则陈酒执修以将命,亦曰乘壶酒,束修,一犬。其以鼎肉,则执以将命。其禽加于一双,则执一双以将命,委其余。犬则执绁;守犬,田犬,则授摈者,既受,乃问犬名。牛则执纼,马则执靮,皆右之。臣则左之。车则说绥,执以将命。甲若有以前之,则执以将命;无以前之,则袒櫜奉胄。哭则执盖。弓则以左手屈韣执拊。剑则启椟盖袭之,加夫桡与剑焉。笏、书、修、苞苴、弓、茵、席、枕、几、颖、杖、琴、瑟、戈有刃者椟、策、龠,其执之皆尚左手。刀却刃授颖。削授拊。凡有刺刃者,以授人则辟刃。

乘兵车,出先刃,入后刃,军尚左,卒尚右。宾客主恭,祭祀主敬,丧事主哀,会同主诩。军旅思险,隐情以虞。

燕侍食于君子,则先饭而后已;毋放饭,毋流歠;小饭而亟之;数毋为口容。客自彻,辞焉则止。客爵居左,其饮居右;介爵、酢爵、僎爵皆居右。羞濡鱼者进尾;冬右腴,夏右鳍;祭膴。凡齐,执之以右,居之于左。赞币自左,诏辞自右。酌尸之仆,如君之仆。其在车则左执辔右受爵,祭左右轨范乃饮。凡羞有俎者,则于俎内祭。君子不食圂腴。小子走而不趋,举爵则坐祭立饮。凡洗必盥。牛羊之肺,离而不提心。凡羞有湇者,不以齐。为君子择葱薤,则绝其本末。羞首者,进喙祭耳。尊者以酌者之左为上尊。尊壶者面其鼻。饮酒者、禨者、醮者,有折俎不坐。未步爵,不尝羞。牛与羊鱼之腥,聂而切之为脍;麋鹿为菹,野豕为轩,皆聂而不切;麇为辟鸡,兔为宛脾,皆聂而切之。切葱若薤,实之酰以柔之。其有折俎者,取祭肺,反之,不坐;燔亦如之。尸则坐。

衣服在躬,而不知其名为罔。其未有烛而有后至者,则以在者告。道瞽亦然。凡饮酒为献主者,执烛抱燋,客作而辞,然后以授人。执烛不让,不辞,不歌。洗盥执食饮者勿气,有问焉,则辟咡而对。为人祭曰致福;为己祭而致膳于君子曰膳;祔练曰告。凡膳告于君子,主人展之,以授使者于阼阶之南,南面再拜稽首送;反命,主人又再拜稽首。其礼:大牢则以牛左肩、臂臑、折九个,少牢则以羊左肩七个,特豕则以豕左肩五个。国家靡敝,则车不雕几,甲不组縢,食器不刻镂,君子不履丝屦,马不常秣。

\hypertarget{header-n524}{%
\subsection{学记}\label{header-n524}}

发虑宪,求善良,足以謏闻,不足以动众;就贤体远,足以动众,未足以化民。君子如欲化民成俗,其必由学乎!

玉不琢,不成器;人不学,不知道。是故古之王者建国君民,教学为先。《兑命》曰:``念终始典于学。''其此之谓乎!

虽有嘉肴,弗食,不知其旨也;虽有至道,弗学,不知其善也。故学然后知不足,教然后知困。知不足,然后能自反也;知困,然后能自强也,故曰:教学相长也。《兑命》曰:``学学半。''其此之谓乎!

古之教者,家有塾,党有庠,术有序,国有学。比年入学,中年考校。一年视离经辨志,三年视敬业乐群,五年视博习亲师,七年视论学取友,谓之小成;九年知类通达,强立而不反,谓之大成。夫然后足以化民易俗,近者说服,而远者怀之,此大学之道也。《记》曰:``蛾子时术之。''其此之谓乎!

大学始教,皮弁祭菜,示敬道也;《宵雅》肄三,官其始也;入学鼓箧,孙其业也;夏楚二物,收其威也;未卜禘不视学,游其志也;时观而弗语,存其心也;幼者听而弗问,学不躐等也。此七者,教之大伦也。《记》曰:``凡学官先事,士先志。''其此之谓乎!

大学之教也时,教必有正业,退息必有居。学,不学操缦,不能安弦;不学博依,不能安《诗》;不学杂服,不能安礼;不兴其艺,不能乐学。故君子之于学也,藏焉,修焉,息焉,游焉。夫然,故安其学而亲其师,乐其友而信其道。是以虽离师辅而不反也。《兑命》曰:``敬孙务时敏,厥修乃来。''其此之谓乎!

今之教者,呻其占毕,多其讯,言及于数,进而不顾其安,使人不由其诚,教人不尽其材;其施之也悖,其求之也佛。夫然,故隐其学而疾其师,苦其难而不知其益也,虽终其业,其去之必速。教之不刑,其此之由乎!

大学之法,禁于未发之谓豫,当其可之谓时,不陵节而施之谓孙,相观而善之谓摩。此四者,教之所由兴也。

发然后禁,则捍格而不胜;时过然后学,则勤苦而难成;杂施而不孙,则坏乱而不修;独学而无友,则孤陋而寡闻;燕朋逆其师;燕辟废其学。此六者,教之所由废也。

君子既知教之所由兴,又知教之所由废,然后可以为人师也。故君子之教喻也,道而弗牵,强而弗抑,开而弗达。道而弗牵则和,强而弗抑则易,开而弗达则思;和易以思,可谓善喻矣。

学者有四失,教者必知之。人之学也,或失则多,或失则寡,或失则易,或失则止。此四者,心之莫同也。知其心,然后能救其失也。教也者,长善而救其失者也。

善歌者,使人继其声;善教者,使人继其志。其言也约而达,微而臧,罕譬而喻,可谓继志矣。

君子知至学之难易,而知其美恶,然后能博喻;能博喻然后能为师;能为师然后能为长;能为长然后能为君。故师也者,所以学为君也。是故择师不可不慎也。《记》曰:``三王四代唯其师。''此之谓乎!

凡学之道,严师为难。师严然后道尊,道尊然后民知敬学。是故君之所不臣于其臣者二:当其为尸则弗臣也,当其为师则弗臣也。大学之礼,虽诏于天子,无北面;所以尊师也。

善学者,师逸而功倍,又从而庸之;不善学者,师勤而功半,又从而怨之。善问者,如攻坚木,先其易者,后其节目,及其久也,相说以解;不善问者反此。善待问者,如撞钟,叩之以小者则小鸣,叩之以大者则大鸣,待其从容,然后尽其声;不善答问者反此。此皆进学之道也。

记问之学,不足以为人师。必也听语乎,力不能问,然后语之;语之而不知,虽舍之可也。

良冶之子,必学为裘;良弓之子,必学为箕;始驾者反之,车在马前。君子察于此三者,可以有志于学矣。

古之学者:比物丑类。鼓无当于五声,五声弗得不和。水无当于五色,五色弗得不章。学无当于五官。五官弗得不治。师无当于五服,五服弗得不亲。

君子曰:大德不官,大道不器,大信不约,大时不齐。察于此四者,可以有志于学矣。

三王之祭川也,皆先河而后海;或源也,或委也。此之谓务本。

\hypertarget{header-n547}{%
\subsection{乐记}\label{header-n547}}

凡音之起,由人心生也。人心之动,物使之然也。感于物而动,故形于声。声相应,故生变;变成方,谓之音;比音而乐之,及干戚羽旄,谓之乐。乐者,音之所由生也;其本在人心之感于物也。是故其哀心感者,其声以杀。其乐心感者,其声噍以缓。其喜心感者,其声发以散。其怒心感者,其声粗以厉。其敬心感者,其声直以廉。其爱心感者,其声和以柔。六者,非性也,感于物而后动。是故先王慎所以感之者。故礼以道其志,乐以和其声,政以一其行,刑以防其奸。礼乐刑政,其极一也;所以同民心而出治道也。凡音者,生人心者也。情动于中,故形于声。声成文,谓之音。是故治世之音安以乐,其政和。乱世之音怨以怒,其政乖。亡国之音哀以思,其民困。声音之道,与政通矣。宫为君,商为臣,角为民,征为事,羽为物。五者不乱,则无怗懘之音矣。宫乱则荒,其君骄。商乱则陂,其官坏。角乱则忧,其民怨。征乱则哀,其事勤。羽乱则危,其财匮。五者皆乱,迭相陵,谓之慢。如此,则国之灭亡无日矣。郑卫之音,乱世之音也,比于慢矣。桑间濮上之音,亡国之音也,其政散,其民流,诬上行私而不可止也。凡音者,生于人心者也。乐者,通伦理者也。是故知声而不知音者,禽兽是也;知音而不知乐者,众庶是也。唯君子为能知乐。是故审声以知音,审音以知乐,审乐以知政,而治道备矣。是故不知声者不可与言音,不知音者不可与言乐。知乐则几于礼矣。礼乐皆得,谓之有德。德者得也。是故乐之隆,非极音也。食飨之礼,非致味也。清庙之瑟,朱弦而疏越,壹倡而三叹,有遗音者矣。大飨之礼,尚玄酒而俎腥鱼,大羹不和,有遗味者矣。是故先王之制礼乐也,非以极口腹耳目之欲也,将以教民平好恶而反人道之正也。人生而静,天之性也;感于物而动,性之欲也。物至知知,然后好恶形焉。好恶无节于内,知诱于外,不能反躬,天理灭矣。夫物之感人无穷,而人之好恶无节,则是物至而人化物也。人化物也者,灭天理而穷人欲者也。于是有悖逆诈伪之心,有淫泆作乱之事。是故强者胁弱,众者暴寡,知者诈愚,勇者苦怯,疾病不养,老幼孤独不得其所,此大乱之道也。是故先王之制礼乐,人为之节;衰麻哭泣,所以节丧纪也;钟鼓干戚,所以和安乐也;昏姻冠笄,所以别男女也;射乡食飨,所以正交接也。礼节民心,乐和民声,政以行之,刑以防之,礼乐刑政,四达而不悖,则王道备矣。乐者为同,礼者为异。同则相亲,异则相敬,乐胜则流,礼胜则离。合情饰貌者礼乐之事也。礼义立,则贵贱等矣;乐文同,则上下和矣;好恶着,则贤不肖别矣。刑禁暴,爵举贤,则政均矣。仁以爱之,义以正之,如此,则民治行矣。乐由中出,礼自外作。乐由中出故静,礼自外作故文。大乐必易,大礼必简。乐至则无怨,礼至则不争。揖让而治天下者,礼乐之谓也。暴民不作,诸侯宾服,兵革不试,五刑不用,百姓无患,天子不怒,如此,则乐达矣。合父子之亲,明长幼之序,以敬四海之内天子如此,则礼行矣。大乐与天地同和,大礼与天地同节。和故百物不失,节故祀天祭地,明则有礼乐,幽则有鬼神。如此,则四海之内,合敬同爱矣。礼者殊事合敬者也;乐者异文合爱者也。礼乐之情同,故明王以相沿也。故事与时并,名与功偕。故钟鼓管磬,羽龠干戚,乐之器也。屈伸俯仰,缀兆舒疾,乐之文也。簠簋俎豆,制度文章,礼之器也。升降上下,周还裼袭,礼之文也。故知礼乐之情者能作,识礼乐之文者能述。作者之谓圣,述者之谓明;明圣者,述作之谓也。乐者,天地之和也;礼者,天地之序也。和故百物皆化;序故群物皆别。乐由天作,礼以地制。过制则乱,过作则暴。明于天地,然后能兴礼乐也。论伦无患,乐之情也;欣喜欢爱,乐之官也。中正无邪,礼之质也,庄敬恭顺。礼之制也。若夫礼乐之施于金石,越于声音,用于宗庙社稷,事乎山川鬼神,则此所与民同也。

王者功成作乐,治定制礼。其功大者其乐备,其治辩者其礼具。干戚之舞非备乐也,孰亨而祀非达礼也。五帝殊时,不相沿乐;三王异世,不相袭礼。乐极则忧,礼粗则偏矣。及夫敦乐而无忧,礼备而不偏者,其唯大圣乎?天高地下,万物散殊,而礼制行矣。流而不息,合同而化,而乐兴焉。春作夏长,仁也;秋敛冬藏,义也。仁近于乐,义近于礼。乐者敦和,率神而从天,礼者别宜,居鬼而从地。故圣人作乐以应天,制礼以配地。礼乐明备,天地官矣。天尊地卑,君臣定矣。卑高已陈,贵贱位矣。动静有常,小大殊矣。方以类聚,物以群分,则性命不同矣。在天成象,在地成形;如此,则礼者天地之别也。地气上齐,天气下降,阴阳相摩,天地相荡,鼓之以雷霆,奋之以风雨,动之以四时,暖之以日月,而百化兴焉。如此则乐者天地之和也。化不时则不生,男女无辨则乱升;天地之情也。及夫礼乐之极乎天而蟠乎地,行乎阴阳而通乎鬼神;穷高极远而测深厚。乐着大始,而礼居成物。着不息者天也,着不动者地也。一动一静者天地之间也。故圣人曰礼乐云。昔者,舜作五弦之琴以歌南风,夔始制乐以赏诸侯。故天子之为乐也,以赏诸侯之有德者也。德盛而教尊,五谷时熟,然后赏之以乐。故其治民劳者,其舞行缀远;其治民逸者,其舞行缀短。故观其舞,知其德;闻其谥,知其行也。《大章》,章之也。《咸池》,备矣。《韶》,继也。《夏》,大也。殷周之乐,尽矣。天地之道,寒暑不时则疾,风雨不节则饥。教者,民之寒暑也;教不时则伤世。事者民之风雨也;事不节则无功。然则先王之为乐也。以法治也,善则行象德矣。夫豢豕为酒,非以为祸也,而狱讼益繁,则酒之流生祸也。是故先王因为酒礼,壹献之礼,宾主百拜,终日饮酒而不得醉焉;此先王之所以备酒祸也。故酒食者所以合欢也;乐者所以象德也;礼者所以缀淫也。是故先王有大事,必有礼以哀之;有大福,必有礼以乐之。哀乐之分,皆以礼终。乐也者,圣人之所乐也,而可以善民心,其感人深,其移风易俗,故先王着其教焉。夫民有血气心知之性,而无哀乐喜怒之常,应感起物而动,然后心术形焉。是故志微杀之音作,而民思忧。啴谐慢易、繁文简节之音作,而民康乐。粗厉猛起、奋末广贲之音作,而民刚毅。廉直、劲正、庄诚之音作,而民肃敬。宽裕肉好、顺成和动之音作,而民慈爱。流辟邪散、狄成涤滥之音作,而民淫乱。是故先王本之情性,稽之度数,制之礼义。合生气之和,道五常之行,使之阳而不散,阴而不密,刚气不怒,柔气不慑,四畅交于中而发作于外,皆安其位而不相夺也;然后立之学等,广其节奏,省其文采,以绳德厚。律小大之称,比终始之序,以象事行。使亲疏贵贱、长幼男女之理,皆形见于乐,故曰:``乐观其深矣。''土敝则草木不长,水烦则鱼鳖不大,气衰则生物不遂,世乱则礼慝而乐淫。是故其声哀而不庄,乐而不安,慢易以犯节,流湎以忘本。广则容奸,狭则思欲,感条畅之气而灭平和之德。是以君子贱之也。凡奸声感人,而逆气应之;逆气成象,而淫乐兴焉。正声感人,而顺气应之;顺气成象,而和乐兴焉。倡和有应,回邪曲直,各归其分;而万物之理,各以其类相动也。是故君子反情以和其志,比类以成其行。奸声乱色,不留聪明;淫乐慝礼,不接心术。惰慢邪辟之气不设于身体,使耳目鼻口、心知百体皆由顺正以行其义。然后发以声音,而文以琴瑟,动以干戚,饰以羽旄,从以箫管。奋至德之光,动四气之和,以着万物之理。是故清明象天,广大象地,终始象四时,周还象风雨。五色成文而不乱,八风从律而不奸,百度得数而有常。小大相成,终始相生。倡和清浊,迭相为经。故乐行而伦清,耳目聪明,血气和平,移风易俗,天下皆宁。故曰:乐者乐也。君子乐得其道,小人乐得其欲。以道制欲,则乐而不乱;以欲忘道,则惑而不乐。是故君子反情以和其志,广乐以成其教,乐行而民乡方,可以观德矣。德者性之端也。乐者德之华也。金石丝竹,乐之器也。诗言其志也,歌咏其声也,舞动其容也。三者本于心,然后乐气从之。是故情深而文明,气盛而化神。和顺积中而英华发外,唯乐不可以为伪。

乐者,心之动也;声者,乐之象也。文采节奏,声之饰也。君子动其本,乐其象,然后治其饰。是故先鼓以警戒,三步以见方,再始以着往,复乱以饬归。奋疾而不拔,极幽而不隐。独乐其志,不厌其道;备举其道,不私其欲。是故情见而义立,乐终而德尊。君子以好善,小人以听过。故曰:生民之道,乐为大焉。乐也者施也;礼也者报也。乐,乐其所自生;而礼,反其所自始。乐章德,礼报情反始也。所谓大辂者,天子之车也。龙旗九旒,天子之旌也。青黑缘者,天子之宝龟也。从之以牛羊之群,则所以赠诸侯也。乐也者,情之不可变者也。礼也者,理之不可易者也。乐统同,礼辨异,礼乐之说,管乎人情矣。穷本知变,乐之情也;着诚去伪,礼之经也。礼乐偩天地之情,达神明之德,降兴上下之神,而凝是精粗之体,领父子君臣之节。是故大人举礼乐,则天地将为昭焉。天地欣合,阴阳相得,煦妪覆育万物,然后草木茂,区萌达,羽翼奋,角觡生,蛰虫昭苏,羽者妪伏,毛者孕鬻,胎生者不殰,而卵生者不殈,则乐之道归焉耳。乐者,非谓黄钟大吕弦歌干扬也,乐之末节也,故童者舞之。铺筵席,陈尊俎,列笾豆,以升降为礼者,礼之末节也,故有司掌之。乐师辨乎声诗,故北面而弦;宗祝辨乎宗庙之礼,故后尸;商祝辨乎丧礼,故后主人。是故德成而上,艺成而下;行成而先,事成而后。是故先王有上有下,有先有后,然后可以有制于天下也。

魏文侯问于子夏曰:``吾端冕而听古乐,则唯恐卧;听郑卫之音,则不知倦。敢问:古乐之如彼何也?新乐之如此何也?''子夏对曰:``今夫古乐,进旅退旅,和正以广。弦匏笙簧,会守拊鼓,始奏以文,复乱以武,治乱以相,讯疾以雅。君子于是语,于是道古,修身及家,平均天下。此古乐之发也。今夫新乐,进俯退俯,奸声以滥,溺而不止;及优侏儒,糅杂子女,不知父子。乐终不可以语,不可以道古。此新乐之发也。今君之所问者乐也,所好者音也!夫乐者,与音相近而不同。''文侯曰:``敢问何如?''子夏对曰:``夫古者,天地顺而四时当,民有德而五谷昌,疾疢不作而无妖祥,此之谓大当。然后圣人作为父子君臣,以为纪纲。纪纲既正,天下大定。天下大定,然后正六律,和五声,弦歌诗颂,此之谓德音;德音之谓乐。《诗》云:『莫其德音,其德克明。克明克类,克长克君,王此大邦;克顺克俾,俾于文王,其德靡悔。既受帝祉,施于孙子。』此之谓也。今君之所好者,其溺音乎?''文侯曰:``敢问溺音何从出也?''子夏对曰:``郑音好滥淫志,宋音燕女溺志,卫音趋数烦志,齐音敖辟乔志;此四者皆淫于色而害于德,是以祭祀弗用也。《诗》云:『肃雍和鸣,先祖是听。』夫肃肃,敬也;雍雍,和也。夫敬以和,何事不行?为人君者谨其所好恶而已矣。君好之,则臣为之。上行之,则民从之。《诗》云:『诱民孔易』,此之谓也。''然后,圣人作为鼗、鼓、椌、楬、埙、篪,此六者德音之音也。然后钟磬竽瑟以和之,干戚旄狄以舞之,此所以祭先王之庙也,所以献酬酳酢也,所以官序贵贱各得其宜也,所以示后世有尊卑长幼之序也。钟声铿,铿以立号,号以立横,横以立武。君子听钟声则思武臣。石声磬,磬以立辨,辨以致死。君子听磬声则思死封疆之臣。丝声哀,哀以立廉,廉以立志。君子听琴瑟之声则思志义之臣。竹声滥,滥以立会,会以聚众。君子听竽笙箫管之声,则思畜聚之臣。鼓鼙之声讙,讙以立动,动以进众。君子听鼓鼙之声,则思将帅之臣。君子之听音,非听其铿枪而已也,彼亦有所合之也。

宾牟贾侍坐于孔子,孔子与之言及乐,曰:``夫《武》之备戒之已久,何也?''对曰:``病不得众也。''``咏叹之,淫液之,何也?''对曰:``恐不逮事也。''``发扬蹈厉之已蚤,何也?''对曰:``及时事也。''``武坐致右宪左,何也?''对曰:``非武坐也。''``声淫及商,何也?''对曰:``非《武》音也。''子曰:``若非《武》音,则何音也?''对曰:``有司失其传也。若非有司失其传,则武王之志荒矣。''子曰:``唯!丘之闻诸苌弘,亦若吾子之言是也。''宾牟贾起,免席而请曰:``夫《武》之备戒之已久,则既闻命矣,敢问:迟之迟而又久,何也?''子曰:``居!吾语汝。夫乐者,象成者也;总干而山立,武王之事也;发扬蹈厉,大公之志也。《武》乱皆坐,周、召之治也。且夫《武》,始而北出,再成而灭商。三成而南,四成而南国是疆,五成而分周公左召公右,六成复缀以崇。天子夹振之而驷伐,盛威于中国也。分夹而进,事早济也,久立于缀,以待诸侯之至也。且女独未闻牧野之语乎?武王克殷反商。未及下车而封黄帝之后于蓟,封帝尧之后于祝,封帝舜之后于陈。下车而封夏后氏之后于杞,投殷之后于宋。封王子比干之墓,释箕子之囚,使之行商容而复其位。庶民弛政,庶士倍禄。济河而西,马散之华山之阳,而弗复乘;牛散之桃林之野,而弗复服。车甲衅而藏之府库,而弗复用。倒载干戈,包之以虎皮;将帅之士,使为诸侯;名之曰建櫜。然后知武王之不复用兵也。散军而郊射,左射狸首,右射驺虞,而贯革之射息也。裨冕搢笏,而虎贲之士说剑也。祀乎明堂而民知孝。朝觐然后诸侯知所以臣,耕藉然后诸侯知所以敬。五者,天下之大教也。食三老五更于大学,天子袒而割牲,执酱而馈,执爵而酳,冕而总干,所以教诸侯之弟也。若此则周道四达,礼乐交通。则夫《武》之迟久,不亦宜乎!''

君子曰:礼乐不可斯须去身。致乐以治心,则易直子谅之心油然生矣。易直子谅之心生则乐,乐则安,安则久,久则天,天则神。天则不言而信,神则不怒而威,致乐以治心者也。致礼以治躬则庄敬,庄敬则严威。心中斯须不和不乐,而鄙诈之心入之矣。外貌斯须不庄不敬,而易慢之心入之矣。故乐也者,动于内者也;礼也者,动于外者也。乐极和,礼极顺,内和而外顺,则民瞻其颜色而弗与争也;望其容貌,而民不生易慢焉。故德辉动于内,而民莫不承听;理发诸外,而民莫不承顺。故曰:致礼乐之道,举而错之,天下无难矣。乐也者,动于内者也;礼也者,动于外者也。故礼主其减,乐主其盈。礼减而进,以进为文:乐盈而反,以反为文。礼减而不进则销,乐盈而不反则放;故礼有报而乐有反。礼得其报则乐,乐得其反则安;礼之报,乐之反,其义一也。夫乐者乐也,人情之所不能免也。乐必发于声音,形于动静,人之道也。声音动静,性术之变,尽于此矣。故人不耐无乐,乐不耐无形。形而不为道,不耐无乱。先王耻其乱,故制雅、颂之声以道之,使其声足乐而不流,使其文足论而不息,使其曲直繁瘠、廉肉节奏足以感动人之善心而已矣。不使放心邪气得接焉,是先王立乐之方也。是故乐在宗庙之中,君臣上下同听之则莫不和敬;在族长乡里之中,长幼同听之则莫不和顺;在闺门之内,父子兄弟同听之则莫不和亲。故乐者审一以定和,比物以饰节;节奏合以成文。所以合和父子君臣,附亲万民也,是先王立乐之方也。故听其雅、颂之声,志意得广焉;执其干戚,习其俯仰诎伸,容貌得庄焉;行其缀兆,要其节奏,行列得正焉,进退得齐焉。故乐者天地之命,中和之纪,人情之所不能免也。夫乐者,先王之所以饰喜也,军旅鈇钺者,先王之所以饰怒也。故先王之喜怒,皆得其侪焉。喜则天下和之,怒则暴乱者畏之。先王之道,礼乐可谓盛矣。

子赣见师乙而问焉,曰:``赐闻声歌各有宜也,如赐者,宜何歌也?''师乙曰:``乙贱工也,何足以问所宜?请诵其所闻,而吾子自执焉:宽而静、柔而正者宜歌颂。广大而静、疏达而信者宜歌大雅。恭俭而好礼者宜歌小雅。正直而静、廉而谦者宜歌风。肆直而慈爱者宜歌商;温良而能断者宜歌齐。夫歌者,直己而陈德也。动己而天地应焉,四时和焉,星辰理焉,万物育焉。故商者,五帝之遗声也。商人识之,故谓之商。齐者三代之遗声也,齐人识之,故谓之齐。明乎商之音者,临事而屡断,明乎齐之音者,见利而让。临事而屡断,勇也;见利而让,义也。有勇有义,非歌孰能保此?故歌者,上如抗,下如队,曲如折,止如槁木,倨中矩,句中钩,累累乎端如贯珠。故歌之为言也,长言之也。说之,故言之;言之不足,故长言之;长言之不足,故嗟叹之;嗟叹之不足,故不知手之舞之,足之蹈之也。''子贡问乐。

\hypertarget{header-n557}{%
\subsection{杂记上}\label{header-n557}}

诸侯行而死于馆,则其复如于其国。如于道,则升其乘车之左毂,以其绥复。其輤有裧,缁布裳帷素锦以为屋而行。至于庙门,不毁墙遂入适所殡,唯輤为说于庙门外。大夫、士死于道,则升其乘车之左毂,以其绥复。如于馆死,则其复如于家。大夫以布为輤而行,至于家而说輤,载以輲车,入自门至于阼阶下而说车,举自阼阶,升适所殡。士輤,苇席以为屋,蒲席以为裳帷。

凡讣于其君,曰:``君之臣某死'';父母、妻、长子,曰:``君之臣某之某死''。君讣于他国之君,曰:``寡君不禄,敢告于执事。'';夫人,曰:``寡小君不禄。'';大子之丧,曰:``寡君之适子某死。''大夫讣于同国:适者,曰:``某不禄'';讣于士,亦曰:``某不禄'';讣于他国之君,曰:``君之外臣寡大夫某死'',讣于适者,曰:``吾子之外私寡大夫某不禄,使某实。''讣于士,亦曰:``吾子之外私寡大夫某不禄,使某实。''士讣于同国大夫,曰:``某死'',讣于士,亦曰:``某死'';讣于他国之君,曰:``君之外臣某死'',讣于大夫,曰:``吾子之外私某死'',讣于士,亦曰:``吾子之外私某死''。大夫次于公馆以终丧,士练而归。士次于公馆,大夫居庐,士居垩室。大夫为其父母兄弟之未为大夫者之丧,服如士服。士为其父母兄弟之为大夫者之丧,服如士服。大夫之适子,服大夫之服。大夫之庶子为大夫,则为其父母服大夫服;其位,与未为大夫者齿。士之子为大夫,则其父母弗能主也,使其子主之。无子,则为之置后。

大夫卜宅与葬日,有司麻衣、布衰、布带,因丧屦,缁布冠不蕤。占者皮弁。如筮,则史练冠长衣以筮。占者朝服。大夫之丧,既荐马。荐马者,哭踊,出乃包奠而读书。大夫之丧,大宗人相,小宗人命龟,卜人作龟。复,诸侯以褒衣冕服,爵弁服,夫人税衣揄狄,狄税素沙。内子以鞠衣,褒衣,素沙。下大夫以襢衣,其余如士。复西上。大夫不揄绞,属于池下。大夫附于士,士不附于大夫,附于大夫之昆弟。无昆弟,则从其昭穆。虽王父母在,亦然。妇附于其夫之所附之妃,无妃。则亦从其昭穆之妃。妾附于妾祖姑,无妾祖姑则亦从其昭穆之妾。男子附于王父则配;女子附于王母,则不配。公子附于公子。君薨,大子号称子,待犹君也。

有三年之练冠,则以大功之麻易之;唯杖屦不易。有父母之丧,尚功衰,而附兄弟之殇则练冠。附于殇,称阳童某甫,不名,神也。凡异居,始闻兄弟之丧,唯以哭对,可也。其始麻,散带绖。未服麻而奔丧,及主人之未成绖也:疏者,与主人皆成之;亲者,终其麻带绖之日数。主妾之丧,则自绖至于练祥,皆使其子主之。其殡祭,不于正室。君不抚仆妾。女君死,则妾为女君之党服。摄女君,则不为先女君之党服。

闻兄弟之丧,大功以上,见丧者之乡而哭。适兄弟之送葬者弗及,遇主人于道,则遂之于墓。凡主兄弟之丧,虽疏亦虞之。

凡丧服未毕,有吊者,则为位而哭拜踊。大夫之哭大夫,弁绖;大夫与殡,亦弁绖。大夫有私丧之葛,则于其兄弟之轻丧,则弁绖。

为长子杖,则其子不以杖即位。为妻,父母在,不杖,不稽颡。母在,不稽颡。稽颡者,其赠也拜。违诸侯之大夫,不反服。

违大夫之诸侯,不反服。丧冠条属,以别吉凶。三年之练冠,亦条属,右缝。小功以下左。缌冠缲缨。大功以上散带。朝服十五升,去其半而缌;加灰,锡也。

诸侯相襚,以后路与冕服。先路与褒衣,不以襚。遣车视牢具。疏布輤,四面有章,置于四隅。载粻,有子曰:``非礼也。丧奠,脯醢而已。''祭称孝子、孝孙,丧称哀子、哀孙。端衰,丧车,皆无等。大白冠,缁布之冠,皆不蕤。委武玄缟而后蕤。大夫冕而祭于公,弁而祭于己。士弁而祭于公,冠而祭于己。士弁而亲迎,然则士弁而祭于己可也。

畅臼以椈,杵以梧。枇以桑,长三尺;或曰五尺。毕用桑,长三尺,刊其柄与末。率带,诸侯、大夫皆五采;士二采。醴者,稻醴也。瓮甒筲衡,实见间而后折入。重,既虞而埋之。

凡妇人,从其夫之爵位。小敛、大敛、启,皆辩拜。朝夕哭,不帷。无柩者不帷。君若载而后吊之,则主人东面而拜,门右北面而踊。出待,反而后奠。子羔之袭也:茧衣裳与税衣纁袡为一,素端一,皮弁一,爵弁一,玄冕一。曾子曰:``不袭妇服。''为君使而死,于公馆,复;私馆不复。公馆者,公宫与公所为也。私馆者,自卿大夫以下之家也。公七踊,大夫五踊,妇人居间,士三踊,妇人皆居间。公袭:卷衣一,玄端一,朝服一,素积一,纁裳一,爵弁二,玄冕一,褒衣一。朱绿带,申加大带于上。小敛环绖,公大夫士一也。公视大敛,公升,商祝铺席,乃敛。鲁人之赠也:三玄二纁,广尺,长终幅。

吊者即位于门西,东面;其介在其东南,北面西上,西于门。主孤西面。相者受命曰:``孤某使某请事。''客曰:``寡君使某,如何不淑!''相者入告,出曰:``孤某须矣。''吊者入,主人升堂,西面。吊者升自西阶,东面,致命曰:``寡君闻君之丧,寡君使某,如何不淑!''子拜稽颡,吊者降,反位。含者执璧将命曰:``寡君使某含。''相者入告,出曰:``孤某须矣。''含者入,升堂,致命。再拜稽颡。含者坐委于殡东南,有苇席;既葬,蒲席。降,出,反位。宰朝服,即丧屦升自西阶,西面,坐取璧,降自西阶以东。襚者曰:``寡君使某襚。''相者入告,出曰:``孤某须矣。''襚者执冕服;左执领,右执要,入,升堂致命曰:``寡君使某襚。''子拜稽颡。委衣于殡东。襚者降,受爵弁服于门内溜,将命,子拜稽颡,如初。受皮弁服于中庭。自西阶受朝服,自堂受玄端,将命,子拜稽颡,皆如初。襚者降,出,反位。宰夫五人,举以东。降自西阶。其举亦西面。上介赗:执圭将命,曰:``寡君使某赗。''相者入告,反命曰:``孤某须矣。''陈乘黄大路于中庭,北辀。执圭将命。客使自下,由路西。子拜稽颡,坐委于殡东南隅。宰举以东。凡将命,乡殡将命,子拜稽颡。西面而坐,委之。宰举璧与圭,宰夫举襚,升自西阶,西面,坐取之,降自西阶。赗者出,反位于门外。上客临曰:``寡君有宗庙之事,不得承事,使一介老某相执綍。''相者反命曰:``孤某须矣。''临者入门右,介者皆从之,立于其左东上。宗人纳宾,升,受命于君;降曰:``孤敢辞吾子之辱,请吾子之复位。''客对曰:``寡君命某,毋敢视宾客,敢辞。''宗人反命曰:``孤敢固辞吾子之辱,请吾子之复位。''客对曰:``寡君命某,毋敢视宾客,敢固辞。''宗人反命曰:``孤敢固辞吾子之辱,请吾子之复位。''客对曰:``寡君命使臣某,毋敢视宾客,是以敢固辞。固辞不获命,敢不敬从。''客立于门西,介立于其左,东上。孤降自阼阶,拜之,升哭,与客拾踊三。客出,送于门外,拜稽颡。

其国有君丧,不敢受吊。外宗房中南面,小臣铺席,商祝铺绞紟衾,士盥于盘北。举迁尸于敛上,卒敛,宰告子,冯之踊。夫人东面坐,冯之兴踊。士丧有与天子同者三:其终夜燎,及乘人,专道而行。

\hypertarget{header-n573}{%
\subsection{杂记下}\label{header-n573}}

有父之丧,如未没丧而母死,其除父之丧也,服其除服。卒事,反丧服。虽诸父昆弟之丧,如当父母之丧,其除诸父昆弟之丧也,皆服其除丧之服。卒事,反丧服。如三年之丧,则既顈,其练祥皆同。王父死,未练祥而孙又死,犹是附于王父也。有殡,闻外丧,哭之他室。入奠,卒奠,出,改服即位,如始即位之礼。大夫、士将与祭于公,既视濯,而父母死,则犹是与祭也,次于异宫。既祭,释服出公门外,哭而归。其它如奔丧之礼。如未视濯,则使人告。告者反,而后哭。如诸父昆弟姑姊妹之丧,则既宿,则与祭。卒事,出公门,释服而后归。其它如奔丧之礼。如同宫,则次于异宫。

曾子问曰:``卿大夫将为尸于公,受宿矣,而有齐衰内丧,则如之何?''孔子曰:``出舍乎公宫以待事,礼也。''孔子曰:``尸弁冕而出,卿、大夫、士皆下之。尸必式,必有前驱。''

父母之丧,将祭,而昆弟死;既殡而祭。如同宫,则虽臣妾,葬而后祭。祭,主人之升降散等,执事者亦散等。虽虞附亦然。自诸侯达诸士,小祥之祭,主人之酢也哜之;众宾兄弟,则皆啐之。大祥:主人啐之,众宾兄弟皆饮之,可也。凡侍祭丧者,告宾祭荐而不食。

子贡问丧,子曰:``敬为上,哀次之,瘠为下。颜色称其情;戚容称其服。''请问兄弟之丧,子曰:``兄弟之丧,则存乎书策矣。''君子不夺人之丧,亦不可夺丧也。孔子曰:``少连、大连善居丧,三日不怠,三月不解,期悲哀,三年忧。东夷之子也。''

三年之丧,言而不语,对而不问:庐,垩室之中,不与人坐焉;在垩室之中,非时见乎母也,不入门。疏衰皆居垩室不庐。庐,严者也。妻视叔父母,姑姊妹视兄弟,长、中、下殇视成人。亲丧外除,兄弟之丧内除。视君之母与妻,比之兄弟。发诸颜色者,亦不饮食也。免丧之外,行于道路,见似目瞿,闻名心瞿。吊死而问疾,颜色戚容必有以异于人也。如此而后可以服三年之丧。其余则直道而行之,是也。

祥,主人之除也,于夕为期,朝服。祥因其故服。子游曰:``既祥,虽不当缟者必缟,然后反服。''当袒,大夫至,虽当踊,绝踊而拜之,反改成踊,乃袭。于士,既事成踊,袭而后拜之,不改成踊。上大夫之虞也,少牢。卒哭成事,附,皆大牢。下大夫之虞也,特牲。卒哭成事,附,皆少牢。祝称卜葬虞,子孙曰哀,夫曰乃,兄弟曰某,卜葬其兄弟曰伯子某。

古者,贵贱皆杖。叔孙武叔朝,见轮人以其杖关毂而輠轮者,于是有爵而后杖也。凿巾以饭,公羊贾为之也。

冒者何也?所以掩形也。自袭以至小敛,不设冒则形,是以袭而后设冒也。或问于曾子曰:``夫既遣而包其余,犹既食而裹其余与?君子既食,则裹其余乎?''曾子曰:``吾子不见大飨乎?夫大飨,既飨,卷三牲之俎归于宾馆。父母而宾客之,所以为哀也!子不见大飨乎!''非为人丧,问与赐与?

三年之丧,以其丧拜;非三年之丧,以吉拜。三年之丧,如或遗之酒肉,则受之必三辞。主人衰绖而受之。如君命,则不敢辞,受而荐之。丧者不遗人,人遗之,虽酒肉,受也。从父昆弟以下,既卒哭,遗人可也。县子曰:``三年之丧,如斩。期之丧,如剡。''三年之丧,虽功衰不吊,自诸侯达诸士。如有服而将往哭之,则服其服而往。期之丧,十一月而练,十三月而祥,十五月禫。练则吊。既葬,大功吊,哭而退,不听事焉。期之丧,未丧,吊于乡人。哭而退,不听事焉。功衰吊,待事不执事。小功缌,执事不与于礼。相趋也,出宫而退。相揖也,哀次而退。相问也,既封而退。相见也。反哭而退。朋友,虞附而退。吊,非从主人也。四十者执綍:乡人五十者从反哭,四十者待盈坎。

丧食虽恶必充饥,饥而废事,非礼也;饱而忘哀,亦非礼也。视不明,听不聪,行不正,不知哀,君子病之。故有疾饮酒食肉,五十不致毁,六十不毁,七十饮酒食肉,皆为疑死。有服,人召之食,不往。大功以下,既葬,适人,人食之,其党也食之,非其党弗食也。功衰食菜果,饮水浆,无盐酪。不能食食,盐酪可也。孔子曰:``身有疡则浴,首有创则沐,病则饮酒食肉。毁瘠为病,君子弗为也。毁而死,君子谓之无子。''

非从柩与反哭,无免于堩。凡丧,小功以上,非虞附练祥,无沐浴。疏衰之丧,既葬,人请见之,则见;不请见人。小功,请见人可也。大功不以执挚。唯父母之丧,不辟涕泣而见人。三年之丧,祥而从政;期之丧,卒哭而从政;九月之丧,既葬而从政;小功缌之丧,既殡而从政。曾申问于曾子曰:``哭父母有常声乎?''曰:``中路婴儿失其母焉,何常声之有?''

卒哭而讳。王父母兄弟,世父叔父,姑姊妹。子与父同讳。母之讳,宫中讳。妻之讳,不举诸其侧;与从祖昆弟同名则讳。以丧冠者,虽三年之丧,可也。既冠于次,入哭踊,三者三,乃出。大功之末,可以冠子,可以嫁子。父,小功之末,可以冠子,可以嫁子,可以取妇。己虽小功,既卒哭,可以冠,取妻;下殇之小功,则不可。凡弁绖,其衰侈袂。

父有服,宫中子不与于乐。母有服,声闻焉不举乐。妻有服,不举乐于其侧。大功将至,辟琴瑟。小功至,不绝乐。

姑姊妹,其夫死,而夫党无兄弟,使夫之族人主丧。妻之党,虽亲弗主。夫若无族矣,则前后家,东西家;无有,则里尹主之。或曰:主之,而附于夫之党。

麻者不绅,执玉不麻。麻不加于采。国禁哭,则止朝夕之奠。即位自因也。童子哭不偯,不踊,不杖,不菲,不庐。孔子曰:``伯母、叔母,疏衰,踊不绝地。姑姊妹之大功,踊绝于地。如知此者,由文矣哉!由文矣哉!''

世柳之母死,相者由左。世柳死,其徒由右相。由右相,世柳之徒为之也。

天子饭,九贝;诸侯七,大夫五,士三。士三月而葬,是月也卒哭;大夫三月而葬,五月而卒哭;诸侯五月而葬,七月而卒哭。士三虞,大夫五,诸侯七。诸侯使人吊,其次:含襚赗临,皆同日而毕事者也,其次如此也。卿大夫疾,君问之无算;士一问之。君于卿大夫,比葬不食肉,比卒哭不举乐;为士,比殡不举乐。升正柩,诸侯执綍五百人,四綍,皆衔枚,司马执铎,左八人,右八人,匠人执羽葆御柩。大夫之丧,其升正柩也,执引者三百人,执铎者左右各四人,御柩以茅。

孔子曰:``管仲镂簋而朱纮,旅树而反坫,山节而藻棁。贤大夫也,而难为上也。晏平仲祀其先人。豚肩不掩豆。贤大夫也,而难为下也。君子上不僭上,下不偪下。''

妇人非三年之丧,不逾封而吊。如三年之丧,则君夫人归。夫人其归也以诸侯之吊礼,其待之也若待诸侯然。夫人至,入自闱门,升自侧阶,君在阼。其它如奔丧礼然。嫂不抚叔,叔不抚嫂。

君子有三患:未之闻,患弗得闻也;既闻之,患弗得学也;既学之,患弗能行也。君子有五耻:居其位,无其言,君子耻之;有其言,无其行,君子耻之;既得之而又失之,君子耻之;地有余而民不足,君子耻之;众寡均而倍焉,君子耻之。

孔子曰:``凶年则乘驽马。祀以下牲。''

恤由之丧,哀公使孺悲之孔子学士丧礼,士丧礼于是乎书。子贡观于蜡。孔子曰:``赐也乐乎?''对曰:``一国之人皆若狂,赐未知其乐也!''子曰:``百日之蜡,一日之泽,非尔所知也。张而不弛,文武弗能也;弛而不张,文武弗为也。一张一弛,文武之道也。''

孟献子曰:``正月日至,可以有事于上帝;七月日至,可有事于祖。''七月而禘,献子为之也。夫人之不命于天子,自鲁昭公始也。外宗为君夫人,犹内宗也。

厩焚,孔子拜乡人为火来者。拜之,士壹,大夫再。亦相吊之道也。孔子曰:``管仲遇盗,取二人焉,上以为公臣,曰:『其所与游辟也,可人也!』管仲死,桓公使为之服。宦于大夫者之为之服也,自管仲始也,有君命焉尔也。''

过而举君之讳,则起。与君之讳同,则称字。内乱不与焉,外患弗辟也。赞,大行曰圭。公九寸,侯、伯七寸,子、男五寸。博三寸,厚半寸。剡上,左右各寸半,玉也。藻三采六等。哀公问子羔曰:``子之食奚当?''对曰:``文公之下执事也。''

成庙则衅之。其礼:祝、宗人、宰夫、雍人,皆爵弁纯衣。雍人拭羊,宗人视之,宰夫北面于碑南,东上。雍人举羊,升屋自中,中屋南面,刲羊,血流于前,乃降。门、夹室皆用鸡。先门而后夹室。其衈皆于屋下。割鸡,门当门,夹室中室。有司皆乡室而立,门则有司当门北面。既事,宗人告事毕,乃皆退。反命于君曰:``衅某庙事毕。''反命于寝,君南乡于门内朝服。既反命,乃退。路寝成则考之而不衅。衅屋者,交神明之道也。凡宗庙之器。其名者成则衅之以豭豚。

诸侯出夫人,夫人比至于其国,以夫人之礼行;至,以夫人入。使者将命曰:``寡君不敏,不能从而事社稷宗庙,使使臣某,敢告于执事。''主人对曰:``寡君固前辞不教矣,寡君敢不敬须以俟命。''有司官陈器皿;主人有司亦官受之。妻出,夫使人致之曰:``某不敏,不能从而共粢盛,使某也敢告于侍者。''主人对曰:``某之子不肖,不敢辟诛,敢不敬须以俟命。''使者退,主人拜送之。如舅在,则称舅;舅没,则称兄;无兄,则称夫。主人之辞曰:``某之子不肖。''如姑姊妹,亦皆称之。

孔子曰:``吾食于少施氏而饱,少施氏食我以礼。吾祭,作而辞曰:『疏食不足祭也。』吾飧,作而辞曰:『疏食也,不敢以伤吾子。』''

纳币一束:束五两,两五寻。妇见舅姑,兄弟、姑姊妹,皆立于堂下,西面北上,是见已。见诸父,各就其寝。女虽未许嫁,年二十而笄,礼之,妇人执其礼。燕则鬈首。

韠:长三尺,下广二尺,上广一尺。会去上五寸,纰以爵韦六寸,不至下五寸。纯以素,紃以五采。

\hypertarget{header-n606}{%
\subsection{丧大记}\label{header-n606}}

疾病,外内皆扫。君大夫彻县,士去琴瑟。寝东首于北牖下。废床。彻亵衣,加新衣,体一人。男女改服。属纩以俟绝气。男子不死于妇人之手,妇人不死于男子之手。君夫人卒于路寝,大夫世妇卒于适寝,内子未命,则死于下室。迁尸于寝,士士之妻皆死于寝。

复,有林麓,则虞人设阶;无林麓,则狄人设阶。小臣复,复者朝服。君以卷,夫人以屈狄;大夫以玄赪,世妇以襢衣;士以爵弁,士妻以税衣。皆升自东荣,中屋履危,北面三号,衣投于前,司命受之,降自西北荣。其为宾,则公馆复,私馆不复;其在野,则升其乘车之左毂而复。复衣不以衣尸,不以敛。妇人复,不以袡。凡复,男子称名,妇人称字。唯哭先复,复而后行死事。

始卒,主人啼,兄弟哭,妇人哭踊。既正尸,子坐于东方,卿大夫父兄子姓立于东方,有司庶士哭于堂下北面;夫人坐于西方,内命妇姑姊妹子姓立于西方,外命妇率外宗哭于堂上北面。大夫之丧,主人坐于东方,主妇坐于西方,其有命夫命妇则坐,无则皆立。士之丧,主人父兄子姓皆坐于东方,主妇姑姊妹子姓皆坐于西方。凡哭尸于室者,主人二手承衾而哭。

君之丧,未小敛,为寄公国宾出;大夫之丧,未小敛,为君命出;士之丧,于大夫不当敛而出。凡主人之出也,徒跣扱衽拊心,降自西阶。君拜寄公国宾于位;大夫于君命,迎于寝门外,使者升堂致命,主人拜于下;士于大夫亲吊则与之哭;不逆于门外,夫人为寄公夫人出,命妇为夫人之命出,士妻不当敛,则为命妇出。

小敛,主人即位于户内,主妇东面,乃敛。卒敛,主人冯之踊,主妇亦如之。主人袒说髦,括发以麻,妇人髽,带麻于房中。彻帷,男女奉尸夷于堂,降拜:君拜寄公国宾,大夫士拜卿大夫于位,于士旁三拜;夫人亦拜寄公夫人于堂上,大夫内子士妻特拜,命妇泛拜众宾于堂上。主人即位,袭带绖踊─-母之丧,即位而免,乃奠。吊者袭裘,加武带绖,与主人拾踊。君丧,虞人出木角,狄人出壶,雍人出鼎,司马县之,乃官代哭,大夫官代哭不县壶,士代哭不以官。君堂上二烛、下二烛,大夫堂上一烛、下二烛,士堂上一烛、下一烛。宾出彻帷。哭尸于堂上,主人在东方,由外来者在西方,诸妇南乡。妇人迎客送客不下堂,下堂不哭;男子出寝门见人不哭。其无女主,则男主拜女宾于寝门内;其无男主,则女主拜男宾于阼阶下。子幼,则以衰抱之,人为之拜;为后者不在,则有爵者辞,无爵者人为之拜。在竟内则俟之,在竟外则殡葬可也。丧有无后,无无主。

君之丧:三日,子、夫人杖,五日既殡,授大夫世妇杖。子、大夫寝门之外杖,寝门之内辑之;夫人世妇在其次则杖,即位则使人执之。子有王命则去杖,国君之命则辑杖,听卜有事于尸则去杖。大夫于君所则辑杖,于大夫所则杖。大夫之丧:三日之朝既殡,主人主妇室老皆杖。大夫有君命则去杖,大夫之命则辑杖;内子为夫人之命去杖,为世妇之命授人杖。士之丧:二日而殡,三日而朝,主人杖,妇人皆杖。于君命夫人之命如大夫,于大夫世妇之命如大夫。子皆杖,不以即位。大夫士哭殡则杖,哭柩则辑杖。弃杖者,断而弃之于隐者。

始死,迁尸于床,幠用敛衾,去死衣,小臣楔齿用角柶,缀足用燕几,君大夫士一也。

管人汲,不说繘、屈之,尽阶不升堂,授御者;御者入浴:小臣四人抗衾,御者二人浴,浴水用盆,沃水用枓,浴用絺巾,挋用浴衣,如它日;小臣爪足,浴余水弃于坎。其母之丧,则内御者抗衾而浴。管人汲,授御者,御者差沐于堂上-─君沐粱,大夫沐稷,士沐粱。甸人为垼于西墙下,陶人出重鬲,管人受沐,乃煮之,甸人取所彻庙之西北厞薪,用爨之。管人授御者沐,乃沐;沐用瓦盘,挋用巾,如它日,小臣爪手翦须,濡濯弃于坎。君设大盘造冰焉,大夫设夷盘造冰焉,士并瓦盘无冰,设床襢笫,有枕。含一床,袭一床迁尸于堂又一床,皆有枕席-─君大夫士一也。

君之丧,子、大夫、公子、众士皆三日不食。子、大夫、公子食粥,纳财,朝一溢米,莫一溢米,食之无算;士疏食水饮,食之无算;夫人世妇诸妻皆疏食水饮,食之无算。大夫之丧,主人室老子姓皆食粥;众士疏食水饮;妻妾疏食水饮。士亦如之。既葬,主人疏食水饮,不食菜果;妇人亦如之。君大夫士一也。练而食菜果,祥而食肉。食粥于盛不盥,食于篹者盥。食菜以酰酱,始食肉者先食干肉,始饮酒者先饮醴酒。期之丧,三不食;食:疏食水饮,不食菜果,三月既葬,食肉饮酒。期终丧,不食肉,不饮酒,父在为母,为妻。九月之丧,食饮犹期之丧也,食肉饮酒,不与人乐之。五月三月之丧,壹不食再不食可也。比葬,食肉饮酒,不与人乐之。叔母、世母、故主、宗子食肉饮酒。不能食粥,羹之以菜可也;有疾,食肉饮酒可也。五十不成丧,七十唯衰麻在身。既葬,若君食之则食之;大夫父之友食之则食之矣。不辟粱肉,若有酒醴则辞。

小敛于户内,大敛于阼。君以簟席,大夫以蒲席,士以苇席。小敛:布绞,缩者一,横者三。君锦衾,大夫缟衾,士缁衾,皆一。衣十有九称,君陈衣于序东;大夫士陈衣于房中;皆西领北上。绞紟不在列。大敛:布绞,缩者三,横者五,布紟二衾。君大夫士一也。君陈衣于庭,百称,北领西上;大夫陈衣于序东,五十称,西领南上;士陈衣于序东,三十称,西领南上。绞紟如朝服,绞一幅为三、不辟,紟五幅、无紞。小敛之衣,祭服不倒。君无襚,大夫士毕主人之祭服;亲戚之衣,受之不以即陈。小敛,君大夫士皆用复衣复衾;大敛,君大夫士祭服无算,君褶衣褶衾,大夫士犹小敛也。袍必有表,不禅,衣必有裳,谓之一称。凡陈衣者实之箧,取衣者亦以箧升,降者自西阶。凡陈衣、不诎,非列采不入,絺绤纻不入。

凡敛者袒,迁尸者袭。君之丧,大胥是敛,众胥佐之;大夫之丧,大胥侍之,众胥是敛;士之丧,胥为侍,士是敛。小敛大敛,祭服不倒,皆左衽结绞不纽。敛者既敛必哭。士与其执事则敛,敛焉则为之壹不食。凡敛者六人。君锦冒黼杀,缀旁七;大夫玄冒黼杀,缀旁五;士缁冒赪杀,缀旁三。凡冒质长与手齐,杀三尺,自小敛以往用夷衾,夷衾质杀之,裁犹冒也。君将大敛,子弁绖,即位于序端,卿大夫即位于堂廉楹西,北面东上,父兄堂下北面,夫人命妇尸西东面,外宗房中南面。小臣铺席,商祝铺绞紟衾衣,士盥于盘,上士举迁尸于敛上。卒敛,宰告,子冯之踊,夫人东面亦如之。大夫之丧,将大敛,既铺绞紟衾衣。君至,主人迎,先入门右,巫止于门外,君释菜,祝先入升堂,君即位于序端,卿大夫即位于堂廉楹西,北面东上;主人房外南面,主妇尸西,东面。迁尸,卒敛,宰告,主人降,北面于堂下,君抚之,主人拜稽颡,君降、升主人冯之,命主妇冯之。士之丧,将大敛,君不在,其余礼犹大夫也。铺绞紟,踊;铺衾,踊;铺衣,踊;迁尸,踊;敛衣,踊;敛衾,踊;敛绞紟,踊。

君抚大夫,抚内命妇;大夫抚室老,抚侄娣。君大夫冯父母、妻、长子,不冯庶子;士冯父母、妻、长子、庶子,庶子有子,则父母不冯其尸。凡冯尸者,父母先,妻子后。君于臣抚之,父母于子执之,子于父母冯之,妇于舅姑奉之,舅姑于妇抚之,妻于夫拘之,夫于妻于昆弟执之。冯尸不当君所。凡冯尸,兴必踊。

父母之丧,居倚庐、不涂,寝苫枕块,非丧事不言。君为庐宫之,大夫士襢之。既葬柱楣,涂庐不于显者。君、大夫、士皆宫之。凡非适子者,自未葬以于隐者为庐。既葬,与人立:君言王事,不言国事;大夫士言公事,不言家事。君既葬,王政入于国,既卒哭而服王事;大夫、士既葬,公政入于家,既卒哭、弁绖带,金革之事无辟也。既练,居垩室,不与人居。君谋国政,大夫、士谋家事。既祥,黝垩。祥而外无哭者;禫而内无哭者,乐作矣故也。禫而从御,吉祭而复寝。

期居庐,终丧不御于内者,父在为母为妻;齐衰期者,大功布衰九月者,皆三月不御于内。妇人不居庐,不寝苫。丧父母,既练而归;期九月者,既葬而归。公之丧,大夫俟练,士卒哭而归。大夫、士父母之葬,既练而归。朔月忌日,则归哭于宗室。诸父兄弟之丧,既卒哭而归。父不次于子,兄不次于弟。

君于大夫、世妇大敛焉;为之赐则小敛焉。于外命妇,既加盖而君至。于士,既殡而往;为之赐,大敛焉。夫人于世妇,大敛焉;为之赐,小敛焉。于诸妻,为之赐,大敛焉。于大夫外命妇,既殡而往。大夫、士既殡而君往焉,使人戒之,主人具殷奠之礼,俟于门外。见马首,先入门右,巫止于门外,祝代之先,君释菜于门内。祝先升自阼阶,负墉南面。君即位于阼。小臣二人执戈立于前,二人立于后。摈者进,主人拜稽颡。君称言,视祝而踊,主人踊。大夫则奠可也。士则出俟于门外,命之反奠,乃反奠。卒奠,主人先俟于门外,君退,主人送于门外,拜稽颡。君于大夫疾,三问之,在殡,三往焉;士疾,壹问之,在殡,壹往焉。君吊则复殡服。夫人吊于大夫、士,主人出迎于门外,见马首,先入门右。夫人入,升堂即位。主妇降自西阶,拜稽颡于下。夫人视世子而踊。奠如君至之礼。夫人退,主妇送于门内,拜稽颡;主人送于大门之外不拜。大夫君不迎于门外。入即位于堂下。主人北面,众主人南面;妇人即位于房中。若有君命,命夫命妇之命,四邻宾客,其君后主人而拜。君吊,见尸柩而后踊。大夫、士若君不戒而往,不具殷奠;君退必奠。

君大棺八寸,属六寸,椑四寸;上大夫大棺八寸,属六寸;下大夫大棺六寸,属四寸,士棺六寸。君里棺用朱绿,用杂金鐕;大夫里棺用玄绿,用牛骨鐕;士不绿。君盖用漆,三衽三束;大夫盖用漆,二衽二束;士盖不用漆,二衽二束。君、大夫鬊爪;实于绿中;士埋之。君殡用輴,攒至于上,毕涂屋;大夫殡以帱,攒置于西序,涂不暨于棺;士殡见衽,涂上帷之。熬,君四种八筐,大夫三种六筐,士二种四筐,加鱼腊焉。饰棺,君龙帷三池,振容。黼荒,火三列,黼三列。素锦褚,加伪荒。纁纽六。齐,五采五贝。黼翣二,黻翣二,画翣二,皆戴圭。鱼跃拂池。君纁戴六,纁披六。大夫画帷二池,不振容。画荒,火三列,黻三列。素锦褚。纁纽二,玄纽二。齐,三采三贝。黻翣二,画翣二,皆戴绥。鱼跃拂池。大夫戴前纁后玄,披亦如之。士布帷布荒,一池,揄绞。纁纽二,缁纽二。齐,三采一贝。画翣二,皆戴绥。士戴前纁后缁,二披用纁。君葬用辁,四綍二碑,御棺用羽葆。大夫葬用辁,二綍二碑,御棺用茅。士葬用国车。二綍无碑,比出宫,御棺用功布。凡封,用綍去碑负引,君封以衡,大夫士以咸。君命毋哗,以鼓封;大夫命毋哭;士哭者相止也。君松椁,大夫柏椁,士杂木椁。棺椁之间,君容柷,大夫容壶,士容甒。君里椁虞筐,大夫不里椁,士不虞筐。

\hypertarget{header-n625}{%
\subsection{祭法}\label{header-n625}}

祭法:有虞氏禘黄帝而郊喾,祖颛顼而宗尧。夏后氏亦禘黄帝而郊鲧,祖颛顼而宗禹。殷人禘喾而郊冥,祖契而宗汤。周人禘喾而郊稷,祖文王而宗武王。

燔柴于泰坛,祭天也;瘗埋于泰折,祭地也;用骍犊。埋少牢于泰昭,祭时也;相近于坎坛,祭寒暑也。王宫,祭日也;夜明,祭月也;幽宗,祭星也;雩宗,祭水旱也;四坎坛,祭四时也。山林、川谷、丘陵,能出云为风雨,见怪物,皆曰神。有天下者,祭百神。诸侯在其地则祭之,亡其地则不祭。

大凡生于天地之间者,皆曰命。其万物死,皆曰折;人死,曰鬼;此五代之所不变也。七代之所以更立者:禘、郊、宗、祖;其余不变也。

天下有王,分地建国,置都立邑,设庙祧坛墠而祭之,乃为亲疏多少之数。是故:王立七庙,一坛一墠,曰考庙,曰王考庙,曰皇考庙,曰显考庙,曰祖考庙;皆月祭之。远庙为祧,有二祧,享尝乃止。去祧为坛,去坛为墠。坛墠,有祷焉祭之,无祷乃止。去墠曰鬼。诸侯立五庙,一坛一墠。曰考庙,曰王考庙,曰皇考庙,皆月祭之;显考庙,祖考庙,享尝乃止。去祖为坛,去坛为墠。坛墠,有祷焉祭之,无祷乃止。去墠为鬼。大夫立三庙二坛,曰考庙,曰王考庙,曰皇考庙,享尝乃止。显考祖考无庙,有祷焉,为坛祭之。去坛为鬼。适士二庙一坛,曰考庙,曰王考庙,享尝乃止。皇考无庙,有祷焉,为坛祭之。去坛为鬼。官师一庙,曰考庙。王考无庙而祭之,去王考曰鬼。庶士庶人无庙,死曰鬼。

王为群姓立社,曰大社。王自为立社,曰王社。诸侯为百姓立社,曰国社。诸侯自立社,曰侯社。大夫以下,成群立社曰置社。

王为群姓立七祀:曰司命,曰中溜,曰国门,曰国行,曰泰厉,曰户,曰灶。王自为立七祀。诸侯为国立五祀,曰司命,曰中溜,曰国门,曰国行,曰公厉。诸侯自为立五祀。大夫立三祀:曰族厉,曰门,曰行。适士立二祀:曰门,曰行。庶士、庶人立一祀,或立户,或立灶。

王下祭殇五:适子、适孙、适曾孙、适玄孙、适来孙。诸侯下祭三,大夫下祭二,适士及庶人,祭子而止。

夫圣王之制祭祀也:法施于民则祀之,以死勤事则祀之,以劳定国则祀之,能御大菑则祀之,能捍大患则祀之。是故厉山氏之有天下也,其子曰农,能殖百谷;夏之衰也,周弃继之,故祀以为稷。共工氏之霸九州岛也,其子曰后土,能平九州岛,故祀以为社。帝喾能序星辰以着众;尧能赏均刑法以义终;舜勤众事而野死。鲧鄣洪水而殛死,禹能修鲧之功。黄帝正名百物以明民共财,颛顼能修之。契为司徒而民成;冥勤其官而水死。汤以宽治民而除其虐;文王以文治,武王以武功,去民之菑。此皆有功烈于民者也。及夫日月星辰,民所瞻仰也;山林川谷丘陵,民所取材用也。非此族也,不在祀典。

\hypertarget{header-n636}{%
\subsection{祭仪}\label{header-n636}}

祭不欲数,数则烦,烦则不敬。祭不欲疏,疏则怠,怠则忘。是故君子合诸天道:春禘秋尝。霜露既降,君子履之,必有凄怆之心,非其寒之谓也。春,雨露既濡,君子履之,必有怵惕之心,如将见之。乐以迎来,哀以送往,故禘有乐而尝无乐。致齐于内,散齐于外。齐之日:思其居处,思其笑语,思其志意,思其所乐,思其所嗜。齐三日,乃见其所为齐者。祭之日:入室,僾然必有见乎其位,周还出户,肃然必有闻乎其容声,出户而听,忾然必有闻乎其叹息之声。是故,先王之孝也,色不忘乎目,声不绝乎耳,心志嗜欲不忘乎心。致爱则存,致悫则着。着存不忘乎心,夫安得不敬乎?君子生则敬养,死则敬享,思终身弗辱也。君子有终身之丧,忌日之谓也。忌日不用,非不祥也。言夫日,志有所至,而不敢尽其私也。唯圣人为能飨帝,孝子为能飨亲。飨者,乡也。乡之,然后能飨焉。是故孝子临尸而不怍。君牵牲,夫人奠盎。君献尸,夫人荐豆。卿大夫相君,命妇相夫人。齐齐乎其敬也,愉愉乎其忠也,勿勿诸其欲其飨之也。文王之祭也:事死者如事生,思死者如不欲生,忌日必哀,称讳如见亲。祀之忠也,如见亲之所爱,如欲色然;其文王与?《诗》云:``明发不寐,有怀二人。''文王之诗也。祭之明日,明发不寐,飨而致之,又从而思之。祭之日,乐与哀半;飨之必乐,已至必哀。

仲尼尝,奉荐而进其亲也悫,其行趋趋以数。已祭,子赣问曰:``子之言祭,济济漆漆然;今子之祭,无济济漆漆,何也?''子曰:``济济者,容也远也;漆漆者,容也自反也。容以远,若容以自反也,夫何神明之及交,夫何济济漆漆之有乎?反馈,乐成,荐其荐俎,序其礼乐,备其百官。君子致其济济漆漆,夫何慌惚之有乎?夫言,岂一端而已?夫各有所当也。''

孝子将祭,虑事不可以不豫;比时具物,不可以不备;虚中以治之。宫室既修,墙屋既设,百物既备,夫妇齐戒沐浴,盛服奉承而进之,洞洞乎,属属乎,如弗胜,如将失之,其孝敬之心至也与!荐其荐俎,序其礼乐,备其百官,奉承而进之。于是谕其志意,以其恍惚以与神明交,庶或飨之。``庶或飨之'',孝子之志也。孝子之祭也,尽其悫而悫焉,尽其信而信焉,尽其敬而敬焉,尽其礼而不过失焉。进退必敬,如亲听命,则或使之也。孝子之祭,可知也,其立之也敬以诎,其进之也敬以愉,其荐之也敬以欲;退而立,如将受命;已彻而退,敬齐之色不绝于面。孝子之祭也,立而不诎,固也;进而不愉,疏也;荐而不欲,不爱也;退立而不如受命,敖也;已彻而退,无敬齐之色,而忘本也。如是而祭,失之矣。孝子之有深爱者,必有和气;有和气者,必有愉色;有愉色者,必有婉容。孝子如执玉,如奉盈,洞洞属属然,如弗胜,如将失之。严威俨恪,非所以事亲也,成人之道也。

先王之所以治天下者五:贵有德,贵贵,贵老,敬长,慈幼。此五者,先王之所以定天下也。贵有德,何为也?为其近于道也。贵贵,为其近于君也。贵老,为其近于亲也。敬长,为其近于兄也。慈幼,为其近于子也。是故至孝近乎王,至弟近乎霸。至孝近乎王,虽天子,必有父;至弟近乎霸,虽诸侯,必有兄。先王之教,因而弗改,所以领天下国家也。子曰:``立爱自亲始,教民睦也。立教自长始,教民顺也。教以慈睦,而民贵有亲;教以敬长,而民贵用命。孝以事亲,顺以听命,错诸天下,无所不行。''

郊之祭也,丧者不敢哭,凶服者不敢入国门,敬之至也。祭之日,君牵牲,穆答君,卿大夫序从。既入庙门,丽于碑,卿大夫袒,而毛牛尚耳,鸾刀以刲,取膟菺,乃退。爓祭,祭腥而退,敬之至也。郊之祭,大报天而主日,配以月。夏后氏祭其闇,殷人祭其阳,周人祭日,以朝及闇。祭日于坛,祭月于坎,以别幽明,以制上下。祭日于东,祭月于西,以别外内,以端其位。日出于东,月生于西。阴阳长短,终始相巡,以致天下之和。天下之礼,致反始也,致鬼神也,致和用也,致义也,致让也。致反始,以厚其本也;致鬼神,以尊上也;致物用,以立民纪也。致义,则上下不悖逆矣。致让,以去争也。合此五者,以治天下之礼也,虽有奇邪,而不治者则微矣。

宰我曰:``吾闻鬼神之名,而不知其所谓。''子曰:``气也者,神之盛也;魄也者,鬼之盛也;合鬼与神,教之至也。众生必死,死必归土:此之谓鬼。骨肉毙于下,阴为野土;其气发扬于上,为昭明,焄蒿,凄怆,此百物之精也,神之着也。因物之精,制为之极,明命鬼神,以为黔首则。百众以畏,万民以服。''圣人以是为未足也,筑为宫室,谓为宗祧,以别亲疏远迩,教民反古复始,不忘其所由生也。众之服自此,故听且速也。二端既立,报以二礼。建设朝事,燔燎膻芗,见以萧光,以报气也。此教众反始也。荐黍稷,羞肝肺首心,见间以侠甒,加以郁鬯,以报魄也。教民相爱,上下用情,礼之至也。

君子反古复始,不忘其所由生也,是以致其敬,发其情,竭力从事,以报其亲,不敢弗尽也。是故昔者天子为藉千亩,冕而朱纮,躬秉耒。诸侯为藉百亩,冕而青纮,躬秉耒,以事天地、山川、社稷、先古,以为醴酪齐盛,于是乎取之,敬之至也。

古者天子、诸侯必有养兽之官,及岁时,齐戒沐浴而躬朝之。牺牷祭牲,必于是取之,敬之至也。君召牛,纳而视之,择其毛而卜之,吉,然后养之。君皮弁素积,朔月,月半,君巡牲,所以致力,孝之至也。古者天子、诸侯必有公桑、蚕室,近川而为之。筑宫仞有三尺,棘墙而外闭之。及大昕之朝,君皮弁素积,卜三宫之夫人世妇之吉者,使入蚕于蚕室,奉种浴于川;桑于公桑,风戾以食之。岁既殚矣,世妇卒蚕,奉茧以示于君,遂献茧于夫人。夫人曰:``此所以为君服与?''遂副袆而受之,因少牢以礼之。古之献茧者,其率用此与!及良日,夫人缫,三盆手,遂布于三宫夫人世妇之吉者使缫;遂朱绿之,玄黄之,以为黼黻文章。服既成,君服以祀先王先公,敬之至也。君子曰:礼乐不可斯须去身。致乐以治心,则易直子谅之心,油然生矣。易直子谅之心生则乐,乐则安,安则久,久则天,天则神。天则不言而信,神则不怒而威。致乐以治心者也。致礼以治躬则庄敬,庄敬则严威。心中斯须不和不乐,而鄙诈之心入之矣;外貌斯须不庄不敬,而慢易之心入之矣。故乐也者,动于内者也,礼也者,动于外者也。乐极和,礼极顺。内和而外顺,则民瞻其颜色而不与争也;望其容貌,而众不生慢易焉。故德辉动乎内,而民莫不承听;理发乎外,而众莫不承顺。故曰:致礼乐之道,而天下塞焉,举而措之无难矣。乐也者,动于内者也;礼也者,动于外者也。故礼主其减,乐主其盈。礼减而进,以进为文;乐盈而反,以反为文。礼减而不进则销,乐盈而不反则放。故礼有报而乐有反。礼得其报则乐,乐得其反则安。礼之报,乐之反,其义一也。

曾子曰:``孝有三:大孝尊亲,其次弗辱,其下能养。''公明仪问于曾子曰:``夫子可以为孝乎?''曾子曰:``是何言与!是何言与!君子之所为孝者:先意承志,谕父母于道。参,直养者也,安能为孝乎?''

曾子曰:``身也者,父母之遗体也。行父母之遗体,敢不敬乎?居处不庄,非孝也;事君不忠,非孝也;莅官不敬,非孝也;朋友不信,非孝也;战陈无勇,非孝也;五者不遂,灾及于亲,敢不敬乎?亨孰膻芗,尝而荐之,非孝也,养也。君子之所谓孝也者,国人称愿然曰:『幸哉有子!』如此,所谓孝也已。众之本教曰孝,其行曰养。养,可能也,敬为难;敬,可能也,安为难;安,可能也,卒为难。父母既没,慎行其身,不遗父母恶名,可谓能终矣。仁者,仁此者也;礼者,履此者也;义者,宜此者也;信者,信此者也;强者,强此者也。乐自顺此生,刑自反此作。''曾子曰:``夫孝,置之而塞乎天地,溥之而横乎四海,施诸后世而无朝夕,推而放诸东海而准,推而放诸西海而准,推而放诸南海而准,推而放诸北海而准。《诗》云:『自西自东,自南自北,无思不服。』此之谓也。''曾子曰:``树木以时伐焉,禽兽以时杀焉。夫子曰:『断一树,杀一兽,不以其时,非孝也。』孝有三:小孝用力,中孝用劳,大孝不匮。思慈爱忘劳,可谓用力矣。尊仁安义,可谓用劳矣。博施备物,可谓不匮矣。父母爱之,嘉而弗忘;父母恶之,惧而无怨;父母有过,谏而不逆;父母既没,必求仁者之粟以祀之。此之谓礼终。''乐正子春下堂而伤其足,数月不出,犹有忧色。门弟子曰:``夫子之足瘳矣,数月不出,犹有忧色,何也?''乐正子春曰:``善如尔之问也!善如尔之问也!吾闻诸曾子,曾子闻诸夫子曰:『天之所生,地之所养,无人为大。』父母全而生之,子全而归之,可谓孝矣。不亏其体,不辱其身,可谓全矣。故君子顷步而弗敢忘孝也。今予忘孝之道,予是以有忧色也。壹举足而不敢忘父母,壹出言而不敢忘父母。壹举足而不敢忘父母,是故道而不径,舟而不游,不敢以先父母之遗体行殆。壹出言而不敢忘父母,是故恶言不出于口,忿言不反于身。不辱其身,不羞其亲,可谓孝矣。''

昔者,有虞氏贵德而尚齿,夏后氏贵爵而尚齿,殷人贵富而尚齿,周人贵亲而尚齿。虞夏殷周,天下之盛王也,未有遗年者。年之贵乎天下,久矣;次乎事亲也。是故朝廷同爵则尚齿。七十杖于朝,君问则席。八十不俟朝,君问则就之,而弟达乎朝廷矣。行,肩而不并,不错则随。见老者,则车徒辟;斑白者不以其任行乎道路,而弟达乎道路矣。居乡以齿,而老穷不遗,强不犯弱,众不暴寡,而弟达乎州巷矣。古之道,五十不为甸徒,颁禽隆诸长者,而弟达乎搜狩矣。军旅什伍,同爵则尚齿,而弟达乎军旅矣。孝弟发诸朝廷,行乎道路,至乎州巷,放乎搜狩,修乎军旅,众以义死之,而弗敢犯也。

祀乎明堂,所以教诸侯之孝也;食三老五更于大学,所以教诸侯之弟也。祀先贤于西学,所以教诸侯之德也;耕藉,所以教诸侯之养也;朝觐,所以教诸侯之臣也。五者,天下之大教也。食三老五更于大学,天子袒而割牲,执酱而馈,执爵而酳,冕而揔干,所以教诸侯之弟也。是故,乡里有齿,而老穷不遗,强不犯弱,众不暴寡,此由大学来者也。天子设四学,当入学,而大子齿。天子巡守,诸侯待于竟。天子先见百年者。八、十九十者东行,西行者弗敢过;西行,东行者弗敢过。欲言政者,君就之可也。壹命齿于乡里,再命齿于族,三命不齿;族有七十者,弗敢先。七十者,不有大故不入朝;若有大故而入,君必与之揖让,而后及爵者。天子有善,让德于天;诸侯有善,归诸天子;卿大夫有善,荐于诸侯;士、庶人有善,本诸父母,存诸长老;禄爵庆赏,成诸宗庙;所以示顺也。昔者,圣人建阴阳天地之情,立以为《易》。易抱龟南面,天子卷冕北面,虽有明知之心,必进断其志焉。示不敢专,以尊天也。善则称人,过则称己。教不伐以尊贤也。孝子将祭祀,必有齐庄之心以虑事,以具服物,以修宫室,以治百事。及祭之日,颜色必温,行必恐,如惧不及爱然。其奠之也,容貌必温,身必诎,如语焉而未之然。宿者皆出,其立卑静以正,如将弗见然。及祭之后,陶陶遂遂,如将复入然。是故,悫善不违身,耳目不违心,思虑不违亲。结诸心,形诸色,而术省之-\/-孝子之志也。建国之神位:右社稷,而左宗庙。

\hypertarget{header-n651}{%
\subsection{祭统}\label{header-n651}}

凡治人之道,莫急于礼。礼有五经,莫重于祭。夫祭者,非物自外至者也,自中出生于心也;心怵而奉之以礼。是故,唯贤者能尽祭之义。贤者之祭也,必受其福。非世所谓福也。福者,备也;备者,百顺之名也。无所不顺者,谓之备。言:内尽于己,而外顺于道也。忠臣以事其君,孝子以事其亲,其本一也。上则顺于鬼神,外则顺于君长,内则以孝于亲。如此之谓备。唯贤者能备,能备然后能祭。是故,贤者之祭也:致其诚信与其忠敬,奉之以物,道之以礼,安之以乐,参之以时。明荐之而已矣。不求其为。此孝子之心也。祭者,所以追养继孝也。孝者畜也。顺于道不逆于伦,是之谓畜。是故,孝子之事亲也,有三道焉:生则养,没则丧,丧毕则祭。养则观其顺也,丧则观其哀也,祭则观其敬而时也。尽此三道者,孝子之行也。既内自尽,又外求助,昏礼是也。故国君取夫人之辞曰:``请君之玉女与寡人共有敝邑,事宗庙社稷。''此求助之本也。夫祭也者,必夫妇亲之,所以备外内之官也;官备则具备。水草之菹,陆产之醢,小物备矣;三牲之俎,八簋之实,美物备矣;昆虫之异,草木之实,阴阳之物备矣。凡天之所生,地之所长,茍可荐者,莫不咸在,示尽物也。外则尽物,内则尽志,此祭之心也。是故,天子亲耕于南郊,以共齐盛;王后蚕于北郊,以共纯服。诸侯耕于东郊,亦以共齐盛;夫人蚕于北郊,以共冕服。天子诸侯非莫耕也,王后夫人非莫蚕也,身致其诚信,诚信之谓尽,尽之谓敬,敬尽然后可以事神明,此祭之道也。及时将祭,君子乃齐。齐之为言齐也。齐不齐以致齐者也。是以君子非有大事也,非有恭敬也,则不齐。不齐则于物无防也,嗜欲无止也。及其将齐也,防其邪物,讫其嗜欲,耳不听乐。故记曰:``齐者不乐'',言不敢散其志也。心不茍虑,必依于道;手足不茍动,必依于礼。是故君子之齐也,专致其精明之德也。故散齐七日以定之,致齐三日以齐之。定之之谓齐。齐者精明之至也,然后可以交于神明也。是故,先期旬有一日,宫宰宿夫人,夫人亦散齐七日,致齐三日。君致齐于外,夫人致齐于内,然后会于大庙。君纯冕立于阼,夫人副袆立于东房。君执圭瓒裸尸,大宗执璋瓒亚裸。及迎牲,君执纼,卿大夫从士执刍。宗妇执盎从夫人荐涚水。君执鸾刀羞哜,夫人荐豆,此之谓夫妇亲之。

及入舞,君执干戚就舞位,君为东上,冕而揔干,率其群臣,以乐皇尸。是故天子之祭也,与天下乐之;诸侯之祭也,与竟内乐之。冕而揔干,率其群臣,以乐皇尸,此与竟内乐之之义也。夫祭有三重焉:献之属,莫重于裸,声莫重于升歌,舞莫重于《武宿夜》,此周道也。凡三道者,所以假于外而以增君子之志也,故与志进退;志轻则亦轻,志重则亦重。轻其志而求外之重也,虽圣人弗能得也。是故君子之祭也,必身自尽也,所以明重也。道之以礼,以奉三重,而荐诸皇尸,此圣人之道也。夫祭有馂;馂者祭之末也,不可不知也。是故古之人有言曰:``善终者如始。''馂其是已。是故古之君子曰:``尸亦馂鬼神之余也,惠术也,可以观政矣。''是故尸谡,君与卿四人馂。君起,大夫六人馂;臣馂君之余也。大夫起,士八人馂;贱馂贵之余也。士起,各执其具以出,陈于堂下,百官进,彻之,下馂上之余也。凡馂之道,每变以众,所以别贵贱之等,而兴施惠之象也。是故以四簋黍见其修于庙中也。庙中者竟内之象也。祭者泽之大者也。是故上有大泽则惠必及下,顾上先下后耳。非上积重而下有冻馁之民也。是故上有大泽,则民夫人待于下流,知惠之必将至也,由馂见之矣。故曰:``可以观政矣。''

夫祭之为物大矣,其兴物备矣。顺以备者也,其教之本与?是故,君子之教也,外则教之以尊其君长,内则教之以孝于其亲。是故,明君在上,则诸臣服从;崇事宗庙社稷,则子孙顺孝。尽其道,端其义,而教生焉。是故君子之事君也,必身行之,所不安于上,则不以使下;所恶于下,则不以事上;非诸人,行诸己,非教之道也。是故君子之教也,必由其本,顺之至也,祭其是与?故曰:祭者,教之本也已。夫祭有十伦焉;见事鬼神之道焉,见君臣之义焉,见父子之伦焉,见贵贱之等焉,见亲疏之杀焉,见爵赏之施焉,见夫妇之别焉,见政事之均焉,见长幼之序焉,见上下之际焉。此之谓十伦。

〔祭有十伦〕铺筵设同几,为依神也;诏祝于室,而出于祊,此交神明之道也。君迎牲而不迎尸,别嫌也。尸在庙门外,则疑于臣,在庙中则全于君;君在庙门外则疑于君,入庙门则全于臣、全于子。是故,不出者,明君臣之义也。夫祭之道,孙为王父尸。所使为尸者,于祭者子行也;父北面而事之,所以明子事父之道也。此父子之伦也。尸饮五,君洗玉爵献卿;尸饮七,以瑶爵献大夫;尸饮九,以散爵献士及群有司,皆以齿。明尊卑之等也。

夫祭有昭穆,昭穆者,所以别父子、远近、长幼、亲疏之序而无乱也。是故,有事于大庙,则群昭群穆咸在而不失其伦。此之谓亲疏之杀也。古者,明君爵有德而禄有功,必赐爵禄于大庙,示不敢专也。故祭之日,一献,君降立于阼阶之南,南乡。所命北面,史由君右执策命之。再拜稽首。受书以归,而舍奠于其庙。此爵赏之施也。君卷冕立于阼,夫人副袆立于东房。夫人荐豆执校,执醴授之执镫。尸酢夫人执柄,夫人受尸执足。夫妇相授受,不相袭处,酢必易爵。明夫妇之别也。凡为俎者,以骨为主。骨有贵贱;殷人贵髀,周人贵肩,凡前贵于后。俎者,所以明祭之必有惠也。是故,贵者取贵骨,贱者取贱骨。贵者不重,贱者不虚,示均也。惠均则政行,政行则事成,事成则功立。功之所以立者,不可不知也。俎者,所以明惠之必均也。善为政者如此,故曰:见政事之均焉。

凡赐爵,昭为一,穆为一。昭与昭齿,穆与穆齿,凡群有司皆以齿,此之谓长幼有序。夫祭有畀辉胞翟阍者,惠下之道也。唯有德之君为能行此,明足以见之,仁足以与之。畀之为言与也,能以其余畀其下者也。辉者,甲吏之贱者也;胞者,肉吏之贱者也;翟者,乐吏之贱者也;阍者,守门之贱者也。古者不使刑人守门,此四守者,吏之至贱者也。尸又至尊;以至尊既祭之末,而不忘至贱,而以其余畀之。是故明君在上,则竟内之民无冻馁者矣,此之谓上下之际。

凡祭有四时:春祭曰礿,夏祭曰禘,秋祭曰尝,冬祭曰烝。礿、禘,阳义也;尝、烝,阴义也。禘者阳之盛也,尝者阴之盛也。故曰:莫重于禘、尝。古者于禘也,发爵赐服,顺阳义也;于尝也,出田邑,发秋政,顺阴义也。故记曰:``尝之日,发公室,示赏也;草艾则墨;未发秋政,则民弗敢草也。''故曰:禘、尝之义大矣。治国之本也,不可不知也。明其义者君也,能其事者臣也。不明其义,君人不全;不能其事,为臣不全。夫义者,所以济志也,诸德之发也。是故其德盛者,其志厚;其志厚者,其义章。其义章者,其祭也敬。祭敬则竟内之子孙莫敢不敬矣。是故君子之祭也,必身亲莅之;有故,则使人可也。虽使人也,君不失其义者,君明其义故也。其德薄者,其志轻,疑于其义,而求祭;使之必敬也,弗可得已。祭而不敬,何以为民父母矣?夫鼎有铭,铭者,自名也。自名以称扬其先祖之美,而明着之后世者也。为先祖者,莫不有美焉,莫不有恶焉,铭之义,称美而不称恶,此孝子孝孙之心也。唯贤者能之。铭者,论譔其先祖之有德善,功烈勋劳庆赏声名列于天下,而酌之祭器;自成其名焉,以祀其先祖者也。显扬先祖,所以崇孝也。身比焉,顺也。明示后世,教也。夫铭者,壹称而上下皆得焉耳矣。是故君子之观于铭也,既美其所称,又美其所为。为之者,明足以见之,仁足以与之,知足以利之,可谓贤矣。贤而勿伐,可谓恭矣。故卫孔悝之鼎铭曰:六月丁亥,公假于大庙。公曰:``叔舅!乃祖庄叔,左右成公。成公乃命庄叔随难于汉阳,即宫于宗周,奔走无射。启右献公。献公乃命成叔,纂乃祖服。乃考文叔,兴旧耆欲,作率庆士,躬恤卫国,其勤公家,夙夜不解,民咸曰:『休哉!』''公曰:``叔舅!予女铭:若纂乃考服。''悝拜稽首曰:``对扬以辟之,勤大命施于烝彝鼎。''此卫孔悝之鼎铭也。古之君子论譔其先祖之美,而明着之后世者也。以比其身,以重其国家如此。子孙之守宗庙社稷者,其先祖无美而称之,是诬也;有善而弗知,不明也;知而弗传,不仁也。此三者,君子之所耻也。昔者,周公旦有勋劳于天下。周公既没,成王、康王追念周公之所以勋劳者,而欲尊鲁;故赐之以重祭。外祭,则郊社是也;内祭,则大尝禘是也。夫大尝禘,升歌《清庙》,下而管《象》;朱干玉戚,以舞《大武》;八佾,以舞《大夏》;此天子之乐也。康周公,故以赐鲁也。子孙纂之,至于今不废,所以明周公之德而又以重其国也。

\hypertarget{header-n661}{%
\subsection{经解}\label{header-n661}}

孔子曰:``入其国,其教可知也。其为人也:温柔敦厚,《诗》教也;疏通知远,《书》教也;广博易良,《乐》教也;洁静精微,《易》教也;恭俭庄敬,《礼》教也;属辞比事,《春秋》教也。故《诗》之失,愚;《书》之失,诬;《乐》之失,奢;《易》之失,贼;《礼》之失,烦;《春秋》之失,乱。其为人也:温柔敦厚而不愚,则深于《诗》者也;疏通知远而不诬,则深于《书》者也;广博易良而不奢,则深于《乐》者也;洁静精微而不贼,则深于《易》者也;恭俭庄敬而不烦,则深于《礼》者也;属辞比事而不乱,则深于《春秋》者也。''

天子者,与天地参。故德配天地,兼利万物,与日月并明,明照四海而不遗微小。其在朝廷,则道仁圣礼义之序;燕处,则听雅、颂之音;行步,则有环佩之声;升车,则有鸾和之音。居处有礼,进退有度,百官得其宜,万事得其序。《诗》云:``淑人君子,其仪不忒。其仪不忒,正是四国。''此之谓也。发号出令而民说,谓之和;上下相亲,谓之仁;民不求其所欲而得之,谓之信;除去天地之害,谓之义。义与信,和与仁,霸王之器也。有治民之意而无其器,则不成。

礼之于正国也:犹衡之于轻重也,绳墨之于曲直也,规矩之于方圜也。故衡诚县,不可欺以轻重;绳墨诚陈,不可欺以曲直;规矩诚设,不可欺以方圆;君子审礼,不可诬以奸诈。是故,隆礼由礼,谓之有方之士;不隆礼、不由礼,谓之无方之民。敬让之道也。故以奉宗庙则敬,以入朝廷则贵贱有位,以处室家则父子亲、兄弟和,以处乡里则长幼有序。孔子曰:``安上治民,莫善于礼。''此之谓也。

故朝觐之礼,所以明君臣之义也。聘问之礼,所以使诸侯相尊敬也。丧祭之礼,所以明臣子之恩也。乡饮酒之礼,所以明长幼之序也。昏姻之礼,所以明男女之别也。夫礼,禁乱之所由生,犹坊止水之所自来也。故以旧坊为无所用而坏之者,必有水败;以旧礼为无所用而去之者,必有乱患。故昏姻之礼废,则夫妇之道苦,而淫辟之罪多矣。乡饮酒之礼废,则长幼之序失,而争斗之狱繁矣。丧祭之礼废,则臣子之恩薄,而倍死忘生者众矣。聘觐之礼废,则君臣之位失,诸侯之行恶,而倍畔侵陵之败起矣。

故礼之教化也微,其止邪也于未形,使人日徙善远罪而不自知也。是以先王隆之也。《易》曰:``君子慎始,差若毫厘,缪以千里。''此之谓也。

\hypertarget{header-n669}{%
\subsection{哀公问}\label{header-n669}}

哀公问于孔子曰:``大礼何如?君子之言礼,何其尊也?''孔子曰:``丘也小人,不足以知礼。''君曰:``否!吾子言之也。''孔子曰:``丘闻之:民之所由生,礼为大。非礼无以节事天地之神也,非礼无以辨君臣上下长幼之位也,非礼无以别男女父子兄弟之亲、昏姻疏数之交也;君子以此之为尊敬然。然后以其所能教百姓,不废其会节。有成事,然后治其雕镂文章黼黻以嗣。其顺之,然后言其丧算,备其鼎俎,设其豕腊,修其宗庙,岁时以敬祭祀,以序宗族。即安其居,节丑其衣服,卑其宫室,车不雕几,器不刻镂,食不贰味,以与民同利。昔之君子之行礼者如此。''

公曰:``今之君子胡莫行之也?''孔子曰:``今之君子,好实无厌,淫德不倦,荒怠傲慢,固民是尽,午其众以伐有道;求得当欲,不以其所。昔之用民者由前,今之用民者由后。今之君子莫为礼也。''

孔子侍坐于哀公,哀公曰:``敢问人道谁为大?''孔子愀然作色而对曰:``君之及此言也,百姓之德也!固臣敢无辞而对?人道,政为大。''公曰:``敢问何谓为政?''孔子对曰:``政者正也。君为正,则百姓从政矣。君之所为,百姓之所从也。君所不为,百姓何从?''公曰:``敢问为政如之何?''孔子对曰:``夫妇别,父子亲,君臣严。三者正,则庶物从之矣。''公曰:``寡人虽无似也,愿闻所以行三言之道,可得闻乎?''孔子对曰:``古之为政,爱人为大;所以治爱人,礼为大;所以治礼,敬为大;敬之至矣,大昏为大。大昏至矣!大昏既至,冕而亲迎,亲之也。亲之也者,亲之也。是故,君子兴敬为亲;舍敬,是遗亲也。弗爱不亲;弗敬不正。爱与敬,其政之本与!''

公曰:``寡人愿有言。然冕而亲迎,不已重乎?''孔子愀然作色而对曰:``合二姓之好,以继先圣之后,以为天地宗庙社稷之主,君何谓已重乎?''公曰:``寡人固!不固,焉得闻此言也。寡人欲问,不得其辞,请少进!''孔子曰:``天地不合,万物不生。大昏,万世之嗣也,君何谓已重焉!''孔子遂言曰:``内以治宗庙之礼,足以配天地之神明;出以治直言之礼,足以立上下之敬。物耻足以振之,国耻足以兴之。为政先礼。礼,其政之本与!''孔子遂言曰:``昔三代明王之政,必敬其妻子也,有道。妻也者,亲之主也,敢不敬与?子也者,亲之后也,敢不敬与?君子无不敬也,敬身为大。身也者,亲之枝也,敢不敬与?不能敬其身,是伤其亲;伤其亲,是伤其本;伤其本,枝从而亡。三者,百姓之象也。身以及身,子以及子,妃以及妃,君行此三者,则忾乎天下矣,大王之道也。如此,国家顺矣。''

公曰:``敢问何谓敬身?''孔子对曰:``君子过言,则民作辞;过动,则民作则。君子言不过辞,动不过则,百姓不命而敬恭,如是,则能敬其身;能敬其身,则能成其亲矣。''公曰:``敢问何谓成亲?''孔子对曰:``君子也者,人之成名也。百姓归之名,谓之君子之子。是使其亲为君子也,是为成其亲之名也已!''孔子遂言曰:``古之为政,爱人为大。不能爱人,不能有其身;不能有其身,不能安土;不能安土,不能乐天;不能乐天,不能成其身。''

公曰:``敢问何谓成身?''孔子对曰:``不过乎物。''公曰:``敢问君子何贵乎天道也?''孔子对曰:``贵其『不已』。如日月东西相从而不已也,是天道也;不闭其久,是天道也;无为而物成,是天道也;已成而明,是天道也。''公曰:``寡人蠢愚,冥烦子志之心也。''孔子蹴然辟席而对曰:``仁人不过乎物,孝子不过乎物。是故,仁人之事亲也如事天,事天如事亲,是故孝子成身。''公曰:``寡人既闻此言也,无如后罪何?''孔子对曰:``君之及此言也,是臣之福也。''

\hypertarget{header-n678}{%
\subsection{仲尼燕居}\label{header-n678}}

仲尼燕居,子张、子贡、言游侍,纵言至于礼。子曰:``居!女三人者,吾语女礼,使女以礼周流无不遍也。''子贡越席而对曰:``敢问何如?''子曰:``敬而不中礼,谓之野;恭而不中礼,谓之给;勇而不中礼,谓之逆。''子曰:``给夺慈仁。''子曰:``师,尔过;而商也不及。子产犹众人之母也,能食之不能教也。''子贡越席而对曰:``敢问将何以为此中者也?''子曰:``礼乎礼!夫礼所以制中也。''

子贡退,言游进曰:``敢问礼也者,领恶而全好者与?''子曰:``然。''``然则何如?''子曰:``郊社之义,所以仁鬼神也;尝禘之礼,所以仁昭穆也;馈奠之礼,所以仁死丧也;射乡之礼,所以仁乡党也;食飨之礼,所以仁宾客也。''子曰:``明乎郊社之义、尝禘之礼,治国其如指诸掌而已乎!是故,以之居处有礼,故长幼辨也。以之闺门之内有礼,故三族和也。以之朝廷有礼,故官爵序也。以之田猎有礼,故戎事闲也。以之军旅有礼,故武功成也。是故,宫室得其度,量鼎得其象,味得其时,乐得其节,车得其式,鬼神得其飨,丧纪得其哀,辨说得其党,官得其体,政事得其施;加于身而错于前,凡众之动得其宜。''

子曰:``礼者何也?即事之治也。君子有其事,必有其治。治国而无礼,譬犹瞽之无相与?伥伥其何之?譬如终夜有求于幽室之中,非烛何见?若无礼则手足无所错,耳目无所加,进退揖让无所制。是故,以之居处,长幼失其别;闺门,三族失其和;朝廷,官爵失其序;田猎,戎事失其策;军旅,武功失其制;宫室,失其度;量鼎,失其象;味,失其时;乐,失其节;车,失其式;鬼神,失其飨;丧纪,失其哀;辩说,失其党;官,失其体;政事,失其施;加于身而错于前,凡众之动,失其宜。如此,则无以祖洽于众也。''

子曰:``慎听之!女三人者,吾语女:礼犹有九焉,大飨有四焉。茍知此矣,虽在畎亩之中事之,圣人已。两君相见,揖让而入门,入门而县兴;揖让而升堂,升堂而乐阕。下管《象》、《武》,《夏》、《龠》序兴。陈其荐俎,序其礼乐,备其百官。如此,而后君子知仁焉。行中规,还中矩,和鸾中采齐,客出以雍,彻以振羽。是故,君子无物而不在礼矣。入门而金作,示情也。升歌《清庙》,示德也。下而管《象》,示事也。是故古之君子,不必亲相与言也,以礼乐相示而已。''

子曰:``礼也者,理也;乐也者,节也。君子无理不动,无节不作。不能《诗》,于礼缪;不能乐,于礼素;薄于德,于礼虚。''子曰:``制度在礼,文为在礼,行之,其在人乎!''子贡越席而对曰:``敢问:夔其穷与?''子曰:``古之人与?古之人也。达于礼而不达于乐,谓之素;达于乐而不达于礼,谓之偏。夫夔,达于乐而不达于礼,是以传此名也,古之人也。''

子张问政,子曰:``师乎!前,吾语女乎?君子明于礼乐,举而错之而已。''子张复问。子曰:``师,尔以为必铺几筵,升降酌献酬酢,然后谓之礼乎?尔以为必行缀兆。兴羽龠,作钟鼓,然后谓之乐乎?言而履之,礼也。行而乐之,乐也。君子力此二者以南面而立,夫是以天下太平也。诸侯朝,万物服体,而百官莫敢不承事矣。礼之所兴,众之所治也;礼之所废,众之所乱也。目巧之室,则有奥阼,席则有上下,车则有左右,行则有随,立则有序,古之义也。室而无奥阼,则乱于堂室也。席而无上下,则乱于席上也。车而无左右,则乱于车也。行而无随,则乱于涂也。立而无序,则乱于位也。昔圣帝明王诸侯,辨贵贱、长幼、远近、男女、外内,莫敢相逾越,皆由此涂出也。''三子者,既得闻此言也于夫子,昭然若发蒙矣。

\hypertarget{header-n687}{%
\subsection{孔子闲居}\label{header-n687}}

孔子闲居,子夏侍。子夏曰:``敢问《诗》云:『凯弟君子,民之父母』,何如斯可谓民之父母矣?''孔子曰:``夫民之父母乎,必达于礼乐之原,以致五至,而行三无,以横于天下。四方有败,必先知之。此之谓民之父母矣。''

子夏曰:``民之父母,既得而闻之矣;敢问何谓『五至』?''孔子曰:``志之所至,诗亦至焉。诗之所至,礼亦至焉。礼之所至,乐亦至焉。乐之所至,哀亦至焉。哀乐相生。是故,正明目而视之,不可得而见也;倾耳而听之,不可得而闻也;志气塞乎天地,此之谓五至。''

子夏曰:``五至既得而闻之矣,敢问何谓三无?''孔子曰:``无声之乐,无体之礼,无服之丧,此之谓三无。''子夏曰:``三无既得略而闻之矣,敢问何诗近之?''孔子曰:``『夙夜其命宥密』,无声之乐也。『威仪逮逮,不可选也』,无体之礼也。『凡民有丧,匍匐救之』,无服之丧也。''

子夏曰:``言则大矣!美矣!盛矣!言尽于此而已乎?''孔子曰:``何为其然也!君子之服之也,犹有五起焉。''子夏曰:``何如?''子曰:``无声之乐,气志不违;无体之礼,威仪迟迟;无服之丧,内恕孔悲。无声之乐,气志既得;无体之礼,威仪翼翼;无服之丧,施及四国。无声之乐,气志既从;无体之礼,上下和同;无服之丧,以畜万邦。无声之乐,日闻四方;无体之礼,日就月将;无服之丧,纯德孔明。无声之乐,气志既起;无体之礼,施及四海;无服之丧,施于孙子。''

子夏曰:``三王之德,参于天地,敢问:何如斯可谓参于天地矣?''孔子曰:``奉三无私以劳天下。''子夏曰:``敢问何谓三无私?''孔子曰:``天无私覆,地无私载,日月无私照。奉斯三者以劳天下,此之谓三无私。其在《诗》曰:『帝命不违,至于汤齐。汤降不迟,圣敬日齐。昭假迟迟,上帝是祗。帝命式于九围。』是汤之德也。天有四时,春秋冬夏,风雨霜露,无非教也。地载神气,神气风霆,风霆流形,庶物露生,无非教也。清明在躬,气志如神,嗜欲将至,有开必先。天降时雨,山川出云。其在《诗》曰:『嵩高惟岳,峻极于天。惟岳降神,生甫及申。惟申及甫,惟周之翰。四国于蕃,四方于宣。』此文武之德也。三代之王也,必先令闻,《诗》云:『明明天子,令闻不已。』三代之德也。『弛其文德,协此四国。』大王之德也。''子夏蹶然而起,负墙而立曰:``弟子敢不承乎!''

\hypertarget{header-n695}{%
\subsection{坊记}\label{header-n695}}

子言之:``君子之道,辟则坊与,坊民之所不足者也。''大为之坊,民犹逾之。故君子礼以坊德,刑以坊淫,命以坊欲。

子云:``小人贫斯约,富斯骄;约斯盗,骄斯乱。''礼者,因人之情而为之节文,以为民坊者也。故圣人之制富贵也使民富不足以骄,贫不至于约,贵不慊于上,故乱益亡。子云:``贫而好乐,富而好礼,众而以宁者,天下其几矣。《诗》云:『民之贪乱,宁为荼毒。』''故制:国不过千乘,都城不过百雉,家富不过百乘。以此坊民,诸侯犹有畔者。

子云:``夫礼者,所以章疑别微,以为民坊者也。''故贵贱有等,衣服有别,朝廷有位,则民有所让。子云:``天无二日,土无二王,家无二主,尊无二上,示民有君臣之别也。''《春秋》不称楚越之王丧,礼君不称天,大夫不称君,恐民之惑也。《诗》云:``相彼盍旦,尚犹患之。''子云:``君不与同姓同车,与异姓同车不同服,示民不嫌也。''以此坊民,民犹得同姓以弒其君。

子云:``君子辞贵不辞贱,辞富不辞贫,则乱益亡。''故君子与其使食浮于人也,宁使人浮于食。子云:``觞酒豆肉让而受恶,民犹犯齿;衽席之上让而坐下,民犹犯贵;朝廷之位让而就贱,民犹犯君。''《诗》云:``民之无良,相怨一方;受爵不让,至于已斯亡。''子云:``君子贵人而贱己,先人而后己,则民作让。''故称人之君曰君,自称其君曰寡君。子云:``利禄,先死者而后生者,则民不偝;先亡者而后存者,则民可以托。''《诗》云:``先君之思,以畜寡人。''以此坊民,民犹偝死而号无告。

子云:``有国家者,贵人而贱禄,则民兴让;尚技而贱车,则民兴艺。''故君子约言,小人先言。

子云:``上酌民言,则下天上施;上不酌民言,则犯也;下不天上施,则乱也。''故君子信让以莅百姓,则民之报礼重。《诗》云:``先民有言,询于刍荛。''

子云:``善则称人,过则称己,则民不争;善则称人,过则称己,则怨益亡。''《诗》云:``尔卜尔筮,履无咎言。''子云:``善则称人,过则称己,则民让善。''《诗》云:``考卜惟王,度是镐京;惟龟正之,武王成之。''子云:``善则称君,过则称己,则民作忠。''《君陈》曰:``尔有嘉谋嘉猷,入告尔君于内,女乃顺之于外,曰:此谋此猷,惟我君之德。于乎!是惟良显哉。''子云:``善则称亲,过则称己,则民作孝。''《大誓》曰:``予克纣,非予武,惟朕文考无罪;纣克予,非朕文考有罪,惟予小子无良。''

子云:``君子弛其亲之过,而敬其美。''《论语》曰:``三年无改于父之道,可谓孝矣。''高宗云:``三年其惟不言,言乃讙。''子云:``从命不忿,微谏不倦,劳而不怨,可谓孝矣。''《诗》云:``孝子不匮。''子云:``睦于父母之党,可谓孝矣。故君子因睦以合族。''《诗》云:``此令兄弟,绰绰有裕;不令兄弟,交相为愈。''

子云:``于父之执,可以乘其车,不可以衣其衣。君子以广孝也。''子云:``小人皆能养其亲,君子不敬,何以辨?''子云:``父子不同位,以厚敬也。''《书》云:``厥辟不辟,忝厥祖。''子云:``父母在,不称老,言孝不言慈;闺门之内,戏而不叹。''君子以此坊民,民犹薄于孝而厚于慈。子云:``长民者,朝廷敬老,则民作孝。''子云:``祭祀之有尸也,宗庙之主也,示民有事也。修宗庙,敬祀事,教民追孝也。''以此坊民,民犹忘其亲。

子云:``敬则用祭器。故君子不以菲废礼,不以美没礼。''故食礼:主人亲馈,则客祭;主人不亲馈,则客不祭。故君子茍无礼,虽美不食焉。《易》曰:``东邻杀牛,不如西邻之禴祭,实受其福。''《诗》云:``既醉以酒,既饱以德。''以此示民,民犹争利而忘义。

子云:``七日戒,三日齐,承一人焉以为尸,过之者趋走,以教敬也。''醴酒在室,醍酒在堂,澄酒在下,示民不淫也。尸饮三,众宾饮一,示民有上下也。因其酒肉,聚其宗族,以教民睦也。故堂上观乎室,堂下观乎上。《诗》云:``礼仪卒度,笑语卒获。''

子云:``宾礼每进以让,丧礼每加以远。''浴于中溜,饭于牖下,小敛于户内,大敛于阼,殡于客位,祖于庭,葬于墓,所以示远也。殷人吊于圹,周人吊于家,示民不偝也。子云:``死,民之卒事也,吾从周。''以此坊民,诸侯犹有薨而不葬者。子云:``升自客阶,受吊于宾位,教民追孝也。''未没丧不称君,示民不争也。故鲁《春秋》记晋丧曰:``杀其君之子奚齐及其君卓。''以此坊民,子犹有弒其父者。

子云:``孝以事君,弟以事长'',示民不贰也,故君子有君不谋仕,唯卜之日称二君。丧父三年,丧君三年,示民不疑也。父母在,不敢有其身,不敢私其财,示民有上下也。故天子四海之内无客礼,莫敢为主焉。故君适其臣,升自阼阶,即位于堂,示民不敢有其室也。父母在,馈献不及车马,示民不敢专也。以此坊民,民犹忘其亲而贰其君。子云:``礼之先币帛也,欲民之先事而后禄也。''先财而后礼,则民利;无辞而行情,则民争。故君子于有馈者,弗能见则不视其馈。《易》曰:``不耕获,不菑畬,凶。''以此坊民,民犹贵禄而贱行。子云:``君子不尽利以遗民。''《诗》云:``彼有遗秉,此有不敛穧,伊寡妇之利。''故君子仕则不稼,田则不渔;食时不力珍,大夫不坐羊,士不坐犬。《诗》云:``采葑采菲,无以下体,德音莫违,及尔同死。''以此坊民,民犹忘义而争利,以亡其身。

子云:``夫礼,坊民所淫,章民之别,使民无嫌,以为民纪者也。''故男女无媒不交,无币不相见,恐男女之无别也。以此坊民,民犹有自献其身。《诗》云:``伐柯如之何?匪斧不克;取妻如之何?匪媒不得;蓺麻如之何?横从其亩;取妻如之何?必告父母。''子云:``取妻不取同姓,以厚别也。''故买妾不知其姓,则卜之。以此坊民,鲁《春秋》犹去夫人之姓曰吴,其死曰孟子卒。子云:``礼,非祭,男女不交爵。''以此坊民,阳侯犹杀缪侯而窃其夫人。故大飨废夫人之礼。子云:``寡妇之子,不有见焉,则弗友也,君子以辟远也。''故朋友之交,主人不在,不有大故,则不入其门。以此坊民,民犹以色厚于德。子云:``好德如好色。''诸侯不下渔色。故君子远色以为民纪。故男女授受不亲。御妇人则进左手。姑姊妹女子子已嫁而反,男子不与同席而坐。寡妇不夜哭。妇人疾,问之不问其疾。以此坊民,民犹淫泆而乱于族。子云:``婚礼,婿亲迎,见于舅姑,舅姑承子以授婿,恐事之违也。''以此坊民,妇犹有不至者。

\hypertarget{header-n712}{%
\subsection{中庸 }\label{header-n712}}

天命之谓性,率性之谓道,修道之谓教。道也者,不可须臾离也,可离非道也。是故君子戒慎乎其所不睹,恐惧乎其所不闻。莫见乎隐,莫显乎微,故君子慎其独也。喜怒哀乐之未发,谓之中;发而皆中节,谓之和;中也者,天下之大本也;和也者,天下之达道也。致中和,天地位焉,万物育焉。

仲尼曰:``君子中庸,小人反中庸,君子之中庸也,君子而时中;小人之中庸也,小人而无忌惮也。``

子曰:``中庸其至矣乎!民鲜能久矣!``

子曰:``道之不行也,我知之矣:知者过之,愚者不及也。道之不明也,我知之矣:贤者过之,不肖者不及也。人莫不饮食也,鲜能知味也。``

子曰:``道其不行矣夫!``

子曰:``舜其大知也与!舜好问而好察迩言,隐恶而扬善,执其两端,用其中于民,其斯以为舜乎!``

子曰:``人皆曰予知,驱而纳诸罟擭陷阱之中,而莫之知辟也。人皆曰予知,
择乎中庸,而不能期月守也。``

子曰:``回之为人也,择乎中庸,得一善,则拳拳服膺弗失之矣。``

子曰:``天下国家可均也,爵禄可辞也,白刃可蹈也,中庸不可能也。``

子路问强,子曰:``南方之强与?北方之强与?抑而强与?宽柔以教,不报无道,南方之强也,君子居之。衽金革,死而不厌,北方之强也,而强者居之。故君子和而不流,强哉矫!中立而不倚,强哉矫!国有道,不变塞焉,强哉矫!国无道,至死不变,强哉矫!``

子曰:``素隐行怪,后世有述焉,吾弗为之矣。君子遵道而行,半涂而废,吾弗能已矣。君子依乎中庸,遁世不见知而不悔,唯圣者能之。``君子之道,费而隐。夫妇之愚,可以与知焉,及其至也,虽圣人亦有所不知焉。夫妇之不肖,可以能行焉;及其至也,虽圣人亦有所不能焉。天地之大也,人犹有所憾。故君子语大,天下莫能载焉;语小,天下莫能破焉。《诗》云:``鸢飞戾天,鱼跃于渊。``言其上下察也。君子之道,造端乎夫妇,及其至也,察乎天地。

子曰:``道不远人,人之为道而远人,不可以为道。《诗》云:`伐柯,伐柯,其则不远。'执柯以伐柯,睨而视之,犹以为远。故君子以人治人,改而止。忠恕违道不远,施诸己而不愿,亦勿施于人。君子之道四,丘未能一焉,所求乎子,以事父,未能也;所求乎臣,以事君,未能也;所求乎弟,以事兄,未能也;
所求乎朋友,先施之,未能也。庸德之行,庸言之谨;有所不足,不敢不勉,有余,不敢尽;言顾行,行顾言,君子胡不慥慥尔!``君子素其位而行,不愿乎其外。素富贵,行乎富贵;素贫贱,行乎贫贱;素夷狄,行乎夷狄;素患难行乎患难,君子无入而不自得焉。在上位不陵下,在下位不援上,正己而不求于人,则无怨。上不怨天,下不尤人。故君子居易以俟命。小人行险以徼幸。

子曰:``射有似乎君子,失诸正鹄,反求诸其身。``君子之道,辟如行远必自迩,辟如登高必自卑。《诗》曰:``妻子好合,如鼓瑟琴。兄弟既翕,和乐且耽。宜尔室家,乐尔妻帑。``

子曰:``父母其顺矣乎!``

子曰:``鬼神之为德,其盛矣乎?视之而弗见,听之而弗闻,体物而不可遗,使天下之人齐明盛服,以承祭祀。洋洋乎如在其上,如在其左右。《诗》曰:`神之格思,不可度思!矧可射思!'夫微之显,诚之不可掩如此夫。``

子曰:``舜其大孝也与!德为圣人,尊为天子,富有四海之内。宗庙飨之,子孙保之。故大德必得其位,必得其禄。必得其名,必得其寿,故天之生物,必因其材而笃焉。故栽者培之,倾者覆之。《诗》曰:`嘉乐君子,宪宪令德。宜
民宜人,受禄于天,保佑命之,自天申之。'故大德者必受命。``

子曰:``无忧者,其惟文王乎!以王季为父,以武王为子,父作之,子述之。
武王缵大王、王季、文王之绪,壹戎衣而有天下。身不失天下之显名,尊为天子,富有四海之内。宗庙飨之,子孙保之。武王末受命,周公成文、武之德,追王大王、王季,上祀先公以天子之礼。斯礼也,达乎诸侯大夫,及士庶人。父为
大夫,子为士,葬以大夫,祭以士。父为士,子为大夫,葬以士,祭以大夫。期之丧,达乎大夫。三年之丧,达乎天子。父母之丧,无贵贱一也。``

子曰:``武王、周公,其达孝矣乎!夫孝者,善继人之志,善述人之事者也。春秋修其祖庙,陈其宗器,设其裳衣,荐其时食。宗庙之礼,所以序昭穆也。序爵,所以辨贵贱也。序事,所以辨贤也。旅酬下为上,所以逮贱也。燕毛,所以序齿也。践其位,行其礼,奏其乐,敬其所尊,爱其所亲,事死如事生,事亡如
事存,孝之至也。郊社之礼,所以事上帝也。宗庙之礼,所以祀乎其先也。明乎郊社之礼、禘尝之义,治国其如示诸掌乎!``

哀公问政。子曰:``文武之政,布在方策。其人存,则其政举;其人亡,则其政息。人道敏政,地道敏树。夫政也者,蒲卢也。故为政在人,取人以身,修身以道,修道以仁。仁者人也。亲亲为大;义者宜也。尊贤为大。亲亲之杀,尊贤之等,礼所生也。在下位不获乎上,民不可得而治矣!故君子不可以不修身;
思修身,不可以不事亲;思事亲,不可以不知人,思知人,不可以不知天。````天下之达道五,所以行之者三。曰:君臣也,父子也,夫妇也,昆弟也,
朋友之交也,五者天下之达道也。知,仁,勇,三者天下之达德也,所以行之者一也。或生而知之,或学而知之,或困而知之,及其知之一也。或安而行之,
或利而行之,或勉强而行之,及其成功,一也。

子曰:``好学近乎知,力行近乎仁,
知耻近乎勇。知斯三者,则知所以修身;知所以修身,则知所以治人;知所以治人,则知所以治天下国家矣。凡为天下国家有九经,曰:修身也。尊贤也,亲亲也,敬大臣也,体群臣也。子庶民也,来百工也,柔远人也,怀诸侯也。修身则道立,尊贤则不惑,亲亲则诸父昆弟不怨,敬大臣则不眩,体群臣则士之报礼重,子庶民则百姓劝,来百工则财用足,柔远人则四方归之,怀诸侯则天下畏之。齐明盛服,非礼不动。
所以修身也;去谗远色,贱货而贵德,所以劝贤也;尊其位,重其禄,同其好恶,
所以劝亲亲也;官盛任使,所以劝大臣也;忠信重禄,所以劝士也;时使薄敛,所以劝百姓也;日省月试,既廪称事,所以劝百工也;送往迎来,嘉善而矜不能,
所以柔远人也;继绝世,举废国,治乱持危。朝聘以时,厚往而薄来,所以怀诸侯也。凡为天下国家有九经,所以行之者一也。凡事豫则立,不豫则废。言前定则不跲,事前定则不困,行前定则不疚,道前定则不穷。在下位不获乎上,民不可得而治矣。获乎上有道,不信乎朋友,
不获乎上矣;信乎朋友有道,不顺乎亲,不信乎朋友矣;顺乎亲有道,反诸身不诚,不顺乎亲矣;诚身有道,不明乎善,不诚乎身矣。诚者,天之道也;诚之者,人之道也。诚者不勉而中,不思而得,从容中道,圣人也。诚之者,择善而固执之者也。博学之,审问之,慎思之,明辨之,笃行之。有弗学,学之弗能,弗措也;有弗问,问之弗知,弗措也;有弗思,思之弗得,弗措也;有弗辨,辨之弗明,
弗措也;有弗行,行之弗笃,弗措也。人一能之己百之,人十能之己千之。果能此道矣。虽愚必明,虽柔必强。``

自诚明谓之性。自明诚谓之教。诚则明矣,明则诚矣。唯天下至诚,为能尽其性;能尽其性,则能尽人之性;能尽人之性,则能尽
物之性;能尽物之性,则可以赞天地之化育;可以赞天地之化育,则可以与天地参矣。其次致曲。曲能有诚,诚则形,形则著,著则明,明则动,动则变,变则化。唯天下至诚为能化。至诚之道,可以前知。国家将兴,必有祯祥;国家将亡,必有妖孽。见乎著龟,动乎四体。祸福将至,善必先知之;不善必先知之。故至诚如神。诚者自成也,而道自道也。诚者物之终始,不诚无物。是故君子诚之为贵。
诚者非自成己而已也,所以成物也。成己仁也;成物知也。性之德也,合外内之道也,故时措之宜也。

故至诚无息,不息则久,久则征;征则悠远,悠远则博厚,博厚则高明。博厚所以载物也;高明所以覆物也;悠久所以成物也。博厚配地,高明配天,
悠久无疆。如此者不见而章,不动而变,无为而成。天地之道,可一言而尽也。其为物不贰,则其生物不测。天地之道,博也,厚也,高也,明也,悠也,久也。今夫天,斯昭昭之多,及其无穷也,日月星辰系焉,万物覆焉。今夫地,一撮土之多。及其广厚,载华岳而不重,振河海而不泄,万物载焉。今夫山,一卷石之多,及其广大,草木生之,禽兽居之,宝藏兴焉,今夫水,一勺之多,及其不测,
鼋、鼍、蛟龙、鱼鳖生焉,货财殖焉。《诗》曰:``惟天之命,于穆不已!``盖曰天之所以为天也。``于乎不显,文王之德之纯!``盖曰文王之所以为文也,纯亦不已。

大哉!圣人之道洋洋乎!发育万物,峻极于天。优优大哉!礼仪三百,威仪三千。待其人然后行。故曰:苟不至德,至道不凝焉。故君子尊德性而道问学。致广大而尽精微。极高明而道中庸。温故而知新,敦厚以崇礼。是故居上不骄,为下不倍;国有道,其言足以兴;国无道,其默足以容。《诗》曰:``既明且哲,以保其身。``其此之谓与!

子曰:``愚而好自用,贱而好自专,生乎今之世,反古之道:如此者,灾及其身者也。``非天子,不议礼,不制度,不考文。今天下车同轨,书同文,行同伦。虽有其位,苟无其德,不敢作礼乐焉;虽有其德。苟无其位,亦不敢作礼乐焉。

子曰:``吾说夏礼,杞不足徵也。吾学殷礼,有宋存焉。吾学周礼,今用之,吾从周。``王天下有三重焉,其寡过矣乎!上焉者虽善无徵,无徵不信,不信民弗从;下焉者虽善不尊,不尊不信,不信民弗从。故君子之道:本诸身,徵诸庶民,考诸三王而不缪,建诸天地而不悖,质诸鬼神而无疑,百世以俟圣人而不惑。质诸鬼神而无疑,知天也;百世以俟圣人而不惑,知人也。是故君子动而世为天下道,行而世为天下法,言而世为天下则。远之则有望,近之则不厌。《诗》曰:``在彼无恶,在此无射。庶几夙夜,以永终誉!``君子未有不如此,而蚤有誉于天下者。

仲尼祖述尧舜,宪章文武:上律天时,下袭水土。辟如天地之无不持载,无不覆帱,辟如四时之错行,如日月之代明。万物并育而不相害,道并行而不相悖,小德川流,大德敦化,此天地之这所以为大也。

唯天下至圣为能聪明睿知,足以有临也;宽裕温柔,足以有容也;发强刚毅,足以有执也;齐庄中正,足以有敬也;文理密察,足以有别也。溥博渊泉,而时出之。溥博如天,渊泉如渊。见而民莫不敬,言而民莫不信,行而民莫不说。是以声名洋溢乎中国,施及蛮貊。舟车所至,人力所通,天之所覆,地之所载,日月所照,霜露所队,凡有血气者,莫不尊亲,故曰配天。

唯天下至诚,为能经纶天下之大经,立天下之大本,知天地之化育。夫焉有所倚?肫肫其仁!渊渊其渊!浩浩其天!苟不固聪明圣知达天德者,其孰能知之?

《诗》曰:``衣锦尚絅``,恶其文之著也。故君子之道,闇然而日章;小人之道,的然而日亡。君子之道:淡而面不厌,简而文,温而理,知远之近,知风之自,知微之显,可与入德矣。《诗》云:``潜虽伏矣,亦孔之昭!``故君子内省不疚,无恶于志。君子之所不可及者,其唯人之所不见乎!《诗》云:``相在尔室,尚不愧于屋漏。``故君子不动而敬,不言而信。《诗》曰:``奏假无言,时靡有争。``是故君子不赏而民劝,不怒而民威于鈇钺。《诗》曰:``不显惟德!百辟其刑之。``是故君子笃恭而天下平。《诗》云:``予怀明德,不大声以色。``子曰:``声色之于以化民。末也。``《诗》曰:``德輶如毛。``毛犹有伦,上天之载,无声无臭,至矣!

\hypertarget{header-n744}{%
\subsection{表记}\label{header-n744}}

子言之:``归乎!君子隐而显,不矜而庄,不厉而威,不言而信。''子曰:``君子不失足于人,不失色于人,不失口于人,是故君子貌足畏也,色足惮也,言足信也。《甫刑》曰:『敬忌而罔有择言在躬。』''子曰:``裼袭之不相因也,欲民之毋相渎也。''子曰:``祭极敬,不继之以乐;朝极辨,不继之以倦。''子曰:``君子慎以辟祸,笃以不掩,恭以远耻。''子曰:``君子庄敬日强,安肆日偷。君子不以一日使其躬儳焉,如不终日。''子曰:``齐戒以事鬼神,择日月以见君,恐民之不敬也。''子曰:``狎侮,死焉而不畏也。''子曰:``无辞不相接也,无礼不相见也;欲民之毋相亵也。《易》曰:『初筮告,再三渎,渎则不告。』''

子言之:``仁者,天下之表也;义者,天下之制也;报者,天下之利也。''子曰:``以德报德,则民有所劝;以怨报怨,则民有所惩。《诗》曰:『无言不雠,无德不报。』《太甲》曰:『民非后无能胥以宁;后非民无以辟四方。』''子曰:``以德报怨,则宽身之仁也;以怨报德,则刑戮之民也。''子曰:``无欲而好仁者,无畏而恶不仁者,天下一人而已矣。是故君子议道自己,而置法以民。''子曰:``仁有三,与仁同功而异情。与仁同功,其仁未可知也;与仁同过,然后其仁可知也。仁者安仁,知者利仁,畏罪者强仁。仁者右也,道者左也。仁者人也,道者义也。厚于仁者薄于义,亲而不尊;厚于义者薄于仁,尊而不亲。道有至,义有考。至道以王,义道以霸,考道以为无失。''

子言之:``仁有数,义有长短小大。中心憯怛,爱人之仁也;率法而强之,资仁者也。《诗》云:『丰水有芑,武王岂不仕!诒厥孙谋,以燕翼子,武王烝哉!』数世之仁也。国风曰:『我今不阅,皇恤我后。』终身之仁也。''子曰:``仁之为器重,其为道远,举者莫能胜也,行者莫能致也,取数多者仁也;夫勉于仁者不亦难乎?是故君子以义度人,则难为人;以人望人,则贤者可知已矣。''子曰:``中心安仁者,天下一人而已矣。大雅曰:『德輶如毛,民鲜克举之;我仪图之,惟仲山甫举之,爱莫助之。』''小雅曰:``高山仰止,景行行止。''子曰:``《诗》之好仁如此;乡道而行,中道而废,忘身之老也,不知年数之不足,俛焉日有孳孳,毙而后已。''子曰:``仁之难成久矣!人人失其所好;故仁者之过易辞也。''子曰:``恭近礼,俭近仁,信近情,敬让以行此,虽有过,其不甚矣。夫恭寡过,情可信,俭易容也;以此失之者,不亦鲜乎?《诗》曰:『温温恭人,惟德之基。』''子曰:``仁之难成久矣,惟君子能之。是故君子不以其所能者病人,不以人之所不能者愧人。是故圣人之制行也,不制以己,使民有所劝勉愧耻,以行其言。礼以节之,信以结之,容貌以文之,衣服以移之,朋友以极之,欲民之有壹也。小雅曰:『不愧于人,不畏于天。』是故君子服其服,则文以君子之容;有其容,则文以君子之辞;遂其辞,则实以君子之德。是故君子耻服其服而无其容,耻有其容而无其辞,耻有其辞而无其德,耻有其德而无其行。是故君子衰绖则有哀色;端冕则有敬色;甲胄则有不可辱之色。《诗》云:『惟鹈在梁,不濡其翼;彼记之子,不称其服。』''

子言之:``君子之所谓义者,贵贱皆有事于天下;天子亲耕,粢盛秬鬯以事上帝,故诸侯勤以辅事于天子。''子曰:``下之事上也,虽有庇民之大德,不敢有君民之心,仁之厚也。是故君子恭俭以求役仁,信让以求役礼,不自尚其事,不自尊其身,俭于位而寡于欲,让于贤,卑己尊而人,小心而畏义,求以事君,得之自是,不得自是,以听天命。《诗》云:『莫莫葛藟,施于条枚;凯弟君子,求福不回。』其舜、禹、文王、周公之谓与!有君民之大德,有事君之小心。《诗》云:『惟此文王,小心翼翼,昭事上帝,聿怀多福,厥德不回,以受方国。』''子曰:``先王谥以尊名,节以壹惠,耻名之浮于行也。是故君子不自大其事,不自尚其功,以求处情;过行弗率,以求处厚;彰人之善而美人之功,以求下贤。是故君子虽自卑,而民敬尊之。''子曰:``后稷,天下之为烈也,岂一手一足哉!唯欲行之浮于名也,故自谓便人。''

子言之:``君子之所谓仁者其难乎!《诗》云:『凯弟君子,民之父母。』凯以强教之;弟以说安之。乐而毋荒,有礼而亲,威庄而安,孝慈而敬。使民有父之尊,有母之亲。如此而后可以为民父母矣,非至德其孰能如此乎?今父之亲子也,亲贤而下无能;母之亲子也,贤则亲之,无能则怜之。母,亲而不尊;父,尊而不亲。水之于民也,亲而不尊;火,尊而不亲。土之于民也,亲而不尊;天,尊而不亲。命之于民也,亲而不尊;鬼,尊而不亲。''子曰:``夏道尊命,事鬼敬神而远之,近人而忠焉,先禄而后威,先赏而后罚,亲而不尊;其民之敝:蠢而愚,乔而野,朴而不文。殷人尊神,率民以事神,先鬼而后礼,先罚而后赏,尊而不亲;其民之敝:荡而不静,胜而无耻。周人尊礼尚施,事鬼敬神而远之,近人而忠焉,其赏罚用爵列,亲而不尊;其民之敝:利而巧,文而不惭,贼而蔽。''子曰:``夏道未渎辞,不求备,不大望于民,民未厌其亲;殷人未渎礼,而求备于民;周人强民,未渎神,而赏爵刑罚穷矣。''子曰:``虞夏之道,寡怨于民;殷周之道,不胜其敝。''子曰:``虞夏之质,殷周之文,至矣。虞夏之文不胜其质;殷周之质不胜其文。''

子言之曰:``后世虽有作者,虞帝弗可及也已矣。君天下,生无私,死不厚其子;子民如父母,有憯怛之爱,有忠利之教;亲而尊,安而敬,威而爱,富而有礼,惠而能散;其君子尊仁畏义,耻费轻实,忠而不犯,义而顺,文而静,宽而有辨。《甫刑》曰:『德威惟威,德明惟明。』非虞帝其孰能如此乎?''子言之:``事君先资其言,拜自献其身,以成其信。是故君有责于其臣,臣有死于其言。故其受禄不诬,其受罪益寡。''子曰:``事君大言入则望大利,小言入则望小利;故君子不以小言受大禄,不以大言受小禄。《易》曰:『不家食,吉。』''子曰:``事君不下达,不尚辞,非其人弗自。小雅曰:『靖共尔位,正直是与;神之听之,式谷以女。』''子曰:``事君远而谏,则谄也;近而不谏,则尸利也。''子曰:``迩臣守和,宰正百官,大臣虑四方。''子曰:``事君欲谏不欲陈。《诗》云:『心乎爱矣,瑕不谓矣;中心藏之,何日忘之。』''子曰:``事君难进而易退,则位有序;易进而难退则乱也。故君子三揖而进,一辞而退,以远乱也。''子曰:``事君三违而不出竟,则利禄也;人虽曰不要,吾弗信也。''子曰:``事君慎始而敬终。''子曰:``事君可贵可贱,可富可贫,可生可杀,而不可使为乱。''子曰:``事君,军旅不辟难,朝廷不辞贱;处其位而不履其事则乱也。故君使其臣得志,则慎虑而从之;否,则孰虑而从之。终事而退,臣之厚也。《易》曰:『不事王侯,高尚其事。』''子曰:``唯天子受命于天,士受命于君。故君命顺则臣有顺命;君命逆则臣有逆命。《诗》曰:『鹊之姜姜,鹑之贲贲;人之无良,我以为君。』''

子曰:``君子不以辞尽人。故天下有道,则行有枝叶;天下无道,则辞有枝叶。是故君子于有丧者之侧,不能赙焉,则不问其所费;于有病者之侧,不能馈焉,则不问其所欲;有客,不能馆,则不问其所舍。故君子之接如水,小人之接如醴;君子淡以成,小人甘以坏。小雅曰:『盗言孔甘,乱是用餤。』''子曰:``君子不以口誉人,则民作忠。故君子问人之寒,则衣之;问人之饥,则食之;称人之美,则爵之。国风曰:『心之忧矣,于我归说。』''子曰:``口惠而实不至,怨菑及其身。是故君子与其有诺责也,宁有已怨。国风曰:『言笑晏晏,信誓旦旦,不思其反;反是不思,亦已焉哉!』''子曰:``君子不以色亲人;情疏而貌亲,在小人则穿窬之盗也与?''子曰:``情欲信,辞欲巧。''

子言之:``昔三代明王皆事天地之神明,无非卜筮之用,不敢以其私,亵事上帝。是故不犯日月,不违卜筮。卜筮不相袭也。大事有时日;小事无时日,有筮。外事用刚日,内事用柔日。不违龟筮。''子曰:``牲牷礼乐齐盛,是以无害乎鬼神,无怨乎百姓。''子曰:``后稷之祀易富也;其辞恭,其欲俭,其禄及子孙。《诗》曰:『后稷兆祀,庶无罪悔,以迄于今。』''子曰:``大人之器威敬。天子无筮;诸侯有守筮。天子道以筮;诸侯非其国不以筮。卜宅寝室。天子不卜处大庙。''子曰:``君子敬则用祭器。是以不废日月,不违龟筮,以敬事其君长,是以上不渎于民,下不亵于上。''

\hypertarget{header-n755}{%
\subsection{缁衣}\label{header-n755}}

子言之曰:``为上易事也,为下易知也,则刑不烦矣。''子曰:``好贤如《缁衣》,恶恶如《巷伯》,则爵不渎而民作愿,刑不试而民咸服。大雅曰:『仪刑文王,万国作孚。』''子曰:``夫民,教之以德,齐之以礼,则民有格心;教之以政,齐之以刑,则民有遁心。故君民者,子以爱之,则民亲之;信以结之,则民不倍;恭以莅之,则民有孙心。《甫刑》曰:『苗民罪用命,制以刑,惟作五虐之刑曰法。是以民有恶德,而遂绝其世也。』''

子曰:``下之事上也,不从其所令,从其所行。上好是物,下必有甚者矣。故上之所好恶,不可不慎也,是民之表也。''子曰:``禹立三年,百姓以仁遂焉,岂必尽仁?《诗》云:『赫赫师尹,民具尔瞻。』《甫刑》曰:『一人有庆,兆民赖之。』大雅曰:『成王之孚,下土之式。』''子曰:``上好仁,则下之为仁争先人。故长民者章志、贞教、尊仁,以子爱百姓;民致行己以说其上矣。《诗》云:『有梏德行,四国顺之。』''

子曰:``王言如丝,其出如纶;王言如纶,其出如綍。故大人不倡游言。可言也,不可行。君子弗言也;可行也,不可言,君子弗行也。则民言不危行,而行不危言矣。《诗》云:『淑慎尔止,不愆于仪。』''子曰:``君子道人以言,而禁人以行。故言必虑其所终,而行必稽其所敝;则民谨于言而慎于行。《诗》云:『慎尔出话,敬尔威仪。』大雅曰:『穆穆文王,于缉熙敬止。』''

子曰:``长民者,衣服不贰,从容有常,以齐其民,则民德壹。《诗》云:『彼都人士,狐裘黄黄,其容不改,出言有章,行归于周,万民所望。』''子曰:``为上可望而知也,为下可述而志也,则君不疑于其臣,而臣不惑于其君矣。《尹吉》曰:『惟尹躬及汤,咸有壹德。』《诗》云:『淑人君子,其仪不忒。』''

子曰:``有国者章义恶,以示民厚,则民情不贰。《诗》云:『靖共尔位,好是正直。』''子曰:``上人疑则百姓惑,下难知则君长劳。故君民者,章好以示民俗,慎恶以御民之淫,则民不惑矣。臣仪行,不重辞,不援其所不及,不烦其所不知,则君不劳矣。《诗》云:『上帝板板,下民卒。』小雅曰:『匪其止共,惟王之邛。』''

子曰:``政之不行也,教之不成也,爵禄不足劝也,刑罚不足耻也。故上不可以亵刑而轻爵。《康诰》曰:『敬明乃罚。』《甫刑》曰:『播刑之不迪。』''

子曰:``大臣不亲,百姓不宁,则忠敬不足,而富贵已过也;大臣不治而迩臣比矣。故大臣不可不敬也,是民之表也;迩臣不可不慎也,是民之道也。君毋以小谋大,毋以远言近,毋以内图外,则大臣不怨,迩臣不疾,而远臣不蔽矣。叶公之顾命曰:『毋以小谋败大作,毋以嬖御人疾庄后,毋以嬖御士疾庄士、大夫、卿士。』''

子曰:``大人不亲其所贤,而信其所贱;民是以亲失,而教是以烦。《诗》云:『彼求我则,如不我得;执我仇仇,亦不我力。』《君陈》曰:『未见圣,若己弗克见;既见圣,亦不克由圣。』''

子曰:``小人溺于水,君子溺于口,大人溺于民,皆在其所亵也。夫水近于人而溺人,德易狎而难亲也,易以溺人;口费而烦,易出难悔,易以溺人;夫民闭于人,而有鄙心,可敬不可慢,易以溺人。故君子不可以不慎也。《太甲》曰:『毋越厥命以自覆也;若虞机张,往省括于厥度则释。』《兑命》曰:『惟口起羞,惟甲胄起兵,惟衣裳在笥,惟干戈省厥躬。』《太甲》曰:『天作孽,可违也;自作孽,不可以逭。』《尹吉》曰:『惟尹躬天,见于西邑;夏自周有终,相亦惟终。』''

子曰:``民以君为心,君以民为体;心庄则体舒,心肃则容敬。心好之,身必安之;君好之,民必欲之。心以体全,亦以体伤;君以民存,亦以民亡。《诗》云:『昔吾有先正,其言明且清,国家以宁,都邑以成,庶民以生;谁能秉国成,不自为正,卒劳百姓。《君雅》曰:『夏日暑雨,小民惟曰怨;资冬祁寒,小民亦惟曰怨。』''

子曰:``下之事上也,身不正,言不信,则义不壹,行无类也。''子曰:``言有物而行有格也;是以生则不可夺志,死则不可夺名。故君子多闻,质而守之;多志,质而亲之;精知,略而行之。《君陈》曰:『出入自尔师虞,庶言同。』《诗》云:『淑人君子,其仪一也。』''

子曰:``唯君子能好其正,小人毒其正。故君子之朋友有乡,其恶有方;是故迩者不惑,而远者不疑也。《诗》云:『君子好仇。』''子曰:``轻绝贫贱,而重绝富贵,则好贤不坚,而恶恶不着也。人虽曰不利,吾不信也。《诗》云:『朋有攸摄,摄以威仪。』''子曰:``私惠不归德,君子不自留焉。《诗》云:『人之好我,示我周行。』''

子曰:``茍有车,必见其轼;茍有衣,必见其敝;人茍或言之,必闻其声;茍或行之,必见其成。《葛覃》曰:『服之无射。』''子曰:``言从而行之,则言不可饰也;行从而言之,则行不可饰也。故君子寡言,而行以成其信,则民不得大其美而小其恶。《诗》云:『自圭之玷,尚可磨也;斯言之玷,不可为也。』小雅曰:『允也君子,展也大成。』《君奭》曰:『昔在上帝,周田观文王之德,其集大命于厥躬。』''子曰:``南人有言曰:『人而无恒,不可以为卜筮。』古之遗言与?龟筮犹不能知也,而况于人乎?《诗》云:『我龟既厌,不我告犹。』《兑命》曰:『爵无及恶德,民立而正事,纯而祭祀,是为不敬;事烦则乱,事神则难。』《易》曰:『不恒其德,或承之羞。恒其德侦,妇人吉,夫子凶。』''

\hypertarget{header-n771}{%
\subsection{奔丧}\label{header-n771}}

奔丧之礼:始闻亲丧,以哭答使者,尽哀;问故,又哭尽哀。遂行,日行百里,不以夜行。唯父母之丧,见星而行,见星而舍。若未得行,则成服而后行。过国至竟,哭尽哀而止。哭辟市朝。望其国竟哭。至于家,入门左,升自西阶,殡东,西面坐,哭尽哀,括发袒,降堂东即位,西乡哭,成踊,袭绖于序东,绞带。反位,拜宾成踊,送宾,反位;有宾后至者,则拜之,成踊、送宾皆如初。众主人兄弟皆出门,出门哭止;阖门,相者告就次。于又哭,括发袒成踊;于三哭,犹括发袒成踊。三日,成服,拜宾、送宾皆如初。

奔丧者非主人,则主人为之拜宾送宾。奔丧者自齐衰以下,入门左中庭北面哭尽哀,免麻于序东,即位袒,与主人哭成踊。于又哭、三哭皆免袒,有宾则主人拜宾、送宾。丈夫妇人之待之也,皆如朝夕哭,位无变也。

奔母之丧,西面哭尽哀,括发袒,降堂东即位,西乡哭,成踊,袭免绖于序东,拜宾、送宾,皆如奔父之礼,于又哭不括发。妇人奔丧,升自东阶,殡东,西面坐,哭尽哀;东髽,即位,与主人拾踊。奔丧者不及殡,先之墓,北面坐,哭尽哀。主人之待之也,即位于墓左,妇人墓右,成踊尽哀括发,东即主人位,绖绞带,哭成踊,拜宾,反位,成踊,相者告事毕。遂冠归,入门左,北面哭尽哀,括发袒成踊,东即位,拜宾成踊。宾出,主人拜送;有宾后至者则拜之成踊;送宾如初。众主人兄弟皆出门,出门哭止,相者告就次。于又哭,括发成踊;于三哭,犹括发成踊。三日成服,于五哭,相者告事毕。为母所以异于父者,壹括发,其余免以终事,他如奔父之礼。

齐衰以下不及殡:先之墓,西面哭尽哀,免麻于东方,即位,与主人哭成踊,袭。有宾则主人拜宾、送宾;宾有后至者,拜之如初。相者告事毕。遂冠归,入门左,北面哭尽哀,免袒成踊,东即位,拜宾成踊,宾出,主人拜送。于又哭,免袒成踊;于三哭,犹免袒成踊。三日成服,于五哭,相者告事毕。

闻丧不得奔丧,哭尽哀;问故,又哭尽哀。乃为位,括发袒成踊,袭绖绞带即位,拜宾反位成踊。宾出,主人拜送于门外,反位;若有宾后至者,拜之成踊,送宾如初。于又哭,括发袒成踊,于三哭,犹括发袒成踊,三日成服,于五哭,拜宾送宾如初。若除丧而后归,则之墓,哭成踊,东括发袒绖,拜宾成踊,送宾反位,又哭尽哀,遂除,于家不哭。主人之待之也,无变于服,与之哭,不踊。自齐衰以下,所以异者,免麻。

凡为位,非亲丧,齐衰以下,皆即位哭尽哀,而东免绖,即位,袒、成踊、袭,拜宾反位,哭成踊,送宾反位,相者告就次。三日,五哭卒,主人出送宾;众主人兄弟皆出门,哭止。相者告事毕。成服拜宾。若所为位家远,则成服而往。齐衰,望乡而哭;大功,望门而哭;小功,至门而哭;缌麻,即位而哭。哭父之党于庙;母妻之党于寝;师于庙门外;朋友于寝门外;所识于野张帷。凡为位不奠。哭天子九,诸侯七,卿大夫五,士三。大夫哭诸侯,不敢拜宾。诸臣在他国,为位而哭,不敢拜宾。与诸侯为兄弟,亦为位而哭。凡为位者壹袒。所识者吊,先哭于家而后之墓,皆为之成踊,从主人北面而踊。凡丧,父在父为主;父没,兄弟同居,各主其丧。亲同,长者主之;不同,亲者主之。闻远兄弟之丧,既除丧而后闻丧,免袒成踊,拜宾则尚左手。无服而为位者,唯嫂叔;及妇人降而无服者麻。凡奔丧,有大夫至,袒,拜之,成踊而后袭;于士,袭而后拜之。

\hypertarget{header-n780}{%
\subsection{问丧}\label{header-n780}}

亲始死,鸡斯徒跣,扱上衽,交手哭。恻怛之心,痛疾之意,伤肾干肝焦肺,水浆不入口,三日不举火,故邻里为之糜粥以饮食之。夫悲哀在中,故形变于外也,痛疾在心,故口不甘味,身不安美也。

三日而敛,在床曰尸,在棺曰柩,动尸举柩,哭踊无数。恻怛之心,痛疾之意,悲哀志懑气盛,故袒而踊之,所以动体安心下气也。妇人不宜袒,故发胸击心爵踊,殷殷田田,如坏墙然,悲哀痛疾之至也。故曰:``辟踊哭泣,哀以送之。送形而往,迎精而反也。''

其往送也,望望然、汲汲然如有追而弗及也;其反哭也,皇皇然若有求而弗得也。故其往送也如慕,其反也如疑。

求而无所得之也,入门而弗见也,上堂又弗见也,入室又弗见也。亡矣丧矣!不可复见矣!故哭泣辟踊,尽哀而止矣。心怅焉怆焉、惚焉忾焉,心绝志悲而已矣。祭之宗庙,以鬼飨之,徼幸复反也。

成圹而归,不敢入处室,居于倚庐,哀亲之在外也;寝苫枕块,哀亲之在土也。故哭泣无时,服勤三年,思慕之心,孝子之志也,人情之实也。

或问曰:``死三日而后敛者,何也?''曰:孝子亲死,悲哀志懑,故匍匐而哭之,若将复生然,安可得夺而敛之也。故曰三日而后敛者,以俟其生也;三日而不生,亦不生矣。孝子之心亦益衰矣;家室之计,衣服之具,亦可以成矣;亲戚之远者,亦可以至矣。是故圣人为之断决以三日为之礼制也。

或问曰:``冠者不肉袒,何也?''曰:冠,至尊也,不居肉袒之体也,故为之免以代之也。

然则秃者不免,伛者不袒,跛者不踊,非不悲也;身有锢疾,不可以备礼也。故曰:丧礼唯哀为主矣。女子哭泣悲哀,击胸伤心;男子哭泣悲哀,稽颡触地无容,哀之至也。

或问曰:``免者以何为也?''曰:不冠者之所服也。《礼》曰:``童子不缌,唯当室缌。''缌者其免也,当室则免而杖矣。

或问曰:``杖者何也?''曰:竹、桐一也。故为父苴杖-\/-苴杖,竹也;为母削杖-\/-削杖,桐也。

或问曰:``杖者以何为也?''曰:孝子丧亲,哭泣无数,服勤三年,身病体羸,以杖扶病也。则父在不敢杖矣,尊者在故也;堂上不杖,辟尊者之处也;堂上不趋,示不遽也。此孝子之志也,人情之实也,礼义之经也,非从天降也,非从地出也,人情而已矣。

\hypertarget{header-n794}{%
\subsection{服问}\label{header-n794}}

传曰:``有从轻而重,公子之妻为其皇姑。有从重而轻,为妻之父母。有从无服而有服,公子之妻为公子之外兄弟。有从有服而无服,公子为其妻之父母。''传曰:``母出,则为继母之党服;母死,则为其母之党服。为其母之党服,则不为继母之党服。''

三年之丧,既练矣,有期之丧,既葬矣,则带其故葛带,绖期之绖,服其功衰。有大功之丧,亦如之。小功,无变也。

麻之有本者,变三年之葛。既练,遇麻断本者,于免,绖之;既免,去绖。每可以绖必绖;既绖,则去之。

小功不易丧之练冠,如免,则绖其缌小功之绖,因其初葛带。缌之麻,不变小功之葛;小功之麻,不变大功之葛。以有本为税。

殇:长、中,变三年之葛。终殇之月算,而反三年之葛。是非重麻,为其无卒哭之税。下殇则否。

君为天子三年,夫人如外宗之为君也。世子不为天子服。君所主:夫人、妻、大子适妇。大夫之适子为君、夫人、大子,如士服。

君之母,非夫人,则群臣无服。唯近臣及仆骖乘从服,唯君所服,服也。公为卿大夫锡衰以居,出亦如之。当事则弁绖。大夫相为,亦然。为其妻,往则服之,出则否。

凡见人无免绖,虽朝于君,无免绖。唯公门有税齐衰。传曰:``君子不夺人之丧,亦不可夺丧也。''传曰:``罪多而刑五,丧多而服五,上附下附列也。''

\hypertarget{header-n805}{%
\subsection{间传}\label{header-n805}}

斩衰何以服苴?苴,恶貌也,所以首其内而见诸外也。斩衰貌若苴,齐衰貌若枲,大功貌若止,小功、缌麻容貌可也,此哀之发于容体者也。

斩衰之哭,若往而不反;齐衰之哭,若往而反;大功之哭,三曲而偯;小功缌麻,哀容可也。此哀之发于声音者也。

斩衰,唯而不对;齐衰,对而不言;大功,言而不议;小功缌麻,议而不及乐。此哀之发于言语者也。

斩衰,三日不食;齐衰,二日不食;大功,三不食;小功缌麻,再不食;士与敛焉,则壹不食。故父母之丧,既殡食粥,朝一溢米,莫一溢米;齐衰之丧,疏食水饮,不食菜果;大功之丧,不食酰酱;小功缌麻,不饮醴酒。此哀之发于饮食者也。

父母之丧,既虞卒哭,疏食水饮,不食菜果;期而小祥,食菜果;又期而大祥,有酰酱;中月而禫,禫而饮醴酒。始饮酒者先饮醴酒。始食肉者先食干肉。

父母之丧,居倚庐,寝苫枕块,不说绖带;齐衰之丧,居垩室,芐翦不纳;大功之丧,寝有席,小功缌麻,床可也。此哀之发于居处者也。

父母之丧,既虞卒哭,柱楣翦屏,芐翦不纳;期而小祥,居垩室,寝有席;又期而大祥,居复寝;中月而禫,禫而床。

斩衰三升,齐衰四升、五升、六升,大功七升、八升、九升,小功十升、十一升、十二升,缌麻十五升去其半,有事其缕、无事其布曰缌。此哀之发于衣服者也。

斩衰三升,既虞卒哭,受以成布六升、冠七升;为母疏衰四升,受以成布七升、冠八升。去麻服葛,葛带三重。期而小祥,练冠縓缘,要绖不除,男子除乎首,妇人除乎带。男子何为除乎首也?妇人何为除乎带也?男子重首,妇人重带。除服者先重者,易服者易轻者。又期而大祥,素缟麻衣。中月而禫,禫而纤,无所不佩。

易服者何?为易轻者也。斩衰之丧,既虞卒哭,遭齐衰之丧,轻者包,重者特。既练,遭大功之丧,麻葛重。齐衰之丧,既虞卒哭,遭大功之丧,麻葛兼服之。斩衰之葛,与齐衰之麻同;齐衰之葛,与大功之麻同;大功之葛,与小功之麻同;小功之葛,与缌之麻同,麻同则兼服之。兼服之服重者,则易轻者也。

\hypertarget{header-n818}{%
\subsection{三年问}\label{header-n818}}

三年之丧何也?曰:称情而立文,因以饰群,别亲疏贵践之节,而不可损益也。故曰:无易之道也。创巨者其日久,痛甚者其愈迟,三年者,称情而立文,所以为至痛极也。斩衰苴杖,居倚庐,食粥,寝苫枕块,所以为至痛饰也。三年之丧,二十五月而毕;哀痛未尽,思慕未忘,然而服以是断之者,岂不送死者有已,复生有节哉?凡生天地之间者,有血气之属必有知,有知之属莫不知爱其类;今是大鸟兽,则失丧其群匹,越月逾时焉,则必反巡,过其故乡,翔回焉,鸣号焉,蹢躅焉,踟蹰焉,然后乃能去之;小者至于燕雀,犹有啁之顷焉,然后乃能去之;故有血气之属者,莫知于人,故人于其亲也,至死不穷。将由夫患邪淫之人与,则彼朝死而夕忘之,然而从之,则是曾鸟兽之不若也,夫焉能相与群居而不乱乎?将由夫修饰之君子与,则三年之丧,二十五月而毕,若驷之过隙,然而遂之,则是无穷也。故先王焉为之立中制节,壹使足以成文理,则释之矣。

然则何以至期也?曰:至亲以期断。是何也?曰:天地则已易矣,四时则已变矣,其在天地之中者,莫不更始焉,以是象之也。然则何以三年也?曰:加隆焉尔也,焉使倍之,故再期也。由九月以下何也?曰:焉使弗及也。故三年以为隆,缌小功以为杀,期九月以为间。上取象于天,下取法于地,中取则于人,人之所以群居和壹之理尽矣。故三年之丧,人道之至文者也,夫是之谓至隆。是百王之所同,古今之所壹也,未有知其所由来者也。孔子曰:``子生三年,然后免于父母之怀;夫三年之丧,天下之达丧也。''

\hypertarget{header-n823}{%
\subsection{深衣}\label{header-n823}}

古者深衣,盖有制度,以应规、矩、绳、权、衡。

短毋见肤,长毋被土。续衽,钩边。要缝半下;袼之高下,可以运肘;袂之长短,反诎之及肘。带下毋厌髀,上毋厌胁,当无骨者。制:十有二幅以应十有二月。

袂圜以应规;曲袷如矩以应方;负绳及踝以应直;下齐如权衡以应平。故规者,行举手以为容;负绳抱方者,以直其政,方其义也。故《易》曰:坤,``六二之动,直以方''也。下齐如权衡者,以安志而平心也。五法已施,故圣人服之。故规矩取其无私,绳取其直,权衡取其平,故先王贵之。故可以为文,可以为武,可以摈相,可以治军旅,完且弗费,善衣之次也。

具父母、大父母,衣纯以缋;具父母,衣纯以青。如孤子,衣纯以素。纯袂、缘、纯边,广各寸半。

\hypertarget{header-n830}{%
\subsection{投壶}\label{header-n830}}

投壶之礼,主人奉矢,司射奉中,使人执壶。主人请曰:``某有枉矢哨壶,请以乐宾。''宾曰:``子有旨酒嘉肴,某既赐矣,又重以乐,敢辞。''主人曰:``枉矢哨壶,不足辞也,敢以请。''宾曰:``某既赐矣,又重以乐,敢固辞。''主人曰:``枉矢哨壶,不足辞也,敢固以请。''宾曰:``某固辞不得命,敢不敬从?''宾再拜受,主人般还,曰:``辟。''主人阼阶上拜送,宾般还,曰:``辟。''已拜,受矢,进即两楹间,退反位,揖宾就筵。

司射进度壶,间以二矢半,反位,设中,东面,执八算兴。

请宾曰:``顺投为入。比投不释,胜饮不胜者,正爵既行,请为胜者立马,一马从二马,三马既立,请庆多马。''请主人亦如之。

命弦者曰:``请奏《狸首》,间若一。''大师曰:``诺。''

左右告矢具,请拾投。有入者,则司射坐而释一算焉。宾党于右,主党于左。

卒投,司射执算曰:``左右卒投,请数。''二算为纯,一纯以取,一算为奇。遂以奇算告曰:``某贤于某若干纯''。奇则曰奇,钧则曰左右钧。

命酌曰:``请行觞。''酌者曰:``诺。''当饮者皆跪奉觞,曰:``赐灌'';胜者跪曰:``敬养''。

正爵既行,请立马。马各直其算。一马从二马,以庆。庆礼曰:``三马既备,请庆多马。''宾主皆曰:``诺。''正爵既行,请彻马。

算多少视其坐。筹,室中五扶,堂上七扶,庭中九扶。算长尺二寸。壶:颈修七寸,腹修五寸,口径二寸半;容斗五升。壶中实小豆焉,为其矢之跃而出也。壶去席二矢半。矢以柘若棘,毋去其皮。鲁令弟子辞曰:毋幠,毋敖,毋偝立,毋逾言;偝立逾言,有常爵。薛令弟子辞曰:毋幠,毋敖,毋偝立,毋逾言;若是者浮。

取半以下为投壶礼,尽用之为射礼。司射、庭长,及冠士立者,皆属宾党;
乐人及使者、童子,皆属主党。

\hypertarget{header-n843}{%
\subsection{儒行}\label{header-n843}}

鲁哀公问于孔子曰:``夫子之服,其儒服与?''孔子对曰:``丘少居鲁,衣逢掖之衣,长居宋,冠章甫之冠。丘闻之也:君子之学也博,其服也乡;丘不知儒服。''

哀公曰:``敢问儒行。''孔子对曰:``遽数之不能终其物,悉数之乃留,更仆未可终也。''

哀公命席。孔子侍曰:``儒有席上之珍以待聘,夙夜强学以待问,怀忠信以待举,力行以待取,其自立有如此者。

儒有衣冠中,动作慎,其大让如慢,小让如伪,大则如威,小则如愧,其难进而易退也,粥粥若无能也。其容貌有如此者。

儒有居处齐难,其坐起恭敬,言必先信,行必中正,道涂不争险易之利,冬夏不争阴阳之和,爱其死以有待也,养其身以有为也。其备豫有如此者。

儒有不宝金玉,而忠信以为宝;不祈土地,立义以为土地;不祈多积,多文以为富。难得而易禄也,易禄而难畜也,非时不见,不亦难得乎?非义不合,不亦难畜乎?先劳而后禄,不亦易禄乎?其近人有如此者。

儒有委之以货财,淹之以乐好,见利不亏其义;劫之以众,沮之以兵,见死不更其守;鸷虫攫搏不程勇者,引重鼎不程其力;往者不悔,来者不豫;过言不再,流言不极;不断其威,不习其谋。其特立有如此者。

``儒有可亲而不可劫也;可近而不可迫也;可杀而不可辱也。其居处不淫,其饮食不溽;其过失可微辨而不可面数也。其刚毅有如此者。

儒有忠信以为甲胄,礼义以为干橹;戴仁而行,抱义而处,虽有暴政,不更其所。其自立有如此者。

儒有一亩之宫,环堵之室,筚门圭窬,蓬户瓮牖;易衣而出,并日而食,上答之不敢以疑,上不答不敢以谄。其仕有如此者。

``儒有今人与居,古人与稽;今世行之,后世以为楷;适弗逢世,上弗援,下弗推,谗谄之民有比党而危之者,身可危也,而志不可夺也,虽危起居,竟信其志,犹将不忘百姓之病也。其忧思有如此者。

儒有博学而不穷,笃行而不倦;幽居而不淫,上通而不困;礼之以和为贵,忠信之美,优游之法,举贤而容众,毁方而瓦合。其宽裕有如此者。

``儒有内称不辟亲,外举不辟怨,程功积事,推贤而进达之,不望其报;君得其志,茍利国家,不求富贵。其举贤援能有如此者。

儒有闻善以相告也,见善以相示也;爵位相先也,患难相死也;久相待也,远相致也。其任举有如此者。

儒有澡身而浴德,陈言而伏,静而正之,上弗知也;粗而翘之,又不急为也;不临深而为高,不加少而为多;世治不轻,世乱不沮;同弗与,异弗非也。其特立独行有如此者。

``儒有上不臣天子,下不事诸侯;慎静而尚宽,强毅以与人,博学以知服;近文章砥厉廉隅;虽分国如锱铢,不臣不仕。其规为有如此者。

儒有合志同方,营道同术;并立则乐,相下不厌;久不相见,闻流言不信;其行本方立义,同而进,不同而退。其交友有如此者。

温良者,仁之本也;敬慎者,仁之地也;宽裕者,仁之作也;孙接者,仁之能也;礼节者,仁之貌也;言谈者,仁之文也;歌乐者,仁之和也;分散者,仁之施也;儒皆兼此而有之,犹且不敢言仁也。其尊让有如此者。

儒有不陨获于贫贱,不充诎于富贵,不慁君王,不累长上,不闵有司,故曰儒。今众人之命儒也妄,常以儒相诟病。''

孔子至舍,哀公馆之,闻此言也,言加信,行加义:``终没吾世,不敢以儒为戏。''

\hypertarget{header-n866}{%
\subsection{大学}\label{header-n866}}

大学之道,在明明德,在亲民,在止于至善。

知止而后有定,定而后能静,静而后能安,安而后能虑,虑而后能得。

物有本末,事有终始,知所先后,则近道矣。

古之欲明明德于天下者,先治其国;欲治其国者,先齐其家;欲齐其家者,先修其身;欲修其身者,先正其心;欲正其心者,先诚其意;欲诚其意者,先致其知,致知在格物。

物格而后知至,知至而后意诚,意诚而后心正,心正而后身修,身修而后家齐,家齐而后国治,国治而后天下平。

自天子以至于庶人,壹是皆以修身为本。

其本乱而末治者否矣,其所厚者薄,而其所薄者厚,未之有也!

此谓知本,此谓知之至也。

所谓诚其意者,毋自欺也,如恶恶臭,如好好色,此之谓自谦,故君子必慎其独也!小人闲居为不善,无所不至,见君子而后厌然,掩其不善,而着其善。人之视己,如见其肺肝然,则何益矣!此谓诚于中,形于外,故君子必慎其独也。曾子曰:``十目所视,十手所指,其严乎!''富润屋,德润身,心广体胖,故君子必诚其意。

《诗》云:``瞻彼淇澳,菉竹猗猗。有斐君子,如切如磋,如琢如磨。瑟兮僴兮,赫兮喧兮。有斐君子,终不可諠兮!''``如切如磋''者,道学也;``如琢如磨''者,自修也;``瑟兮僴兮''者,恂栗也;``赫兮喧兮''者,威仪也;``有斐君子,终不可諠兮''者,道盛德至善,民之不能忘也。

《诗》云:``于戏前王不忘!''君子贤其贤而亲其亲,小人乐其乐而利其利,此以没世不忘也。

《康诰》曰:``克明德。''《太甲》曰:``顾諟天之明命。''《帝典》曰:``克明峻德。''皆自明也。

汤之盘铭曰:``茍日新,日日新,又日新。''《康诰》曰:``作新民。''《诗》曰:``周虽旧邦,其命惟新。''是故君子无所不用其极。

《诗》云:``邦畿千里,惟民所止。''《诗》云:``缗蛮黄鸟,止于丘隅。''子曰:``于止,知其所止,可以人而不如鸟乎?''《诗》云:``穆穆文王,于缉熙敬止!''为人君,止于仁;为人臣,止于敬;为人子,止于孝;为人父,止于慈;与国人交,止于信。

子曰:``听讼,吾犹人也,必也使无讼乎!''无情者不得尽其辞,大畏民志。此谓知本。

所谓修身在正其心者:身有所忿懥,则不得其正;有所恐惧,则不得其正;有所好乐,则不得其正;有所忧患,则不得其正。心不在焉,视而不见,听而不闻,食而不知其味。此谓修身在正其心。

所谓齐其家在修其身者:人之其所亲爱而辟焉,之其所贱恶而辟焉,之其所畏敬而辟焉,之其所哀矜而辟焉,之其所敖惰而辟焉。故好而知其恶,恶而知其美者,天下鲜矣!故谚有之曰:``人莫知其子之恶,莫知其苗之硕。''此谓身不修不可以齐其家。

所谓治国必先齐其家者,其家不可教而能教人者,无之。故君子不出家而成教于国:孝者,所以事君也;弟者,所以事长也;慈者,所以使众也。《康诰》曰:``如保赤子'',心诚求之,虽不中不远矣。未有学养子而后嫁者也!一家仁,一国兴仁;一家让,一国兴让;一人贪戾,一国作乱。其机如此。此谓一言偾事,一人定国。尧、舜率天下以仁,而民从之;桀、纣率天下以暴,而民从之。其所令反其所好,而民不从。是故君子有诸己而后求诸人,无诸己而后非诸人。所藏乎身不恕,而能喻诸人者,未之有也。故治国在齐其家。《诗》云:``桃之夭夭,其叶蓁蓁;之子于归,宜其家人。''宜其家人,而后可以教国人。《诗》云:``宜兄宜弟。''宜兄宜弟,而后可以教国人。《诗》云:``其仪不忒,正是四国。''其为父子兄弟足法,而后民法之也。此谓治国在齐其家。

所谓平天下在治其国者:上老老而民兴孝,上长长而民兴弟,上恤孤而民不倍,是以君子有絜矩之道也。

所恶于上,毋以使下;所恶于下,毋以事上;所恶于前,毋以先后;所恶于后,毋以从前;所恶于右,毋以交于左;所恶于左,毋以交于右。此之谓絜矩之道。

《诗》云:``乐只君子,民之父母。''民之所好好之,民之所恶恶之,此之谓民之父母。《诗》云:``节彼南山,维石岩岩。赫赫师尹,民具尔瞻。''有国者不可以不慎,辟则为天下戮矣。《诗》云:``殷之未丧师,克配上帝。仪监于殷,峻命不易。''道得众则得国,失众则失国。

是故君子先慎乎德。有德此有人,有人此有土,有土此有财,有财此有用。德者本也,财者末也,外本内末,争民施夺。是故财聚则民散,财散则民聚。是故言悖而出者,亦悖而入;货悖而入者,亦悖而出。

《康诰》曰:``惟命不于常!''道善则得之,不善则失之矣。

楚书曰:``楚国无以为宝,惟善以为宝。''舅犯曰:``亡人无以为宝,仁亲以为宝。''

《秦誓》曰:``若有一介臣,断断兮无他技,其心休休焉,其如有容焉。人之有技,若己有之;人之彦圣,其心好之,不啻若自其口出。实能容之,以能保我子孙黎民,尚亦有利哉!人之有技,媢嫉以恶之;人之彦圣,而违之俾不通。实不能容,以不能保我子孙黎民,亦曰殆哉!''唯仁人放流之,迸诸四夷,不与同中国,此谓唯仁人为能爱人,能恶人。见贤而不能举,举而不能先,命也;见不善而不能退,退而不能远,过也。好人之所恶,恶人之所好,是谓拂人之性,灾必逮夫身。是故君子有大道,必忠信以得之,骄泰以失之。

生财有大道。生之者众,食之者寡,为之者疾,用之者舒,则财恒足矣。仁者以财发身,不仁者以身发财。未有上好仁而下不好义者也,未有好义其事不终者也,未有府库财非其财者也。孟献子曰:``畜马乘,不察于鸡豚;伐冰之家,不畜牛羊;百乘之家,不畜聚敛之臣。与其有聚敛之臣,宁有盗臣。''此谓国不以利为利,以义为利也。长国家而务财用者,必自小人矣。彼为善之,小人之使为国家,灾害并至。虽有善者,亦无如之何矣!此谓国不以利为利,以义为利也。

\hypertarget{header-n895}{%
\subsection{冠义}\label{header-n895}}

凡人之所以为人者,礼义也。礼义之始,在于正容体、齐颜色、顺辞令。容体正,颜色齐,辞令顺,而后礼义备。以正君臣、亲父子、和长幼。君臣正,父子亲,长幼和,而后礼义立。故冠而后服备,服备而后容体正、颜色齐、辞令顺。故曰:冠者,礼之始也。是故古者圣王重冠。

古者冠礼筮日筮宾,所以敬冠事,敬冠事所以重礼;重礼所以为国本也。

故冠于阼,以着代也;醮于客位,三加弥尊,加有成也;已冠而字之,成人之道也。见于母,母拜之;见于兄弟,兄弟拜之;成人而与为礼也。玄冠、玄端奠挚于君,遂以挚见于乡大夫、乡先生;以成人见也。

成人之者,将责成人礼焉也。责成人礼焉者,将责为人子、为人弟、为人臣、为人少者之礼行焉。将责四者之行于人,其礼可不重与?

故孝弟忠顺之行立,而后可以为人;可以为人,而后可以治人也。故圣王重礼。故曰:冠者,礼之始也,嘉事之重者也。是故古者重冠;重冠故行之于庙;行之于庙者,所以尊重事;尊重事而不敢擅重事;不敢擅重事,所以自卑而尊先祖也。

\hypertarget{header-n903}{%
\subsection{昏义}\label{header-n903}}

昏礼者,将合二姓之好,上以事宗庙,而下以继后世也。故君子重之。是以昏礼纳采、问名、纳吉、纳征、请期,皆主人筵几于庙,而拜迎于门外,入,揖让而升,听命于庙,所以敬慎重正昏礼也。

父亲醮子,而命之迎,男先于女也。子承命以迎,主人筵几于庙,而拜迎于门外。婿执雁入,揖让升堂,再拜奠雁,盖亲受之于父母也。降,出御妇车,而婿授绥,御轮三周。先俟于门外,妇至,婿揖妇以入,共牢而食,合卺而酳,所以合体同尊卑以亲之也。

敬慎重正而后亲之,礼之大体,而所以成男女之别,而立夫妇之义也。男女有别,而后夫妇有义;夫妇有义,而后父子有亲;父子有亲,而后君臣有正。故曰:昏礼者,礼之本也。

夫礼始于冠,本于昏,重于丧祭,尊于朝聘,和于射乡-\/-此礼之大体也。

夙兴,妇沐浴以俟见;质明,赞见妇于舅姑,执笲、枣、栗、段修以见,赞醴妇,妇祭脯醢,祭醴,成妇礼也。舅姑入室,妇以特豚馈,明妇顺也。厥明,舅姑共飨妇以一献之礼,奠酬。舅姑先降自西阶,妇降自阼阶,以着代也。

成妇礼,明妇顺,又申之以着代,所以重责妇顺焉也。妇顺者,顺于舅姑,和于室人;而后当于夫,以成丝麻布帛之事,以审守委积盖藏。是故妇顺备而后内和理;内和理而后家可长久也;故圣王重之。

是以古者妇人先嫁三月,祖祢未毁,教于公宫,祖祢既毁,教于宗室,教以妇德、妇言、妇容、妇功。教成祭之,牲用鱼,芼之以苹藻,所以成妇顺也。

古者天子后立六宫、三夫人、九嫔、二十七世妇、八十一御妻,以听天下之内治,以明章妇顺;故天下内和而家理。天子立六官、三公、九卿、二十七大夫、八十一元士,以听天下之外治,以明章天下之男教;故外和而国治。故曰:天子听男教,后听女顺;天子理阳道,后治阴德;天子听外治,后听内职。教顺成俗,外内和顺,国家理治,此之谓盛德。

是故男教不修,阳事不得,适见于天,日为之食;妇顺不修,阴事不得,适见于天,月为之食。是故日食则天子素服而修六官之职,荡天下之阳事;月食则后素服而修六宫之职,荡天下之阴事。故天子与后,犹日之与月、阴之与阳,相须而后成者也。天子修男教,父道也;后修女顺,母道也。故曰:天子之与后,犹父之与母也。故为天王服斩衰,服父之义也;为后服资衰,服母之义也。

\hypertarget{header-n915}{%
\subsection{乡饮酒义}\label{header-n915}}

乡饮酒之义:主人拜迎宾于庠门之外,入,三揖而后至阶,三让而后升,所以致尊让也。盥洗扬觯,所以致洁也。拜至,拜洗,拜受,拜送,拜既,所以致敬也。尊让洁敬也者,君子之所以相接也。君子尊让则不争,洁敬则不慢,不慢不争,则远于斗辨矣;不斗辨则无暴乱之祸矣,斯君子之所以免于人祸也,故圣人制之以道。

乡人、士、君子,尊于房户之间,宾主共之也。尊有玄酒,贵其质也。羞出自东房,主人共之也。洗当东荣,主人之所以自洁,而以事宾也。

宾主象天地也;介僎象阴阳也;三宾象三光也;让之三也,象月之三日而成魄也;四面之坐,象四时也。

天地严凝之气,始于西南,而盛于西北,此天地之尊严气也,此天地之义气也。天地温厚之气,始于东北,而盛于东南,此天地之盛德气也,此天地之仁气也。主人者尊宾,故坐宾于西北,而坐介于西南以辅宾,宾者接人以义者也,故坐于西北。主人者,接人以德厚者也,故坐于东南。而坐僎于东北,以辅主人也。仁义接,宾主有事,俎豆有数曰圣,圣立而将之以敬曰礼,礼以体长幼曰德。德也者,得于身也。故曰:古之学术道者,将以得身也。是故圣人务焉。

祭荐,祭酒,敬礼也。哜肺,尝礼也。啐酒,成礼也。于席末,言是席之正,非专为饮食也,为行礼也,此所以贵礼而贱财也。卒觯,致实于西阶上,言是席之上,非专为饮食也,此先礼而后财之义也。先礼而后财,则民作敬让而不争矣。

乡饮酒之礼:六十者坐,五十者立侍,以听政役,所以明尊长也。六十者三豆,七十者四豆,八十者五豆,九十者六豆,所以明养老也。民知尊长养老,而后乃能入孝弟。民入孝弟,出尊长养老,而后成教,成教而后国可安也。君子之所谓孝者,非家至而日见之也;合诸乡射,教之乡饮酒之礼,而孝弟之行立矣。

孔子曰:``吾观于乡,而知王道之易易也。''

主人亲速宾及介,而众宾自从之。至于门外,主人拜宾及介,而众宾自入;贵贱之义别矣。三揖至于阶,三让以宾升,拜至、献、酬、辞让之节繁。及介省矣。至于众宾升受,坐祭,立饮。不酢而降;隆杀之义别矣。

主人酬介工入,升歌三终,主人献之;笙入三终,主人献之;间歌三终,合乐三终,工告乐备,遂出。一人扬觯,乃立司正焉,知其能和乐而不流也。

宾酬主人,主人酬介,介酬众宾,少长以齿,终于沃洗者焉。知其能弟长而无遗矣。

降,说屦升坐,修爵无数。饮酒之节,朝不废朝,莫不废夕。宾出,主人拜送,节文终遂焉。知其能安燕而不乱也。

贵贱明,隆杀辨,和乐而不流,弟长而无遗,安燕而不乱,此五行者,足以正身安国矣。彼国安而天下安。故曰:``吾观于乡,而知王道之易易也。''

乡饮酒之义:立宾以象天,立主以象地,设介僎以象日月,立三宾以象三光。古之制礼也,经之以天地,纪之以日月,参之以三光,政教之本也。

亨狗于东方,祖阳气之发于东方也。洗之在阼,其水在洗东,祖天地之左海也。尊有玄酒,教民不忘本也。

宾必南乡。东方者春,春之为言蠢也,产万物者圣也。南方者夏,夏之为言假也,养之、长之、假之,仁也。西方者秋,秋之为言愁也,愁之以时察,守义者也。北方者冬,冬之言中也,中者藏也。是以天子之立也,左圣乡仁,右义偝藏也。介必东乡,介宾主也。主人必居东方,东方者春,春之为言蠢也,产万物者也;主人者造之,产万物者也。月者三日则成魄,三月则成时,是以礼有三让,建国必立三卿。三宾者,政教之本,礼之大参也。

\hypertarget{header-n933}{%
\subsection{射义}\label{header-n933}}

古者诸侯之射也,必先行燕礼;卿、大夫、士之射也,必先行乡饮酒之礼。故燕礼者,所以明君臣之义也;乡饮酒之礼者,所以明长幼之序也。

故射者,进退周还必中礼,内志正,外体直,然后持弓矢审固;持弓矢审固,然后可以言中,此可以观德行矣。

其节:天子以《驺虞》为节;诸侯以《狸首》为节;卿大夫以《采苹》为节;士以《采繁》为节。《驺虞》者,乐官备也,《狸首》者,乐会时也;《采苹》者,乐循法也;《采繁》者,乐不失职也。是故天子以备官为节;诸侯以时会天子为节;卿大夫以循法为节;士以不失职为节。故明乎其节之志,以不失其事,则功成而德行立,德行立则无暴乱之祸矣。功成则国安。故曰:射者,所以观盛德也。

是故古者天子以射选诸侯、卿、大夫、士。射者,男子之事也,因而饰之以礼乐也。故事之尽礼乐,而可数为,以立德行者,莫若射,故圣王务焉。

是故古者天子之制,诸侯岁献贡士于天子,天子试之于射宫。其容体比于礼,其节比于乐,而中多者,得与于祭。其容体不比于礼,其节不比于乐,而中少者,不得与于祭。数与于祭而君有庆;数不与于祭而君有让。数有庆而益地;数有让而削地。故曰:射者,射为诸侯也。是以诸侯君臣尽志于射,以习礼乐。夫君臣习礼乐而以流亡者,未之有也。

故《诗》曰:``曾孙侯氏,四正具举;大夫君子,凡以庶士,小大莫处,御于君所,以燕以射,则燕则誉。''言君臣相与尽志于射,以习礼乐,则安则誉也。是以天子制之,而诸侯务焉。此天子之所以养诸侯,而兵不用,诸侯自为正之具也。

孔子射于矍相之圃,盖观者如堵墙。射至于司马,使子路执弓矢,出延射曰:``贲军之将,亡国之大夫,与为人后者不入,其余皆入。''盖去者半,入者半。又使公罔之裘、序点,扬觯而语,公罔之裘扬觯而语曰:``幼壮孝弟,耆耋好礼,不从流俗,修身以俟死者,不,在此位也。''盖去者半,处者半。序点又扬觯而语曰:``好学不倦,好礼不变,旄期称道不乱者,不,在此位也。''盖仅有存者。

射之为言者绎也,或曰舍也。绎者,各绎己之志也。故心平体正,持弓矢审固;持弓矢审固,则射中矣。故曰:为人父者,以为父鹄;为人子者,以为子鹄;为人君者,以为君鹄;为人臣者,以为臣鹄。故射者各射己之鹄。故天子之大射谓之射侯;射侯者,射为诸侯也。射中则得为诸侯;射不中则不得为诸侯。

天子将祭,必先习射于泽。泽者,所以择士也。已射于泽,而后射于射宫。射中者得与于祭;不中者不得与于祭。不得与于祭者有让,削以地;得与于祭者有庆,益以地。进爵绌地是也。

故男子生,桑弧蓬矢六,以射天地四方。天地四方者,男子之所有事也。故必先有志于其所有事,然后敢用谷也。饭食之谓也。

射者,仁之道也。射求正诸己,己正然后发,发而不中,则不怨胜己者,反求诸己而已矣。孔子曰:``君子无所争,必也射乎!揖让而升,下而饮,其争也君子。''

孔子曰:``射者何以射?何以听?循声而发,发而不失正鹄者,其唯贤者乎!若夫不肖之人,则彼将安能以中?''《诗》云:``发彼有的,以祈尔爵。''祈,求也;求中以辞爵也。酒者,所以养老也,所以养病也;求中以辞爵者,辞养也。

\hypertarget{header-n948}{%
\subsection{燕义}\label{header-n948}}

古者周天子之官,有庶子官。庶子官职诸侯、卿、大夫、士之庶子之卒,掌其戒令,与其教治,别其等,正其位。国有大事,则率国子而致于大子,唯所用之。若有甲兵之事,则授之以车甲,合其卒伍,置其有司,以军法治之,司马弗正。凡国之政事,国子存游卒,使之修德学道,春合诸学,秋合诸射,以考其艺而进退之。

诸侯燕礼之义:君立阼阶之东南,南乡尔卿,大夫皆少进,定位也;君席阼阶之上,居主位也;君独升立席上,西面特立,莫敢适之义也。设宾主,

饮酒之礼也;使宰夫为献主,臣莫敢与君亢礼也;不以公卿为宾,而以大夫为宾,为疑也,明嫌之义也;宾入中庭,君降一等而揖之,礼之也。

君举旅于宾,及君所赐爵,皆降再拜稽首,升成拜,明臣礼也;君答拜之,礼无不答,明君上之礼也。臣下竭力尽能以立功于国,君必报之以爵禄,故臣下皆务竭力尽能以立功,是以国安而君宁。礼无不答,言上之不虚取于下也。上必明正道以道民,民道之而有功,然后取其什一,故上用足而下不匮也;是以上下和亲而不相怨也。和宁,礼之用也;此君臣上下之大义也。故曰:燕礼者,所以明君臣之义也。

席:小卿次上卿,大夫次小卿,士、庶子以次就位于下。献君,君举旅行酬;而后献卿,卿举旅行酬;而后献大夫,大夫举旅行酬;而后献士,士举旅行酬;而后献庶子。俎豆、牲体、荐羞,皆有等差,所以明贵贱也。

\hypertarget{header-n956}{%
\subsection{聘义}\label{header-n956}}

聘礼,上公七介,侯、伯五介,子、男三介,所以明贵贱也。介绍而传命,君子于其所尊弗敢质,敬之至也。三让而后传命,三让而后入庙门,三揖而后至阶,三让而后升,所以致尊让也。

君使士迎于竟,大夫郊劳,君亲拜迎于大门之内而庙受,北面拜贶,拜君命之辱,所以致敬也。敬让也者,君子之所以相接也。故诸侯相接以敬让,则不相侵陵。

卿为上摈,大夫为承摈,士为绍摈。君亲礼宾,宾私面、私觌、致饔饩、还圭璋、贿赠、飨食燕,所以明宾客君臣之义也。

故天子制诸侯,比年小聘,三年大聘,相厉以礼。使者聘而误,主君弗亲飨食也。所以愧厉之也。诸侯相厉以礼,则外不相侵,内不相陵。此天子之所以养诸侯,兵不用而诸侯自为正之具也。

以圭璋聘,重礼也;已聘而还圭璋,此轻财而重礼之义也。诸侯相厉以轻财重礼,则民作让矣。主国待客,出入三积,饩客于舍,五牢之具陈于内,米三十车,禾三十车,刍薪倍禾,皆陈于外,乘禽日五双,群介皆有饩牢,壹食再飨,燕与时赐无数,所以厚重礼也。古之用财者不能均如此,然而用财如此其厚者,言尽之于礼也。尽之于礼,则内君臣不相陵,而外不相侵。故天子制之,而诸侯务焉尔。

聘射之礼,至大礼也。质明而始行事,日几中而后礼成,非强有力者弗能行也。故强有力者,将以行礼也。酒清,人渴而不敢饮也;肉干,人饥而不敢食也;日莫人倦,齐庄正齐,而不敢解惰。以成礼节,以正君臣,以亲父子,以和长幼。此众人之所难,而君子行之,故谓之有行;有行之谓有义,有义之谓勇敢。故所贵于勇敢者,贵其能以立义也;所贵于立义者,贵其有行也;所贵于有行者,贵其行礼也。故所贵于勇敢者,贵其敢行礼义也。故勇敢强有力者,天下无事,则用之于礼义;天下有事,则用之于战胜。用之于战胜则无敌,用之于礼义则顺治;外无敌,内顺治,此之谓盛德。故圣王之贵勇敢强有力如此也。勇敢强有力而不用之于礼义战胜,而用之于争斗,则谓之乱人。刑罚行于国,所诛者乱人也。如此则民顺治而国安也。

子贡问于孔子曰:``敢问君子贵玉而贱玟者何也?为玉之寡而玟之多与?''孔子曰:``非为玟之多故贱之也、玉之寡故贵之也。夫昔者君子比德于玉焉:温润而泽,仁也;缜密以栗,知也;廉而不刿,义也;垂之如队,礼也;叩之其声清越以长,其终诎然,乐也;瑕不掩瑜、瑜不掩瑕,忠也;孚尹旁达,信也;气如白虹,天也;精神见于山川,地也;圭璋特达,德也。天下莫不贵者,道也。《诗》云:『言念君子,温其如玉。』故君子贵之也。''

\hypertarget{header-n966}{%
\subsection{丧服四制}\label{header-n966}}

凡礼之大体,体天地,法四时,则阴阳,顺人情,故谓之礼。訾之者,是不知礼之所由生也。夫礼,吉凶异道,不得相干,取之阴阳也。丧有四制,变而从宜,取之四时也。有恩有理,有节有权,取之人情也。恩者仁也,理者义也,节者礼也,权者知也。仁义礼智,人道具矣。

其恩厚者,其服重;故为父斩衰三年,以恩制者也。门内之治,恩掩义;门外之治,义断恩。资于事父以事君,而敬同,贵贵尊尊,义之大者也。故为君亦斩衰三年,以义制者也。

三日而食,三月而沐,期而练,毁不灭性,不以死伤生也。丧不过三年,苴衰不补,坟墓不培;祥之日,鼓素琴,告民有终也;以节制者也。资于事父以事母,而爱同。天无二日,土无二王,国无二君,家无二尊,以一治之也。故父在,为母齐衰期者,见无二尊也。

杖者何也?爵也。三日授子杖,五日授大夫杖,七日授士杖。或曰担主;或曰辅病,妇人、童子不杖,不能病也。百官备,百物具,不言而事行者,扶而起;言而后事行者,杖而起;身自执事而后行者,面垢而已。秃者不髽,伛者不袒,跛者不踊。老病不止酒肉。凡此八者,以权制者也。

始死,三日不怠,三月不解,期悲哀,三年忧-\/-恩之杀也。圣人因杀以制节,此丧之所以三年。贤者不得过,不肖者不得不及,此丧之中庸也,王者之所常行也。《书》曰:``高宗谅闇,三年不言'',善之也;王者莫不行此礼。何以独善之也?曰:高宗者武丁;武丁者,殷之贤王也。继世即位而慈良于丧,当此之时,殷衰而复兴,礼废而复起,故善之。善之,故载之书中而高之,故谓之高宗。三年之丧,君不言,《书》云:``高宗谅闇,三年不言'',此之谓也。然而曰``言不文''者,谓臣下也。

礼:斩衰之丧,唯而不对;齐衰之丧,对而不言;大功之丧,言而不议;缌小功之丧,议而不及乐。

父母之丧,衰冠绳缨菅屦,三日而食粥,三月而沐,期十三月而练冠,三年而祥。比终兹三节者,仁者可以观其爱焉,知者可以观其理焉,强者可以观其志焉。礼以治之,义以正之,孝子弟弟贞妇,皆可得而察焉。

\end{document}
