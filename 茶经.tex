\PassOptionsToPackage{unicode=true}{hyperref} % options for packages loaded elsewhere
\PassOptionsToPackage{hyphens}{url}
%
\documentclass[]{article}
\usepackage{lmodern}
\usepackage{amssymb,amsmath}
\usepackage{ifxetex,ifluatex}
\usepackage{fixltx2e} % provides \textsubscript
\ifnum 0\ifxetex 1\fi\ifluatex 1\fi=0 % if pdftex
  \usepackage[T1]{fontenc}
  \usepackage[utf8]{inputenc}
  \usepackage{textcomp} % provides euro and other symbols
\else % if luatex or xelatex
  \usepackage{unicode-math}
  \defaultfontfeatures{Ligatures=TeX,Scale=MatchLowercase}
\fi
% use upquote if available, for straight quotes in verbatim environments
\IfFileExists{upquote.sty}{\usepackage{upquote}}{}
% use microtype if available
\IfFileExists{microtype.sty}{%
\usepackage[]{microtype}
\UseMicrotypeSet[protrusion]{basicmath} % disable protrusion for tt fonts
}{}
\IfFileExists{parskip.sty}{%
\usepackage{parskip}
}{% else
\setlength{\parindent}{0pt}
\setlength{\parskip}{6pt plus 2pt minus 1pt}
}
\usepackage{hyperref}
\hypersetup{
            pdfborder={0 0 0},
            breaklinks=true}
\urlstyle{same}  % don't use monospace font for urls
\setlength{\emergencystretch}{3em}  % prevent overfull lines
\providecommand{\tightlist}{%
  \setlength{\itemsep}{0pt}\setlength{\parskip}{0pt}}
\setcounter{secnumdepth}{0}
% Redefines (sub)paragraphs to behave more like sections
\ifx\paragraph\undefined\else
\let\oldparagraph\paragraph
\renewcommand{\paragraph}[1]{\oldparagraph{#1}\mbox{}}
\fi
\ifx\subparagraph\undefined\else
\let\oldsubparagraph\subparagraph
\renewcommand{\subparagraph}[1]{\oldsubparagraph{#1}\mbox{}}
\fi

% set default figure placement to htbp
\makeatletter
\def\fps@figure{htbp}
\makeatother


\date{}

\begin{document}

\hypertarget{header-n16}{%
\section{茶经}\label{header-n16}}

\begin{center}\rule{0.5\linewidth}{\linethickness}\end{center}

\tableofcontents

\begin{center}\rule{0.5\linewidth}{\linethickness}\end{center}

\hypertarget{header-n22}{%
\subsection{一之源}\label{header-n22}}

茶者,南方之嘉木也,一尺二尺,乃至数十尺。其巴山峡川有两人合抱者,伐而掇之,其树如瓜芦,叶如栀子,花如白蔷薇,实如栟榈,蒂如丁香,根如胡桃。

其字或从草,或从木,或草木并。其名一曰茶,二曰槚,三曰蔎,四曰茗,五曰荈。

其地,上者生烂石,中者生砾壤,下者生黄土。

凡艺而不实,植而罕茂,法如种瓜,三岁可采。野者上,园者次;阳崖阴林,紫者上,绿者次;笋者上,牙者次;叶卷上,叶舒次。阴山坡谷者,不堪采掇,性凝滞,结瘕疾。

茶之为用,味至寒,为饮最宜精行俭德之人。若热渴、凝闷、脑疼、目涩、四肢乏、百节不舒,聊四五啜,与醍醐、甘露抗衡也。

采不时,造不精,杂以卉莽,饮之成疾。

茶为累也,亦犹人参。上者生上党,中者生百济、新罗,下者生高丽。有生泽州、易州、幽州、檀州者,为药无效,况非此者,设服荠苨,使六疾不瘳。知人参为累,则茶累尽矣。

\hypertarget{header-n23}{%
\subsection{二之具}\label{header-n23}}

籝,一曰篮,一曰笼,一曰筥。以竹织之,受五升,或一斗、二斗、三斗者,茶人负以采茶也。

灶无用突者,釜用唇口者。甑,或木或瓦,匪腰而泥,篮以箪之,篾以系之。始其蒸也,入乎箪,既其熟也,出乎箪。釜涸注于甑中,又以谷木枝三亚者制之,散所蒸牙笋并叶,畏流其膏。

杵臼,一曰碓,惟恒用者佳。

规,一曰模,一曰棬。以铁制之,或圆或方或花。

承,一曰台,一曰砧。以石为之,不然以槐、桑木半埋地中,遣无所摇动。

檐,一曰衣。以油绢或雨衫单服败者为之,以檐置承上,又以规置檐上,以造茶也。茶成,举而易之。

芘莉,一曰羸子,一曰篣筤。以二小竹长三赤,躯二赤五寸,柄五寸,以篾织,方眼如圃,人土罗阔二赤,以列茶也。

棨,一曰锥刀,柄以坚木为之,用穿茶也。

扑,一曰鞭。以竹为之,穿茶以解茶也。

焙,凿地深二尺,阔二尺五寸,长一丈,上作短墙,高二尺,泥之。

贯,削竹为之,长二尺五寸,以贯茶焙之。

棚,一曰栈,以木构于焙上,编木两层,高一尺,以焙茶也。茶之半干升下棚,全干升上棚。

穿,江东淮南剖竹为之,巴川峡山纫谷皮为之。江东以一斤为上穿,半斤为中穿,四两五两为小穿。峡中以一百二十斤为上,八十斤为中穿,五十斤为小穿。字旧作钗钏之``钏'',字或作贯串,今则不然。如磨、扇、弹、钻、缝五字,文以平声书之,义以去声呼之,其字以穿名之。

育,以木制之,以竹编之,以纸糊之,中有隔,上有覆,下有床,傍有门,掩一扇,中置一器,贮煻煨火,令煴煴然,江南梅雨时焚之以火。

\hypertarget{header-n25}{%
\subsection{四之器}\label{header-n25}}

风炉(灰承) 筥 炭挝 火筴 鍑 交床 夹纸囊 碾拂末 罗 合 则 水方 漉水囊 瓢
竹筴 鹾簋揭 碗 熟 盂 畚 札 涤方 滓方 巾 具列 都篮

风炉{[}灰承{]}

风炉:以铜、铁铸之,如古鼎形。厚三分,缘阔九分,令六分虚中,致其污墁。凡三足,古文书二十一字:一足云:``坎上巽下离于中'';一足云:``体均五行去百疾'';一足云:``圣唐灭胡明年铸。''其三足之间,设三窗,底一窗以为通飙漏烬之所。上并古文书六字:一窗之上书``伊公''二字;一窗之上书``羹陆''二字;一窗之上书``氏茶''二字,所谓``伊公羹、陆氏茶''也。置滞(土旁){[}土臬{]},于其内设三格:其一格有翟焉,翟者,火禽也,画一卦曰离;其一格有彪焉,彪者,风兽也,画一卦曰巽;其一格有鱼焉,鱼者,水虫也,画一卦曰坎。巽主风,离主火,坎主水,风能兴火,火能熟水,故备其三卦焉。其饰,以连葩、垂蔓、曲水、方文之类。其炉,或锻铁为之,或运泥为之.其灰承,作三足铁{[}木半{]}抬之。

筥:以竹织之,高一尺二寸,径阔七寸。或用藤,作木楦如筥形织之。六出圆眼。其底盖若莉箧口①,铄之。

炭挝:以铁六棱制之。长一尺,锐上丰中。执细头,系一小{[}钅展{]},以饰挝也。若今之河陇军人木吾也。或作槌,或作斧,随其便也。

火筴:一名箸,若常用者,圆直一尺三寸。顶平截,无葱薹句鏁之属。以铁或熟铜制之。

鍑(音辅,或作釜,或作鬴):以生铁为之。今人有业冶者,所谓急铁,其铁以耕刀之趄炼而铸之。内抹土而外抹沙。土滑于内,易其摩涤;沙涩于外,吸其炎焰。方其耳,以令正也。广其缘,以务远也。长其脐,以守中也。脐长,则沸中;沸中,末易扬,则其味淳也。洪州以瓷为之,莱州以石为之。瓷与石皆雅器也,性非坚实,难可持久。用银为之,至洁,但涉于侈丽。稚则雅矣,洁亦洁矣,若用之恒,而卒归于铁也。

交床:以十字交之,剜中令虚,以支鍑也。

夹:以小青竹为之,长一尺二寸。令一寸有节,节以上剖之,以炙茶也。彼竹之筱,津润于火,假其香洁以益茶味。恐非林谷间莫之致。或用精铁、熟铜之类,取其久也。

纸囊:以剡藤纸白厚者夹缝之,以贮所炙茶,使不泄其香也。

碾:以桔木为之,次以梨,桑、桐、柘为之。内圆而外方。内圆,备于运行也;外方,制其倾危也。内容堕而外无余木。堕,形如车轮,不辐而轴焉。长九寸,阔一寸七分。堕径三寸八分,中厚一寸,边厚半寸。轴中方而执圆。其拂未,以鸟羽制之。

罗、合:罗末,以合贮之,以则置合中。用巨竹剖而屈之,以纱绢衣之。其合,以竹节为之,或屈杉以漆之。高三寸,盖一寸,底二才,口径四寸。

则:以海贝、蜗蛤之属,或以铜、铁,竹匕、策之类。则者,量也,准也,度也。凡煮水一升,用末方寸匕'',若好薄者减之,故云则也。

水方:以稠榜木(原注,音胄,木名也。]槐、楸、梓等合之,其里井外缝漆之。受一斗。

漉水囊:若常用者。其格,以生铜铸之,以备水湿无有苔秽、腥涩之意;以熟铜、苔秽;铁,腥涩也。林栖谷隐者,或用之竹木。木与竹非持久涉远之具,故用之生铜,其囊,织青竹以卷之,裁碧缣以缝之,细翠钿以缀之,又作油绿囊以贮之。圆径五寸,柄一寸五分。

瓢:一曰牺、杓,剖瓠为之,或刊木为之。晋舍人杜毓《荈赋》云:``酌之以瓠''。瓠,瓢也,口阔,胚薄,柄短。永嘉中,余姚人虞洪入瀑布山采茗,遇一道士云:``吾,丹丘子,祈子他日瓯牺之余,乞相遗也。''牺,木杓也。今常用以梨木为之。

竹筴:或以桃、柳、蒲葵木为之,或以柿心木为之。长一尺,银裹两头。

鹾簋:以瓷为之,圆径四寸,若合形。或瓶、或缶。贮盐花也。其揭,竹制,长四寸一分,阔九分。揭,策也。

熟盂:以贮熟水。或瓷、或砂。受二升。

碗:越州上,鼎州、婺州次;丘州上,寿州、洪州次。或者以邢州处越州上,殊为不然。若邢瓷类银,越瓷类玉,邢不如越一也;若邢瓷类雪,则越瓷类冰,邢不如越二也;邢瓷白而茶色丹,越瓷青而茶色绿,邢不如越三也。晋杜琉《荈赋》所谓:``器择陶拣,出自东瓯''。瓯,越州也,瓯越上。口唇不卷,底卷而浅,受半升以下。越州瓷、丘瓷皆青,青则益茶,茶作红白之色。邢州瓷白,茶色红;寿州瓷黄,茶色紫;洪州瓷褐,茶色黑;悉不宜茶。

畚:以白蒲卷而编之,可贮碗十枚,或用筥。其纸帊以剡纸夹缝令方,亦十之也。

札:缉栟榈皮,以茱萸莫木夹而缚之,或截竹束而管之,若巨笔形。

涤方:以贮洗涤之余。水方,受八升。

滓方:以集诸滓,制如涤方,处五升。

巾:以拖縍布为之。长二尺,作二枚,互用之,以洁诸器。

具列:或作床,或作架。或纯木、纯竹而制之;或木或竹\ldots{}\ldots{},黄黑可扃而漆者。长三尺,阔二尺,高六寸。具列者,悉敛诸器物,悉以陈列也。

都篮:以悉设诸器而名之,以竹蔑,内作三角方眼,外以双蔑阔者经之,以单蔑纤者缚之,递压双经,作方眼,使玲成。高一尺五寸,底阔一尺,高二寸,长二尺四寸,阔二尺。

\hypertarget{header-n26}{%
\subsection{五之煮}\label{header-n26}}

凡炙茶,慎勿于风烬间炙,熛焰如钻,使凉炎不均。特以逼火,屡其翻正,候炮出培塿状蟆背,然后去火五寸。卷而舒,则本其始,又炙之。若火干者,以气熟止;日干者,以柔止。

其始,若茶之至嫩者,蒸罢热捣,叶烂而芽笋存焉。假以力者,持千钧杵亦不之烂,如漆科珠,壮士接之,不能驻其指。及就,则似无穰骨也。炙之,则其节若倪倪如婴儿之臂耳。既而,承热用纸囊贮之,精华之气无所散越,候寒末之。{[}原注:末之上者,其屑如细米;末之下者,其屑如菱角。{]}

其火,用炭,次用劲薪。{[}原注:谓桑、槐、桐、枥之类也。{]}其炭曾经燔炙为膻腻所及,及膏木、败器,不用之。{[}原注:膏木,谓柏、松、桧也。败器,谓朽废器也。{]}古人有劳薪之味,信哉!

其水,用山水上,江水中,井水下。{[}原注:《荈赋》所谓``水则岷方之注,挹彼清流。''{]}其山水拣乳泉、石池漫流者上;其瀑涌湍漱,勿食之。久食,令人有颈疾。又水流于山谷者,澄浸不泄,自火天至霜郊以前,或潜龙蓄毒于其间,饮者可决之,以流其恶,使新泉涓涓然,酌之。其江水,取去人远者。井,取汲多者。

其沸,如鱼目,微有声,为一沸;缘边如涌泉连珠,为二沸;腾波鼓浪,为三沸,已上,水老,不可食也。初沸,则水合量,调之以盐味,谓弃其啜余,{[}原注:啜,尝也,市税反,又市悦反。{]}无乃\href{}{卤舀}而钟其一味乎,{[}原注:{[}卤舀{]},古暂反。{[}卤监{]},吐滥反。无味也。{]}第二沸,出水一瓢,以竹环激汤心,则量末当中心而下。有顷,势若奔涛溅沫,以所出水止之,而育其华也。

凡酌至诸碗,令沫饽均。{[}原注:字书并《本草》:``沫、饽,均茗沫也。''饽蒲笏反。{]}沫饽,汤之华也。华之薄者曰沫,厚者曰饽,轻细者曰花,花,如枣花漂漂然于环池之上;又如回潭曲渚青萍之始生;又如晴天爽朗,有浮云鳞然。其沫者,若绿钱浮于水湄;又如菊英堕于樽俎之中。饽者,以滓煮之,及沸,则重华累沫,皤皤然若积雪耳。《荈赋》所谓``焕如积雪,烨若春{[}莆方攵{]},有之。

第一煮沸水,弃其上有水膜如黑云母,饮之则其味不正。其第一者为隽永,{[}原注:徐县、全县二反。至美者曰隽永。隽,味也。永,长也。史长曰隽永,《汉书》蒯通著《隽永》二十篇也。{]}或留熟盂以贮之,以备育华救沸之用,诸第一与第二、第三碗次之,第四、第五碗外,非渴甚莫之饮。凡煮水一升,酌分五碗,{[}原注:碗数少至三,多至五;若人多至十,加两炉。{]}乘热连饮之。以重浊凝其下,精英浮其上。如冷,则精英随气而竭,饮啜不消亦然矣。

茶性俭,不宜广,广则其味黯澹。且如一满碗,啜半而味寡,况其广乎!其色缃也,其馨{[}上必下土右欠{]}
也,{[}原注:香至美曰{[}上必下土右欠{]}。{[}上必下土右欠{]}
,音备。{]}其味甘,槚 也;不甘而苦,荈也;啜苦咽甘,茶也。

\hypertarget{header-n27}{%
\subsection{六之饮}\label{header-n27}}

翼而飞,毛而走,呿而言,此三者俱生于天地间,饮啄以活,饮之时义远矣哉!至若救渴,饮之以浆;蠲忧忿,饮之以酒;荡昏寐,饮之以茶。

茶之为饮,发乎神农氏,闻于鲁周公,齐有晏婴,汉有杨雄、司马相如,吴有韦曜,晋有刘琨、张载、远祖纳、谢安、左思之徒,皆饮焉。滂时浸俗,盛于国朝,两都并荆俞{[}原注:俞,当作渝。巴渝也{]}间,以为比屋之饮。

饮有粗茶、散茶、末茶、饼茶者。乃斫、乃熬、乃炀、乃舂,贮于瓶缶之中,以汤沃焉,谓之痷茶。或用葱、姜、枣、桔皮、茱萸、薄荷之等,煮之百沸,或扬令滑,或煮去沫,斯沟渠间弃水耳,而习俗不已。

于戏!天育有万物,皆有至妙,人之所工,但猎浅易。所庇者屋,屋精极;所著者衣,衣精极;所饱者饮食,食与酒皆精极之;{[}译者注:此处有脱文{]}茶有九难:一曰造,二曰别,三曰器,四曰火,五曰水,六曰炙,七曰末,八曰煮,九曰饮。阴采夜焙,非造也。嚼味嗅香,非别也。膻鼎腥瓯,非器也。膏薪庖炭,非火也。飞湍壅潦,非水也。非炙也。碧粉缥尘,非末也。操艰搅遽,非煮也。夏兴冬废,非饮也。

夫珍鲜馥烈者,其碗数三;次之者,碗数五。若座客数至五,行三碗;至七,行五碗;若六人以下,不约碗数,但阙一人而已,其隽永补所阙人。

\hypertarget{header-n28}{%
\subsection{七之事}\label{header-n28}}

三皇炎帝。神农氏。周鲁周公旦。齐相晏婴。汉仙人丹丘子。黄山君司马文。园令相如。杨执戟雄。吴归命侯。韦太傅弘嗣。晋惠帝。刘司空琨。琨兄子兖州刺史演。张黄门孟阳。傅司隶咸。江洗马充。孙参军楚。左记室太冲。陆吴兴纳。纳兄子会稽内史俶。谢冠军安石。郭弘农璞。桓扬州温。杜舍人毓。武康小山寺释法瑶。沛国夏侯恺。馀姚虞洪。北地傅巽。丹阳弘君举。安任育。宣城秦精。敦煌单道开。剡县陈务妻。广陵老姥。河内山谦之。后魏琅琊王肃。宋新安王子鸾。鸾弟豫章王子尚。鲍昭妹令晖。八公山沙门谭济。齐世祖武帝。梁·刘廷尉。陶先生弘景。皇朝徐英公绩。

《神农·食经》:``茶茗久服,令人有力、悦志''。

周公《尔雅》:``槚,苦茶。''《广雅》云:``荆巴间采叶作饼,叶老者饼成,以米膏出之,欲煮茗饮,先灸,令赤色,捣末置瓷器中,以汤浇覆之,用葱、姜、橘子芼之,其饮醒酒,令人不眠。''

《晏子春秋》:``婴相齐景公时,食脱粟之饭,灸三戈五卯茗莱而已。''

司马相如《凡将篇》:``乌啄桔梗芫华,款冬贝母木蘖蒌,芩草芍药桂漏芦,蜚廉雚菌荈诧,白敛白芷菖蒲,芒消莞椒茱萸。''

《方言》:``蜀西南人谓茶曰葭。''

《吴志·韦曜传》:``孙皓每飨宴坐席,无不率以七胜为限。虽不尽入口,皆浇灌取尽,曜饮酒不过二升,皓初礼异,密赐茶荈以代酒。''

《晋中兴书》:``陆纳为吴兴太守,时卫将军谢安常欲诣纳,纳兄子俶怪纳,无所备,不敢问之,乃私蓄十数人馔。安既至,所设唯茶果而已。俶遂陈盛馔珍羞必具,及安去,纳杖俶四十,云:`汝既不能光益叔父,柰何秽吾素业?'''

《晋书》:``桓温为扬州牧,性俭,每燕饮,唯下七奠,拌茶果而已。''

《搜神记》:``夏侯恺因疾死,宗人字苟奴,察见鬼神,见恺来收马,并病其妻,着平上帻单衣入,坐生时西壁大床,就人觅茶饮。''

刘琨《与兄子南兖州刺史演书》云:``前得安州干姜一斤、桂一斤、黄芩一斤,皆所须也,吾体中溃闷,常仰真茶,汝可置之。''

傅咸《司隶教》曰:``闻南方有以困蜀妪作茶粥卖,为帘事打破其器具。又卖饼于市,而禁茶粥以蜀姥何哉!''

《神异记》:``馀姚人虞洪入山采茗,遇一道士牵三青牛,引洪至瀑布山曰:`予丹丘子也。闻子善具饮,常思见惠。山中有大茗可以相给,祈子他日有瓯牺之余,乞相遗也。'因立奠祀。后常令家人入山,获大茗焉。''

左思《娇女诗》:``吾家有娇女,皎皎颇白皙。小字为纨素,口齿自清历。有姊字惠芳,眉目粲如画。驰骛翔园林,果下皆生摘。贪华风雨中,倏忽数百适。心为茶荈剧,吹嘘对鼎䥶。''

张孟阳《登成都楼诗》云:``借问杨子舍,想见长卿庐。程卓累千金,骄侈拟五侯。门有连骑客,翠带腰吴钩。鼎食随时进,百和妙且殊。披林采秋橘,临江钓春鱼。黑子过龙醢,果馔逾蟹蝑。芳茶冠六情,溢味播九区。人生苟安乐,兹土聊可娱。''

傅巽《七诲》:``蒲桃、宛柰、齐柿、燕栗、峘阳黄梨、巫山朱橘、南中茶子、西极石蜜。''

弘君举食檄:寒温既毕,应下霜华之茗,三爵而终,应下诸蔗、木瓜、元李、杨梅、五味橄榄、悬豹、葵羹各一杯。孙楚歌:`茱萸出芳树颠,鲤鱼出洛水泉,白盐出河东,美豉出鲁渊。姜桂茶荈出巴蜀,椒橘、木兰出高山,蓼苏出沟渠,精稗出中田。'''

华佗《食论》:``苦茶久食益意思。''

壶居士《食忌》:``苦茶久食羽化。与韭同食,令人体重。''郭璞《尔雅注》云:``树小似栀子,冬生叶,可煮羹饮,今呼早取为茶,晚取为茗,或一曰荈,蜀人名之苦茶。''

《世说》:``任瞻字育长,少时有令名。自过江失志,既下饮,问人云:`此为茶为茗?'觉人有怪色,乃自分明云:`向问饮为热为冷?'''

《续搜神记·晋武帝》:``宣城人秦精,常入武昌山采茗,遇一毛人长丈余,引精至山下,示以丛茗而去。俄而复还,乃探怀中橘以遗精,精怖,负茗而归。''

晋四王起事,惠帝蒙尘,还洛阳,黄门以瓦盂盛茶上至尊。

《异苑》:``剡县陈务妻少,与二子寡居,好饮茶茗。以宅中有古冢,每饮,辄先祀之。二子患之曰:`古冢何知?徒以劳。'意欲掘去之,母苦禁而止。其夜梦一人云:吾止此冢三百余年,卿二子恒欲见毁,赖相保护,又享吾佳茗,虽潜壤朽骨,岂忘翳桑之报。及晓,于庭中获钱十万,似久埋者,但贯新耳。母告,二子惭之,从是祷馈愈甚。''

《广陵耆老传》:``晋元帝时有老姥,每旦独提一器茗,往市鬻之,市人竞买,自旦至夕,其器不减,所得钱散路傍孤贫乞人。人或异之,州法曹絷之狱中,至夜,老姥执所鬻茗器,从狱牖中飞出。''

《艺术传》:``敦煌人单道开不畏寒暑,常服小石子。所服药有松桂蜜之气,所余茶苏而已。''释道该说《续名僧传》:``宋释法瑶姓杨氏,河东人,永嘉中过江遇沈台真,请真君武康小山寺,年垂悬车,饭所饮茶,永明中敕吴兴礼致上京,年七十九。''

《宋江氏家传》:``江统字应迁,愍怀太子洗马,常上疏谏云:`今西园卖酰面蓝子菜茶之属,亏败国体。'''

《宋录》:``新安王子鸾、豫章王子尚,诣昙济道人于八公山,道人设茶茗,子尚味之曰:此甘露也,何言茶茗。''

王微《杂诗》:``寂寂掩高阁,寥寥空广厦。待君竟不归,收领今就槚。

鲍昭妹令晖着《香茗赋》。

南齐世祖武皇帝遗诏:``我灵座上,慎勿以牲为祭,但设饼果、茶饮、干饭、酒脯而已。''

梁刘孝绰、谢晋安王饷米等,启传诏:李孟孙宣教旨,垂赐米、酒、瓜、笋、菹、脯、酢、茗八种,气苾新城,味芳云松。江潭抽节,迈昌荇之珍;疆场擢翘,越葺精之美。羞非纯束野麏,裛似雪之驴;鲊异陶瓶河鲤,操如琼之粲。茗同食粲酢,颜望楫免,千里宿舂,省三月种聚。小人怀惠,大懿难忘。陶弘景《杂录》:``苦茶轻换膏,昔丹丘子青山君服之。''

《后魏录》:``琅琊王肃仕南朝,好茗饮莼羹。及还北地,又好羊肉酪浆,人或问之:茗何如酪?肃曰:茗不堪与酪为奴。''

《桐君录》:``西阳武昌庐江昔陵好茗,皆东人作清茗。茗有饽,饮之宜人。凡可饮之物,皆多取其叶,天门冬、拔揳取根,皆益人。又巴东别有真茗茶,煎饮令人不眠。俗中多煮檀叶,并大皂李作茶,并冷。又南方有瓜芦木,亦似茗,至苦涩,取为屑茶,饮亦可通夜不眠。煮盐人但资此饮,而交广最重,客来先设,乃加以香芼辈。《坤元录》:``辰州溆浦县西北三百五十里无射山,云蛮俗当吉庆之时,亲族集会,歌舞于山上,山多茶树。''

《括地图》:``临遂县东一百四十里有茶溪。''

山谦之《吴兴记》:``乌程县西二十里有温山,出御荈。《夷陵图经》:``黄牛、荆门、女观望州等山,茶茗出焉。''

《永嘉图经》:``永嘉县东三百里有白茶山。''

《淮阴图经》:``山阳县南二十里有茶坡。''

《茶陵图经》云:``茶陵者,所谓陵谷,生茶茗焉。''《本草·木部》:``茗,苦茶,味甘苦,微寒,无毒,主瘘疮,利小便,去痰渴热,令人少睡。秋采之苦,主下气消食。注云:春采之。''

《本草·菜部》:``苦茶,一名荼,一名选,一名游冬。生益州川谷山陵道傍,凌冬不死。三月三日采干。注云:疑此即是今茶,一名荼,令人不眠。本草注。''按《诗》云``谁谓荼苦'',又云``堇荼如饴'',皆苦菜也。陶谓之苦茶,木类,非菜流。茗,春采谓之苦?茶。

《枕中方》:``疗积年瘘,苦茶、蜈蚣并灸,令香熟,等分捣筛,煮甘草汤洗,以末傅之。''

《孺子方》:``疗小儿无故惊蹶,以葱须煮服之。''

\hypertarget{header-n29}{%
\subsection{八之出}\label{header-n29}}

山南以峡州上,襄州、荆州次,衡州下,金州、梁州又下。

淮南以光州上,义阳郡、舒州次,寿州下,蕲州、黄州又下。

浙西以湖州上,常州次,宣州、杭州、睦州、歙州下,润州、苏州又下。

剑南以彭州上,绵州、蜀州次,邛州次,雅州、泸州下,眉州、汉州又下。

浙东以越州上,明州、婺州次,台州下。

黔中生恩州、播州、费州、夷州,江南生鄂州、袁州、吉州,岭南生福州、建州、韶州、象州。其恩、播、费、夷、鄂、袁、吉、福、建、泉、韶、象十一州未详。往往得之,其味极佳。

\hypertarget{header-n30}{%
\subsection{九之略}\label{header-n30}}

其造具,若方春禁火之时,于野寺山园丛手而掇,乃蒸,乃舂,乃以火干之,则又棨、朴、焙、贯、相、穿、育等七事皆废。其煮器,若松间石上可坐,则具列,废用槁薪鼎枥之属,则风炉、灰承、炭挝、火筴、交床等废;若瞰泉临涧,则水方、涤方、漉水囊废。若五人已下,茶可末而精者,则罗废;若援藟跻嵒,引絙入洞,于山口灸而末之,或纸包合贮,则碾、拂末等废;既瓢碗、筴、札、熟盂、醝簋悉以一筥盛之,则都篮废。但城邑之中,王公之门,二十四器阙一则茶废矣!

\hypertarget{header-n31}{%
\subsection{十之图}\label{header-n31}}

以绢素或四幅或六幅,分布写之,陈诸座隅,则茶之源、之具、之造、之器、之煮、之饮、之事、之出、之略,目击而存,于是《茶经》之始终备焉。

\end{document}
