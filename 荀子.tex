\PassOptionsToPackage{unicode=true}{hyperref} % options for packages loaded elsewhere
\PassOptionsToPackage{hyphens}{url}
%
\documentclass[]{article}
\usepackage{lmodern}
\usepackage{amssymb,amsmath}
\usepackage{ifxetex,ifluatex}
\usepackage{fixltx2e} % provides \textsubscript
\ifnum 0\ifxetex 1\fi\ifluatex 1\fi=0 % if pdftex
  \usepackage[T1]{fontenc}
  \usepackage[utf8]{inputenc}
  \usepackage{textcomp} % provides euro and other symbols
\else % if luatex or xelatex
  \usepackage{unicode-math}
  \defaultfontfeatures{Ligatures=TeX,Scale=MatchLowercase}
\fi
% use upquote if available, for straight quotes in verbatim environments
\IfFileExists{upquote.sty}{\usepackage{upquote}}{}
% use microtype if available
\IfFileExists{microtype.sty}{%
\usepackage[]{microtype}
\UseMicrotypeSet[protrusion]{basicmath} % disable protrusion for tt fonts
}{}
\IfFileExists{parskip.sty}{%
\usepackage{parskip}
}{% else
\setlength{\parindent}{0pt}
\setlength{\parskip}{6pt plus 2pt minus 1pt}
}
\usepackage{hyperref}
\hypersetup{
            pdfborder={0 0 0},
            breaklinks=true}
\urlstyle{same}  % don't use monospace font for urls
\setlength{\emergencystretch}{3em}  % prevent overfull lines
\providecommand{\tightlist}{%
  \setlength{\itemsep}{0pt}\setlength{\parskip}{0pt}}
\setcounter{secnumdepth}{0}
% Redefines (sub)paragraphs to behave more like sections
\ifx\paragraph\undefined\else
\let\oldparagraph\paragraph
\renewcommand{\paragraph}[1]{\oldparagraph{#1}\mbox{}}
\fi
\ifx\subparagraph\undefined\else
\let\oldsubparagraph\subparagraph
\renewcommand{\subparagraph}[1]{\oldsubparagraph{#1}\mbox{}}
\fi

% set default figure placement to htbp
\makeatletter
\def\fps@figure{htbp}
\makeatother


\date{}

\begin{document}

\hypertarget{header-n0}{%
\section{荀子}\label{header-n0}}

\begin{center}\rule{0.5\linewidth}{\linethickness}\end{center}

\tableofcontents

\begin{center}\rule{0.5\linewidth}{\linethickness}\end{center}

\hypertarget{header-n10}{%
\subsection{劝学}\label{header-n10}}

君子曰:学不可以已。青、取之于蓝,而青于蓝;冰、水为之,而寒于水。木直中绳,輮以为轮,其曲中规,虽有槁暴,不复挺者,輮使之然也。故木受绳则直,金就砺则利,君子博学而日参省乎己,则知明而行无过矣。故不登高山,不知天之高也;不临深溪,不知地之厚也;不闻先王之遗言,不知学问之大也。干、越、夷、貉之子,生而同声,长而异俗,教使之然也。诗曰:``嗟尔君子,无恒安息。靖共尔位,好是正直。神之听之,介尔景福。''神莫大于化道,福莫长于无祸。

吾尝终日而思矣,不如须臾之所学也。吾尝跂而望矣,不如登高之博见也。登高而招,臂非加长也,而见者远;顺风而呼,声非加疾也,而闻者彰。假舆马者,非利足也,而致千里;假舟楫者,非能水也,而绝江河。君子生非异也,善假于物也。

南方有鸟焉,名曰蒙鸠,以羽为巢,而编之以发,系之苇苕,风至苕折,卵破子死。巢非不完也,所系者然也。西方有木焉,名曰射干,茎长四寸,生于高山之上,而临百仞之渊,木茎非能长也,所立者然也。蓬生麻中,不扶而直;白沙在涅,与之俱黑。兰槐之根是为芷,其渐之滫,君子不近,庶人不服。其质非不美也,所渐者然也。故君子居必择乡,游必就士,所以防邪辟而近中正也。

物类之起,必有所始。荣辱之来,必象其德。肉腐出虫,鱼枯生蠹。怠慢忘身,祸灾乃作。强自取柱,柔自取束。邪秽在身,怨之所构。施薪若一,火就燥也,平地若一,水就湿也。草木畴生,禽兽群焉,物各从其类也。是故质的张,而弓矢至焉;林木茂,而斧斤至焉;树成荫,而众鸟息焉。醯酸,而蚋聚焉。故言有招祸也,行有招辱也,君子慎其所立乎!

积土成山,风雨兴焉;积水成渊,蛟龙生焉;积善成德,而神明自得,圣心备焉。故不积蹞步,无以致千里;不积小流,无以成江海。骐骥一跃,不能十步;驽马十驾,功在不舍。锲而舍之,朽木不折;锲而不舍,金石可镂。螾无爪牙之利,筋骨之强,上食埃土,下饮黄泉,用心一也。蟹六跪而二螯,非蛇蟺之穴,无可寄托者,用心躁也。是故无冥冥之志者,无昭昭之明;无惛惛之事者,无赫赫之功。行衢道者不至,事两君者不容。目不能两视而明,耳不能两听而聪。螣蛇无足而飞,梧鼠五技而穷。诗曰:``尸鸠在桑,其子七兮。淑人君子,其仪一兮。其仪一兮,心如结兮。''故君子结于一也。

昔者瓠巴鼓瑟,而流鱼出听;伯牙鼓琴,而六马仰秣。故声无小而不闻,行无隐而不形。玉在山而草木润,渊生珠而崖不枯。为善不积邪,安有不闻者乎!

学恶乎始?恶乎终?曰:其数则始乎诵经,终乎读礼;其义则始乎为士,终乎为圣人。真积力久则入。学至乎没而后止也。故学数有终,若其义则不可须臾舍也。为之人也,舍之禽兽也。故书者、政事之纪也;诗者、中声之所止也;礼者、法之大兮,类之纲纪也。故学至乎礼而止矣。夫是之谓道德之极。礼之敬文也,乐之中和也,诗书之博也,春秋之微也,在天地之间者毕矣。

君子之学也,入乎耳,着乎心,布乎四体,形乎动静。端而言,蝡而动,一可以为法则。小人之学也,入乎耳,出乎口;口耳之间,则四寸耳,曷足以美七尺之躯哉!古之学者为己,今之学者为人。君子之学也,以美其身;小人之学也,以为禽犊。故不问而告谓之傲,问一而告二谓之囋。傲、非也,囋、非也;君子如向矣。

学莫便乎近其人。礼乐法而不说,诗书故而不切,春秋约而不速。方其人之习君子之说,则尊以遍矣,周于世矣。故曰:学莫便乎近其人。

学之经莫速乎好其人,隆礼次之。上不能好其人,下不能隆礼,安特将学杂识志,顺诗书而已耳。则末世穷年,不免为陋儒而已。将原先王,本仁义,则礼正其经纬蹊径也。若挈裘领,诎五指而顿之,顺者不可胜数也。不道礼宪,以诗书为之,譬之犹以指测河也,以戈舂黍也,以锥餐壶也,不可以得之矣。故隆礼,虽未明,法士也;不隆礼,虽察辩,散儒也。

问楛者,勿告也;告楛者,勿问也;说楛者,勿听也。有争气者,勿与辩也。故必由其道至,然后接之;非其道则避之。故礼恭,而后可与言道之方;辞顺,而后可与言道之理;色从而后可与言道之致。故未可与言而言,谓之傲;可与言而不言,谓之隐;不观气色而言,谓瞽。故君子不傲、不隐、不瞽,谨顺其身。诗曰:``匪交匪舒,天子所予。''此之谓也。

百发失一,不足谓善射;千里蹞步不至,不足谓善御;伦类不通,仁义不一,不足谓善学。学也者,固学一之也。一出焉,一入焉,涂巷之人也;其善者少,不善者多,桀纣盗跖也;全之尽之,然后学者也。

君子知夫不全不粹之不足以为美也,故诵数以贯之,思索以通之,为其人以处之,除其害者以持养之。使目非是无欲见也,使口非是无欲言也,使心非是无欲虑也。及至其致好之也,目好之五色,耳好之五声,口好之五味,心利之有天下。是故权利不能倾也,群众不能移也,天下不能荡也。生乎由是,死乎由是,夫是之谓德操。德操然后能定,能定然后能应。能定能应,夫是之谓成人。天见其明,地见其光,君子贵其全也。

\hypertarget{header-n16}{%
\subsection{修身}\label{header-n16}}

见善,修然必以自存也;见不善,愀然必以自省也。善在身,介然必以自好也;不善在身,菑然必以自恶也。故非我而当者,吾师也;是我而当者,吾友也;谄谀我者,吾贼也。故君子隆师而亲友,以致恶其贼。好善无厌,受谏而能诫,虽欲无进,得乎哉!小人反是:致乱而恶人之非己也;致不肖而欲人之贤己也;心如虎狼,行如禽兽,而又恶人之贼己也。谄谀者亲,谏争者疏,修正为笑,至忠为贼,虽欲无灭亡,得乎哉!诗曰:``嗡嗡呰呰,亦孔之哀。谋之其臧,则具是违;谋之不臧,则具是依。''此之谓也。

扁善之度以治气养生,则后彭祖;以修身自名,则配尧禹。宜于时通,利以处穷,礼信是也。凡用血气、志意、知虑,由礼则治通,不由礼则勃乱提僈;食饮,衣服、居处、动静,由礼则和节,不由礼则触陷生疾;容貌、态度、进退、趋行,由礼则雅,不由礼则夷固、僻违、庸众而野。故人无礼则不生,事无礼则不成,国家无礼则不宁。诗曰:``礼仪卒度,笑语卒获。''此之谓也。

以善先人者谓之教,以善和人者谓之顺;以不善先人者谓之谄,以不善和人者谓之谀。是是非非谓之知,非是是非谓之愚。伤良曰谗,害良曰贼。是谓是,非谓非曰直。窃货曰盗,匿行曰诈,易言曰诞。趣舍无定谓之无常。保利弃义谓之至贼。多闻曰博,少闻曰浅。多见曰闲,少见曰陋。难进曰偍,易忘曰漏。少而理曰治,多而乱曰秏。

治气养心之术:血气刚强,则柔之以调和;知虑渐深,则一之以易良;勇胆猛戾,则辅之以道顺;齐给便利,则节之以动止;狭隘褊小,则廓之以广大;卑湿重迟贪利,则抗之以高志;庸众驽散,则劫之以师友;怠慢僄弃,则照之以祸灾;愚款端悫,则合之以礼乐,通之以思索。凡治气养心之术,莫径由礼,莫要得师,莫神一好。夫是之谓治气养心之术也。

志意修则骄富贵,道义重则轻王公;内省而外物轻矣。传曰:``君子役物,小人役于物。''此之谓矣。身劳而心安,为之;利少而义多,为之;事乱君而通,不如事穷君而顺焉。故良农不为水旱不耕,良贾不为折阅不市,士君子不为贫穷怠乎道。

体恭敬而心忠信,术礼义而情爱人;横行天下,虽困四夷,人莫不贵。劳苦之事则争先,饶乐之事则能让,端悫诚信,拘守而详;横行天下,虽困四夷,人莫不任。体倨固而心埶诈,术顺墨而精杂污;横行天下,虽达四方,人莫不贱。劳苦之事则偷儒转脱,饶乐之事则佞兑而不曲,辟违而不悫,程役而不录:横行天下,虽达四方,人莫不弃。

行而供冀,非渍淖也;行而俯项,非击戾也;偶视而先俯,非恐惧也。然夫士欲独修其身,不以得罪于比俗之人也。

夫骥一日而千里,驽马十驾,则亦及之矣。将以穷无穷,逐无极与?其折骨绝筋,终身不可以相及也。将有所止之,则千里虽远,亦或迟、或速、或先、或后,胡为乎其不可以相及也!不识步道者,将以穷无穷,逐无极与?意亦有所止之与?夫``坚白''、``同异''、``有厚无厚''之察,非不察也,然而君子不辩,止之也。倚魁之行,非不难也,然而君子不行,止之也。故学曰迟。彼止而待我,我行而就之,则亦或迟、或速、或先、或后,胡为乎其不可以同至也!故蹞步而不休,跛鳖千里;累土而不辍,丘山崇成。厌其源,开其渎,江河可竭。一进一退,一左一右,六骥不致。彼人之才性之相县也,岂若跛鳖之与六骥足哉!然而跛鳖致之,六骥不致,是无它故焉,或为之,或不为尔!道虽迩,不行不至;事虽小,不为不成。其为人也多暇日者,其出入不远矣。

好法而行,士也;笃志而体,君子也;齐明而不竭,圣人也。人无法,则伥伥然;有法而无志其义,则渠渠然;依乎法,而又深其类,然后温温然。

礼者、所以正身也,师者、所以正礼也。无礼何以正身?无师吾安知礼之为是也?礼然而然,则是情安礼也;师云而云,则是知若师也。情安礼,知若师,则是圣人也。故非礼,是无法也;非师,是无师也。不是师法,而好自用,譬之是犹以盲辨色,以聋辨声也,舍乱妄无为也。故学也者,礼法也。夫师、以身为正仪,而贵自安者也。诗云:``不识不知,顺帝之则。''此之谓也。

端悫顺弟,则可谓善少者矣;加好学逊敏焉,则有钧无上,可以为君子者矣。偷儒惮事,无廉耻而嗜乎饮食,则可谓恶少者矣;加愓悍而不顺,险贼而不弟焉,则可谓不详少者矣,虽陷刑戮可也。老老而壮者归焉,不穷穷而通者积焉,行乎冥冥而施乎无报,而贤不肖一焉。人有此三行,虽有大过,天其不遂乎!

君子之求利也略,其远害也早,其避辱也惧,其行道理也勇。君子贫穷而志广,富贵而体恭,安燕而血气不惰,劳倦而容貌不枯,怒不过夺,喜不过予。君子贫穷而志广,隆仁也;富贵而体恭,杀埶也;安燕而血气不衰,柬理也;劳倦而容貌不枯,好交也;怒不过夺,喜不过予,是法胜私也。书曰:``无有作好,遵王之道。无有作恶,遵王之路。''此言君子之能以公义胜私欲也。

\hypertarget{header-n20}{%
\subsection{不苟}\label{header-n20}}

君子行不贵苟难,说不贵苟察,名不贵苟传,唯其当之为贵。故怀负石而投河,
是行之难为者也,而申徒狄能之;然而君子不贵者,非礼义之中也。``山渊平'',
``天地比'',``齐秦袭'',``入乎耳,出乎口'',``钩有须'',``卵有毛'',是说
之难持者也,而惠施邓析能之。然而君子不贵者,非礼义之中也。盗跖贪凶,名声
若日月,与舜禹俱传而不息;然而君子不贵者,非礼义之中也。故曰:君子行不贵
苟难,说不贵苟察,名不贵苟传,唯其当之为贵。诗曰:``物其有矣,惟其时矣。''
此之谓也。

君子易知而难狎,易惧而难胁,畏患而不避义死,欲利而不为所非,交亲而不
比,言辩而不辞,荡荡乎其有以殊于世也。

君子能亦好,不能亦好;小人能亦丑,不能亦丑。君子能则宽容易直以开道人,
不能则恭敬繜绌以畏事人;小人能则倨傲僻违以骄溢人,不能则妒嫉怨诽以倾覆人。
故曰:君子能则人荣学焉,不能则人乐告之;小人能则人贱学焉,不能则人羞告之。
是君子小人之分也。

君子宽而不僈,廉而不刿,辩而不争,察而不激,直立而不胜,坚强而不暴,
柔从而不流,恭敬谨慎而容。夫是之谓至文。诗曰:``温温恭人,惟德之基。''此
之谓也。

君子崇人之德,扬人之美,非谄谀也;正义直指,举人之过,非毁疵也;言己
之光美,拟于舜禹,参于天地,非夸诞也;与时屈伸,柔从若蒲苇,非慑怯也;刚
强猛毅,靡所不信,非骄暴也;以义变应,知当曲直故也。诗曰:``左之左之,君
子宜之;右之右之,君子有之。''此言君子以义屈信变应故也。

君子小人之反也:君子大心则敬天而道,小心则畏义而节;知则明通而类,愚
则端悫而法;见由则恭而止,见闭则敬而齐;喜则和而理,忧则静而理;通则文而
明,穷则约而详。小人则不然:大心则慢而暴,小心则淫而倾;知则攫盗而渐,愚
则毒贼而乱;见由则兑而倨,见闭则怨而险;喜则轻而翾,忧则挫而慑;通则骄而
偏,穷则弃而儑。传曰:``君子两进,小人两废。''此之谓也。

君子治治,非治乱也。曷谓邪?曰:礼义之谓治,非礼义之谓乱也。故君子者,
治礼义者也,非治非礼义者也。然则国乱将弗治与?曰:国乱而治之者,非案乱而
治之之谓也。去乱而被之以治。人污而修之者,非案污而修之之谓也,去污而易之
以修。故去乱而非治乱也,去污而非修污也。治之为名,犹曰君子为治而不为乱,
为修而不为污也。

君子絜其身而同焉者合矣,善其言而类焉者应矣。故马鸣而马应之,牛鸣而牛
应之,非知也,其埶然也。故新浴者振其衣,新沐者弹其冠,人之情也。其谁能以
己之潐潐,受人之掝掝者哉!

君子养心莫善于诚,致诚则无它事矣。惟仁之为守,惟义之为行。诚心守仁则
形,形则神,神则能化矣。诚心行义则理,理则明,明则能变矣。变化代兴,谓之
天德。天不言而人推其高焉,地不言而人推其厚焉,四时不言而百姓期焉。夫此有
常,以至其诚者也。君子至德,嘿然而喻,未施而亲,不怒而威:夫此顺命,以慎
其独者也。善之为道者,不诚则不独,不独则不形,不形则虽作于心,见于色,出
于言,民犹若未从也;虽从必疑。天地为大矣,不诚则不能化万物;圣人为知矣,
不诚则不能化万民;父子为亲矣,不诚则疏;君上为尊矣,不诚则卑。夫诚者,君
子之所守也,而政事之本也,唯所居以其类至。操之则得之,舍之则失之。操而得
之则轻,轻则独行,独行而不舍,则济矣。济而材尽,长迁而不反其初,则化矣。

君子位尊而志恭,心小而道大;所听视者近,而所闻见者远。是何邪?则操术
然也。故千人万人之情,一人之情也。天地始者,今日是也。百王之道,后王是也。
君子审后王之道,而论百王之前,若端拜而议。推礼义之统,分是非之分,总天下
之要,治海内之众,若使一人。故操弥约,而事弥大。五寸之矩,尽天下之方也。
故君子不下室堂,而海内之情举积此者,则操术然也。

有通士者,有公士者,有直士者,有悫士者,有小人者。上则能尊君,下则能
爱民,物至而应,事起而辨,若是则可谓通士矣。不下比以闇上,不上同以疾下,
分争于中,不以私害之,若是则可谓公士矣。身之所长,上虽不知,不以悖君;身
之所短,上虽不知,不以取赏;长短不饰,以情自竭,若是则可谓直士矣。庸言必
信之,庸行必慎之,畏法流俗,而不敢以其所独甚,若是则可谓悫士矣。言无常信,
行无常贞,唯利所在,无所不倾,若是则可谓小人矣。

公生明,偏生闇,端悫生通,诈伪生塞,诚信生神,夸诞生惑。此六生者,君
子慎之,而禹桀所以分也。

欲恶取舍之权:见其可欲也,则必前后虑其可恶也者;见其可利也,则必前后
虑其可害也者,而兼权之,孰计之,然后定其欲恶取舍。如是则常不失陷矣。凡人
之患,偏伤之也。见其可欲也,则不虑其可恶也者;见其可利也,则不虑其可害也
者。是以动则必陷,为则必辱,是偏伤之患也。

人之所恶者,吾亦恶之。夫富贵者,则类傲之;夫贫贱者,则求柔之。是非仁
人之情也,是奸人将以盗名于晻世者也,险莫大焉。故曰:盗名不如盗货。田仲史
不如盗也。

\hypertarget{header-n24}{%
\subsection{荣辱}\label{header-n24}}

憍泄者,人之殃也;恭俭者,偋五兵也。虽有戈矛之刺,不如恭俭之利也。故与人善言,暖于布帛;伤人之言,深于矛戟。故薄薄之地,不得履之,非地不安也,危足无所履者,凡在言也。巨涂则让,小涂则殆,虽欲不谨,若云不使。

快快而亡者、怒也,察察而残者、忮也,博而穷者、訾也,清之而俞浊者、口也,豢之而俞瘠者、交也,辩而不说者、争也,直立而不见知者、胜也,廉而不见贵者、刿也,勇而不见惮者、贪也,信而不见敬者、好剸行也。此小人之所务,而君子之所不为也。

斗者,忘其身者也,忘其亲者也,忘其君者也。行其少顷之怒,而丧终身之躯,然且为之,是忘其身也;家室立残,亲戚不免乎刑戮,然且为之,是忘其亲也;君上之所恶也,刑法之所大禁也,然且为之,是忘其君也。忧忘其身,内忘其亲,上忘其君,是刑法之所不舍也,圣王之所不畜也。乳彘触虎,乳狗不远游,不忘其亲也。人也,忧忘其身,内忘其亲,上忘其君,则是人也,而曾狗彘之不若也。

凡斗者,必自以为是,而以人为非也。己诚是也,人诚非也,则是己君子,而人小人也;以君子与小人相贼害也,忧以忘其身,内以忘其亲,上以忘其君,岂不过甚矣哉!是人也,所谓以狐父之戈钃牛矢也。将以为智邪?则愚莫大焉;将以为利邪?则害莫大焉;将以为荣邪?则辱莫大焉;将以为安邪?则危莫大焉。人之有斗,何哉?我欲属之狂惑疾病邪?则不可,圣王又诛之。我欲属之鸟鼠禽兽邪?则又不可,其形体又人,而好恶多同。人之有斗,何哉?我甚丑之。

有狗彘之勇者,有贾盗之勇者,有小人之勇者,有士君子之勇者。争饮食,无廉耻,不知是非,不辟死伤,不畏众强,牟牟然惟利饮食之见,是狗彘之勇也。为事利,争货财,无辞让,果敢而振,猛贪而戾,牟牟然惟利之见,是贾盗之勇也。轻死而暴,是小人之勇也。义之所在,不倾于权,不顾其利,举国而与之不为改视,重死持义而不桡,是士君子之勇也。

鯈魾者,浮阳之鱼也,胠于沙而思水,则无逮矣。挂于患而思谨,则无益矣。自知者不怨人,知命者不怨天;怨人者穷,怨天者无志。失之己,反之人,岂不迂乎哉!

荣辱之大分,安危利害之常体:先义而后利者荣,先利而后义者辱;荣者常通,辱者常穷;通者常制人,穷者常制于人:是荣辱之大分也。材悫者常安利,荡悍者常危害;安利者常乐易,危害者常忧险;乐易者常寿长,忧险者常夭折:是安危利害之常体也。

夫天生蒸民,有所以取之:志意致修,德行致厚,智虑致明,是天子之所以取天下也。政令法,举措时,听断公,上则能顺天子之命,下则能保百姓,是诸侯之所以取国家也。志行修,临官治,上则能顺上,下则能保其职,是士大夫之所以取田邑也。循法则、度量、刑辟、图籍、不知其义,谨守其数,慎不敢损益也;父子相传,以持王公,是故三代虽亡,治法犹存,是官人百吏之所以取禄职也。孝弟原悫,軥录疾力,以敦比其事业,而不敢怠傲,是庶人之所以取暖衣饱食,长生久视,以免于刑戮也。饰邪说,文奸言,为倚事,陶诞突盗,惕悍憍暴,以偷生反侧于乱世之间,是奸人之所以取危辱死刑也。其虑之不深,其择之不谨,其定取舍楛僈,是其所以危也。

材性知能,君子小人一也;好荣恶辱,好利恶害,是君子小人之所同也;若其所以求之之道则异矣:小人也者,疾为诞而欲人之信己也,疾为诈而欲人之亲己也,禽兽之行而欲人之善己也;虑之难知也,行之难安也,持之难立也,成则必不得其所好,必遇其所恶焉。故君子者,信矣,而亦欲人之信己也;忠矣,而亦欲人之亲己也;修正治辨矣,而亦欲人之善己也;虑之易知也,行之易安也,持之易立也,成则必得其所好,必不遇其所恶焉。是故穷则不隐,通则大明,身死而名弥白。小人莫不延颈举踵而愿曰:``知虑材性,固有以贤人矣。''夫不知其与己无以异也。则君子注错之当,而小人注错之过也。故孰察小人之知能,足以知其有余,可以为君子之所为也。譬之越人安越,楚人安楚,君子安雅。是非知能材性然也,是注错习俗之节异也。仁义德行,常安之术也,然而未必不危也;污僈突盗,常危之术也,然而未必不安也。故君子道其常,而小人道其怪。

凡人有所一同:饥而欲食,寒而欲暖,劳而欲息,好利而恶害,是人之所生而有也,是无待而然者也,是禹桀之所同也。目辨白黑美恶,耳辨声音清浊,口辨酸咸甘苦,鼻辨芬芳腥臊,骨体肤理辨寒暑疾养,是又人之所常生而有也,是无待而然者也,是禹桀之所同也。可以为尧禹,可以为桀跖,可以为工匠,可以为农贾,在埶注错习俗之所积耳。为尧禹则常安荣,为桀跖则常危辱;为尧禹则常愉佚,为工匠农贾则常烦劳;然而人力为此,而寡为彼,何也?曰:陋也。尧禹者,非生而具者也,夫起于变故,成乎修为,待尽而后备者也。人之生固小人,无师无法则唯利之见耳。人之生固小人,又以遇乱世,得乱俗,是以小重小也,以乱得乱也。君子非得埶以临之,则无由得开内焉。今是人之口腹,安知礼义?安知辞让?安知廉耻隅积?亦呥呥而嚼,乡乡而饱已矣。人无师无法,则其心正其口腹也。今使人生而未尝睹刍豢稻粱也,惟菽藿糟糠之为睹,则以至足为在此也,俄而粲然有秉刍豢稻梁而至者,则瞲然视之曰:此何怪也?彼臭之而嗛于鼻,尝之而甘于口,食之而安于体,则莫不弃此而取彼矣。今以夫先王之道,仁义之统,以相群居,以相持养,以相藩饰,以相安固邪。以夫桀跖之道,是其为相县也,几直夫刍豢稻梁之县糟糠尔哉!然而人力为此,而寡为彼,何也?曰:陋也。陋也者,天下之公患也,人之大殃大害也。故曰:仁者好告示人。告之、示之、靡之、儇之、鈆之、重之,则夫塞者俄且通也,陋者俄且僩也,愚者俄且知也。是若不行,则汤武在上曷益?桀纣在上曷损?汤武存,则天下从而治,桀纣存,则天下从而乱。如是者,岂非人之情,固可与如此,可与如彼也哉!

人之情,食欲有刍豢,衣欲有文绣,行欲有舆马,又欲夫余财蓄积之富也;然而穷年累世不知不足,是人之情也。今人之生也,方知畜鸡狗猪彘,又蓄牛羊,然而食不敢有酒肉;余刀布,有囷窌,然而衣不敢有丝帛;约者有筐箧之藏,然而行不敢有舆马。是何也?非不欲也,几不长虑顾后,而恐无以继之故也?于是又节用御欲,收歛蓄藏以继之也。是于己长虑顾后,几不甚善矣哉!今夫偷生浅知之属,曾此而不知也,粮食大侈,不顾其后,俄则屈安穷矣。是其所以不免于冻饿,操瓢囊为沟壑中瘠者也。况夫先王之道,仁义之统,诗书礼乐之分乎!彼固为天下之大虑也,将为天下生民之属,长虑顾后而保万世也。其流长矣,其温厚矣,其功盛姚远矣,非顺孰修为之君子,莫之能知也。故曰:短绠不可以汲深井之泉,知不几者不可与及圣人之言。夫诗书礼乐之分,固非庸人之所知也。故曰:一之而可再也,有之而可久也,广之而可通也,虑之而可安也,反鈆察之而俞可好也。以治情则利,以为名则荣,以群则和,以独则足,乐意者其是邪!

夫贵为天子,富有天下,是人情之所同欲也;然则从人之欲,则埶不能容,物不能赡也。故先王案为之制礼义以分之,使有贵贱之等,长幼之差,知愚能不能之分,皆使人载其事,而各得其宜。然后使谷禄多少厚薄之称,是夫群居和一之道也。故仁人在上,则农以力尽田,贾以察尽财,百工以巧尽械器,士大夫以上至于公侯,莫不以仁厚知能尽官职。夫是之谓至平。故或禄天下,而不自以为多,或监门御旅,抱关击柝,而不自以为寡。故曰:``斩而齐,枉而顺,不同而一。''夫是之谓人伦。诗曰:``受小共大共,为下国骏蒙。''此之谓也。

\hypertarget{header-n28}{%
\subsection{非相}\label{header-n28}}

相人,古之人无有也,学者不道也。古者有姑布子卿,今之世梁有唐举,相人之形状颜色,而知其吉凶妖祥,世俗称之。古之人无有也,学者不道也。故相形不如论心,论心不如择术;形不胜心,心不胜术;术正而心顺之,则形相虽恶而心术善,无害为君子也。形相虽善而心术恶,无害为小人也。君子之谓吉,小人之谓凶。故长短小大,善恶形相,非吉凶也。古之人无有也,学者不道也。

盖帝尧长,帝舜短;文王长,周公短;仲尼长,子弓短。昔者卫灵公有臣曰公孙吕,身长七尺,面长三尺,焉广三寸,鼻目耳具,而名动天下。楚之孙叔敖,期思之鄙人也,突秃长左,轩较之下,而以楚霸。叶公子高,微小短瘠,行若将不胜其衣然。白公之乱也,令尹子西,司马子期,皆死焉,叶公子高入据楚,诛白公,定楚国,如反手尔,仁义功名善于后世。故事不揣长,不揳大,不权轻重,亦将志乎尔。长短大小,美恶形相,岂论也哉!且徐偃王之状,目可瞻马。仲尼之状,面如蒙倛。周公之状,身如断菑。皋陶之状,色如削瓜。闳夭之状,面无见肤。傅说之状,身如植鳍。伊尹之状,面无须麋。禹跳汤偏。尧舜参牟子。从者将论志意,比类文学邪?直将差长短,辨美恶,而相欺傲邪?

古者桀纣长巨姣美,天下之杰也。筋力越劲,百人之敌也,然而身死国亡,为天下大僇,后世言恶,则必稽焉。是非容貌之患也,闻见之不众,议论之卑尔。今世俗之乱君,乡曲之儇子,莫不美丽姚冶,奇衣妇饰,血气态度拟于女子;妇人莫不愿得以为夫,处女莫不愿得以为士,弃其亲家而欲奔之者,比肩并起;然而中君羞以为臣,中父羞以为子,中兄羞以为弟,中人羞以为友;俄则束乎有司,而戮乎大市,莫不呼天啼哭,苦伤其今,而后悔其始,是非容貌之患也,闻见之不众,议论之卑尔!然则,从者将孰可也!

人有三不祥:幼而不肯事长,贱而不肯事贵,不肖而不肯事贤,是人之三不祥也。人有三必穷:为上则不能爱下,为下则好非其上,是人之一必穷也;乡则不若,偝则谩之,是人之二必穷也;知行浅薄,曲直有以相县矣,然而仁人不能推,知士不能明,是人之三必穷也。人有此三数行者,以为上则必危,为下则必灭。诗曰:``雨雪瀌瀌,宴然聿消,莫肯下隧,式居屡骄。''此之谓也。

人之所以为人者何已也?曰:以其有辨也。饥而欲食,寒而欲暖,劳而欲息,好利而恶害,是人之所生而有也,是无待而然者也,是禹桀之所同也。然则人之所以为人者,非特以二足而无毛也,以其有辨也。今夫狌狌形状亦二足而无毛也,然而君子啜其羹,食其胾。故人之所以为人者,非特以其二足而无毛也,以其有辨也。夫禽兽有父子,而无父子之亲,有牝牡而无男女之别。故人道莫不有辨。

辨莫大于分,分莫大于礼,礼莫大于圣王;圣王有百,吾孰法焉?曰:文久而灭,节族久而绝,守法数之有司,极礼而褫。故曰:欲观圣王之迹,则于其粲然者矣,后王是也。彼后王者,天下之君也;舍后王而道上古,譬之是犹舍己之君,而事人之君也。故曰:欲观千岁,则数今日;欲知亿万,则审一二;欲知上世,则审周道;欲审周道,则审其人所贵君子。故曰:以近知远,以一知万,以微知明,此之谓也。

夫妄人曰:``古今异情,其所以治乱者异道。''而众人惑焉。彼众人者,愚而无说,陋而无度者也。其所见焉,犹可欺也,而况于千世之传也?妄人者,门庭之间,犹可诬欺也,而况于千世之上乎?圣人何以不可欺?曰:圣人者,以己度者也。故以人度人,以情度情,以类度类,以说度功,以道观尽,古今一也。类不悖,虽久同理,故乡乎邪曲而不迷,观乎杂物而不惑,以此度之。五帝之外无传人,非无贤人也,久故也。五帝之中无传政,非无善政也,久故也。禹汤有传政而不若周之察也,非无善政也,久故也。传者久则论略,近则论详,略则举大,详则举小。愚者闻其略而不知其详,闻其详而不知其大也。是以文久而灭,节族久而绝。

凡言不合先王,不顺礼义,谓之奸言;虽辩,君子不听。法先王,顺礼义,党学者,然而不好言,不乐言,则必非诚士也。故君子之于言也,志好之,行安之,乐言之,故君子必辩。凡人莫不好言其所善,而君子为甚。故赠人以言,重于金石珠玉;观人以言,美于黼黻文章;听人以言,乐于钟鼓琴瑟。故君子之于言无厌。鄙夫反是:好其实不恤其文,是以终身不免埤污佣俗。故易曰:``括囊无咎无誉。''腐儒之谓也。

凡说之难,以至高遇至卑,以至治接至乱。未可直至也,远举则病缪,近世则病佣。善者于是间也,亦必远举而不缪,近世而不佣,与时迁徙,与世偃仰,缓急嬴绌,府然若渠匽檃栝之于己也。曲得所谓焉,然而不折伤。故君子之度己则以绳,接人则用抴。度己以绳,故足以为天下法则矣;接人用抴,故能宽容,因众以成天下之大事矣。故君子贤而能容罢,知而能容愚,博而能容浅,粹而能容杂,夫是之谓兼术。诗曰:``徐方既同,天子之功。''此之谓也。

谈说之术:矜庄以莅之,端诚以处之,坚强以持之,譬称以喻之,分别以明之,欣驩芬芗以送之,宝之,珍之,贵之,神之。如是则说常无不受。虽不说人,人莫不贵。夫是之谓为能贵其所贵。传曰:``唯君子为能贵其所贵。''此之谓也。

君子必辩。凡人莫不好言其所善,而君子为甚焉。是以小人辩言险,而君子辩言仁也。言而非仁之中也,则其言不若其默也,其辩不若其吶也。言而仁之中也,则好言者上矣,不好言者下也。故仁言大矣:起于上所以道于下,政令是也;起于下所以忠于上,谋救是也。故君子之行仁也无厌、志好之、行安之,乐言之;故言君子必辩。小辩不如见端,见端不如见本分。小辩而察,见端而明,本分而理;圣人士君子之分具矣。有小人之辩者,有士君子之辩者,有圣人之辩者:不先虑,不早谋,发之而当,成文而类,居错迁徙,应变不穷,是圣人之辩者也。先虑之,早谋之,斯须之言而足听,文而致实,博而党正,是士君子之辩者也。听其言则辞辩而无统,用其身则多诈而无功,上不足以顺明王,下不足以和齐百姓,然而口舌之均,应唯则节,足以为奇伟偃却之属,夫是之谓奸人之雄。圣王起,所以先诛也,然后盗贼次之。盗贼得变,此不得变也。

\hypertarget{header-n32}{%
\subsection{非十二子}\label{header-n32}}

假今之世,饰邪说,文奸言,以枭乱天下,矞宇嵬琐使天下混然不知是非治乱
之所在者,有人矣。

纵情性,安恣孳,禽兽行,不足以合文通治;然而其持之有故,其言之成理,
足以欺惑愚众;是它嚣魏牟也。

忍情性,綦溪利跂,苟以分异人为高,不足以合大众,明大分,然而其持之有
故,其言之成理,足以欺惑愚众:是陈仲史也。

不知壹天下建国家之权称,上功用,大俭约,而僈差等,曾不足以容辨异,县
君臣;然而其持之有故,其言之成理,足以欺惑愚众:是墨翟宋钘也。

尚法而无法,下修而好作,上则取听于上,下则取从于俗,终日言成文典,反
紃察之,则倜然无所归宿,不可以经国定分;然而其持之有故,其言之成理,足以
欺惑愚众:是慎到田骈也。

不法先王,不是礼义,而好治怪说,玩琦辞,甚察而不惠,辩而无用,多事而
寡功,不可以为治纲纪;然而其持之有故,其言之成理,足以欺惑愚众;是惠施邓
析也。

略法先王而不知其统,犹然而犹材剧志大,闻见杂博。案往旧造说,谓之五行,
甚僻违而无类,幽隐而无说,闭约而无解。案饰其辞,而只敬之,曰:此真先君子
之言也。子思唱之,孟轲和之。世俗之沟犹瞀儒、嚾嚾然不知其所非也,遂受而传
之,以为仲尼子弓为兹厚于后世:是则子思孟轲之罪也。

若夫总方略,齐言行,壹统类,而群天下之英杰,而告之以大古,教之以至顺,
奥窔之间,簟席之上,敛然圣王之文章具焉,佛然平世之俗起焉,六说者不能入也,
十二子者不能亲也。无置锥之地,而王公不能与之争名,在一大夫之位,则一君不
能独畜,一国不能独容,成名况乎诸侯,莫不愿以为臣,是圣人之不得埶者也,仲
尼子弓是也。一天下,财万物,长养人民,兼利天下,通达之属莫不从服,六说者
立息,十二子者迁化,则圣人之得埶者,舜禹是也。

今夫仁人也,将何务哉?上则法舜禹之制,下则法仲尼子弓之义,以务息十二
子之说。如是则天下之害除,仁人之事毕,圣王之迹着矣。

信信、信也,疑疑、亦信也。贵贤、仁也,贱不肖、亦仁也。言而当、知也,默而当,亦知也,故知默犹知言也。故多言而类,圣人也;少言而法,君子也;多
言无法,而流湎然,虽辩,小人也。故劳力而不当民务,谓之奸事,劳知而不律先
王,谓之奸心;辩说譬谕,齐给便利,而不顺礼义,谓之奸说。此三奸者,圣王之
所禁也。知而险,贼而神,为诈而巧,言无用而辩,辩不惠而察,治之大殃也。行
辟而坚,饰非而好,玩奸而泽,言辩而逆,古之大禁也。知而无法,勇而无惮,察
辩而操僻,淫大而用之,好奸而与众,利足而迷,负石而坠,是天下之所弃也。

兼服天下之心:高上尊贵,不以骄人;聪明圣知,不以穷人;齐给速通,不争
先人;刚毅勇敢,不以伤人;不知则问,不能则学,虽能必让,然后为德。遇君则
修臣下之义,遇乡则修长幼之义,遇长则修子弟之义,遇友则修礼节辞让之义,遇
贱而少者,则修告导宽容之义。无不爱也,无不敬也,无与人争也,恢然如天地之
苞万物。如是,则贤者贵之,不肖者亲之;如是,而不服者,则可谓訞怪狡猾之人
矣,虽则子弟之中,刑及之而宜。诗云:``匪上帝不时,殷不用旧;虽无老成人,
尚有典刑;曾是莫听,大命以倾。''此之谓也。

古之所谓仕士者,厚敦者也,合群者也,乐富贵者也,乐分施者也,远罪过者
也,务事理者也,羞独富者也。今之所谓仕士者,污漫者也,贼乱者也,恣孳者也,
贪利者也;触抵者也,无礼义而唯权埶之嗜者也。

古之所谓处士者,德盛者也,能静者也,修正者也,知命者也,箸是者也。今
之所谓处士者,无能而云能者也,无知而云知者也,利心无足,而佯无欲者也,行
伪险秽,而强高言谨悫者也,以不俗为俗,离纵而跂訾者也。

士君子之所能不能为:君子能为可贵,而不能使人必贵己;能为可信,而不能
使人必信己;能为可用,而不能使人必用己。故君子耻不修,不耻见污;耻不信,
不耻不见信;耻不能,不耻不见用。是以不诱于誉,不恐于诽,率道而行,端然正
己,不为物倾侧:夫是之谓诚君子。诗云:``温温恭人,维德之基。''此之谓也。

士君子之容:其冠进,其衣逢,其容良;俨然,壮然,祺然,蕼然,恢恢然,
广广然,昭昭然,荡荡然-是父兄之容也。其冠进,其衣逢,其容悫;俭然,恀然,
辅然,端然,訾然,洞然,缀缀然,瞀瞀然是子弟之容也。

吾语汝学者之嵬容:其冠絻,其缨禁缓,其容简连;填填然,狄狄然,莫莫然,
瞡瞡然,瞿瞿然,尽尽然,盱盱然;酒食声色之中,则瞒瞒然,瞑瞑然;礼节之中,
则疾疾然,訾訾然;劳苦事业之中,则儢儢然,离离然,偷儒而罔,无廉耻而忍謑
诟是学者之嵬也。

弟陀其冠,衶禫其辞,禹行而舜趋:是子张氏之贱儒也。正其衣冠,齐其颜色,
嗛然而终日不言、是子夏氏之贱儒也。偷儒惮事,无廉耻而耆饮食,必曰君子固不
用力:是子游氏之贱儒也。彼君子则不然:佚而不惰,劳而不僈,宗原应变,曲得
其宜,如是然后圣人也。

\hypertarget{header-n36}{%
\subsection{仲尼}\label{header-n36}}

仲尼之门,五尺之竖子,言羞称乎五伯。是何也?曰:然!彼诚可羞称也。齐桓五伯之盛者也,前事则杀兄而争国;内行则姑姊妹之不嫁者七人,闺门之内,般乐奢汏,以齐之分奉之而不足;外事则诈邾袭莒,并国三十五。其事行也若是其险污淫汏也。彼固曷足称乎大君子之门哉!

若是而不亡,乃霸,何也?曰:于乎!夫齐桓公有天下之大节焉,夫孰能亡之?倓然见管仲之能足以托国也,是天下之大知也。安忘其怒,出忘其雠,遂立为仲父,是天下之大决也。立以为仲父,而贵戚莫之敢妒也;与之高国之位,而本朝之臣莫之敢恶也;与之书社三百,而富人莫之敢距也;贵贱长少,秩秩焉,莫不从桓公而贵敬之,是天下之大节也。诸侯有一节如是,则莫之能亡也;桓公兼此数节者而尽有之,夫又何可亡也!其霸也,宜哉!非幸也,数也。

然而仲尼之门,五尺之竖子,言羞称五伯,是何也?曰:然!彼非本政教也,非致隆高也,非綦文理也,非服人之心也。乡方略,审劳佚,畜积修斗,而能颠倒其敌者也。诈心以胜矣。彼以让饰争,依乎仁而蹈利者也,小人之杰也,彼固曷足称乎大君子之门哉!

彼王者则不然:致贤而能以救不肖,致强而能以宽弱,战必能殆之而羞与之斗,委然成文,以示之天下,而暴国安自化矣。有灾缪者,然后诛之。故圣王之诛也綦省矣。文王诛四,武王诛二,周公卒业,至于成王,则安以无诛矣。故道岂不行矣哉!文王载百里地,而天下一;桀纣舍之,厚于有天下之埶,而不得以匹夫老。故善用之,则百里之国足以独立矣;不善用之,则楚六千里而为雠人役。故人主不务得道,而广有其埶,是其所以危也。

持宠处位,终身不厌之术:主尊贵之,则恭敬而僔;主信爱之,则谨慎而嗛;主专任之,则拘守而详:主安近之,则慎比而不邪;主疏远之,则全一而不倍;主损绌之,则恐惧而不怨。贵而不为夸,信而不处谦,任重而不敢专。财利至,则善而不及也,必将尽辞让之义,然后受。福事至则和而理,祸事至则静而理。富则广施,贫则用节。可贵可贱也,可富可贫也,可杀而不可使为奸也:是持宠处位终身不厌之术也。虽在贫穷徒处之埶,亦取象于是矣。夫是之谓吉人。诗云:``媚兹一人,应侯顺德,永言孝思,昭哉嗣服。''此之谓也。

求善处大重,理任大事,擅宠于万乘之国,必无后患之术,莫若好同之,援贤博施,除怨而无妨害人。能耐任之,则慎行此道也;能而不耐任,且恐失宠,则莫若早同之,推贤让能,而安随其后。如是,有宠则必荣,失宠则必无罪。是事君者之宝,而必无后患之术也。故知者之举事也,满则虑嗛,平则虑险,安则虑危,曲重其豫,犹恐及其祸,是以百举而不陷也。孔子曰:``巧而好度,必节;勇而好同,必胜;知而好谦,必贤。''此之谓也。愚者反是:处重擅权,则好专事而妒贤能,抑有功而挤有罪,志骄盈而轻旧怨,以吝啬而不行施,道乎上为重,招权于下以妨害人。虽欲无危,得乎哉!是以位尊则必危,任重则必废,擅宠则必辱,可立而待也,可炊而竟也。是何也?则堕之者众,而持之者寡矣。天下之行术,以事君则必通,以为仁则必圣,立隆而勿贰也。然后恭敬以先之,忠信以统之,慎谨以行之,端悫以守之,顿穷则从之疾力以申重之。君虽不知,无怨疾之心;功虽甚大,无伐德之色;省求多功,爱敬不倦;如是则常无不顺矣。以事君则必通,以为仁则必圣,夫之谓天下之行术。

少事长,贱事贵,不肖事贤,是天下之通义也。有人也,埶不在人上,而羞为人下,是奸人之心也。志不免乎奸心,行不免乎奸道,而求有君子圣人之名,辟之,是犹伏而咶天,救经而引其足也。说必不行矣,俞务而俞远。故君子时诎则诎,时伸则伸也。

\hypertarget{header-n40}{%
\subsection{儒效}\label{header-n40}}

大儒之效:武王崩,成王幼,周公屏成王而及武王,以属天下,恶天下之倍周也。履天子之籍,听天下之断,偃然如固有之,而天下不称贪焉。杀管叔,虚殷国,而天下不称戾焉。兼制天下,立七十一国,姬姓独居五十三人,而天下不称偏焉。教诲开导成王,使谕于道,而能揜迹于文武。周公归周,反籍于成王,而天下不辍事周;然而周公北面而朝之。天子也者,不可以少当也,不可以假摄为也;能则天下归之,不能则天下去之,是以周公屏成王而及武王,以属天下,恶天下之离周也。成王冠,成人,周公归周,反籍焉,明不灭主之义也。周公无天下矣;乡有天下,今无天下,非擅也;成王乡无天下,今有天下,非夺也;变埶次序节然也。故以枝代主而非越也;以弟诛兄而非暴也;君臣易位而非不顺也。因天下之和,遂文武之业,明主枝之义,抑亦变化矣,天下厌然犹一也。非圣人莫之能为。夫是之谓大儒之效。

秦昭王问孙卿子曰:``儒无益于人之国。''

孙卿子曰:``儒者法先王,隆礼义,谨乎臣子而致贵其上者也。人主用之,则埶在本朝而宜;不用,则退编百姓而悫;必为顺下矣。虽穷困冻餧,必不以邪道为贪。无置锥之地,而明于持社稷之大义。嘄呼而莫之能应,然而通乎财万物,养百姓之经纪。埶在人上,则王公之材也;在人下,则社稷之臣,国君之宝也;虽隐于穷阎漏屋,人莫不贵之,道诚存也。

``仲尼将为司寇,沈犹氏不敢朝饮其羊,公慎氏出其妻,慎溃氏踰境而徙,鲁之粥牛马者不豫贾,修正以待之也。居于阙党,阙党之子弟罔不分,有亲者取多,孝弟以化之也。儒者在本朝则美政,在下位则美俗。儒之为人下如是矣。''

王曰:``然则其为人上何如?''

孙卿曰:``其为人上也,广大矣!志意定乎内,礼节修乎朝,法则度量正乎官,忠信爱利形乎下。行一不义,杀一无罪,而得天下,不为也。此若义信乎人矣,通于四海,则天下应之如讙。是何也?则贵名白而天下治也。故近者歌讴而乐之,远者竭蹶而趋之,四海之内若一家,通达之属莫不从服。夫是之谓人师。诗曰:`自西自东,自南自北,无思不服。'此之谓也。夫其为人下也如彼,其为人上也如此,何谓其无益于人之国也!''

昭王曰:``善!''

先王之道,人之隆也,比中而行之.曷谓中?曰:礼义是也.道者,非天之道,非地之道,人之所以道也,君子之所道也.君子之所谓贤者,非能遍能人之所能之谓也;君子之所谓知者,非能遍知人之所知之谓也;君子之所谓辩者,非能遍辩人之所辩之谓也;君子之所谓察者,非能遍察人之所察之谓也;有所止矣。相高下,视硗肥,序五种,君子不如农人;通货财,相美恶,辩贵贱,君子不如贾人;设规矩,陈绳墨,便备用,君子不如工人;不恤是非然不然之情,以相荐樽,以相耻怍,君子不若惠施、邓析。若夫谲德而定次,量能而授官,使贤不肖皆得其位,能不能皆得其官,万物得其宜,事变得其应,慎墨不得进其谈,惠施、邓析不敢窜其察,言必当理,事必当务,是然后君子之所长也。

凡事行,有益于理者,立之;无益于理者,废之。夫是之谓中事。凡知说,有益于理者,为之;无益于理者,舍之。夫是之谓中说。事行失中,谓之奸事;知说失中,谓之奸道。奸事、奸道,治世之所弃,而乱世之所从服也。若夫充虚之相施易也,``坚白''``同异''之分隔也,是聪耳之所不能听也,明目之所不能见也,辩士之所不能言也,虽有圣人之知,未能偻指也。不知无害为君子,知之无损为小人。工匠不知,无害为巧;君子不知,无害为治。王公好之则乱法,百姓好之则乱事。而狂惑戆陋之人,乃始率其群徒,辩其谈说,明其辟称,老身长子,不知恶也。夫是之谓上愚,曾不如相鸡狗之可以为名也。诗曰:``为鬼为蜮,则不可得,有腼面目,视人罔极。作此好歌,以极反侧。''此之谓也。

我欲贱而贵,愚而智,贫而富,可乎?

曰:其唯学乎。彼学者,行之,曰士也;敦慕焉,君子也;知之,圣人也。上为圣人,下为士、君子,孰禁我哉!乡也混然涂之人也,俄而并乎尧禹,岂不贱而贵矣哉!乡也效门室之辨,混然曾不能决也,俄而原仁义,分是非,圆回天下于掌上,而辩黑白,岂不愚而知矣哉!乡也胥靡之人,俄而治天下之大器举在此,岂不贫而富矣哉!今有人于此,屑然藏千溢之宝,虽行貣而食,人谓之富矣。彼宝也者,衣之不可衣也,食之不可食也,卖之不可偻售也,然而人谓之富,何也?岂不大富之器诚在此也?是杅杅亦富人已,岂不贫而富矣哉!故君子无爵而贵,无禄而富,不言而信,不怒而威,穷处而荣,独居而乐!岂不至尊、至富、至重、至严之情举积此哉!

故曰:贵名不可以比周争也,不可以夸诞有也,不可以埶重胁也,必将诚此然后就也。争之则失,让之则至;遵道则积,夸诞则虚。故君子务修其内,而让之于外;务积德于身,而处之以遵道。如是,则贵名起如日月,天下应之如雷霆。故曰:君子隐而显,微而明,辞让而胜。诗曰:``鹤鸣于九皋,声闻于天。''此之谓也。鄙夫反是:比周而誉俞少,鄙争而名俞辱,烦劳以求安利,其身俞危。诗曰:``民之无良,相怨一方,受爵不让,至于己斯亡。''此之谓也。

故能小而事大,辟之是犹力之少而任重也,舍粹折无适也。身不肖而诬贤,是犹伛伸而好升高也,指其顶者愈众。故明主谲德而序位,所以为不乱也;忠臣诚能然后敢受职,所以为不穷也。分不乱于上,能不穷于下,治辩之极也。诗曰:``平平左右,亦是率从。''是言上下之交不相乱也。

以从俗为善,以货财为宝,以养生为己至道,是民德也。行法至坚,不以私欲乱所闻:如是,则可谓劲士矣。行法至坚,好修正其所闻,以桥饰其情性;其言多当矣,而未谕也;其行多当矣,而未安也;其知虑多当矣,而未周密也;上则能大其所隆,下则能开道不己若者:如是,则可谓笃厚君子矣。修百王之法,若辨白黑;应当时之变,若数一二;行礼要节而安之,若生四枝;要时立功之巧,若诏四时;平正和民之善,亿万之众而搏若一人:如是,则可谓圣人矣。

井井兮其有理也,严严兮其能敬己也,分分兮其有终始也,猒猒兮其能长久也,乐乐兮其执道不殆也,照照兮其用知之明也,修修兮其用统类之行也,绥绥兮其有文章也,熙熙兮其乐人之臧也,隐隐兮其恐人之不当也:如是,则可谓圣人矣。此其道出乎一。曷谓一?曰:执神而固。曷谓神?曰:尽善挟治之谓神,万物莫足以倾之之谓固。神固之谓圣人。

圣人也者,道之管也:天下之道管是矣,百王之道一是矣。故诗书礼乐之道归是矣。诗言是其志也,书言是其事也,礼言是其行也,乐言是其和也,春秋言是其微也,故风之所以为不逐者,取是以节之也,小雅之所以为小雅者,取是而文之也,大雅之所以为大雅者,取是而光之也,颂之所以为至者,取是而通之也。天下之道毕是矣。乡是者臧,倍是者亡;乡是如不臧,倍是如不亡者,自古及今,未尝有也。

客有道曰:孔子曰:``周公其盛乎!身贵而愈恭,家富而愈俭,胜敌而愈戒。''

应之曰:是殆非周公之行,非孔子之言也。武王崩,成王幼,周公屏成王而及武王,履天子之籍,负扆而立,诸侯趋走堂下。当是时也,夫又谁为恭矣哉!兼制天下立七十一国,姬姓独居五十三人焉;周之子孙,苟不狂惑者,莫不为天下之显诸侯。孰谓周公俭哉!武王之诛纣也,行之日以兵忌,东面而迎太岁,至泛而泛,至怀而坏,至共头而山隧。霍叔惧曰:``出三日而五灾至,无乃不可乎?''周公曰:``刳比干而囚箕子,飞廉、恶来知政,夫又恶有不可焉!''遂选马而进,朝食于戚,暮宿于百泉,旦厌于牧之野。鼓之而纣卒易乡,遂乘殷人而诛纣。盖杀者非周人,因殷人也。故无首虏之获,无蹈难之赏。反而定三革,偃五兵,合天下,立声乐,于是武象起而韶护废矣。四海之内,莫不变心易虑以化顺之。故外阖不闭,跨天下而无蕲。当是时也,夫又谁为戒矣哉!

造父者,天下之善御者也,无舆马则无所见其能。羿者,天下之善射者也,无弓矢则无所见其巧。大儒者,善调一天下者也,无百里之地,则无所见其功。舆固马选矣,而不能以至远,一日而千里,则非造父也。弓调矢直矣,而不能射远中微,则非羿也。用百里之地,而不能以调一天下,制强暴,则非大儒也。

彼大儒者,虽隐于穷阎漏屋,无置锥之地,而王公不能与之争名;在一大夫之位,则一君不能独畜,一国不能独容,成名况乎诸侯,莫不愿得以为臣。用百里之地,而千里之国莫能与之争胜;笞棰暴国,齐一天下,而莫能倾也。是大儒之征也。其言有类,其行有礼,其举事无悔,其持险应变曲当。与时迁徙,与世偃仰,千举万变,其道一也。是大儒之稽也。其穷也俗儒笑之;其通也英杰化之,嵬琐逃之,邪说畏之,众人媿之。通则一天下,穷则独立贵名,天不能死,地不能埋,桀跖之世不能污,非大儒莫之能立,仲尼、子弓是也。

故有俗人者,有俗儒者,有雅儒者,有大儒者。不学问,无正义,以富利为隆,是俗人者也。逢衣浅带,解果其冠,略法先王而足乱世术,缪学杂举,不知法后王而一制度,不知隆礼义而杀诗书;其衣冠行伪已同于世俗矣,然而不知恶;其言议谈说已无异于墨子矣,然而明不能别;呼先王以欺愚者而求衣食焉;得委积足以揜其口,则扬扬如也;随其长子,事其便辟,举其上客,亿然若终身之虏而不敢有他志:是俗儒者也。法后王,一制度,隆礼义而杀诗书;其言行已有大法矣,然而明不能齐法教之所不及,闻见之所未至,则知不能类也;知之曰知之,不知曰不知,内不自以诬,外不自以欺,以是尊贤畏法而不敢怠傲:是雅儒者也。法先王,统礼义,一制度;以浅持博,以古持今,以一持万;苟仁义之类也,虽在鸟兽之中,若别白黑;倚物怪变,所未尝闻也,所未尝见也,卒然起一方,则举统类而应之,无所儗作;张法而度之,则晻然若合符节:是大儒者也。

故人主用俗人,则万乘之国亡;用俗儒,则万乘之国存;用雅儒,则千乘之国安;用大儒,则百里之地,久而后三年,天下为一,诸侯为臣;用万乘之国,则举错而定,一朝而伯。

不闻不若闻之,闻之不若见之,见之不若知之,知之不若行之。学至于行之而止矣。行之,明也;明之为圣人。圣人也者,本仁义,当是非,齐言行,不失豪厘,无他道焉,已乎行之矣。故闻之而不见,虽博必谬;见之而不知,虽识必妄;知之而不行,虽敦必困。不闻不见,则虽当,非仁也。其道百举而百陷也。

故人无师无法而知,则必为盗,勇则必为贼,云能则必为乱,察则必为怪,辩则必为诞;人有师有法,而知则速通,勇则速畏,云能则速成,察则速尽,辩则速论。故有师法者,人之大宝也;无师法者,人之大殃也。人无师法,则隆性矣;有师法,则隆积矣。而师法者,所得乎积,非所受乎性。性不足以独立而治。性也者,吾所不能为也,然而可化也。积也者,非吾所有也,然而可为也。注错习俗,所以化性也;并一而不二,所以成积也。习俗移志,安久移质。并一而不二,则通于神明,参于天地矣。

故积土而为山,积水而为海,旦暮积谓之岁,至高谓之天,至下谓之地,宇中六指谓之极,涂之人百姓,积善而全尽,谓之圣人。彼求之而后得,为之而后成,积之而后高,尽之而后圣,故圣人也者,人之所积也。人积耨耕而为农夫,积斲削而为工匠,积反货而为商贾,积礼义而为君子。工匠之子,莫不继事,而都国之民安习其服,居楚而楚,居越而越,居夏而夏,是非天性也,积靡使然也。故人知谨注错,慎习俗,大积靡,则为君子矣。纵情性而不足问学,则为小人矣;为君子则常安荣矣,为小人则常危辱矣。凡人莫不欲安荣而恶危辱,故唯君子为能得其所好,小人则日徼其所恶。诗曰:``维此良人,弗求弗迪;唯彼忍心,是顾是复。民之贪乱,宁为荼毒。''此之谓也。

人论:志不免于曲私,而冀人之以己为公也;行不免于污漫,而冀人之以己为修也;甚愚陋沟瞀,而冀人之以己为知也:是众人也。志忍私,然后能公;行忍情性,然后能修;知而好问,然后能才;公修而才,可谓小儒矣。志安公,行安修,知通统类:如是则可谓大儒矣。大儒者,天子三公也;小儒者,诸侯、大夫、士也;众人者,工农商贾也。礼者、人主之所以为群臣寸尺寻丈检式也。人伦尽矣。

君子言有坛宇,行有防表,道有一隆。言政治之求,不下于安存;言志意之求,不下于士;言道德之求,不二后王。道过三代谓之荡,法二后王谓之不雅。高之下之,小之巨之,不外是矣。是君子之所以骋志意于坛宇宫廷也。故诸侯问政,不及安存,则不告也。匹夫问学,不及为士,则不教也。百家之说,不及后王,则不听也。夫是之谓君子言有坛宇,行有防表也。

\hypertarget{header-n44}{%
\subsection{王制}\label{header-n44}}

请问为政?曰:贤能不待次而举,罢不能不待须而废,元恶不待教而诛,中庸不待政而化。分未定也,则有昭缪。虽王公士大夫之子孙也,不能属于礼义,则归
之庶人。虽庶人之子孙也,积文学,正身行,能属于礼义,则归之卿相士大夫。故
奸言,奸说,奸事,奸能,遁逃反侧之民,职而教之,须而待之,勉之以庆赏,惩
之以刑罚。安职则畜,不安职则弃。五疾,上收而养之,材而事之,官施而衣食之,
兼覆无遗。才行反时者死无赦。夫是之谓天德,是王者之政也。

听政之大分:以善至者待之以礼,以不善至者待之以刑。两者分别,则贤不肖
不杂,是非不乱。贤不肖不杂,则英杰至,是非不乱,则国家治。若是,名声日闻,
天下愿,令行禁止,王者之事毕矣。

凡听:威严猛厉,而不好假道人,则下畏恐而不亲,周闭而不竭。若是,则大
事殆乎弛,小事殆乎遂。和解调通,好假道人,而无所凝止之,则奸言并至,尝试
之说锋起。若是,则听大事烦,是又伤之也。故法而不议,则法之所不至者必废。
职而不通,则职之所不及者必队。故法而议,职而通,无隐谋,无遗善,而百事无
过,非君子莫能。故公平者,听之衡也;中和者,听之绳也。其有法者以法行,无
法者以类举,听之尽也。偏党而不经,听之辟也。故有良法而乱者,有之矣,有君
子而乱者,自古及今,未尝闻也。传曰:``治生乎君子,乱生乎小人。''此之谓也。

分均则不偏,埶齐则不壹,众齐则不使。有天有地,而上下有差;明王始立,
而处国有制。夫两贵之不能相事,两贱之不能相使,是天数也。埶位齐,而欲恶同,
物不能澹则必争;争则必乱,乱则穷矣。先王恶其乱也,故制礼义以分之,使有贫
富贵贱之等,足以相兼临者,是养天下之本也。书曰:``维齐非齐。''此之谓也。

马骇舆,则君子不安舆;庶人骇政,则君子不安位。马骇舆,则莫若静之;庶
人骇政,则莫若惠之。选贤良,举笃敬,兴孝弟,收孤寡,补贫穷。如是,则庶人
安政矣。庶人安政,然后君子安位。传曰:``君者、舟也,庶人者、水也;水则载
舟,水则覆舟。''此之谓也。故君人者,欲安、则莫若平政爱民矣;欲荣、则莫若
隆礼敬士矣;欲立功名、则莫若尚贤使能矣。-是人君之大节也。三节者当,则其
余莫不当矣。三节者不当,则其余虽曲当,犹将无益也。孔子曰:``大节是也,小
节是也,上君也;大节是也,小节一出焉,一入焉,中君也;大节非也,小节虽是
也,吾无观其余矣。''

成侯、嗣公聚敛计数之君也,未及取民也。子产取民者也,未及为政也。管仲
为政者也,未及修礼也。故修礼者王,为政者强,取民者安,聚敛者亡。故王者富
民,霸者富士,仅存之国富大夫,亡国富筐箧,实府库。筐箧已富,府库已实,而
百姓贫:夫是之谓上溢而下漏。入不可以守,出不可以战,则倾覆灭亡可立而待也。
故我聚之以亡,敌得之以强。聚敛者,召寇、肥敌、亡国、危身之道也,故明君不
蹈也。

王夺之人,霸夺之与,强夺之地。夺之人者臣诸侯,夺之与者友诸侯,夺之地
者敌诸侯。臣诸侯者王,友诸侯者霸,敌诸侯者危。

用强者:人之城守,人之出战,而我以力胜之也,则伤人之民必甚矣;伤人之
民甚,则人之民必恶我甚矣;人之民恶我甚,则日欲与我斗。人之城守,人之出战,
而我以力胜之,则伤吾民必甚矣;伤吾民甚,则吾民之恶我必甚矣;吾民之恶我甚,
则日不欲为我斗。人之民日欲与我斗,吾民日不欲为我斗,是强者之所以反弱也。
地来而民去,累多而功少,虽守者益,所以守者损,是以大者之所以反削也。诸侯
莫不怀交接怨,而不忘其敌,伺强大之间,承强大之敝,此强大之殆时也。

知强大者不务强也,虑以王命,全其力,凝其德。力全则诸侯不能弱也,德凝
则诸侯不能削也,天下无王霸主,则常胜矣:是知强道者也。

彼霸者则不然:辟田野,实仓廪,便备用,案谨募选阅材伎之士,然后渐庆赏
以先之,严刑罚以纠之。存亡继绝,卫弱禁暴,而无兼幷之心,则诸侯亲之矣。修
友敌之道,以敬接诸侯,则诸侯说之矣。所以亲之者,以不幷也;幷之见,则诸侯
疏矣。所以说之者,以友敌也;臣之见,则诸侯离矣。故明其不幷之行,信其友敌
之道,天下无王霸主,则常胜矣。是知霸道者也。

闵王毁于五国,桓公劫于鲁庄,无它故焉,非其道而虑之以王也。

彼王者不然:仁眇天下,义眇天下,威眇天下。仁眇天下,故天下莫不亲也;
义眇天下,故天下莫不贵也;威眇天下,故天下莫敢敌也。以不敌之威,辅服人之
道,故不战而胜,不攻而得,甲兵不劳而天下服,是知王道者也。知此三具者,欲
王而王,欲霸而霸,欲强而强矣。

王者之人:饰动以礼义,听断以类,明振毫末,举措应变而不穷,夫是之谓有
原。是王者之人也。

王者之制:道不过三代,法不二后王;道过三代谓之荡,法二后王谓之不雅。
衣服有制,宫室有度,人徒有数,丧祭械用皆有等宜。声、则非雅声者举废,色、
则凡非旧文者举息,械用,则凡非旧器者举毁,夫是之谓复古,是王者之制也。

王者之论:无德不贵,无能不官,无功不赏,无罪不罚。朝无幸位,民无幸生。
尚贤使能,而等位不遗;析愿禁悍,而刑罚不过。百姓晓然皆知夫为善于家,而取
赏于朝也;为不善于幽,而蒙刑于显也。夫是之谓定论。是王者之论也。

王者之法:等赋、政事、财万物,所以养万民也。田野什一,关市几而不征,
山林泽梁,以时禁发而不税。相地而衰政。理道之远近而致贡。通流财物粟米,无
有滞留,使相归移也,四海之内若一家。故近者不隐其能,远者不疾其劳,无幽闲
隐僻之国,莫不趋使而安乐之。夫是之为人师。是王者之法也。

北海则有走马吠犬焉,然而中国得而畜使之。南海则有羽翮、齿革、曾青、丹
干焉,然而中国得而财之。东海则有紫紶、鱼盐焉,然而中国得而衣食之。西海则
有皮革、文旄焉,然而中国得而用之。故泽人足乎木,山人足乎鱼,农夫不斲削、
不陶冶而足械用,工贾不耕田而足菽粟。故虎豹为猛矣,然君子剥而用之。故天之
所覆,地之所载,莫不尽其美,致其用,上以饰贤良,下以养百姓而安乐之。夫是
之谓大神。诗曰:``天作高山,大王荒之;彼作矣,文王康之。''此之谓也。

以类行杂,以一行万。始则终,终则始,若环之无端也,舍是而天下以衰矣。
天地者,生之始也;礼义者,治之始也;君子者,礼义之始也;为之,贯之,积重
之,致好之者,君子之始也。故天地生君子,君子理天地;君子者,天地之参也,
万物之摠也,民之父母也。无君子,则天地不理,礼义无统,上无君师,下无父子,
夫是之谓至乱。君臣、父子、兄弟、夫妇,始则终,终则始,与天地同理,与万世
同久,夫是之谓大本。故丧祭、朝聘、师旅一也;贵贱、杀生、与夺一也;君君、
臣臣、父父、子子、兄兄、弟弟一也;农农、士士、工工、商商一也。

水火有气而无生,草木有生而无知,禽兽有知而无义,人有气、有生、有知,
亦且有义,故最为天下贵也。力不若牛,走不若马,而牛马为用,何也?曰:人能
群,彼不能群也。人何以能群?曰:分。分何以能行?曰:义。故义以分则和,和
则一,一则多力,多力则强,强则胜物;故宫室可得而居也。故序四时,裁万物,
兼利天下,无它故焉,得之分义也。

故人生不能无群,群而无分则争,争则乱,乱则离,离则弱,弱则不能胜物;
故宫室不可得而居也,不可少顷舍礼义之谓也。能以事亲谓之孝,能以事兄谓之弟,
能以事上谓之顺,能以使下谓之君。

君者,善群也。群道当,则万物皆得其宜,六畜皆得其长,群生皆得其命。故
养长时,则六畜育;杀生时,则草木殖;政令时,则百姓一,贤良服。

圣王之制也:草木荣华滋硕之时,则斧斤不入山林,不夭其生,不绝其长也。
鼋鼍鱼鳖鳅鳣孕别之时,罔罟毒药不入泽,不夭其生,不绝其长也。春耕、夏耘、
秋收、冬藏,四者不失时,故五谷不绝,而百姓有余食也。污池渊沼川泽,谨其时
禁,故鱼鳖优多,而百姓有余用也。斩伐养长不失其时,故山林不童,而百姓有余
材也。

圣王之用也:上察于天,下错于地,塞备天地之间,加施万物之上,微而明,
短而长,狭而广,神明博大以至约。故曰:一与一是为人者,谓之圣人。

序官:宰爵知宾客、祭祀、响食牺牲之牢数。司徒知百宗、城郭、立器之数。
司马知师旅、甲兵、乘白之数。修宪命,审诗商,禁淫声,以时顺修,使夷俗邪音
不敢乱雅,大师之事也。修堤梁,通沟浍,行水潦,安水臧,以时决塞,岁虽凶败
水旱,使民有所耘艾,司空之事也。相高下,视肥硗,序五种,省农功,谨蓄藏,
以时顺修,使农夫朴力而寡能,治田之事也。修火宪,养山林薮泽草木、鱼鳖、百
索,以时禁发,使国家足用,而财物不屈,虞师之事也。顺州里,定廛宅,养六畜,
闲树艺,劝教化,趋孝弟,以时顺修,使百姓顺命,安乐处乡,乡师之事也。论百
工,审时事,辨功苦,尚完利,便备用,使雕琢文采不敢专造于家,工师之事也。
相阴阳,占祲兆,钻龟陈卦,主攘择五卜,知其吉凶妖祥,伛巫跛击之事也。修采
清,易道路,谨盗贼,平室律,以时顺修,使宾旅安而货财通,治市之事也。抃急
禁悍,防淫除邪,戮之以五刑,使暴悍以变,奸邪不作,司寇之事也。本政教,正
法则,兼听而时稽之,度其功劳,论其庆赏,以时慎修,使百吏免尽,而众庶不偷,
冢宰之事也。论礼乐,正身行,广教化,美风俗,兼覆而调一之,辟公之事也。全
道德,致隆高,綦文理,一天下,振毫末,使天下莫不顺比从服,天王之事也。故
政事乱,则冢宰之罪也;国家失俗,则辟公之过也;天下不一,诸侯俗反,则天王
非其人也。

具具而王,具具而霸,具具而存,具具而亡。用万乘之国者,威强之所以立也,
名声之所以美也,敌人之所以屈也,国之所以安危臧否也,制与在此,亡乎人。王、
霸、安存、危殆、灭亡,制与在我,亡乎人。夫威强未足以殆邻敌也,名声未足以
县天下也,则是国未能独立也,岂渠得免夫累乎?天下胁于暴国,而党为吾所不欲
于是者,日与桀同事同行,无害为尧。是非功名之所就也,非存亡安危之所堕也。
功名之所就,存亡安危之所堕,必将于愉殷赤心之所。诚以其国为王者之所亦王,
以其国为危殆灭亡之所亦危殆灭亡。殷之日,案以中立,无有所偏,而为纵横之事,
偃然案兵无动,以观夫暴国之相卒也。案平政教,审节奏,砥砺百姓,为是之日,
而兵剸天下劲矣。案然修仁义,伉隆高,正法则,选贤良,养百姓,为是之日,而
名声剸天下之美矣。权者重之,兵者劲之,名声者美之。夫尧舜者一天下也,不能
加毫末于是矣。

权谋倾覆之人退,则贤良知圣之士案自进矣。刑政平,百姓和,国俗节,则兵
劲城固,敌国案自诎矣。务本事,积财物,而勿忘栖迟薛越也,是使群臣百姓皆以
制度行,则财物积,国家案自富矣。三者体此而天下服,暴国之君案自不能用其兵
矣。何则?彼无与至也。彼其所与至者,必其民也。其民之亲我,欢若父母,好我
芳如芝兰,反顾其上则若灼黥,若仇雠;彼人之情性也虽桀跖,岂有肯为其所恶,
贼其所好者哉!彼以夺矣。故古之人,有以一国取天下者,非往行之也,修政其所,
天下莫不愿,如是而可以诛暴禁悍矣。故周公南征而北国怨,曰:``何独不来也!''
东征而西国怨,曰:``何独后我也!''孰能有与是斗者与?安以其国为是者王。殷
之日,安以静兵息民,慈爱百姓,辟田野,实仓廪,便备用,安谨募选阅材伎之士,
然后渐赏庆以先之,严刑罚以防之,择士之知事者,使相率贯也,是以厌然畜积修
饰,而物用之足也。兵革器械者,彼将日日暴露毁折之中原;我将修饰之,拊循之,
掩盖之于府库。货财粟米者,彼将日日栖迟薛越之中野,我今将畜积幷聚之于仓廪。
材伎股肱健勇爪牙之士,彼将日日挫顿竭之于仇敌,我今将来致之,幷阅之,砥砺
之于朝廷。如是,则彼日积敝,我日积完;彼日积贫,我日积富;彼日积劳,我日
积佚。君臣上下之间者,彼将厉厉焉日日相离疾也,我将顿顿焉日日相亲爱也,以
是待其敝。安以其国为是者霸。立身则从佣俗,事行则遵佣故,进退贵贱则举佣士,
之所以接下之人百姓者则庸宽惠,如是者则安存。立身则轻楛,事行则蠲疑,进退
贵贱则举佞侻,之所以接下之人百姓者则好取侵夺,如是者危殆。立身则憍暴,事
行则倾覆,进退贵贱则举幽险诈故,之所以接下之人百姓者,则好用其死力矣,而
慢其功劳,好用其籍敛矣,而忘其本务,如是者灭亡。此五等者,不可不善择
也,王、霸、安存、危殆、灭亡之具也。善择者制人,不善择者人制之。善择之者
王,不善择之者亡。夫王者之与亡者,制人之与人制之也,是其为相县也亦远矣。

\hypertarget{header-n48}{%
\subsection{富国}\label{header-n48}}

万物同宇而异体,无宜而有用为人,数也。人伦并处,同求而异道,同欲而异
知,生也。皆有可也,知愚同;所可异也,知愚分。埶同而知异,行私而无祸,纵
欲而不穷,则民心奋而不可说也。如是,则知者未得治也;知者未得治,则功名未
成也;功名未成,则群众未县也;群众未县,则君臣未立也。无君以制臣,无上以
制下,天下害生纵欲。欲恶同物,欲多而物寡,寡则必争矣。故百技所成,所以养
一人也。而能不能兼技,人不能兼官。离居不相待则穷,群居而无分则争;穷者患
也,争者祸也,救患除祸,则莫若明分使群矣。强胁弱也,知惧愚也,民下违上,
少陵长,不以德为政:如是,则老弱有失养之忧,而壮者有分争之祸矣。事业所恶
也,功利所好也,职业无分:如是,则人有树事之患,而有争功之祸矣。男女之合,
夫妇之分,婚姻娉内,送逆无礼:如是,则人有失合之忧,而有争色之祸矣。故知
者为之分也。

足国之道:节用裕民,而善臧其余。节用以礼,裕民以政。彼裕民,故多余。
裕民则民富,民富则田肥以易,田肥以易则出实百倍。上以法取焉,而下以礼节用
之,余若丘山,不时焚烧,无所臧之。夫君子奚患乎无余?故知节用裕民,则必有
仁圣贤良之名,而且有富厚丘山之积矣。此无他故焉,生于节用裕民也。不知节用
裕民则民贫,民贫则田瘠以秽,田瘠以秽则出实不半;上虽好取侵夺,犹将寡获也。
而或以无礼节用之,则必有贪利纠譑之名,而且有空虚穷乏之实矣。此无他故焉,
不知节用裕民也。康诰曰:``弘覆乎天,若德裕乃身。''此之谓也。

礼者,贵贱有等;长幼有差,贫富轻重皆有称者也。故天子袾裷衣冕,诸侯玄
裷衣冕,大夫裨冕,士皮弁服。德必称位,位必称禄,禄必称用,由士以上则必以
礼乐节之,众庶百姓则必以法数制之。量地而立国,计利而畜民,度人力而授事,
使民必胜事,事必出利,利足以生民,皆使衣食百用出入相揜,必时臧余,谓之称
数。故自天子通于庶人,事无大小多少,由是推之。故曰:``朝无幸位,民无幸生。''
此之谓也。轻田野之赋,平关市之征,省商贾之数,罕兴力役,无夺农时,如是则
国富矣。夫是之谓以政裕民。

人之生不能无群,群而无分则争,争则乱,乱则穷矣。故无分者,人之大害也;
有分者,天下之本利也;而人君者,所以管分之枢要也。故美之者,是美天下之本
也;安之者,是安天下之本也;贵之者,是贵天下之本也。古者先王分割而等异之
也,故使或美,或恶,或厚,或薄,或佚或乐,或劬或劳,非特以为淫泰夸丽之声,
将以明仁之文,通仁之顺也。故为之雕琢、刻镂、黼黻文章,使足以辨贵贱而已,
不求其观;为之钟鼓、管磬、琴瑟、竽笙,使足以辨吉凶、合欢、定和而已,不求
其余;为之宫室、台榭,使足以避燥湿、养德、辨轻重而已,不求其外。诗曰:
``雕琢其章,金玉其相,亹亹我王,纲纪四方。''此之谓也。

若夫重色而衣之,重味而食之,重财物而制之,合天下而君之,非特以为淫泰
也,固以为主天下,治万变,材万物,养万民,兼制天下者,为莫若仁人之善也夫。
故其知虑足以治之,其仁厚足以安之,其德音足以化之,得之则治,失之则乱。百
姓诚赖其知也,故相率而为之劳苦以务佚之,以养其知也;诚美其厚也,故为之出
死断亡以覆救之,以养其厚也;诚美其德也,故为之雕琢、刻镂、黼黻、文章以藩
饰之,以养其德也。故仁人在上,百姓贵之如帝,亲之如父母,为之出死断亡而愉
者,无它故焉,其所是焉诚美,其所得焉诚大,其所利焉诚多。诗曰:``我任我辇,
我车我牛,我行既集,盖云归哉!''此之谓也。

故曰:君子以德,小人以力;力者,德之役也。百姓之力,待之而后功;百姓
之群,待之而后和;百姓之财,待之而后聚;百姓之埶,待之而后安;百姓之寿,
待之而后长;父子不得不亲,兄弟不得不顺,男女不得不欢。少者以长,老者以养。
故曰:``天地生之,圣人成之。''此之谓也。

今之世而不然:厚刀布之敛,以夺之财;重田野之赋,以夺之食;苛关市之征,
以难其事。不然而已矣:有掎絜伺诈,权谋倾覆,以相颠倒,以靡敝之。百姓晓然
皆知其污漫暴乱,而将大危亡也。是以臣或弒其君,下或杀其上,粥其城,倍其节,
而不死其事者,无他故焉,人主自取之。诗曰:``无言不雠,无德不报。''此之谓
也。

兼足天下之道在明分:掩地表亩,刺屮殖谷,多粪肥田,是农夫众庶之事也。
守时力民,进事长功,和齐百姓,使人不偷,是将率之事也。高者不旱,下者不水,
寒暑和节,而五谷以时孰,是天之事也。若夫兼而覆之,兼而爱之,兼而制之,岁
虽凶败水旱,使百姓无冻餧之患,则是圣君贤相之事也。

墨子之言昭昭然为天下忧不足。夫不足非天下之公患也,特墨子之私忧过计也。
今是土之生五谷也,人善治之,则亩数盆,一岁而再获之。然后瓜桃枣李一本数以
盆鼓;然后荤菜百疏以泽量;然后六畜禽兽一而剸车;鼋、鼍、鱼、鳖、鳅、鳣以
时别,一而成群;然后飞鸟、凫、雁若烟海;然后昆虫万物生其间,可以相食养者,
不可胜数也。夫天地之生万物也,固有余,足以食人矣;麻葛茧丝、鸟兽之羽毛齿
革也,固有余,足以衣人矣。夫有余不足,非天下之公患也,特墨子之私忧过计也。

天下之公患,乱伤之也。胡不尝试相与求乱之者谁也?我以墨子之``非乐''也,
则使天下乱;墨子之``节用''也,则使天下贫,非将堕之也,说不免焉。墨子大有
天下,小有一国,将蹙然衣粗食恶,忧戚而非乐。若是则瘠,瘠则不足欲;不足欲
则赏不行。墨子大有天下,小有一国,将少人徒,省官职,上功劳苦,与百姓均事
业,齐功劳。若是则不威;不威则罚不行。赏不行,则贤者不可得而进也;罚不行,
则不肖者不可得而退也。贤者不可得而进也,不肖者不可得而退也,则能不能不可
得而官也。若是,则万物失宜,事变失应,上失天时,下失地利,中失人和,天下
敖然,若烧若焦,墨子虽为之衣褐带索,嚽菽饮水,恶能足之乎?既以伐其本,竭
其原,而焦天下矣。

故先王圣人为之不然:知夫为人主上者,不美不饰之不足以一民也,不富不厚
之不足以管下也,不威不强之不足以禁暴胜悍也,故必将撞大钟,击鸣鼓,吹笙竽,
弹琴瑟,以塞其耳;必将錭琢刻镂,黼黻文章,以塞其目;必将刍豢稻粱,五味芬
芳,以塞其口。然后众人徒,备官职,渐庆赏,严刑罚,以戒其心。使天下生民之
属,皆知己之所愿欲之举在是于也,故其赏行;皆知己之所畏恐之举在是于也,故
其罚威。赏行罚威,则贤者可得而进也,不肖者可得而退也,能不能可得而官也。
若是则万物得宜,事变得应,上得天时,下得地利,中得人和,则财货浑浑如泉源,
汸汸如河海,暴暴如丘山,不时焚烧,无所臧之。夫天下何患乎不足也?故儒术诚
行,则

(缺两行)

撞钟击鼓而和。诗曰:``钟鼓喤喤,管磬玱玱,降福穰穰,降福简简,威仪反
反。既醉既饱,福禄来反。''此之谓也。故墨术诚行,则天下尚俭而弥贫,非斗而
日争,劳苦顿萃,而愈无功,愀然忧戚非乐,而日不和。诗曰:``天方荐瘥,丧乱
弘多,民言无嘉,憯莫惩嗟。''此之谓也。

垂事养民,拊循之,唲呕之,冬日则为之饘粥,夏日则为之瓜麮,以偷取少顷
之誉焉,是偷道也。可以少顷得奸民之誉,然而非长久之道也;事必不就,功必不
立,是奸治者也。傮然要时务民,进事长功,轻非誉而恬失民,事进矣,而百姓疾
之,是又偷偏者也。徙坏堕落,必反无功。故垂事养誉,不可;以遂功而忘民,亦
不可。皆奸道也。

故古人为之不然:使民夏不宛喝,冬不冻寒,急不伤力,缓不后时,事成功立,
上下俱富;而百姓皆爱其上,人归之如流水,亲之欢如父母,为之出死断亡而愉者,
无它故焉,忠信、调和、均辨之至也。故国君长民者,欲趋时遂功,则和调累解,
速乎急疾;忠信均辨,说乎庆赏矣;必先修正其在我者,然后徐责其在人者,威乎
刑罚。三德者诚乎上,则下应之如景向,虽欲无明达,得乎哉!书曰:``乃大明服,
惟民其力懋,和而有疾。''此之谓也。

故不教而诛,则刑繁而邪不胜;教而不诛,则奸民不惩;诛而不赏,则勤厉之
民不劝;诛赏而不类,则下疑俗险而百姓不一。故先王明礼义以壹之,致忠信以爱
之,尚贤使能以次之,爵服庆赏以申重之,时其事,轻其任,以调齐之,潢然兼覆
之,养长之,如保赤子。若是,故奸邪不作,盗贼不起,而化善者劝勉矣。是何邪?
则其道易,其塞固,其政令一,其防表明。故曰:上一则下一矣,上二则下二矣。
辟之若屮木枝叶必类本。此之谓也。

不利而利之,不如利而后利之之利也。不爱而用之,不如爱而后用之之功也。
利而后利之,不如利而不利者之利也。爱而后用之,不如爱而不用者之功也。利而
不利也,爱而不用也者,取天下者也。利而后利之,爱而后用之者,保社稷者也。
不利而利之,不爱而用之者,危国家者也。

观国之治乱臧否,至于疆易而端已见矣。其候缴支缭,其竟关之政尽察是
乱国已。入其境,其田畴秽,都邑露是贪主已。观其朝廷,则其贵者不贤;观
其官职,则其治者不能;观其便嬖,则其信者不悫是闇主已。凡主相臣下百吏
之属,其于货财取与计数也,顺孰尽察;其礼义节奏也,芒轫僈楛是辱国已。
其耕者乐田,其战士安难,其百吏好法,其朝廷隆礼,其卿相调议是治国已。
观其朝廷,则其贵者贤;观其官职,则其治者能;观其便嬖,则其信者悫是明
主已。凡主相臣下百吏之属,其于货财取与计数也,宽饶简易;其于礼义节奏也,
陵谨尽察是荣国已。贤齐则其亲者先贵,能齐则其故者先官,其臣下百吏,污
者皆化而修,悍者皆化而愿,躁者皆化而悫是明主之功已。

观国之强弱贫富有征验:上不隆礼则兵弱,上不爱民则兵弱,已诺不信则兵弱,
庆赏不渐则兵弱,将率不能则兵弱。上好功则国贫,上好利则国贫,士大夫众则国
贫,工商众则国贫,无制数度量则国贫。下贫则上贫,下富则上富。故田野县鄙者,
财之本也;垣窌仓廪者,财之末也。百姓时和,事业得叙者,货之源也;等赋府库
者,货之流也。故明主必谨养其和,节其流,开其源,而时斟酌焉。潢然使天下必
有余,而上不忧不足。如是,则上下俱富,交无所藏之。是知国计之极也。故禹十
年水,汤七年旱,而天下无菜色者,十年之后,年谷复熟,而陈积有余。是无它故
焉,知本末源流之谓也。故田野荒而仓廪实,百姓虚而府库满,夫是之谓国蹶。伐
其本,竭其源,而并之其末,然而主相不知恶也,则其倾覆灭亡可立而待也。以国
持之,而不足以容其身,夫是之谓至贫,是愚主之极也。将以取富而丧其国,将以
取利而危其身,古有万国,今有十数焉,是无它故焉,其所以失之一也。君人者亦
可以觉矣。百里之国,足以独立矣。

凡攻人者,非以为名,则案以为利也;不然则忿之也。仁人之用国,将修志意,
正身行,伉隆高,致忠信,期文理。布衣紃屦之士诚是,则虽在穷阎漏屋,而王公
不能与之争名;以国载之,则天下莫之能隐匿也。若是则为名者不攻也。将辟田野,
实仓廪,便备用,上下一心,三军同力,与之远举极战则不可;境内之聚也保固;
视可,午其军,取其将,若拨麷。彼得之,不足以药伤补败。彼爱其爪牙,畏其仇
敌,若是则为利者不攻也。将修大小强弱之义,以持慎之,礼节将甚文,圭璧将甚
硕,货赂将甚厚,所以说之者,必将雅文辩慧之君子也。彼苟有人意焉,夫谁能忿
之?若是,则忿之者不攻也。为名者否,为利者否,为忿者否,则国安于盘石,寿
于旗翼。人皆乱,我独治;人皆危,我独安;人皆丧失之,我按起而治之。故仁人
之用国,非特将持其有而已也,又将兼人。诗曰:``淑人君子,其仪不忒;其仪不
忒,正是四国。''此之谓也。

持国之难易:事强暴之国难,使强暴之国事我易。事之以货宝,则货宝单,而
交不结;约信盟誓,则约定而畔无日;割国之锱铢以赂之,则割定而欲无厌。事之
弥烦,其侵人愈甚,必至于资单国举然后已。虽左尧而右舜,未有能以此道得免焉
者也。譬之是犹使处女婴宝珠,佩宝玉,负戴黄金,而遇中山之盗也,虽为之逢蒙
视,诎要挠腘,君卢屋妾,由将不足以免也。故非有一人之道也,直将巧繁拜请而
畏事之,则不足以持国安身。故明君不道也。必将修礼以齐朝,正法以齐官,平政
以齐民;然后节奏齐于朝,百事齐于官,众庶齐于下。如是,则近者竞亲,远方致
愿,上下一心,三军同力,名声足以暴炙之,威强足以捶笞之,拱揖指挥,而强暴
之国莫不趋使,譬之是犹乌获与焦侥搏也。故曰:事强暴之国难,使强暴之国事我
易。此之谓也。

\hypertarget{header-n52}{%
\subsection{王霸}\label{header-n52}}

国者,天下之利用也;人主者,天下之利埶也。得道以持之,则大安也,大荣
也,积美之源也;不得道以持之,则大危也,大累也,有之不如无之;及其綦也,
索为匹夫不可得也,齐愍、宋献是也。故人主天下之利埶也,然而不能自安也,安
之者必将道也。

故用国者,义立而王,信立而霸,权谋立而亡。三者明主之所谨择也,仁
人之所务白也。絜国以呼礼义,而无以害之,行一不义,杀一无罪,而得天下,仁
者不为也。擽然扶持心国,且若是其固也。之所与为之者,之人则举义士也;之所
以为布陈于国家刑法者,则举义法也;主之所极然帅群臣而首乡之者,则举义志也。
如是则下仰上以义矣,是綦定也;綦定而国定,国定而天下定。仲尼无置锥之地,
诚义乎志意,加义乎身行,箸之言语,济之日,不隐乎天下,名垂乎后世。今亦以
天下之显诸侯,诚义乎志意,加义乎法则度量,箸之以政事,案申重之以贵贱杀生,
使袭然终始犹一也。如是,则夫名声之部发于天地之间也,岂不如日月雷霆然矣哉!
故曰:以国齐义,一日而白,汤武是也。汤以亳,武王以鄗,皆百里之地也,天下
为一,诸侯为臣,通达之属,莫不从服,无它故焉,以义济矣。是所谓义立而
王也。

德虽未至也,义虽未济也,然而天下之理略奏矣,刑赏已诺信乎天下矣,臣下
晓然皆知其可要也。政令已陈,虽睹利败,不欺其民;约结已定,虽睹利败,不欺
其与。如是,则兵劲城固,敌国畏之;国一綦明,与国信之;虽在僻陋之国,威动
天下,五伯是也。非本政教也,非致隆高也,非綦文理也,非服人之心也,乡方略,
审劳佚,谨畜积,修战备,齺然上下相信,而天下莫之敢当。故齐桓、晋文、楚庄、
吴阖闾、越勾践,是皆僻陋之国也,威动天下,强殆中国,无它故焉,略信也。-
-是所谓信立而霸也。

絜国以呼功利,不务张其义,齐其信,唯利之求,内则不惮诈其民,而求小利
焉;外则不惮诈其与,而求大利焉,内不修正其所以有,然常欲人之有。如是,则
臣下百姓莫不以诈心待其上矣。上诈其下,下诈其上,则是上下析也。如是,则敌
国轻之,与国疑之,权谋日行,而国不免危削,綦之而亡,齐闵、薛公是也。故用
强齐,非以修礼义也,非以本政教也,非以一天下也,绵绵常以结引驰外为务。故
强、南足以破楚,西足以诎秦,北足以败燕,中足以举宋。及以燕赵起而攻之,若
振槁然,而身死国亡,为天下大戮,后世言恶,则必稽焉。是无它故焉,唯其不由
礼义,而由权谋也。

三者明主之所以谨择也,而仁人之所以务白也。善择者制人,不善择者人制之。

国者、天下之大器也,重任也,不可不善为择所而后错之,错险则危;不可不
善为择道然后道之,涂薉则塞;危塞则亡。彼国错者,非封焉之谓也,何法之道,
谁子之与也。故道王者之法,与王者之人为之,则亦王;道霸者之法,与霸者之人
为之,则亦霸;道亡国之法,与亡国之人为之,则亦亡。三者明主之所以谨择
也,而仁人之所以务白也。

故国者、重任也,不以积持之则不立。故国者,世所以新者也,是惮,惮、非
变也,改王改行也。故一朝之日也,一日之人也,然而厌焉有千岁之国,何也?曰:
援夫千岁之信法以持之也,安与夫千岁之信士为之也。人无百岁之寿,而有千岁之
信士,何也?曰:以夫千岁之法自持者,是乃千岁之信士矣。故与积礼义之君子为
之则王,与端诚信全之士为之则霸,与权谋倾覆之人为之则亡。三者明主之所
以谨择也,仁人之所以务白也。善择之者制人,不善择之者人制之。

彼持国者,必不可以独也,然则强固荣辱在于取相矣。身能相能,如是者王,
身不能,知恐惧而求能者,如是者强;身不能,不知恐惧而求能者,安唯便僻左右
亲比己者之用,如是者危削;綦之而亡。国者,巨用之则大,小用之则小;綦大而
王,綦小而亡,小巨分流者存。巨用之者,先义而后利,安不恤亲疏,不恤贵贱,
唯诚能之求,夫是之谓巨用之。小用之者,先利而后义,安不恤是非,不治曲直,
唯便僻亲比己者之用,夫是之谓小用之。巨用之者若彼,小用之者若此,小巨分流
者,亦一若彼,一若此也。故曰:``粹而王,驳而霸,无一焉而亡。''此之谓也。

国无礼则不正。礼之所以正国也,譬之:犹衡之于轻重也,犹绳墨之于曲直也,
犹规矩之于方圆也,既错之而人莫之能诬也。诗云:``如霜雪之将将,如日月之光
明,为之则存,不为则亡。''此之谓也。

国危则无乐君,国安则无忧民。乱则国危,治则国安。今君人者,急逐乐而缓
治国,岂不过甚矣哉!譬之是由好声色,而恬无耳目也,岂不哀哉!夫人之情,目
欲綦色,耳欲綦声,口欲綦味,鼻欲綦臭,心欲綦佚。此五綦者,人情之所必
不免也。养五綦者有具。无其具,则五綦者不可得而致也。万乘之国,可谓广大富
厚矣,加有治辨强固之道焉,若是则恬愉无患难矣,然后养五綦之具具也。故百乐
者,生于治国者也;忧患者,生于乱国者也。急逐乐而缓治国者,非知乐者也。故
明君者,必将先治其国,然后百乐得其中。闇君者,必将急逐乐而缓治国,故忧患
不可胜校也,必至于身死国亡然后止也,岂不哀哉!将以为乐,乃得忧焉;将以为
安,乃得危焉;将以为福,乃得死亡焉,岂不哀哉!于乎!君人者,亦可以察若言
矣。故治国有道,人主有职。若夫贯日而治详,一日而曲列之,是所使夫百吏官人
为也,不足以是伤游玩安燕之乐。若夫论一相以兼率之,使臣下百吏莫不宿道乡方
而务,是夫人主之职也。若是则一天下,名配尧禹。之主者,守至约而详,事至佚
而功,垂衣裳,不下簟席之上,而海内之人莫不愿得以为帝王。夫是之谓至约,乐
莫大焉。

人主者,以官人为能者也;匹夫者,以自能为能者也。人主得使人为之,匹夫
则无所移之。百亩一守,事业穷,无所移之也。今以一人兼听天下,日有余而治不
足者,使人为之也。大有天下,小有一国,必自为之然后可,则劳苦秏(卒页)莫甚
焉。如是,则虽臧获不肯与天子易埶业。以是县天下,一四海,何故必自为之?为
之者,役夫之道也,墨子之说也。论德使能而官施之者,圣王之道也,儒之所谨守
也。传曰:农分田而耕,贾分货而贩,百工分事而劝,士大夫分职而听,建国诸侯
之君分土而守,三公摠方而议,则天子共己而已矣。出若入若,天下莫不平均,莫
不治辨,是百王之所同也,而礼法之大分也。

百里之地,可以取天下。是不虚;其难者在人主之知之也。取天下者,非
负其土地而从之之谓也,道足以壹人而已矣。彼其人苟壹,则其土地奚去我而适它?
故百里之地,其等位爵服,足以容天下之贤士矣;其官职事业,足以容天下之能士
矣;循其旧法,择其善者而明用之,足以顺服好利之人矣。贤士一焉,能士官焉,
好利之人服焉,三者具而天下尽,无有是其外矣。故百里之地,足以竭埶矣。致忠
信,箸仁义,足以竭人矣。两者合而天下取,诸侯后同者先危。诗曰:``自西自东,
自南自北,无思不服。''一人之谓也。

羿、蜂门者,善服射者也;王良、造父者,善服驭者也。聪明君子者,善服人
者也。人服而埶从之,人不服而埶去之,故王者已于服人矣。故人主欲得善射
射远中微,则莫若羿、蜂门矣;欲得善驭及速致远,则莫若王良、造父矣。欲
得调壹天下,制秦楚,则莫若聪明君子矣。其用知甚简,其为事不劳,而功名致大,
甚易处而极可乐也。故明君以为宝,而愚者以为难。夫贵为天子,富有天下,名为
圣王,兼制人,人莫得而制也,是人情之所同欲也,而王者兼而有是者也。重色而
衣之,重味而食之,重财物而制之,合天下而君之,饮食甚厚,声乐甚大,台谢甚
高,园囿甚广,臣使诸侯,一天下,是又人情之所同欲也,而天子之礼制如是者也。
制度以陈,政令以挟,官人失要则死,公侯失礼则幽,四方之国,有侈离之德则必
灭,名声若日月,功绩如天地,天下之人应之如景向,是又人情之所同欲也,而王
者兼而有是者也。故人之情,口好味,而臭味莫美焉;耳好声,而声乐莫大焉;目
好色,而文章致繁,妇女莫众焉;形体好佚,而安重闲静莫愉焉;心好利,而谷禄
莫厚焉。合天下之所同愿兼而有之,睪牢天下而制之若制子孙,人苟不狂惑戆陋者,
其谁能睹是而不乐也哉!欲是之主,并肩而存;能建是之士,不世绝;千岁而不合,
何也?曰:人主不公,人臣不忠也。人主则外贤而偏举,人臣则争职而妒贤,是其
所以不合之故也。人主胡不广焉,无恤亲疏,无偏贵贱,惟诚能之求?若是,则人
臣轻职业让贤,而安随其后。如是,则舜禹还至,王业还起;功壹天下,名配舜禹,
物由有可乐,如是其美焉者乎!呜呼!君人者,亦可以察若言矣。杨朱哭衢涂,曰:
``此夫过举蹞步,而觉跌千里者夫!''哀哭之。此亦荣辱、安危、存亡之衢已,此
其为可哀,甚于衢涂。呜呼!哀哉!君人者,千岁而不觉也。

无国而不有治法,无国而不有乱法;无国而不有贤士,无国而不有罢士;无国
而不有愿民,无国而不有悍民;无国而不有美俗,无国而不有恶俗。两者并行而国
在,上偏而国安,在下偏而国危;上一而王,下一而亡。故其法治,其佐贤,其民
愿,其俗美,而四者齐,夫是之谓上一。如是则不战而胜,不攻而得,甲兵不劳而
天下服。故汤以亳,文王以鄗,皆百里之地也,天下为一,诸侯为臣,通达之属,
莫不从服,无它故焉,四者齐也。桀纣即厚有天下之埶,索为匹夫而不可得也,是
无它故焉,四者并亡也。故百王之法不同,若是所归者一也。

上莫不致爱其下,而制之以礼。上之于下,如保赤子,政令制度,所以接下之
人百姓,有不理者如豪末,则虽孤独鳏寡必不加焉。故下之亲上,欢如父母,可杀
而不可使不顺。君臣上下,贵贱长幼,至于庶人,莫不以是为隆正;然后皆内自省,
以谨于分。是百王之所同也,而礼法之枢要也。然后农分田而耕,贾分货而贩,百
工分事而劝,士大夫分职而听,建国诸侯之君分土而守,三公总方而议,则天子共
己而止矣。出若入若,天下莫不均平,莫不治辨。是百王之所同,而礼法之大分也。
若夫贯日而治平,权物而称用,使衣服有制,宫室有度,人徒有数,丧祭械用皆有
等宜,以是用挟于万物,尺寸寻丈,莫得不循乎制度数量然后行,则是官人使吏之
事也,不足数于大君子之前。故君人者,立隆政本朝而当,所使要百事者诚仁人也,
则身佚而国治,功大而名美,上可以王,下可以霸。立隆正本朝而不当,所使要百
事者非仁人也,则身劳而国乱,功废而名辱,社稷必危,是人君者之枢机也。故能
当一人而天下取,失当一人而社稷危。不能当一人,而能当千百人者,说无之有也。
既能当一人,则身有何劳而为?垂衣裳而天下定。故汤用伊尹,文王用吕尚,武王
用召公,成王用周公旦。卑者五伯,齐桓公闺门之内,县乐、奢泰、游抏之修,于
天下不见谓修,然九合诸侯,一匡天下,为五伯长,是亦无他故焉,知一政于管仲
也,是君人者之要守也。知者易为之兴力,而功名綦大。舍是而孰足为也?故古之
人,有大功名者,必道是者也。丧其国危其身者,必反是者也。故孔子曰:``知者
之知,固以多矣,有以守少,能无察乎?愚者之知,固以少矣,有以守多,能无狂
乎?''此之谓也。

治国者分已定,则主相臣下百吏,各谨其所闻,不务听其所不闻;各谨其所见,
不务视其所不见。所闻所见诚以齐矣。则虽幽闲隐辟,百姓莫敢不敬分安制,以化
其上,是治国之征也。

主道治近不治远,治明不治幽,治一不治二。主能治近则远者理,主能治明则
幽者化,主能当一则百事正。夫兼听天下,日有余而治不足者,如此也,是治之极
也。既能治近,又务治远;既能治明,又务见幽;既能当一,又务正百,是过者也,
过犹不及也。辟之是犹立直木而求其影之枉也。不能治近,又务治远;不能察明,
又务见幽;不能当一,又务正百,是悖者也。辟之是犹立枉木而求其影之直也。故
明主好要,而闇主好详;主好要则百事详,主好详则百事荒。君者、论一相,陈一
法,明一指,以兼覆之,兼照之,以观其盛者也。相者,论列百官之长,要百事之
听,以饰朝廷臣下百吏之分,度其功劳,论其庆赏,岁终奉其成功以效于君。当则
可,不当则废。故君人劳于索之,而休于使之。

用国者,得百姓之力者富,得百姓之死者强,得百姓之誉者荣。三得者具
而天下归之,三得者亡而天下去之;天下归之之谓王,天下去之之谓亡。汤武者,
修其道,行其义,兴天下同利,除天下同害,天下归之。故厚德音以先之,明礼义
以道之,致忠信以爱之,赏贤使能以次之,爵服赏庆以申重之,时其事,轻其任,
以调齐之,潢然兼覆之,养长之,如保赤子。生民则致宽,使民则綦理,辩政令制
度,所以接天下之人百姓,有非理者如豪末,则虽孤独鳏寡,必不加焉。是故百姓
贵之如帝,亲之如父母,为之出死断亡而不愉者,无它故焉,道德诚明,利泽诚厚
也。乱世则不然,污漫突盗以先之,权谋倾覆以示之,俳优、侏儒、妇女之请谒以
悖之,使愚诏知,使不肖临贤,生民则致贫隘,使民则极劳苦。是故,百姓贱之如
尪,恶之如鬼,日欲司间而相与投借之,去逐之。卒有寇难之事,又望百姓之为己
死,不可得也,说无以取之焉。孔子曰:``审吾所以适人,适人之所以来我也。''
此之谓也。

伤国者,何也?曰:以小人尚民而威,以非所取于民而巧,是伤国之大灾也。
大国之主也,而好见小利,是伤国。其于声色、台榭、园囿也,愈厌而好新,是伤
国。不好修正其所以有,啖啖常欲人之有,是伤国。三邪者在匈中,而又好以权谋
倾覆之人,断事其外,若是,则权轻名辱,社稷必危,是伤国者也。大国之主也,
不隆本行,不敬旧法,而好诈故,若是,则夫朝廷群臣,亦从而成俗于不隆礼义而
好倾覆也。朝廷群臣之俗若是,则夫众庶百姓亦从而成俗于不隆礼义而好贪利矣。
君臣上下之俗,莫不若是,则地虽广,权必轻;人虽众,兵必弱;刑罚虽繁,令不
下通。夫是之谓危国,是伤国者也。

儒者为之不然,必将曲辨:朝廷必将隆礼义而审贵贱,若是、则士大夫莫不敬
节死制者矣。百官则将齐其制度,重其官秩,若是、则百吏莫不畏法而遵绳矣。关
市几而不征,质律禁止而不偏,如是、则商贾莫不敦悫而无诈矣。百工将时斩伐,
佻其期日,而利其巧任,如是,则百工莫不忠信而不楛矣。县鄙则将轻田野之税,
省刀布之歛,罕举力役,无夺农时,如是、农夫莫不朴力而寡能矣。士大夫务节死
制,然而兵劲。百吏畏法循绳,然后国常不乱。商贾敦悫无诈,则商旅安,货通财,
而国求给矣。百工忠信而不楛,则器用巧便而财不匮矣。农夫朴力而寡能,则上不
失天时,下不失地利,中得人和,而百事不废。是之谓政令行,风俗美,以守则固,
以征则强,居则有名,动则有功。此儒之所谓曲辨也。

\hypertarget{header-n56}{%
\subsection{君道}\label{header-n56}}

有乱君,无乱国;有治人,无治法,羿之法非亡也,而羿不世中;禹之法犹存,而夏不世王。故法不能独立,类不能自行;得其人则存,失其人则亡。法者、治之端也;君子者、法之原也。故有君子,则法虽省,足以遍矣;无君子,则法虽具,失先后之施,不能应事之变,足以乱矣。不知法之义,而正法之数者,虽博临事必乱。故明主急得其人,而闇主急得其埶。急得其人,则身佚而国治,功大而名美,上可以王,下可以霸;不急得其人,而急得其埶,则身劳而国乱,功废而名辱,社稷必危。故君人者,劳于索之,而休于使之。书曰:``惟文王敬忌,一人以择。''此之谓也。

合符节,别契券者,所以为信也;上好权谋,则臣下百吏诞诈之人乘是而后欺。探筹、投钩者,所以为公也;上好曲私,则臣下百吏乘是而后偏。衡石称县者,所以为平也;上好覆倾,则臣下百吏乘是而后险。斗斛敦概者,所以为啧也;上好贪利,则臣下百吏乘是而后丰取刻与,以无度取于民。故械数者,治之流也,非治之原也;君子者,治之原也。官人守数,君子养原;原清则流清,原浊则流浊。故上好礼义,尚贤使能,无贪利之心,则下亦将綦辞让,致忠信,而谨于臣子矣。如是则虽在小民,不待合符节,别契券而信,不待探筹投钩而公,不待冲石称县而平,不待斗斛敦概而啧。故赏不用而民劝,罚不用而民服,有司不劳而事治,政令不烦而俗美。百姓莫敢不顺上之法,象上之志,而劝上之事,而安乐之矣。故借歛忘费,事业忘劳,寇难忘死,城郭不待饰而固,兵刃不待陵而劲,敌国不待服而诎,四海之民不待令而一,夫是之谓至平。诗曰:``王犹允塞,徐方既来。''此之谓也。

请问为人君?曰:以礼分施,均遍而不偏。请问为人臣?曰:以礼侍君,忠顺而不懈。请问为人父?曰:宽惠而有礼。请问为人子?曰:敬爱而致文。请问为人兄?曰:慈爱而见友。请问为人弟?曰:敬诎而不苟。请问为人夫?曰:致功而不流,致临而有辨。请问为人妻?曰:夫有礼则柔从听侍,夫无礼则恐惧而自竦也。此道也,偏立而乱,俱立而治,其足以稽矣。请问兼能之奈何?曰:审之礼也。古者先王审礼以方皇周浃于天下,动无不当也。故君子恭而不难,敬而不巩,贫穷而不约,富贵而不骄,并遇变态而不穷,审之礼也。故君子之于礼,敬而安之;其于事也,径而不失;其于人也,寡怨宽裕而无阿;其为身也,谨修饰而不危;其应变故也,齐给便捷而不惑;其于天地万物也,不务说其所以然,而致善用其材;其于百官之事伎艺之人也,不与之争能,而致善用其功;其待上也,忠顺而不懈;其使下也,均遍而不偏;其交游也,缘类而有义;其居乡里也,容而不乱。是故穷则必有名,达则必有功,仁厚兼覆天下而不闵,明达用天地理万变而不疑,血气和平,志意广大,行义塞于天地之间,仁智之极也。夫是之谓圣人;审之礼也。

请问为国?曰闻修身,未尝闻为国也。君者仪也,民者景也,仪正而景正。君者盘也,民者水也,盘圆而水圆。君者盂也,盂方而水方。君射则臣决。楚庄王好细腰,故朝有饿人。故曰:闻修身,未尝闻为国也。

君者,民之原也;原清则流清,原浊则流浊。故有社稷者而不能爱民,不能利民,而求民之亲爱己,不可得也。民不亲不爱,而求为己用,为己死,不可得也。民不为己用,不为己死,而求兵之劲,城之固,不可得也。兵不劲,城不固,而求敌之不至,不可得也。敌至而求无危削,不灭亡,不可得也。危削灭亡之情,举积此矣,而求安乐,是狂生者也。狂生者,不胥时而落。故人主欲强固安乐,则莫若反之民;欲附下一民,则莫若反之政;欲修政美俗,则莫若求其人。彼或蓄积而得之者不世绝。彼其人者,生乎今之世,而志乎古之道。以天下之王公莫好之也,然而是子独好之;以天下之民莫为之也,然而是子独为之。好之者贫,为之者穷,然而是子犹将为之也,不为少顷辍焉。晓然独明于先王之所以得之,所以失之,知国之安危臧否,若别白黑。是其人也,大用之,则天下为一,诸侯为臣;小用之,则威行邻敌;纵不能用,使无去其疆域,则国终身无故。故君人者,爱民而安,好士而荣,两者无一焉而亡。诗曰:``介人维藩,大师为垣。''此之谓也。

道者,何也?曰:君之所道也。君者,何也?曰:能群也。能群也者,何也?曰:善生养人者也,善班治人者也,善显设人者也,善藩饰人者也。善生养人者人亲之,善班治人者人安之,善显设人者人乐之,善藩饰人者人荣之。四统者俱,而天下归之,夫是之谓能群。不能生养人者,人不亲也;不能班治人者,人不安也;不能显设人者,人不乐也;不能藩饰人者,人不荣也。四统者亡,而天下去之,夫是之谓匹夫。故曰:道存则国存,道亡则国亡。省工贾,众农夫,禁盗贼,除奸邪:是所以生养之也。天子三公,诸侯一相,大夫擅官,士保职,莫不法度而公:是所以班治之也。论德而定次,量能而授官,皆使人载其事,而各得其所宜,上贤使之为三公,次贤使之为诸侯,下贤使之为士大夫:是所以显设之也。修冠弁衣裳,黼黻文章,琱琢刻镂,皆有等差:是所以藩饰之也。故由天子至于庶人也,莫不骋其能,得其志,安乐其事,是所同也;衣暖而食充,居安而游乐,事时制明而用足,是又所同也。若夫重色而成文章,重味而成珍备,是所衍也。圣王财衍,以明辨异,上以饰贤良而明贵贱,下以饰长幼而明亲疏。上在王公之朝,下在百姓之家,天下晓然皆知其所以为异也,将以明分达治而保万世也。故天子诸侯无靡费之用,士大夫无流淫之行,百吏官人无怠慢之事,众庶百姓无奸怪之俗,无盗贼之罪,其能以称义遍矣。故曰:治则衍及百姓,乱则不足及王公。此之谓也。

至道大形:隆礼至法则国有常,尚贤使能则民知方,纂论公察则民不疑,赏克罚偷则民不怠,兼听齐明则天下归之;然后明分职,序事业,材技官能,莫不治理,则公道达而私门塞矣,公义明而私事息矣:如是,则德厚者进而佞说者止,贪利者退而廉节者起。书曰:``先时者杀无赦,不逮时者杀无赦。''人习其事而固,人之百事,如耳目鼻口之不可以相借官也。故职分而民不慢,次定而序不乱,兼听齐明而百姓不留:如是,则臣下百吏至于庶人,莫不修己而后敢安止,诚能而后敢受职;百姓易俗,小人变心,奸怪之属莫不反悫:夫是之谓政教之极。故天子不视而见,不听而聪,不虑而知,不动而功,块然独坐而天下从之如一体,如四胑之从心:夫是之谓大形。诗曰:``温温恭人,维德之基。''此之谓也。

为人主者,莫不欲强而恶弱,欲安而恶危,欲荣而恶辱,是禹桀之所同也。要此三欲,辟此三恶,果何道而便?曰:在慎取相,道莫径是矣。故知而不仁,不可;仁而不知,不可;既知且仁,是人主之宝也,王霸之佐也。不急得,不知;得而不用,不仁。无其人而幸有其功,愚莫大焉。今人主有大患:使贤者为之,则与不肖者规之;使知者虑之,则与愚者论之;使修士行之,则与污邪之人疑之,虽欲成功,得乎哉!譬之,是犹立直木而恐其景之枉也,惑莫大焉!语曰:好女之色,恶者之孽也;公正之士,众人之痤也;修道之人,污邪之贼也。今使污邪之人,论其怨贼,而求其无偏,得乎哉!譬之,是犹立枉木而求其景之直也,乱莫大焉。

故古之人为之不然:其取人有道,其用人有法。取人之道,参之以礼;用人之法,禁之以等。行义动静,度之以礼;知虑取舍,稽之以成;日月积久,校之以功,故卑不得以临尊,轻不得以县重,愚不得以谋知,是以万举而不过也。故校之以礼,而观其能安敬也;与之举措迁移,而观其能应变也;与之安燕,而观其能无流慆也;接之以声色、权利、忿怒、患险,而观其能无离守也。彼诚有之者,与诚无之者,若白黑然,可诎邪哉!故伯乐不可欺以马,而君子不可欺以人,此明王之道也。

人主欲得善射射远中微者,县贵爵重赏以招致之。内不可以阿子弟,外不可以隐远人,能中是者取之;是岂不必得之之道也哉!虽圣人不能易也。欲得善驭及速致远者,一日而千里,县贵爵重赏以招致之。内不可以阿子弟,外不可以隐远人,能致是者取之;是岂不必得之之道也哉!虽圣人不能易也。欲治国驭民,调壹上下,将内以固城,外以拒难,治则制人,人不能制也;乱则危辱灭亡,可立而待也。然而求卿相辅佐,则独不若是其公也,案唯便嬖亲比己者之用也,岂不过甚矣哉!故有社稷者,莫不欲强,俄则弱矣;莫不欲安,俄则危矣;莫不欲存,俄则亡矣。古有万国,今有十数焉,是无他故,莫不失之是也。故明主有私人以金石珠玉,无私人以官职事业,是何也?曰:本不利于所私也。彼不能而主使之,则是主闇也;臣不能而诬能,则是臣诈也。主闇于上,臣诈于下,灭亡无日,俱害之道也。夫文王非无贵戚也,非无子弟也,非无便嬖也,倜然乃举太公于州人而用之,岂私之也哉!以为亲邪?则周姬姓也,而彼姜姓也;以为故邪?则未尝相识也;以为好丽邪?则夫人行年七十有二,(齿军)然而齿堕矣。然而用之者,夫文王欲立贵道,欲白贵名,以惠天下,而不可以独也。非于是子莫足以举之,故举是子而用之。于是乎贵道果立,贵名果白,兼制天下,立七十一国姬姓独居五十三人。周之子孙,苟非狂惑者,莫不为天下之显诸侯,如是者能爱人也。故举天下之大道,立天下之大功,然后隐其所怜所爱,其下犹足以为天下之显诸侯。故曰:唯明主为能爱其所爱,闇主则必危其所爱。此之谓也。

墙之外,目不见也;里之前,耳不闻也;而人主之守司,远者天下,近者境内,不可不略知也。天下之变,境内之事,有弛易齵差者矣,而人主无由知之,则是拘胁蔽塞之端也。耳目之明,如是其狭也;人主之守司,如是其广也;其中不可以不知也,如是其危也。然则人主将何以知之?曰:便嬖左右者,人主之所以窥远收众之门户牖向也,不可不早具也。故人主必将有便嬖左右足信者,然后可。其知惠足使规物,其端诚足使定物,然后可;夫是之谓国具。人主不能不有游观安燕之时,则不得不有疾病物故之变焉。如是,国者,事物之至也如泉原,一物不应,乱之端也。故曰:人主不可以独也。卿相辅佐,人主之基杖也,不可不早具也。故人主必将有卿相辅佐足任者,然后可。其德音足以填抚百姓,其知虑足以应待万变,然后可;夫是之谓国具。四邻诸侯之相与,不可以不相接也,然而不必相亲也,故人主必将有足使喻志决疑于远方者,然后可。其辩说足以解烦,其知虑足以决疑,其齐断足以距难,不还秩,不反君,然而应薄扞患,足以持社稷,然后可,夫是之谓国具。故人主无便嬖左右足信者,谓之闇;无卿相辅佐足任使者,谓之独;所使于四邻诸侯者非其人,谓之孤;孤独而晻,谓之危。国虽若存,古之人曰亡矣。诗曰:``济济多士,文王以宁。''此之谓也。

材人:愿悫拘录,计数纤啬,而无敢遗丧,是官人使吏之材也。修饬端正,尊法敬分,而无倾侧之心,守职修业,不敢损益,可传世也,而不可使侵夺,是士大夫官师之材也。知隆礼义之为尊君也,知好士之为美名也,知爱民之为安国也,知有常法之为一俗也,知尚贤使能之为长功也,知务本禁末之为多材也,知无与下争小利之为便于事也,知明制度,权物称用之为不泥也,是卿相辅佐之材也,未及君道也。能论官此三材者而无失其次,是谓人主之道也。若是则身佚而国治,功大而名美,上可以王,下可以霸,是人主之要守也。人主不能论此三材者,不知道此道,安值将卑埶出劳,并耳目之乐,而亲自贯日而治详,一日而曲辨之,虑与臣下争小察而綦偏能,自古及今,未有如此而不乱者也。是所谓视乎不可见,听乎不可闻,为乎不可成,此之谓也。

\hypertarget{header-n60}{%
\subsection{臣道}\label{header-n60}}

人臣之论:有态臣者,有篡臣者,有功臣者,有圣臣者。内不足使一民,外不足使距难,百姓不亲,诸侯不信;然而巧敏佞说,善取宠乎上,是态臣者也。上不忠乎君,下善取誉乎民,不恤公道通义,朋党比周,以环主图私为务,是篡臣者也。内足使以一民,外足使以距难,民亲之,士信之,上忠乎君,下爱百姓而不倦,是功臣者也。上则能尊君,下则能爱民,政令教化,刑下如影,应卒遇变,齐给如响,推类接誉,以待无方,曲成制象,是圣臣者也。故用圣臣者王,用功臣者强,用篡臣者危,用态臣者亡。态臣用则必死,篡臣用则必危,功臣用则必荣,圣臣用则必尊。故齐之苏秦,楚之州侯,秦之张仪,可谓态臣者也。韩之张去疾,赵之奉阳,齐之孟尝,可谓篡臣也。齐之管仲,晋之咎犯,楚之孙叔敖,可谓功臣矣。殷之伊尹,周之太公,可谓圣臣矣。是人臣之论也,吉凶贤不肖之极也。必谨志之!而慎自为择取焉,足以稽矣。

从命而利君谓之顺,从命而不利君谓之谄;逆命而利君谓之忠,逆命而不利君谓之篡;不恤君之荣辱,不恤国之臧否,偷合苟容以持禄养交而已耳,谓之国贼。君有过谋过事,将危国家陨社稷之惧也;大臣父兄,有能进言于君,用则可,不用则去,谓之谏;有能进言于君,用则可,不用则死,谓之争;有能比知同力,率群臣百吏而相与强君挢君,君虽不安,不能不听,遂以解国之大患,除国之大害,成于尊君安国,谓之辅;有能抗君之命,窃君之重,反君之事,以安国之危,除君之辱,功伐足以成国之大利,谓之拂。故谏争辅拂之人,社稷之臣也,国君之宝也,明君之所尊厚也,而闇主惑君以为己贼也。故明君之所赏,闇君之所罚也;闇君之所赏,明君之所杀也。伊尹箕子可谓谏矣,比干子胥可谓争矣,平原君之于赵可谓辅矣,信陵君之于魏可谓拂矣。传曰:``从道不从君。''此之谓也。故正义之臣设,则朝廷不颇;谏争辅拂之人信,则君过不远;爪牙之士施,则仇雠不作;边境之臣处,则疆垂不丧,故明主好同而闇主好独,明主尚贤使能而飨其盛,闇主妒贤畏能而灭其功,罚其忠,赏其贼,夫是之谓至闇,桀纣所以灭也。

事圣君者,有听从无谏争;事中君者,有谏争无谄谀;事暴君者,有补削无挢拂。迫胁于乱时,穷居于暴国,而无所避之,则崇其美,扬其善,违其恶,隐其败,言其所长,不称其所短,以为成俗。诗曰:``国有大命,不可以告人,妨其躬身。''此之谓也。

恭敬而逊,听从而敏,不敢有以私决择也,不敢有以私取与也,以顺上为志,是事圣君之义也。忠信而不谀,谏争而不谄,挢然刚折端志而无倾侧之心,是案曰是,非案曰非,是事中君之义也。调而不流,柔而不屈,宽容而不乱,晓然以至道而无不调和也,而能化易,时关内之,是事暴君之义也。若驭朴马,若养赤子,若食餧人。故因其惧也而改其过,因其忧也而辨其故,因其喜也而入其道,因其怒也而除其怨,曲得所谓焉。书曰:``从命而不拂,微谏而不倦,为上则明,为下则逊。''此之谓也。

事人而不顺者,不疾者也;疾而不顺者,不敬者也;敬而不顺者,不忠者也;忠而不顺者,无功者也;有功而不顺者,无德者也。故无德之为道也,伤疾、堕功、灭苦,故君子不为也。

有大忠者,有次忠者,有下忠者,有国贼者:以德覆君而化之,大忠也;以德调君而辅之,次忠也;以是谏非而怒之,下忠也;不恤君之荣辱,不恤国之臧否,偷合苟容以持禄养交而已耳,国贼也。若周公之于成王也,可谓大忠矣;若管仲之于桓公,可谓次忠矣;若子胥之于夫差,可谓下忠矣;若曹触龙之于纣者,可谓国贼矣。

仁者必敬人。凡人非贤,则案不肖也。人贤而不敬,则是禽兽也;人不肖而不敬,则是狎虎也。禽兽则乱,狎虎则危,灾及其身矣。诗曰:``不敢暴虎,不敢冯河。人知其一,莫知其它。战战兢兢如临深渊,如履薄冰。''此之谓也。故仁者必敬人。敬人有道,贤者则贵而敬之,不肖者则畏而敬之;贤者则亲而敬之,不肖者则疏而敬之。其敬一也,其情二也。若夫忠信端悫,而不害伤,则无接而不然,是仁人之质也。忠信以为质,端悫以为统,礼义以为文,伦类以为理,喘而言,臑而动,而一可以为法则。诗曰:``不僭不贼,鲜不为则。''此之谓也。

恭敬、礼也;调和、乐也;谨慎、利也;斗怒、害也。故君子安礼乐利,谨慎而无斗怒,是以百举而不过也。小人反是。

通忠之顺,权险之平,祸乱之从声,三者非明主莫之能知也。争然后善,戾然后功,生死无私,致忠而公,夫是之谓通忠之顺,信陵君似之矣。夺然后义,杀然后仁,上下易位然后贞,功参天地,泽被生民,夫是之谓权险之平,汤武是也。过而通情,和而无经,不恤是非,不论曲宜,偷合苟容,迷乱狂生,夫是之谓祸乱之从声,飞廉恶来是也。传曰:``斩而齐,枉而顺,不同而一。''诗曰:``受小球大球,为下国缀旒。''此之谓也。

\hypertarget{header-n64}{%
\subsection{致士}\label{header-n64}}

衡听、显幽、重明、退奸、进良之术:朋党比周之誉,君子不听;残贼加累之
谮,君子不用;隐忌雍蔽之人,君子不近;货财禽犊之请,君子不许。凡流言、流说、流事、流谋、流誉、流愬,不官而衡至者,君子慎之,闻听而明誉之,定其当
而当,然后士其刑赏而还与之;如是则奸言、奸说、奸事、奸谋、奸誉、奸愬,莫
之试也;忠言、忠说、忠事、忠谋、忠誉、忠愬、莫不明通,方起以尚尽矣。夫是
之谓衡听、显幽、重明、退奸、进良之术。

川渊深而鱼鳖归之,山林茂而禽兽归之,刑政平而百姓归之,礼义备而君子归
之。故礼及身而行修,义及国而政明,能以礼挟而贵名白,天下愿,令行禁止,王
者之事毕矣。诗曰:``惠此中国,以绥四方。''此之谓也。川渊者,鱼龙之居也,
山林者、鸟兽之居也,国家者、士民之居也。川渊枯、则鱼龙去之,山林险,则鸟
兽去之,国家失政、则士民去之。无土则人不安居,无人则土不守,无道法则人不
至,无君子则道不举。故土之与人也,道之与法也者,国家之本作也。君子也者,
道法之摠要也,不可少顷旷也。得之则治,失之则乱;得之则安,失之则危;得之
则存,失之则亡,故有良法而乱者有之矣,有君子而乱者,自古及今,未尝闻也,
传曰:``治生乎君子,乱生于小人。''此之谓也。

得众动天。美意延年。诚信如神,夸诞逐魂。

人主之害,不在乎不言用贤,而在乎不诚必用贤。夫言用贤者,口也;却贤者,
行也,口行相反,而欲贤者之至,不肖者之退也,不亦难乎!夫耀蝉者,务在明其
火,振其树而已;火不明,虽振其树,无益也。今人主有能明其德者,则天下归之,
若蝉之归明火也。

临事接民,而以义变应,宽裕而多容,恭敬以先之,政之始也。然后中和察断
以辅之,政之隆也。然后进退诛赏之,政之终也。故一年与之始,三年与之终。用
其终为始,则政令不行,而上下怨疾,乱所以自作也。书曰:``义刑义杀;勿庸以
即,女惟曰:未有顺事。''言先教也。

程者、物之准也,礼者、节之准也;程以立数,礼以定伦;德以叙位,能以授
官。凡节奏欲陵,而生民欲宽;节奏陵而文,生民宽而安;上文下安,功名之极也,
不可以加矣。

君者、国之隆也,父者、家之隆也。隆一而治,二而乱。自古及今,未有二隆
争重,而能长久者。

师术有四而博习不与焉:尊严而惮,可以为师;耆艾而信,可以为师;诵
说而不陵不犯,可以为师;知微而论,可以为师:故师术有四而博习不与焉。
水深而回,树落则粪本,弟子通利则思师。诗曰:``无言不雠,无德不报。''此之
谓也。

赏不欲僭,刑不欲滥。赏僭则利及小人,刑滥则害及君子。若不幸而过,宁僭
勿滥。与其害善,不若利淫。

\hypertarget{header-n68}{%
\subsection{议兵}\label{header-n68}}

临武君与孙卿子议兵于赵孝成王前,王曰:请问兵要?

临武君对曰:上得天时,下得地利,观敌之变动,后之发,先之至,此用兵之要术也。

孙卿子曰:不然!臣所闻古之道,凡用兵攻战之本,在乎壹民。弓矢不调,则羿不能以中微;六马不和,则造父不能以致远;士民不亲附,则汤武不能以必胜也。故善附民者,是乃善用兵者也。故兵要在乎善附民而已。

临武君曰:不然。兵之所贵者埶利也,所行者变诈也。善用兵者,感忽悠闇,莫知其所从出。孙吴用之无敌于天下,岂必待附民哉!

孙卿子曰:不然。臣之所道,仁者之兵,王者之志也。君之所贵,权谋埶利也;所行,攻夺变诈也;诸侯之事也。仁人之兵,不可诈也;彼可诈者,怠慢者也,路亶者也,君臣上下之间,涣然有离德者也。故以桀诈桀,犹巧拙有幸焉。以桀诈尧,譬之:若以卵投石,以指挠沸;若赴水火,入焉焦没耳。故仁人上下,百将一心,三军同力;臣之于君也,下之于上也,若子之事父,弟之事兄,若手臂之扞头目而覆胸腹也,诈而袭之,与先惊而后击之,一也。且仁人之用十里之国,则将有百里之听;用百里之国,则将有千里之听;用千里之国,则将有四海之听,必将聪明警戒和传而一。故仁人之兵,聚则成卒,散则成列,延则若莫邪之长刃,婴之者断;兑则若莫邪之利锋,当之者溃,圜居而方止,则若盘石然,触之者角摧,案角鹿埵陇种东笼而退耳。且夫暴国之君,将谁与至哉?彼其所与至者,必其民也,而其民之亲我欢若父母,其好我芬若椒兰,彼反顾其上,则若灼黥,若雠仇;人之情,虽桀跖,岂又肯为其所恶,贼其所好者哉!是犹使人之子孙自贼其父母也,彼必将来告之,夫又何可诈也!故仁人用国日明,诸侯先顺者安,后顺者危,虑敌之者削,反之者亡。诗曰:``武王载发,有虔秉钺;如火烈烈,则莫我敢遏。''此之谓也。

孝成王、临武君曰:善!请问王者之兵,设何道何行而可?

孙卿子曰:凡在大王,将率末事也。臣请遂道王者诸侯强弱存亡之效,安危之埶:君贤者其国治,君不能者其国乱;隆礼贵义者其国治,简礼贱义者其国乱;治者强,乱者弱,是强弱之本也。上足卬则下可用也,上不卬则下不可用也;下可用则强,下不可用则弱,是强弱之常也。隆礼效功,上也;重禄贵节,次也;上功贱节,下也,是强弱之凡也。好士者强,不好士者弱;爱民者强,不爱民者弱;政令信者强,政令不信者弱;民齐者强,民不齐者弱;赏重者强,赏轻者弱;刑威者强,刑侮者弱;械用兵革攻完便利者强,械用兵革窳楛不便利者弱。重用兵者强,轻用兵者弱;权出一者强,权出二者弱,是强弱之常也。

齐人隆技击,其技也,得一首者,则赐赎锱金,无本赏矣。是事小敌毳,则偷可用也,事大敌坚,则涣然离耳。若飞鸟然,倾侧反复无日,是亡国之兵也,兵莫弱是矣。是其去赁市佣而战之几矣。

魏氏之武卒,以度取之,衣三属之甲,操十二石之弩,负服矢五十个,置戈其上,冠胄带剑,赢三日之粮,日中而趋百里,中试则复其户,利其田宅,是数年而衰,而未可夺也,改造则不易周也,是故地虽大,其税必寡,是危国之兵也。

秦人其生民郏阨,其使民也酷烈,劫之以埶,隐之以阨,忸之以庆赏,酋之以刑罚,使天下之民,所以要利于上者,非斗无由也。阨而用之,得而后功之,功赏相长也,五甲首而隶五家,是最为众强长久,多地以正,故四世有胜,非幸也,数也。

故齐之技击,不可以遇魏氏之武卒;魏氏之武卒,不可以遇秦之锐士;秦之锐士,不可以当桓文之节制;桓文之节制,不可以敌汤武之仁义;有遇之者,若以焦熬投石焉。兼是数国者,皆干赏蹈利之兵也,佣徒鬻卖之道也,未有贵上安制綦节之理也。诸侯有能微妙之以节,则作而兼殆之耳。故招近募选,隆埶诈,尚功利,是渐之也;礼义教化,是齐之也。故以诈遇诈,犹有巧拙焉;以诈遇齐,辟之犹以锥刀堕太山也,非天下之愚人莫敢试。故王者之兵不试。汤武之诛桀纣也,拱挹指麾,而强暴之国莫不趋使,诛桀纣若诛独夫。故泰誓曰:``独夫纣。''此之谓也。故兵大齐则制天下,小齐则治邻敌。若夫招近募选,隆埶诈,尚功利之兵,则胜不胜无常,代翕代张,代存代亡,相为雌雄耳矣。夫是之谓盗兵,君子不由也。

故齐之田单,楚之庄蹻,秦之卫鞅,燕之缪虮,是皆世俗所谓善用兵者也,是其巧拙强弱,则未有以相君也。若其道一也,未及和齐也;掎契司诈,权谋倾覆,未免盗兵也。齐桓、晋文、楚庄、吴阖闾、越勾践是皆和齐之兵也,可谓入其域矣,然而未有本统也,故可以霸而不可以王;是强弱之效也。

孝成王、临武君曰善!请问为将?

孙卿子曰:知莫大乎弃疑,行莫大乎无过,事莫大乎无悔,事至无悔而止矣,成不可必也。故制号政令欲严以威,庆赏刑罚欲必以信,处舍收藏欲周以固,徙举进退欲安以重,欲疾以速;窥敌观变欲潜以深,欲伍以参;遇敌决战必道吾所明,无道吾所疑:夫是之谓六术。无欲将而恶废,无急胜而忘败,无威内而轻外,无见利而不顾其害,凡虑事欲孰而用财欲泰:夫是之谓五权。所以不受命于主有三:可杀而不可使处不完,可杀而不可使击不胜,可杀而不可使欺百姓:夫是之谓三至。凡受命于主而行三军,三军既定,百官得序,群物皆正,则主不能喜,敌不能怒:夫是之谓至臣。虑必先事,而申之以敬,慎终如始,终始如一:夫是之谓大吉。凡百事之成也,必在敬之;其败也,必在慢之。故敬胜怠则吉,怠胜敬则灭;计胜欲则从,欲胜计则凶。战如守,行如战,有功如幸,敬谋无圹,敬事无圹,敬吏无圹,敬众无圹,敬敌无圹:夫是之谓五无圹。谨行此六术、五权、三至,而处之以恭敬无圹,夫是之谓天下之将,则通于神明矣。

临武君曰:善!请问王者之军制?

孙卿子曰:将死鼓,御死辔,百吏死职,士大夫死行列。闻鼓声而进,闻金声而退,顺命为上,有功次之;令不进而进,犹令不退而退也,其罪惟均。不杀老弱,不猎禾稼,服者不禽,格者不舍,奔命者不获。凡诛,非诛其百姓也,诛其乱百姓者也;百姓有扞其贼,则是亦贼也。以故顺刃者生,苏刃者死,奔命者贡。微子开封于宋,曹触龙断于军,殷之服民,所以养生之者也,无异周人。故近者歌讴而乐之,远者竭蹙而趋之,无幽闲辟陋之国,莫不趋使而安乐之,四海之内若一家,通达之属莫不从服,夫是之谓人师。诗曰:``自西自东,自南自北,无思不服。''此之谓也。王者有诛而无战,城守不攻,兵格不击,上下相喜则庆之,不屠城,不潜军,不留众,师不越时。故乱者乐其政,不安其上,欲其至也。

临武君曰:善!

陈嚣问孙卿子曰:先生议兵,常以仁义为本;仁者爱人,义者循理,然则又何以兵为?凡所为有兵者,为争夺也。

孙卿子曰:非汝所知也!彼仁者爱人,爱人故恶人之害之也;义者循理,循理故恶人之乱之也。彼兵者所以禁暴除害也,非争夺也。故仁者之兵,所存者神,所过者化,若时雨之降,莫不说喜。是以尧伐驩兜,舜伐有苗,禹伐共工,汤伐有夏,文王伐崇,武王伐纣,此四帝两王,皆以仁义之兵,行于天下也。故近者亲其善,远方慕其德,兵不血刃,远迩来服,德盛于此,施及四极。诗曰:``淑人君子,其仪不忒,其仪不忒,正是四国。''此之谓也。

李斯问孙卿子曰:秦四世有胜,兵强海内,威行诸侯,非以仁义为之也,以便从事而已。

孙卿子曰:非汝所知也!汝所谓便者,不便之便也;吾所谓仁义者,大便之便也。彼仁义者,所以修政者也;政修则民亲其上,乐其君,而轻为之死。故曰:凡在于军,将率末事也。秦四世有胜,諰諰然常恐天下之一合而轧己也,此所谓末世之兵,未有本统也。故汤之放桀也,非其逐之鸣条之时也;武王之诛纣也,非以甲子之朝而后胜之也,皆前行素修也,所谓仁义之兵也。今女不求之于本,而索之于末,此世之所以乱也。

礼者、治辨之极也,强固之本也,威行之道也,功名之总也,王公由之所以得天下也,不由所以陨社稷也。故坚甲利兵不足以为胜,高城深池不足以为固,严令繁刑不足以为威。由其道则行,不由其道则废。

楚人鲛革犀兕以为甲,鞈坚如金石;宛钜铁矛,惨如蜂虿,轻利僄遫,卒如飘风;然而兵殆于垂沙,唐蔑死。庄蹻起,楚分而为三四,是岂无坚甲利兵也哉!其所以统之者非其道故也。汝颍以为险,江汉以为池,限之以邓林,缘之以方城;然而秦师至,而鄢郢举,若振槁然,是岂无固塞隘阻也哉!其所以统之者非其道故也。纣刳比干,囚箕子,为炮烙刑,杀戮无时,臣下懔然莫必其命,然而周师至,而令不行乎下,不能用其民,是岂令不严,刑不繁也哉!其所以统之者非其道故也。

古之兵,戈矛弓矢而已矣,然而敌国不待试而诎;城郭不辨,沟池不抇,固塞不树,机变不张;然而国晏然不畏外而固者,无它故焉,明道而钧分之,时使而诚爱之,下之和上也如影向,有不由令者,然后俟之以刑。故刑一人而天下服,罪人不邮其上,知罪之在己也。是故刑罚省而威流,无它故焉,由其道故也。古者帝尧之治天下也,盖杀一人,刑二人,而天下治。传曰:``威厉而不试,刑错而不用。''此之谓也。

凡人之动也,为赏庆为之,则见害伤焉止矣。故赏庆、刑罚、埶诈,不足以尽人之力,致人之死。为人主上者也,其所以接下之百姓者,无礼义忠信,焉虑率用赏庆、刑罚、埶诈,除阨其下,获其功用而已矣。大寇则至,使之持危城则必畔,遇敌处战则必北,劳苦烦辱则必奔,霍焉离耳,下反制其上。故赏庆、刑罚、埶诈之为道者,佣徒鬻卖之道也,不足以合大众,美国家,故古之人羞而不道也。故厚德音以先之,明礼义以道之,致忠信以爱之,尚贤使能以次之,爵服庆赏以申之,时其事,轻其任,以调齐之,长养之,如保赤子。政令以定,风俗以一,有离俗不顺其上,则百姓莫不敦恶,莫不毒孽,若祓不祥;然后刑于是起矣。是大刑之所加也,辱孰大焉!将以为利邪?则大刑加焉,身苟不狂惑戆陋,谁睹是而不改也哉!然后百姓晓然皆知循上之法,像上之志,而安乐之。于是有能化善、修身、正行、积礼义、尊道德,百姓莫不贵敬,莫不亲誉;然后赏于是起矣。是高爵丰禄之所加也,荣孰大焉!将以为害邪?则高爵丰禄以持养之;生民之属,孰不愿也!雕雕焉县贵爵重赏于其前,县明刑大辱于其后,虽欲无化,能乎哉!故民归之如流水,所存者神,所为者化。××××之属为之化而顺,暴悍勇力之属为之化而愿,旁辟曲私之属为之化而公,矜纠收缭之属为之化而调,夫是之谓大化至一。诗曰:``王犹允塞,徐方既来。''此之谓也。

凡兼人者有三术:有以德兼人者,有以力兼人者,有以富兼人者。彼贵我名声,美我德行,欲为我民,故辟门除涂,以迎吾入。因其民,袭其处,而百姓皆安。立法施令,莫不顺比。是故得地而权弥重,兼人而兵俞强:是以德兼人者也。非贵我名声也,非美我德行也,彼畏我威,劫我埶,故民虽有离心,不敢有畔虑,若是则戎甲俞众,奉养必费。是故得地而权弥轻,兼人而兵俞弱:是以力兼人者也。非贵我名声也,非美我德行也,用贫求富,用饥求饱,虚腹张口,来归我食。若是,则必发夫掌窌之粟以食之,委之财货以富之,立良有司以接之,已期三年,然后民可信也。是故得地而权弥轻,兼人而国俞贫:是以富兼人者也。故曰:以德兼人者王,以力兼人者弱,以富兼人者贫,古今一也。

兼幷易能也,唯坚凝之难焉。齐能幷宋,而不能凝也,故魏夺之。燕能幷齐,而不能凝也,故田单夺之。韩之上地,方数百里,完全富足而趋赵,赵不能凝也,故秦夺之。故能幷之,而不能凝,则必夺;不能幷之,又不能凝其有,则必亡。能凝之,则必能幷之矣。得之则凝,兼幷无强。古者汤以薄,武王以滈,皆百里之地也,天下为一,诸侯为臣,无他故焉,能凝之也。故凝士以礼,凝民以政;礼修而士服,政平而民安;士服民安,夫是之谓大凝。以守则固,以征则强,令行禁止,王者之事毕矣。

\hypertarget{header-n72}{%
\subsection{强国}\label{header-n72}}

刑范正,金锡美,工冶巧,火齐得,剖刑而莫邪已。然而不剥脱,不砥厉,则
不可以断绳。剥脱之,砥厉之,则劙盘盂,刎牛马,忽然耳。彼国者,亦强国之剖
刑已。然而不教诲,不调一,则入不可以守,出不可以战。教诲之,调一之,则兵
劲城固,敌国不敢婴也。彼国者亦有砥厉,礼义节奏是也。故人之命在天,国之命
在礼。人君者,隆礼尊贤而王,重法爱民而霸,好利多诈而危,权谋倾覆幽险而亡。

威有三:有道德之威者,有暴察之威者,有狂妄之威者此三威者,不可不
孰察也。礼义则修,分义则明,举错则时,爱利则形。如是,百姓贵之如帝,高之
如天,亲之如父母,畏之如神明。故赏不用而民劝,罚不用而威行,夫是之谓道德
之威。礼乐则不修,分义则不明,举错则不时,爱利则不形;然而其禁暴也察,其
诛不服也审,其刑罚重而信,其诛杀猛而必,黭然而雷击之,如墙厌之。如是,百
姓劫则致畏,嬴则敖上,执拘则{[}最{]}聚,得间则散,敌中则夺,非劫之以形埶,非
振之以诛杀,则无以有其下,夫是之谓暴察之威。无爱人之心,无利人之事,而日
为乱人之道,百姓讙敖,则从而执缚之,刑灼之,不和人心。如是,下比周贲溃以
离上矣,倾覆灭亡,可立而待也,夫是之谓狂妄之威。此三威者,不可不孰察
也。道德之威成乎安强,暴察之威成乎危弱,狂妄之威成乎灭亡也。

公孙子曰:子发将西伐蔡,克蔡,获蔡侯,归致命曰:``蔡侯奉其社稷,而归
之楚;舍属二三子而治其地。''既,楚发其赏,子发辞曰:``发诫布令而敌退,是
主威也;徙举相攻而敌退,是将威也;合战用力而敌退,是众威也。臣舍不宜以众
威受赏。''

讥之曰:子发之致命也恭,其辞赏也固。夫尚贤使能,赏有功,罚有罪,非独
一人为之也,彼先王之道也,一人之本也,善善恶恶之应也,治必由之,古今一也。
古者明主之举大事,立大功也,大事已博,大功已立,则君享其成,群臣享其功,
士大夫益爵,官人益秩,庶人益禄。是以为善者劝,为不善者沮,上下一心,三军
同力,是以百事成,而功名大也。今子发独不然:反先王之道,乱楚国之法,堕兴
功之臣,耻受赏之属,无僇乎族党,而抑卑其后世,案独以为私廉,岂不过甚矣哉!
故曰:子发之致命也恭,其辞赏也固。

荀卿子说齐相曰:处胜人之埶,行胜人之道,天下莫忿,汤武是也。处胜人之
埶,不以胜人之道,厚于有天下之埶,索为匹夫不可得也,桀纣是也。然则得胜人
之埶者,其不如胜人之道远矣!夫主相者,胜人以埶也,是为是,非为非,能为能,
不能为不能,并己之私欲,必以道,夫公道通义之可以相兼容者,是胜人之道也。
今相国上则得专主,下则得专国,相国之于胜人之埶,亶有之矣。然则胡不驱此胜
人之埶,赴胜人之道,求仁厚明通之君子而托王焉,与之参国政,正是非!如是,
则国孰敢不为义矣!君臣上下,贵贱长少,至于庶人,莫不为义,则天下孰不欲合
义矣!贤士愿相国之朝,能士愿相国之官,好利之民莫不愿以齐为归,是一天下也。
相国舍是而不为,案直为是世俗之所以为,则女主乱之宫,诈臣乱之朝,贪吏乱之
官,众庶百姓皆以争夺贪利为俗,曷若是而可以持国乎?今巨楚县吾前,大燕(鱼酋)
吾后,劲魏钩吾右,西壤之不绝若绳,楚人则乃有襄贲开阳以临吾左,是一国作谋,
则三国必起而乘我。如是,则齐必断而为四、三,国若假城然耳,必为天下大笑。
曷若两者孰足为也!夫桀纣,圣王之后子孙也,有天下者之世也,埶籍之所存,天
下之宗室也,土地之大,封内千里,人之众数以亿万,俄而天下倜然举去桀纣而奔
汤武,反然举恶桀纣而贵汤武。是何也?夫桀纣何失?而汤武何得也?曰:是无它
故焉,桀纣者善为人所恶也,而汤武者善为人所好也。人之所恶何也?曰:污漫、
争夺、贪利是也。人之所好者何也?曰:礼义、辞让、忠信是也。今君人者,譬称
比方则欲自并乎汤武,若其所以统之,则无以异于桀纣,而求有汤武之功名,可乎?
故凡得胜者,必与人也;凡得人者,必与道也。道也者,何也?礼义、辞让、忠信
是也。故自四五万而往者,强胜非众之力也,隆在信矣。自数百里而往者,安固非
大之力也,隆在修政矣。今已有数万之众者也,陶诞比周以争与;已有数百里之国
者也,污漫突盗以争地;然则是弃己之所安强,而争己之所以危弱也;损己之所不
足,以重己之所有余。若是其悖缪也,而求有汤武之功名,可乎!辟之,是犹伏而
咶天,救经而引其足也。说必不行矣,愈务而愈远。为人臣者,不恤己行之不行,
苟得利而已矣,是渠冲入穴而求利也,是仁人之所羞而不为也。故人莫贵乎生,莫
乐乎安;所以养生安乐者,莫大乎礼义。人知贵生乐安而弃礼义,辟之,是犹欲寿
而歾颈也,愚莫大焉。故君人者,爱民而安,好士而荣,两者亡一焉而亡。诗曰:
``价人维藩,大师维垣。''此之谓也。

力术止,义术行,曷谓也?曰:秦之谓也。威强乎汤武,广大乎舜禹,然而忧
患不可胜校也。諰諰然常恐天下之一合而轧己也,此所谓力术止也。曷谓乎威强乎
汤武?汤武也者,乃能使说己者使耳。今楚、父死焉,国举焉,负三王之庙,而辟
于陈蔡之间,视可司间,案欲剡其胫而以蹈秦之腹,然而秦使左案左,使右案右,
是乃使雠人役也;此所谓威强乎汤武也。曷谓广大乎舜禹也?曰:古者百王之一天
下,臣诸侯也,未有过封内千里者也。今秦南乃有沙羡与俱,是乃江南也。北与胡
貉为邻,西有巴戎,东在楚者乃界于齐,在韩者踰常山乃有临虑,在魏者乃据圉津
即去大梁百有二十里耳!其在赵者剡然有苓而据松柏之塞,负西海而固常山,
是地遍天下也。威动海内,强殆中国,然而忧患不可胜校也,諰諰然常恐天下之一
合而轧己也;此所谓广大乎舜禹也。然则奈何?曰:节威反文,案用夫端诚信全之
君子治天下焉,因与之参国政,正是非,治曲直,听咸阳,顺者错之,不顺者而后
诛之。若是,则兵不复出于塞外,而令行于天下矣。若是,则虽为之筑明堂于塞外
而朝诸侯,殆可矣。假今之世,益地不如益信之务也。

应侯问孙卿子曰:入秦何见?

孙卿子曰:其固塞险,形埶便,山林川谷美,天材之利多,是形胜也。入境,
观其风俗,其百姓朴,其声乐不流污,其服不佻,甚畏有司而顺,古之民也。及都
邑官府,其百吏肃然,莫不恭俭、敦敬、忠信而不楛,古之吏也。入其国,观其士
大夫,出于其门,入于公门;出于公门,归于其家,无有私事也;不比周,不朋党,
倜然莫不明通而公也,古之士大夫也。观其朝廷,其朝闲,听决百事不留,恬然如
无治者,古之朝也。故四世有胜,非幸也,数也。是所见也。故曰:佚而治,约而
详,不烦而功,治之至也,秦类之矣。虽然,则有其諰矣。兼是数具者而尽有之,
然而县之以王者之功名,则倜倜然其不及远矣!是何也?则其殆无儒邪!故曰粹而
王,驳而霸,无一焉而亡。此亦秦之所短也。

积微:月不胜日,时不胜月,岁不胜时。凡人好敖慢小事,大事至然后兴之务
之,如是,则常不胜夫敦比于小事者矣。是何也?则小事之至也数,其县日也博,
其为积也大;大事之至也希,其县日也浅,其为积也小。故善日者王,善时者霸,
补漏者危,大荒者亡。故王者敬日,霸者敬时,仅存之国危而后戚之。亡国至亡而
后知亡,至死而后知死,亡国之祸败,不可胜悔也。霸者之善箸焉,可以时托也;
王者之功名,不可胜日志也。财物货宝以大为重,政教功名反是能积微者速成。
诗曰:``德輶如毛,民鲜克举之。''此之谓也。

凡奸人之所以起者,以上之不贵义,不敬义也。夫义者,所以限禁人之为恶与
奸者也。今上不贵义,不敬义,如是,则天下之人百姓,皆有弃义之志,而有趋奸
之心矣,此奸人之所以起也。且上者下之师也,夫下之和上,譬之犹响之应声,影
之像形也。故为人上者,不可不顺也。夫义者,内节于人,而外节于万物者也;上
安于主,而下调于民者也;内外上下节者,义之情也。然则凡为天下之要,义为本,
而信次之。古者禹汤本义务信而天下治,桀纣弃义倍信而天下乱。故为人上者,必
将慎礼义,务忠信,然后可。此君人者之大本也。堂上不粪,则郊草不瞻旷芸;白
刃扞乎胸,则目不见流矢;拔戟加乎首,则十指不辞断;非不以此为务也,疾养缓
急之有相先者也。

\hypertarget{header-n76}{%
\subsection{天论}\label{header-n76}}

天行有常,不为尧存,不为桀亡。应之以治则吉,应之以乱则凶。强本而节用,则天不能贫;养备而动时,则天不能病;修道而不贰,则天不能祸。故水旱不能使之饥,寒暑不能使之疾,祆怪不能使之凶。本荒而用侈,则天不能使之富;养略而动罕,则天不能使之全;倍道而妄行,则天不能使之吉。故水旱未至而饥,寒暑未薄而疾,祆怪未至而凶受时与治世同,而殃祸与治世异,不可以怨天,其道然也。故明于天人之分,则可谓至人矣。

不为而成,不求而得,夫是之谓天职。如是者,虽深、其人不加虑焉;虽大、不加能焉;虽精、不加察焉,夫是之谓不与天争职。天有其时,地有其财,人有其治,夫是之谓能参。舍其所以参,而愿其所参,则惑矣。

列星随旋,日月递照,四时代御,阴阳大化,风雨博施,万物各得其和以生,各得其养以成,不见其事,而见其功,夫是之谓神。皆知其所以成,莫知其无形,夫是之谓天功。唯圣人为不求知天。

天职既立,天功既成,形具而神生,好恶喜怒哀乐臧焉,夫是之谓天情。耳目鼻口形能各有接而不相能也,夫是之谓天官。心居中虚,以治五官,夫是之谓天君。财非其类以养其类,夫是之谓天养。顺其类者谓之福,逆其类者谓之祸,夫是之谓天政。暗其天君,乱其天官,弃其天养,逆其天政,背其天情,以丧天功,夫是之谓大凶。圣人清其天君,正其天官,备其天养,顺其天政,养其天情,以全其天功。如是,则知其所为,知其所不为矣;则天地官而万物役矣。其行曲治,其养曲适,其生不伤,夫是之谓知天。

故大巧在所不为,大智在所不虑。所志于天者,已其见象之可以期者矣;所志于地者,已其见宜之可以息者矣:所志于四时者,已其见数之可以事者矣;所志于阴阳者,已其见和之可以治者矣。官人守天,而自为守道也。

治乱,天邪?曰:日月星辰瑞历,是禹桀之所同也,禹以治,桀以乱;治乱非天也。

时邪?曰:繁启蕃长于春夏,畜积收臧于秋冬,是禹桀之所同也,禹以治,桀以乱;治乱非时也。

地邪?曰:得地则生,失地则死,是又禹桀之所同也,禹以治,桀以乱;治乱非地也。诗曰:``天作高山,大王荒之。彼作矣,文王康之。''此之谓也。

天不为人之恶寒也辍冬,地不为人之恶辽远也辍广,君子不为小人之匈匈也辍行。天有常道矣,地有常数矣,君子有常体矣。君子道其常,而小人计其功。诗曰:``礼义之不愆,何恤人之言兮!''此之谓也。

楚王后车千乘,非知也;君子啜菽饮水,非愚也;是节然也。若夫志意修,德行厚,知虑明,生于今而志乎古,则是其在我者也。故君子敬其在己者,而不慕其在天者;小人错其在己者,而慕其在天者。君子敬其在己者,而不慕其在天者,是以日进也;小人错其在己者,而慕其在天者,是以日退也。故君子之所以日进,与小人之所以日退,一也。君子小人之所以相县者,在此耳。

星队木鸣,国人皆恐。曰:是何也?曰:无何也!是天地之变,阴阳之化,物之罕至者也。怪之,可也;而畏之,非也。夫日月之有蚀,风雨之不时,怪星之党见,是无世而不常有之。上明而政平,则是虽并世起,无伤也;上闇而政险,则是虽无一至者,无益也。夫星之队,木之鸣,是天地之变,阴阳之化,物之罕至者也;怪之,可也;而畏之,非也。

物之已至者,人祆则可畏也:楛耕伤稼,楛耨失岁,政险失民;田薉稼恶,籴贵民饥,道路有死人:夫是之谓人祆。政令不明,举错不时,本事不理,勉力不时,则牛马相生,六畜作祆:夫是之谓人祆。礼义不修,内外无别,男女淫乱,则父子相疑,上下乖离,寇难并至:夫是之谓人祆。祆是生于乱。三者错,无安国。其说甚尔,其菑甚惨。勉力不时,则牛马相生,六畜作祆,可怪也,而亦可畏也。传曰:``万物之怪书不说。''无用之辩,不急之察,弃而不治。若夫君臣之义,父子之亲,夫妇之别,则日切瑳而不舍也。

雩而雨,何也?曰:无何也,犹不雩而雨也。日月食而救之,天旱而雩,卜筮然后决大事,非以为得求也,以文之也。故君子以为文,而百姓以为神。以为文则吉,以为神则凶也。

在天者莫明于日月,在地者莫明于水火,在物者莫明于珠玉,在人者莫明于礼义。故日月不高,则光明不赫;水火不积,则晖润不博;珠玉不睹乎外,则王公不以为宝;礼义不加于国家,则功名不白。故人之命在天,国之命在礼。君人者,隆礼尊贤而王,重法爱民而霸,好利多诈而危,权谋倾覆幽险而亡矣。

大天而思之,孰与物畜而制之!从天而颂之,孰与制天命而用之!望时而待之,孰与应时而使之!因物而多之,孰与骋能而化之!思物而物之,孰与理物而勿失之也!愿于物之所以生,孰与有物之所以成!故错人而思天,则失万物之情。

百王之无变,足以为道贯。一废一起,应之以贯,理贯不乱。不知贯,不知应变。贯之大体未尝亡也。乱生其差,治尽其详。故道之所善,中则可从,畸则不可为,匿则大惑。水行者表深,表不明则陷。治民者表道,表不明则乱。礼者,表也。非礼,昏世也;昏世,大乱也。故道无不明,外内异表,隐显有常,民陷乃去。

万物为道一偏,一物为万物一偏。愚者为一物一偏,而自以为知道,无知也。慎子有见于后,无见于先。老子有见于诎,无见于信。墨子有见于齐,无见于畸。宋子有见于少,无见于多。有后而无先,则群众无门。有诎而无信,则贵贱不分。有齐而无畸,则政令不施,有少而无多,则群众不化。书曰:``无有作好,遵王之道;无有作恶,遵王之路。''此之谓也。

\hypertarget{header-n80}{%
\subsection{正论}\label{header-n80}}

世俗之为说者曰:``主道利周。''

是不然。主者、民之唱也,上者、下之仪也。彼将听唱而应,视仪而动;唱默则民无应也,仪隐则下无动也;不应不动,则上下无以相有也。若是,则与无上同也!不祥莫大焉。故上者、下之本也。上宣明,则下治辨矣;上端诚,则下愿悫矣;上公正,则下易直矣。治辨则易一,愿悫则易使,易直则易知。易一则强,易使则功,易知则明,是治之所由生也。上周密,则下疑玄矣;上幽险,则下渐诈矣;上偏曲,则下比周矣。疑玄则难一,渐诈则难使,比周则难知。难一则不强,难使则不功,难知则不明,是乱之所由作也。故主道利明不利幽,利宣不利周。故主道明则下安,主道幽则下危。故下安则贵上,下危则贱上。故上易知,则下亲上矣;上难知,则下畏上矣。下亲上则上安,下畏上则上危。故主道莫恶乎难知,莫危乎使下畏己。传曰:``恶之者众则危。''书曰:``克明明德。''诗曰:``明明在下。''故先王明之,岂特玄之耳哉!

世俗之为说者曰:``桀纣有天下,汤武篡而夺之。''

是不然。以桀纣为常有天下之籍则然,亲有天下之籍则不然,天下谓在桀纣则不然。古者天子千官,诸侯百官。以是千官也,令行于诸夏之国,谓之王。以是百官也,令行于境内,国虽不安,不至于废易遂亡,谓之君。圣王之子也,有天下之后也,埶籍之所在也,天下之宗室也,然而不材不中,内则百姓疾之,外则诸侯叛之,近者境内不一,遥者诸侯不听,令不行于境内,甚者诸侯侵削之,攻伐之。若是,则虽未亡,吾谓之无天下矣。圣王没,有埶籍者罢不足以县天下,天下无君;诸侯有能德明威积,海内之民莫不愿得以为君师;然而暴国独侈,安能诛之,必不伤害无罪之民,诛暴国之君,若诛独夫。若是,则可谓能用天下矣。能用天下之谓王。汤武非取天下也,修其道,行其义,兴天下之同利,除天下之同害,而天下归之也。桀纣非去天下也,反禹汤之德,乱礼义之分,禽兽之行,积其凶,全其恶,而天下去之也。天下归之之谓王,天下去之之谓亡。故桀纣无天下,汤武不弒君,由此效之也。汤武者,民之父母也;桀纣者、民之怨贼也。今世俗之为说者,以桀纣为君,而以汤武为弒,然则是诛民之父母,而师民之怨贼也,不祥莫大焉。以天下之合为君,则天下未尝合于桀纣也。然则以汤武为弒,则天下未尝有说也,直堕之耳。

故天子唯其人。天下者,至重也,非至强莫之能任;至大也,非至辨莫之能分;至众也,非至明莫之能和。此三至者,非圣人莫之能尽。故非圣人莫之能王。圣人备道全美者也,是县天下之权称也。桀纣者、其志虑至险也,其志意至闇也,其行为至乱也;亲者疏之,贤者贱之,生民怨之。禹汤之后也,而不得一人之与;刳比干,囚箕子,身死国亡,为天下之大僇,后世之言恶者必稽焉,是不容妻子之数也。故至贤畴四海,汤武是也;至罢不能容妻子,桀纣是也。今世俗之为说者,以桀纣为有天下,而臣汤武,岂不过甚矣哉!譬之,是犹伛巫跛匡大自以为有知也。

故可以有夺人国,不可以有夺人天下;可以有窃国,不可以有窃天下也。可以夺之者可以有国,而不可以有天下;窃可以得国,而不可以得天下。是何也?曰:国、小具也,可以小人有也,可以小道得也,可以小力持也;天下者、大具也,不可以小人有也,不可以小道得也,不可以小力持也。国者、小人可以有之,然而未必不亡也;天下者,至大也,非圣人莫之能有也。

世俗之为说者曰:``治古无肉刑,而有象刑:墨黥,慅婴,共、艾毕,剕、枲屦,杀、赭衣而不纯。治古如是。''

是不然。以为治邪?则人固莫触罪,非独不用肉刑,亦不用象刑矣。以为人或触罪矣,而直轻其刑,然则是杀人者不死,伤人者不刑也。罪至重而刑至轻,庸人不知恶矣,乱莫大焉。凡刑人之本,禁暴恶恶,且惩其未也。杀人者不死,而伤人者不刑,是谓惠暴而宽贼也,非恶恶也。故象刑殆非生于治古,并起于乱今也。

治古不然。凡爵列、官职、赏庆、刑罚,皆报也,以类相从者也。一物失称,乱之端也。夫德不称位,能不称官,赏不当功,罚不当罪,不祥莫大焉。昔者武王伐有商,诛纣,断其首,县之赤旆。夫征暴诛悍,治之盛也。杀人者死,伤人者刑,是百王之所同也,未有知其所由来者也。

刑称罪,则治;不称罪,则乱。故治则刑重,乱则刑轻,犯治之罪固重,犯乱之罪固轻也。书曰:``刑罚世轻世重。''此之谓也。

世俗之为说者曰:``汤武不善禁令。''曰:``是何也?''曰:``楚越不受制。''

是不然。汤武者、至天下之善禁令者也。汤居亳,武王居鄗,皆百里之地也,天下为一,诸侯为臣,通达之属,莫不振动从服以化顺之,曷为楚越独不受制也!

彼王者之制也,视形埶而制械用,称远迩而等贡献,岂必齐哉!故鲁人以榶,卫人用柯,齐人用一革,土地刑制不同者,械用、备饰不可不异也。故诸夏之国同服同仪,蛮、夷、戎、狄之国同服不同制。封内甸服,封外侯服,侯卫宾服,蛮夷要服,戎狄荒服。甸服者祭,侯服者祀,宾服者享,要服者贡,荒服者终王。日祭、月祀、时享、岁贡、终王,夫是之谓视形埶而制械用,称远近而等贡献;是王者之制也。

彼楚越者,且时享、岁贡,终王之属也,必齐之日祭月祀之属,然后曰受制邪?是规磨之说也。沟中之瘠也,则未足与及王者之制也。语曰:``浅不足与测深,愚不足与谋智,坎井之蛙,不可与语东海之乐。''此之谓也。

世俗之为说者曰:``尧舜擅让。''

是不然。天子者,埶位至尊,无敌于天下,夫有谁与让矣?道德纯备,智惠甚明,南面而听天下,生民之属莫不震动从服以化顺之。天下无隐士,无遗善,同焉者是也,异焉者非也。夫有恶擅天下矣。

曰:``死而擅之。''

是又不然。圣王在上,决德而定次,量能而授官,皆使民载其事而各得其宜。不能以义制利,不能以伪饰性,则兼以为民。圣王已没,天下无圣,则固莫足以擅天下矣。天下有圣,而在后子者,则天下不离,朝不易位,国不更制,天下厌然,与乡无以异也;以尧继尧,夫又何变之有矣!圣不在后子而在三公,则天下如归,犹复而振之矣。天下厌然,与乡无以异也;以尧继尧,夫又何变之有矣!唯其徙朝改制为难。故天子生则天下一隆,致顺而治,论德而定次,死则能任天下者必有之矣。夫礼义之分尽矣,擅让恶用矣哉!

曰:``老衰而擅。''

是又不然。血气筋力则有衰,若夫智虑取舍则无衰。

曰:``老者不堪其劳而休也。''

是又畏事者之议也。天子者埶至重而形至佚,心至愉而志无所诎,而形不为劳,尊无上矣。衣被则服五采,杂间色,重文绣,加饰之以珠玉;食饮则重大牢而备珍怪,期臭味,曼而馈,伐皋而食,雍而彻乎五祀,执荐者百余人,侍西房;居则设张容,负依而坐,诸侯趋走乎堂下;出户而巫觋有事,出门而宗祝有事,乘大路趋越席以养安,侧载睪芷以养鼻,前有错衡以养目,和鸾之声,步中武象,趋中韶护以养耳,三公奉軶、持纳,诸侯持轮、挟舆、先马,大侯编后,大夫次之,小侯元士次之,庶士介而夹道,庶人隐窜,莫敢视望。居如大神,动如天帝。持老养衰,犹有善于是者与?不老者、休也,休犹有安乐恬愉如是者乎?故曰:诸侯有老,天子无老。

有擅国,无擅天下,古今一也。夫曰尧舜擅让,是虚言也,是浅者之传,陋者之说也,不知逆顺之理,小大、至不至之变者也,未可与及天下之大理者也。

世俗之为说者曰:``尧舜不能教化。''是何也?曰:``朱象不化。''

是不然也:尧舜至天下之善教化者也。南面而听天下,生民之属莫不振动从服以化顺之。然而朱象独不化,是非尧舜之过,朱象之罪也。尧舜者、天下之英也;朱象者、天下之嵬,一时之琐也。今世俗之为说者,不怪朱象,而非尧舜,岂不过甚矣哉!夫是之谓嵬说。羿蜂门者、天下之善射者也,不能以拨弓曲矢中微;王梁造父者、天下之善驭者也,不能以辟马毁舆致远。尧舜者、天下之善教化者也,不能使嵬琐化。何世而无嵬?何时而无琐?自太皞燧人莫不有也。故作者不祥,学者受其殃,非者有庆。诗曰:``下民之孽,匪降自天。噂沓背憎,职竞由人。''此之谓也。

世俗之为说者曰:``太古薄背,棺厚三寸,衣衾三领,葬田不妨田,故不掘也;乱今厚葬饰棺,故抇也。''

是不及知治道,而不察于抇不抇者之所言也。凡人之盗也,必以有为,不以备不足,则以重有余也。而圣王之生民也,皆使富厚优犹知足,而不得以有余过度。故盗不窃,贼不刺,狗豕吐菽粟,而农贾皆能以货财让。风俗之美,男女自不取于涂,而百姓羞拾遗。故孔子曰:``天下有道,盗其先变乎!''虽珠玉满体,文绣充棺,黄金充椁,加之以丹矸,重之以曾青,犀象以为树,琅玕、龙兹、华觐以为实,人犹莫之抇也。是何故也?则求利之诡缓,而犯分之羞大也。

夫乱今然后反是。上以无法使,下以无度行;知者不得虑,能者不得治,贤者不得使。若是,则上失天性,下失地利,中失人和。故百事废,财物诎,而祸乱起。王公则病不足于上,庶人则冻餧羸瘠于下。于是焉桀纣群居,而盗贼击夺以危上矣。安禽兽行,虎狼贪,故脯巨人而炙婴儿矣。若是则有何尤抇人之墓,抉人之口而求利矣哉!虽此裸而薶之,犹且必抇也,安得葬薶哉!彼乃将食其肉而龁其骨也。

夫曰:太古薄背,故不抇也;乱今厚葬,故抇也。是特奸人之误于乱说,以欺愚者而淖陷之,以偷取利焉。夫是之谓大奸。传曰:``危人而自安,害人而自利。''此之谓也。

子宋子曰:``明见侮之不辱,使人不斗。人皆以见侮为辱,故斗于也;知见侮之为不辱,则不斗矣。''

应之曰:然则以人之情为不恶侮乎?

曰:``恶而不辱也。''

曰:若是,则必不得所求焉。凡人之斗也,必以其恶之为说,非以其辱之为故也。今俳优、侏儒、狎徒詈侮而不斗者,是岂钜知见侮之为不辱哉。然而不斗者,不恶故也。今人或入其央渎,窃其猪彘,则援剑戟而逐之,不避死伤。是岂以丧猪为辱也哉!然而不惮斗者,恶之故也。虽以见侮为辱也,不恶则不斗;虽知见侮为不辱,恶之则必斗。然则斗与不斗邪,亡于辱之与不辱也,乃在于恶之与不恶也。夫今子宋子不能解人之恶侮,而务说人以勿辱也,岂不过甚矣哉!金舌弊口,犹将无益也。不知其无益,则不知;知其无益也,直以欺人,则不仁。不仁不知,辱莫大焉。将以为有益于人,则与无益于人也,则得大辱而退耳!说莫病是矣。

子宋子曰:``见侮不辱。''

应之曰:凡议必先立隆正,然后可也。无隆正则是非不分,而辨讼不决,故所闻曰:``天下之大隆,是非之封界,分职名象之所起,王制是也。''故凡言议期命是非,以圣王为师。而圣王之分,荣辱是也。

是有两端矣。有义荣者,有埶荣者;有义辱者,有埶辱者。志意修,德行厚,知虑明,是荣之由中出者也,夫是之谓义荣。爵列尊,贡禄厚,形埶胜,上为天子诸侯,下为卿相士大夫,是荣之从外至者也,夫是之谓埶荣。流淫污僈,犯分乱理,骄暴贪利,是辱之由中出者也,夫是之谓义辱。詈侮捽搏,捶笞膑脚,斩断枯磔,借靡后缚,是辱之由外至者也,夫是之谓埶辱。是荣辱之两端也。

故君子可以有埶辱,而不可以有义辱;小人可以有埶荣,而不可以有义荣。有埶辱无害为尧,有埶荣无害为桀。义荣埶荣,唯君子然后兼有之;义辱埶辱,唯小人然后兼有之。是荣辱之分也。圣王以为法,士大夫以为道,官人以为守,百姓以成俗,万世不能易也。

今子宋子则不然,独诎容为己,虑一朝而改之,说必不行矣。譬之,是犹以砖涂塞江海也,以焦侥而戴太山也,蹎跌碎折,不待顷矣。二三子之善于子宋子者,殆不若止之,将恐得伤其体也。

子宋子曰:``人之情,欲寡,而皆以己之情,为欲多,是过也。''故率其群徒,辨其谈说,明其譬称,将使人知情之欲寡也。

应之曰:然则亦以人之情为目不欲綦色,耳不欲綦声,口不欲綦味,鼻不欲綦臭,形不欲綦佚此五綦者,亦以人之情为不欲乎?

曰:``人之情,欲是已。''

曰:若是,则说必不行矣。以人之情为欲,此五綦者而不欲多,譬之,是犹以人之情为欲富贵而不欲货也,好美而恶西施也。古之人为之不然。以人之情为欲多而不欲寡,故赏以富厚而罚以杀损也。是百王之所同也。故上贤禄天下,次贤禄一国,下贤禄田邑,愿悫之民完衣食。今子宋子以是之情为欲寡而不欲多也,然则先王以人之所不欲者赏,而以人之欲者罚邪?乱莫大焉。今子宋子严然而好说,聚人徒,立师学,成文典,然而说不免于以至治为至乱也,岂不过甚矣哉!

\hypertarget{header-n84}{%
\subsection{礼论}\label{header-n84}}

礼起于何也?曰:人生而有欲,欲而不得,则不能无求。求而无度量分界,则不能不争;争则乱,乱则穷。先王恶其乱也,故制礼义以分之,以养人之欲,给人之求。使欲必不穷于物,物必不屈于欲。两者相持而长,是礼之所起也。

故礼者养也。刍豢稻梁,五味调香,所以养口也;椒兰芬苾,所以养鼻也;雕琢刻镂,黼黻文章,所以养目也;钟鼓管磬,琴瑟竽笙,所以养耳也;疏房檖貌,越席床笫几筵,所以养体也。故礼者养也。

君子既得其养,又好其别。曷谓别?曰:贵贱有等,长幼有差,贫富轻重皆有称者也。故天子大路越席,所以养体也;侧载睪芷,所以养鼻也;前有错衡,所以养目也;和鸾之声,步中武象,趋中韶护,所以养耳也;龙旗九斿,所以养信也;寝兕持虎,蛟韅、丝末、弥龙,所以养威也;故大路之马必信至,教顺,然后乘之,所以养安也。孰知夫出死要节之所以养生也!孰知夫出费用之所以养财也!孰知夫恭敬辞让之所以养安也!孰知夫礼义文理之所以养情也!故人苟生之为见,若者必死;苟利之为见,若者必害;苟怠惰偷懦之为安,若者必危;苟情说之为乐,若者必灭。故人一之于礼义,则两得之矣;一之于情性,则两丧之矣。故儒者将使人两得之者也,墨者将使人两丧之者也,是儒墨之分也。

礼有三本:天地者,生之本也;先祖者,类之本也;君师者,治之本也。无天地,恶生?无先祖,恶出?无君师,恶治?三者偏亡,焉无安人。故礼、上事天,下事地,尊先祖,而隆君师。是礼之三本也。

故王者天太祖,诸侯不敢坏,大夫士有常宗,所以别贵始;贵始得之本也。郊止乎天子,而社止于诸侯,道及士大夫,所以别尊者事尊,卑者事卑,宜大者巨,宜小者小也。故有天下者事七世,有一国者事五世,有五乘之地者事三世,有三乘之地者事二世,持手而食者不得立宗庙,所以别积厚,积厚者流泽广,积薄者流泽狭也。

大飨,尚玄尊,俎生鱼,先大羹,贵食饮之本也。飨,尚玄尊而用酒醴,先黍稷而饭稻粱。祭,齐大羹而饱庶羞,贵本而亲用也。贵本之谓文,亲用之谓理,两者合而成文,以归大一,夫是之谓大隆。故尊之尚玄酒也,俎之尚生鱼也,豆之先大羹也,一也。利爵之不醮也,成事之俎不尝也,三臭之不食也,一也。大昏之未发齐也,太庙之未入尸也,始卒之未小敛也,一也。大路之素未集也,郊之麻絻也,丧服之先散麻也,一也。三年之丧,哭之不反也,清庙之歌,一唱而三叹也,县一钟,尚拊膈,朱弦而通越也,一也。

凡礼,始乎梲,成乎文,终乎悦校。故至备,情文俱尽;其次,情文代胜;其下复情以归大一也。天地以合,日月以明,四时以序,星辰以行,江河以流,万物以昌,好恶以节,喜怒以当,以为下则顺,以为上则明,万变不乱,贰之则丧也。礼岂不至矣哉!立隆以为极,而天下莫之能损益也。本末相顺,终始相应,至文以有别,至察以有说,天下从之者治,不从者乱,从之者安,不从者危,从之者存,不从者亡,小人不能测也。

礼之理诚深矣,``坚白''``同异''之察入焉而溺;其理诚大矣,擅作典制辟陋之说入焉而丧;其理诚高矣,暴慢恣孳轻俗以为高之属入焉而队。故绳墨诚陈矣,则不可欺以曲直;衡诚县矣,则不可欺以轻重;规矩诚设矣,则不可欺以方圆;君子审于礼,则不可欺以诈伪。故绳者,直之至;衡者,平之至;规矩者,方圆之至;礼者,人道之极也。然而不法礼,不足礼,谓之无方之民;法礼,足礼,谓之有方之士。礼之中焉能思索,谓之能虑;礼之中焉能勿易,谓之能固。能虑、能固,加好者焉,斯圣人矣。故天者,高之极也;地者,下之极也;无穷者,广之极也;圣人者,人道之极也。故学者,固学为圣人也,非特学无方之民也。

礼者,以财物为用,以贵贱为文,以多少为异,以隆杀为要。文理繁,情用省,是礼之隆也。文理省,情用繁,是礼之杀也。文理情用相为内外表墨,并行而杂,是礼之中流也。故君子上致其隆,下尽其杀,而中处其中。步骤驰骋厉鹜不外是矣。是君子之坛宇宫廷也。人有是,士君子也;外是,民也;于是其中焉,方皇周挟,曲得其次序,是圣人也。故厚者,礼之积也;大者,礼之广也;高者,礼之隆也;明者,礼之尽也。诗曰:``礼仪卒度,笑语卒获。''此之谓也。

礼者,谨于治生死者也。生、人之始也,死、人之终也,终始俱善,人道毕矣。故君子敬始而慎终,终始如一,是君子之道,礼义之文也。夫厚其生而薄其死,是敬其有知,而慢其无知也,是奸人之道而倍叛之心也。君子以倍叛之心接臧谷,犹且羞之,而况以事其所隆亲乎!故死之为道也,一而不可得再复也,臣之所以致重其君,子之所以致重其亲,于是尽矣。故事生不忠厚,不敬文,谓之野;送死不忠厚,不敬文,谓之瘠。君子贱野而羞瘠,故天子棺椁七重,诸侯五重,大夫三重,士再重。然后皆有衣衾多少厚薄之数,皆有翣菨文章之等,以敬饰之,使生死终始若一;一足以为人愿,是先王之道,忠臣孝子之极也。天子之丧动四海,属诸侯;诸侯之丧动通国,属大夫;大夫之丧动一国,属修士;修士之丧动一乡,属朋友;庶人之丧合族党,动州里;刑余罪人之丧,不得合族党,独属妻子,棺椁三寸,衣衾三领,不得饰棺,不得昼行,以昏殣,凡缘而往埋之,反无哭泣之节,无衰麻之服,无亲疏月数之等,各反其平,各复其始,已葬埋,若无丧者而止,夫是之谓至辱。

礼者,谨于吉凶不相厌者也。紸纩听息之时,则夫忠臣孝子亦知其闵矣,然而殡敛之具,未有求也;垂涕恐惧,然而幸生之心未已,持生之事未辍也。卒矣,然后作具之。故虽备家必踰日然后能殡,三日而成服。然后告远者出矣,备物者作矣。故殡久不过七十日,速不损五十日。是何也?曰:远者可以至矣,百求可以得矣,百事可以成矣;其忠至矣,其节大矣,其文备矣。然后月朝卜日,月夕卜宅,然后葬也。当是时也,其义止,谁得行之?其义行,谁得止之?故三月之葬,其貌以生设饰死者也,殆非直留死者以安生也,是致隆思慕之义也。

丧礼之凡,变而饰,动而远,久而平。故死之为道也,不饰则恶,恶则不哀;尒则翫,翫则厌,厌则忘,忘则不敬。一朝而丧其严亲,而所以送葬之者,不哀不敬,则嫌于禽兽矣,君子耻之。故变而饰,所以灭恶也;动而远,所以遂敬也;久而平,所以优生也。

礼者、断长续短,损有余,益不足,达爱敬之文,而滋成行义之美者也。故文饰、麤恶,声乐、哭泣,恬愉、忧戚;是反也;然而礼兼而用之,时举而代御。故文饰、声乐、恬愉,所以持平奉吉也;麤恶、哭泣、忧戚,所以持险奉凶也。故其立文饰也,不至于窕冶;其立麤恶也,不至于瘠弃;其立声乐、恬愉也,不至于流淫、惰慢;其立哭泣、哀戚也,不至于隘慑伤生,是礼之中流也。

故情貌之变,足以别吉凶,明贵贱亲疏之节,期止矣。外是,奸也;虽难,君子贱之。故量食而食之,量要而带之,相高以毁瘠,是奸人之道,非礼义之文也,非孝子之情也,将以有为者也。故说豫、娩泽,忧戚、萃恶,是吉凶忧愉之情发于颜色者也。歌谣、謷笑、哭泣、谛号,是吉凶忧愉之情发于声音者也。刍豢、稻梁、酒醴,餰鬻、鱼肉、菽藿、酒浆,是吉凶忧愉之情发于食饮者也。卑絻、黼黻、文织,资麤、衰绖、菲繐、菅屦,是吉凶忧愉之情发于衣服者也。疏房、檖貌、越席、床笫、几筵,属茨、倚庐、席薪、枕块,是吉凶忧愉之情发于居处者也。两情者,人生固有端焉。若夫断之继之,博之浅之,益之损之,类之尽之,盛之美之,使本末终始,莫不顺比,足以为万世则,则是礼也。非顺孰修为之君子,莫之能知也。

故曰:性者、本始材朴也;伪者、文理隆盛也。无性则伪之无所加,无伪则性不能自美。性伪合,然后成圣人之名,一天下之功于是就也。故曰:天地合而万物生,阴阳接而变化起,性伪合而天下治。天能生物,不能辨物也,地能载人,不能治人也;宇中万物生人之属,待圣人然后分也。诗曰:``怀柔百神,及河乔岳。''此之谓也。

丧礼者,以生者饰死者也,大象其生以送其死也。故事死如生,事亡如存,终始一也。始卒,沐浴、鬠体、饭唅,象生执也。不沐则濡栉三律而止,不浴则濡巾三式而止。充耳而设瑱,饭以生稻,唅以槁骨,反生术矣。设亵衣,袭三称,缙绅而无钩带矣。设掩面儇目,鬠而不冠笄矣。书其名,置于其重,则名不见而柩独明矣。荐器:则冠有鍪而毋縰,瓮庑虚而不实,有簟席而无床笫,木器不成斲,陶器不成物,薄器不成内,笙竽具而不和,琴瑟张而不均,舆藏而马反,告不用也。具生器以适墓,象徙道也。略而不尽,貌而不功,趋舆而藏之,金革辔靷而不入,明不用也。象徙道,又明不用也,是皆所以重哀也。故生器文而不功,明器貌而不用。凡礼,事生,饰欢也;送死,饰哀也;祭祀,饰敬也;师旅,饰威也。是百王之所同,古今之所一也,未有知其所由来者也。故圹垄、其貌象室屋也;棺椁、其貌象版盖斯象拂也;无帾丝歶缕翣,其貌以象菲帷帱尉也。抗折,其貌以象槾茨番阏也。故丧礼者,无他焉,明死生之义,送以哀敬,而终周藏也。故葬埋,敬藏其形也;祭祀,敬事其神也;其铭诔系世,敬传其名也。事生,饰始也;送死,饰终也;终始具,而孝子之事毕,圣人之道备矣。刻死而附生谓之墨,刻生而附死谓之惑,杀生而送死谓之贼。大象其生以送其死,使死生终始莫不称宜而好善,是礼义之法式也,儒者是矣。

三年之丧,何也?曰:称情而立文,因以饰群,别亲疏贵贱之节,而不可益损也。故曰:无适不易之术也。创巨者其日久,痛甚者其愈迟,三年之丧,称情而立文,所以为至痛极也。齐衰、苴杖、居庐、食粥、席薪、枕块,所以为至痛饰也。三年之丧,二十五月而毕,哀痛未尽,思慕未忘,然而礼以是断之者,岂不以送死有已,复生有节也哉!凡生天地之间者,有血气之属必有知,有知之属莫不爱其类。今夫大鸟兽则失亡其群匹,越月踰时,则必反铅;过故乡,则必徘徊焉,鸣号焉,踯躅焉,踟蹰焉,然后能去之也。小者是燕爵,犹有啁焦之顷焉,然后能去之。故有血气之属莫知于人,故人之于其亲也,至死无穷。将由夫愚陋淫邪之人与,则彼朝死而夕忘之;然而纵之,则是曾鸟兽之不若也,彼安能相与群居而无乱乎!将由夫修饰之君子与,则三年之丧,二十五月而毕,若驷之过隙,然而遂之,则是无穷也。故先王圣人安为之立中制节,一使足以成文理,则舍之矣。

然则何以分之?曰:至亲以期断。是何也?曰:天地则已易矣,四时则已无矣,其在宇中者莫不更始矣,故先王案以此象之也。然则三年何也?曰:加隆焉,案使倍之,故再期也。由九月以下何也?曰:案使不及也。故三年以为隆,缌麻、小功以为杀,期、九月以为间。上取象于天,下取象于地,中取则于人,人所以群居和一之理尽矣。故三年之丧,人道之至文者也,夫是之谓至隆。是百王之所同也,古今之所一也。

君之丧,所以取三年,何也?曰:君者、治辨之主也,文理之原也,情貌之尽也,相率而致隆之,不亦可乎?诗曰:``恺悌君子,民之父母。''彼君子者,固有为民父母之说焉。父能生之,不能养之;母能食之,不能教诲之;君者,已能食之矣,又善教诲之者也。三年毕矣哉!乳母、饮食之者也,而三月;慈母、衣被之者也,而九月;君曲备之者也,三年毕乎哉!得之则治,失之则乱,文之至也。得之则安,失之则危,情之至也。两至者俱积焉,以三年事之,犹未足也,直无由进之耳。故社,祭社也;稷、祭稷也;郊者,并百王于上天而祭祀之也。

三月之殡,何也?曰:大之也,重之也。所致隆也,所致亲也,将举措之,迁徙之,离宫室而归丘陵也,先王恐其不文也,是以繇其期,足之日也。故天子七月,诸侯五月,大夫三月,皆使其须足以容事,事足以容成,成足以容文,文足以容备,曲容备物之谓道矣。

祭者、志意思慕之情也。愅诡唈僾而不能无时至焉。故人之欢欣和合之时,则夫忠臣孝子亦愅诡而有所至矣。彼其所至者,甚大动也;案屈然已,则其于志意之情者惆然不嗛,其于礼节者阙然不具。故先王案为之立文,尊尊亲亲之义至矣。故曰:祭者、志意思慕之情也。忠信爱敬之至矣,礼节文貌之盛矣,苟非圣人,莫之能知也。圣人明知之,士君子安行之,官人以为守,百姓以成俗;其在君子以为人道也,其在百姓以为鬼事也。故钟鼓管磬,琴瑟竽笙,韶夏护武,汋桓箾简象,是君子之所以为愅诡其所喜乐之文也。齐衰、苴杖、居庐、食粥、席薪、枕块,是君子之所以为愅诡其所哀痛之文也。师旅有制,刑法有等,莫不称罪,是君子之所以为愅诡其所敦恶之文也。卜筮视日、斋戒、修涂、几筵、馈荐、告祝,如或飨之。物取而皆祭之,如或尝之。毋利举爵,主人有尊,如或觞之。宾出,主人拜送,反易服,即位而哭,如或去之。哀夫!敬夫!事死如事生,事亡如事存,状乎无形,影然而成文。

\hypertarget{header-n88}{%
\subsection{乐论}\label{header-n88}}

夫乐者、乐也,人情之所必不免也。故人不能无乐,乐则必发于声音,形于动静;而人之道,声音动静,性术之变尽是矣。故人不能不乐,乐则不能无形,形而不为道,则不能无乱。先王恶其乱也,故制雅颂之声以道之,使其声足以乐而不流,使其文足以辨而不諰,使其曲直繁省廉肉节奏,足以感动人之善心,使夫邪污之气无由得接焉。是先王立乐之方也,而墨子非之奈何!

故乐在宗庙之中,君臣上下同听之,则莫不和敬;闺门之内,父子兄弟同听之,则莫不和亲;乡里族长之中,长少同听之,则莫不和顺。故乐者审一以定和者也,比物以饰节者也,合奏以成文者也;足以率一道,足以治万变。是先王立乐之术也,而墨子非之奈何!

故听其雅颂之声,而志意得广焉;执其干戚,习其俯仰屈伸,而容貌得庄焉;行其缀兆,要其节奏,而行列得正焉,进退得齐焉。故乐者、出所以征诛也,入所以揖让也;征诛揖让,其义一也。出所以征诛,则莫不听从;入所以揖让,则莫不从服。故乐者、天下之大齐也,中和之纪也,人情之所必不免也。是先王立乐之术也,而墨子非之奈何!

且乐者、先王之所以饰喜也;军旅鈇钺者,先王之所以饰怒也。先王喜怒皆得其齐焉。是故喜而天下和之,怒而暴乱畏之。先王之道,礼乐正其盛者也。而墨子非之。故曰:墨子之于道也,犹瞽之于白黑也,犹聋之于清浊也,犹欲之楚而北求之也。

夫声乐之入人也深,其化人也速,故先王谨为之文。乐中平则民和而不流,乐肃庄则民齐而不乱。民和齐则兵劲城固,敌国不敢婴也。如是,则百姓莫不安其处,乐其乡,以至足其上矣。然后名声于是白,光辉于是大,四海之民莫不愿得以为师,是王者之始也。乐姚冶以险,则民流僈鄙贱矣;流僈则乱,鄙贱则争;乱争则兵弱城犯,敌国危之如是,则百姓不安其处,不乐其乡,不足其上矣。故礼乐废而邪音起者,危削侮辱之本也。故先王贵礼乐而贱邪音。其在序官也,曰:``修宪命,审诗商,禁淫声,以时顺修,使夷俗邪音不敢乱雅,太师之事也。''

墨子曰:``乐者、圣王之所非也,而儒者为之过也。''君子以为不然。乐者,圣王之所乐也,而可以善民心,其感人深,其移风易俗。故先王导之以礼乐,而民和睦。夫民有好恶之情,而无喜怒之应则乱;先王恶其乱也,故修其行,正其乐,而天下顺焉。故齐衰之服,哭泣之声,使人之心悲。带甲婴胄,歌于行伍,使人之心伤;姚冶之容,郑卫之音,使人之心淫;绅、端、章甫,舞韶歌武,使人之心庄。故君子耳不听淫声,目不视邪色,口不出恶言,此三者,君子慎之。

凡奸声感人而逆气应之,逆气成象而乱生焉;正声感人而顺气应之,顺气成象而治生焉。唱和有应,善恶相象,故君子慎其所去就也。君子以钟鼓道志,以琴瑟乐心;动以干戚,饰以羽旄,从以磬管。故其清明象天,其广大象地,其俯仰周旋有似于四时。故乐行而志清,礼修而行成,耳目聪明,血气和平,移风易俗,天下皆宁,美善相乐。故曰:乐者、乐也。君子乐得其道,小人乐得其欲;以道制欲,则乐而不乱;以欲忘道,则惑而不乐。故乐者,所以道乐也,金石丝竹,所以道德也;乐行而民乡方矣。故乐也者,治人之盛者也,而墨子非之。

且乐也者,和之不可变者也;礼也者,理之不可易者也。乐合同,礼别异,礼乐之统,管乎人心矣。穷本极变,乐之情也;着诚去伪,礼之经也。墨子非之,几遇刑也。明王已没,莫之正也。愚者学之,危其身也。君子明乐,乃其德也。乱世恶善,不此听也。于乎哀哉!不得成也。弟子勉学,无所营也。

声乐之象:鼓大丽,钟统实,磬廉制,竽笙箫和,筦钥发猛,埙篪翁博,瑟易良,琴妇好,歌清尽,舞意天道兼。鼓其乐之君邪。故鼓似天,钟似地,磬似水,竽笙箫和筦钥,似星辰日月,鼗柷、拊鞷、椌楬似万物。曷以知舞之意?曰:目不自见,耳不自闻也,然而治俯仰、诎信、进退、迟速,莫不廉制,尽筋骨之力,以要钟鼓俯会之节,而靡有悖逆者,众积意謘謘乎!

吾观于乡,而知王道之易易也。主人亲速宾及介,而众宾皆从之。至于门外,主人拜宾及介,而众宾皆入;贵贱之义别矣。三揖至于阶,三让以宾升。拜至、献、酬,辞让之节繁,及介省矣。至于众宾,升受、坐祭、立饮,不酢而降;隆杀之义辨矣。工入,升歌三终,主人献之;笙入三终,主人献之;间歌三终,合乐三终,工告乐备,遂出。二人扬觯,乃立司正,焉知其能和乐而不流也。宾酬主人,主人酬介,介酬众宾,少长以齿,终于沃洗者,焉知其能弟长而无遗也。降,说屦升坐,修爵无数。饮酒之节,朝不废朝,莫不废夕。宾出,主人拜送,节文终遂,焉知其能安燕而不乱也。贵贱明,隆杀辨,和乐而不流,弟长而无遗,安燕而不乱,此五行者,足以正身安国矣。彼国安而天下安。故曰:吾观于乡,而知王道之易易也。

乱世之征:其服组,其容妇。其俗淫,其志利,其行杂,其声乐险,其文章匿而采,其养生无度,其送死瘠墨,贱礼义而贵勇力,贫则为盗,富则为贼;治世反是也。

\hypertarget{header-n92}{%
\subsection{解蔽}\label{header-n92}}

凡人之患,蔽于一曲,而闇于大理。治则复经,两疑则惑矣。天下无二道,圣人无两心。今诸侯异政,百家异说,则必或是或非,或治或乱。乱国之君,乱家之人,此其诚心,莫不求正而以自为也。妒缪于道,而人诱其所迨也。私其所积,唯恐闻其恶也。倚其所私,以观异术,唯恐闻其美也。是以与治虽走,而是己不辍也。岂不蔽于一曲,而失正求也哉!心不使焉,则白黑在前而目不见,雷鼓在侧而耳不闻,况于使者乎?德道之人,乱国之君非之上,乱家之人非之下,岂不哀哉!

故为蔽:欲为蔽,恶为蔽,始为蔽,终为蔽,远为蔽,近为蔽,博为蔽,浅为蔽,古为蔽,今为蔽。凡万物异则莫不相为蔽,此心术之公患也。

昔人君之蔽者,夏桀殷纣是也。桀蔽于末喜斯观,而不知关龙逢,以惑其心,而乱其行。桀蔽于妲己、飞廉,而不知微子启,以惑其心,而乱其行。故群臣去忠而事私,百姓怨非而不用,贤良退处而隐逃,此其所以丧九牧之地,而虚宗庙之国也。桀死于鬲山,纣县于赤旆。身不先知,人又莫之谏,此蔽塞之祸也。成汤监于夏桀,故主其心而慎治之,是以能长用伊尹,而身不失道,此其所以代夏王而受九有也。文王监于殷纣,故主其心而慎治之,是以能长用吕望,而身不失道,此其所以代殷王而受九牧也。远方莫不致其珍;故目视备色,耳听备声,口食备味,形居备宫,名受备号,生则天下歌,死则四海哭。夫是之谓至盛。诗曰:``凤凰秋秋,其翼若干,其声若箫。有凤有凰,乐帝之心。''此不蔽之福也。

昔人臣之蔽者,唐鞅奚齐是也。唐鞅蔽于欲权而逐载子,奚齐蔽于欲国而罪申生;唐鞅戮于宋,奚齐戮于晋。逐贤相而罪孝兄,身为刑戮,然而不知,此蔽塞之祸也。故以贪鄙、背叛、争权而不危辱灭亡者,自古及今,未尝有之也。鲍叔、宁戚、隰朋仁知且不蔽,故能持管仲,而名利福禄与管仲齐。召公、吕望仁知且不蔽,故能持周公而名利福禄与周公齐。传曰:``知贤之为明,辅贤之谓能,勉之强之,其福必长。''此之谓也。此不蔽之福也。

昔宾孟之蔽者,乱家是也。墨子蔽于用而不知文。宋子蔽于欲而不知得。慎子蔽于法而不知贤。申子蔽于埶而不知知。惠子蔽于辞而不知实。庄子蔽于天而不知人。故由用谓之道,尽利矣。由欲谓之道,尽嗛矣。由法谓之道,尽数矣。由埶谓之道,尽便矣。由辞谓之道,尽论矣。由天谓之道,尽因矣。此数具者,皆道之一隅也。夫道者体常而尽变,一隅不足以举之。曲知之人,观于道之一隅,而未之能识也。故以为足而饰之,内以自乱,外以惑人,上以蔽下,下以蔽上,此蔽塞之祸也。孔子仁知且不蔽,故学乱术足以为先王者也。一家得周道,举而用之,不蔽于成积也。故德与周公齐,名与三王并,此不蔽之福也。

圣人知心术之患,见蔽塞之祸,故无欲、无恶、无始、无终、无近、无远、无博、无浅、无古、无今,兼陈万物而中县衡焉。是故众异不得相蔽以乱其伦也。

何谓衡?曰:道。故心不可以不知道;心不知道,则不可道,而可非道。人孰欲得恣,而守其所不可,以禁其所可?以其不可道之心取人,则必合于不道人,而不合于道人。以其不可道之心与不道人论道人,乱之本也。夫何以知?曰:心知道,然后可道;可道然后守道以禁非道。以其可道之心取人,则合于道人,而不合于不道之人矣。以其可道之心与道人论非道,治之要也。何患不知?故治之要在于知道。

人何以知道?曰:心。心何以知?曰:虚壹而静。心未尝不臧也,然而有所谓虚;心未尝不两也,然而有所谓壹;心未尝不动也,然而有所谓静。人生而有知,知而有志;志也者,臧也;然而有所谓虚;不以所已臧害所将受谓之虚。心生而有知,知而有异;异也者,同时兼知之;同时兼知之,两也;然而有所谓一;不以夫一害此一谓之壹。心卧则梦,偷则自行,使之则谋;故心未尝不动也;然而有所谓静;不以梦剧乱知谓之静。未得道而求道者,谓之虚壹而静。作之:则将须道者之虚则人,将事道者之壹则尽,尽将思道者静则察。知道察,知道行,体道者也。虚壹而静,谓之大清明。万物莫形而不见,莫见而不论,莫论而失位。坐于室而见四海,处于今而论久远。疏观万物而知其情,参稽治乱而通其度,经纬天地而材官万物,制割大理而宇宙里矣。恢恢广广,孰知其极?睪睪广广,孰知其德?涫涫纷纷,孰知其形?明参日月,大满八极,夫是之谓大人。夫恶有蔽矣哉!

心者,形之君也,而神明之主也,出令而无所受令。自禁也,自使也,自夺也,自取也,自行也,自止也。故口可劫而使墨云,形可劫而使诎申,心不可劫而使易意,是之则受,非之则辞。故曰:心容其择也无禁,必自现,其物也杂博,其情之至也不贰。诗云:``采采卷耳,不盈倾筐。嗟我怀人,寘彼周行。''倾筐易满也,卷耳易得也,然而不可以贰周行。故曰:心枝则无知,倾则不精,贰则疑惑。以赞稽之,万物可兼知也。身尽其故则美。类不可两也,故知者择一而壹焉。

农精于田,而不可以为田师;贾精于市,而不可以为市师;工精于器,而不可以为器师。有人也,不能此三技,而可使治三官。曰:精于道者也。精于物者也。精于物者以物物,精于道者兼物物。故君子壹于道,而以赞稽物。壹于道则正,以赞稽物则察;以正志行察论,则万物官矣。昔者舜之治天下也,不以事诏而万物成。处一危之,其荣满侧;养一之微,荣矣而未知。故道经曰:``人心之危,道心之微。''危微之几,惟明君子而后能知之。故人心譬如盘水,正错而勿动,则湛浊在下,而清明在上,则足以见鬒眉而察理矣。微风过之,湛浊动乎下,清明乱于上,则不可以得大形之正也。心亦如是矣。故导之以理,养之以清,物莫之倾,则足以定是非决嫌疑矣。小物引之,则其正外易,其心内倾,则不足以决麤理矣。故好书者众矣,而仓颉独传者,壹也;好稼者众矣,而后稷独传者,壹也。好乐者众矣,而夔独传者,壹也;好义者众矣,而舜独传者,壹也。倕作弓,浮游作矢,而羿精于射;奚仲作车,乘杜作乘马,而造父精于御:自古及今,未尝有两而能精者也。曾子曰:``是其庭可以搏鼠,恶能与我歌矣!''

空石之中有人焉,其名曰觙。其为人也,善射以好思。耳目之欲接,则败其思;蚊虻之声闻,则挫其精。是以辟耳目之欲,而远蚊虻之声,闲居静思则通。思仁若是,可谓微乎?孟子恶败而出妻,可谓能自强矣,未及思也;有子恶卧而焠掌,可谓能自忍矣;未及好也。辟耳目之欲,远蚊虻之声,可谓危矣;未可谓微也。夫微者,至人也。至人也,何忍!何强!何危!故浊明外景,清明内景,圣人纵其欲,兼其情,而制焉者理矣;夫何强!何忍!何危!故仁者之行道也,无为也;圣人之行道也,无强也。仁者之思也恭,圣者之思也乐。此治心之道也。

凡观物有疑,中心不定,则外物不清。吾虑不清,未可定然否也。冥冥而行者,见寝石以为伏虎也,见植林以为后人也:冥冥蔽其明也。醉者越百步之沟,以为蹞步之浍也;俯而出城门,以为小之闺也:酒乱其神也。厌目而视者,视一为两;掩耳而听者,听漠漠而以为哅哅:埶乱其官也。故从山上望牛者若羊,而求羊者不下牵也:远蔽其大也。从山下望木者,十仞之木若箸,而求箸者不上折也:高蔽其长也。水动而景摇,人不以定美恶:水埶玄也。瞽者仰视而不见星,人不以定有无:用精惑也。有人焉以此时定物,则世之愚者也。彼愚者之定物,以疑决疑,决必不当。夫苟不当,安能无过乎?

夏首之南有人焉;曰涓蜀梁。其为人也,愚而善畏。明月而宵行,俯见其影,以为伏鬼也;仰视其发,以为立魅也。背而走,比至其家,失气而死。岂不哀哉!凡人之有鬼也,必以其感忽之间,疑玄之时定之。此人之所以无有而有无之时也,而己以定事。故伤于湿而痹,痹而击鼓烹豚,则必有敝鼓丧豚之费矣,而未有俞疾之福也。故虽不在夏首之南,则无以异矣。

凡以知,人之性也;可以知,物之理也。以可以知人之性,求可以知物之理,而无所疑止之,则没世穷年不能无也。其所以贯理焉虽亿万,已不足浃万物之变,与愚者若一。学、老身长子,而与愚者若一,犹不知错,夫是之谓妄人。故学也者,固学止之也。恶乎止之?曰:止诸至足。曷谓至足?曰:圣王。圣也者,尽伦者也;王也者,尽制者也;两尽者,足以为天下极矣。故学者以圣王为师,案以圣王之制为法,法其法以求其统类,以务象效其人。向是而务,士也;类是而几,君子也;知之,圣人也。故有知非以虑是,则谓之惧;有勇非以持是,则谓之贼;察孰非以分是,则谓之篡;多能非以修荡是,则谓之知;辩利非以言是,则谓之詍。传曰:``天下有二:非察是,是察非。''谓合王制不合王制也。天下不以是为隆正也,然而犹有能分是非、治曲直者邪?

若夫非分是非,非治曲直,非辨治乱,非治人道,虽能之无益于人,不能无损于人;案直将治怪说,玩奇辞,以相挠滑也;案强钳而利口,厚颜而忍诟,无正而恣孳,妄辨而几利;不好辞让,不敬礼节,而好相推挤:此乱世奸人之说也,则天下之治说者,方多然矣。传曰:``析辞而为察,言物而为辨,君子贱之。博闻强志,不合王制,君子贱之。''此之谓也。

为之无益于成也,求之无益于得也,忧戚之无益于几也,则广焉能弃之矣,不以自妨也,不少顷干之胸中。不慕往,不闵来,无邑怜之心,当时则动,物至而应,事起而辨,治乱可否,昭然明矣。

周而成,泄而败,明君无之有也。宣而成,隐而败,闇君无之有也。故人君者,周则谗言至矣,直言反矣;小人迩而君子远矣!诗云:``墨以为明,狐狸而苍。''此言上幽而下险也。君人者,宣则直言至矣,而谗言反矣;君子迩而小人远矣!诗云:``明明在下,赫赫在上。''此言上明而下化也。

\hypertarget{header-n96}{%
\subsection{正名}\label{header-n96}}

后王之成名:刑名从商,爵名从周,文名从礼,散名之加于万物者,则从诸夏之成俗曲期,远方异俗之乡,则因之而为通。

散名之在人者:生之所以然者谓之性;性之和所生,精合感应,不事而自然谓之性。性之好、恶、喜、怒、哀、乐谓之情。情然而心为之择谓之虑。心虑而能为之动谓之伪;虑积焉,能习焉,而后成谓之伪。正利而为谓之事。正义而为谓之行。所以知之在人者谓之知;知有所合谓之智。所以能之在人者谓之能;能有所合谓之能。性伤谓之病。节遇谓之命:是散名之在人者也,是后王之成名也。

故王者之制名,名定而实辨,道行而志通,则慎率民而一焉。故析辞擅作名,以乱正名,使民疑惑,人多辨讼,则谓之大奸。其罪犹为符节度量之罪也。故其民莫敢托为奇辞以乱正名,故其民悫;悫则易使,易使则公。其民莫敢托为奇辞以乱正名,故壹于道法,而谨于循令矣。如是则其迹长矣。迹长功成,治之极也。是谨于守名约之功也。今圣王没,名守慢,奇辞起,名实乱,是非之形不明,则虽守法之吏,诵数之儒,亦皆乱也。若有王者起,必将有循于旧名,有作于新名。然则所为有名,与所缘以同异,与制名之枢要,不可不察也。

异形离心交喻,异物名实玄纽,贵贱不明,同异不别;如是,则志必有不喻之患,而事必有困废之祸。故知者为之分别制名以指实,上以明贵贱,下以辨同异。贵贱明,同异别,如是则志无不喻之患,事无困废之祸,此所为有名也。

然则何缘而以同异?曰:缘天官。凡同类同情者,其天官之意物也同。故比方之疑似而通,是所以共其约名以相期也。形体、色理以目异;声音清浊、调竽、奇声以耳异;甘、苦、咸、淡、辛、酸、奇味以口异;香、臭、芬、郁、腥、臊、漏庮、奇臭以鼻异;疾、痒、凔、热、滑、铍、轻、重以形体异;说、故、喜、怒、哀、乐、爱、恶、欲以心异。心有征知。征知,则缘耳而知声可也,缘目而知形可也。然而征知必将待天官之当簿其类,然后可也。五官簿之而不知,心征知而无说,则人莫不然谓之不知。此所缘而以同异也。

然后随而命之,同则同之,异则异之。单足以喻则单,单不足以喻则兼;单与兼无所相避则共;虽共不为害矣。知异实者之异名也,故使异实者莫不异名也,不可乱也,犹使同实者莫不同名也。

故万物虽众,有时而欲无举之,故谓之物;物也者,大共名也。推而共之,共则有共,至于无共然后止。有时而欲偏举之,故谓之鸟兽。鸟兽也者,大别名也。推而别之,别则有别,至于无别然后至。

名无固宜,约之以命,约定俗成谓之宜,异于约则谓之不宜。名无固实,约之以命实,约定俗成,谓之实名。名有固善,径易而不拂,谓之善名。

物有同状而异所者,有异状而同所者,可别也。状同而为异所者,虽可合,谓之二实。状变而实无别而为异者,谓之化。有化而无别,谓之一实。此事之所以稽实定数也。此制名之枢要也。后王之成名,不可不察也。

``见侮不辱'',``圣人不爱己'',``杀盗非杀人也'',此惑于用名以乱名者也。验之所为有名,而观其孰行,则能禁之矣。``山渊平'',``情欲寡'',``刍豢不加甘,大钟不加乐'',此惑于用实,以乱名者也。验之所缘以同异,而观其孰调,则能禁之矣。``非而谒楹'',``有牛马非马也,''此惑于用名以乱实者也。验之名约,以其所受,悖其所辞,则能禁之矣。

凡邪说辟言之离正道而擅作者,无不类于三惑者矣。故明君知其分而不与辨也。夫民易一以道,而不可与共故。故明君临之以埶,道之以道,申之以命,章之以论,禁之以刑。故民之化道也如神,辨说恶用矣哉!今圣王没,天下乱,奸言起,君子无埶以临之,无刑以禁之,故辨说也。实不喻,然后命,命不喻,然后期,期不喻,然后说,说不喻,然后辨。故期命辨说也者,用之大文也,而王业之始也。名闻而实喻,名之用也。累而成文,名之丽也。用丽俱得,谓之知名。名也者,所以期累实也。辞也者,兼异实之名以论一意也。辨说也者,不异实名以喻动静之道也。期命也者,辨说之用也。辨说也者,心之象道也。心也者,道之工宰也。道也者,治之经理也。心合于道,说合于心,辞合于说。正名而期,质请而喻,辨异而不过,推类而不悖。听则合文,辨则尽故。以正道而辨奸,犹引绳以持曲直。是故邪说不能乱,百家无所窜。有兼听之明,而无矜奋之容;有兼覆之厚,而无伐德之色。说行则天下正,说不行则白道而冥穷。是圣人之辨说也。诗曰:``颙颙卬卬,如圭如璋,令闻令望,岂弟君子,四方为纲。''此之谓也。

辞让之节得矣,长少之理顺矣;忌讳不称,祅辞不出。以仁心说,以学心听,以公心辨。不动乎众人之非誉,不治观者之耳目,不赂贵者之权埶,不利传辟者之辞。故能处道而不贰,咄而不夺,利而不流,贵公正而贱鄙争,是士君子之辨说也。诗曰:``长夜漫兮,永思骞兮,大古之不慢兮,礼义之不愆兮,何恤人之言兮!''此之谓也。

君子之言,涉然而精,俛然而类,差差然而齐。彼正其名,当其辞,以务白其志义者也。彼名辞也者,志义之使也,足以相通,则舍之矣。苟之,奸也。故名足以指实,辞足以见极,则舍之矣。外是者,谓之讱,是君子之所弃,而愚者拾以为己宝。故愚者之言,芴然而粗,啧然而不类,誻誻然而沸,彼诱其名,眩其辞,而无深于其志义者也。故穷借而无极,甚劳而无功,贪而无名。故知者之言也,虑之易知也,行之易安也,持之易立也,成则必得其所好,而不遇其所恶焉。而愚者反是。诗曰:``为鬼为蜮,则不可得。有腼面目,视人罔极。作此好歌,以极反侧。''此之谓也。

凡语治而待去欲者,无以道欲而困于有欲者也。凡语治而待寡欲者,无以节欲而困于多欲者也。有欲无欲,异类也,生死也,非治乱也。欲之多寡,异类也,情之数也,非治乱也。欲不待可得,而求者从所可。欲不待可得,所受乎天也;求者从所可,所受乎心也。所受乎天之一欲,制于所受乎心之多,固难类所受乎天也。人之所欲生甚矣,人之恶死甚矣;然而人有从生成死者,非不欲生而欲死也,不可以生而可以死也。故欲过之而动不及,心止之也。心之所可中理,则欲虽多,奚伤于治?欲不及而动过之,心使之也。心之所可失理,则欲虽寡,奚止于乱?故治乱在于心之所可,亡于情之所欲。不求之其所在,而求之其所亡,虽曰我得之,失之矣。

性者、天之就也;情者、性之质也;欲者、情之应也。以所欲为可得而求之,情之所必不免也。以为可而道之,知所必出也。故虽为守门,欲不可去,性之具也。虽为天子,欲不可尽。欲虽不可尽,可以近尽也。欲虽不可去,求可节也。所欲虽不可尽,求者犹近尽;欲虽不可去,所求不得,虑者欲节求也。道者、进则近尽,退则节求,天下莫之若也。

凡人莫不从其所可,而去其所不可。知道之莫之若也,而不从道者,无之有也。假之有人而欲南,无多;而恶北,无寡,岂为夫南之不可尽也,离南行而北走也哉!今人所欲,无多;所恶,无寡,岂为夫所欲之不可尽也,离得欲之道,而取所恶也哉!故可道而从之,奚以损之而乱?不可道而离之,奚以益之而治?故知者论道而已矣,小家珍说之所愿者皆衰矣。凡人之取也,所欲未尝粹而来也;其去也,所恶未尝粹而往也。故人无动而不可以不与权俱。衡不正,则重县于仰,而人以为轻;轻县于俛,而人以为重;此人所以惑于轻重也。权不正,则祸托于欲,而人以为福;福托于恶,而人以为祸;此亦人所以惑于祸福也。道者,古今之正权也;离道而内自择,则不知祸福之所托。易者,以一易一,人曰:无得亦无丧也,以一易两,人曰:无丧而有得也。以两易一,人曰:无得而有丧也。计者取所多,谋者从所可。以两易一,人莫之为,明其数也。从道而出,犹以一易两也,奚丧!离道而内自择,是犹以两易一也,奚得!其累百年之欲,易一时之嫌,然且为之,不明其数也。

有尝试深观其隐而难者:志轻理而不重物者,无之有也;外重物而不内忧者,无之有也;行离理而不外危者,无之有也;外危而不内恐者,无之有也。心忧恐,则口衔刍豢而不知其味,耳听钟鼓而不知其声,目视黼黻而不知其状,轻暖平簟而体不知其安。故向万物之美而不能嗛也。假而得间而嗛之,则不能离也。故向万物之美而盛忧,兼万物之美而盛害,如此者,其求物也,养生也?粥寿也?故欲养其欲而纵其情,欲养其性而危其形,欲养其乐而攻其心,欲养其名而乱其行,如此者,虽封侯称君,其与夫盗无以异;乘轩戴絻,其与无足无以异。夫是之谓以己为物役矣。

心平愉,则色不及佣而可以养目,声不及佣而可以养耳,蔬食菜羹而可以养口,麤布之衣,麤紃之履,而可以养体。局室、芦帘、稿蓐、敝机筵,而可以养形。故虽无万物之美而可以养乐,无埶列之位而可以养名。如是而加天下焉,其为天下多,其私乐少矣。夫是之谓重己役物。

无稽之言,不见之行,不闻之谋,君子慎之。

\hypertarget{header-n100}{%
\subsection{性恶}\label{header-n100}}

人之性恶,其善者伪也。今人之性,生而有好利焉,顺是,故争夺生而辞
让亡焉;生而有疾恶焉,顺是,故残贼生而忠信亡焉;生而有耳目之欲,有好声色
焉,顺是,故淫乱生而礼义文理亡焉。然则从人之性,顺人之情,必出于争夺,合于犯分乱理,而归于暴。故必将有师法之化,礼义之道,然后出于辞让,合于文
理,而归于治。用此观之,人之性恶明矣,其善者伪也。

故枸木必将待檃栝、烝矫然后直;钝金必将待砻厉然后利;今人之性恶,必将
待师法然后正,得礼义然后治,今人无师法,则偏险而不正;无礼义,则悖乱而不
治,古者圣王以人性恶,以为偏险而不正,悖乱而不治,是以为之起礼义,制法度,
以矫饰人之情性而正之,以扰化人之情性而导之也,始皆出于治,合于道者也。今
人之化师法,积文学,道礼义者为君子;纵性情,安恣孳,而违礼义者为小人。用
此观之,人之性恶明矣,其善者伪也。

孟子曰:``今之学者,其性善。''

曰:是不然。是不及知人之性,而不察乎人之性伪之分者也。凡性者,天之就
也,不可学,不可事。礼义者,圣人之所生也,人之所学而能,所事而成者也。不
可学,不可事,而在人者,谓之性;可学而能,可事而成之在人者,谓之伪。是性
伪之分也。今人之性,目可以见,耳可以听;夫可以见之明不离目,可以听之聪不
离耳,目明而耳聪,不可学明矣。

孟子曰:``今人之性善,将皆失丧其性故也。''

曰:若是则过矣。今人之性,生而离其朴,离其资,必失而丧之。用此观之,
然则人之性恶明矣。所谓性善者,不离其朴而美之,不离其资而利之也。使夫资朴
之于美,心意之于善,若夫可以见之明不离目,可以听之聪不离耳,故曰目明而耳
聪也。今人之性,饥而欲饱,寒而欲暖,劳而欲休,此人之情性也。今人见长而不
敢先食者,将有所让也;劳而不敢求息者,将有所代也。夫子之让乎父,弟之让乎
兄,子之代乎父,弟之代乎兄,此二行者,皆反于性而悖于情也;然而孝子之道,
礼义之文理也。故顺情性则不辞让矣,辞让则悖于情性矣。用此观之,人之性恶明
矣,其善者伪也。

问者曰:``人之性恶,则礼义恶生?''

应之曰:凡礼义者,是生于圣人之伪,非故生于人之性也。故陶人埏埴而为器,
然则器生于陶人之伪,非故生于人之性也。故工人斲木而成器,然则器生于工人之
伪,非故生于人之性也。圣人积思虑,习伪故,以生礼义而起法度,然则礼义法度
者,是生于圣人之伪,非故生于人之性也。若夫目好色,耳好听,口好味,心好利,
骨体肤理好愉佚,是皆生于人之情性者也;感而自然,不待事而后生之者也。夫感
而不能然,必且待事而后然者,谓之生于伪。是性伪之所生,其不同之征也。

故圣人化性而起伪,伪起而生礼义,礼义生而制法度;然则礼义法度者,是圣
人之所生也。故圣人之所以同于众,其不异于众者,性也;所以异而过众者,伪也。
夫好利而欲得者,此人之情性也。假之有弟兄资财而分者,且顺情性,好利而欲得,
若是,则兄弟相拂夺矣;且化礼义之文理,若是,则让乎国人矣。故顺情性则弟兄
争矣,化礼义则让乎国人矣。

凡人之欲为善者,为性恶也。夫薄愿厚,恶愿美,狭愿广,贫愿富,贱愿贵,
苟无之中者,必求于外。故富而不愿财,贵而不愿埶,苟有之中者,必不及于外。
用此观之,人之欲为善者,为性恶也。今人之性,固无礼义,故强学而求有之也;
性不知礼义,故思虑而求知之也。然则性而已,则人无礼义,不知礼义。人无礼义
则乱,不知礼义则悖。然则性而已,则悖乱在己。用此观之,人之性恶明矣,其善
者伪也。

孟子曰:``人之性善。''

曰:是不然。凡古今天下之所谓善者,正理平治也;所谓恶者,偏险悖乱也:
是善恶之分也矣。今诚以人之性固正理平治邪,则有恶用圣王,恶用礼义哉?虽有
圣王礼义,将曷加于正理平治也哉?今不然,人之性恶。故古者圣人以人之性恶,
以为偏险而不正,悖乱而不治,故为之立君上之埶以临之,明礼义以化之,起法正
以治之,重刑罚以禁之,使天下皆出于治,合于善也。是圣王之治而礼义之化也。
今当试去君上之埶,无礼义之化,去法正之治,无刑罚之禁,倚而观天下民人之相
与也。若是,则夫强者害弱而夺之,众者暴寡而哗之,天下悖乱而相亡,不待顷矣。
用此观之,然则人之性恶明矣,其善者伪也。

故善言古者,必有节于今;善言天者,必有征于人。凡论者贵其有辨合,有符
验。故坐而言之,起而可设,张而可施行。今孟子曰:``人之性善。''无辨合符验,
坐而言之,起而不可设,张而不可施行,岂不过甚矣哉!故性善则去圣王,息礼义
矣。性恶则与圣王,贵礼义矣。故檃栝之生,为枸木也;绳墨之起,为不直也;立
君上,明礼义,为性恶也。用此观之,然则人之性恶明矣,其善者伪也。

直木不待檃栝而直者,其性直也。枸木必将待檃栝烝矫然后直者,以其性不直
也。今人之性恶,必将待圣王之治,礼义之化,然后始出于治,合于善也。用此观
之,人之性恶明矣,其善者伪也。

问者曰:``礼义积伪者,是人之性,故圣人能生之也。''

应之曰:是不然。夫陶人埏埴而生瓦,然则瓦埴岂陶人之性也哉?工人斲木而
生器,然则器木岂工人之性也哉?夫圣人之于礼义也,辟则陶埏而生之也。然则礼
义积伪者,岂人之本性也哉!凡人之性者,尧舜之与桀跖,其性一也;君子之与小
人,其性一也。今将以礼义积伪为人之性邪?然则有曷贵尧禹,曷贵君子矣哉!凡
贵尧禹君子者,能化性,能起伪,伪起而生礼义。然则圣人之于礼义积伪也,亦犹
陶埏而为之也。用此观之,然则礼义积伪者,岂人之性也哉!所贱于桀跖小人者,
从其性,顺其情,安恣孳,以出乎贪利争夺。故人之性恶明矣,其善者伪也。天非
私曾骞孝己而外众人也,然而曾骞孝己独厚于孝之实,而全于孝之名者,何也?以
綦于礼义故也。天非私齐鲁之民而外秦人也,然而于父子之义,夫妇之别,不如齐
鲁之孝具敬文者,何也?以秦人从情性,安恣孳,慢于礼义故也,岂其性异矣哉!

``涂之人可以为禹。''曷谓也?

曰:凡禹之所以为禹者,以其为仁义法正也。然则仁义法正有可知可能之理。
然而涂之人也,皆有可以知仁义法正之质,皆有可以能仁义法正之具,然则其可以
为禹明矣。今以仁义法正为固无可知可能之理邪?然则唯禹不知仁义法正,不能仁
义法正也。将使涂之人固无可以知仁义法正之质,而固无可以能仁义法正之具邪?
然则涂之人也,且内不可以知父子之义,外不可以知君臣之正。今不然。涂之人者,
皆内可以知父子之义,外可以知君臣之正,然则其可以知之质,可以能之具,其在
涂之人明矣。今使涂之人者,以其可以知之质,可以能之具,本夫仁义法正之可知
可能之理,可能之具,然则其可以为禹明矣。今使涂之人伏术为学,专心一志,思
索孰察,加日县久,积善而不息,则通于神明,参于天地矣。故圣人者,人之所积
而致矣。

曰:``圣可积而致,然而皆不可积,何也?''

曰:可以而不可使也。故小人可以为君子,而不肯为君子;君子可以为小人,
而不肯为小人。小人君子者,未尝不可以相为也,然而不相为者,可以而不可使也。
故涂之人可以为禹,则然;涂之人能为禹,则未必然也。虽不能为禹,无害可以为
禹。足可以遍行天下,然而未尝有遍行天下者也。夫工匠农贾,未尝不可以相为事
也,然而未尝能相为事也。用此观之,然则可以为,未必能也;虽不能,无害可以
为。然则能不能之与可不可,其不同远矣,其不可以相为明矣。

尧问于舜曰:``人情何如?''舜对曰:``人情甚不美,又何问焉!妻子具而孝
衰于亲,嗜欲得而信衰于友,爵禄盈而忠衰于君。人之情乎!人之情乎!甚不美,
又何问焉!唯贤者为不然。''

有圣人之知者,有士君子之知者,有小人之知者,有役夫之知者。多言则文而
类,终日议其所以,言之千举万变,其统类一也:是圣人之知也。少言则径而省,
论而法,若佚之以绳:是士君子之知也。其言也谄,其行也悖,其举事多悔:是小
人之知也。齐给便敏而无类,杂能旁魄而无用,析速粹孰而不急,不恤是非,不论
曲直,以期胜人为意,是役夫之知也。

有上勇者,有中勇者,有下勇者。天下有中,敢直其身;先王有道,敢行其意;
上不循于乱世之君,下不俗于乱世之民;仁之所在无贫穷,仁之所亡无富贵;天下
知之,则欲与天下同苦乐之;天下不知之,则傀然独立天地之间而不畏:是上勇也。
礼恭而意俭,大齐信焉,而轻货财;贤者敢推而尚之,不肖者敢援而废之:是中勇
也。轻身而重货,恬祸而广解苟免,不恤是非然不然之情,以期胜人为意:是下勇
也。

繁弱、钜黍古之良弓也;然而不得排檠则不能自正。桓公之葱,太公之阙,文
王之录,庄君之曶,阖闾之干将、莫邪、钜阙、辟闾,此皆古之良剑也;然而不加
砥厉则不能利,不得人力则不能断。骅骝、騹骥、纤离、绿耳,此皆古之良马也;
然而必前有衔辔之制,后有鞭策之威,加之以造父之驶,然后一日而致千里也。夫人虽有性质美而心辩知,必将求贤师而事之,择良友而友之。得贤师而事之,则所
闻者尧舜禹汤之道也;得良友而友之,则所见者忠信敬让之行也。身日进于仁义而
不自知也者,靡使然也。今与不善人处,则所闻者欺诬诈伪也,所见者污漫淫邪贪
利之行也,身且加于刑戮而不自知者,靡使然也。传曰:``不知其子视其友,不知
其君视其左右。''靡而已矣!靡而已矣!

\hypertarget{header-n104}{%
\subsection{君子}\label{header-n104}}

天子无妻,告人无匹也。四海之内无客礼,告无适也。足能行,待相者然后进;
口能言,待官人然后诏。不视而见,不听而聪,不言而信,不虑而知,不动而功,
告至备也。天子也者,埶至重,形至佚,心至愈,志无所诎,形无所劳,尊无上矣。
诗曰:``普天之下,莫非王土;率土之滨,莫非王臣。''此之谓也。

圣王在上,分义行乎下,则士大夫无流淫之行,百吏官人无怠慢之事,众庶百
姓无奸怪之俗,无盗贼之罪,莫敢犯上之大禁,天下晓然皆知夫盗窃之不可以为富
也,皆知夫贼害之不可以为寿也,皆知夫犯上之禁不可以为安也。由其道则人得其
所好焉,不由其道则必遇其所恶焉。是故刑罚綦省而威行如流,世晓然皆知夫为奸
则虽隐窜逃亡之由不足以免也,故莫不服罪而请。书云:``凡人自得罪。''此之谓
也。

故刑当罪则威,不当罪则侮;爵当贤则贵,不当贤则贱。古者刑不过罪,爵不
踰德。故杀其父而臣其子,杀其兄而臣其弟。刑罚不怒罪,爵赏不踰德,分然各以
其诚通。是以为善者劝,为不善者沮;刑罚綦省,而威行如流,政令致明,而化易
如神。传曰:``一人有庆,兆民赖之。''此之谓也。

乱世则不然:刑罚怒罪,爵赏踰德,以族论罪,以世举贤。故一人有罪,而三
族皆夷,德虽如舜,不免刑均,是以族论罪也。先祖当贤,后子孙必显,行虽如桀
纣,列从必尊,此以世举贤也。以族论罪,以世举贤,虽欲无乱,得乎哉!诗曰:
``百川沸腾,山冢崒崩,高岸为谷,深谷为陵。哀今之人,胡憯莫惩!''此之谓也。
论法圣王,则知所贵矣;以义制事,则知所利矣。论知所贵,则知所养矣;事知所
利,则动知所出矣。二者是非之本,得失之原也。故成王之于周公也,无所往
而不听,知所贵也。桓公之于管仲也,国事无所往而不用,知所利也。吴有伍子胥
而不能用,国至于亡,倍道失贤也。故尊圣者王,贵贤者霸,敬贤者存,慢贤者亡,
古今一也。故尚贤,使能,等贵贱,分亲疏,序长幼,此先王之道也。故尚贤使能,
则主尊下安;贵贱有等,则令行而不流;亲疏有分,则施行而不悖;长幼有序,则
事业捷成而有所休。故仁者,仁此者也;义者,分此者也;节者,死生此者也;忠
者,惇慎此者也;兼此而能之备矣;备而不矜,一自善也,谓之圣。不矜矣,夫故
天下不与争能,而致善用其功。有而不有也,夫故为天下贵矣。诗曰:``淑人君子,
其仪不忒;其仪不忒,正是四国。''此之谓也。

\hypertarget{header-n108}{%
\subsection{成相}\label{header-n108}}

请成相:世之殃,愚闇愚闇堕贤良!人主无贤,如瞽无相,何伥伥!请布基,慎圣人,愚而自专事不治。主忌苟胜,群臣莫谏,必逢灾。论臣过,反其施,尊主安国尚贤义。拒谏饰非,愚而上同,国必祸。曷谓``罢''?国多私,比周还主党与施。远贤近谗,忠臣蔽塞主埶移。曷谓``贤''?明君臣,上能尊主下爱民。主诚听之,天下为一海内宾。主之孽,谗人达,贤能遁逃国乃蹙。愚以重愚,闇以重闇,成为桀。世之灾,妒贤能,飞廉知政任恶来。卑其志意,大其园圃高其台。武王怒,师牧野,纣卒易乡启乃下。武王善之,封之于宋立其祖。世之衰,谗人归,比干见刳箕子累。武王诛之,吕尚招麾殷民怀。世之祸,恶贤士,子胥见杀百里徙。穆公任之,强配五伯六卿施。世之愚,恶大儒,逆斥不通孔子拘。展禽三绌,春申道缀,基毕输。请牧基,贤者思,尧在万世如见之。谗人罔极,险陂倾侧此之疑。基必施,辨贤罢,文武之道同伏戏,由之者治,不由者乱,何疑为?凡成相,辨法方,至治之极复后王。慎墨季惠,百家之说欺不详。治复一,修之吉,君子执之心如结,众人贰之,谗夫弃之,形是诘。水至平,端不倾,心术如此象圣人。人而有埶,直而用抴必参天。世无王,穷贤良,暴人刍豢,仁人糟糠;礼乐息灭,圣人隐伏,墨术行。治之经,礼与刑,君子以修百姓宁。明德慎罚,国家既治四海平。治之志,后埶富,君子诚之好以待。处之敦固,有深藏之,能远思。思乃精,志之荣,好而壹之神以成。精神相反,一而不贰、为圣人。治之道,美不老,君子由之佼以好。下以教诲子弟,上以事祖考。成相竭,辞不蹙,君子道之顺以达。宗其贤良,辨其殃孽。

请成相,道圣王,尧舜尚贤身辞让,许由善卷,重义轻利行显明。尧让贤,以为民,泛利兼爱德施均。辨治上下,贵贱有等明君臣。尧授能,舜遇时,尚贤推德天下治。虽有圣贤,适不遇世,孰知之?尧不德,舜不辞,妻以二女任以事。大人哉舜,南面而立万物备。舜授禹,以天下,尚得推贤不失序。外不避仇,内不阿亲,贤者予。禹劳心力,尧有德,干戈不用三苗服。举舜甽亩,任之天下,身休息。得后稷,五谷殖;夔为乐正鸟兽服;契为司徒,民知孝弟尊有德。禹有功,抑下鸿,辟除民害逐共工。北决九河,通十二渚,疏三江。禹傅土,平天下,躬亲为民行劳苦。得益、皋陶、横革、直成、为辅。契玄王,生昭明,居于砥石迁于商,十有四世,乃有天乙是成汤。天乙汤,论举当,身让卞随举牟光。道古贤圣基必张。

愿陈辞,世乱恶善不此治。隐过疾贤,长由奸诈鲜无灾。患难哉!阪为先,圣知不用愚者谋。前车已覆,后未知更,何觉时?不觉悟,不知苦,迷惑失指易上下。中不上达,蒙揜耳目塞门户。门户塞,大迷惑,悖乱昏莫不终极;是非反易,比周欺上恶正直。正直恶,心无度,邪枉辟回失道途。己无邮人,我独自美,岂独无故?不知戒,后必有,恨后遂过不肯悔。谗夫多进,反复言语生诈态。人之态,不如备,争宠嫉贤利恶忌;妒功毁贤,下歛党与上蔽匿。上壅蔽,失辅埶,任用谗夫不能制。郭公长父之难,厉王流于彘。周幽厉,所以败,不听规谏忠是害。嗟我何人,独不遇时当乱世!欲衷对,言不从,恐为子胥身离凶;进谏不听,刭而独鹿弃之江。观往事,以自戒,治乱是非亦可识。托于成相以喻意。

请成相,言治方,君论有五约以明。君谨守之,下皆平正,国乃昌。臣下职,莫游食,务本节用财无极。事业听上,莫得相使,一民力。守其职,足衣食,厚薄有等明爵服。利往卬上,莫得擅与,孰私得?君法明,论有常,表仪既设民知方。进退有律,莫得贵贱、孰私王?君法仪,禁不为,莫不说教名不移。修之者荣,离之者辱,孰它师?刑称陈,守其银,下不得用轻私门。罪祸有律,莫得轻重威不分。请牧基,明有祺,主好论议必善谋。五听修领,莫不理续主执持。听之经,明其请,参伍明谨施赏刑。显者必得,隐者复显,民反诚。言有节,稽其实,信诞以分赏刑必。下不欺上,皆以情言,明若日。上通利,隐远至,观法不法见不视。耳目既显,吏敬法令莫敢恣。君教出,行有律,吏谨将之无铍滑。下不私请,各以宜,舍巧拙。臣谨修,君制变,公察善思论不乱。以治天下,后世法之成律贯。

\hypertarget{header-n112}{%
\subsection{赋}\label{header-n112}}

爰有大物,非丝非帛,文理成章;非日非月,为天下明。生者以寿,死者以葬。
城郭以固,三军以强。粹而王,驳而伯,无一焉而亡。臣愚不识,敢请之王?

王曰:此夫文而不采者欤?简然易知,而致有理者欤?君子所敬,而小人所不
者欤?性不得则若禽兽,性得之则甚雅似者欤?匹夫隆之则为圣人,诸侯隆之则一
四海者欤?致明而约,甚顺而体,请归之礼。礼。

皇天隆物,以示施下民,或厚或薄,常不齐均。桀纣以乱,汤武以贤。涽涽淑
淑,皇皇穆穆。周流四海,曾不崇日。君子以修,跖以穿室。大参乎天,精微而无
形,行义以正,事业以成。可以禁暴足穷,百姓待之而后泰宁。臣愚不识,愿问其
名。

曰:此夫安宽平而危险隘者邪?修洁之为亲,而杂污之为狄者邪?甚深藏而外
胜敌者邪?法禹舜而能弇迹者邪?行为动静待之而后适者邪?血气之精也,志意之
荣也,百姓待之而后宁也,天下待之而后平也,明达纯粹而无疵也,夫是之谓君子
之知知。

有物于此,居则周静致下,动则綦高以钜,圆者中规,方者中矩,大参天地,
德厚尧禹,精微乎毫毛,而充盈乎大寓。忽兮其极之远也,攭兮其相逐而反也,卬
卬兮天下之咸蹇也。德厚而不捐,五采备而成文,往来惛惫,通于大神,出入甚极,
莫知其门。天下失之则灭,得之则存。弟子不敏,此之愿陈,君子设辞,请测意之。

曰:此夫大而不塞者与?充盈大宇而不窕,入却穴而不偪者与?行远疾速,而
不可托讯者与?往来惛惫,而不可为固塞者与?暴至杀伤,而不亿忌者与?功被天
下,而不私置者与?托地而游宇,友风而子雨,冬日作寒,夏日作暑,广大精神,
请归之云云。

有物于此,(人蠡)(人蠡)兮其状,屡化如神,功被天下,为万世文。礼乐以成,
贵贱以分,养老长幼,待之而后存。名号不美,与``暴''为邻。功立而身废,事成
而家败。弃其耆老,收其后世。人属所利,飞鸟所害。臣愚不识,请占之五泰。

五泰占之曰:此夫身女好,而头马首者与?屡化而不寿者与?善壮而拙老者与?
有父母而无牝牡者与?冬伏而夏游,食桑而吐丝,前乱而后治,夏生而恶暑,喜湿
而恶雨,蛹以为母,蛾以为父,三俯三起,事乃大已,夫是之谓蚕理。蚕

有物于此,生于山阜,处于室堂。无知无巧,善治衣裳。不盗不窃,穿窬而行。
日夜合离,以成文章。以能合从,又善连衡。下覆百姓,上饰帝王。功业甚博,不
见贤良。时用则存,不用则亡。臣愚不识,敢请之王。

王曰:此夫始生钜,其成功小者邪?长其尾而锐其剽者邪?头铦达而尾赵缭者
邪?一往一来,结尾以为事。无羽无翼,反复甚极。尾生而事起,尾邅而事已。簪
以为父,管以为母。既以缝表,又以连里:夫是之谓箴理。箴

天下不治,请陈佹诗:天地易位,四时易乡。列星殒坠,旦暮晦盲。幽闇登昭,
日月下藏。公正无私,见谓从横。志爱公利,重楼疏堂。无私罪人,憼革贰兵。道
德纯备,谗口将将。仁人绌约,敖暴擅强。天下幽险,恐失世英。螭龙为蝘蜓,鸱
枭为凤凰。比干见刳,孔子拘匡。昭昭乎其知之明也,郁郁乎其遇时之不祥也,拂
乎其欲礼义之大行也,闇乎天下之晦盲也,皓天不复,忧无疆也。千岁必反,古之
常也。弟子勉学,天不忘也。圣人共手,时几将矣。与愚以疑,愿闻反辞。

其小歌曰:念彼远方,何其塞矣,仁人绌约,暴人衍矣。忠臣危殆,谗人服矣。

琁、玉、瑶、珠,不知佩也,杂布与帛,不知异也。闾娵子奢,莫之媒也;嫫
母力父,是之喜也。以盲为明,以聋为聪,以危为安,以吉为凶。呜呼!上天!曷
维其同!

\hypertarget{header-n116}{%
\subsection{大略}\label{header-n116}}

君人者,隆礼尊贤而王,重法爱民而霸,好利多诈而危。欲近四旁,莫如中央,故王者必居天下之中,礼也。

天子外屏,诸侯内屏,礼也。外屏、不欲见外也;内屏、不欲见内也。

诸侯召其臣,臣不俟驾,颠倒衣裳而走,礼也。诗曰:``颠之倒之,自公召之。''天子召诸侯,诸侯辇舆就马,礼也。诗曰:``我出我舆,于彼牧矣。自天子所,谓我来矣。''

天子山冕,诸侯玄冠,大夫裨冕,士韦弁,礼也。

天子御珽,诸侯御荼,大夫服笏,礼也。

天子雕弓,诸侯彤弓,大夫黑弓,礼也。

诸侯相见,卿为介,以其教士毕行,使仁居守。

聘人以圭,问士以璧,召人以瑗,绝人以玦,反绝以环。

人主仁心设焉,知其役也,礼其尽也,故王者先仁而后礼,天施然也。

聘礼志曰:``币厚则伤德,财侈则殄礼。''礼云礼云,玉帛云乎哉!诗曰:``物其指矣,唯其偕矣。''不时宜,不敬文,不驩欣,虽指非礼也。

水行者表深,使人无陷;治民者表乱,使人无失,礼者,其表也。先王以礼义表天下之乱;今废礼者,是弃表也,故民迷惑而陷祸患,此刑罚之所以繁也。

舜曰:``维予从欲而治。''故礼之生,为贤人以下至庶民也,非为成圣也;然而亦所以成圣也,不学不成;尧学于君畴,舜学于务成昭,禹学于西王国。

五十不成丧,七十唯衰存。

亲迎之礼,父南向而立,子北面而跪,醮而命之:``往迎尔相,成我宗事,隆率以敬先妣之嗣,若则有常。''子曰:``诺!唯恐不能,敢忘命矣!''

夫行也者,行礼之谓也。礼也者,贵者敬焉,老者孝焉,长者弟焉,幼者慈焉,贱者惠焉。

赐予其宫室,犹用庆赏于国家也;忿怒其臣妾,犹用刑罚于万民也。

君子之于子,爱之而勿面,使之而勿视,道之以道而勿强。

礼以顺人心为本,故亡于礼经而顺于人心者,皆礼也。

礼之大凡:事生、饰驩也,送死、饰哀也,军旅、施威也。

亲亲、故故、庸庸、劳劳,仁之杀也;贵贵、尊尊、贤贤、老老、长长、义之伦也。行之得其节,礼之序也。仁、爱也,故亲;义、理也,故行;礼、节也,故成。仁有里,义有门;仁、非其里而处之,非仁也;义,非其门而由之,非义也。推恩而不理,不成仁;遂理而不敢,不成义;审节而不和,不成礼;和而不发,不成乐。故曰:仁义礼乐,其致一也。君子处仁以义,然后仁也;行义以礼,然后义也;制礼反本成末,然后礼也。三者皆通,然后道也。

货财曰赙,舆马曰赗,衣服曰襚,玩好曰赠,玉贝曰唅。赙赗、所以佐生也,赠襚、所以送死也。送死不及柩尸,吊生不及悲哀,非礼也。故吉行五十,奔丧百里,赗赠及事,礼之大也。

礼者、政之挽也;为政不以礼,政不行矣。

天子即位,上卿进曰:``如之何忧之长也?能除患则为福,不能除患则为贼。''授天子一策。中卿进曰:``配天而有下土者,先事虑事,先患虑患。先事虑事谓之接,接则事优成。先患虑患谓之豫,豫则祸不生。事至而后虑者谓之后,后则事不举。患至而后虑者谓之困,困则祸不可御。''授天子二策。下卿进曰:``敬戒无怠,庆者在堂,吊者在闾。祸与福邻,莫知其门。豫哉!豫哉!万民望之。''授天子三策。

禹见耕者耦、立而式,过十室之邑、必下。

杀大蚤,朝大晚,非礼也。治民不以礼,动斯陷矣。

平衡曰拜,下衡曰稽首,至地曰稽颡。

大夫之臣,拜不稽首,非尊家臣也,所以辟君也。

一命齿于乡,再命齿于族,三命,族人虽七十不敢先。上大夫,中大夫,下大夫。

吉事尚尊,丧事尚亲。

聘、问也。享、献也。私觌、私见也。

言语之美,穆穆皇皇。朝廷之美,济济鎗鎗。

为人臣下者,有谏而无讪,有亡而无疾,有怨而无怒。

君于大夫,三问其疾,三临其丧;于士,一问,一临。诸侯非问疾吊丧不之臣之家。

既葬,君若父之友食之则食矣,不辟梁肉,有醴酒则辞。

寝不踰庙,燕衣不踰祭服,礼也。

汤之咸,见夫妇。夫妇之道,不可不正也,君臣父子之本也。咸、感也,以高下下,以男下女,柔上而刚下。

聘士之义,亲迎之道,重始也。

礼者,人之所履也,失所履,必颠蹶陷溺。所失微而其为乱大者,礼也。

礼之于正国家也,如权衡之于轻重也,如绳墨之于曲直也。故人无礼不生,事无礼不成,国家无礼不宁。君臣不得不尊,父子不得不亲,兄弟不得不顺,夫妇不得不驩,少者以长,老者以养。故天地生之,圣人成之。

和鸾之声,步中武象,趋中韶护。君子听律习容而后出。

霜降逆女,冰泮杀止,十日一御。

坐视膝,立视足,应对言语视面。立视前六尺而大之六六三十六,三丈六尺。

文貌情用,相为内外表里。礼之中焉,能思索谓之能虑。

礼者,本末相顺,终始相应。

礼者,以财物为用,以贵贱为文,以多少为异。

下臣事君以货,中臣事君以身,上臣事君以人。

易曰:``复自道,何其咎?''春秋贤穆公,以为能变也。

士有妒友,则贤交不亲;君有妒臣,则贤人不至。蔽公者谓之昧,隐贤者谓之妒,奉妒昧者谓之交谲。交谲之人,妒昧之臣,国之薉孽也。

口能言之,身能行之,国宝也。口不能言,身能行之,国器也。口能言之,身不能行,国用也。口言善,身行恶,国妖也。治国者敬其宝,爱其器,任其用,除其妖。

不富无以养民情,不教无以理民性。故家五亩宅,百亩田,务其业,而勿夺其时,所以富之也。立大学,设庠序,修六礼,明七教,所以道之也。诗曰:``饮之食之,教之诲之。''王事具矣。

武王始入殷,表商容之闾,释箕子之囚,哭比干之墓,天下乡善矣。

天下国有俊士,世有贤人。迷者不问路,溺者不问遂,亡人好独。诗曰:``我言维服,勿用为笑。先民有言,询于刍荛。''言博问也。

有法者以法行,无法者以类举。以其本知其末,以其左知其右,凡百事异理而相守也。庆赏刑罚,通类而后应;政教习俗,相顺而后行。

八十者一子不事,九十者举家不事,废疾非人不养者,一人不事,父母之丧,三年不事,齐衰大功,三月不事,从诸侯来,与新有昏,期不事。

子谓子家驹续然大夫,不如晏子;晏子功用之臣也,不如子产;子产惠人也,不如管仲;管仲之为人,力功不力义,力知不力仁,野人也,不可为天子大夫。

孟子三见宣王,不言事。门人曰:``曷为三遇齐王而不言事?''孟子曰:``吾先攻其邪心。''

公行子之之燕,遇曾元于涂,曰:``燕君何如?''曾元曰:``志卑。志卑者轻物,轻物者不求助;苟不求助,何能举?氐羌之虏也,不忧其系垒也,而忧其不焚也。利夫秋毫,害靡国家,然且为之,几为知计哉!''

今夫亡箴者,终日求之而不得;其得之也,非目益明也,眸而见之也。心之于虑亦然。

``义''与``利''者,人之所两有也。虽尧舜不能去民之欲利;然而能使其欲利不克其好义也。虽桀纣不能去民之好义;然而能使其好义不胜其欲利也。故义胜利者为治世,利克义者为乱世。上重义则义克利,上重利则利克义。故天子不言多少,诸侯不言利害,大夫不言得丧,士不通货财。有国之君不息牛羊,错质之臣不息鸡豚,冢卿不修币,大夫不为场园,从士以上皆羞利而不与民争业,乐分施而耻积藏;然故民不困财,贫窭者有所窜其手。

文王诛四,武王诛二,周公卒业,至成康则案无诛已。

多积财而羞无有,重民任而诛不能,此邪行之所以起,刑罚之所以多也。

上好义,则民闇饰矣!上好富,则民死利矣!二者治乱之衢也。民语曰:``欲富乎?忍耻矣!倾绝矣!绝故旧矣!与义分背矣!''上好富,则人民之行如此,安得不乱!

汤旱而祷曰:``政不节与?使民疾与?何以不雨至斯极也!宫室荣与?妇谒盛与?何以不雨至斯之极也!苞苴行与?谗夫兴与?何以不雨至斯极也!''

天之生民,非为君也;天之立君,以为民也。故古者,列地建国,非以贵诸侯而已;列官职,差爵禄,非以尊大夫而已。

主道知人,臣道知事。故舜之治天下,不以事诏而万物成。农精于田,而不可以为田师,工贾亦然。

以贤易不肖,不待卜而后知吉。以治伐乱,不待战而后知克。

齐人欲伐鲁,忌卞庄子,不敢过卞。晋人欲伐卫,畏子路,不敢过蒲。

不知而问尧舜,无有而求天府。曰:先王之道,则尧舜已;六贰之博,则天府已。

君子之学如蜕,翻然迁之。故其行效,其立效,其置颜色、出辞气效。无留善,无宿问。

善学者尽其理,善行者究其难。

君子立志如穷,虽天子三公问正,以是非对。

君子隘穷而不失,劳倦而不苟,临患难而不忘细席之言。岁不寒无以知松柏,事不难无以知君子无日不在是。

雨小,汉故潜。夫尽小者大,积微者箸,德至者色泽洽,行尽而声问远,小人不诚于内而求之于外。

言而不称师谓之畔,教而不称师谓之倍。倍畔之人,明君不内,朝士大夫遇诸涂不与言。

不足于行者,说过;不足于信者,诚言。故春秋善胥命,而诗非屡盟,其心一也。善为诗者不说,善为易者不占,善为礼者不相,其心同也。

曾子曰:``孝子言为可闻,行为可见。言为可闻,所以说远也;行为可见,所以说近也;近者说则亲,远者悦则附;亲近而附远,孝子之道也。''

曾子行,晏子从于郊,曰:``婴闻之:君子赠人以言,庶人赠人以财。婴贫无财,请假于君子,赠吾子以言:乘舆之轮,太山之木也,示诸檃栝,三月五月,为帱采,敝而不反其常。君子之檃栝,不可不谨也。慎之!兰茞槁本,渐于蜜醴,一佩易之。正君渐于香酒,可谗而得也。君子之所渐,不可不慎也。''

人之于文学也,犹玉之于琢磨也。诗曰:``如切如磋,如琢如磨。''谓学问也。和之璧,井里之厥也,玉人琢之,为天子宝。子赣季路故鄙人也,被文学,服礼义,为天下列士。

学问不厌,好士不倦,是天府也。

君子疑则不言,未问则不言,道远日益矣。

多知而无亲,博学而无方,好多而无定者,君子不与。

少不讽诵,壮不论议,虽可,未成也。

君子壹教,弟子壹学,亟成。

君子进则益上之誉,而损下之忧。不能而居之,诬也;无益而厚受之,窃也。学者非必为仕,而仕者必如学。

子贡问于孔子曰:``赐倦于学矣,愿息事君。''孔子曰:``诗云:`温恭朝夕,执事有恪。'事君难,事君焉可息哉!''``然则,赐愿息事亲。''孔子曰:``诗云:`孝子不匮,永锡尔类。'事亲难,事亲焉可息哉!''``然则赐愿息于妻子。''孔子曰:``诗云:`刑于寡妻,至于兄弟,以御于家邦。'妻子难,妻子焉可息哉!''``然则赐愿息于朋友。''孔子曰:``诗云:`朋友攸摄,摄以威仪。'朋友难,朋友焉可息哉!''``然则赐愿息耕。''孔子曰:``诗云:`昼尔于茅,宵尔索绹,亟其乘屋,其始播百谷。'耕难,耕焉可息哉!''``然则赐无息者乎?''孔子曰:``望其圹,皋如也,颠如也,鬲如也,此则知所息矣。''子贡曰:``大哉!死乎!君子息焉,小人休焉。''

国风之好色也,传曰:``盈其欲而不愆其止。其诚可比于金石,其声可内于宗庙。''小雅不以于污上,自引而居下,疾今之政以思往者,其言有文焉,其声有哀焉。

国将兴,必贵师而重傅,贵师而重傅,则法度存。国将衰,必贱师而轻傅;贱师而轻傅,则人有快;人有快则法度坏。

古者匹夫五十而士。天子诸侯子十九而冠,冠而听治,其教至也。

君子也者而好之,其人也;其人而不教,不祥。非君子而好之,非其人也;非其人而教之,赍盗粮,借贼兵也。

不自嗛其行者,言滥过。古之贤人,贱为布衣,贫为匹夫,食则饘粥不足,衣则竖褐不完;然而非礼不进,非义不受,安取此?

子夏家贫,衣若县鹑。人曰:``子何不仕?''曰:``诸侯之骄我者,吾不为臣;大夫之骄我者,吾不复见。柳下惠与后门者同衣,而不见疑,非一日之闻也。争利如蚤甲,而丧其掌。''

君人者不可以不慎取臣,匹夫不可不慎取友。友者、所以相有也。道不同,何以相有也?均薪施火,火就燥;平地注水,水流湿。夫类之相从也,如此其着也,以友观人,焉所疑?取友善人,不可不慎,是德之基也。诗曰:``无将大车,维尘冥冥。''言无与小人处也。

蓝苴路作,似知而非。懦弱易夺,似仁而非。悍戆好斗,似勇而非。

仁义礼善之于人也,辟之若货财粟米之于家也,多有之者富,少有之者贫,至无有者穷。故大者不能,小者不为,是弃国捐身之道也。

凡物有乘而来,乘其出者,是其反也。

流言灭之,货色远之。祸之所由生也,生自纤纤也。是故君子蚤绝之。

言之信者,在乎区盖之间。疑则不言,未问则不言。

知者明于事,达于数,不可以不诚事也。故曰:``君子难说,说之不以道,不说也。''

语曰:``流丸止于瓯臾,流言止于知者。''此家言邪说之所以恶儒者也。是非疑,则度之以远事,验之以近物,参之以平心,流言止焉,恶言死焉。

曾子食鱼,有余,曰:``泔之。''门人曰:``泔之伤人,不若奥之。''曾子泣涕曰:``有异心乎哉!''伤其闻之晚也。

无用吾之所短,遇人之所长。故塞而避所短,移而从所仕。疏知而不法,辨察而操僻,勇果而无礼,君子之所憎恶也。

多言而类,圣人也;少言而法,君子也;多言无法,而流湎然,虽辩,小人也。

国法禁拾遗,恶民之串以无分得也,有夫分义,则容天下而治;无分义,则一妻一妾而乱。

天下之人,唯各特意哉,然而有所共予也。言味者予易牙,言音者予师旷,言治者予三王。三王既以定法度,制礼乐而传之,有不用而改自作,何以异于变易牙之和,更师旷之律?无三王之法,天下不待亡,国不待死。

饮而不食者,蝉也;不饮不食者,浮蝣也。

虞舜、孝己孝而亲不爱,比干、子胥忠而君不用,仲尼、颜渊知而穷于世。劫迫于暴国而无所辟之,则崇其善,扬其美,言其所长,而不称其所短也。

惟惟而亡者,诽也;博而穷者,訾也;清之而俞浊者,口也。

君子能为可贵,不能使人必贵己;能为可用,不能使人必用己。

诰誓不及五帝,盟诅不及三王,交质子不及五伯。

\hypertarget{header-n120}{%
\subsection{宥坐}\label{header-n120}}

孔子观于鲁桓公之庙,有欹器焉,孔子问于守庙者曰:``此为何器?''守庙者曰:``此盖为宥坐之器,''孔子曰:``吾闻宥坐之器者,虚则欹,中则正,满则覆。''孔子顾谓弟子曰:``注水焉。''弟子挹水而注之。中而正,满而覆,虚而欹,孔子喟然而叹曰:``吁!恶有满而不覆者哉!''子路曰:``敢问持满有道乎?''孔子曰:``聪明圣知,守之以愚;功被天下,守之以让;勇力抚世,守之以怯,富有四海,守之以谦:此所谓挹而损之之道也。''

孔子为鲁摄相,朝七日而诛少正卯。门人进问曰:``夫少正卯鲁之闻人也,夫子为政而始诛之,得无失乎,''孔子曰:``居,吾语女其故。人有恶者五,而盗窃不与焉:一曰:心达而险;二曰:行辟而坚;三曰:言伪而辩;四曰:记丑而博;五曰:顺非而泽此五者有一于人,则不得免于君子之诛,而少正卯兼有之。故居处足以聚徒成群,言谈足饰邪营众,强足以反是独立,此小人之桀雄也,不可不诛也。是以汤诛尹谐,文王诛潘止,周公诛管叔,太公诛华仕,管仲诛付里乙,子产诛邓析史付,此七子者,皆异世同心,不可不诛也。诗曰:`忧心悄悄,愠于群小。'小人成群,斯足忧也。''

孔子为鲁司寇,有父子讼者,孔子拘之,三月不别。其父请止,孔子舍之。季孙闻之,不说,曰:``是老也欺予。语予曰:为国家必以孝。今杀一人以戮不孝!又舍之。''冉子以告。孔子慨然叹曰:``呜呼!上失之,下杀之,其可乎?不教其民,而听其狱,杀不辜也。三军大败,不可斩也;狱犴不治,不可刑也,罪不在民故也。嫚令谨诛,贼也。今生也有时,歛也无时,暴也;不教而责成功,虐也。已此三者,然后刑可即也。书曰:`义刑义杀,勿庸以即,予维曰未有顺事。'言先教也。故先王既陈之以道,上先服之;若不可,尚贤以綦之;若不可,废不能以单之;綦三年而百姓从风矣。邪民不从,然后俟之以刑,则民知罪矣。诗曰:`尹氏大师,维周之氐;秉国之均,四方是维;天子是庳,卑民不迷。'是以威厉而不试,刑错而不用,此之谓也。今之世则不然:乱其教,繁其刑,其民迷惑而堕焉,则从而制之,是以刑弥繁,而邪不胜。三尺之岸而虚车不能登也,百仞之山任负车登焉,何则?陵迟故也。数仞之墙而民不踰也,百仞之山而竖子冯而游焉,陵迟故也。今之世陵迟已久矣,而能使民勿踰乎,诗曰:`周道如砥,其直如矢。君子所履,小人所视。眷焉顾之,潸焉出涕。'岂不哀哉!''

诗曰:``瞻彼日月,悠悠我思。道之云远,曷云能来。''子曰:``伊稽首不其有来乎?''

孔子观于东流之水。子贡问于孔子曰:``君子之所以见大水必观焉者,是何?''孔子曰:``夫水遍与诸生而无为也,似德。其流也埤下,裾拘必循其理,似义,其洸洸乎不淈尽,似道。若有决行之,其应佚若声响,其赴百仞之谷不惧,似勇。主量必平,似法。盈不求概,似正。淖约微达,似察。以出以入以就鲜絜,似善化。其万折也必东,似志。是故见大水必观焉。

孔子曰:``吾有耻也,吾有鄙也,吾有殆也:幼不能强学,老无以教之,吾耻之,去其故乡,事君而达,卒遇故人曾无旧言,吾鄙之;与小人处者,吾殆之也。''

孔子曰:``如垤而进,吾与之;如丘而止,吾已矣。''今学曾未如(月尤)赘,则具然欲为人师。

孔子南适楚,厄于陈蔡之间,七日不火食,藜羹不糁,弟子皆有饥色。子路进而问之曰:``由闻之:为善者天报之以福,为不善者天报之以祸,今夫子累德积义怀美,行之日久矣,奚居之隐也?''孔子曰:``由不识,吾语女。女以知者为必用邪?王子比干不见剖心乎!女以忠者为必用邪?关龙逢不见刑乎!女以谏者为必用邪?吴子胥不磔姑苏东门外乎!夫遇不遇者,时也;贤不肖者,材也;君子博学深谋,不遇时者多矣!由是观之,不遇世者众矣,何独丘也哉!且夫芷兰生于深林,非以无人而不芳。君子之学,非为通也,为穷而不困,忧而意不衰也,知祸福终始而心不惑也。夫贤不肖者,材也;为不为者,人也;遇不遇者,时也;死生者,命也。今有其人,不遇其时,虽贤,其能行乎?苟遇其时,何难之有!故君子博学深谋,修身端行,以俟其时。''孔子曰:``由!居!吾语女。昔晋公子重耳霸心生于曹,越王句践霸心生于会稽,齐桓公小白霸心生于莒。故居不隐者思不远,身不佚者志不广;女庸安知吾不得之桑落之下?''

子贡观于鲁庙之北堂,出而问于孔子曰:``乡者赐观于太庙之北堂,吾亦未辍,还复瞻被九盖皆继,被有说邪?匠过绝邪?''孔子曰:``太庙之堂亦尝有说,官致良工,因丽节文,非无良材也,盖曰贵文也。''

\hypertarget{header-n124}{%
\subsection{子道}\label{header-n124}}

入孝出弟,人之小行也。上顺下笃,人之中行也;从道不从君,从义不从父,
人之大行也。若夫志以礼安,言以类使,则儒道毕矣。虽尧舜不能加毫末于是矣。
孝子所不从命有三:从命则亲危,不从命则亲安,孝子不从命乃衷;从命则亲辱,
不从命则亲荣,孝子不从命乃义;从命则禽兽,不从命则修饰,孝子不从命乃敬。
故可以从命而不从,是不子也;未可以从而从,是不衷也;明于从不从之义,而能
致恭敬,忠信、端悫、以慎行之,则可谓大孝矣。传曰:``从道不从君,从义不从
父。''此之谓也。故劳苦、雕萃而能无失其敬,灾祸、患难而能无失其义,则不幸
不顺见恶而能无失其爱,非仁人莫能行。诗曰:``孝子不匮。''此之谓也。

鲁哀公问于孔子曰:``子从父命,孝乎?臣从君命,贞乎?''三问,孔子不对。
孔子趋出以语子贡曰:``乡者,君问丘也,曰:`子从父命,孝乎?臣从君命,贞
乎?'三问而丘不对,赐以为何如?''子贡曰:``子从父命,孝矣。臣从君命,贞
矣,夫子有奚对焉?''孔子曰:``小人哉!赐不识也!昔万乘之国,有争臣四人,
则封疆不削;千乘之国,有争臣三人,则社稷不危;百乘之家,有争臣二人,则宗
庙不毁。父有争子,不行无礼;士有争友,不为不义。故子从父,奚子孝?臣从君,
奚臣贞?审其所以从之之谓孝、之谓贞也。''

子路问于孔子曰:``有人于此,夙兴夜寐,耕耘树艺,手足胼胝,以养其亲,
然而无孝之名,何也?''孔子曰:``意者身不敬与?辞不逊与?色不顺与?古之人
有言曰:`衣与!缪与!不女聊。'今夙兴夜寐,耕耘树艺,手足胼胝,以养其亲,
无此三者,则何为而无孝之名也?意者所友非人邪?''孔子曰:``由志之,吾语女。
虽有国士之力,不能自举其身。非无力也,势不可也。故入而行不修,身之罪也;
出而名不章,友之过也。故君子入则笃行,出则友贤,何为而无孝之名也!''

子路问于孔子曰:``鲁大夫练而床,礼邪?''孔子曰:``吾不知也。''子路出,
谓子贡曰:``吾以为夫子无所不知,夫子徒有所不知。''子贡曰:``汝何问哉?''
子路曰:``由问:`鲁大夫练而床,礼邪?'夫子曰:`吾不知也。'''子贡曰:
``吾将为女问之。''子贡问曰:``练而床,礼邪?''孔子曰;``非礼也。''子贡出,
谓子路曰:``女谓夫子为有所不知乎!夫子徒无所不知。女问非也。礼:居是邑不
非其大夫。''

子路盛服而见孔子,孔子曰:``由,是裾裾何也?昔者江出于岷山,其始出也,
其源可以滥觞,及其至江之津也,不放舟,不避风,则不可涉也。非维下流水多邪?
今女衣服既盛,颜色充盈,天下且孰肯谏女矣!子路趋而出,改服而入,盖犹若也。
孔子曰:``由志之!吾语汝:奋于言者华,奋于行者伐,色知而有能者,小人也。
故君子知之曰知之,不知曰不知,言之要也;能之曰能之,不能曰不能,行之至也。
言要则知,行至则仁;既仁且知,夫恶有不足矣哉!''

子路入,子曰:``由!知者若何?仁者若何?''子路对曰:``知者使人知己,
仁者使人爱己。''子曰:``可谓士矣。''子贡入,子曰:``赐!知者若何?仁者若
何?''子贡对曰:``知者知人,仁者爱人。''子曰:``可谓士君子矣。''颜渊入,
子曰:``回!知者若何?仁者若何?''颜渊对曰:``知者自知,仁者自爱。''子曰:
``可谓明君子矣。''

子路问于孔子曰:``君子亦有忧乎?''孔子曰:``君子其未得也,则乐其意,既已得之,又乐其治。是以有终生之乐,无一日之忧。小人者其未得也,则忧不得;
既已得之,又恐失之。是以有终身之忧,无一日之乐也。''

\hypertarget{header-n128}{%
\subsection{法行}\label{header-n128}}

公输不能加于绳墨,圣人不能加于礼。礼者,众人法而不知,圣人法而知之。

曾子曰:``无内人之疏而外人之亲,无身不善而怨人,无刑己至而呼天。内人
之疏而外人之亲,不亦反乎!身不善而怨人,不亦远乎!刑己至而呼天,不亦晚乎!
诗曰:`涓涓源水,不雝不塞。毂已破碎,乃大其辐。事已败矣,乃重太息。'其
云益乎!''

曾子病,曾元持足,曾子曰:``元!志之!吾语汝。夫鱼鳖鼋鼍犹以渊为浅而
堀其中,鹰鸢犹以山为卑而增巢其上,及其得也必以饵。故君子能无以利害义,则
耻辱亦无由至矣。''

子贡问于孔子曰:``君子之所以贵玉而贱(王民)者,何也?为夫玉之少而(王民)
之多邪?''孔子曰:``恶!赐!是何言也!夫君子岂多而贱之,少而贵之哉!夫玉
者,君子比德焉。温润而泽,仁也;栗而理,知也;坚刚而不屈,义也;廉而不刿,
行也;折而不挠,勇也;瑕适并见,情也;扣之,其声清扬而远闻,其止辍然,辞
也。故虽有(王民)之雕雕,不若玉之章章。诗曰:`言念君子,温其如玉。'此之
谓也。''

曾子曰:``同游而不见爱者,吾必不仁也;交而不见敬者,吾必不长也;临财
而不见信者,吾必不信也。三者在身曷怨人!怨人者穷,怨天者无识。失之己而反
诸人,岂不亦迂哉!''

南郭惠子问于子贡曰:``夫子之门何其杂也?''子贡曰:``君子正身以俟,欲
来者不距,欲去者不止。且夫良医之门多病人,檃栝之侧多枉木,是以杂也。''

孔子曰:``君子有三恕:有君不能事,有臣而求其使,非恕也;有亲不能报,
有子而求其孝,非恕也;有兄不能敬,有弟而求其听令,非恕也。士明于此三恕,
则可以端身矣。''

孔子曰:``君子有三思而不可不思也:少而不学,长无能也;老而不教,死无
思也;有而不施,穷无与也。是故君子少思长,则学;老思死,则教;有思穷,则
施也。''

\hypertarget{header-n132}{%
\subsection{哀公}\label{header-n132}}

鲁哀公问于孔子曰:``吾欲论吾国之士,与之治国,敢问如何取之邪?''孔子
对曰:``生今之世,志古之道:居今之俗,服古之服;舍此而为非者,不亦鲜乎!''
哀公曰:``然则夫章甫絇屦,绅带而搢笏者,此贤乎?''孔子对曰:``不必然,夫
端衣玄裳,絻而乘路者,志不在于食荤;斩衰菅屦,杖而啜粥者,志不在于酒肉。
生今之世,志古之道;居今之俗,服古之服;舍此而为非者,虽有,不亦鲜乎!''
哀公曰:``善!''

孔子曰:``人有五仪:有庸人,有士,有君子,有贤人,有大圣。''哀公曰:
``敢问何如斯可谓庸人矣?''孔子对曰:``所谓庸人者,口不道善言,心不知邑邑;
不知选贤人善士托其身焉以为己忧;动行不知所务,止立不知所定;日选择于物,
不知所贵;从物如流,不知所归;五凿为正,心从而坏:如此则可谓庸人矣。''哀
公曰:``善!敢问何如斯可谓士矣?''孔子对曰:``所谓士者,虽不能尽道术,必
有率也;虽不能遍美善,必有处也。是故知不务多,务审其所知;言不务多,务审
其所谓;行不务多,务审其所由。故知既已知之矣,言既已谓之矣,行既已由之矣,
则若性命肌肤之不可易也。故富贵不足以益也,卑贱不足以损也:如此则可谓士矣。''
哀公曰:``善!敢问何如斯可谓之君子矣?''孔子对曰:``所谓君子者,言忠信而
心不德,仁义在身而色不伐,思虑明通而辞不争,故犹然如将可及者,君子也。''
哀公曰:``善!敢问何如斯可谓贤人矣?''孔子对曰:``所谓贤人者,行中规绳而
不伤于本,言足法于天下而不伤于身,富有天下而无怨财,布施天下而不病贫:如
此则可谓贤人矣。''哀公曰:``善!敢问何如斯可谓大圣矣?''孔子对曰:``所谓
大圣者,知通乎大道,应变而不穷,辨乎万物之情性者也。大道者,所以变化遂成
万物也;情性者,所以理然不取舍也。是故其事大辨乎天地,明察乎日月,总要万
物于风雨,缪缪肫肫,其事不可循,若天之嗣,其事不可识,百姓浅然不识其邻:
若此则可谓大圣矣。''哀公曰:``善!''

鲁哀公问舜冠于孔子,孔子不对。三问不对。哀公曰:``寡人问舜冠于子,何
以不言也?''孔子曰:``古之王者,有务而拘领者矣,其政好生而恶杀焉。是以凤
在列树,麟在郊野,乌鹊之巢可俯而窥也。君不此问,而问舜冠,所以不对也。''

鲁哀公问于孔子曰:``寡人生于深宫之中,长于妇人之手,寡人未尝知哀也,
未尝知忧也,未尝知劳也,未尝知惧也,未尝知危也。''孔子曰:``君之所问,圣
君之问也,丘、小人也,何足以知之?''曰:``非吾子无所闻之也。''孔子曰:
``君入庙门而右,登自胙阶,仰视榱栋,俯见几筵,其器存,其人亡,君以此思哀,
则哀将焉而不至矣?君昧爽而栉冠,平明而听朝,一物不应,乱之端也,君以此思
忧,则忧将焉而不至矣?君平明而听朝,日昃而退,诸侯之子孙必有在君之末庭者,
君以思劳,则劳将焉而不至矣?君出鲁之四门,以望鲁四郊,亡国之虚则必有数盖
焉,君以此思惧,则惧将焉而不至矣?且丘闻之,君者,舟也;庶人者,水也。水
则载舟,水则覆舟,君以此思危,则危将焉而不至矣?''

鲁哀公问于孔子曰:``绅委章甫有益于仁乎?''孔子蹴然曰:``君号然也?资
衰苴杖者不听乐,非耳不能闻也,服使然也。黼衣黼裳者不茹荤,非口不能味也,
服使然也。且丘闻之,好肆不守折,长者不为市。窃其有益与其无益,君其知之矣。''

鲁哀公问于孔子曰:``请问取人。''孔子对曰:``无取健,无取詌,无取口啍。
健、贪也;詌、乱也;口啍、诞也。故弓调而后求劲焉,马服而后求良焉,士信悫
而后求知能焉。士不信尒而有多知能,譬之其豺狼也,不可以身尒也。语曰:`桓
公用其贼,文公用其盗。故明主任计不信怒,闇主信怒不任计。计胜怒则强,怒胜
计则亡。''

定公问于颜渊曰:``子亦闻东野毕之善驭乎?''颜渊对曰:``善则善矣,虽然,
其马将失。''定公不悦,入谓左右曰:``君子固谗人乎!。''三日而校来谒,曰:
``东野毕之马失。两骖列,两服入厩。''定公越席而起曰:``趋驾召颜渊!''颜渊
至,定公曰:``前日寡人问吾子,吾子曰:`东野毕之驶善则善矣,虽然,其马将
失。'不识吾子何以知之?''颜渊对曰:``臣以政知之。昔舜巧于使民,而造父巧
于使马;舜不穷其民,造父不穷其马;是以舜无失民,造父无失马。今东野毕之驭,
上车执辔衔,体正矣;步骤驰骋,朝礼毕矣;历险致远,马力尽矣;然犹求马不已,
是以知之也。''定公曰:``善,可得少进乎?''颜渊对曰:``臣闻之,鸟穷则啄,
兽穷则攫,人穷则诈。自古及今,未有穷其下而能无危者也。''

\hypertarget{header-n136}{%
\subsection{尧问}\label{header-n136}}

尧问于舜曰:``我欲致天下,为之奈何?''对曰:``执一无失,行微无怠,忠
信无倦,而天下自来。执一如天地,行微如日月,忠诚盛于内,贲于外,形于四海,
天下其在一隅邪!夫有何足致也!''

魏武侯谋事而当,群臣莫能逮,退朝而有喜色。吴起进曰:``亦尝有以楚庄王
之语,闻于左右者乎?''武侯曰:``楚庄王之语何如?''吴起对曰:``楚庄王谋事
而当,群臣莫能逮,退朝有忧色。申公巫臣进问曰:`王朝而有忧色,何也?'庄
王曰:`不谷谋事而当,群臣莫能逮,是以忧也。其在中蘬之言也,曰:``诸侯自
为得师者王,得友者霸,得疑者存,自为谋而莫己若者亡。''今以不谷之不肖,而
群臣莫能逮,吾国几于亡乎!是以忧也。'楚庄王以忧,而君以喜。''武侯逡巡再
拜曰:``天使夫子振寡人之过也。''

伯禽将归于鲁,周公谓伯禽之傅曰:``汝将行,盍志而子美德乎?''对曰:
``其为人宽,好自用以慎。此三者,其美德也。''周公曰:``呜呼!以人恶为美德
乎?君子好以道德,故其民归道。彼其宽也,出无辨矣,女又美之!彼其好自用也,
是所以窭小也。君子力如牛,不与牛争力;走如马,不与马争走;知如士,不与士
争知。彼争者均者之气也,女又美之!彼其慎也,是其所以浅也。闻之曰:`无越
踰不见士。'见士问曰:`无乃不察乎?'不闻即物少至,少至则浅。彼浅者,贱
人之道也,女又美之!吾语女:我、文王之为子,武王之为弟,成王之为叔父,吾
于天下不贱矣;然而吾所执贽而见者十人,还贽而相见者三十人,貌执之士者百有
余人,欲言而请毕事者千有余人,于是吾仅得三士焉,以正吾身,以定天下。吾所
以得三士者,亡于十人与三十人中,乃在百人与千人之中。故上士吾薄为之貌,下
士吾厚为之貌,人人皆以我为越踰好士,然故士至;士至而后见物,见物然后知其
是非之所在。戒之哉!女以鲁国骄人,几矣!夫仰禄之士犹可骄也,正身之士不可
骄也。彼正身之士,舍贵而为贱,舍富而为贫,舍佚而为劳,颜色黎黑而不失其所,
是以天下之纪不息,文章不废也。''

语曰:缯丘之封人,见楚相孙叔敖曰:``吾闻之也:处官久者士妒之,禄厚者
民怨之,位尊者君恨之。为相国有此三者,而不得罪于楚之士民何也?''孙叔敖曰:
``吾三相楚而心愈卑,每益禄而施愈博,位滋尊而礼愈恭,是以不得罪于楚之士民
也。''

子贡问于孔子曰:``赐为人下而未知也。''孔子曰:``为人下者乎?其犹土也。
深抇之而得甘泉焉,树之而五谷蕃焉,草木殖焉,禽兽育焉;生则立焉,死则入焉;
多其功,而不``息''德。为人下者其犹土也。''

昔虞不用宫之奇而晋幷之,莱不用子马而齐幷之,纣刳王子比干而武王得之。
不亲贤用知,故身死国亡也。

为说者曰:``孙卿不及孔子。''是不然。孙卿迫于乱世,遒于严刑,上无贤主,
下遇暴秦,礼义不行,教化不成,仁者绌约,天下冥冥,行全刺之,诸侯大倾。当
是时也,知者不得虑,能者不得治,贤者不得使。故君上蔽而无睹,贤人距而不受。
然则孙卿怀将圣之心,蒙佯狂之色,视天下以愚。诗曰:``既明且哲,以保其身。''
此之谓也。是其所以名声不白,徒与不众,光辉不博也。今之学者,得孙卿之遗言
余教,足以为天下法式表仪。所存者神,所过者化,观其善行,孔子弗过。世不详
察,云非圣人,奈何!天下不治,孙卿不遇时也。德若尧禹,世少知之;方术不用,
为人所疑;其知至明,循道正行,足以为纪纲。呜呼!贤哉!宜为帝王。天地不知,
善桀纣,杀贤良,比干剖心,孔子拘匡,接舆避世,箕子佯狂,田常为乱,阖闾擅
强。为恶得福,善者有殃。今为说者,又不察其实,乃信其名。时世不同,誉何由
生?不得为政,功安能成?志修德厚,孰谓不贤乎!

\end{document}
