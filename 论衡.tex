\PassOptionsToPackage{unicode=true}{hyperref} % options for packages loaded elsewhere
\PassOptionsToPackage{hyphens}{url}
%
\documentclass[]{article}
\usepackage{lmodern}
\usepackage{amssymb,amsmath}
\usepackage{ifxetex,ifluatex}
\usepackage{fixltx2e} % provides \textsubscript
\ifnum 0\ifxetex 1\fi\ifluatex 1\fi=0 % if pdftex
  \usepackage[T1]{fontenc}
  \usepackage[utf8]{inputenc}
  \usepackage{textcomp} % provides euro and other symbols
\else % if luatex or xelatex
  \usepackage{unicode-math}
  \defaultfontfeatures{Ligatures=TeX,Scale=MatchLowercase}
\fi
% use upquote if available, for straight quotes in verbatim environments
\IfFileExists{upquote.sty}{\usepackage{upquote}}{}
% use microtype if available
\IfFileExists{microtype.sty}{%
\usepackage[]{microtype}
\UseMicrotypeSet[protrusion]{basicmath} % disable protrusion for tt fonts
}{}
\IfFileExists{parskip.sty}{%
\usepackage{parskip}
}{% else
\setlength{\parindent}{0pt}
\setlength{\parskip}{6pt plus 2pt minus 1pt}
}
\usepackage{hyperref}
\hypersetup{
            pdfborder={0 0 0},
            breaklinks=true}
\urlstyle{same}  % don't use monospace font for urls
\setlength{\emergencystretch}{3em}  % prevent overfull lines
\providecommand{\tightlist}{%
  \setlength{\itemsep}{0pt}\setlength{\parskip}{0pt}}
\setcounter{secnumdepth}{0}
% Redefines (sub)paragraphs to behave more like sections
\ifx\paragraph\undefined\else
\let\oldparagraph\paragraph
\renewcommand{\paragraph}[1]{\oldparagraph{#1}\mbox{}}
\fi
\ifx\subparagraph\undefined\else
\let\oldsubparagraph\subparagraph
\renewcommand{\subparagraph}[1]{\oldsubparagraph{#1}\mbox{}}
\fi

% set default figure placement to htbp
\makeatletter
\def\fps@figure{htbp}
\makeatother


\date{}

\begin{document}

\hypertarget{header-n0}{%
\section{论衡}\label{header-n0}}

\begin{center}\rule{0.5\linewidth}{\linethickness}\end{center}

\tableofcontents

\begin{center}\rule{0.5\linewidth}{\linethickness}\end{center}

\hypertarget{header-n6}{%
\subsection{卷一}\label{header-n6}}

\hypertarget{header-n7}{%
\subsubsection{逢遇篇}\label{header-n7}}

操行有常贤,仕宦无常遇。贤不贤,才也;遇不遇,时也。才高行洁,不可保以必尊贵;能薄操浊,不可保以必卑贱。或高才洁行,不遇退在下流;薄能浊操,遇,在众上。世各自有以取士,士亦各自得以进。进在遇,退在不遇。处尊居显,未必贤,遇也;位卑在下,未必愚,不遇也。故遇,或抱洿行,尊於桀之朝;不遇,或持洁节,卑於尧之廷。所以遇不遇非一也:或时贤而辅恶;或以大才从於小才;或俱大才,道有清浊;或无道德而以技合;或无技能,而以色幸。

伍员、帛喜,俱事夫差,帛喜尊重,伍员诛死。此异操而同主也。或操同而主异,亦有遇不遇,伊尹、箕子,是也。伊尹、箕子才俱也,伊尹为相,箕子为奴;伊尹遇成汤,箕子遇商纣也。夫以贤事贤君,君欲为治,臣以贤才辅之,趋舍偶合,其遇固宜;以贤事恶君,君不欲为治,臣以忠行佐之,操志乖忤,不遇固宜。

或以贤圣之臣,遭欲为治之君,而终有不遇,孔子、孟轲是也。孔子绝粮陈、蔡,孟轲困於齐、梁,非时君主不用善也,才下知浅,不能用大才也。夫能御骥騄者,必王良也;能臣禹、稷、皋陶者,必尧、舜也。御百里之手,而以调千里之足,必摧衡折轭之患;有接具臣之才,而以御大臣之知,必有闭心塞意之变。故至言弃捐,圣贤距逆,非憎圣贤,不甘至言也。圣贤务高,至言难行也。夫以大才干小才,小才不能受,不遇固宜。

或以大才之臣,遇大才之主,乃有遇不遇,虞舜、许由、太公、伯夷是也。虞舜、许由俱圣人也,并生唐世,俱面於尧。虞舜绍帝统,许由入山林。太公、伯夷俱贤也,并出周国,皆见武王;太公受封,伯夷饿死。夫贤圣道同,志合趋齐,虞舜、太公行耦,许由、伯夷操违者,生非其世。出非其时也。道虽同,同中有异,志虽合,合中有离。何则?道有精粗,志有清浊也。许由,皇者之辅也,生於帝者之时;伯夷,帝者之佐也,出於王者之世,并由道德,俱发仁义,主行道德,不清不留;主为仁义,不高不止,此其所以不遇也。尧溷,舜浊;武王诛残,太公讨暴,同浊皆粗,举措均齐,此其所以为遇者也。故舜王天下,皋陶佐政,北人无择深隐不见;禹王天下,伯益辅治,伯成子高委位而耕。非皋陶才愈无择,伯益能出子高也,然而皋陶、伯益进用,无择、子高退隐,进用行耦,退隐操违也。退隐势异,身虽屈,不愿进;人主不须其言,废之,意亦不恨,是两不相慕也。

商鞅三说秦孝公,前二说不听,後一说用者:前二,帝王之论;後一,霸者之议也。夫持帝王之论,说霸者之主,虽精见距;更调霸说,虽粗见受。何则?精遇孝公所不欲得,粗遇孝公所欲行也。故说者不在善,在所说者善之;才不待贤,在所事者贤之。马圄之说无方,而野人说之;子贡之说有义,野人不听。吹籁工为善声,因越王不喜,更为野声,越王大说。故为善於不欲得善之主,虽善不见爱;为不善於欲得不善之主,虽不善不见憎。此以曲伎合,合则遇,不合则不遇。

或无伎,妄以奸巧合上志,亦有以遇者,窃簪之臣,鸡鸣之客是。窃簪之臣,亲於子反,鸡鸣之客,幸於孟尝。子反好偷臣,孟尝爱伪客也。以有补於人君,人君赖之,其遇固宜。或无补益,为上所好,籍孺、邓通是也。籍孺幸於孝惠,邓通爱於孝文,无细简之才,微薄之能,偶以形佳骨娴,皮媚色称。夫好容,人所好也,其遇固宜。或以丑面恶色,称媚於上,嫫母、无盐是也。嫫母进於黄帝,无盐纳於齐王。故贤不肖可豫知,遇难先图。何则?人主好恶无常,人臣所进无豫,偶合为是,适可为上。进者未必贤,退者未必愚;合幸得进,不幸失之。

世俗之议曰:``贤人可遇,不遇,亦自其咎也:生不希世准主,观鉴治内,调能定说,审词际会。能进有补赡主,何不遇之有?今则不然,作无益之能,纳无补之说,以夏进炉,以冬奏扇,为所不欲得之事,献所不欲闻之语,其不遇祸幸矣,何福佑之有乎?''

进能有益,纳说有补,人之所知也;或以不补而得佑,或以有益而获罪。且夏时炉以炙湿,冬时扇以火。世可希,主不可准也;说可转,能不可易也。世主好文,己为文则遇;主好武,己则不遇。主好辩,有口则遇;主不好辩,己则不遇。文主不好武,武主不好文;辩主不好行,行主不好辩。文与言,尚可暴习;行与能,不可卒成。学不宿习,无以明名。名不素著,无以遇主。仓猝之业,须臾之名,日力不足。不预闻,何以准主而纳其说,进身而托其能哉?昔周人有仕数不遇,年老白首,泣涕於涂者。人或问之:``何为泣乎?''对曰:``吾仕数不遇,自伤年老失时,是以泣也。''人曰:``仕奈何不一遇也?''对曰:``吾年少之时,学为文。文德成就,始欲仕宦,人君好用老。用老主亡,後主又用武,吾更为武。武节始就,武主又亡。少主始立,好用少年,吾年又老,是以未尝一遇。''仕宦有时,不可求也。夫希世准主,尚不可为,况节高志妙,不为利动,性定质成,不为主顾者乎?

且夫遇也,能不预设,说不宿具,邂逅逢喜,遭触上意,故谓之遇。如准主调说,以取尊贵,是名为揣,不名曰遇。春种谷生,秋刈谷收,求物物得,作事事成,不名为遇。不求自至,不作自成,是名为遇。犹拾遗於涂,摭弃於野,若天授地生,鬼助神辅,禽息之精阴庆,鲍叔之魂默举,若是者,乃遇耳。今俗人即不能定遇不遇之论,又就遇而誉之,因不遇而毁之,是据见效,案成事,不能量操审才能也。

\hypertarget{header-n19}{%
\subsubsection{累害篇}\label{header-n19}}

凡人仕宦有稽留不进,行节有毁伤不全,罪过有累积不除,声名有暗昧不明,才非下,行非悖也;又知非昬,策非昧也;逢遭外祸,累害之也。非唯人行,凡物皆然,生动之类,咸被累害。累害自外,不由其内。夫不本累害所从生起,而徒归责於被累害者,智不明,暗塞於理者也.物以春生,人保之;以秋成,人必不能保之。卒然牛马践根,刀镰割茎,生者不育,至秋不成。不成之类,遇害不遂,不得生也。夫鼠涉饭中,捐而不食。捐饭之味,与彼不污者钧,以鼠为害,弃而不御。君子之累害,与彼不育之物,不御之饭,同一实也,俱由外来,故为累害。

修身正行,不能来福;战栗戒慎,不能避祸。祸福之至,幸不幸也。故曰:得非己力,故谓之福;来不由我,故谓之祸。不由我者,谓之何由?由乡里与朝廷也。夫乡里有三累,朝廷有三害。累生於乡里,害发於朝廷,古今才洪行淑之人遇此多矣。

何谓三累三害?

凡人操行,不能慎择友,友同心恩笃,异心疏薄,疏薄怨恨,毁伤其行,一累也。人才高下,不能钧同,同时并进,高者得荣,下者惭恚,毁伤其行,二累也。人之交游,不能常欢,欢则相亲,忿则疏远,疏远怨恨,毁伤其行,三累也。位少人众,仕者争进,进者争位,见将相毁,增加傅致,将昧不明,然纳其言,一害也。将吏异好,清浊殊操,清吏增郁郁之白,举涓涓之言,浊吏怀恚恨,徐求其过,因纤微之谤,被以罪罚,二害也。将或幸佐吏之身,纳信其言,佐吏非清节,必拔人越次。迕失其意,毁之过度;清正之仕,抗行伸志,遂为所憎,毁伤於将,三害也。夫未进也,身被三累;已用也,身蒙三害,虽孔丘、墨翟不能自免,颜回、曾参不能全身也。

动百行,作万事,嫉妒之人,随而云起,枳棘钩挂容体,蜂虿之党,啄螫怀操岂徒六哉!六者章章,世曾不见。夫不原士之操行有三累,仕宦有三害,身完全者谓之洁,被毁谤者谓之辱;官升进者谓之善,位废退者谓之恶。完全升进,幸也,而称之;毁谤废退,不遇也,而訾之:用心若此,必为三累三害也。

论者既不知累害(所从生,又不知被累害)者行贤洁也,以涂博泥,以黑点缯,孰有知之?清受尘,白取垢,青蝇所污,常在练素。处颠者危,势丰者亏,颓坠之类,常在悬垂。屈平洁白,邑犬群吠,吠所怪也,非俊疑杰,固庸能也。伟士坐以俊杰之才,招致群吠之声。夫如是,岂宜更勉奴下,循不肖哉?不肖奴下,非所勉也,岂宜更偶俗全身以弭谤哉?偶俗全身,则乡原也。乡原之人,行全无阙,非之无举,刺之无刺也。此又孔子之所罪,孟轲之所愆也。

古贤美极,无以卫身。故循性行以俟累害者,果贤洁之人也!极累害之谤,而贤洁之实见焉。立贤洁之迹,毁谤之尘安得不生?弦者思折伯牙之指,御者愿摧王良之手。何则?欲专良善之名,恶彼之胜己也。是故魏女色艳,郑袖劓之;朝吴忠贞,无忌逐之。戚施弥妒,蘧除多佞。是故湿堂不洒尘,卑屋不蔽风;风冲之物不得育,水湍之岸不得峭。如是,牖里、陈蔡可得知,而沉江蹈河也。以轶才取容媚於俗,求全功名於将,不遭邓析之祸,取子胥之诛,幸矣。孟贲之尸,人不刃者,气绝也。死灰百斛,人不沃者,光灭也。动身章智,显光气於世;奋志敖党,立卓异於俗,固常通人所谗嫉也。以方心偶俗之累,求益反损,盖孔子所以忧心,孟轲所以惆怅也。

德鸿者招谤,为士者多口。以休炽之声,弥口舌之患,求无危倾之害,远矣。臧仓之毁未尝绝也,公伯寮之溯未尝灭也。垤成丘山,污为江河矣。夫如是市虎之讹,投杼之误,不足怪,则玉变为石,珠化为砾,不足诡也。何则?昧心冥冥之知使之然也。文王所以为粪土,而恶来所以为金玉也,非纣憎圣而好恶也,心知惑蔽。蔽惑不能审,则微子十去,比干五剖,未足痛也。故三监谗圣人,周公奔楚。後母毁孝子,伯奇放流。当时周世孰有不惑乎?後《鸱鸮》作,而《黍离》兴,讽咏之者,乃悲伤之。故无雷风之变,周公之恶不灭;当夏不陨霜,邹衍之罪不除。德不能感天,诚不能动变,君子笃信审己也,安能遏累害於人?

圣贤不治名,害至不免辟,形章墨短,掩匿白长;不理身冤,不弭流言,受垢取毁,不求洁完,故恶见而善不彰,行缺而迹不显。邪伪之人,治身以巧俗,修诈以偶众。犹漆盘盂之工,穿墙不见;弄丸剑之倡,手指不知也。世不见短,故共称之;将不闻恶,故显用之。夫如是,世俗之所谓贤洁者,未必非恶;所谓邪污者,未必非善也。

或曰:``言有招患,行有召耻,所在常由小人。''夫小人性患耻者也,含邪而生,怀伪而游,沐浴累害之中,何招召之有?故夫火生者不伤湿,水居者无溺患。火不苦热,水不痛寒,气性自然焉,招之?君子也,以忠言招患,以高行招耻,何世不然?

然而太山之恶,君子不得名;毛发之善,小人不得有也。以玷污言之,清受尘而白取垢;以毁谤言之,贞良见妒,高奇见噪;以遇罪言之,忠言招患,高行招耻;以不纯言之,玉有瑕而珠有毁。焦陈留君兄,名称兗州,行完迹洁,无纤芥之毁;及其当为从事,刺史焦康绌而不用。何则?众好纯誉之人,非真贤也。公侯已下,玉石杂糅。贤士之行,善恶相苞。夫采玉者破石拔玉,选士者弃恶取善。夫如是,累害之人负世以行,指击之者从何往哉?

\hypertarget{header-n33}{%
\subsubsection{命禄篇}\label{header-n33}}

凡人遇偶及遭累害,皆由命也。有死生寿夭之命,亦有贵贱贫富之命。自王公逮庶人,圣贤及下愚,凡有首目之类,含血之属,莫不有命。命当贫贱,虽富贵之,犹涉祸患矣。命当富贵,虽贫贱之,犹逢福善矣。故命贵从贱地自达,命贱从富位自危。故夫富贵若有神助,贫贱若有鬼祸。命贵之人,俱学独达,并仕独迁;命富之人,俱求独得,并为独成。贫贱反此,难达,难迁,难得,难成;获过受罪,疾病亡遗,失其富贵,贫贱矣。是故才高行厚,未必保其必富贵;智寡德薄,未可信其必贫贱。或时才高行厚,命恶,废而不进;知寡德薄,命善,兴而超逾。故夫临事知愚,操行清浊,性与才也;仕宦贵贱,治产贫富,命与时也。命则不可勉,时则不可力,知者归之於天,故坦荡恬忽。虽其贫贱。

使富贵若凿沟伐薪,加勉力之趋,致强健之势,凿不休则沟深,斧不止则薪多,无命之人,皆得所愿,安得贫贱凶危之患哉?然则,或时沟未通而遇湛,薪未多而遇虎。仕宦不贵,治产之富,凿沟遇湛、伐薪逢虎之类也。

有才不得施,有智不得行,或施而功不立,或行而事不成,虽才智如孔子,犹无成立之功。世俗见人节行高,则曰:``贤哲如此,何不贵?''见人谋虑深,则曰:``辩慧如此,何不富?''贵富有命禄,不在贤哲与辩慧。故曰:富不可以筹策得,贵不可以才能成。智虑深而无财,才能高而无官。怀银纾紫,未必稷、契之才;积金累玉,未必陶硃之智。或时下愚而千金,顽鲁而典城。故官御同才,其贵殊命;治生钧知,其富异禄。禄命有贫富,知不能丰杀;命有贵贱,才不能进退。成王之才不如周公,桓公之知不若管仲,然成、桓受尊命,而周、管禀卑秩也。案古人君希有不学於人臣,知博希有不为父师。然而人君犹以无能处主位,人臣犹以鸿才为厮役。故贵贱在命,不在智愚;贫富在禄,不在顽慧。世之论事者以才高当为将相,能下者宜为农商,见智能之士官位不至,怪而訾之曰:``是必毁於行操。''行操之士亦怪毁之曰:``是必乏於才知。''殊不知才知行操虽高,官位富禄有命。才智之人,以吉盛时举事而福至,人谓才智明审;凶哀祸来,谓愚暗。不知吉凶之命,盛衰之禄也。

白圭、子贡,转货致富,积累金玉,人谓术善学明。主父偃辱贱於齐,排摈不用;赴阙举疏,遂用於汉,官至齐相。赵人徐乐亦上书,与偃章会,上善其言,征拜为郎。人谓偃之才,乐之慧,非也。儒者明说一经,习之京师,明如匡稚圭,深如赵子都,初阶甲乙之科,迁转至郎博士,人谓经明才高所得,非也。而说若范雎之干秦明,封为应侯;蔡泽之说范雎,拜为客卿,人谓雎、泽美善所致,非也。皆命禄贵富善至之时也。孔子曰:``死生有命,富贵在天。''鲁平公欲见孟子,嬖人臧仓毁孟子而止。孟子曰:``天也!''孔子圣人,孟子贤者,诲人安道,不失是非,称言命者,有命审也。

《淮南书》曰:``仁鄙在时不在行,利害在命黥不在智。''贾生曰:``天不可与期,道不可与谋,迟速有命,焉识其时?''高祖击布,为流矢所中,疾甚。吕后迎良医,医曰:``可治。''高祖骂之曰:``吾以布衣提三尺剑取天下,此非天命乎!命乃在天,虽扁鹊何益?''韩信与帝论兵,谓高祖曰:``陛下所谓天授,非智力所得。''扬子云曰:``遇不遇,命也。''太史公曰:``富贵不违贫贱,贫贱不违富贵。''是谓从富贵为贫贱,从贫贱为富贵也。

夫富贵不欲为贫贱,贫贱自至;贫贱不求为富贵,富贵自得也。春夏囚死,秋冬王相,非能为之也;日朝出而暮入,非求之也,天道自然。代王自代入为文帝,周亚夫以庶子为条侯,此时代王非太子,亚夫非适嗣,逢时遇会,卓然卒至。命贫以力勤致富,富至而死;命贱以才能取贵,贵至而免。才力而致富贵,命禄不能奉持,犹器之盈量,手之持重也。器受一升,以一升则平,受之如过一升,则满溢也;手举一钧,以一钧则平,举之过一钧,则踬仆矣。前世明是非归之於命也,命审然也。

信命者,则可幽居俟时,不须劳精苦形求索之也。犹珠玉之在山泽,天命难知,人不耐审,虽有厚命,犹不自信,故必求之也。如自知,虽逃富避贵,终不得离。故曰:力胜贫,慎胜祸。勉力勤事以致富,砥才明操以取贵;废时失务,欲望富贵,不可得也。虽云有命,当须索之。如信命不求,谓当自至,可不假而自得,不作而自成,不行而自至?夫命富之人,筋力自强;命贵之人,才智自高,若千里之马,头目蹄足自相副也。有求而不得者矣,未必不求而得之者也。精学不求贵,贵自至矣:力作不求富,富自到矣。富贵之福,不可求致;贫贱之祸,不可苟除也。由此言之,有富贵之命,不求自得。

信命者曰:``自知吉,不待求也。天命吉厚,不求自得;天命凶厚,求之无益。''夫物不求而自生,则人亦有不求贵而贵者矣。人情有不教而自善者,有教而终不善者矣,天性,犹命也。越王翳逃山中,至诚不愿。自冀得代。越人熏其穴,遂不得免,强立为君。而天命当然,虽逃避之,终不得离。故夫不求自得之贵欤!

\hypertarget{header-n44}{%
\subsubsection{气寿篇}\label{header-n44}}

凡人禀命有二品,一曰所当触值之命,二曰强弱寿夭之命。所当触值,谓兵烧压溺也。强寿弱夭,谓禀气渥薄也。兵烧压溺,遭以所禀为命,未必有审期也。若夫强弱夭寿以百为数,不至百者,气自不足也。夫禀气渥则其体强,体强则其命长;气薄则其体弱,体弱则命短。命短则多病,寿短。始生而死,未产而伤,禀之薄弱也。渥强之人,不卒其寿,若夫无所遭遇,虚居困劣,短气而死,此禀之薄,用之竭也。此与始生而死,未产而伤,一命也,皆由禀气不足,不自致於百也。

人之禀气,或充实而坚强,或虚劣而软弱。充实坚强,其年寿;虚劣软弱,失弃其身。天地生物,物有不遂;父母生子,子有不就。物有为实,枯死而堕;人有为兒,夭命而伤。使实不枯,亦至满岁;使兒不伤,亦至百年。然为实、兒而死枯者,禀气薄,则虽形体完,其虚劣气少,不能充也。兒生,号啼之声鸿朗高暢者寿,嘶喝湿下者夭。何则?禀寿夭之命,以气多少为主性也。妇人疏字者子活,数乳者子死。何则?疏而气渥,子坚强;数而气薄,子软弱也。怀子而前已产子死,则谓所怀不活。名之曰怀,其意以为已产之子死,故感伤之子失其性矣。所产子死、所怀子凶者,字乳亟数,气薄不能成也;虽成人形体,则易感伤,独先疾病,病独不治。

百岁之命,是其正也。不能满百者,虽非正,犹为命也。譬犹人形一丈,正形也,名男子为丈夫,尊公妪为丈人。不满丈者,失其正也,虽失其正,犹乃为形也。夫形不可以不满丈之故谓之非形,犹命不可以不满百之故谓之非命也。非天有长短之命,而人各有禀受也。由此言之,人受气命於天,卒与不卒,同也。语曰:``图王不成,其弊可以霸。''霸者,王之弊也。霸本当至於王,犹寿当至於百也。不能成王,退而为霸;不能至百,消而为夭。王霸同一业,优劣异名;寿夭或一气,长短殊数。何以知不满百为夭者百岁之命也?以其形体小大长短同一等也。百岁之身,五十之体,无以异也;身体不异,血气不殊;鸟兽与人异形,故其年寿与人殊数。

何以明人年以百为寿也?世间有矣。儒者说曰:太平之时,人民侗长,百岁左右,气和之所生也。《尧典》曰:``朕在位七十载。''求禅得舜,舜征三十岁在位。尧退而老,八岁而终,至殂落,九十八岁。未在位之时,必已成人,今计数百有余矣。又曰:``舜生三十,征用三十,在位五十载,陟方乃死。''适百岁矣。文王谓武王曰:``我百,尔九十。吾与尔三焉。''文王九十七而薨,武王九十三而崩。周公,武王之弟也,兄弟相差,不过十年。武王崩,周公居摄七年,复政退老,出入百岁矣。邵公,周公之兄也,至康王之时,尚为太保,出入百有余岁矣。圣人禀和气,故年命得正数。气和为治平,故太平之世多长寿人。百岁之寿,盖人年之正数也,犹物至秋而死,物命之正期也。物先秋後秋,则亦如人死或增百岁,或减百也;先秋後秋为期,增百减百为数。物或出地而死,犹人始生而夭也;物或逾秋不死,亦如人年多度百至於三百也。传称:老子二百余岁,邵公百八十。高宗享国百年,周穆王享国百年,并未享国之时,皆出百三十四十岁矣。

\hypertarget{header-n52}{%
\subsection{卷二}\label{header-n52}}

\hypertarget{header-n53}{%
\subsubsection{幸偶篇}\label{header-n53}}

凡人操行,有贤有愚,及遭祸福,有幸有不幸;举事有是有非,及触赏罚,有偶有不偶。并时遭兵,隐者不中。同日被霜,蔽者不伤。中伤未必恶,隐蔽未必善。隐蔽幸,中伤不幸。俱欲纳忠,或赏或罚;并欲有益,或信或疑。赏而信者未必真,罚而疑者未必伪。赏信者偶,罚疑不偶也。

孔子门徒七十有余,颜回蚤夭。孔子曰:``不幸短命死矣!''短命称不幸,则知长命者幸也,短命者不幸也。服圣贤之道,讲仁义之业,宜蒙福佑。伯牛有疾,亦复颜回之类,俱不幸也。蝼蚁行於地,人举足而涉之。足所履,蝼蚁荏笮死;足所不蹈,全活不伤。火燔野草,车轹所致,火所不燔,俗或喜之,名曰幸草。夫足所不蹈,火所不及,未必善也,举火行有适然也。由是以论,痈疽之发,亦一实也。气结阏积,聚为痈;溃为疽创,流血出脓,岂痈疽所发,身之善穴哉?营卫之行,遇不通也。蜘蛛结网,蜚虫过之,或脱或获;猎者张罗,百兽群扰,或得或失。渔者罾江河之鱼,或存或亡。或奸盗大辟而不知,或罚赎小罪而发觉:灾气加人,亦此类也。不幸遭触而死,幸者免脱而生,不幸者,不侥幸也。孔子曰:``人之生也直,罔之生也幸。''则夫顺道而触者,为不幸矣。立岩墙之下,为坏所压;蹈圻岸之上,为崩所坠,轻遇无端,故为不幸。鲁城门久朽欲顿,孔子过之,趋而疾行。左右曰:``久矣。''孔子曰:``恶其久也。''孔子戒慎已甚,如过遭坏,可谓不幸也。故孔子曰:``君子有不幸而无有幸,小人有幸而无不幸。''又曰:``君子处易以俟命,小人行险以徼幸。''

佞幸之徒,闳孺、籍孺之辈,无德薄才,以色称媚,不宜爱而受宠,不当亲而得附,非道理之宜。故太史公为之作传,邪人反道而受恩宠,与此同科,故合其名谓之《佞幸》。无德受恩,无过遇祸,同一实也。俱禀元气,或独为人,或为禽兽。并为人,或贵或贱,或贫或富。富或累金,贫或乞食;贵至封侯,贱至奴仆。非天禀施有左右也,人物受性有厚薄也。俱行道德,祸福不钧;并为仁义,利害不同。晋文修文德,徐偃行仁义,文公以赏赐,偃王以破灭。鲁人为父报仇,安行不走,追者舍之;牛缺为盗所夺,和意不恐,盗还杀之。文德与仁义同,不走与不恐等,然文公、鲁人得福,偃王、牛缺得祸者,文公、鲁人幸,而偃王、牛缺不幸也。韩昭侯醉卧而寒,典冠加之以衣,觉而问之,知典冠爱己也,以越职之故,加之以罪。卫之骖乘者,见御者之过,从後呼车,有救危之义,不被其罪。夫骖乘之呼车,典冠之加衣,同一意也。加衣恐主之寒,呼车恐君之危,仁惠之情,俱发於心。然而於韩有罪,於卫为忠,骖乘偶,典冠不偶也。

非唯人行,物亦有之。长数仞之竹,大连抱之木,工技之人,裁而用之,或成器而见举持,或遗材而遭废弃。非工技之人有爱憎也,刀斧如有偶然也。蒸谷为饭,酿饭为酒。酒之成也,甘苦异味;饭之熟也,刚柔殊和。非庖厨酒人有意异也,手指之调有偶适也。调饭也殊筐而居,甘酒也异器而处,虫堕一器,酒弃不饮;鼠涉一筐,饭捐不食。夫百草之类,皆有补益,遭医人采掇,成为良药;或遗枯泽,为火所烁。等之金也,或为剑戟,或为锋钴。同之木也,或梁於宫,或柱於桥。俱之火也,或烁脂烛,或燔枯草。均之土也,或基殿堂,或涂轩户。皆之水也,或溉鼎釜,或澡腐臭。物善恶同,遭为人用,其不幸偶,犹可伤痛,况含精气之徒乎!

虞舜圣人也,在世宜蒙全安之福。父顽母,弟象敖狂,无过见憎,不恶而嚚得罪,不幸甚矣!孔子,舜之次也。生无尺土,周流应聘,削迹绝粮。俱以圣才,并不幸偶。舜尚遭尧受禅,孔子已死於阙里。以圣人之才,犹不幸偶,庸人之中,被不幸偶,祸必众多矣!

\hypertarget{header-n61}{%
\subsubsection{命义篇}\label{header-n61}}

墨家之论,以为人死无命;儒家之议,以为人死有命。言有命者,见子夏言``死生有命,富贵在天。''言无命者,闻历阳之都,一宿沉而为湖;秦将白起坑赵降卒於长平之下,四十万众,同时皆死;春秋之时,败绩之军,死者蔽草,尸且万数;饥馑之岁,饿者满道;温气疫疬,千户灭门,如必有命,何其秦、齐同也?言有命者曰:夫天下之大,人民之众,一历阳之都,一长平之坑,同命俱死,未可怪也。命当溺死,故相聚於历阳;命当压死,故相积於长平。犹高祖初起,相工入丰、沛之邦,多封侯之人矣,未必老少男女俱贵而有相也,卓砾时见,往往皆然。而历阳之都,男女俱没,长平之坑,老少并陷,万数之中,必有长命未当死之人。遭时衰微,兵革并起,不得终其寿。人命有长短,时有盛衰,衰则疾病,被灾蒙祸之验也。''

宋、卫、陈、郑同日并灾,四国之民,必有禄盛未当衰之人,然而俱灭,国祸陵之也。故国命胜人命,寿命胜禄命。人有寿夭之相,亦有贫富贵贱之法,俱见於体。故寿命修短,皆禀於天;骨法善恶,皆见於体。命当夭折,虽禀异行,终不得长;禄当贫贱,虽有善性,终不得遂。项羽且死,顾谓其徒曰:``吾败乃命,非用兵之过。''此言实也。实者项羽用兵过於高祖,高祖之起,有天命焉。国命系於众星,列宿吉凶,国有祸福;众星推移,人有盛衰。人之有吉凶,犹岁之有丰耗,命有衰盛,物有贵贱。一岁之中,一贵一贱;一寿之间,一衰一盛。物之贵贱,不在丰耗;人之衰盛,不在贤愚。子夏曰``死生有命,富贵在天'',而不曰``死生在天,富贵有命''者,何则?死生者,无象在天,以性为主。禀得坚强之性,则气渥厚而体坚强,坚强则寿命长,寿命长则不夭死。禀性软弱者,气少泊而性羸窳,羸窳则寿命短,短则蚤死。故言``有命'',命则性也。至於富贵所禀,犹性所禀之气,得众星之精。众星在天,天有其象。得富贵象则富贵,得贫贱象则贫贱,故曰``在天''。在天如何?天有百官,有众星。天施气而众星布精,天所施气,众星之气在其中矣。人禀气而生,含气而长,得贵则贵,得贱则贱;贵或秩有高下,富或资有多少,皆星位尊卑小大之所授也。故天有百官,天有众星,地有万民,五帝、三王之精。天有王梁、造父,人亦有之,禀受其气,故巧於御。

传曰:``说命有三,一曰正命,二曰随命,三曰遭命。''正命,谓本禀之自得吉也。性然骨善,故不假操行以求福而吉自至,故曰正命。随命者,戳力操行而吉福至,纵情施欲而凶祸到,故曰随命。遭命者,行善得恶,非所冀望,逢遭於外而得凶祸,故曰遭命。凡人受命,在父母施气之时,已得吉凶矣。夫性与命异,或性善而命凶,或性恶而命吉。操行善恶者,性也;祸福吉凶者,命也。或行善而得祸,是性善而命凶;或行恶而得福,是性恶而命吉也。性自有善恶,命自有吉凶。使命吉之人,虽不行善,未必无福;凶命之人,虽勉操行,未必无祸。孟子曰:``求之有道,得之有命。''性善乃能求之,命善乃能得之。性善命凶,求之不能得也。行恶者祸随而至。而盗跖、庄蹻横行天下,聚党数千,攻夺人物,断斩人身,无道甚矣,宜遇其祸,乃以寿终。夫如是,随命之说,安所验乎?遭命者,行善於内,遭凶於外也。若颜渊、伯牛之徒,如何遭凶?颜渊、伯牛,行善者也,当得随命,福佑随至,何故遭凶?颜渊困於学,以才自杀;伯牛空居而遭恶疾。及屈平、伍员之徒,尽忠辅上,竭王臣之节,而楚放其身,吴烹其尸。行善当得随命之福,乃触遭命之祸,何哉?言随命则无遭命,言遭命则无随命,儒者三命之说,竟何所定?且命在初生,骨表著见。今言随操行而至,此命在末,不在本也。则富贵贫贱皆在初禀之时,不在长大之後,随操行而至也。正命者,至百而死;随命者,五十而死。遭命者,初禀气时遭凶恶也,谓妊娠之时遭得恶也,或遭雷雨之变,长大夭死。此谓三命。

亦有三性:有正,有随,有遭。正者,禀五常之性也;随者,随父母之性;遭者,遭得恶物象之故也。故妊妇食兔,子生缺脣。《月令》曰:``是月也,雷将发声。''有不戒其容者,生子不备,必有大凶,喑聋跛盲。气遭胎伤,故受性狂悖。羊舌似我初生之时,声似豺狼,长大性恶,被祸而死。在母身时,遭受此性,丹硃、商均之类是也。性命在本,故《礼》有胎教之法:子在身时,席不正不坐,割不正不食,非正色目不视,非正声耳不听。及长,置以贤师良傅,教君臣父子之道,贤不肖在此时矣。受气时,母不谨慎,心妄虑邪,则子长大,狂悖不善,形体丑恶。素女对黄帝陈五女之法,非徒伤父母之身,乃又贼男女之性。

人有命,有禄,有遭遇,有幸偶。命者,贫富贵贱也;禄者,盛衰兴废也。以命当富贵,遭当盛之禄,常安不危;以命当贫贱,遇当衰之禄,则祸殃乃至,常苦不乐。遭者,遭逢非常之变,若成汤囚夏台,文王厄牖里矣。以圣明之德,而有囚厄之变,可谓遭矣。变虽甚大,命善禄盛,变不为害,故称遭逢之祸。晏子所遭,可谓大矣。直兵指胸,白刃如颈,蹈死亡之地,当剑戟之锋,执死得生还。命善禄盛,遭逢之祸,不能害也。历阳之都,长平之坑,其中必有命善禄盛之人,一宿同填而死。遭逢之祸大,命善禄盛不能却也。譬犹水火相更也,水盛胜火,火盛胜水。遇者,遇其主而用也。虽有善命盛禄,不遇知己之主,不得效验。幸者,谓所遭触得善恶也。获罪得脱,幸也。无罪见拘,不幸也。执拘未久,蒙令得出,命善禄盛,夭灾之祸不能伤也。偶者,谓事君也。以道事君,君善其言,遂用其身,偶也。行与主乖,退而远,不偶也。退远未久,上官录召,命善禄盛,不偶之害不能留也。

故夫遭遇幸偶,或与命禄并,或与命离。遭遇幸偶,遂以成完;遭遇不幸偶,遂以败伤,是与命并者也。中不遂成,善转为恶,是与命禄离者也。故人之在世,有吉凶之命,有盛衰之,重以遭遇幸偶之逢,获从生死而卒其善恶之行,得其胸中之志,希矣。

\hypertarget{header-n70}{%
\subsubsection{无形篇}\label{header-n70}}

人禀元气於天,各受寿夭之命,以立长短之形,犹陶者用土为簋廉,冶者用铜为柈杅矣。器形已成,不可小大;人体已定,不可减增。用气为性,性成命定。体气与形骸相抱,生死与期节相须。形不可变化,命不可减加。以陶冶言之,人命短长,可得论也。

或难曰:``陶者用埴为簋廉,簋廉壹成,遂至毁败,不可复变。若夫冶者用铜为柈,杅虽已成器,犹可复烁。柈可得为尊,尊不可为簋。人禀气於天,虽各受寿夭之命,立以形体,如得善道神药,形可变化,命可加增。

曰:冶者变更成器,须先以火燔烁,乃可大小短长。人冀延年,欲比於铜器,宜有若炉炭之化乃易形形易寿亦可增。人何由变易其形,便如火烁铜器乎?《礼》曰:``水潦降,不献鱼鳖。''何则?雨水暴下,虫蛇变化,化为鱼鳖。离本真暂变之虫,臣子谨慎,故不敢献。人愿身之变,冀若虫蛇之化乎?夫虫蛇未化者,不若不化者。虫蛇未化,人不食也;化为鱼鳖,人则食之。食则寿命乃短,非所冀也。岁月推移,气变物类,虾蟆为鹑,雀为蜃蛤。人愿身之变,冀若鹑与蜃蛤鱼鳖之类也?人设捕蜃蛤,得者食之。虽身之不化,寿命不得长,非所冀也。鲁公牛哀寝疾,七日变而成虎。鲧殛羽山,化为黄能。愿身变者,冀牛哀之为虎,鲧之为能乎?则夫虎、能之寿,不能过人。天地之性,人最为贵。变人之形,更为禽兽,非所冀也。凡可冀者,以老翁变为婴兒,其次白发复黑,齿落复生,身气丁强,超乘不衰,乃可贵也。徒变其形,寿命不延,其何益哉?

且物之变,随气,若应政治,有所象为,非天所欲寿长之故,变易其形也,又非得神草珍药食之而变化也。人恆服药固寿,能增加本性,益其身年也。遭时变化,非天之正气、人所受之真性也。天地不变,日月不易,星辰不没,正也。人受正气,故体不变。时或男化为女,女化为男,由高岸为谷,深谷为陵也。应政为变,为政变,非常性也。汉兴,老父授张良书,已化为石。是以石之精,为汉兴之瑞也。犹河精为人持璧与秦使者,秦亡之征也。蚕食桑老,绩而为茧,茧又化而为蛾;蛾有两翼,变去蚕形。蛴螬化为复育,复育转而为蝉;蝉生两翼,不类蛴螬。凡诸命蠕蜚之类,多变其形,易其体。至人独不变者,禀得正也。生为婴兒,长为丈夫,老为父翁。从生至死,未尝变更者,天性然也。天性不变者,不可令复变;变者,不可不变。若夫变者之寿,不若不变者。人欲变其形,辄增益其年,可也;如徒变其形而年不增,则蝉之类也,何谓人愿之?

龙之为虫,一存一亡,一短一长。龙之为性也,变化斯须,辄复非常。由此言之,人,物也,受不变之形,形不可变更,年不可增减。传称高宗有桑谷之异。悔过反政,享福百年,是虚也。传言宋景公出三善言,荧惑却三舍,延年二十一载,是又虚也。又言秦缪公有明德,上帝赐之十九年,是又虚也。称赤松、王乔好道为仙,度世不死,是又虚也。假令人生立形谓之甲,终老至死,常守甲形。如好道为仙,未有使甲变为乙者也。夫形不可变更,年不可减增。何则?形、气、性,天也。形为春,气为夏。人以气为寿,形随气而动。气性不均,则於体不同。牛寿半马,马寿半人,然则牛马之形与人异矣。禀牛马之形,当自得牛马之寿;牛马之不变为人,则年寿亦短於人。世称高宗之徒,不言其身形变异。而徒言其增延年寿,故有信矣。

形之血气也,犹囊之贮粟米也。一石囊之高大,亦适一石。如损益粟米,囊亦增减。人以气为寿,气犹粟米,形犹囊也。增减其寿,亦当增减其身,形安得如故?如以人形与囊异,气与粟米殊,更以苞瓜喻之。苞瓜之汁,犹人之血也;其肌,犹肉也。试令人损益苞瓜之汁,令其形如故,耐为之乎?人不耐损益苞瓜之汁,天安耐增减人之年?人年不可增减,高宗之徒,谁益之者?而云增加。如言高宗之徒,形体变易,其年亦增,乃可信也。今言年增,不言其体变,未可信也。何则?人禀气於天,气成而形立,则命相须以至终死。形不可变化,年亦不可增加。以何验之?人生能行,死则僵仆,死则气减形消而坏。禀生人形,不可得变,其年安可增?人生至老,身变者,发与肤也。人少则发黑,老则发白,白久则黄。发之变,形非变也。人少则肤白,老则肤黑,黑久则黯,若有垢矣。发黄而肤为垢,故《礼》曰:``黄耇无疆。''发肤变异,故人老寿迟死,骨肉不可变更,寿极则死矣。五行之物,可变改者,唯土也。埏以为马,变以为人,是谓未入陶灶更火者也。如使成器,入灶更火,牢坚不可复变。今人以为天地所陶冶矣,形已成定,何可复更也?

图仙人之形,体生毛,臂变为翼,行於云则年增矣,千岁不死。此虚图也。世有虚语,亦有虚图。假使之然,蝉蛾之类,非真正人也。海外三十五国,有毛民羽民,羽则翼矣。毛羽之民土形所出,非言为道身生毛羽也。禹、益见西王母,不言有毛羽。不死之民,亦在外国,不言有毛羽。毛羽之民,不言之死;不死之民,不言毛羽。毛羽未可以效不死,仙人之有翼,安足以验长寿乎?

\hypertarget{header-n80}{%
\subsubsection{率性篇}\label{header-n80}}

论人之性,定有善有恶。其善者,固自善矣;其恶者,故可教告率勉,使之为善。凡人君父审观臣子之性,善则养育劝率,无令近恶;近恶则辅保禁防,令渐於善,善渐於恶,恶化於善,成为性行。召公戒成曰:``今王初服厥命,於戏!若生子罔不在厥初生。''生子谓十五子,初生意於善,终以善;初生意於恶,终以恶。《诗》曰:``彼姝者子,何以与之?''传言:譬犹练丝,染之蓝则青,染之丹则赤。十五之子其犹丝也,其有所渐化为善恶,犹蓝丹之染练丝,使之为青赤也。青赤一成,真色无异。是故扬子哭岐道,墨子哭练丝也。盖伤离本,不可复变也。人之性,善可变为恶,恶可变为善,犹此类也。逢生麻间,不扶自直;白纱入缁,不练自黑。彼蓬之性不直,纱之质不黑,麻扶缁染,使之直黑。夫人之性犹蓬纱也,在所渐染而善恶变矣。

王良、造父称为善御,能使不良为良也。如徒能御良,其不良者不能驯服,此则驵工庸师服驯技能,何奇而世称之?故曰:王良登车,马不罢驽;尧、舜为政,民无狂愚。传曰:``尧、舜之民可比屋而封,桀、纣之民可比屋而诛。''斯民也,三代所以直道而行也。圣主之民如彼,恶主之民如此,竟在化不在性也。闻伯夷之风者,贪夫廉而懦夫有立志;闻柳下惠之风者,薄夫敦而鄙夫宽。徒闻风名,犹或变节,况亲接形面相敦告乎?孔门弟子七十之徒,皆任卿相之用,被服圣教,文才雕琢,知能十倍,教训之功而渐渍之力也。未入孔子之门时,闾巷常庸无奇,其尤甚不率者,唯子路也。世称子路无恆之庸人,未入孔门时,戴鸡佩豚,勇猛无礼,闻诵读之声,摇鸡奋豚,扬脣吻之音,聒贤圣之耳,恶至甚矣。孔子引而教之,渐渍磨历,阖导牖进,猛气消损,骄节屈折,卒能政事,序在四科。斯盖变性使恶为善之明效也。

夫肥沃墝埆,土地之本性也。肥而沃者性美,树稼丰茂。墝而埆者性恶,深耕细锄,厚加粪壤,勉致人功,以助地力,其树稼与彼肥沃者相似类也。地之高下,亦如此焉。以锸凿地,以埤增下,则其下与高者齐;如复增锸,则夫下者不徒齐者也,反更为高,而其高者反为下。使人之性有善有恶,彼地有高有下,勉致其教令之善,则将善者同之矣。善以化渥,酿其教令,变更为善。善则且更宜反过於往善,犹下地增加锸更崇於高地也。赐不受命而货殖焉,赐本不受天之富命,所加货财积聚,为世富人者,得货殖之术也。夫得其术,虽不受命,犹自益饶富。性恶之人,益不禀天善性,得圣人之教,志行变化。世称利剑有千金之价。棠溪、鱼肠之属,龙泉、太阿之辈,其本铤,山中之恆铁也。冶工锻炼,成为銛利,岂利剑之锻与炼,乃异质哉?工良师巧,炼一数至也。试取东下直一金之剑,更熟锻炼,足其火,齐其銛,犹千金之剑也。夫铁石天然,尚为锻炼者变易故质,况人含五常之性,贤圣未之熟锻炼耳,奚患性之不善哉?古贵良医者,能知笃剧之病所从生起,而以针药治而已之。如徒知病之名而坐观之,何以为奇?夫人有不善,则乃性命之疾也,无其教治,而欲令变更,岂不难哉!

天道有真伪。真者固自与天相应,伪者人加知巧,亦与真者无以异也。何以验之?《禹贡》曰``璆琳琅玕'',此则土地所生真玉珠也。然而道人消烁五石,作五色之玉,比之真玉,光不殊别,兼鱼蚌之珠,与《禹贡》璆琳皆真玉珠也。然而随侯以药作珠,精耀如真,道士之教至,知巧之意加也。阳遂取火於天,五月丙午日中之时,消炼五石,铸以为器,磨历生光,仰以向日,则火来至。此真取火之道也。今妄取刀剑月,摩拭朗白,仰以向日,亦得火焉。夫月非阳遂也,所以耐取火者,摩拭之所致也。今夫性恶之人,使与性善者同类乎?可率勉之令其为善;使之异类乎,亦可令与道人之所铸玉、随侯之所作珠、人之所摩刀剑月焉,教导以学,渐渍以德,亦将日有仁义之操。黄帝与炎帝争为天子,教熊罴貔虎以战於阪泉之野,三战得志,炎帝败绩。尧以天下让舜,鲧为诸侯,欲得三公,而尧不听,怒其猛兽,欲以为乱,比兽之角可以为城,举尾以为旌,奋心盛气,阻战为强。夫禽兽与人殊形,犹可命战,况人同类乎?推此以论,``百兽率舞'',``潭鱼出听'',``六马仰秣'',不复疑矣。异类以殊为同,同类以钧为异,所由不在於物,在於人也。凡含血气者,教之所以异化也。三苗之民,或贤或不肖,尧、舜齐之,恩教加也。楚、越之人,处庄、岳之间,经历岁月,变为舒缓,风俗移也。故曰:``齐舒缓,秦慢易,楚促急,燕戆投''。以庄、岳言之,四国之民,更相出入,久居单处,性必变易。

夫性恶者,心比木石。木石犹为人用,况非木石!在君子之迹,庶几可见。有痴狂之疾,歌啼於路,不晓东西,不睹燥湿,不觉疾病,不知饥饱,性已毁伤,不可如何。前无所观,却无所畏也。是故王法不废学校之官,不除狱理之吏,欲令凡众见礼仪之教。学校勉其前,法禁防其後,使丹硃之志亦将可勉。何以验之?三军之士,非能制也,勇将率勉,视死如归。且阖庐尝试其士於五湖之侧,皆加刃於肩,血流至地。句践亦试其士於寝宫之庭,赴火死者,不可胜数。夫刃火,非人性之所贪也,二主激率,念不顾生。是故军之法轻刺血。孟贲勇也,闻军令惧。是故叔孙通制定礼仪,拔剑争功之臣,奉礼拜伏,初骄倨而後逊顺,教威德,变易性也。不患性恶,患其不服圣教,自遇而以生祸也。

豆麦之种,与稻梁殊,然食能去饥。小人君子,禀性异类乎?譬诸五谷皆为用,实不异而效殊者,禀气有厚泊,故性有善恶也。残则受仁之气泊,而怒则禀勇渥也。仁泊则戾而少愈,勇渥则猛而无义,而又和气不足,喜怒失时,计虑轻愚。妄行之人,罪故为恶。人受五常,含五脏,皆具於身。禀之泊少,故其操行不及善人,犹或厚或泊也。非厚与泊殊其酿也,曲孽多少使之然也。是故酒之泊厚,同一曲孽;人之善恶,共一元气,气有少多,鼓性有贤愚。

西门豹急,佩韦以自缓;董安於缓,带弦以自促。急之与缓,俱失中和,然而韦弦附身,成为完具之人。能纳韦弦之教,补接不足,则豹、安於之名可得参也。贫劣宅屋不具墙壁宇达,人指訾之。如财货富愈,起屋筑墙,以自蔽鄣,为之具宅,人弗复非。魏之行田百亩,鄴独二百,西门豹灌以漳水,成为膏腴,则亩收一锺。夫人之质犹鄴田,道教犹漳水也。患不能化,不患人性之难率也。雒阳城中之道无水,水工激上洛中之水,日夜驰流,水工之功也。由此言之,迫近君子,而仁义之道数加於身,孟母之徙宅,盖得其验。人间之水污浊,在野外者清洁,俱为一水,源从天涯,或浊或清,所在之势使之然也。南越王赵他,本汉贤人也,化南夷之俗,背畔王制,椎髻箕坐,好之若性。陆贾说以汉德,惧以圣威,蹶然起坐,心觉改悔,奉制称蕃,其於椎髻箕坐也,恶之若性。前则若彼,後则若此。由此言之,亦在於教,不独在性也。

\hypertarget{header-n90}{%
\subsubsection{吉验篇}\label{header-n90}}

凡人禀贵命於天,必有吉验见於地。见於地,故有天命也。验见非一,或以人物,或以祯祥,或以光气。

传言黄帝妊二十月而生,生而神灵,弱而能言。长大率诸侯,诸侯归之;教熊罴战,以伐炎帝,炎帝败绩。性与人异,故在母之身留多十月;命当为帝,故能教物,物为之使。尧体就之如日,望之若云。洪水滔天,蛇龙为害,尧使禹治水,驱蛇龙,水治东流,蛇龙潜处。有殊奇之骨,故有诡异之验;有神灵之命,故有验物之效。天命当贵,故从唐侯入嗣帝后之位。舜未逢尧,鳏在侧陋。瞽瞍与象谋欲杀之。使之完廪,火燔其下;令之浚井,土掩其上。舜得下廪,不被火灾;穿井旁出,不触土害。尧闻征用,试之於职。官治职修,事无废乱。使入大麓之野,虎狼不搏,蝮蛇不噬;逢烈风疾雨,行不迷惑。夫人欲杀之,不能害,之毒螫之野,禽虫不能伤,卒受帝命,践天子祚。

后稷之母,履大人迹,或言衣帝喾之服,坐息帝喾之处,妊身。怪而弃之隘巷,牛马不敢践之;寘之冰上,鸟以翼覆之,庆集其身。母知其神怪,乃收养之。长大佐尧,位至司马。乌孙王号昆莫,匈奴攻杀其父,而昆莫生,弃於野,乌衔肉往食之。单于怪之,以为神,而收长。及壮,使兵,数有功。单于乃复以其父之民予昆莫,令长守於西城。夫后稷不当弃,故牛马不践,鸟以羽翼覆爱其身;昆莫不当死,故乌衔肉就而食之。北夷橐离国王侍婢有娠,王欲杀之。婢对曰:``有气大如鸡子,从天而下,我故有娠''。後产子,捐於猪溷中,猪以口气嘘之,不死;复徙置马栏中,欲使马借杀之,马复以口气嘘之,不死。王疑以为天子,令其母收取,奴畜之,名东明,令牧牛马。东明善射,王恐夺其国也,欲杀之。东明走,南至掩淲水,以弓击水,鱼鳖浮为桥,东明得渡,鱼鳖解散,追兵不得渡,因都王夫馀。故北夷有夫馀国焉。东明之母初妊时,见气从天下,及生,弃之,猪马以气吁之而生之。长大,王欲杀之,以弓击水,鱼鳖为桥。天命不当死,故有猪马之救;命当都王夫馀,故有鱼鳖为桥之助也。伊尹且生之时,其母梦人谓已曰:``臼出水,疾东走。''母顾!明旦视臼出水,即东走十里,顾其乡,皆为水矣。伊尹命不当没,故其母感梦而走。推此以论,历阳之都,其策命若伊尹之类,必有先时感动在他地之效。

齐襄公之难,桓公为公子,与子纠争立。管仲辅子纠,鲍叔佐桓公。管仲与桓公争,引弓射之,中其带钩。夫人身长七尺,带约其要,钩挂於带,在身所掩,不过一寸之内,既微小难中,又滑泽钴靡,锋刃中钩者,莫不蹉跌。管仲射之,正中其钩中,矢触因落,不跌中旁肉。命当富贵,有神灵之助,故有射钩不中之验。楚共王有五子:子招、子圉、子干、子晰、弃疾。五人皆有宠,共王无适立,乃望祭山川,请神决之。乃与巴姬埋璧於太室之庭,令五子齐而入拜。康王跨之;子圉肘加焉;子干、子晰皆远之;弃疾弱,抱而入,再拜皆压纽。故共王死,招为康王,至子失之;圉为灵王,及身而弑;子干为王,十有余日;子晰不立,又惧诛死,皆绝无後。弃疾後立,竟续楚祀,如其神符。其王日之长短,与拜去璧远近相应也。夫璧在地中,五子不知,相随入拜,远近不同,压纽若神将教跽之矣。晋屠岸贾作难,诛赵盾之子。朔死,其妻有遗腹子。及岸贾闻之,索於宫,母置兒於裤中,祝曰:``赵氏宗灭乎?若当啼。即不灭,若无声。''及索之,而终不啼,遂脱得活。程婴齐负之,匿於山中。至景公时,韩厥言於景公,景公乃与韩厥共立赵孤,续赵氏祀,是为文子。当赵孤之无声,若有掩其口者矣。由此言之,赵文子立,命也。

高皇帝母曰刘媪,尝息大泽之陂,梦与神遇。是时雷电晦冥,蛟龙在上。及生而有美。性好用酒,尝从王媪、武负贳酒,饮醉止卧,媪、负见其身常有神怪。每留饮醉,酒售数倍。後行泽中,手崭大蛇,一妪当道而哭,云:``赤帝子杀吾子。''此验既著闻矣。秦始皇帝常曰:``东南有天子气''。於是东游以厌当之。高祖之气也,与吕后隐於芒、山泽间。吕后与人求之,见其上常有气直起,往求,辄得其处。後与项羽约,先入秦关,王之。高祖先至,项羽怨恨。范增曰:``吾令人望其气,气皆为龙,成五采,此皆天子之气也。急击之''。高祖往谢项羽。羽与亚父谋杀高祖,使项庄拔剑起舞。项伯知之,因与项庄俱起。每剑加高祖之上,项伯辄以身覆高祖之身,剑遂不得下,杀势不得成。会有张良、樊哙之救,卒得免脱,遂王天下。初妊身有蛟龙之神;既生,酒舍见云气之怪;夜行斩蛇,蛇妪悲哭;始皇、吕后,望见光气;项羽谋杀,项伯为蔽,谋遂不成,遭得良、哙,盖富贵之验,气见而物应、人助辅援也。窦太后弟名曰广国,年四五岁,家贫,为人所掠卖。其家不知其所在。传卖十余家。至宜阳,为其主人入山作炭。暮寒,卧炭下百余人,炭崩尽压死,广国独得脱。自卜数日当为侯,从其家之长安,闻窦皇后新立,家在清河观津,乃上书自陈。窦皇后言於景帝,召见问其故,果是,乃厚赐之。景帝立,拜广国为章武侯。夫积炭崩,百余人皆死,广国独脱,命当富贵,非徒得活,又封为侯。虞子大,陈留东莞人也。其生时以夜,适免母身,母见其上若一匹练状,经上天。明以问人,人皆曰:``吉,贵。''气与天通,长大仕宦,位至司徒公。广文伯河东蒲坂人也,其生亦以夜半时,适生,有人从门呼其父名。父出应之,不见人,有一木杖植其门侧,好善异於众,其父持杖入门以示人,人占曰:``吉''。文伯长大学宦,位至广汉太守。文伯当富贵,故父得赐杖,杖当子力矣。光武帝建平元年十二月甲子生於济阳宫後殿第二内中,皇考为济阳令,时夜无火,室内自明。皇考怪之,即召功曹吏充兰,使出问卜工。兰与马下卒苏永俱之卜王长孙所。长孙卜,谓永、兰曰:``此吉事也。毋多言。''是岁,有禾生景天中,三本一茎九穗,长於禾一二尺,盖嘉禾也。元帝之初,有凤凰下济阳宫,故今济阳宫有凤凰庐。始与李父等俱起,到柴界中,遇贼兵,惶惑走济阳旧庐。比到,见光若火,正赤,在旧庐道南,光耀憧憧上属天,有顷,不见。王莽时,谒者苏伯阿能望气,使过舂陵,城郭郁郁葱葱。及光武到河北,与伯阿见,问曰:``卿前过舂陵,何用知其气佳也?''伯阿对曰:``见其郁郁葱葱耳。''盖天命当兴,圣王当出,前後气验,照察明著。继体守文,因据前基,禀天光气,验不足言。创业龙兴,由微贱起於颠沛;若高祖、光武者,曷尝无天人神怪光显之验乎!

\hypertarget{header-n99}{%
\subsection{卷三}\label{header-n99}}

\hypertarget{header-n100}{%
\subsubsection{偶会篇}\label{header-n100}}

命,吉凶之主也。自然之道,适偶之数,非有他气旁物厌胜感动使之然也。世谓子胥伏剑,屈原自沉,子兰、宰嚭诬谗,吴、楚之君冤杀之也。偶二子命当绝,子兰、宰嚭适为谗,而怀王、夫差适信奸也。君适不明,臣适为谗,二子之命,偶自不长。二偶三合,似若有之,其实自然,非他为也。夏、殷之朝适穷,桀、纣之恶适稔,商、周之数适起,汤、武之德适丰。关龙逢杀,箕子、比干囚死,当桀、纣恶盛之时,亦二子命讫之期也。任伊尹之言,纳吕望之议,汤、武且兴之会,亦二臣当用之际也。人臣命有吉凶,贤不肖之主与之相逢。文王时当昌,吕望命当贵;高宗治当平,傅说德当遂。非文王、高宗为二臣生,吕望、傅说为两君出也。君明臣贤,光曜相察;上修下治,度数相得。颜渊死,子曰``天丧予''。子路死,子曰``天祝予。''孔子自伤之辞,非实然之道也。孔子命不王,二子寿不长也。不王不长,所禀不同,度数并放,适相应也。二龙之祆当效,周历适闿椟;褒姒当丧周国,幽王禀性偶恶。非二龙使历王发孽,褒姒令幽王愚惑也。遭逢会遇,自相得也。僮谣之语当验,斗鸡之变适生;瞿鹆之占当应,鲁昭之恶适成。非僮谣致斗竞,瞿鹆招君恶也。期数自至,人行偶合也。尧命当禅舜,丹硃为无道;虞统当传夏,商均行不轨。非舜、禹当得天下,能使二子恶也;美恶是非适相逢也。火星与昴星出入,昴星低时火星出,昴星见时火星伏,非火之性厌服昴也,时偶不并,度转乖也。正月建寅,斗魁破申,非寅建使申破也,转运之衡,偶自应也。父殁而子嗣,姑死而妇代,非子妇嗣代使父姑终殁也,老少年次自相承也。世谓秋气击杀谷草,谷草不任,雕伤而死。此言失实。夫物以春生夏长,秋而熟老,适自枯死,阴气适盛,与之会遇。何以验之?物有秋不死者,生性未极也。人生百岁而终,物生一岁而死,死谓阴气杀之,人终触何气而亡?论者犹或谓鬼丧之。夫人终鬼来,物死寒至,皆适遭也。人终见鬼,或见鬼而不死;物死触寒,或触寒而不枯。坏屋所压,崩崖所坠,非屋精崖气杀此人也。屋老崖沮,命凶之人,遭居适履。月毁於天,螺消於渊。风从虎,云从龙。同类通气,性相感动。若夫物事相遭,吉凶同时,偶适相遇,非气感也。杀人者罪至大辟。杀者罪当重,死者。命当尽也。故害气下降,囚命先中;圣王德施,厚禄先逢。是故德令降於殿堂,命长之囚,出於牢中。天非为囚未当死,使圣王出德令也,圣王适下赦,拘囚适当免死。犹人以夜卧昼起矣,夜月光尽,不可以作,人力亦倦,欲壹休息;昼日光明,人卧亦觉,力亦复足。非天以日作之,以液息之也,作与日相应,息与夜相得也。

雁鹄集於会稽,去避碣石之寒,来遭民田之毕,蹈履民田,啄食草粮。粮尽食索,春雨适作,避热北去,复之碣石。象耕灵陵,亦如此焉。传曰:``舜葬苍梧,象为之耕。禹葬会稽,鸟为之佃。''失事之实,虚妄之言也。丈夫有短寿之相,娶必得早寡之妻;早寡之妻,嫁亦遇夭折之夫也。世曰:``男女早死者,夫贼妻,妻害夫。''非相贼害,命有然也。使火燃,以水沃之,可谓水贼火。火适自灭,水适自覆,两各自败,不为相贼。今男女之早夭,非水沃火之比,适自灭覆之类也。贼父之子,妨兄之弟,与此同召。同宅而处,气相加凌,羸瘠消单,至於死亡,可谓相贼。或客死千里之外,兵烧厌溺,气不相犯,相贼如何?王莽姑正君,许嫁二夫,二夫死,当适赵而王薨。气未相加,遥贼三家,何其痛也!黄次公取邻巫之女,卜谓女相贵,故次公位至丞相。其实不然。次公当贵,行与女会;女亦自尊,故入次公门。偶适然自相遭遇,时也。

无禄之人,商而无盈,农而无播,非其性贼货而命妨谷也。命贫,居无利之货,禄恶,殖不滋之谷也。世谓宅有吉凶,徙有岁月。实事则不然。天道难知,假令有命凶之人,当衰之家,治宅遭得不吉之地,移徙适触岁月之忌。一家犯忌,口以十数,坐而死者,必禄衰命泊之人也。推此以论,仕宦进退迁徙,可复见也。时适当退,君用谗口;时适当起,贤人荐己。故仕且得官也,君子辅善;且失位也,小人毁奇。公伯寮诉子路於季孙,孔子称命。鲁人臧仓谗孟子於平公,孟子言天。道未当行,与谗相遇;天未与己,恶人用口。故孔子称命,不怨公伯寮;孟子言天,不尤臧仓,诚知时命当自然也。

推此以论,人君治道功化,可复言也。命当贵,时适平;期当乱,禄遭衰。治乱成败之时,与人兴衰吉凶适相遭遇。因此论圣贤迭起,犹此类也。圣主龙兴於仓卒,良辅超拔於际会。世谓韩信、张良辅助汉王,故秦灭汉兴,高祖得王。夫高祖命当自王,信、良之辈时当自兴,两相遭遇,若故相求。是故高祖起於丰、沛,丰、沛子弟相多富贵,非天以子弟助高祖也,命相小大,适相应也。赵简子废太子伯鲁,立庶子无恤,无恤遭贤,命亦当君赵也。世谓伯鲁不肖,不如无恤;伯鲁命当贱,知虑多泯乱也。韩生仕至太傅,世谓赖倪宽。实谓不然,太傅当贵,遭与倪宽遇也。赵武藏於裤中,终日不啼,非或掩其口,阏其声也;命时当生,睡卧遭出也。故军功之侯,必斩兵死之头;富家之商必夺贫室之财。削土免侯,罢退令相,罪法明白,禄秩适极。故历气所中,必加命短之人;凶岁所著,必饥虚耗之家矣。

\hypertarget{header-n107}{%
\subsubsection{骨相篇}\label{header-n107}}

人曰命难知。命甚易知。知之何用?用之骨体。人命禀於天,则有表候见于体。察表候以知命,犹察斗斛以知容矣。表候者,骨法之谓也。传言黄帝龙颜,颛顼戴午,帝喾骈齿,尧眉八采,舜目重瞳,禹耳三漏,汤臂再肘,文王四乳,武王望阳,周公背偻,皋陶马口,孔子反羽。斯十二圣者,皆在帝王之位,或辅主忧世,世所共闻,儒所共说,在经传者较著可信。若夫短书俗记、竹帛胤文,非儒者所见,众多非一。苍颉四目,为黄帝史。晋公子重耳仳胁,为诸侯霸。苏秦骨鼻,为六国相。张仪仳胁,亦相秦、魏。项羽重瞳,云虞舜之後,与高祖分王天下。陈平贫而饮食之足,貌体佼好,而众人怪之,曰:``平何食而肥?''及韩信为滕公所鉴,免於鈇质,亦以面状有异。面状肥佼,亦一相也。高祖隆准、龙颜、美须,左股有七十二黑子。单父吕公善相,见高祖状貌,奇之,因以其女妻高祖,吕后是也,卒生孝惠帝、鲁元公主。高祖为泗上亭长,当去归之田,与吕后及两子居田。有一老公过,请饮,因相吕后曰:``夫人,天下贵人也。''令相两子,见孝惠曰:``夫人所以贵者,乃此男也。''相鲁元,曰:``皆贵。''老公去,高祖从外来,吕后言於高祖。高祖追及老公,止使自相。老公曰:``乡者夫人婴兒相皆似君,君相贵不可言也。''後高祖得天下,如老公言。推此以况一室之人,皆有富贵之相矣。类同气钧,性体法相固自相似。异气殊类,亦两相遇。富贵之男娶得富贵之妻,女亦得富贵之男。夫二相不钧而相遇,则有立死;若未相适,有豫亡之祸也。王莽姑正君许嫁,至期当行时,夫辄死。如此者再,乃献之赵王,赵王未取又薨。清河南宫大有与正君父稚君善者,遇相君曰:``贵为天下母。''是时,宣帝世,元帝为太子,稚君乃因魏郡都尉纳之太子,太子幸之,生子君上。宣帝崩,太子立,正君为皇后,君上为太子。元帝崩,太子立,是为成帝,正君为皇太后,竟为天下母,夫正君之相当为天下母,而前所许二家及赵王,为无天下父之相,故未行而二夫死,赵王薨。是则二夫、赵王无帝王大命,而正君不当与三家相遇之验也。丞相黄次公,故为阳夏游徼,与善相者同车俱行,见一妇人年十七八,相者指之曰:``此妇人当大富贵,为封侯者夫人。''次公止车,审视之,相者曰:``今此妇人不富贵,卜书不用也。''次公问之,乃其旁里人巫家子也,即娶以为妻。其後次公果大富贵,位至丞相,封为列侯。夫次公富贵,妇人当配之,故果相遇,遂俱富贵。使次公命贱,不得妇人为偶,不宜为夫妇之时,则有二夫、赵王之祸。夫举家皆富贵之命,然後乃任富贵之事。骨法形体,有不应者,择必别离死亡,不得久享介福。故富贵之家,役使奴僮,育养牛马,必有与众不同者矣。僮奴则有不死亡之相,牛马则有数字乳之性,田则有种孳速熟之谷,商则有居善疾售之货。是故知命之人,见富贵於贫贱,睹贫贱於富贵。案骨节之法,察皮肤之理,以审人之性命,无不应者。

赵简子使姑布子卿相诸子,莫吉,至翟婢之子无恤而以为贵。无恤最贤,又有贵相,简子後废太子,而立无恤,卒为诸侯,襄子是矣。相工相黥布,当先刑而乃王,後竟被刑乃封王。卫青父郑季与杨信公主家僮卫媪通,生青。在建章宫时,钳徒相之,曰:``贵至封侯。''青曰:``人奴之道,得不笞骂足矣,安敢望封侯?''其後青为军吏,战数有功,超封增官,遂为大将军,封为万户侯。周亚夫未封侯之时,许负相之,曰:``君後三岁而入将相,持国秉,贵重矣,於人臣无两。其後九岁而君饿死。''亚夫笑曰:``臣之兄已代侯矣,有如父卒,子当代,亚夫何说侯乎?然既巳贵,如负言,又何说饿死?指示我!''许负指其口,有纵理入口,曰:``此饿死法也。''居三岁,其兄绛侯胜有罪,文帝择绛侯子贤者,推亚夫,乃封条侯,续绛侯後。文帝之後六年,匈奴入边,乃以亚夫为将军。至景帝之时,亚夫为丞相,後以疾免。其子为亚夫买工官尚方甲盾五百被可以为葬者,取庸苦之,不与钱。庸知其盗买官器,怨而上告其子。景帝下吏责问,因不食五日,呕血而死。当邓通之幸文帝也,贵在公卿之上,赏赐亿万,与上齐体。相工相之曰:``当贫贱饿死。''文帝崩,景帝立,通有盗铸钱之罪,景帝考验,通亡,寄死人家,不名一钱。

韩太傅为诸生时,借相工五十钱,与之俱入璧雍之中,相璧雍弟子谁当贵者。相工指倪宽曰:``彼生当贵,秩至三公。''韩生谢遣相工,通刺倪宽,结胶漆之交,尽筋力之敬,徙舍从宽,深自附纳之。宽尝甚病,韩生养视如仆状,恩深逾於骨肉。後名闻於天下。倪宽位至御史大夫,州郡丞旨召请,擢用举在本朝,遂至太傅。夫钳徒、许负及相邓通、倪宽之工,可谓知命之工矣。故知命之工,察骨体之证,睹富贵贫贱,犹人见盘盂之器,知所设用也。善器必用贵人,恶器必施贱者,尊鼎不在陪厕之侧,匏瓜不在殿堂之上,明矣。富贵之骨,不遇贫贱之苦;贫贱之相,不遭富贵之乐,亦犹此也。器之盛物,有斗石之量,犹人爵有高下之差也。器过其量,物溢弃遗;爵过其差,死亡不存。论命者如比之於器,以察骨体之法,则命在於身形,定矣。非徒富贵贫贱有骨体也,而操行清浊亦有法理。贵贱贫富,命也;操行清浊,性也。非徒命有骨法,性亦有骨法。唯知命有明相,莫知性有骨法,此见命之表证,不见性之符验也。范蠡去越,自齐遗大夫种书曰:``飞鸟尽,良弓藏,狡兔死,走犬烹。越王为人长颈鸟喙,可与共患难,不可与共容乐。子何不去?''大夫种不能去,称疾不朝,赐剑而死。大梁人尉缭,说秦始皇以并天下之计,始皇从其册,与之亢礼,衣服饮食与之齐同。缭曰:``秦王为人,隆准长目,鸷膺豺声,少恩,虎视狼心,居约易以下人;得志亦轻视人。我布衣也,然见我,常身自下我。诚使秦王须得志,天下皆为虏矣。不可与交游。''乃亡去。故范蠡、尉缭见性行之证,而以定处来事之实,实有其效,如其法相。由此言之,性命系於形体,明矣。以尺书所载,世所共见,准况古今,不闻者必众多非一,皆有其实。禀气於天,立形於地,察在地之形,以知在天之命,莫不得其实也。有传孔子相澹台子羽、唐举占蔡泽不验之文,此失之不审,何隐匿微妙之表也。相或在内,或在外,或在形体,或在声气。察外者遗其内;在形体者,亡其声气。孔子适郑,与弟子相失,孔子独立郑东门。郑人或问子贡曰:``东门有人,其头似尧,其项若皋陶,肩类子产。然自腰以下,不及禹三寸,傫傫若丧家之狗。''子贡以告孔子,孔子欣然笑曰:``形状未也。如丧家狗,然哉!然哉!''夫孔子之相,郑人失其实。郑人不明,法术浅也。孔子之失子羽,唐举惑於蔡泽,犹郑人相孔子,不能具见形状之实也。

\hypertarget{header-n113}{%
\subsubsection{初禀篇}\label{header-n113}}

人生性命当富贵者,初禀自然之气,养育长大,富贵之命效矣。文王得赤雀,武王得白鱼赤乌。儒者论之,以为雀则文王受命,鱼乌则武王受命;文、武受命於天,天用雀与鱼乌命授之也。天用赤雀命文王,文王不受,天复用鱼乌命武王也。若此者,谓本无命於天,修己行善,善行闻天,天乃授以帝王之命也,故雀与鱼乌,天使为王之命也。王所奉以行诛者也。如实论之,非命也。命,谓初所禀得而生也。人生受性,则受命矣。性命俱禀,同时并得,非先禀性,後乃受命也。何以明之?弃事尧为司马,居稷官,故为后稷。曾孙公刘居邰,後徙居邠。後孙古公亶甫三子:太伯、仲雍、季历,季历生文王昌。昌在襁褓之中,圣瑞见矣。故古公曰:``我世当有兴者,其在昌乎!''於是太伯知之,乃辞之吴,文身断发,以让王季。文王受命,谓此时也,天命在人本矣,太王古公见之早也。此犹为未,文王在母身之中已受命也。王者一受命,内以为性,外以为体。体者,面辅骨法,生而禀之。

吏秩百石以上,王侯以下,郎将大夫,以至元士,外及刺史太守,居禄秩之吏,禀富贵之命,生而有表见於面,故许负、姑布子卿辄见其验。仕者随秩迁转,迁转之人,或至公卿,命禄尊贵,位望高大。王者尊贵之率,高大之最也。生有高大之命,其时身有尊贵之奇,古公知之,见四乳之怪也。夫四乳,圣人证也,在母身中,禀天圣命,岂长大之後,修行道德,四乳乃生?以四乳论望羊,亦知为胎之时已受之矣。刘媪息於大泽,梦与神遇,遂生高祖,此时已受命也。光武生於济阳宫,夜半无火,内中光明。军下卒苏永谓公曹史充兰曰:``此吉事也,毋多言!''此时已受命。独谓文王、武王得赤雀、鱼乌乃受命,非也。上天壹命,王者乃兴,不复更命也。得富贵大命,自起王矣。何以验之?富家之翁,资累千金。生有富骨,治生积货,至於年老,成为富翁矣。夫王者,天下之翁也,禀命定於身中,犹鸟之别雄雌於卵壳之中也。卵壳孕而雌雄生,日月至而骨节强,强则雄,自率将雌。雄非生长之後,或教使为雄,然後乃敢将雌,此气性刚强自为之矣。夫王者,天下之雄也,其命当王。王命定於怀妊,犹富贵骨生,鸟雄卵成也。非唯人,鸟也,万物皆然。草木生於实核,出土为栽蘖,稍生茎叶,成为长短巨细,皆有实核。王者,长巨之最也。硃草之茎如针,紫芝之栽如豆,成为瑞矣。王者禀气而生,亦犹此也。

或曰:``王者生禀天命,及其将王,天复命之。犹公卿以下,诏书封拜,乃敢即位。赤雀鱼乌,上天封拜之命也。天道人事,有相命使之义。''自然无为,天之道也。命文以赤雀,武以白鱼,是有为也。管仲与鲍叔分财取多,鲍叔不与,管仲不求。内有以相知,视彼犹我,取之不疑。圣人起王,犹管之取财也。朋友彼我无有授与之义,上天自然,有命使之验,是则天道有为,朋友自然也。当汉祖斩大蛇之时,谁使斩者?岂有天道先至,而乃敢斩之哉?勇气奋发,性自然也。夫斩大蛇,诛秦杀项,同一实也。周之文、武命伐殷,亦一义也。高祖不受命使之将,独谓文、武受雀鱼之命,误矣。难曰:《康王之诰》曰:``冒闻於上帝,帝休,天乃大命文王。''如无命史,经何为言天乃大命文王?所谓大命者,非天乃命文王也,圣人动作,天命之意也,与天合同,若天使之矣。《书》方激劝康叔,勉使为善,故言文王行道,上闻於天,天乃大命之也。《诗》曰:``乃眷西顾,此惟予度。''与此同义。天无头面,眷顾如何?人有顾睨,以人效天,事易见,故曰眷顾。天乃大命文王,眷顾之义,实天之命也。何以验之?``夫大人与天地合其德,与日月合其明,与四时合其序,与鬼神合其吉凶,先天而天不违,後天而奉天时。''如必须天有命,乃以从事,安得先天而後天乎?以其不待天命,直以心发,故有先天後天之勤。言合天时,故有不违奉天之文。《论语》曰:``大哉!尧之为君!唯天为大,唯尧则之。''王者则天不违,奉天之义也。推自然之性,与天合同,是则所谓``大命文王''也,自文王意,文王自为,非天驱赤雀,使告文王,云当为王,乃敢起也。然则文王赤雀,及武王白鱼,非天之命,昌炽佑也。吉人举事,无不利者。人徒不召而至,瑞物不招而来,黯然谐合,若或使之。出门闻吉,顾睨见善,自然道也。文王当兴,赤雀适来;鱼跃乌飞,武王偶见:非天使雀至、白鱼来也,吉物动飞,而圣遇也。白鱼入於王舟,王阳曰:``偶适也。''光禄大夫刘琨,前为弘农太守,虎渡何。光武皇帝曰:``偶适自然,非或使之也。''故夫王阳之言``适'',光武之曰``偶'',可谓合於自然也。

\hypertarget{header-n119}{%
\subsubsection{本性篇}\label{header-n119}}

情性者,人治之本,礼乐所由生也。故原情性之极,礼为之防,乐为之节。性有卑谦辞让,故制礼以适其宜;情有好恶喜怒哀乐,故作乐以通其敬。礼所以制,乐所为作者,情与性也。昔儒旧生,著作篇章,莫不论说,莫能实定。

周人世硕,以为``人性有善恶,举人之善性,养而致之则善长;性恶,养而致之则恶长''。如此,则性各有阴阳,善恶在所养焉。故世子作《养书》一篇。密子贱、漆雕开、公孙尼子之徒,亦论情性,与世子相出入,皆言性有善有恶。

孟子作《性善》之篇,以为``人性皆善,及其不善,物乱之也''。谓人生於天地,皆禀善性,长大与物交接者,放纵悖乱,不善日以生矣。若孟子之言,人幼小之时,无有不善也。微子曰``我旧云孩子,王子不出。''纣为孩子时,微子睹其不善之性。性恶不出众庶,长大为乱不变,故云也。羊舌食我初生之时,叔姬视之,及堂,闻其啼声而还,曰:``其声,豺狼之声也。野心无亲,非是莫灭羊舌氏。隧不肯见。及长,祁胜为乱,食我与焉。国人杀食我。羊舌氏由是灭矣。纣之恶在孩子之时;食我之乱见始生之声。孩子始生,未与物接,谁令悖者?丹硃生於唐宫,商均生於虞室。唐、虞之时,可比屋而封,所与接者,必多善矣。二帝之旁,必多贤矣。然而丹硃傲,商均虐,并失帝统,历世为戒。且孟子相人以眸子焉,心清而眸子,心浊而眸子眊。人生目辄眊了,眊禀之於天,不同气也;非幼小之时,长大与人接乃更眊也。性本自然,善恶有质。孟子之言情性,未为实也。然而性善之论,亦有所缘。或仁或义,性术乖也。动作趋翔,性识诡也。面色或白或黑,身形或长或短,至老极死,不可变易,天性然也。皆知水土物器形性不同,而莫知善恶禀之异也。一岁婴兒无争夺之心,长大之後,或渐利色,狂心悖行,由此生也。

告子与孟生同时,其论性无善恶之分,譬之湍水,决之东则东,决之西则西,夫水无分於东西,犹人无分於善恶也。夫告子之言,谓人之性与水同也。使性若水,可以水喻性,犹金之为金,木之为木也。人善因善,恶亦因恶,初禀天然之姿,受纯壹之质,故生而兆见,善恶可察。无分於善恶,可推移者,谓中人也,不善不恶,须教成者也。故孔子曰:``中人以上可以语上也;中人以下,不可以语上也。''告子之以决水喻者,徒谓中人,不指极善极恶也。孔子曰:``性相近也,习相远也。''夫中人之性,在所习焉。习善而为善,习恶而为恶也。至於极善极恶,非复在习。故孔子曰:``惟上智与下愚不移。''性有善不善,圣化贤教,不能复移易也。孔子,道德之祖,诸子之中最卓者也,而曰``上智下愚不移'',故知告子之言,未得实也。夫告子之言,亦有缘也。《诗》曰:``彼姝之子,何以与之。''其传曰:``譬犹练丝,染之蓝则青,染之硃则赤。''夫决水使之东西,犹染丝令之青赤也。丹硃、商均已染於唐、虞之化矣,然而丹硃傲而商均虐者,至恶之质,不受蓝硃变也。

孙卿有反孟子,作《性恶》之篇,以为``人性恶,其善者伪也''。性恶者,以为人生皆得恶性也;伪者,长大之後,勉使为善也。若孙卿之言,人幼小无有善也。稷为兒,以种树为戏;孔子能行,以俎豆为弄。石生而坚,兰生而香。禀善气,长大就成,故种树之戏为唐司马;俎豆之弄,为周圣师。禀兰石之性,故有坚香之验。夫孙卿之言,未为得实。然而性恶之言,有缘也。一岁婴兒,无推让之心,见食,号欲食之;睹好,啼欲玩之。长大之後,禁情割欲,勉励为善矣。刘子政非之曰:``如此,则天无气也。阴阳善恶不相当,则人之为善安从生?''

陆贾曰:``天地生人也,以礼义之性。人能察己所以受命则顺,顺之谓道。''夫陆贾知人礼义为性,人亦能察己所以受命。性善者,不待察而自善;性恶者,虽能察之,犹背礼畔义,义挹於善不能为也。故贪者能言廉,乱者能言治。盗跖非人之窃也,庄蹻刺人之滥也,明能察己,口能论贤,性恶不为,何益於善?陆贾之言未能得实。

董仲舒览孙、孟之书,作《情性》之说曰:``天之大经,一阴一阳。人之大经,一情一性。性生於阳,情生於阴。阴气鄙,阳气仁。曰性善者,是见其阳也;谓恶者,是见其阴者也。''若仲舒之言,谓孟子见其阳,孙卿见其阴也。处二家各有见,可也。不处人情性,情性有善有恶,未也。夫人情性,同生於阴阳,其生於阴阳,有渥有泊。玉生於石,有纯有驳,性情生於阴阳,安能纯善?仲舒之言,未能得实。

刘子政曰:``性,生而然者也,在於身而不发;情,接於物而然者也,出形於外。形外则谓之阳;不发者则谓之阴。''夫子政之言,谓性在身而不发。情接於物,形出於外,故谓之阳;性不发,不与物接,故谓之阴。夫如子政之言,乃谓情为阳、性为阴也。不据本所生起,苟以形出与不发见定阴阳也。必以形出为阳,性亦与物接,造此必於是,颠沛必於是。恻隐不忍,仁之气也;卑歉辞让,性之发也,有与接会,故恻隐卑谦,形出於外。谓性在内,不与物接,恐非其实。不论性之善恶,徒议外内阴阳,理难以知。且从子政之言,以性为阴,情为阳,夫人禀情,竟有善恶不也?

自孟子以下至刘子政,鸿儒博生,闻见多矣。然而论情性竟无定是。唯世硕、公孙尼子之徒,颇得其正。由此言之,事易知,道难论也。酆文茂记,繁如荣华,恢谐剧谈,甘如饴蜜,未必得实。实者,人性有善有恶,犹人才有高有下也。高不可下,下不可高。谓性无善恶,是谓人才无高下也。禀性受命,同一实也。命有贵贱,性有善恶。谓性无善恶,是谓人命无贵贱也。

九州田土之性,善恶不均。故有黄赤黑之别,上中下之差;水潦不同,故有清浊之流,东西南北之趋。人禀天地之性,怀五常之气,或仁或义,性术乖也;动作趋翔,或重或轻,性识诡也。面色或白或黑,身形或长或短,至老极死不可变易,天性然也。余固以孟轲言人性善者,中人以上者也;孙卿言人性恶者,中人以下者也;扬雄言人性善恶混者,中人也。若反经合道,则可以为教;尽性之理,则未也。

\hypertarget{header-n132}{%
\subsubsection{物势篇}\label{header-n132}}

儒者论曰:``天地故生人。''此言妄也。夫天地合气,人偶自生也;犹夫妇合气,子则自生也。夫妇合气,非当时欲得生子;情欲动而合,合而生子矣。且夫妇不故生子,以知天地不故生人也。然则人生於天地也,犹鱼之於渊,饥虱之於人也。因气而生,种类相产,万物生天地之间,皆一实也。传曰:天地不故生人,人偶自生。

若此,论事者何故云``天地为炉,万物为铜,阴阳为火,造化为工''乎?案陶冶者之用烁铜燔器,故为之也。而云天地不故生人,人偶自生耳,可谓陶冶者不故为器而器偶自成乎?夫比不应事,未可谓喻;文不称实,未可谓是也。曰:``是喻人禀气不能纯一,若烁铜之下形,燔器之得火也,非谓天地生人与陶冶同也。''兴喻人皆引人事。人事有体,不可断绝。以目视头,头不得不动;以手相足,足不得不摇。目与头同形,手与足同体。今夫陶冶者,初埏埴作器,必模范为形,故作之也;燃炭生火,必调和炉灶,故为之也。及铜烁不能皆成,器燔不能尽善,不能故生也。夫天不能故生人,则其生万物,亦不能故也。天地合气,物偶自生矣。夫耕耘播种,故为之也;及其成与不熟,偶自然也。

何以验之?如天故生万物,当令其相亲爱,不当令之相贼害也。或曰:五行之气,天生万物。以万物含五行之气,五行之气更相贼害。曰:天自当以一行之气生万物,令之相亲爱,不当令五行之气反使相贼害也。或曰:欲为之用,故令相贼害;贼害相成也。故天用五行之气生万物,人用万物作万事。不能相制,不能相使,不相贼害,不成为用。金不贼木,木不成用。火不烁金,金不成器。故诸物相贼相利,含血之虫相胜服、相啮噬、相啖食者,皆五行气使之然也。''曰:``天生万物欲令相为用,不得不相贼害也。则生虎狼蝮蛇及蜂虿之虫,皆贼害人,天又欲使人为之用邪?且一人之身,含五行之气,故一人之行,有五常之操。五常,五行之道也。五藏在内,五行气俱。如论者之言,含血之虫,怀五行之气,辄相贼害。一人之身,胸怀五藏,自相贼也;一人之操,行义之心,自相害也。且五行之气相贼害,含血之虫相胜服,其验何在?曰:寅,木也,其禽虎也;戍,土也,其禽犬也。丑、未,亦土也,丑禽牛,未禽羊也。木胜土,故犬与牛羊为虎所服也。亥水也,其禽豕也;巳,火也,其禽蛇也;子亦水也,其禽鼠也。午亦火也,其禽马也。水胜火,故豕食蛇;火为水所害,故马食鼠屎而腹胀。曰:审如论者之言,含血之虫,亦有不相胜之效。午,马也,子,鼠也,酉,鸡也,卯兔也。水胜火,鼠何不逐马?金胜木,鸡何不啄兔?亥,豕也,(未,羊也。)丑,牛也。土胜水,牛羊何不杀豕?巳,蛇也。申,猴也。火胜金,蛇何不食獼猴?獼猴者,畏鼠也。啮獼猴者,犬也。鼠,水。獼猴,金也。水不胜金,獼猴何故畏鼠也?戍,土也,申,猴也。土不胜金,猴何故畏犬?东方,木也,其星仓龙也。西方,金也,其星白虎也;南方,火也,其星硃鸟也。北方,水也,其星玄武也。天有四星之精,降生四兽之体。含血之虫,以四兽为长,四兽含五行之气最较郑鼇案龙虎交不相贼,鸟龟会不相害。以四兽验之,以十二辰之禽效之,五行之虫以气性相刻,则尤不相应。

凡万物相刻贼,含血之虫则相服,至於相啖食者,自以齿牙顿利,筋力优劣,动作巧便,气势勇桀。若人之在世,势不与适,力不均等,自相胜服。以力相服,则以刃相贼矣。夫人以刃相贼,犹物以齿角爪牙相触刺也。力强角利,势烈牙长,则能胜;气微爪短,胆小距顿,则服畏也。人有勇怯,故战有胜负,胜者未必受金气,负者未必得木精也。孔子畏阳虎,却行流汗,阳虎未必色白,孔子未必面青也。鹰之击鸠雀,鸮之啄鹄雁,未必鹰鸮、生於南方,而鸠雀鹄雁产於西方也,自是筋力勇怯相胜服也。

一堂之上,必有论者;一乡之中,必有讼者。讼必有曲直,论必有是非,非而曲者为负,是而直者为胜。亦或辩口利舌,辞喻横出为胜;或诎弱缀跲,连蹇不比者为负。以舌论讼,犹以剑戟斗也。利剑长戟,手足健疾者胜;顿刀短矛,手足缓留者负。夫物之相胜,或以筋力,或以气势,或以巧便。小有气势,口足有便,则能以小而制大;大无骨力,角翼不劲,则以大而服小。鹊食蝟皮,博劳食蛇,蝟、蛇不便也。蚊虻之力,不如牛马,牛马困於蚊虻,蚊虻乃有势也。鹿之角,足以触犬,獼猴之手,足以博鼠,然而鹿制於犬,獼猴服於鼠,角爪不利也。故十年之牛,为牧竖所驱;长仞之象,为越僮所钩,无便故也。故夫得其便也,则以小能胜大;无其便也,则以强服於羸也。

\hypertarget{header-n140}{%
\subsubsection{奇怪篇}\label{header-n140}}

儒者称圣人之生,不因人气,更禀精於天。禹母吞薏苡而生禹,故夏姓曰姒;卨母吞燕卵而生卨,故殷姓曰子。后稷母履大人迹而生后稷,故周姓曰姬。《诗》曰:``不坼不副''。是生后稷。说者又曰:``禹、卨逆生,闿母背而出;后稷顺生,不坼不副。不感动母体,故曰``不坼不副''。逆生者子孙逆死,顺生者子孙顺亡。故桀、纣诛死,赧王夺邑''。言之有头足,故人信其说;明事以验证,故人然其文。谶书又言:``尧母庆都野出,赤龙感己,遂生尧''。《高祖本纪》言:刘媪尝息大泽之陂,梦与神遇。是时,雷电晦冥,太公往视,见蛟龙於上。已而有身,遂生高祖。其言神验,文又明著,世儒学者,莫谓不然。如实论之,虚妄言也。

彼《诗》言``不坼不副'',言其不感动母体,可也;言其母背而出,妄也。夫蝉之生复育也,闿背而出。天之生圣子,与复育同道乎?兔吮毫而怀子,及其子生,从口而出。案禹母吞薏苡,卨母咽燕卵,与兔吮毫同实也。禹、卨之母生,宜皆从口,不当闿背。夫如是,闿背之说,竟虚妄也。世间血刃死者多,未必其先祖初为人者生时逆也。秦失天下,阎乐斩胡亥,项羽诛子婴。秦之先祖伯翳,岂逆生乎?如是为顺逆之说,以验三家之祖,误矣。

且夫薏苡,草也;燕卵,鸟也;大人迹,土也,三者皆形,非气也,安能生人?说圣者,以为禀天精微之气,故其为有殊绝之知。今三家之生,以草、以鸟、以土,可谓精微乎?天地之性,唯人为贵,则物贱矣。今贵人之气,更禀贱物之精,安能精微乎?夫令鸠雀施气於雁鹄,终不成子者,何也?鸠雀之身小,雁鹄之形大也。今燕之身不过五寸,薏苡之茎不过数尺,二女吞其卵实,安能成七尺之形乎?烁一鼎之铜,以灌一钱之形,不能成一鼎,明矣。今谓大人天神,故其迹巨。巨迹之人,一鼎之烁铜也;姜原之身,一钱之形也。使大人施气於姜原,姜原之身小,安能尽得其精?不能尽得其精,则後稷不能成人。尧、高祖审龙之子,子性类父,龙能乘云,尧与高祖亦宜能焉。万物生於土,各似本种;不类土者,生不出於土,土徒养育之也。母之怀子,犹土之育物也。尧、高祖之母,受龙之施,犹土受物之播也。物生自类本种,夫二帝宜似龙也。且夫含血之类,相与为牝牡;牝牡之会,皆见同类之物。精感欲动,乃能授施。若夫牡马见雌牛,雄雀见牝鸡,不相与合者,异类故也。今龙与人异类,何能感於人而施气?或曰:夏之衰,二龙斗於庭,吐漦於地。龙亡漦在,椟而藏之。至周幽王发出龙漦,化为玄鼋,入於後宫,与处女交,遂生褒姒。玄鼋与人异类,何以感於处女而施气乎?夫玄鼋所交非正,故褒姒为祸,周国以亡。以非类妄交,则有非道妄乱之子。今尧、高祖之母,不以道接会,何故二帝贤圣,与褒姒异乎?或曰:``赵简子病,五日不知人。觉言,我之帝所,有熊来,帝命我射之,中熊,死;有罴来,我又射之,中罴,罴死。後问当道之鬼,鬼曰:``熊罴,晋二卿之先祖也。''熊罴物也,与人异类,何以施类於人,而为二卿祖?夫简子所射熊罴,二卿祖当亡,简子当昌之祆也。简子见之,若寝梦矣。空虚之象,不必有实。假令有之,或时熊罴先化为人。乃生二卿。鲁公牛哀病化为虎。人化为兽,亦如兽为人。玄鼋入後宫,殆先化为人。天地之间,异类之物,相与交接,未之有也。

天人同道,好恶均心。人不好异类,则天亦不与通。人虽生於天,犹虮虱生於人也。人不好虮虱,天无故欲生於人。何则?异类殊性,情欲不相得也。天地,夫妇也,天施气於地以生物。人转相生,精微为圣,皆因父气,不更禀取。如更禀者为圣,、後稷不圣。如圣人皆当更禀,十二圣不皆然也。黄帝、帝喾、帝颛顼、帝舜之母,何所受气?文王、武王、周公、孔子之母,何所感吞?

此或时见三家之姓,曰姒氏、子氏、姬氏,则因依放,空生怪说,犹见鼎湖之地,而著黄帝升天之说矣。失道之意,还反其字。苍颉作书,与事相连。姜原履大人迹。迹者基也,姓当为其下土,乃为女旁臣,非基迹之字,不合本事,疑非实也。以周姬况夏殷,亦知子之与姒,非燕子、薏苡也。或时禹、契、後稽之母适欲怀妊,遭吞薏苡、燕卵,履大人迹也。世好奇怪,古今同情。不见奇怪,谓德不异,故因以为姓。世间诚信,因以为然。圣人重疑,因不复定。世士浅论,因不复辨。儒生是古,因生其说。《被诗》言``不坼不副''者,言後稽之生,不感动母身也。儒生穿凿,因造禹、契逆生之说。感於龙,梦与神遇,犹此率也。尧、高祖之母,适欲怀妊,遭逢雷龙载云雨而行,人见其形,遂谓之然。梦与神遇,得圣子之象也。梦见鬼合之,非梦与神遇乎,安得其实!``野出感龙'',及``蛟龙居上'',或尧、高祖受富贵之命。龙为吉物,遭加其上,吉祥之瑞,受命之证也。光武皇帝产於济阳宫,凤皇集於地,嘉禾生於屋。圣人之生,齐鸟吉物之为瑞应。必以奇吉之物见而子生,谓之物之子,是则光武皇帝嘉禾之精,凤皇之气欤?案《帝系》之篇及《三代世表》,禹,鲧之子也;卨、稷皆帝喾之子,其母皆帝喾之妃也,及尧,亦喾之子。帝王之妃,何为适草野?古时虽质,礼已设制,帝王之妃,何为浴於水?夫如是,言圣人更禀气於天,母有感吞者,虚妄之言也。实者,圣人自有种族,如文、武各有类。孔子吹律,自知殷後;项羽重瞳,自知虞舜苗裔也。五帝、三王皆祖黄帝。黄帝圣人,本禀贵命,故其子孙皆为帝王。帝王之生,必有怪奇,不见於物,则效於梦矣。

\hypertarget{header-n149}{%
\subsection{卷四}\label{header-n149}}

\hypertarget{header-n150}{%
\subsubsection{书虚篇}\label{header-n150}}

世信虚妄之书,以为载於竹帛上者,皆贤圣所传,无不然之事,故信而是之,讽而读之;睹真是之传,与虚妄之书相违,则并谓短书不可信用。夫幽冥之实尚可知,沈隐之情尚可定,显文露书,是非易见,笼总并传,非实事,用精不专,无思於事也。

夫世间传书诸子之语,多欲立奇造异,作惊目之论,以骇世俗之人;为谲诡之书,以著殊异之名。传书言:延陵季子出游,见路有遗金。当夏五月,有披裘而薪者,季子呼薪者曰:``取彼地金来。''薪者投镰於地,瞋目拂手而言曰:``何子居之高,视之下,仪貌之壮,语言之野也!吾当夏五月,披裘而薪,岂取金者哉?''季子谢之,请问姓字。薪者曰:``子皮相之士也!何足语姓字!''遂去不顾。世以为然,殆虚言也。夫季子耻吴之乱,吴欲共立以为主,终不肯受,去之延陵,终身不还,廉让之行,终始若一。许由让天下,不嫌贪封侯。伯夷委国饥死,不嫌贪刀钩。廉让之行,大可以况小,小难以况大。季子能让吴位,何嫌贪地遗金?季子使於上国,道过徐。徐君好其宝剑,未之即予。还而徐君死,解剑带冢树而去。廉让之心,耻负其前志也。季子不负死者,弃其宝剑,何嫌一叱生人取金於地?季子未去吴乎?公子也;已去吴乎,延陵君也。公子与君,出有前後,车有附从,不能空行於涂,明矣。既不耻取金,何难使左右?而烦披裘者?世称柳下惠之行,言其能以幽冥自修洁也。贤者同操,故千岁交志。置季子於冥昧之处,尚不取金,况以白日,前後备具,取金於路,非季子之操也。或时季子实见遗金,怜披裘薪者,欲以益之;或时言取彼地金,欲以予薪者,不自取也。世俗传言,则言季子取遗金也。

传书或言:颜渊与孔子俱上鲁太山,孔子东南望,吴阊门外有系白马,引颜渊指以示之曰:``若见吴昌门乎?''颜渊曰:``见之。''孔子曰:``门外何有?''曰``有如系练之状。''孔子抚其目而正之,因与俱下。下而颜渊发白齿落,遂以病死。盖以精神不能若孔子,强力自极,精华竭尽,故早夭死。世俗闻之,皆以为然。如实论之,殆虚言也。案《论语》之文,不见此言。考《六经》之传,亦无此语。夫颜渊能见千里之外,与圣人同,孔子、诸子,何讳不言?盖人目之所见,不过十里。过此不见,非所明察,远也。传曰:``太山之高巍然,去之百里,不见垂,远也。''案鲁去吴,千有余里,使离硃望之,终不能见,况使颜渊,何能审之?如才庶几者,明目异於人,则世宜称亚圣,不宜言离硃。人目之视也,物大者易察,小者难审。使颜渊处昌门之外,望太山之形,终不能见。况从太山之上,察白马之色,色不能见,明矣。非颜渊不能见,孔子亦不能见也。何以验之?耳目之用,均也。目不能见百里,则耳亦不能闻也。陆贾曰:``离娄之明,不能察帷薄之内;师旷之聪,不能闻百里之外。''昌门之与太山,非直帷薄之内、百里之外也。

秦武王与孟说举鼎不任,绝脉而死。举鼎用力,力由筋脉,筋脉不堪,绝伤而死,道理宜也。今颜渊用目望远,望远目睛不任,宜盲眇,发白齿落,非其致也。发白齿落,用精於学,勤力不休,气力竭尽,故至於死。伯奇放流,首发早白。《诗》云:``惟忧用老。''伯奇用忧,而颜渊用睛,暂望仓卒,安能致此?

儒书言:舜葬於苍梧、禹葬於会稽者,巡狩年老,道死边土。圣人以天下为家,不别远近,不殊内外,故遂止葬。夫言舜、禹,实也;言其巡狩,虚也。舜之与尧,俱帝者也,共五千里之境,同四海之内;二帝之道,相因不殊。《尧典》之篇,舜巡狩东至岱宗,南至霍山,西至太华,北至恆山。以为四岳者,四方之中,诸侯之来,并会岳下,幽深远近,无不见者,圣人举事,求其宜适也。禹王如舜,事无所改,巡狩所至,以复如舜。舜至苍梧,禹到会稽,非其实也。实舜、禹之时,鸿水未治,尧传於舜,舜受为帝,与禹分部,行治鸿水。尧崩之後,舜老,亦以传於禹。舜南治水,死於苍梧;禹东治水,死於会嵇。贤圣家天下,故因葬焉。吴君高说:会稽本山名,夏禹巡守,会计於此山,因以名郡,故曰会稽。夫言因山名郡可也,言禹巡狩会计於此山,虚也。巡狩本不至会稽,安得会计於此山?宜听君高之说,诚会稽为会计,禹到南方,何所会计?如禹始东死於会稽,舜亦巡狩,至於苍梧,安所会计?百王治定则出巡,巡则辄会计,是则四方之山皆会计也。百王太平,升封太山。太山之上,封可见者七十有二,纷纶湮灭者,不可胜数。如审帝王巡狩辄会计,会计之地如太山封者,四方宜多。夫郡国成名,犹万物之名,不可说也。独为会稽立欤?周时旧名吴、越也,为吴、越立名,从何往哉?六国立名,状当如何?天下郡国且百余,县邑出万,乡亭聚里,皆有号名,贤圣之才莫能说。君高能说会稽,不能辨定方名。会计之说,未可从也。巡狩考正法度,禹时吴为裸国,断发文身,考之无用,会计如何?

传书言:舜葬於苍梧,象为之耕;禹葬会稽,鸟为之田。盖以圣德所致,天使鸟兽报佑之也。世莫不然。考实之,殆虚言也。夫舜、禹之德不能过尧,尧葬於冀州,或言葬於崇山,冀州鸟兽不耕,而鸟兽独为舜、禹耕,何天恩之偏驳也?或曰:``舜、禹治水,不得宁处,故舜死於苍梧,禹死於会稽。勤苦有功,故天报之;远离中国,故天痛之。''夫天报舜、禹,使鸟田象耕,何益舜、禹?天欲报舜、禹,宜使苍梧、会稽常祭祀之。使鸟兽田耕,不能使人祭。祭加舜、禹之墓,田施人民之家,天之报佑圣人,何其拙也,且无益哉!由此言之,鸟田象耕,报佑舜、禹,非其实也。实者,苍梧多象之地,会稽众鸟所居。《禹贡》曰:``彭蠡既潴,阳鸟攸居。''天地之情,鸟兽之行也。象自蹈土,鸟自食苹。土蹶草尽,若耕田状,壤靡泥易,人随种之,世俗则谓为舜、禹田。海陵麋田,若象耕状,何尝帝王葬海陵者邪?

传书言:吴王夫差杀伍子胥,煮之於镬,乃以鸱夷橐投之於江。子胥恚恨,驱水为涛,以溺杀人。今时会稽丹徒大江、钱塘浙江,皆立子胥之庙。盖欲慰其恨心,止其猛涛也。夫言吴王杀子胥投之於江,实也;言其恨恚驱水为涛者,虚也。屈原怀恨,自投湘江,湘江不为涛;申徒狄蹈河而死,河水不为涛。世人必曰:``屈原、申徒狄不能勇猛,力怒不如子胥。''夫卫菹子路而汉烹彭越,子胥勇猛不过子路、彭越。然二士不能发怒於鼎镬之中,以烹汤菹汁渖漎旁人。子胥亦自先入镬,後乃入江;在镬中之时,其神安居?岂怯於镬汤,勇於江水哉!何其怒气前後不相副也?且投於江中,何江也?有丹徒大江,有钱唐浙江,有吴通陵江。或言投於丹徒大江,无涛,欲言投於钱唐浙江。浙江、山阴江、上虞江皆有涛,三江有涛,岂分橐中之体,散置三江中乎?人若恨恚也,仇雠未死,子孙遗在,可也。今吴国已灭,夫差无类,吴为会稽,立置太守,子胥之神,复何怨苦,为涛不止,欲何求索?吴、越在时,分会稽郡,越治山阴,吴都今吴,馀暨以南属越,钱唐以北属吴。钱唐之江,两国界也。山阴、上虞在越界中,子胥入吴之江为涛,当自上吴界中,何为入越之地?怨恚吴王、发怒越江,违失道理,无神之验也。

且夫水难驱,而人易从也。生任筋力,死用精魂。子胥之生,不能从生人营卫其身,自令身死,筋力消绝,精魂飞散,安能为涛?使子胥之类数百千人,乘船渡江,不能越水。一子胥之身,煮汤镬之中,骨肉糜烂,成为羹菹,何能有害也?周宣王杀其臣杜伯,燕简公杀其臣庄子义。其後杜伯射宣王,庄子义害简公,事理似然,犹为虚言。今子胥不能完体,为杜伯、子义之事以报吴王,而驱水往来,岂报仇之义、有知之验哉?俗语不实,成为丹青;丹青之文,贤圣惑焉。夫地之有百川也,犹人之有血脉也。血脉流行,泛扬动静,自有节度。百川亦然,其朝夕往来,犹人之呼吸气出入也。天地之性,上古有之,《经》曰:``江、汉朝宗於海。''唐、虞之前也,其发海中之时,漾驰而已;入三江之中,殆小浅狭,水激沸起,故腾为涛。广陵曲江有涛,文人赋之。大江浩洋,曲江有涛,竟以隘狭也。吴杀其身,为涛广陵,子胥之神,竟无知也。溪谷之深,流者安洋,浅多沙石,激扬为濑。夫涛濑,一也。谓子胥为涛,谁居溪谷为濑者乎?案涛入三江,岸沸踊,中央无声。必以子胥为涛,子胥之身,聚岸涯也?涛之起也,随月盛衰,小大满损不齐同。如子胥为涛,子胥之怒,以月为节也?三江时风,扬疾之波亦溺杀人,子胥之神,复为风也?秦始皇渡湘水,遭风,问湘山何祠。左右对曰:``尧之女,舜之妻也。''始皇太怒,使刑徒三千人,斩湘山之树而履之。夫谓子胥之神为涛,犹谓二女之精为风也。

传书言:孔子当泗水而葬,泗水为之却流。此言孔子之德,能使水却,不湍其墓也。世人信之。是故儒者称论,皆言孔子之後当封,以泗水却流为证。如原省之,殆虚言也。夫孔子死,孰与其生?生能操行,慎道应天,死,操行绝,天佑至德,故五帝、三王招致瑞应,皆以生存,不以死亡。孔子生时,推排不容,故叹曰:``凤鸟不至,河不出图,吾已矣夫!''生时无佑,死反有报乎?孔子之死,五帝、三王,之死也。五帝、三王无佑,孔子之死独有天报,是孔子之魂圣,五帝之精不能神也。泗水无知,为孔子却流,天神使之。然则,孔子生时,天神不使人尊敬。如泗水却流,天欲封孔子之後,孔子生时,功德应天,天不封其身,乃欲封其後乎?是盖水偶自却流。江河之流,有回复之处;百川之行,或易道更路,与却流无以异。则泗水却流,不为神怪也。

传书称:魏公子之德,仁惠下士,兼及鸟兽。方与客饮,有鹯击鸠。鸠走,巡於公子案下。追击,杀於公子之前,公子耻之,即使人多设罗,得鹯数十枚,责让以击鸠之罪。击鸠之鹯,低头不敢仰视,公子乃杀之。鹯世称之曰:``魏公子为鸠报仇。''此虚言也。夫鹯,物也,情心不同,音语不通。圣人不能使鸟兽为义理之行,公子何人,能使鹯低头自责?鸟为者以千万数,向击鸠蜚去,安可复得?能低头自责,是圣鸟也。晓公子之言,则知公子之行矣。知公子之行,则不击鸠於其前。人犹不能改过,鸟与人异,谓之能悔,世俗之语,失物类之实也。或时公子实捕鹯,鹯得。人持其头,变折其颈,疾痛低垂,不能仰视。缘公子惠义之人,则因褒称,言鹯服过。盖言语之次,空生虚妄之美;功名之下,常有非实之加。

传书言:齐桓公妻姑姊妹七人。此言虚也。夫乱骨肉,犯亲戚,无上下之序者,禽兽之性,则乱不知伦理。案桓公九合诸侯,一匡天下,道之以德,将之以威,以故诸侯服从,莫敢不率,非内乱怀鸟兽之性者所能为也。夫率诸侯朝事王室,耻上无势而下无礼也。外耻礼之不存,内何犯礼而自坏?外内不相副,则功无成而威不立矣。世称桀、纣之恶,不言淫於亲戚。实论者谓夫桀、纣恶微於亡秦,亡秦过泊於王莽,无淫乱之言。桓公妻姑姊七人,恶浮於桀、纣,而过重於秦、莽也。《春秋》采毫毛之美,贬纤芥之恶。桓公恶大,不贬何哉?鲁文姜,齐襄公之妹也,襄公通焉。《春秋》经曰:``庄二年冬,夫人姜氏会齐侯於郜。''《春秋》何尤於襄公,而书其奸?何宥於桓公,隐而不讥?如经失之,传家左丘明、公羊、谷梁何讳不言?案桓公之过,多内宠,内嬖如夫人者六。有五公子争立,齐乱,公薨三月乃讣。世闻内嬖六人,嫡庶无别,则言乱於姑姊妹七人矣。

传书言:齐桓公负妇人而朝诸侯,此言桓公之淫乱无礼甚也。夫桓公大朝之时,负妇人於背,其游宴之时,何以加此?方修士礼,崇历肃敬,负妇人於背,何以能率诸侯朝事王室?葵丘之会,桓公骄矜,当时诸侯畔者九国。睚眦不得,九国畔去,况负妇人,淫乱之行,何以肯留?或曰:``管仲告诸侯:吾君背有疽创,不得妇人,疮不衰愈。诸侯信管仲,故无畔者。''夫十室之邑,必有忠信若孔子。当时诸侯千人以上,必知方术治疽,不用妇人。管仲为君讳也,诸侯知仲为君讳而欺己,必恚怒而畔去,何以能久统会诸侯,成功於霸?或曰:``桓公实无道,任贤相管仲,故能霸天下。''夫无道之人,与狂无异,信谗远贤,反害仁义,安能任管仲,能养人令之成事:桀杀关龙逢,纣杀王子比干,无道之君莫能用贤。使管仲贤,桓公不能用;用管仲,故知桓公无乱行也。有贤明之君,故有贞良之臣。臣贤,君明之验,奈何谓之有乱?难曰:``卫灵公无道之君,时知贤臣。管仲为辅,何明桓公不为乱也?''夫灵公无道,任用三臣,仅以不丧,非有功行也。桓公尊九九之人,拔宁戚於车下,责苞茅不贡,运兵功楚,九合诸侯,一匡天下,千世一出之主也。而云负妇人於背,虚矣。说《尚书》者曰:``周公居摄,带天子之绶,戴天子之冠,负扆南面而朝诸侯。''户牖之间曰扆,南面之坐位也。负南面乡坐,扆在後也。桓公朝诸侯之时,或南面坐,妇人立於後也。世俗传云,则曰负妇人於背矣。此则夔一足、宋丁公凿井得一人之语也。唐、虞时,夔为大夫,性知音乐,调声悲善。当时人曰:``调乐如夔一足矣。''世俗传言:``夔一足。''案秩宗官缺,帝舜博求,众称伯夷,伯夷稽首让於夔龙。秩宗卿官,汉之宗正也。断足,非其理也。且一足之人,何用行也?夏後孔甲,田於东蓂山,天雨晦冥,入於民家,主人方乳。或曰:``後来之子必贵。''或曰:``不胜,之子必贱。''孔甲曰:``为余子,孰能贱之?''遂载以归,析缭,斧斩其足,卒为守者。孔甲之欲贵之子,有余力矣,断足无宜,故为守者。今夔一足,无因趋步,坐调音乐,可也;秩宗之官,不宜一足,犹守者断足,不可贵也。孔甲不得贵之子,伯夷不得让於夔焉。宋丁公者,宋人也。未凿井时,常有寄汲,计之,日去一人作。自凿井後,不复寄汲,计之,日得一人之作。故曰:``宋丁公凿井得一人。''俗传言曰:``丁公凿井得一人於井中。''夫人生於人,非生於土也。穿土凿井,无为得人。推此以论,负妇人之语,犹此类也。负妇人而坐,则云妇人在背。知妇人在背非道,则生管仲以妇人治疽之言矣。使桓公用妇人彻胤服,妇人於背;女气疮可去,以妇人治疽。方朝诸侯,桓公重衣,妇人袭裳,女气分隔,负之何益?桓公思士,作庭燎而夜坐,以思致士,反以白日负妇人见诸侯乎?

传书言聂正为严翁仲刺杀韩王,此虚也。夫聂政之时,韩列侯也。列侯之三年,聂政刺韩相侠累。十二年列侯卒。与聂政杀侠累,相去十七年。而言聂政刺杀韩王,短书小传,竟虚不可信也。

传书又言:燕太子丹使刺客荆轲刺秦王,不得,诛死。後高渐丽复以击筑见秦王,秦王说之;知燕太子之客,乃冒其眼,使之击筑。渐丽乃置铅於筑中以为重,当击筑,秦王膝进,不能自禁。渐丽以筑击秦王颡,秦王病伤,三月而死。夫言高渐丽以筑击秦王,实也;言中秦王病伤三月而死,虚也。夫秦王者,秦始皇帝也。

始皇二十年,燕太子丹使荆轲刺始皇,始皇杀轲,明矣。二十一年,使将军王翦功燕,得太子首;二十五年,遂伐燕,而虏燕王嘉。後不审何年,高渐丽以筑击始皇,不中,诸渐丽。当二十七年,游天下,到会稽,至琅邪,北至劳、盛山,并海,西至平原津而病,到沙丘平台,始皇崩。夫谶书言始皇还,到沙丘而亡;传书又言病筑疮三月而死於秦。一始皇之身,世或言死於沙丘,或言死於秦,其死言恆病疮。传书之言,多失其实,世俗之人,不能定也。

\hypertarget{header-n168}{%
\subsubsection{变虚篇}\label{header-n168}}

传书曰:宋景公之时,荧惑守心,公惧,召子韦而问之曰:``荧惑在心,何也?''子韦曰:``荧惑,天罚也,心,宋分野也,祸当君。虽然,可移於宰相。''公曰:``宰相所使治国家也,而移死焉,不祥。''子韦曰:``可移於民。''公曰:``民死,寡人将谁为也?宁独死耳。''子韦曰:``可移於岁。''公曰:``民饥,必死。为人君而欲杀其民以自活也,其谁以我为君者乎?是寡人命固尽也,子毋复言。''子韦退走,北面再拜曰:``臣敢贺君。天之处高而耳卑,君有君人之言三,天必三赏君。今夕星必徙三舍,君延命二十一年。''公曰:``奚知之?''对曰:``君有三善,故有三赏,星必三徙。徙行七星,星当一年,三七二十一,故君命延二十一岁。臣请伏於殿下以伺之,星必不徙,臣请死耳。''是夕也,火星果徙三舍。如子韦之言,则延年审得二十一岁矣。星徙审则延命,延命明则景公为善,天佑之也。则夫世间人能为景公之行者,则必得景公佑矣。此言虚也。何则?皇天迁怒,使荧惑本景公身为有恶而守心,则虽听子韦言,犹无益也。使其不为景公,则虽不听子韦之言,亦无损也。

齐景公时有彗星,使人禳之。晏子曰:``无益也,只取诬焉。天道不暗,不贰其命,若之何禳之也?且天之有彗,以除秽也。君无秽德,又何禳焉?若德之秽,禳之何益?《诗》曰:``惟此文王,小心翼翼,昭事上帝,聿怀多福;厥德不回,以受方国。''君无回德,方国将至,何患於彗?《诗》曰:我无所监,夏後及商,用乱之故,民卒流亡。若德回乱,民将流亡,祝史之为,无能补也。公说,乃止。齐君欲禳彗星之凶,犹子韦欲移荧惑之祸也。宋君不听,犹晏子不肯从也。则齐君为子韦,晏子为宋君也。同变共祸,一事二人。天犹贤宋君,使荧惑徙三舍,延二十一年,独不多晏子使彗消而增其寿,何天佑善偏驳之齐一也?人君有善行,善行动於心,善言出於意,同由共本,一气不异。宋景公出三善言,则其先三善言之前,必有善行也。有善行,必有善政,政善,则嘉瑞臻,福祥至,荧惑之星无为守心也。使景公有失误之行,以致恶政,恶政发,则妖异见,荧惑之守心,桑谷不生朝。高宗消桑谷之变,以政不以言;景公却荧惑之异,亦宜以行。景公有恶行,故荧惑守心。不改政修行,坐出三善言,安能动天?天安肯应!何以效之?使景公出三恶言,能使荧惑守心乎?夫三恶言不能使荧惑守心,三善言安能使荧惑退徙三舍?以三善言获二十一年,如有百善言,得千岁之寿乎?非天佑善之意,应诚为福之实也。

子韦之言:``天处高而听卑,君有君人之言三,天必三赏君。''夫天体也,与地无异。诸有体者,耳咸附於首。体与耳殊,未之有也。天之去人,高数万里,使耳附天,听数万里之语,弗能闻也。人坐楼台之上,察地之蝼蚁,尚不见其体,安能闻其声。何则?蝼蚁之体细,不若人形大,声音孔气不能达也。今天之崇高非直楼台,人体比於天,非若蝼蚁於人也。谓天非若蝼蚁于人也。谓天闻人言,随善恶为吉凶,误矣。四夷入诸夏,因译而通。同形均气,语不相晓。虽五帝三王,不能去译独晓四夷,况天与人异体、音与人殊乎?人不晓天所为,天安能知人所行。使天体乎,耳高不能闻人言;使天气乎,气若云烟,安能听人辞?说灾变之家曰:``人在天地之间,犹鱼在水中矣。其能以行动天地,犹鱼鼓而振水也,鱼动而水荡气变。''此非实事也。假使真然,不能至天。鱼长一尺,动於水中,振旁侧之水,不过数尺,大若不过与人同,所振荡者不过百步,而一里之外淡然澄静,离之远也。今人操行变气,远近宜与鱼等;气应而变,宜与水均。以七尺之细形,形中之微气,不过与一鼎之蒸火同。从下地上变皇天,何其高也!且景公贤者也。贤者操行,上不及圣人,下不过恶人。世间圣人,莫不尧、舜,恶人,莫不桀、纣。尧、舜操行多善,无移荧惑之效;桀、纣之政多恶,有反景公脱祸之验。景公出三善言,延年二十一岁,是则尧、舜宜获千岁,桀纣宜为殇子。今则不然,各随年寿,尧、舜、桀、纣皆近百载。是竟子韦之言妄,延年之语虚也。且子韦之言曰:``荧惑,天使也;心,宋分野也。祸当君。''若是者,天使荧惑加祸於景公也,如何可移於将相、若岁与国民乎?天之有荧惑也,犹王者之有方伯也。诸侯有当死之罪,使方伯围守其国,国君问罪於臣,臣明罪在君。虽然,可移於臣子与人民。设国君计其言,令其臣归罪於国人,方伯闻之,肯听其言,释国君之罪,更移以付国人乎?方伯不听者,自国君之罪,非国人之辜也。方伯不听自国人之罪,荧惑安肯移祸於国人!若此,子韦之言妄也。曰:景公听乎言、庸何能动天?使诸侯不听其臣言,引过自予。方伯闻其言,释其罪,委之去乎?方伯不释诸侯之罪,荧惑安肯徙去三舍?夫听与不听,皆无福善,星徙之实,未可信用。天人同道,好恶不殊。人道不然,则知天无验矣。

宋、卫、陈、郑之俱灾也,气变见天。梓慎知之,请於子产有以除之,子产不听。天道当然,人事不能却也。使子产听梓慎,四国能无灾乎?尧遭鸿水时,臣必有梓慎、子韦之知矣。然而不却除者,尧与子产同心也。案子韦之言曰:``荧惑,天使也;心,宋分野也。祸当君。''审如此言,祸不可除,星不可却也。若夫寒温失和,风雨不时,政事之家,谓之失误所致,可以善政贤行变而复也。若荧惑守心,若必死,犹亡祸安可除?修政改行,安能却之?善政贤行,尚不能却,出虚华之三言,谓星却而祸除,增寿延年,享长久之福,误矣。观子韦之言景公,言荧惑之祸,非寒暑风雨之类,身死命终之祥也。国且亡,身且死,祆气见於天,容色见於面。面有容色,虽善操行不能灭,死征已见也。在体之色,不可以言行灭;在天之妖,安可以治除乎?人病且死,色见於面,人或谓之曰:``此必死之征也。虽然,可移於五邻,若移於奴役。''当死之人,正言不可,容色肯为善言之故灭,而当死之命,肯为之长乎?气不可灭,命不可长。然则荧惑安可却?景公之年安可增乎?由此言之,荧惑守心,未知所为,故景公不死也。

且言``星徙三舍''者,何谓也?星三徙於一舍乎?一徙历於三舍也?案子韦之言曰:``君有君人之言三,天必三赏君,今夕星必徙三舍。''若此,星竟徙三舍也。夫景公一坐有三善言,星徙三舍,知有十善言,星徙十舍乎?荧惑守心,为善言却,如景公复出三恶言,荧惑食心乎?为善言却,为恶言进,无善无恶,荧惑安居不行动乎?或时荧惑守心为旱灾,不为君薨。子韦不知,以为死祸。信俗至诚之感,荧惑去处星,必偶自当去,景公自不死,世则谓子韦之言审,景公之诚感天矣。亦或时子韦知星行度适自去,自以著己之知,明君臣推让之所致;见星之数七,因言星七舍,复得二十一年,因以星舍计年之数。是与齐太卜无以异也。齐景公问太卜曰:``子之道何能?''对曰:``能动地。''晏子往见公,公曰:``寡人问太卜曰:`子道何能?'对曰:`能动地。'地固可动乎?''晏子嘿然不对,出见太卜曰:``昔吾见钩星在房、心之间,地其动乎?''太卜曰:``然。''晏子出,太卜走见公:``臣非能动地,地固将自动。''夫子韦言星徙,犹太卜言地动也。地固且自动,太卜言己能动之。星固将自徙,子韦言君能徙之。使晏子不言钩星在房、心,则太卜之奸对不觉。宋无晏子之知臣,故子韦之一言,遂为其是。案《子韦书录序秦》亦言:``子韦曰:`君出三善言,荧惑宜有动'。''於是候之,果徙舍。''不言``三''。或时星当自去,子韦以为验,实动离舍,世增言``三''。既空增三舍之数,又虚生二十一年之寿也。

\hypertarget{header-n177}{%
\subsection{卷五}\label{header-n177}}

\hypertarget{header-n178}{%
\subsubsection{异虚篇}\label{header-n178}}

殷高宗之时,桑谷俱生於朝,七日而大拱。高宗召其相而问之,相曰:``吾虽知之,弗能言也。''问祖己,祖己曰:``夫桑谷者,野草也,而生於朝,意朝亡乎?''高宗恐骇,侧身而行道,思索先王之政,明养老之义,兴灭国,继绝世,举佚民。桑谷亡。三年之後,诸侯以译来朝者六国,遂享百年之福。高宗,贤君也,而感桑谷生。而问祖己,行祖己之言,修政改行。桑谷之妖亡,诸侯朝而年长久。修善之义笃,故瑞应之福渥。此虚言也。

祖己之言``朝当亡''哉!夫朝之当亡,犹人当死。人欲死,怪出。国欲亡,期尽。人死命终,死不复生,亡不复存。祖己之言政,何益於不亡?高宗之修行,何益於除祸?夫家人见凶修善,不能得吉;高宗见妖改政,安能除祸?除祸且不能,况能招致六国,延期至百年乎!故人之死生,在於命之夭寿,不在行之善恶;国之存亡,在期之长短,不在於政之得失。案祖己之占,桑谷为亡之妖,亡象已见,虽修孝行,其何益哉!何以效之?

鲁昭公之时,瞿鹆来巢。师己采文、成之世童谣之语,有瞿鹆之言,见今有来巢之验,则占谓之凶。其後,昭公为季氏所逐,出於齐,国果空虚,都有虚验。故野鸟来巢,师己处之,祸竟如占。使昭公闻师己之言,修行改政为善,居高宗之操,终不能消。何则?瞿鹆之谣已兆,出奔之祸已成也。瞿鹆之兆,已出於文、成之世矣。根生,叶安得不茂?源发,流安得不广?此尚为近,未足以言之。夏将衰也,二龙战於庭,吐漦而去,夏王椟而藏之。夏亡,传於殷;殷亡,传於周,皆莫之发。至幽王之时,发而视之,漦流於庭,化为玄鼋,走入後宫,与妇人交,遂生褒姒。褒姒归周,历王惑乱,国遂灭亡。幽、历王之去夏世,以为千数岁,二龙战时,幽、厉、褒姒等未为人也。周亡之妖,已出久矣。妖出,祸安得不就?瑞见,福安得不至?若二龙战时言曰:``余褒之二君也。''是则褒姒当生之验也。龙称褒,褒姒不得不生,生则厉王不得不恶,恶则国不得不亡。征已见,虽五圣十贤相与却之,终不能消。善恶同实:善祥出,国必兴;恶祥见,朝必亡。谓恶异可以善行除,是谓善瑞可以恶政灭也。

河源出於昆仑,其流播於九河。使尧、禹却以善政,终不能还者,水势当然,人事不能禁也。河源不可禁,二龙不可除,则桑谷不可却也。王命之当兴也,犹春气之当为夏也。其当亡也,犹秋气之当为冬也。见春之微叶,知夏有茎叶。睹秋之零实,知冬之枯萃。桑谷之生,其犹春叶秋实也,必然犹验之。今详修政改行,何能除之?夫以周亡之祥,见於夏时,又何以知桑谷之生,不为纣亡出乎!或时祖己言之,信野草之占,失远近之实。高宗问祖己之後,侧身行道,六国诸侯偶朝而至,高宗之命自长未终,则谓起桑谷之问,改行修行,享百年之福矣。夫桑谷之生,殆为纣出,亦或时吉而不凶,故殷朝不亡,高宗寿长。祖己信野草之占,谓之当亡之征。

汉孝武皇帝之时,获白麟戴两角而共牴,使谒者终军议之。军曰:``夫野兽而共一角,象天下合同为一也。''麒麟野兽也,桑谷野草也,俱为野物,兽草何别?终军谓兽为吉,祖己谓野草为凶。高宗祭成汤之庙,有蜚雉升鼎而雊。祖己以为远人将有来者,说《尚书》家谓雉凶,议驳不同。且从祖己之言,雉来吉也,雉伏於野草之中,草覆野鸟之形,若民人处草庐之中,可谓其人吉而庐凶乎?民人入都,不谓之凶,野草生朝,何故不吉?雉则民人之类。如谓含血者吉,长狄来至,是吉也,何故谓之凶?如以从夷狄来者不吉,介葛卢来朝,是凶也。如以草木者为凶,硃草、蓂荚出,是不吉也。硃草、蓂荚,皆草也,宜生於野,而生於朝,是为不吉。何故谓之瑞?一野之物,来至或出,吉凶异议。硃草荚善草,故为吉,则是以善恶为吉凶,不以都野为好丑也。周时天下太平,越尝献雉於周公。高宗得之而吉。雉亦草野之物,何以为吉?如以雉所分有似於士,则麏亦仍有似君子;公孙术得白鹿,占何以凶?然则雉之吉凶未可知,则夫桑谷之善恶未可验也。桑谷或善物,象远方之士将皆立於高宗之朝,故高宗获吉福,享长久也。

说灾异之家,以为天有灾异者,所以谴告王者,信也。夫王者有过,异见於国;不改,灾见草本;不改,灾见於五谷;不改,灾至身。左氏《春秋传》曰:``国之将亡,鲜不五稔。''灾见於五谷,五谷安得熟?不熟,将亡之征。灾亦有且亡五谷不熟之应。天不熟,或为灾,或为福。祸福之实未可知,桑谷之言安可审?论说之家著於书记者皆云:``天雨谷者凶。''传书曰:``苍颉作书,天雨谷,鬼夜哭。''此方凶恶之应。和者,天用成谷之道,从天降而和,且犹谓之善,况所成之谷从雨下乎!极论订之,何以为凶?夫阴阳和则谷稼成,不则被灾害。阴阳和者,谷之道也,何以谓之凶?丝成帛,缕成布。赐人丝缕,犹为重厚,况遗人以成帛与织布乎?夫丝缕犹阴阳,帛布犹成谷也。赐人帛,不谓之恶,天与之谷何,故谓之凶?夫雨谷吉凶未可定,桑谷之言未可知也。

使暢草生於周之时,天下太平,人来献暢草。暢草亦草野之物也,与彼桑谷何异?如以夷狄献之则为吉,使暢草生於周家,肯谓之善乎?夫暢草可以炽酿,芬香暢达者,将祭灌暢降神。设自生於周朝,与嘉禾、硃草、蓂荚之类不殊矣。然则桑亦食蚕,蚕为丝,丝为帛,帛为衣。衣以入宗庙为朝服,与暢无异。何以谓之凶?卫献公太子至灵台,蛇绕左轮。御者曰:``太子下拜,吾闻国君之子,蛇绕车轮左者速得国。''太子遂不下,反乎舍。御人见太子,太子曰:``吾闻为人子者,尽和顺於君,不行私欲,共严承令,不逆君安。今吾得国,是君失安也。见国之利而忘君安,非子道也。得国而拜,其非君欲。废子道者不孝,逆君欲则不忠。而欲我行之,殆欲吾国之危明矣。''投殿将死,其御止之,不能禁,遂伏剑而死。夫蛇绕左轮,审为太子速得国,太子宜不死,献公宜疾薨。今献公不死,太子伏剑,御者之占,俗之虚言也。或时蛇为太子将死之妖,御者信俗之占,故失吉凶之实。夫桑谷之生,与蛇饶左轮相似类也。蛇至实凶,御者以为吉。桑谷实吉,祖己以为凶。

禹南济於江,有黄龙负舟。舟中之人五色无主。禹乃嘻笑而称曰:``我受命於天,竭力以劳万民。生,寄也;死,归也。何足以滑和,视龙犹蝘蜓也。''龙去而亡。案古今龙至皆为吉,而禹独谓黄龙凶者,见其负舟,舟中之人恐也。夫以桑谷比於龙,吉凶虽反,盖相似。野草生於朝,尚为不吉,殆有若黄龙负舟之异。故为吉而殷朝不亡。

晋文公将与楚成王战於城濮,彗星出楚。楚操其柄,以问咎犯,咎犯对曰:``以彗斗,倒之者胜。''文公梦与成王博,成王在上,盬其脑。问咎犯,咎犯曰:``君得天而成王伏其罪,战必大胜。''文公从之,大破楚师。向令文公问庸臣,必曰不胜。何则?彗星无吉,搏在上无凶也。夫桑谷之占,占为凶,犹晋当彗末,博在下为不吉也。然而吉者,殆有若对彗见天之诡。故高宗长久,殷朝不亡。使文公不问咎犯,咎犯不明其吉,战以大胜,世人将曰:``文公以至贤之德,破楚之无道。天虽见妖,卧有凶梦,犹灭妖消凶以获福。''殷无咎犯之异知,而有祖己信常之占,故桑谷之文,传世不绝,转祸为福之言,到今不实。

\hypertarget{header-n190}{%
\subsubsection{感虚篇}\label{header-n190}}

儒者传书言:``尧之时,十日并出,万物焦枯。尧上射十日,九日去,一日常出''。此言虚也。夫人之射也,不过百步,矢力尽矣。日之行也,行天星度。天之去人,以万里数,尧上射之,安能得日?使尧之时,天地相近,不过百步,则尧射日,矢能及之;过百步,不能得也。假使尧时天地相近,尧射得之,犹不能伤日。伤日何肯去?何则?日,火也。使在地之火附一把矩,人从旁射之,虽中,安能灭之?地火不为见射而灭,天火何为见射而去?此欲言尧以精诚射之,精诚所加,金石为亏,盖诚无坚则亦无远矣。夫水与火,各一性也。能射火而灭之,则当射水而除之。洪水之时,流滥中国,为民大害。尧何不推精诚射而除之?尧能射日,使火不为害,不能射河,使水不为害。夫射水不能却水,则知射日之语,虚非实也。或曰:``日,气也。射虽不及,精诚灭之''。夫天亦远,使其为气,则与日月同;使其为体,则与金石等。以尧之精诚,灭日亏金石,上射日则能穿天乎?世称桀、纣之恶,射天而殴地;誉高宗之德,政消桑谷。今尧不能以德灭十日,而必射之;是德不若高宗,恶与桀、纣同也。安能以精诚获天之应也?

传书言:武王伐纣,渡孟津,阳侯之波逆流而击,疾风晦冥,人马不见。於是武王左操黄钺,右执白旄,瞋目而麾之曰:``余在,天下谁敢害吾意者!''於是风霁波罢。此言虚也。武王渡孟津时,士众喜乐,前歌後舞。天人同应,人喜天怒,非实宜也。前歌後舞,未必其实。麾风而止之,迹近为虚。夫风者,气也;论者以为天地之号令也。武王诛纣是乎,天当安静以佑之;如诛纣非乎,而天风者,怒也。武王不奉天令,求索己过,瞋目言曰``余在,天下谁敢害吾者'',重天怒、增己之恶也,风何肯止?父母怒,子不改过,瞋目大言,父母肯贳之乎?如风天所为,祸气自然,是亦无知,不为瞋目麾之故止。夫风犹雨也,使武王瞋目以旄麾雨而止之乎?武王不能止雨,则亦不能止风。或时武王适麾之,风偶自止,世褒武王之德,则谓武王能止风矣。

传书言:鲁〔阳〕公与韩战,战酣,日暮,公援戈而麾之,日为之反三舍。此言虚也。凡人能以精诚感动天,专心一意,委务积神,精通於天,天为变动,然尚未可谓然。〔阳〕公志在战,为日暮一麾,安能令日反?使圣人麾日,日终之反。〔阳〕公何人,而使日反乎?《鸿范》曰:``星有好风,星有好雨。日月之行,则有冬有夏。月之从星,则有风雨。''夫星与日月同精,日月不从星,星辄复变。明日月行有常度,不得从星之好恶也,安得从〔阳〕公之所欲?星之在天也,为日月舍,犹地有邮亭,为长吏廨也。二十八舍有分度,一舍十度,或增或减。言日反三舍,乃三十度也。日,日行一度。一麾之间,反三十日时所在度也?如谓舍为度,三度亦三日行也。一麾之间,令日却三日也。宋景公推诚出三善言,荧惑徙三舍。实论者犹谓之虚。〔阳〕公争斗,恶日之暮,以此一戈麾,无诚心善言,日为之反,殆非其意哉!且日,火也,圣人麾火,终不能却;〔阳〕公麾日,安能使反?或时战时日正卯,战迷,谓日之暮,麾之,转左曲道,日若却。世好神怪,因谓之反,不道所谓也。

传书言:荆轲为燕子谋刺秦王,白虹贯日。卫先生为秦画长平之事,太白蚀昴。此言精感天,天为变动也。夫言白虹贯日,太白蚀昴,实也。言荆轲之谋,卫先生之画,感动皇天,故白虹贯日,太白蚀昴者,虚也。夫以箸撞钟,以算击鼓,不能鸣者,所用撞击之者,小也。今人之形不过七尺,以七尺形中精神,欲有所为,虽积锐意,犹箸撞钟、算击鼓也,安能动天?精非不诚,所用动者小也。且所欲害者人也,人不动,天反动乎!问曰:``人之害气,能相动乎?''曰:``不能!''``豫让欲害赵襄子,襄子心动。贯高欲篡高祖,高祖亦心动。二子怀精,故两主振感。''曰:``祸变且至,身自有怪,非适人所能动也。何以验之?时或遭狂人於途,以刃加己,狂人未必念害己身也,然而己身先时已有妖怪矣。由此言之,妖怪之至,祸变自凶之象,非欲害己者之所为也。且凶之人卜得恶兆,筮得凶卦,出门见不吉,占危睹祸气,祸气见於面,犹白虹太白见於天也。变见於天,妖出於人,上下适然,自相应也。''

传书言:``燕太子丹朝於秦,不得去,从秦王求归。秦王执留之,与之誓曰:`使日再中,天雨粟,令乌白头,马生角,厨门木象生肉足,乃得归。'当此之时,天地佑之,日为再中,天雨粟,乌白头,马生角,厨门木象生肉足。秦王以为圣,乃归之。''此言虚也。燕太子丹何人,而能动天?圣人之拘,不能动天,太子丹贤者也,何能致此?夫天能佑太子,生诸瑞以免其身,则能和秦王之意以解其难。见拘一事而易,生瑞五事而难。舍一事之易,为五事之难,何天之不惮劳也?汤困夏台,文王拘羑里,孔子厄陈、蔡。三圣之困,天不能佑,使拘之者睹佑知圣,出而尊厚之。或曰:``拘三圣者,不与三誓,三圣心不愿,故佑圣之瑞无因而至。天之佑人,犹借人以物器矣。人不求索,则弗与也。''曰:``太子愿天下瑞之时,岂有语言乎!''心愿而已。然汤闭於夏台,文王拘於羑里,时心亦愿出;孔子厄陈、蔡,心愿食。天何不令夏台、

羑里关钥毁败,汤、文涉出;雨粟陈、蔡,孔子食饱乎?太史公曰:``世称太子丹之令天雨粟、马生角,大抵皆虚言也。''太史公书汉世实事之人,而云``虚言'',近非实也。

传书言:杞梁氏之妻向城而哭,城为之崩。此言杞梁从军不还,其妻痛之,向城而哭,至诚悲痛,精气动城,故城为之崩也。夫言向城而哭者,实也。城为之崩者,虚也。夫人哭悲莫过雍门子。雍门子哭对孟尝君,孟尝君为之於邑。盖哭之精诚,故对向之者凄怆感动也。夫雍门子能动孟尝之心,不能感孟尝衣者,衣不知恻怛,不以人心相关通也。今城,土也。土犹衣也,无心腹之藏,安能为悲哭感动而崩?使至诚之声能动城土,则其对林木哭,能折草破木乎?向水火而泣,能涌水灭火乎?夫草木水火与土无异,然杞梁之妻不能崩城,明矣。或时城适自崩,杞梁妻适哭。下世好虚,不原其实,故崩城之名,至今不灭。

传书言:邹衍无罪,见拘於燕,当夏五月,仰天而叹,天为陨霜。此与杞梁之妻哭而崩城,无以异也。言其无罪见拘,当夏仰天而叹,实也。言天为之雨霜,虚也。夫万人举口并解吁嗟,犹未能感天,皱衍一人冤而壹叹,安能下霜?邹衍之冤不过曾子、伯奇。曾子见疑而吟,伯奇被逐而歌。疑、〔逐〕与拘同。吟、歌与叹等。曾子、伯奇不能致寒,邹衍何人,独能雨霜?被逐之冤,尚未足言。申生伏剑,子胥刎颈。实孝而赐死,诚忠而被诛。且临死时,皆有声辞,声辞出口,与仰天叹无异。天不为二子感动,独为邹衍动,岂天痛见拘,不悲流血哉?伯冤痛相似,而感动不同也?夫然一炬火,爨一镬水,终日不能热也;倚一尺冰,置庖厨中,终夜不能寒也。何则?微小之感不能动大巨也。今邹衍之叹,不过如一炬、尺冰,而皇天巨大,不徒镬水庖厨之丑类也。一仰天叹,天为陨霜。何天之易感,霜之易降也?夫哀与乐同,喜与怒均。衍兴怨痛,使天下霜,使衍蒙非望之赏,仰天而笑,能以冬时使天热乎?变复之家曰:``人君秋赏则温,夏罚则寒。''寒不累时,则霜不降,温不兼日,则冰不释。一夫冤而一叹,天辄下霜,何气之易变,时之易转也?寒温自有时,不合变复之家。且从变复之说,或时燕王好用刑,寒气应至;而衍囚拘而叹,叹时霜适自下。世见适叹而霜下,则谓邹衍叹之致也。

传书言:师旷奏《白雪》之曲,而神物下降,风雨暴至。平公因之癃病,晋国赤地。或言师旷《清角》之曲,一奏之,有云从西北起:再奏之,大风至,大雨随之,裂帷幕,破俎豆,堕廊瓦。坐者散走。平公恐惧,伏乎廊室。晋国大旱,赤地三年;平公癃病。夫《白雪》与《清角》,或同曲而异名,其祸败同一实也。传书之家,载以为是;世俗观见,信以为然。原省其实,殆虚言也。夫《清角》,何音之声而致此?``《清角》,木音也,故致风雨,如木为风,雨与风俱。''三尺之木,数弦之声,感动天地,何其神也!此复一哭崩城、一叹下霜之类也。师旷能鼓《清角》,必有所受,非能质性生出之也。其初受学之时,宿昔习弄,非直一再奏也。审如传书之言,师旷学《清角》时,风雨当至也。

传书言:``瓠芭鼓瑟,渊鱼出听;师旷鼓琴,六马仰秣''。或言:``师旷鼓《清角》,一奏之,有玄鹤二八自南方来,集於廊门之危;再奏之而列;三奏之,延颈而鸣,舒翼而舞,音中宫商之声,声吁於天。平公大悦,坐者皆喜''。《尚书》曰:``击石拊石,百兽率舞。''此虽奇怪,然尚可信。何则?鸟兽好悲声,耳与人耳同也。禽兽见人欲食,亦欲食之;闻人之乐,何为不乐?然而``鱼听''、``仰秣''、``玄鹤延颈''、``百兽率舞'',盖且其实;风雨之至、晋国大旱、赤地三年、平公癃病,殆虚言也。或时奏《清角》时,天偶风雨、风雨之後,晋国适旱;平公好乐,喜笑过度,偶发癃病。传书之家,信以为然,世人观见,遂以为实。实者乐声不能致此。何以验之?风雨暴至,是阴阳乱也。乐能乱阴阳,则亦能调阴阳也。王者何须修身正行,扩施善政?使鼓调阴阳之曲,和气自至,太平自立矣。

传书言:``汤遭七年旱,以身祷於桑林,自责以六过,天乃雨''。或言:``五年。祷辞曰:`余一人有罪,无及万夫。万夫有罪,在余一人。天以一人不敏,使上帝鬼神伤民之命'。於是剪其发,丽其手,自以为牲,用祈福於上帝。上帝甚说,时雨乃至。言汤以身祷於桑林自责,若言剪发丽手,自以为牲,用祈福於帝者,实也。言雨至为汤自责以身祷之故,殆虚言也。孔子疾病,子路请祷。孔子曰:``有诸?''子路曰:``有之;《诔》曰:`祷尔於上下神祗。'''孔子曰:``丘之祷,久矣。''圣人修身正行,素祷之日久,天地鬼神知其无罪,故曰祷久矣。《易》曰:``大人与天地合其德,与日月合其明,与四时合其叙,与鬼神合其吉凶。''此言圣人与天地、鬼神同德行也。即须祷以得福,是不同也。汤与孔子俱圣人也,皆素祷之日久。孔子不使子路祷以治病,汤何能以祷得雨?孔子素祷,身犹疾病。汤亦素祷,岁犹大旱。然则天地之有水旱,犹人之有疾病也。疾不可以自责除,水旱不可以祷谢去,明矣。汤之致旱,以过乎?是不与天地同德也。今不以过致旱乎?自责祷谢,亦无益也。人形长七尺,形中有五常,有瘅热之病,深自克责,犹不能愈,况以广大之天,自有水旱之变。汤用七尺之形,形中之诚,自责祷谢,安能得雨邪?人在层台之上,人从层台下叩头,求请台上之物。台上之人闻其言,则怜而与之;如不闻其言,虽至诚区区,终无得也。夫天去人,非徒层台之高也,汤虽自责,天安能闻知而与之雨乎?夫旱,火变也;湛,水异也。尧遭洪水,可谓湛矣。尧不自责以身祷祈,必舜、禹治之,知水变必须治也。除湛不以祷祈,除旱亦宜如之。由此言之,汤之祷祈,不能得雨。或时旱久,时当自雨;汤以旱久,亦适自责。世人见雨之下,随汤自责而至,则谓汤以祷祈得雨矣。

传书言:``仓颉作书,天雨粟,鬼夜哭。''此言文章兴而乱渐见,故其妖变致天雨粟、鬼夜哭也。夫言天雨粟、鬼夜哭,实也。言其应仓颉作书,虚也。夫河出图,洛出《书》,圣帝明王之瑞应也。图书文章,与仓颉所作字画何以异?天地为图书,仓颉作文字,业与天地同,指与鬼神合,何非何恶而致雨粟鬼哭之怪?使天地鬼神恶人有书,则其出图书,非也;天不恶人有书,作书何非而致此怪?或时仓颉适作书,天适雨粟,鬼偶夜哭,而雨粟、鬼神哭自有所为。世见应书而至,则谓作书生乱败之象,应事而动也。``天雨谷'',论者谓之从天而下,〔应〕变而生。如以云雨论之,雨谷之变,不足怪也。何以验之?夫云〔雨〕出於丘山,降散则为雨矣。人见其从上而坠,则谓之天雨水也。夏日则雨水,冬日天寒则雨凝而为雪,皆由云气发於丘山,不从天上降集於地,明矣。夫谷之雨,犹复云〔布〕之亦从地起,因与疾风俱飘,参於天,集於地。人见其从天落也,则谓之天雨谷。建武三十一年中,陈留雨谷,谷下蔽地。案视谷形,若茨而黑,有似於稗实也。此或时夷狄之地,生出此谷。夷狄不粒食,此谷生於草野之中,成熟垂委於地,遭疾风暴起,吹扬与之俱飞,风衰谷集,坠於中国。中国见之,谓之雨谷。何以效之?野火燔山泽,山泽之中,草木皆烧,其叶为灰,疾风暴起,吹扬之,参天而飞,风衰叶下,集於道路。夫``天雨谷''者,草木叶烧飞而集之类也。而世以为雨谷,作传书者以〔为〕变怪。天主施气,地主产物。有叶、实可啄食者,皆地所生,非天所为也。今谷非气所生,须土以成。虽云怪变,怪变因类。生地之物,更从天集,生天之物,可从地出乎?地之有万物,犹天之有列星也。星不更生於地,谷何独生於天乎?传书又言:伯益作井,龙登玄云,神栖昆仑。言龙井有害,故龙神为变也。夫言龙登玄云,实也。言神栖昆仑,又言为作井之故,龙登神去,虚也。夫作井而饮,耕田而食,同一实也。伯益作井,致有变动。始为耕耘者,何故无变?神农之桡木为耒,教民耕耨,民始食谷,谷始播种。耕土以为田,凿地以为井。井出水以救渴,田出谷以拯饥,天地鬼神所欲为也,龙何故登玄云?神何故栖昆仑?夫龙之登玄云,古今有之,非始益作井而乃登也。方今盛夏,雷雨时至,龙多登云。云龙相应,龙乘云雨而行,物类相致,非有为也。尧时,五十之民,击壤於涂。观者曰:``大哉,尧之德也!''击壤者曰:``吾日出而作,日入而息,凿井而饮,耕田而食。尧何等力?''尧时已有井矣。唐、虞之时,豢龙、御龙,龙常在朝。夏末政衰,龙乃隐伏。非益凿井,龙登云也。所谓神者,何神也?百神皆是。百神何故恶人为井?使神与人同,则亦宜有饮之欲。有饮之欲,憎井而去,非其实也。夫益殆之凿井,龙不为凿井登云,神不栖於昆仑,传书意妄,造生之也。

传书言:梁山崩,壅河三日不流,晋君忧之。晋伯宗以辇者之言,令景公素缟而哭之,河水为之流通。此虚言也。夫山崩壅河,犹人之有痈肿,血脉不通也。治痈肿者,可复以素服哭泣之声治乎?尧之时,洪水滔天,怀山襄陵。帝尧吁嗟,博求贤者。水变甚於河壅,尧忧深於景公,不闻以素缟哭泣之声能厌胜之。尧无贤人若辇者之术乎?将洪水变大,不可以声服除也?如素缟而哭,悔过自责也,尧、禹之治水以力役,不自责。梁山,尧时山也;所壅之河,尧时河也。山崩河壅,天雨水踊,二者之变无以殊也。尧、禹治洪水以力役,辇者治壅河用自责。变同而治异,人钧而应殊,殆非贤圣变复之实也。凡变复之道,所以能相感动者,以物类也。有寒则复之以温,温复解之以寒。故以龙致雨,以刑逐〔景〕,皆缘五行之气用相感胜之。山崩壅河,素缟哭之,於道何意乎?此或时何壅之时,山初崩,土积聚,水未盛。三日之後,水盛土散,稍坏沮矣。坏沮水流,竟注东去。遭伯宗得辇者之言,因素缟而哭,哭之因流,流时谓之河变,起此而复,其实非也。何以验之?使山恆自崩乎,素缟哭无益也。使其天变应之,宜改政治。素缟而哭,何政所改而天变复乎?

传书言:曾子之孝,与母同气。曾子出薪於野,有客至而欲去,曾母曰:``愿留,参方到。''即以右手扼其左臂。曾子左臂立痛,即驰至问母:``臂何故痛?''母曰:``今者客来欲去,吾扼臂以呼汝耳。''盖以至孝,与父母同气,体有疾病,精神辄感。曰:此虚也。夫孝悌之至,通於神明,乃谓德化至天地。俗人缘此而说,言孝悌之至,精气相动。如曾母臂痛,曾子臂亦辄痛,曾母病,曾子亦病〔乎〕?曾母死,曾子辄死乎?考事,曾母先死,曾子不死矣。此精气能小相动,不能大相感也。世称申喜夜闻其母歌,心动,开关问歌者为谁,果其母。盖闻母声,声音相感,心悲意动,开关而问,盖其实也。今曾母在家,曾子在野,不闻号呼之声,母小扼臂,安能动子?疑世人颂成,闻曾子之孝天下少双,则为空生母扼臂之说也。

世称:南阳卓公为缑氏令,蝗不入界。盖以贤明至诚,灾虫不入其县也。此又虚也。夫贤明至诚之化,通於同类,能相知心,然後慕服。蝗虫,闽虻之类也,何知何见而能知卓公之化?使贤者处深野之中,闽虻能不入其舍乎?闽虻不能避贤者之舍,蝗虫何能不入卓公之县?如谓蝗虫变与闽虻异,夫寒温亦灾变也,使一郡皆寒,贤者长一县,一县之界能独温乎?夫寒温不能避贤者之县,蝗虫何能不入卓公之界?夫如是,蝗虫适不入界,卓公贤名称於世,世则谓之能却蝗虫矣。何以验之?夫蝗之集於野,非能普博尽蔽地也,往往积聚多少有处。非所积之地,则盗跖所居;所少之野,则伯夷所处也。集过有多少,不能尽蔽覆也。夫集地有多少,则其过县有留去矣。多少不可以验善恶;有无安可以明贤不肖也?盖时蝗自过,不谓贤人界不入明矣。

\hypertarget{header-n209}{%
\subsection{卷六}\label{header-n209}}

\hypertarget{header-n210}{%
\subsubsection{福虚篇}\label{header-n210}}

世论行善者福至,为恶者祸来。福祸之应,皆天也,人为之,天应之。阳恩,人君赏其行;阴惠,天地报其德。无贵贱贤愚,莫谓不然。徒见行事有其文传,又见善人时遇福,故遂信之,谓之实然。斯言或时贤圣欲劝人为善,著必然之语,以明德报;或福时适遇者以为然。如实论之,安得福佑乎?

禁惠王食寒菹而得蛭,因遂吞之,腹有疾而不能食。令尹问:``王安得此疾也?''王曰:``我食寒菹而得蛭,念谴之而不行其罪乎?是废法而威不立也,非所以使国人闻之也;谴而行诛乎?则庖厨监食者法皆当死,心又不忍也。吾恐左右见之也,因遂吞之。''令尹避席再拜而贺曰:``臣闻天道无亲,唯德是辅。王有仁德,天之所奉也,病不为伤。''是夕也,惠王之後而蛭出,及久患心腹之积皆愈。故天之亲德也,可谓不察乎!曰:此虚言也。案惠王之吞蛭,不肖之主也。有不肖之行,天不佑也。何则?惠王不忍谴蛭,恐庖厨监食法皆诛也。一国之君,专擅赏罚;而赦,人君所为也。惠王通谴菹中何故有蛭,庖厨监食皆当伏法。然能终不以饮食行诛於人,赦而不罪,惠莫大焉。庖厨罪觉而不诛,自新而改後。惠王赦细而活微,身安不病。今则不然,强食害己之物,使监食之臣不闻其过,失御下之威,无御非之心,不肖一也。使庖厨监食失甘苦之和,若尘土落於菹中,大如虮虱,非意所能览,非目所能见,原心定罪,不明其过,可谓惠矣。今蛭广有分数,长有寸度,在寒菹中,眇目之人犹将见之,臣不畏敬,择濯不谨,罪过至重。惠王不谴,不肖二也。菹中不当有蛭,不食投地;如恐左右之见,怀屏隐匿之处,足以使蛭不见,何必食之?如不可食之物,误在菹中,可复隐匿而强食之,不肖三也。有不肖之行,而天佑之,是天报佑不肖人也。不忍谴蛭,世谓之贤。贤者操行,多若吞蛭之类。吞蛭天除其病,是则贤者常无病也。贤者德薄,未足以言。圣人纯道,操行少非,为推不忍之行,以容人之过。必众多矣。然而武王不豫,孔子疾病,天之佑人,何不实也?或时惠王吞蛭,蛭偶自出。食生物者无有不死,腹中热也。初吞时蛭〕未死,而腹中热,蛭动作,故腹中痛。须臾,蛭死腹中,痛亦止。蛭之性食血,惠王心腹之积,殆积血也。故食血之虫死,而积血之病愈。犹狸之性食鼠,人有鼠病,吞狸自愈。物类相胜,方药相使也。食蛭虫而病愈,安得怪乎?食生物无不死,死无不出,之後蛭出,安得佑乎?令尹见惠王有不忍之德,知蛭入腹中必当死出,因再拜,病贺不为伤。著已知来之德,以喜惠王之心,是与子韦之言星徙、太卜之言地动无以异也。

宋人有好善行者,三世不改,家无故黑牛生白犊。以问孔子,孔子曰:``此吉祥也,以享鬼神。''即以犊祭。一年,其父无故而盲。牛又生白犊。其父又使其子问孔子,孔子曰:``吉祥也,以享鬼神。''复以犊祭。一年,其子无故而盲。其後楚攻宋,围其城。当此之时,易子而食之,析骸而炊之。此独以父子俱盲之故,得毋乘城。军罢围解,父子俱视。此修善积行神报之效也。曰:此虚言也。夫宋人父子修善如此,神报之,何必使之先盲後视哉?不盲常视,不能护乎?此神不能护不盲之人,则亦不能以盲护人矣。使宋、楚之君合战顿兵,流血僵尸,战夫禽获,死亡不还。以盲之故,得脱不行,可谓神报之矣。今宋、楚相攻,两军未合,华元、子反结言而退,二军之众,并全而归,兵矢之刃无顿用者。虽有乘城之役,无死亡之患。为善人报者,为乘城之间乎?使时不盲,亦犹不死。盲与不盲,俱得脱免,神使之盲,何益於善!当宋国乏粮之时也,盲人之家,岂独富哉?俱与乘城之家易子
骸,反以穷厄独盲无见,则神报佑人,失善恶之实也。宋人父子前偶自以风寒发盲,围解之後,盲偶自愈。世见父子修善,又用二白犊祭,宋、楚相攻独不乘城,围解之後父子皆视,则谓修善之报、获鬼神之佑矣。

楚相孙叔敖为兒之时,见两头蛇,杀而埋之,归,对其母泣。母问其故,对曰:``我闻见两头蛇死。向者,出见两头蛇,恐去母死,是以泣也。''其母日:
``今蛇何在?''对日:``我恐後人见之,即杀而埋之。''其母日:``吾闻有阴德者,天必报之。汝必不死,天必报汝。''叔敖竟不死,遂为楚相。埋一蛇,获二佑,天报善明矣。曰:此虚言矣。夫见两头蛇辄死者,俗言也;有阴德天报之福者,俗议也。叔敖信俗言而埋蛇,其母信俗议而必报,是谓死生无命,在一蛇之死。齐孟尝君田文以五月五日生,其父田婴让其母曰:``何故举之?''曰:``君所以不举五月子,何也?''婴曰:``五月子长与户同,杀其父母。''曰:``人命在天乎?在户乎?如在天,君何忧也;如在户,则宜高其户耳,谁而及之者!''
後文长与一户同,而婴不死。是则五月举子之忌,无效验也。夫恶见两头蛇,犹五月举子也。五月举子,其父不死,则知见两头蛇者,无殃祸也。由此言之,见两头蛇自不死,非埋之故也。埋一蛇,获二福,如埋十蛇,得几佑乎?埋蛇恶人复见,叔敖贤也。贤者之行,岂徒埋蛇一事哉?前埋蛇之时,多所行矣。禀天善性,动有贤行。贤行之人,宜见吉物,无为乃见杀人之蛇。岂叔敖未见蛇之时有恶,天欲杀之,见其埋蛇,除其过,天活之哉?石生而坚,兰生而香。如谓叔敖之贤在埋蛇之时,非生而禀之也。

儒家之徒董无心,墨家之役缠子,相见讲道。缠子称墨家佑鬼神,是引秦穆公有明德,上帝赐之十九年,缠子难以尧、舜不赐年,桀、纣不夭死。尧、舜、桀、纣犹为尚远,且近难以秦穆公、晋文公。夫谥者,行之迹也,迹生时行,以为死谥。穆者误乱之名,文者德惠之表。有误乱之行,天赐之年;有德惠之操,天夺其命乎?案穆公之霸,不过晋文;晋文之谥,美於穆公。天不加晋文以命,独赐穆公以年,是天报误乱,与``穆公''同也。天下善人寡,恶人众。善人顺道,恶人违天。然夫恶人之命不短,善人之年不长。天不命善人常享一百载之寿,恶人为殇子恶死,何哉?

\hypertarget{header-n218}{%
\subsubsection{祸虚篇}\label{header-n218}}

世谓受福佑者,既以为行善所致;又谓被祸害者,为恶所得。以为有沉恶伏过,天地罚之,鬼神报之。天地所罚,小大犹发;鬼神所报,远近犹至。

传曰:``子夏丧其子而丧其明,曾子吊之,哭。子夏曰:`天乎!予之无罪也!'曾子怒曰:`商,汝何无罪也?吾与汝事夫子於洙、泗之间,退而老於西河之上,使西河之民疑汝於夫子,尔罪一也;丧尔亲,使民未有异闻,尔罪二也;丧尔子,丧尔明,尔罪三也。而曰,汝何无罪欤?'子夏投其杖而拜,曰:`吾过矣,吾过矣!吾离群而索居,亦以久矣!'''夫子夏丧其明,曾子责以罪,子夏投杖拜曾子之言,盖以天实罚过,故目失其明,已实有之,故拜受其过。始闻暂见,皆以为然;熟考论之,虚妄言也。夫失明犹失听也。失明则盲,失听则聋。病聋不谓之有过,失明谓之有罪,惑也。盖耳目之病,犹心腹之有病也。耳目失明听,谓之有罪,心腹有病,可谓有过乎?伯牛有疾,孔子自牖执其手,曰:``
亡之,命矣夫!斯人也而有斯疾也!''原孔子言,谓伯牛不幸,故伤之也。如伯牛以过致疾,天报以恶与子夏同,孔子宜陈其过,若曾子谓子夏之状。今乃言命,命非过也。且天之罚人,犹人君罪下也。所罚服罪,人君赦之。子夏服过,拜以自悔,天德至明,宜愈其盲。如非天罪,子夏失明,亦换三罪。且丧明之病,孰与被厉之病?丧明有三罪,被厉有十过乎?颜渊早夭,子路菹醢。早死、菹醢,极祸也。以丧明言之,颜渊、子路有百罪也。由此言之,曾子之言误矣。然子夏之丧明,丧其子也。子者人情所通,亲者人所力报也。丧亲民无闻,丧子失其明,此恩损於亲而爱增於子也。增则哭泣无数,数哭中风,目失明矣。曾子因俗之议,以著子夏三罪。子夏亦缘俗议,因以失明,故拜受其过。曾子、子夏未离於俗,故孔子门叙行,未在上第也。

秦襄王赐白起剑,白起伏剑将自刎,曰:``我有何罪於天乎?''良久,曰:
``我固当死。长平之战,赵卒降者数十万,我诈而尽坑之,是足以死。''遂自杀。白起知己前罪,服更後罚也。夫白起知己所以罪,不知赵卒所以坑。如天审罚有过之人,赵降卒何辜於天?如用兵妄伤杀,则四十万众必有不亡,不亡之人,何故以其善行无罪而竟坑之?卒不得以善蒙天之佑,白起何故独以其罪伏天之诛?由此言之,白起之言过矣。

秦二世使使者诏杀蒙恬,蒙恬喟然叹曰:``我何过於天,无罪而死!''良久,徐曰:``恬罪故当死矣。夫起临洮属之辽东,城径万里,此其中不能毋绝地脉。此乃恬之罪也。''即吞药自杀。太史公非之曰:``夫秦初灭诸侯,天下心未定,夷伤未瘳,而恬为名将,不以此时强谏,救百姓之急,养老矜孤,修众庶之和,阿意兴功,此其〔兄〕弟〔遇〕诛,不亦宜乎!何与乃罪地脉也?''夫蒙恬之言既非,而太史公非之亦未是。何则?蒙恬绝脉,罪至当死。地养万物,何过於人,而蒙恬绝其脉?知己有绝地脉之罪,不知地脉所以绝之过。自非如此,与不自非何以异?太史公为非恬之为名将,不能以强谏,故致此祸。夫当谏不谏,故致受死亡之戮。身任李陵,坐下蚕室,如太史公之言,所任非其人,故残身之戮,天命而至也。非蒙恬以不强谏,故致此祸,则己下蚕室,有非者矣。己无非,则其非蒙恬,非也。作伯夷之传,〔列〕善恶之行云:``七十子之徒,仲尼独荐颜渊好学。然回也屡空,糟糠不厌,卒夭死。天之报施善人如何哉!盗跖日杀不辜,肝人之肉,暴戾恣睢,聚党数千,横行天下,竟以寿终。是独遵何哉?''若此言之,颜回不当早夭,盗跖不当全活也。不怪颜渊不当夭,而独谓蒙恬当死,过矣。汉将李广与望气王朔燕语曰:``自汉击匈奴,而广未常不在其中,而诸校尉以下,才能不及中,然以胡军攻取侯者数十人。而广不为後人,然终无尺〔寸〕之功,以得封邑者,何也?岂吾相不当侯?且固命也?''朔曰:``将军自念,岂常有恨者乎?''广曰:``吾为陇西太守,羌常反,吾诱而降之八百余人;吾诈而同日杀之。至今恨之,独此矣。''朔曰:``祸莫大於杀已降,此乃将军所以不得侯者也。
''李广然之,闻者信之。夫不侯犹不王者也。不侯何恨,不王何负乎?孔子不王,论者不谓之有负;李广不侯,王朔谓之有恨。然则王朔之言,失论之实矣。论者以为人之封侯,自有天命。天命之符,见於骨体。大将军卫青在建章宫时,钳徒相之,曰:``贵至封侯。''後竟以功封万户侯。卫青未有功,而钳徒见其当封之证。由此言之,封侯有命,非人操行所能得也。钳徒之言实而有效,王朔之言虚而无验也。多横恣而不罹祸,顺道而违福,王朔之说,白起自非、蒙恬自咎之类也。仓卒之世,以财利相劫杀者众。同车共船,千里为商,至阔迥之地,杀其人而并取其财,尸捐不收,骨暴不葬,在水为鱼鳖之食,在土为蝼蚁之粮;惰窳之人,不力农勉商,以积谷货,遭岁饥馑,腹饿不饱,椎人若畜,割而食之,无君子小人,并为鱼肉:人所不能知,吏所不能觉。千人以上,万人以下,计一聚之中,生者百一,死者十九。可谓无道至痛甚矣,皆得阳达富厚安乐。天不责其无仁义之心,道相并杀;非其无力作而仓卒以人为食,加以渥祸,使之夭命,章其阴罪,明示世人,使知不可为非之验,何哉?王朔之言,未必审然。

传书:``李斯妒同才,幽杀韩非於秦,後被车裂之罪,商鞅欺旧交,擒魏公子卬,後受诛死之祸。''彼欲言其贼贤欺交,故受患祸之报也。夫韩非何过而为李斯所幽?公子卬何罪而为商鞅所擒?车裂诛死,贼贤欺交,幽死见擒,何以致之?如韩非、公子卬有恶,天使李斯、商鞅报之,则李斯、商鞅为天奉诛,宜蒙其赏,不当受其祸。如韩非、公子卬无恶,非天所罚,李斯、商鞅不得幽擒。论者说曰:``韩非、公子卬有阴恶伏罪,人不闻见,天独知之,故受戮殃。''夫诸有罪之人,非贼贤则逆道。如贼贤,则被所贼者何负?如逆道,则被所逆之道何非?

凡人穷达祸福之至,大之则命,小之则时。太公穷贱,遭周文而得封。甯戚隐厄,逢齐桓而见官。非穷贱隐厄有非,而得封见官有是也。穷达有时,遭遇有命也。太公、甯戚,贤者也,尚可谓有非。圣人,纯道者也。虞舜为父弟所害,几死再三;有遇唐尧,尧禅舜。立为帝。尝见害,未有非;立为帝,未有是。前时未到,後则命时至也。案古人君臣困穷,後得达通,未必初有恶天祸其前,卒有善神佑其後也。一身之行,一行之操,结发终死,前後无异。然一成一败,一进一退,一穷一通,一全一坏,遭遇适然,命时当也。

\hypertarget{header-n227}{%
\subsubsection{龙虚篇}\label{header-n227}}

盛夏之时,雷电击折树木,发坏室屋,俗谓天取龙,谓龙藏於树木之中,匿於屋室之间也,雷电击折树木,发坏屋室,则龙见於外。龙见,雷取以升天。世无愚智贤不肖,皆谓之然。如考实之,虚妄言也。

夫天之取龙何意邪?如以龙神为天使,犹贤臣为君使也,反报有时,无为取也。如以龙遁逃不还,非神之行,天亦无用为也。如龙之性当在天,在天上者固当生子,无为复在地。如龙有升降,降龙生子於地,子长大,天取之,则世名雷电为天怒,取龙之子,无为怒也。且龙之所居,常在水泽之中,不在木中屋间。何以知之?叔向之母曰:``深山大泽,实生龙蛇。''传曰:``山致其高,云雨起焉。水致其深,蛟龙生焉。''传又言:``禹渡於江,黄龙负船。''``荆次非渡淮,两龙绕舟。''``东海之上,有A丘欣,勇而有力,出过神渊,使御者饮马,马饮因没。欣怒,拔剑入渊追马,见两蛟方食其马,手剑击杀两蛟。''由是言之,蛟与龙常在渊水之中,不在木中屋间明矣。在渊水之中,则鱼鳖之类。鱼鳖之类,何为上天?天之取龙,何用为哉?如以天神乘龙而行,神恍惚无形,出入无间,无为乘龙也。如仙人骑龙,天为仙者取龙,则仙人含天精气,形轻飞腾,若鸿鹄之状,无为骑龙也。世称黄帝骑龙升天,此言盖虚,犹今谓天取龙也。

且世谓龙升天者,必谓神龙。不神,不升天;升天,神之效也。天地之性,人为贵,则龙贱矣。贵者不神,贱者反神乎?如龙之性有神与不神,神者升天,不神者不能。龟蛇亦有神与不神,神龟神蛇,复升天乎?且龙禀何气而独神?天有仓龙、白虎、硃鸟、玄武之象也,地亦有龙、虎、鸟、龟之物。四星之精,降生四兽。虎鸟与龟不神,龙何故独神也?人为倮虫之长,龙为鳞虫之长。俱为物长,谓龙升天,人复升天乎?龙与人同,独谓能升天者,谓龙神也。世或谓圣人神而先知,犹谓神龙能升天也。因谓圣人先知之明,论龙之才,谓龙升天,故其宜也。

天地之间,恍惚无形,寒暑风雨之气乃为神。今龙有形,有形则行,行则食,食则物之性也。天地之性,有形体之类,能行食之物,不得为神。何以言之,龙有体也。传曰:``鳞虫三百,龙为之长。''龙为鳞虫之长,安得无体?何以言之,孔子曰:``龙食於清,游於清。龟食於清;游於浊;鱼食於浊,游於浊。丘上不及龙,下不为鱼,中止其龟与!''

《山海经》言:四海之外,有乘龙蛇之人。世俗画龙之象,马首蛇尾。由此言之,马、蛇之类也。慎子曰:``蜚龙乘云,腾蛇游雾,云罢雨霁,与蚓蚁同矣。
''韩子曰:``龙之为虫也,鸣可狎而骑也。然喉下有逆鳞尺余,人或婴之,必杀人矣。''比之为蚓蚁,又言虫可狎而骑,蛇、马之类明矣。传曰:``纣作象箸而箕子泣。''泣之者,痛其极也。夫有象箸,必有玉杯。玉杯所盈,象箸所挟,则必龙肝豹胎。夫龙肝可食,其龙难得。难得则愁下,愁下则祸生,故从而痛之。如龙神,其身不可得杀,其肝何可得食?禽兽肝胎非一,称龙肝豹胎者,人得食而知其味美也。春秋之时,龙见於绛郊。魏献子问於蔡墨曰:``吾闻之,虫莫智於龙,以其不生得也。谓之智,信乎?''对曰:``人实不知,非龙实智。古者畜龙,故国有豢龙氏,有御龙氏。''献子曰:``是二者,吾亦闻之,而不知其故。是何谓也?''对曰:``昔有飂叔〔安〕有裔子曰董父,实甚好龙,能求其嗜欲以饮食之,龙多归之。乃扰畜龙,以服事舜,而锡之姓曰董,氏曰豢龙,封诸鬲川,鬲夷氏是其後也。故帝舜氏世有畜龙。及有夏,孔甲扰於帝,帝赐之乘龙,河、汉各二,各有雌雄,孔甲不能食也,而未获豢龙氏。有陶唐氏既衰,其後有刘累学扰龙於豢龙氏,以事孔甲,能饮食龙。夏后嘉之,赐氏曰御龙,以更豕韦之後。龙一雌死,潜醢以食夏后.夏后〔亨〕之。既而使求,
惧而不得,迁於鲁县,范氏其後也。''献子曰:``今何故无之?''对曰:``夫物有其官,官修其方,朝夕思之。一日失职,则死及之,失官不食。官宿其业,其物乃至。若泯弃之,物乃低伏,郁湮不育。''由此言之,龙可畜又可食也。可食之物,不能神矣。世无其官,又无董父、後刘之人,故潜藏伏匿,出见希疏;出又乘云,与人殊路,人谓之神。如存其官而有其人,则龙,牛之类也,何神之有?以《山海经》言之,以慎子、韩子证之,以俗世之画验之,以箕子之泣订之,以蔡墨之对论之,知龙不能神,不能升天,天不以雷电取龙,明矣。世俗言龙神而升天者,妄矣。

世俗之言,亦有缘也。短书言:``龙无尺木,无以升天。''又曰``升天'',又言``尺木'',谓龙从木中升天也。彼短书之家,世俗之人也。见雷电发时,龙随而起,当雷电〔击〕树木之时,龙适与雷电俱在树木之侧,雷电去,龙随而上,故谓从树木之中升天也。实者雷龙同类,感气相致,故《易》曰:``云从龙,风从虎。''又言:``虎啸谷风至,龙兴景云起。''龙与云相招,虎与风相致,故董仲舒雩祭之法,设土龙以为感也。夫盛夏太阳用事,云雨干之。太阳火也,云雨水也,〔水〕火激薄则鸣而为雷。龙闻雷声则起,起而云至,云至而龙乘之。云雨感龙,龙亦起云而升天。天极雷高,云消复降。人见其乘云则谓``升天'',见天为雷电则为``天取龙''。世儒读《易》文,见传言,皆知龙者云之类。拘俗人之议,不能通其说;又见短书为证,故遂谓``天取龙''。

天不取龙,龙不升天。当丘欣之杀两蛟也,手把其尾,拽而出之至渊之外,雷电击之。蛟则龙之类也。蛟龙见而云雨至,云雨至则雷电击。如以天实取龙,龙为天用,何以死蛟〔不〕为取之?且鱼在水中,亦随云雨,蜚而乘云雨非升天也。龙,鱼之类也,其乘雷电犹鱼之飞也。鱼随云雨,不谓之神,龙乘雷电独谓之神。世俗之言,失其实也。物在世间,各有所乘。水蛇乘雾,龙乘云,鸟乘风。见龙乘云,独谓之神,失龙之实,诬龙之能也。

然则龙之所以为神者,以能屈伸其体,存亡其形。屈伸其体,存亡其形,未足以为神也。豫让吞炭,漆身为厉,人不识其形。子贡灭须为妇人,人不知其状;龙变体自匿,人亦不能觉,变化藏匿者巧也。物性亦有自然,

狌狌知往,乾鹊知来,鹦鹉能言,三怪比龙,性变化也。如以巧为神,豫让、子贡神也。孔子曰:``游者可为网,飞者可为矰。至於龙也,吾不知其乘风云上升。今日见老子,其犹龙乎!''夫龙乘云而上,云消而下。物类可察,上下可知;而云孔子不知。以孔子之圣,尚不知龙,况俗人智浅,好奇之性,无实可之心,谓之龙神而升天,不足怪也。

\hypertarget{header-n239}{%
\subsubsection{雷虚篇}\label{header-n239}}

盛夏之时,雷电迅疾,击折树木,坏败室屋,时犯杀人。世俗以为``击折树木、坏败室屋''者,天取龙;其``犯杀人''也,谓之〔有〕阴过,饮食人以不洁净,天怒,击而杀之。隆隆之声,天怒之音,若人之呴吁矣。世无愚智,莫谓不然。推人道以论之,虚妄之言也。

夫雷之发动,一气一声也,折木坏屋亦犯杀人,犯杀人时亦折木坏屋。独谓折木坏屋者,天取龙;犯杀人,罚阴过,与取龙吉凶不同,并时共声,非道也。论者以为``隆隆''者,天怒呴吁之声也。此便於罚过,不宜於取龙。罚过,天怒可也;取龙,龙何过而怒之?如龙神,天取之,不宜怒。如龙有过,与人同罪,杀而已,何为取也?杀人,怒可也。取龙,龙何过而怒之?杀人不取;杀龙取之。人龙之罪何别?而其杀之何异?然则取龙之说既不可听,罚过之言复不可从。

何以效之?案雷之声迅疾之时,人仆死於地,隆隆之声临人首上,故得杀人。审隆隆者天怒乎?怒用口之怒气杀人也。口之怒气,安能杀人?人为雷所杀,询其身体,若燔灼之状也。如天用口怒,口怒生火乎?且口着乎体,口之动与体俱。当击折之时,声着於地;其衰也,声着於天。夫如是,声着地之时,口至地,体亦宜然。当雷迅疾之时,仰视天,不见天之下,不见天之下,则夫隆隆之声者,非天怒也。天之怒与人无异。人怒,身近人则声疾,远人则声微。今天声近,其体远,非怒之实也。且雷声迅疾之时,声东西或南北,如天怒体动,口东西南北,仰视天亦宜东西南北。或曰:``天已东西南北矣,云雨冥晦,人不能见耳。''夫千里不同风,百里不共雷。《易》曰:``震惊百里。''雷电之地,〔云〕雨晦冥,百里之外无雨之处,宜见天之东西南北也。口着於天,天宜随口,口一移普天皆移,非独雷雨之地,天随口动也。且所谓怒者,谁也?天神邪?苍苍之天也?如谓天神,神怒无声;如谓苍苍之天,天者体不怒,怒用口。且天地相与,夫妇也,其即民父母也。子有过,父怒,笞之致死,而母不哭乎?今天怒杀人,地宜哭之。独闻天之怒,不闻地之哭。如地不能哭,则天亦不能怒。且有怒则有喜。人有阴过,亦有阴善。有阴过,天怒杀之;如有阴善,天亦宜以善赏之。隆隆之声谓天之怒,如天之喜,亦哂然而笑。人有喜怒,故谓天喜怒、推人以知天,知天本於人。如人不怒,则亦无缘谓天怒也。缘人以知天,宜尽人之性。人性怒则呴吁,喜则歌笑。比闻天之怒,希闻天之喜;比见天之罚,希见天之赏。岂天怒不喜,贪於罚,希於赏哉?何怒罚有效,喜赏无验也?

且雷之击也,``折木坏屋'',``时犯杀人'',以为天怒。时或徒雷,无所折败,亦不杀人,天空怒乎?人君不空喜怒,喜怒必有赏罚。无所罚而空怒,是天妄也。妄则失威,非天行也。政事之家,以寒温之气,为喜怒之候,人君喜即天温,〔怒〕则天寒。雷电之日,天必寒也。高祖之先刘媪曾息大泽之陂,梦与神遇,此时雷电晦冥。天方施气,宜喜之时也,何怒而雷?如用击折者为怒,不击折者为喜,则夫隆隆之声,不宜同音。人怒喜异声,天怒喜同音,与人乖异,则人何缘谓之天怒?且``饮食人以不洁净'',小过也。以至尊之身,亲罚小过,非尊者之宜也。尊不亲罚过,故王不亲诛罪。天尊於王,亲罚小过,是天德劣於王也。且天之用心,犹人之用意。人君罪恶,初闻之时,怒以非之;及其诛之,哀以怜之。故《论语》曰:``如得其情,则哀怜而勿喜。''纣至恶也,武王将诛,哀而怜之。故《尚书》曰:``予惟率夷怜尔。''人君诛恶,怜而杀之;天之罚过,怒而击之。是天少恩而人多惠也。说雨者以为天施气。天施气,气渥为雨,故雨润万物,名曰澍。人不喜,不施恩。天不说,不降雨。谓雷,天怒;雨者,天喜也。雷起常与雨俱,如论之言,天怒且喜也。人君赏罚不同日,天之怒喜不殊时,天人相违,赏罚乖也。且怒喜具形,乱也。恶人为乱,怒罚其过;罚之以乱,非天行也。冬雷人谓之阳气泄,春雷谓之阳气发。夏雷不谓阳气盛,谓之天怒,竟虚言也。

人在天地之间,物也。物,亦物也。物之饮食,天不能知。人之饮食,天独知之。万物於天,皆子也;父母於子,恩德一也。岂为贵贤加意,贱愚不察乎?何其察人之明,省物之暗也!犬豕食,人腐臭食之,天不杀也。如以人贵而独禁之,则鼠洿人饮食,人不知,误而食之,天不杀也。如天能原鼠,则亦能原人,人误以不洁净饮食人,人不知而食之耳,岂故举腐臭以予之哉?如故予之,人亦不肯食。吕后断戚夫人手,去其眼,置於厕中,以为人豕。呼人示之,人皆伤心;惠帝见之,疾卧不起。吕后故为,天不罚也。人误不知,天辄杀之,不能原误,失而责故,天治悖也。

夫人食不净之物,口不知有其洿也;如食,已知之,名曰肠洿。戚夫人入厕,身体辱之,与洿何以别?肠之与体何以异?为肠不为体,伤洿不病辱,非天意也。且人闻人食不清之物,心平如故,观戚夫人者,莫不伤心。人伤,天意悲矣。夫悲戚夫人则怨吕后,案吕后之崩,未必遇雷也。道士刘春荧惑楚王英,使食不清。春死,未必遇雷也。建初四年夏六月,雷击杀会稽〔鄞〕专日食羊五头皆死。夫羊何阴过,而雷杀之?舟人洿溪上流,人饮下流,舟人不雷死。

天神之处天,犹王者之居也。王者居重关之内,则天之神宜在隐匿之中。王者居宫室之内,则天亦有太微、紫宫、轩辕、文昌之坐。王者与人相远,不知人之阴恶。天神在四宫之内,何能见人暗过?王者闻人进,以人知。天知人恶,亦宜因鬼。使天问过於鬼神,则其诛之,宜使鬼神。如使鬼神,则天怒,鬼神也,非天也。

且王断刑以秋,天之杀用夏,此王者用刑违天时。奉天而行,其诛杀也,宜法象上天。天杀用夏,王诛以秋,天人相违,非奉天之义也。或论曰:``饮食〔人〕不洁净,天之大恶也。杀大恶,不须时。''王者大恶,谋反大逆无道也。天之大恶,饮食人不洁清。天〔人〕所恶,小大不均等也。如小大同,王者宜法天,制饮食人不洁清之法为死刑也。圣王有天下,制刑不备此法,圣王阙略,有遗失也?或论曰:``鬼神治阴,王者治阳。阴过暗昧,人不能觉,故使鬼神主之。''
曰:``阴过非一也,何不尽杀?案一过,非治阴之义也。天怒不旋日,人怨不旋踵。人有阴过,或时有用冬,未必专用夏也。以冬过误,不辄击杀,远至於夏,非不旋日之意也。

图画之工,图雷之状,累累如连鼓之形;又图一人,若力士之容,谓之雷公,使之左手引连鼓,右手推椎,若击之状。其意以为雷声隆隆者,连鼓相扣击之〔音〕也;其魄然若敝裂者,椎所击之声也;其杀人也,引连鼓相椎,并击之矣。世又信之,莫谓不然。如复原之,虚妄之象也。夫雷,非声则气也。声与气,安可推引而为连鼓之形乎?如审可推引,则是物也。相扣而音鸣者,非鼓即钟也。夫隆隆之声,鼓与钟邪?如审是也,钟鼓不〔而〕空悬,须有笋虡,然後能安,然後能鸣。今钟鼓无所悬着,雷公之足,无所蹈履,安得而为雷?或曰:``如此固为神。如必有所悬着,足有所履,然後而为雷,是与人等也,何以为神?''曰:神者,恍惚无形,出入无门,上下无垠,故谓之神。今雷公有形,雷声有器,安得为神?如无形,不得为之图象;如有形,不得谓之神。谓之神龙升天,实事者谓之不然,以人时或见龙之形也。以其形见,故图画升龙之形也;以其可画,故有不神之实。

难曰:``人亦见鬼之形,鬼复神乎?''曰:人时见鬼,有见雷公者乎?鬼名曰神,其行蹈地,与人相似。雷公头不悬於天,足不蹈於地,安能为雷公?飞者皆有翼,物无翼而飞,谓仙人。画仙人之形,为之作翼。如雷公与仙人同,宜复着翼。使雷公不飞,图雷家言其飞,非也;使实飞,不为着翼,又非也。夫如是,图雷之家,画雷之状,皆虚妄也。且说雷之家,谓雷,天怒呴吁也;图雷之家,谓之雷公怒引连鼓也。审如说雷之家,则图雷之家非;审如图雷之家,则说雷之家误。二家相违也,并而是之,无是非之分。无是非之分,故无是非之实。无以定疑论,故虚妄之论胜也。

《礼》曰:``刻尊为雷之形,一出一入,一屈一伸,为相校轸则鸣。''校轸之状,郁律垒之类也,此象类之矣。气相校轸分裂,则隆隆之声,校轸之音也。魄然若\{敝衣\}裂者,气射之声也。气射中人,人则死矣。实说,雷者太阳之激气也。何以明之?正月阳动,故正月始雷。五月阳盛,故五月雷迅。秋冬阳衰,故秋冬雷潜。盛夏之时,太阳用事,阴气乘之。阴阳分〔争〕,则相校轸。校轸则激射。激射为毒,中人辄死,中木木折,中屋屋坏。人在木下屋间,偶中而死矣。何以验之?试以一斗水灌冶铸之火,气激\{敝衣\}裂,若雷之音矣。或近之,必灼人体。天地为炉大矣,阳气为火猛矣,云雨为水多矣,分争激射,安得不迅?中伤人身,安得不死?当冶工之消铁也,以士为形,燥则铁下,不则跃溢而射。射中人身,则皮肤灼剥。阳气之热,非直消铁之烈也;阴气激之,非直土泥之湿也;阳气中人,非直灼剥之痛也。

夫雷,火也。〔火〕气剡人,人不得无迹。如炙处状似文字,人见之,谓天记书其过,以示百姓。是复虚妄也。使人尽有过,天用雷杀人。杀人当彰其恶,以惩其後,明著其文字,不当暗昧。《图》出於河,《书》出於洛。河图、洛书,天地所为,人读知之。今雷死之书,亦天所为也,何故难知?如以〔殪〕人皮不可书,鲁惠公夫人仲子,宁武公女也,生而有文在掌,曰``为鲁夫人'',文明可知,故仲子归鲁。雷书不著,故难以惩後。夫如是,火剡之迹,非天所刻画也。或颇有而增其语,或无有而空生其言,虚妄之俗,好造怪奇。何以验之?雷者火也,以人中雷而死,即询其身,中头则须发烧燋,中身则皮肤灼焚,临其尸上闻火气,一验也。道术之家,以为雷烧石,色赤,投於井中,石燋井寒,激声大鸣,若雷之状,二验也。人伤於寒,寒气入腹,腹中素温,温寒分争,激气雷鸣,三验也。当雷之时,电光时见大,若火之耀,四验也。当雷之击,时或燔人室屋,及地草木,五验也。夫论雷之为火有五验,言雷为天怒无一效。然则雷为天怒,虚妄之言。

〔难〕曰:``《论语》云:`迅雷风烈必变。'《礼记》曰:`有疾风迅雷甚雨则必变,虽夜必兴,衣服、冠而坐。'惧天怒,畏罚及己也。如雷不为天怒,其击不为罚过,则君子何为为雷变动、朝服而正坐〔乎〕?''曰:天之与人犹父子,有父为之变,子安能忽?故天变,己亦宜变,顺天时,示己不违也。人闻犬声於外,莫不惊骇,竦身侧耳以审听之。况闻天变异常之声,轩盍迅疾之音乎?《论语》所指,《礼记》所谓,皆君子也。君子重慎,自知无过,如日月之蚀,无阴暗食人以不洁清之事,内省不惧,何畏於雷?审如不畏雷,则其变动不足以效天怒。何则?不为己也。如审畏雷,亦不足以效罚阴过。何则?雷之所击,多无过之人。君子恐偶遇之,故恐惧变动。夫如是,君子变动,不能明雷为天怒,而反著雷之妄击也。妄击不罚过,故人畏之。如审罚过,有过小人乃当惧耳,君子之人无为恐也。宋王问唐鞅曰:``寡人所杀戮者众矣,而群臣愈不畏,其故何也?''唐鞅曰:``王之所罪,尽不善者也。罚不善,善者胡为畏?王欲群臣之畏也,不若毋辨其善与不善而时罪之,斯群臣畏矣。''宋王行其言,群臣畏惧,宋国大恐〕。夫宋王妄刑,故宋国大恐。惧雷电妄击,故君子变动。君子变动,宋国大恐之类也。

\hypertarget{header-n256}{%
\subsection{卷七}\label{header-n256}}

\hypertarget{header-n257}{%
\subsubsection{道虚篇}\label{header-n257}}

儒书言:黄帝采首山铜,铸鼎於荆山下。鼎既成,有龙垂胡髯,下迎黄帝。黄帝上骑龙,群臣,後宫从上七十余人,龙乃上去。余小臣不得上,乃悉持龙髯。龙髯拔,堕黄帝之弓,百姓仰望黄帝既上天,乃抱其弓与龙胡髯吁号。故後世因其处曰``鼎湖'',其弓曰``乌号''。《太史公记》诔五帝,亦云:黄帝封禅已,仙云。群臣朝其衣冠。因葬埋之。

曰:此虚言也。实``黄帝''者何等也?号乎,谥乎?如谥,臣子所诔列也。诔生时所行为之谥。黄帝好道,遂以升天,臣子诔之,宜以仙升,不当以``黄''
谥。《谥法》曰:``静民则法曰黄。''``黄''者,安民之谥,非得道之称也。百王之谥,文则曰文,武则曰``武''。文武不失实,所以劝操行也。如黄帝之时质,未有谥乎?名之为黄帝,何世之人也?使黄帝之臣子,知君,使後世之人,迹其行。黄帝之世,号谥有无,虽疑未定,``黄''非升仙之称,明矣。

龙不升天,黄帝骑之,乃明黄帝不升天也。龙起云雨,因乘而行;云散雨止,降复入渊。如实黄帝骑龙,随溺於渊也。案黄帝葬於桥山,犹曰群臣葬其衣冠。审骑龙而升天,衣不离形;如封禅已,仙去。衣冠亦不宜遗。黄帝实仙不死而升天,臣子百姓所亲见也。见其升天,知其不死,必也。葬不死之衣冠,与实死者无以异,非臣子实事之心,别生於死之意也。

载太山之上者,七十有二君,皆劳情苦思,忧念王事,然後功成事立,致治太平。太平则天下和安,乃升太山而封禅焉。夫修道求仙,与忧职勤事不同。心思道则忘事,忧事则害性。世称尧若腊,舜若腒,心愁忧苦,形体赢癯。使黄帝致太平乎,则其形体宜如尧、舜。尧、舜不得道,黄帝升天,非其实也。使黄帝废事修道,则心意调和,形体肥劲,是与尧、舜异也,异则功不同矣。功不同,天下未太平而升封,又非实也。五帝三王皆有圣德之优者,黄帝〔亦〕在上焉。如圣人皆仙,仙者非独黄帝;如圣人不仙,黄帝何为独仙?世见黄帝好方术,方术仙者之业,则谓帝仙矣。又见鼎湖之名,则言黄帝采首山铜铸鼎,而龙垂胡髯迎黄帝矣。是与说会稽之山无以异也。夫山名曰``会稽'',即云夏禹巡狩,会计於此山上,故曰``会稽''。夫禹至会稽治水不巡狩,犹黄帝好方伎不升天也。无会计之事,犹无铸鼎龙垂胡髯之实也。里名胜母,可谓实有子胜其母乎?邑名朝歌,可谓民朝起者歌乎?

儒书言:淮南王学道,招会天下有道之人,倾一国之尊,下道术之士。是以道术之士,并会淮南,奇方异术,莫不争出。王遂得道,举家升天。畜产皆仙,犬吠於天上,鸡鸣於云中。此言仙药有余,犬鸡食之,并随王而升天也。好道学仙之人,皆谓之然。此虚言也。

夫人,物也,虽贵为王侯,性不异於物。物无不死,人安能仙?鸟有毛羽,能飞,不能升天。人无毛羽,何用飞升?使有毛羽,不过与鸟同;况其无有,升天如何?案能飞升之物,生有毛羽之兆;能驰走之物,生有蹄足之形。
驰走不能飞升,飞升不能驰走。禀性受气,形体殊别也。今人禀驰走之性,故生无毛羽之兆,长大至老,终无奇怪。好道学仙,中生毛羽,终以飞升。使物性可变,金木水火可革更也。虾蟆化为鹑,雀入水为蜃蛤,禀自然之性,非学道所能为也。好道之人,恐其或若等之类,故谓人能生毛羽,毛羽备具,能升天也。且夫物之生长,无卒成暴起,皆有浸渐。为道学仙之人,能先生数寸之毛羽,从地自奋,升楼台之陛,乃可谓升天。今无小升之兆,卒有大飞之验,何方术之学成无浸渐也?

毛羽大效,难以观实。且以人髯发物色少老验之。物生也色青,其熟也色黄。人之少也发黑,其老也发白。黄为物熟验,白为人老效。物黄,人虽灌溉壅养,终不能青;发白,虽吞药养性,终不能黑。黑青不可复还,老衰安可复却?黄之与白,犹肉腥炙之燋,鱼鲜煮之熟也。燋不可复令腥,熟不可复令鲜。鲜腥犹少壮,燋熟犹衰老也。天养物,能使物暢至秋,不得延之至春;吞药养性,能令人无病,不能寿之为仙。为仙体轻气强,犹未能升天,令见轻强之验,亦无毛羽之效,何用升天?天之与地,皆体也。地无下,则天无上矣。天无上升之路,何如?穿天之体?人力不能入。如天之门在西北,升天之人,宜从昆仑上。淮南之国,在地东南。如审升天,宜举家先从昆仑,乃得其阶。如鼓翼邪飞,趋西北之隅,是则淮南王有羽翼也。今不言其从之昆仑,亦不言其身生羽翼,空言升天,竟虚非实也。

案淮南王刘安,孝武皇帝之时也。父长,以罪迁蜀严道,至雍道死。安嗣为王,恨父徙死,怀反逆之心,招会术人,欲为大事。伍被之属充满殿堂,作道术之书,发怪奇之文,合景乱首。《八公之传》欲示神奇,若得道之状,道终不成,效验不立,乃与伍被谋为反事,事觉自杀,或言诛死。诛死、自杀,同一实也。世见其书,深冥奇怪,又观《八公之传》似若有效,则传称淮南王仙而升天,失其实也。

儒书言:卢敖游乎北海,经乎太阴,入乎玄关,至於蒙谷之上,见一士焉:深目玄准,雁颈而〔鸢〕肩,浮上而杀下,轩轩然方迎风而舞。顾见卢敖,樊然下其臂,遁逃乎碑下。敖乃视之,方卷然龟背而食合梨。卢敖仍与之语曰:``吾子唯以敖为倍俗,去群离党,穷观於六合之外者,非敖而己?敖幼而游,至长不偷〕解,周行四极,唯北阴之未窥。今卒睹夫子於是,殆可与敖为友乎?''若士者悖然而笑曰:``嘻!子中州之民也,不宜远至此。此犹光日月而戴列星,四时之所行,阴阳之所生也。此其比夫不名之地,犹突兀也。若我南游乎罔浪之野,北息乎沉薶之乡,西穷乎杳冥之党,而东贯湏懞之先。此其下无地,上无天,听焉无闻,而视焉则营;此其外犹有状,有状之余,壹举而能千万里,吾犹未能之在。今子游始至於此,乃语穷观,岂不亦远哉?然子处矣。吾与汗漫期於九垓之上,吾不可久。''若士者举臂而纵身,逐入云中。卢敖目仰而视之,不见,乃止喜心不怠,怅若有丧,曰:``吾比夫子也,犹黄鹄之与壤虫也,终日行,而不离咫尺,而自以为远,岂不悲哉!''

若卢敖者,唯龙无翼者升则乘云。卢敖言若士者有翼,言乃可信。今不言有翼,何以升云?且凡能轻举入云中者,饮食与人殊之故也。龙食与蛇异,故其举措与蛇不同。闻为道者,服金玉之精,食紫芝之英。食精身轻,故能神仙。若士者食合蜊之肉,与庸民同食,无精轻之验,安能纵体而升天?闻食气者不食物,食物者不食气。若士者食物如不食气,则不能轻举矣。

或时卢敖学道求仙,游乎北海,离众远去,无得道之效,惭於乡里,负於论议。自知以必然之事见责於世,则作夸诞之语,云见一士,其意以为有〔仙〕,求之未得,期数未至也。淮南王刘安坐反而死,天下并闻,当时并见,儒书尚有言其得道仙去,鸡犬升天者;况卢敖一人之身,独行绝迹之地,空造幽冥之语乎?是与河东蒲坂项曼都之语,无以异也。曼都好道学仙,委家亡去,三年而返。家问其状,曼都曰:``去时不能自知,忽见若卧形,有仙人数人,将我上天,离月数里而止。见月上下幽冥,幽冥不知东西。居月之旁,其寒凄怆。口饥欲食,仙人辄饮我以流霞一杯,每饮一杯,数月不饥。不知去几何年月,不知以何为过,忽然若卧,复下至此。''河东号之曰``斥仙''。实论者闻之,乃知不然。夫曼都能上天矣,何为不仙?已三年矣,何故复还?夫人去民间,升皇天之上,精气形体,有变於故者矣。万物变化,无复还者。复育化为蝉,羽翼既成,不能复化为复育。能升之物,皆有羽翼,升而复降,羽翼如故。见曼都之身有羽翼乎,言乃可信;身无羽翼,言虚妄也。虚则与卢敖同一实也。或时曼都好道,默委家去,周章远方,终无所得,力勌望极,默复归家,惭愧无言,则言上天。其意欲言道可学得,审有仙人;己殆有过,故成而复斥,升而复降。

儒书言:齐王疾痏,使人之宋迎文挚。文挚至,视王之疾,谓太子曰:``王之疾,必可已也。''虽然,王之疾已,则必杀挚也''。太子曰:``何故?''文挚对曰:``非怒王,疾不可治也。王怒,则挚必死。''太子顿首强请曰:``苟已王之疾,臣与臣之母以死争之於王,必幸臣之母。愿先生之勿患也。''文挚曰:``
诺,请以死为王。''与太子期,将往,不至者三,齐王固已怒矣。文挚至,不解屦登床,履衣,问王之疾。王怒而不与言。文挚因出辞以重王怒。王叱而起,疾乃遂已。王大怒不悦,将生烹文挚。太子与王后急争之而不能得,果以鼎生烹文挚。爨之三日三夜,颜色不变。文挚曰:``诚欲杀我,则胡不覆之,以绝阴阳之气?''王使覆之,文挚乃死。夫文挚,道人也,入水不濡,入火不燋,故在鼎三日三夜,颜色不变。此虚言也。

夫文挚而烹三日三夜,颜色不变,为一覆之故绝气而死,非得道之验也。诸生息之物,气绝则死。死之物,烹之辄烂。致生息之物密器之中,覆盖其口,漆涂其隙,中外气隔,息不得泄,有顷死也。如置汤镬之中,亦辄烂矣。何则?体同气均,禀性於天,共一类也。文挚不息乎?与金石同,入汤不烂,是也。令文挚息乎?烹之不死,非也。令文挚言,言则以声,声以呼吸。呼吸之动,因血气之发。血气之发,附於骨肉。骨肉之物,烹之辄死。今言烹之不死,一虚也。既能烹煮不死,此真人也,与金石同。金石虽覆盖,与不覆盖者无以异也。今言文挚覆之则死,二虚也。置人寒水之中,无汤火之热,鼻中口内不通於外,斯须之顷,气绝而死矣。寒水沉人,尚不得生,况在沸汤之中,有猛火之烈乎?言其入汤不死,三虚也。人没水中,口不见於外,言音不扬。烹文挚之时,身必没於鼎中。没则口不见,口不见则言不扬。文挚之言,四虚也。烹辄死之人,三日三夜,颜色不变,痴愚之人,尚知怪之。使齐王无知,太子群臣宜见其奇。奇怪文挚,则请出尊宠敬事,从之问道。今言三日三夜,无臣子请出之言,五虚也。此或时闻文挚实烹,烹而且死。世见文挚为道人也,则为虚生不死之语矣。犹黄帝实死也,传言升天;淮南坐反,书言度世。世好传虚,故文挚之语,传至於今。

世无得道之效,而有有寿之人,世见长寿之人,学道为仙,逾百不死,共谓之仙矣。何以明之?如武帝之时,有李少君,以祠灶、辟谷、却老方见上,上尊重之。少君匿其年及所生长,常自谓七十,而能使物却老。其游以方遍诸侯。无妻。人闻其能使物及不老,更馈遗之,常余钱金衣食。人皆以为不治产业饶给,又不知其何许人,愈争事之。少君资好方,善为巧发奇中。尝从武安侯饮,座中有年九十馀者,少君乃言其王父游射处。老人为兒时,从父,识其处。一座尽惊。少君见上,上有古铜器,问少君。少君曰:``此器齐桓公十五年陈於柏寝。''已而案其刻,果齐桓公器,一宫尽惊,以为少君数百岁人也。久之,少君病死。今世所谓得道之人,李少君之类也。少君死於人中,人见其尸,故知少君性寿之人也。如少君处山林之中,入绝迹之野,独病死於岩石之间,尸为虎狼狐狸之食,则世复以为真仙去矣。

世学道之人无少君之寿,年未至百,与众俱死。愚夫无知之人,尚谓之尸解而去,其实不死。所谓尸解者,何等也?谓身死精神去乎,谓身不死得免去皮肤也?如谓身死精神去乎,是与死无异,人亦仙人也;如谓不死免去皮肤乎,诸学道死者骨肉具在,与恆死之尸无以异也。夫蝉之去复育,龟之解甲,蛇之脱皮,鹿之堕角,壳皮之物解壳皮,持骨肉去,可谓尸解矣。今学道而死者,尸与复育相似,尚未可谓之尸解。何则?案蝉之去复育,无以神於复育,况不相似复育,谓之尸解,盖复虚妄失其实矣。太史公与李少君同世并时,少君之死,临尸者虽非太史公,足以见其实矣。如实不死。尸解而去,太史公宜纪其状,不宜言死,其处座中年九十老父为兒时者,少君老寿之效也。或少君年十四五,老父为兒,随其王父。少君年二百岁而死,何为不识?武帝去桓公铸铜器,且非少君所及见也。或时闻宫殿之内有旧铜器,或案其刻以告之者,故见而知之。今时好事之人,见旧剑古钩,多能名之,可复谓目见其铸作之时乎?

世或言:东方朔亦道人也,姓金氏,字曼倩。变姓易名,游宦汉朝。外有仕宦之名,内乃度世之人。此又虚也。

夫朔与少君并在武帝之时,太史公所及见也。少君有〔谷〕道祠灶却老之方,又名齐桓公所铸鼎,知九十老人王父所游射之验,然尚无得道之实,而徒性寿迟死之人也。况朔无少君之方术效验,世人何见谓之得道?案武帝之时,道人文成、五利之辈,入海求仙人,索不死之药,有道术之验,故为上所信。朔无入海之使,无奇怪之效也。如使有奇,不过少君之类,及文成、五利之辈耳,况谓之有道?此或时偶复若少君矣,自匿所生之处,当时在朝之人,不知其故,朔盛称其年长,人见其面状少,性又恬淡,不好仕宦,善达占卜射覆,为怪奇之戏,世人则谓之得道之人矣。

世或以老子之道为可以度世,恬淡无欲,养精爱气。夫人以精神为寿命,精神不伤则寿命长而不死。成事:老子行之,逾百度世,为真人矣。

夫恬淡少欲,孰与鸟兽?鸟兽亦老而死。鸟兽含情欲,有与人相类者矣,未足以言。草木之生何情欲?而春生秋死乎?夫草木无欲,寿不逾岁;人多情欲,寿至於百。此无情欲者反夭,有情欲者寿也。夫如是,老子之术,以恬淡无欲、延寿度世者,复虚也。或时老子,李少君之类也,行恬淡之道,偶其性命亦自寿长。世见其命寿,又闻其恬淡,谓老子以术度世矣。

世或以辟谷不食为道术之人,谓王子乔之辈,以不食谷,与恆人殊食,故与恆人殊寿,逾百度世,逐为仙人。此又虚也。

夫人之生也,禀食饮之性,故形上有口齿,形下有孔窍。口齿以噍食,孔窍以注泻。顺此性者,为得天正道,逆此性者为违所禀受。失本气於天,何能得久寿?使子乔生无齿口孔窍,是禀性与人殊。禀性与人殊,尚未可谓寿,况形体均同而以所行者异?言其得度世,非性之实也。夫人之不食也,犹身之不衣也。衣以温肤,食以充腹。肤温腹饱,精神明盛。如饥而不饱,寒而不温,则有冻饿之害矣。冻饿之人,安能久寿?且人之生也,以食为气,犹草木生以土为气矣。拔草木之根,使之离土,则枯而蚤死。闭人之口,使之不食,则饿而不寿矣。

道家相夸曰:``真人食气''。以气而为食,故传曰:``食气者寿而不死'',虽不谷饱,亦以气盈。''此又虚也。

夫气谓何气也?如谓阴阳之气,阴阳之气,不能饱人,人或咽气,气满腹胀,不能餍饱。如谓百药之气,人或服药,食一合屑,吞数十丸,药力烈盛,胸中愦毒,不能饱人。食气者必谓吹呴呼吸,吐故纳新也,昔有彭祖尝行之矣,不能久寿,病而死矣。

道家或以导气养性,度世而不死,以为血脉在形体之中,不动摇屈伸,则闭塞不通。不通积聚,则为病而死。此又虚也。

夫人之形,犹草木之体也。草木在高山之巅,当疾风之冲,昼夜动摇者,能复胜彼隐在山谷间,鄣於疾风者乎?案草木之生,动摇者伤而不暢,人之导引动摇形体者,何故寿而不死?夫血脉之藏於身也,犹江河之流地。江河之流,浊而不清,血脉之动,亦扰不安。不安,则犹人勤苦无聊也,安能得久生乎?

道家或以服食药物,轻身益气,延年度世。此又虚也。

夫服食药物,轻身益气,颇有其验。若夫延年度世,世无其效。百药愈病,病愈而气复,气复而身轻矣。凡人禀性,身本自轻,气本自长,中於风湿,百病伤之,故身重气劣也。服食良药,身气复故,非本气少身重,得药而乃气长身更轻也,禀受之时,本自有之矣。故夫服食药物除百病,令身轻气长,复其本性,安能延年至於度世?有血脉之类,无有不生,生无不死。以其生,故知其死也。天地不生,故不死;阴阳不生,故不死。死者,生之效;生者,死之验也。夫有始者必有终,有终者必有死。唯无终始者,乃长生不死。人之生,其犹〔冰〕也。水凝而为冰,气积而为人。冰极一冬而释,人竟百岁而死。人可令不死,冰可令不释乎?诸学仙术,为不死之方,其必不成,犹不能使冰终不释也。

\hypertarget{header-n287}{%
\subsubsection{语增篇}\label{header-n287}}

传语曰:圣人忧世,深思事勤,愁扰精神,感动形体,故称``尧若腊,舜若腒,桀、纣之君,垂腴尺余。''夫言圣人忧世念人,身体赢恶,不能身体肥泽,可也;言尧、舜若腊与腒,桀、纣垂腴尺余,增之也。

齐桓公云:``寡人未得仲父极难,既得仲父甚易。''桓公不及尧、舜,仲父不及禹、契,桓公犹易,尧、舜反难乎?以桓公得管仲易,知尧、舜得禹、契不难。夫易则少忧,少忧则不愁,不愁则身体不癯。舜承尧太平,尧、舜袭德。功假荒服,尧尚有忧,舜安〔而〕无事。故《经》曰:``上帝引逸'',谓虞舜也。舜承安继治,任贤使能,恭己无为而天下治。故孔子曰:``巍巍乎!舜、禹之有天下而不与焉。''夫不与尚谓之癯若腒,如德劣承衰,若孔子栖栖,周流应聘,身不得容,道不得行,可骨立〔皮〕附,僵仆道路乎?纣为长夜之饮,糟丘酒池,沉湎於酒,不舍昼夜,是必以病。病则不甘饮食,不甘饮食,则肥腴不得至尺。《经》曰:``惟湛乐是从,时亦罔有克寿。''魏公子无忌为长夜之饮,困毒而死。纣虽未死,宜赢癯矣。然桀、纣同行则宜同病,言其腴垂过尺余,非徒增之,又失其实矣。

传语又称:``纣力能索铁伸钩,抚梁易柱。''言其多力也。``蜚廉、恶来之徒,并幸受宠。言好伎力之主致伎力之士也。或言武王伐纣,兵不血刃。''夫以索铁伸钩之力,辅以蜚廉、恶来之徒,与周军相当,武王德虽盛,不能夺纣素所厚之心;纣虽恶,亦不失所与同行之意。虽为武王所擒,时亦宜杀伤十百人。今言``不血刃,''非纣多力之效,蜚廉、恶来助纣之验也。

案武王之符瑞,不过高祖。武王有白鱼、赤乌之佑,高祖有断大蛇、老妪哭於道之瑞。武王有八百诸侯之助,高祖有天下义兵之佐。武王之相,望羊而已;高祖之相,龙颜、隆准、项紫、美须髯,身有七十二黑子。高祖又逃吕后於泽中,吕后辄见上有云气之验,武王不闻有此。夫相多於望羊,瑞明於鱼乌,天下义兵并来会汉,助强於诸侯。武王承纣,高祖袭秦,二世之恶,隆盛於纣,天下畔秦,宜多於殷。案高祖伐秦,还破项羽,战场流血,暴尸万数,失军亡众,几死一再,然後得天下,用兵苦,诛乱剧。独云周兵不血刃,非其实也。言其易,可也;言不血刃,增之也。案周取殷之时,太公《阴谋》之书,食小兒丹,教云亡殷,兵到牧野,晨举脂烛。察《武成》之篇,牧野之战,血流浮杵,赤志千里。由此言之,周之取殷,与汉、秦一实也。而云取殷易,兵不血刃,美武王之德,增益其实也。凡天下之事,不可增损,考察前後,效验自列。自列,则是非之实有所定矣。世称纣力能索铁伸钩;又称武王伐之兵不血刃。夫以索铁伸钩之力当人,则是孟贲、夏育之匹也;以不血刃之德取人,是则三皇、五帝之属也。以索铁之力,不宜受服;以不血刃之德,不宜顿兵。今称纣力,则武王德贬;誉武王,则纣力少。索铁、不血刃,不得两立;殷、周之称,不得二全。不得二全,则必一非。

孔子曰:``纣之不善,不若是之甚也。是以君子恶居下流,天下之恶皆归焉。
''孟子曰:``吾於《武成》,取二三策耳。以至仁伐不仁,如何其血之浮杵也?''
若孔子言,殆〔且〕浮杵;若孟子之言,近不血刃。浮杵过其实,不血刃亦失其正。一圣一贤,共论一纣,轻重殊称,多少异实。纣之恶不若王莽。纣杀比干,莽鸩平帝;纣以嗣立,莽盗汉位。杀主隆於诛臣,嗣立顺於盗位,士众所畔,宜甚於纣。汉诛王莽,兵顿昆阳,死者万数,军至渐台,血流没趾。而独谓周取天下,兵不血刃,非其实也。

传语曰:``文王饮酒千钟,孔子百觚。''欲言圣人德盛,能以德将酒也。如一坐千钟百觚,此酒徒,非圣人也。饮酒有法,胸腹小大,与人均等。饮酒用千钟,用肴宜尽百牛,百觚则宜用十羊。夫以千钟百牛、百觚十羊言之,文王之身如防风之君,孔子之体如长狄之人,乃能堪之。案文王、孔子之体,不能及防风、长狄,以短小之身,饮食众多,是缺文王之广,贬孔子之崇也。

案《酒诰》之篇,``朝夕曰祀兹酒'',此言文王戒慎酒也。朝夕戒慎,则民化之。外出戒慎之教,内饮酒尽千钟,导民率下,何以致化?承纣疾恶,何以自别?且千钟之效,百觚之验,何所用哉?使文王、孔子因祭用酒乎?则受福胙不能厌饱。因飨射之用酒乎,飨射饮酒,自有礼法。如私燕赏赐饮酒乎?则赏赐饮酒,宜与下齐。赐尊者之前,三觞而退,过於三觞,醉酗生乱。文王、孔子,率礼之人也,赏赉左右,至於醉酗乱身:自用酒千钟百觚,大之则为桀、纣,小之则为酒徒,用何以立德成化,表名垂誉乎?世闻``德将毋醉''之言,见圣人有多德之效,则虚增文王以为千钟,空益孔子以百觚矣。

传语曰:``纣沉湎於酒,以糟为丘,以酒为池,牛饮者三千人,为长夜之饮,亡其甲子。''夫纣虽嗜酒,亦欲以为乐。令酒池在中庭乎?则不当言为长夜之饮。坐在深室之中,闭窗举烛,故曰长夜。令坐於室乎?每当饮者,起之中庭,乃复还坐,则是烦苦相藉,不能甚乐。令池在深室之中,则三千人宜临池坐,前俯饮池酒,仰食肴膳,倡乐在前,乃为乐耳。如审临池而坐,则前饮害於肴膳,倡乐之作不得在前。夫饮食既不以礼,临池牛饮,则其啖肴不复用杯,亦宜就鱼肉而虎食。则知夫酒池牛饮,非其实也。

传又言:纣悬肉以为林,令男女倮而相逐其间,是为醉乐淫戏无节度也。夫肉当内於口,口之所食,宜洁不辱。今言男女倮相逐其间,何等洁者?如以醉而不计洁辱,则当其浴於酒中,而倮相逐於肉间。何为不肯浴於酒中?以不言浴於酒,知不倮相逐於肉间。

传者之说,或言:``车行洒,骑行炙,百二十日为一夜。''夫言:``用酒为池,''则言其车行酒非也;言其``悬肉为林,''即言骑行炙非也。或时纣沉湎覆酒,滂沲於地,即言以酒为池。酿酒糟积聚,则言糟为丘。悬肉以林,则言肉为林。林中幽冥,人时走戏其中,则言倮相逐。或时载酒用鹿车,则言车行酒、骑行炙。或时十数夜,则言其百二十。或时醉不知问日数,则言其亡甲子。周公封康叔,告以纣用酒期於悉极,欲以戒之也。而不言糟丘酒池,悬肉为林,长夜之饮,亡其甲子。圣人不言,殆非实也。

传言曰:``纣非时与三千人牛饮於酒池。''夫夏官百,殷二百,周三百。纣之所与相乐,非民,必臣也;非小臣,必大官,其数不能满三千人。传书家欲恶纣,故言三千人,增其实也。

传语曰:``周公执贽下白屋之士。''谓候之也。夫三公,鼎足之臣,王者之贞干也;白屋之士,闾巷之微贱者也。三公倾鼎足之尊,执贽候白屋之士,非其实也。时或待士卑恭,不骄白屋,人则言其往候白屋;或时起白屋之士,以璧迎礼之,人则言其执贽以候其家也。

传语曰:``尧、舜之俭,茅茨不剪,采椽不斫。夫言茅茨采椽,可也;言不剪不斫,增之也。《经》曰``弼成五服''。五服,五采服也。服五采之服,又茅茨、采椽,何宫室衣服之不相称也?服五采,画日月星辰,茅茨、采椽,非其实也。

传语曰:``秦始皇帝燔烧诗书,坑杀儒士。''言燔烧诗书,灭去《五经》文书也;坑杀儒士者,言其皆挟经传文书之人也。烧其书,坑其人,诗书绝矣。言烧燔诗书、坑杀儒士,实也;言其欲灭诗书,故坑杀其人,非其诚,又增之也。

秦始皇帝三十四年,置酒咸阳台,儒士七十人前为寿。仆射周青臣进颂始皇之德。齐淳於越进谏始皇不封子弟功臣自为〔挟〕辅,刺周青臣以为面谀。始皇下其议於丞相李斯。李斯非淳于越曰:``诸生不师今而学古,以非当世,惑乱黔首。臣请敕史官,非秦记皆烧之;非博士官所职,天下有敢藏诗书、百家语、诸刑书者;悉诣守尉集烧之;有敢偶语诗书,弃市;以古非今者,族灭;吏见知弗举,与同罪。''始皇许之。明年,三十五年,诸生在咸阳者,多为妖言。始皇使御史案问诸生,诸生传相告引者,自除犯禁者四百六十七人,皆坑之。燔诗书,起淳于越之谏;坑儒士,起自诸生为妖言,见坑者四百六十七人。传增言坑杀儒士,欲绝诗书,又言尽坑之。此非其实,而又增之。

传语曰:``町町若荆轲之闾。''言荆轲为燕太子丹刺秦王,後诛轲九族,其後恚恨不已,复夷轲之一里,一里皆灭,故曰町町。此言增之也。

夫秦虽无道,无为尽诛荆轲之里。始皇幸梁山之宫,从山上望见丞相李斯车骑甚盛,恚,出言非之。其後左右以告李斯,李斯立损车骑。始皇知左右泄其言,莫知为谁,尽捕诸在旁者皆杀之。其後坠星下东郡,至地为石,民或刻其石曰``
始皇帝死,地分''。皇帝闻之,令御史逐问,莫服,尽取石旁人诛之。夫诛从行於梁山宫及诛石旁人,欲得泄言、刻石者,不能审知,故尽诛之。荆轲之闾,何罪於秦而尽诛之?如刺秦王在闾中,不知为谁,尽诛之,可也;荆轲已死,刺者有人,一里之民,何为坐之?始皇二十年,燕使荆轲刺秦王,秦王觉之,体解轲以徇,不言尽诛其闾。彼或时诛轲九族,九族众多,同里而处,诛其九族,一里且尽,好增事者,则言町町也。

\hypertarget{header-n308}{%
\subsection{卷八}\label{header-n308}}

\hypertarget{header-n309}{%
\subsubsection{儒增篇}\label{header-n309}}

儒书称:``尧、舜之德,至优至大,天下太平,一人不刑。''又言:``文、武之隆,遗在成、康,刑错不用四十余年。''是欲称尧、舜,褒文、武也。夫为言不益,则美不足称;为文不渥,则事不足褒。尧、舜虽优,不能使一人不刑;文、武虽盛,不能使刑不用。言其犯刑者少,用刑希疏,可也;言其一人不刑,刑错不用,增之也。

夫能使一人不刑,则能使一国不伐;能使刑错不用,则能使兵寝不施。案尧伐丹水,舜征有苗,四子服罪,刑兵设用。成王之时,四国篡畔,淮夷、徐戎,并为患害。夫刑人用刀,伐人用兵,罪人用法,诛人用武。武、法不殊,兵、刀不异。巧论之人,不能别也。夫德劣故用兵,犯法故施刑。刑与兵,犹足与翼也,走用足,飞用翼。形体虽异,其行身同。刑之与兵,全众禁邪,其实一也。称兵之用,言刑之不施,是犹人〔身〕缺目完,以目完称人体全,不可从也。人桀於刺虎,怯於击人,而以刺虎称谓之勇,不可听也。身无败缺,勇无不进,乃为全耳。今称``一人不刑,''不言一兵不用;褒``刑错不用,''不言一人不畔:未得为优,未可谓盛也。

儒书称:``楚养由基善射,射一杨叶,百发能百中之。''是称其巧於射也。夫言其时射一杨叶中之,可也;言其百发而百中,增之也。

夫一杨叶射而中之,中之一再,行败穿不可复射矣。如就叶悬於树而射之,虽不欲射叶,杨叶繁茂,自中之矣。是必使上取杨叶,一一更置地而射之也。射之数十行,足以见巧;观其射之者亦皆知射工,亦必不至於百,明矣。言事者好增巧美,数十中之,则言其百中矣。百与千,数之大者也。实欲言``十''则言``
百'',百则言``千''矣。是与《书》言``协和万邦'',《诗》曰``子孙千亿'',同一意也。

儒书言:``卫有忠臣弘演,为卫哀公使,未还,狄人攻哀公而杀之,尽食其肉,独舍其肝。弘演使还,致命於肝,痛哀公之死,身肉尽,肝无所附,引刀自刳其腹,尽出其腹实,乃内哀公之肝而死。''言此者,欲称其忠矣。言其自刳内哀公之肝而死,可也;言尽出其腹实乃内哀公之肝,增之也。

人以刃相刺,中五藏辄死。何则?五藏,气之主也,犹头,脉之凑也。头一断,手不能取他人之头著之於颈,奈何独能先出其腹实,乃内哀公之肝?腹实出,辄死,则手不能复把矣。如先内哀公之肝,乃出其腹实,则文当言``内哀公之肝,出其腹实。''今先言``尽出其腹实,内哀公之肝,''又言``尽'',增其实也。

儒书言:``楚熊渠子出,见寝石,以为伏虎,将弓射之,矢没其卫。''或曰:养由基见寝石,以为兕也,射之,矢饮羽。''或言:``李广''。便是熊渠、养由基、李广主名不审,无实也。或以为``虎'',或以为``兕'',兕、虎俱猛,一实也。或言``没卫'',或言饮羽,羽则卫,言不同耳,要取以寝石似虎、兕,畏惧加精,射之入深也。夫言以寝石为虎,射之矢入,可也;言其没卫,增之也。

夫见似虎者,意以为是,张弓射之,盛精加意,则其见真虎,与是无异。射似虎之石,矢入没卫,若射真虎之身,矢洞度乎?石之质难射,肉易射也。以射难没卫言之,则其射易者洞不疑矣。善射者能射远中微,不失毫厘,安能使弓弩更多力乎?养由基从军,射晋侯中其目。夫以匹夫射万乘之主,其加精倍力,必与射寝石等。当中晋侯之目也,可复洞达於项乎?如洞达於项,晋侯宜死。

车张十石之弩,恐不能入〔石〕一寸,〔矢〕摧为三,况以一人之力,引微弱之弓,虽加精诚,安能没卫?人之精乃气也,气乃力也。有水火之难,惶惑恐惧,举徙器物,精诚至矣,素举一石者,倍举二石。然则,见伏石射之,精诚倍故,不过入一寸,如何谓之没卫乎?如有好用剑者,见寝石,惧而斫之,可复谓能断石乎?以勇夫空拳而暴虎者,卒然见寝石,以手椎之,能令石有迹乎?巧人之精,与拙人等;古人之诚与今人同。使当今射工,射禽兽於野,其欲得之,不余精力乎?及其中兽,不过数寸。跌误中石,不能内锋,箭摧折矣。夫如是,儒书之言楚熊渠子、养由基、李广射寝石,矢没卫饮羽者,皆增之也。

儒书称:``鲁般、墨子之巧,刻木为鸢,飞之三日而不集''。夫言其以木为鸢飞之,可也;言其三日不集,增之也。

夫刻木为鸢以象鸢形,安能飞而不集乎?既能飞翔,安能至於三日?如审有机关,一飞遂翔,不可复下,则当言遂飞,不当言三日。犹世传言曰:``鲁般巧,亡其母也。''言巧工为母作木车马、木人御者,机关备具,载母其上,一驱不还,遂失其母。如木鸢机关备具,与木车马等,则遂飞不集。机关为须臾间,不能远过三日,则木车等亦宜三日止於道路,无为径去以失其母。二者必失实者矣。

书说:孔子不能容於世,周流游说七十余国,未尝得安。夫言周流不遇,可也;言干七十国,增之也。

案《论语》之篇、诸子之书,孔子自卫反鲁,在陈绝粮,削迹於卫,忘味於齐,伐树於宋,并费与顿牟,至不能十国。传言七十国,非其实也。或时干十数国也,七十之说,文书传之,因言干七十国矣。

《论语》曰:``孔子问公叔文子於公明贾曰:`信乎,夫子不言、不笑、不取乎?'公明贾对曰:`以告者,过也。夫子时然後言,人不厌其言也;乐然後笑,人不厌其笑也;义然後取,人不厌其取也。'子曰:`岂其然乎!岂其然乎!
'''夫公叔文子实时言、时笑、义取,人传说称之;言其不言、不笑、不取也,俗言竟增之也。

书言:秦缪公伐郑,过晋不假途,晋襄公率羌戎要击於崤塞之下,匹马只轮无反者。时秦遣三大夫孟明视、西乞术、白乙丙皆得复还。夫三大夫复还,车马必有归者;文言匹马只轮无反者,增其实也。

书称:``齐之孟尝,魏之信陵,赵之平原,楚之春申君,待士下客,招会四方,各三千人。''欲言下士之至,趋之者众也。夫言士多,可也;言其三千,增之也。

四君虽好士,士至虽众,不过各千余人。书则言三千矣。夫言众必言千数,言少则言无一。世俗之情,言事之失也。

传记言:``高子羔之丧亲,泣血三年未尝见齿。君子以为难。''难为故也。夫不以为非实而以为难,君子之言误矣。高子泣血,殆必有之。何则?荆和献宝於楚,楚刖其足,痛宝不进,己情不达,泣涕,涕尽因续以血。今高子痛亲,哀极涕竭血随而出,实也。而云三年未尝见齿,是增之也。

言未尝见齿,欲言其不言、不笑也。孝子丧亲不笑,可也,安得不言?言安得不见齿?孔子曰:``言不文。''或时不言,传则言其不见齿;或时传则言其不见齿三年矣。高宗谅阴,三年不言。尊为天子,不言,而其文言不言,犹疑於增,况高子位贱,而曰未尝见齿,是必增益之也。

儒书言:禽息荐百里奚,缪公未听,禽息出,当门仆头碎首而死。缪公痛之,乃用百里奚。此言贤者荐善,不爱其死,仆头碎首而死,以达其友也。世士相激,文书传称之,莫谓不然。夫仆头以荐善,古今有之。禽息仆头,盖其实也;言碎首而死,是增之也。

夫人之扣头,痛者血流,虽忿恨惶恐,无碎首者。非首不可碎,人力不能自碎也。执刃刎颈,树锋刺胸,锋刃之助,故手足得成势也。言禽息举椎自击,首碎,不足怪也;仆头碎首,力不能自将也。有扣头而死者,未有使头破首碎者也。此时或扣头荐百里奚,世空言其死;若或扣头而死,世空言其首碎也。

儒书言:荆轲为燕太子刺秦王,操匕首之剑,刺之不得。秦王拔剑击之。轲以匕首掷秦王不中,中铜柱,入尺。欲言匕首之利,荆轲势盛,投锐利之刃,陷坚强之柱,称荆轲之勇,故增益其事也。夫言入铜柱,实也;言其入尺,增之也。

夫铜虽不若匕首坚刚,入之不过数寸,殆不能入尺。以入尺言之,设中秦王,匕首洞过乎?车张十石之弩,射垣木之表,尚不能入尺。以荆轲之手力,投轻小之匕首,身被龙渊之剑刃,入坚刚之铜柱,是荆轲之力劲於十石之弩,铜柱之坚不若木表之刚也。世称荆轲之勇,不言其多力。多力之人,莫若孟贲。使孟贲挝铜柱,能〔洞〕出一尺乎?此亦或时匕首利若干将、莫邪,所刺无前,所击无下,故有入尺之效。夫称干将、莫邪,亦过其实。刺击无前下,亦入铜柱尺之类也。

儒书言:``董仲舒读《春秋》,专精一思,志不在他,三年不窥园菜。''夫言不窥园菜,实也;言三年,增之也。

仲舒虽精,亦时解休,解休之间,犹宜游於门庭之侧;则能至门庭,何嫌不窥园菜?闻用精者,察物不见,存道以亡身;不闻不至门庭,坐思三年,不及窥园也。《尚书毋佚》曰``君子所其毋逸,先知稼穑之艰难,乃佚''。者也。人之筋骨,非木非石,不能不解。故张而不弛,文王不为;弛而不张,文王不行;一弛一张,文王以为常。圣人材优,尚有弛张之时。仲舒材力劣於圣,安能用精三年不休?

儒书言:夏之方盛也,远方图物,贡金九牧,铸鼎象物,而为之备,故入山泽不逢恶物,用辟神奸,故能叶於上下,以承天休。

夫金之性,物也,用远方贡之为美,铸以为鼎,用象百物之奇,安能入山泽不逢恶物,辟除神奸乎?周时天下太平,越裳献白雉,倭人贡鬯草。食白雉,服鬯草,不能除凶;金鼎之器,安能辟奸?且九鼎之来,德盛之瑞也。服瑞应之物,不能致福。男子服玉,女子服珠。珠玉於人,无能辟除。宝奇之物,使为兰服,作牙身,或言有益者,九鼎之语也。夫九鼎无能辟除,传言能辟神奸,是则书增其文也。

世俗传言:``周鼎不爨自沸;不投物,物自出。''此则世俗增其言也,儒书增其文也,是使九鼎以无怪空为神也。且夫谓周之鼎神者,何用审之?周鼎之金,远方所贡,禹得铸以为鼎也。其为鼎也,有百物之象。如为远方贡之为神乎,远方之物安能神?如以为禹铸之为神乎,禹圣不能神,圣人身不能神,铸器安能神?如以金之物为神乎,则夫金者石之类也,石不能神,金安能神?以有百物之象为神乎,夫百物之象犹雷樽也,雷樽刻画云雷之形,云雷在天,神於百物,云雷之象不能神,百物之象安能神也?

传言:秦灭周,周之九鼎入於秦。

案本事,周赧王之时,秦昭王使将军攻王赧,王赧惶惧奔秦,顿首受罪,尽献其邑三十六、口三万。秦受其献还王赧。王赧卒,秦王取九鼎宝器矣。若此者,九鼎在秦也。始皇二十八年,北游至琅邪,还过彭城,齐戒祷祠,欲出周鼎,使千人没泗水之中,求弗能得。案时,昭王之後三世得始皇帝,秦无危乱之祸,鼎宜不亡,亡时殆在周。传言王赧奔秦,秦取九鼎,或时误也。传又言:``宋太丘社亡,鼎没水中彭城下,其後二十九年,秦并天下。''若此者,鼎未入秦也。其亡,从周去矣,未为神也。

春秋之时,五石陨於宋。五石者星也,星之去天,犹鼎之亡於地也。星去天不为神,鼎亡於地何能神?春秋之时,三山亡,犹太丘社之去宋,五星之去天。三山亡,五石陨,太丘社去,皆自有为。然鼎亡,亡亦有应也。未可以亡之故,乃谓之神。如鼎与秦三山同乎,亡不能神。如有知欲辟危乱之祸乎,则更桀、纣之时矣。衰乱无道,莫过桀、纣,桀、纣之时,鼎不亡去。周之衰乱,未若桀、纣。留无道之桀、纣,去衰末之周,非止去之宜神有知之验也。或时周亡之时,将军摎人众见鼎盗取,奸人铸烁以为他器,始皇求不得也。後因言有神名,则空生没於泗水之语矣。

孝文皇帝之时,赵人新垣平上言:``周鼎亡在泗水中。今河溢,通於泗水。臣望东北,汾阴直有金气,意周鼎出乎!兆见弗迎则不至。''於是文帝使使治庙汾阴,南临河,欲祠出周鼎。人有上书告新垣平所言神器事皆诈也,於是下平事於吏。吏治,诛新垣平。夫言鼎在泗水中,犹新垣平诈言鼎有神气见也。

\hypertarget{header-n344}{%
\subsubsection{艺增篇}\label{header-n344}}

世谷所患,患言事增其实;著文垂辞,辞出溢其真,称美过其善,进恶没其罪。何则?俗人好奇。不奇,言不用也。故誉人不增其美,则闻者不快其意;毁人不益其恶,则听者不惬於心。闻一增以为十,见百益以为千。使夫纯朴之事,十剖百判;审然之语,千反万畔。墨子哭於练丝,杨子哭於歧道,盖伤失本,悲离其实也。蜚流之言,百传之语,出小人之口,驰闾巷之间,其犹是也。诸子之文,笔墨之疏,〔大〕贤所著,妙思所集,宜如其实,犹或增之。傥经艺之言,如其实乎?言审莫过圣人,经艺万世不易,犹或出溢,增过其实。增过其实,皆有事为,不妄乱误以少为多也?然而必论之者,方言经艺之增与传语异也。经增非一,略举较著,令怳惑之人,观览采择,得以开心通意,晓解觉悟。

《尚书》曰:``协和万国'',是美尧德致太平之化,化诸夏并及夷狄也。言协和方外,可也;言万国,增之也。

夫唐之与周,俱治五千里内。周时诸侯千七百九十三国,荒服、戎服、要服及四海之外不粒食之民,若穿胸、儋耳、焦侥、跋踵之辈,并合其数,不能三千。天之所覆,地之所载,尽於三千之中矣。而《尚书》云万国,褒增过实以美尧也。欲言尧之德大,所化者众,诸夏夷狄,莫不雍和,故曰万国。犹《诗》言``子孙千亿''矣,美周宣王之德能慎天地,天地祚之,子孙众多,至於千亿。言子孙众多,可也;言千亿,增之也。夫子孙虽众,不能千亿,诗人颂美,增益其实。案后稷始受邰封,讫於宣王,宣王以至外族内属,血脉所连,不能千亿。夫千与万,数之大名也。万言众多,故《尚书》言万国,《诗》言千亿。

《诗》云:``鹤鸣九皋,声闻於天。''言鹤鸣九折之泽,声犹闻於天,以喻君子修德穷僻,名犹达朝廷也。〔言〕其闻高远,可矣;言其闻於天,增之也。

彼言声闻於天,见鹤鸣於云中,从地听之,度其声鸣於地,当复闻於天也。夫鹤鸣云中,人闻声仰而视之,目见其形。耳目同力,耳闻其声,则目见其形矣。然则耳目所闻见,不过十里,使参天之鸣,人不能闻也。何则?天之去人以万数远,则目不能见,耳不能闻。今鹤鸣从下闻之,鹤鸣近也。以从下闻其声,则谓其鸣於地,当复闻於天,失其实矣。其鹤鸣於云中,人从下闻之,如鸣於九皋。人无在天上者,何以知其闻於天上也?无以知,意从准况之也。诗人或时不知,至诚以为然;或时知而欲以喻事,故增而甚之。

《诗》曰:``维周黎民,靡有孑遗''是谓周宣王之时,遭大旱之灾也。诗人伤早之甚,民被其害,言无有孑遗一人不愁痛者。夫早甚,则有之矣;言无孑遗一人,增之也。

夫周之民,犹今之民也。使今之民也,遭大旱之灾,贫羸无蓄积,扣心思雨;若其富人,谷食饶足者,廪囷不空,口腹不饥,何愁之有?天之旱也,山林之间不枯,犹地之水,丘陵之上不湛也。山林之间,富贵之人,必有遣脱者矣,而言靡有孑遗,增益其文,欲言旱甚也。

《易》曰:``丰其屋,蔀其家,窥其户,阒其无人也。''非其无人也,无贤人也。《尚书》曰:``毋旷庶官。''旷,空;庶,众也。毋空众官,置非其人,与空无异,故言空也。

夫不肖者皆怀五常,才劣不逮,不成纯贤,非狂妄顽嚚,身中无一知也。德有大小,材有高下,居官治职,皆欲勉效在官。《尚书》之官,《易》之户中,犹能有益,如何谓之空而无人?《诗》曰:``济济多士,文王以宁。''此言文王得贤者多而不肖者少也。今《易》宜言``阒其少人'',《尚书》宜言``无少众官''
。以少言之,可也;言空而无人,亦尤甚焉。

五谷之於人也,食之皆饱。稻粱之味,甘而多腴。豆麦虽粝,亦能愈饥。食豆麦者,皆谓粝而不甘,莫谓腹空无所食。竹木之杖,皆能扶病。竹杖之力,弱劣不及木。或操竹杖,皆谓不劲,莫谓手空无把持。夫不肖之臣,豆麦、竹杖之类也。《易》持其具臣在户,言无人者,恶之甚也。《尚书》众官,亦容小材,而云无空者,刺之甚也。

《论语》曰:``大哉!尧之为君也。荡荡乎民无能名焉。''传曰:``有年五十击壤於路者,观者曰:`大哉!尧德乎!'击壤者曰:`吾日出而作,日入而息,凿井而饮,耕田而食,尧何等力!''此言荡荡无能名之效也。言荡荡,可也;乃欲言民无能名,增之也。四海之大,万民之众,无能名尧之德者,殆不实也。

夫击壤者曰:``尧何等力?'''欲言民无能名也。观者曰:``大哉!尧之德乎!''此何等民者,犹能知之。实有知之者,云无,竟增之。

儒书又言:``尧、舜之民,可比屋而封。''言其家有君子之行,可皆官也。夫言可封,可也;言比屋,增之也

人年五十为人父,为人父而不知君,何以示子?太平之世,家为君子,人有礼义,父不失礼,子不废行。夫有行者有知,知君莫如臣,臣贤能知君,能知其君,故能治其民。今不能知尧,何可封官?年五十击壤於路,与竖子未成人者为伍,何等贤者?子路使子羔为郈宰,孔子以为不可:未学,无所知也。击壤者无知,官之如何?称尧之荡荡,不能述其可比屋而封;言贤者可比屋而封,不能议让其愚。而无知之,夫击壤者,难以言比屋,比屋难以言荡荡。二者皆增之,所由起,美尧之德也。

《尚书》曰:``祖伊谏纣曰:今我民罔不欲丧。''罔,无也;我天下民无不欲王亡者。夫言欲王之亡,可也;言无不,增之也。

纣虽恶,民臣蒙恩者非一,而祖伊增语,欲以惧纣也。故曰:语不益,心不惕;心不惕,行不易。增其语欲以惧之,冀其警悟也。

苏秦说齐王曰:``临淄之中,车毂击,人肩磨,举袖成幕,连衽成帷,挥汗成雨。''齐虽炽盛,不能如此。苏秦增语,激齐王也。祖伊之谏纣,犹苏秦之说齐王也。贤圣增文,外有所为,内未必然。何以明之?夫《武成》之篇,言武王伐纣,血流浮杵。助战者多,故至血流如此。皆欲纣之亡也,土崩瓦解,安肯战乎?然祖伊之言民无不欲,如苏秦增语。《武成》言血流浮杵,亦太过焉。死者血流,安能浮杵?案武王伐纣於牧之野。河北地高,壤靡不干燥。兵顿血流,辄燥入土,安得杵浮?且周、殷士卒,皆赍盛粮,无杵臼之事,安得杵而浮之?言血流杵,欲言诛纣,惟兵顿士伤,故至浮杵。

《春秋》``庄公七年:夏四月辛卯,夜中恆星不见,星霣如雨。''《公羊传》曰:``如雨者何?非雨也。非雨则曷为谓之如雨?不修《春秋》曰:雨星,不及地尺而复。君子修之,`星如雨'。''不修《春秋》者,未修《春秋》时《鲁史记》,曰``雨星不及地尺如复''。君子者,谓孔子也。孔子修之,``星如雨''。如雨者,如雨状也。山气为云,上不及天,下而为雨。星陨不及地,上复在天,故曰如雨。孔子正言也。夫星霣或时至地,或时不能,尺丈之数难审也。《史记》言尺,亦以太甚矣。夫地有楼台山陵,安得言尺?孔子言如雨,得其实矣。孔子作《春秋》,故正言如雨。如孔子不作,不及地尺之文,遂传至今。

光武皇帝之时,郎中汝南贲光上书言:``孝文皇帝时居明光宫,天下断狱三人。''颂美文帝,陈其效实。光武皇帝曰:``孝文时不居明光宫,断狱不三人。''
积善修德,美名流之,是以君子恶居下流。夫贲光上书於汉,汉为今世,增益功美,犹过其实,况上古帝王久远,贤人从後褒述,失实离本,独已多矣。不遭光武论,千世之後,孝文之事,载在经艺之上,人不知其增,居明光宫,断狱三人,而遂为实事也。

\hypertarget{header-n367}{%
\subsection{卷九}\label{header-n367}}

\hypertarget{header-n368}{%
\subsubsection{问孔篇}\label{header-n368}}

世儒学者,好信师而是古,以为贤圣所言皆无非,专精讲习,不知难问。夫贤圣下笔造文,用意详审,尚未可谓尽得实,况仓卒吐言,安能皆是?不能皆是,时人不知难;或是,而意沉难见,时人不知问。案贤圣之言,上下多相违;其文,前後多相伐者。世之学者,不能知也。

论者皆云:``孔门之徒,七十子之才,胜今之儒。''此言妄也。彼见孔子为师,圣人传道,必授异才,故谓之殊。夫古人之才,今人之才也。今谓之英杰,古以为圣神,故谓七十子历世希有。使当今有孔子之师,则斯世学者,皆颜、闵之徒也;使无孔子,则七十子之徒,今之儒生也。何以验之?以学於孔子,不能极问也。圣人之言,不能尽解;说道陈义,不能辄形。不能辄形,宜问以发之;不能尽解,宜难以极之。皋陶陈道帝舜之前,浅略未极。禹问难之,浅言复深,略指复分。盖起问难此说激而深切、触而著明也。

孔子笑子游之弦歌,子游引前言以距孔子。自今案《论语》之文,孔子之言多若笑弦歌之辞,弟子寡若子游之难,故孔子之言遂结不解。以七十子不能难,世之儒生,不能实道是非也。

凡学问之法,不为无才,难於距师,核道实义,证定是非也。问难之道,非必对圣人及生时也。世之解说说人者,非必须圣人教告,乃敢言也。苟有不晓解之问,〔追〕难孔子,何伤於义?诚有传圣业之知,伐孔子之说,何逆於理?谓问孔子之言,难其不解之文,世间弘才大知生,能答问、解难之人,必将贤吾世间难问之言是非。

孟懿子问孝。子曰:``毋违。''樊迟御,子告之曰:``孟孙问孝於我,我对曰`毋违'。''樊迟曰:``何谓也?''子曰:``生,事之以礼;死,葬之以礼。''
问曰:孔子之言毋违,毋违者,礼也。孝子亦当先意承志,不当违亲之欲。孔子言毋违,不言违礼。懿子听孔子之言,独不为嫌於毋违志乎。樊迟问何谓,孔子乃言``生,事之以礼;死,葬之以礼,祭之以礼。''使樊迟不问,毋违之说,遂不可知也。懿子之才,不过樊迟,故《论语》篇中不见言行。樊迟不晓,懿子必能晓哉?

孟武伯问孝,子曰:``父母,唯其疾之忧。''武伯善忧父母,故曰``唯其疾之忧''。武伯忧亲,懿子违礼。攻其短,答武伯云``父母,唯其疾之忧'',对懿子亦宜言唯水火之变乃违礼。周公告小才敕,大材略。子游之大材也,孔子告之敕;懿子小才也,告之反略。违周公之志,攻懿子之短,失道理之宜。弟子不难,何哉?如以懿子权尊,不敢极言,则其对武伯亦宜但言毋忧而已。俱孟氏子也,权尊钧同,敕武伯而略懿子,未晓其故也。使孔子对懿子极言毋违礼,何害之有?专鲁莫过季氏,讥八佾之舞庭,刺太山之旅祭,不惧季氏增邑不隐讳之害,独畏答懿子极言之罪,何哉?且问孝者非一,皆有御者,对懿子言,不但心服臆肯,故告樊迟。

孔子曰:``富与贵,是人之所欲也,不以其道得之,不居也;贫与贱,是人之所恶也,不以其道得之,不去也。''此言人当由道义得,不当苟取也;当守节安贫,不当妄去也。

夫言不以其道,得富贵不居,可也;不以其道,得贫贱如何?富贵顾可去,去贫贱何之?去贫贱,得富贵也。不得富贵,不去贫贱。如谓得富贵不以其道,则不去贫贱邪?则所得富贵,不得贫贱也。贫贱何故当言得之?顾当言贫与贱是人之所恶也,不以其道去之,则不去也。当言去,不当言得。得者,施於得之也。今去之,安得言得乎?独富贵当言得耳。何者?得富贵,乃去贫贱也。是则以道去贫贱如何?修身行道,仕得爵禄、富贵。得爵禄、富贵,则去贫贱矣。不以其道去贫贱如何?毒苦贫贱,起为奸盗,积聚货财,擅相官秩,是为不以其道。七十子既不问,世之学者亦不知难。使此言意不解而文不分,是谓孔子不能吐辞也;使此言意结文又不解,是孔子相示未形悉也。弟子不问,世俗不难,何哉?

孔子曰:``公冶长可妻也,虽在缧绁之中,非其罪也。''以其子妻之。

问曰:孔子妻公冶长者,何据见哉?据年三十可妻邪,见其行贤可妻也?如据其年三十,不宜称在缧绁;如见其行贤,亦不宜称在缧绁。何则?诸入孔子门者,皆有善行,故称备徒役。徒役之中无妻,则妻之耳,不须称也。如徒役之中多无妻,公冶长尤贤,故独妻之,则其称之宜列其行,不宜言其在缧绁也。何则?世间强受非辜者多,未必尽贤人也。恆人见枉,众多非一,必以非辜为孔子所妻,则是孔子不妻贤,妻冤也。案孔子之称公冶长,有非辜之言,无行能之文。实不贤,孔子妻之,非也;实贤,孔子称之不具,亦非也。诚似妻南容云,国有道不废,国无道免於刑戮,具称之矣。

子谓子贡曰:``汝与回也,孰愈?''曰:``赐也,何敢望回?回也,闻一以知十;赐也,闻一以知二。''子曰:``弗如也,吾与汝俱不如也。''是贤颜渊试以问子贡也。

问曰:孔子所以教者,礼让也。子路,为国以礼,其言不让,孔子非之。使子贡实愈颜渊,孔子问之,犹曰不如,使实不及,亦曰不如,非失对欺师,礼让之言宜谦卑也。今孔子出言,欲何趣哉?使孔子知颜渊愈子贡,则不须问子贡。使孔子实不知,以问子贡,子贡谦让亦不能知。使孔子徒欲表善颜渊,称颜渊贤,门人莫及,於名多矣,何须问於子贡?子曰:``贤哉,回也!''又曰:``吾与回言终日,不违如愚。''又曰:``回也,其心三月不违仁。''三章皆直称,不以他人激。至是一章,独以子贡激之,何哉?

或曰:欲抑子贡也。当此之时,子贡之名凌颜渊之上,孔子恐子贡志骄意溢,故抑之也。夫名在颜渊之上,当时所为,非子贡求胜之也。实子贡之知何如哉?使颜渊才在己上,己自服之,不须抑也。使子贡不能自知,孔子虽言,将谓孔子徒欲抑已。由此言之,问与不问,无能抑扬。

宰我昼寝。子曰:``朽木不可雕也,粪土之墙不可圬也,於予予何诛。''是恶宰予之昼寝。

问曰:昼寝之恶也,小恶也;朽木粪土,败毁不可复成之物,大恶也。责小过以大恶,安能服人?使宰我性不善,如朽木粪土,不宜得入孔子之门,序在四科之列。使性善,孔子恶之,恶之太甚,过也;人之不仁,疾之已甚,乱也。孔子疾宰予,可谓甚矣。使下愚之人涉耐罪,狱吏令以大辟之罪,必冤而怨邪?将服而自咎也?使宰我愚,则与涉耐罪之人同志;使宰我贤,知孔子责人,几微自改矣。明文以识之,流言以过之,以其言示端而已自改。自改不在言之轻重,在宰予能更与否。

《春秋》之义,采毫毛之善,贬纤介之恶,褒毫毛以巨大,以巨大贬纤介。观《春秋》之义,肯是之乎?不是,则宰我不受;不受,则孔子之言弃矣。圣人之言与文相副,言出於口,文立於策,俱发於心,其实一也。孔子作《春秋》,不贬小以大。其非宰予也,以大恶细,文语相违,服人如何?

子曰:``始吾於人也,听其言而信其行;今吾於人也,听其言而观其行。於予予改是。''盖起宰予昼寝,更知人之术也。

问曰:人之昼寝,安足以毁行?毁行之人,昼夜不卧,安足以成善?以昼寝而观人善恶,能得其实乎?案宰予在孔子之门,序於四科,列在赐上。如性情怠,不可雕琢,何以致此?使宰我以昼寝自致此,才复过人远矣。如未成就,自谓已足,不能自知,知不明耳,非行恶也。晓敕而已,无为改术也。如自知未足,倦极昼寝,是精神索也。精神索至於死亡,岂徒寝哉?且论人之法,取其行则弃其言,取其言则弃其行。今宰予虽无力行,有言语。用言,令行缺,有一概矣。今孔子起宰予昼寝,听其言,观其行,言行相应,则谓之贤。是孔子备取人也。毋求备於一人之义,何所施?

子张问:``令尹子文三仕为令尹,无喜色;三已之,无愠色;旧令尹之政,必以告新令尹。何如?''子曰:``忠矣。''曰:``仁矣乎?''曰:``未知,焉得仁?''子文曾举楚子玉代己位而伐宋,以百乘败而丧其众,不知如此,安得为仁?

问曰:子文举子玉,不知人也。智与仁,不相干也。有不知之性,何妨为仁之行?五常之道,仁、义、礼、智、信也。五者各别,不相须而成。故有智人、有仁人者,有礼人、有义人者。人有信者未必智,智者未必仁,仁者未必礼,礼者未必义。子文智蔽於子玉,其仁何毁?谓仁,焉得不可?且忠者,厚也。厚人,仁矣。孔子曰:``观过,斯知仁矣。''子文有仁之实矣。孔子谓忠非仁,是谓父母非二亲,配匹非夫妇也。

哀公问:``弟子孰谓好学?''孔子对曰:``有颜回者,不迁怒,不贰过,不幸短命死矣。今也则亡,未闻好学者也。''

夫颜渊所以死者,审何用哉?令自以短命,犹伯牛之有疾也。人生受命,皆全当洁。今有恶疾,故曰无命。人生皆当受天长命,今得短命,亦宜曰无命。如〔命〕有短长,则亦有善恶矣。言颜渊短命,则宜言伯牛恶命;言伯牛无命,则宜言颜渊无命。一死一病,皆痛云命。所禀不异,文语不同。未晓其故也。

哀公问孔子孰为好学。孔子对曰:``有颜回者好学,今也则亡。不迁怒,不贰过。''何也?曰:并攻哀公之性,迁怒、贰过故也。因其问则并以对之,兼以攻上之短,不犯其罚。

问曰:康子亦问好学,孔子亦对之以颜渊。康子亦有短,何不并对以攻康子?康子,非圣人也,操行犹有所失。成事,康子患盗,孔子对曰:``苟子之不欲,虽赏之不窃。''由此言之,康子以欲为短也。不攻,何哉?

孔子见南子,子路不悦。子曰:``予所鄙者,天厌之!天厌之!''南子,卫灵公夫人也,聘孔子,子路不说,谓孔子淫乱也。孔子解之曰:我所为鄙陋者,天厌杀我。至诚自誓,不负子路也。

问曰:孔子自解,安能解乎?使世人有鄙陋之行,天曾厌杀之,可引以誓;子路闻之,可信以解;今未曾有为天所厌者也,曰天厌之,子路肯信之乎?行事,雷击杀人,水火烧溺人,墙屋压填人。如曰雷击杀我,水火烧溺我,墙屋压填我,子路颇信之;今引未曾有之祸,以自誓於子路,子路安肯晓解而信之?行事,适有卧厌不悟者,谓此为天所厌邪?案诸卧厌不悟者,未皆为鄙陋也。子路入道虽浅,犹知事之实。事非实,孔子以誓,子路必不解矣。

孔子称曰:``死生有命,富贵在天。''若此者,人之死生自有长短,不在操行善恶也。成事,颜渊蚤死,孔子谓之短命。由此知短命夭死之人,必有邪行也。子路入道虽浅,闻孔子之言,知死生之实。孔子誓以``予所鄙者,天厌之''!独不为子路言:夫子惟命未当死,天安得厌杀之乎?若此,誓子路以天厌之,终不见信。不见信,则孔子自解,终不解也。《尚书》曰:``毋若丹硃敖,惟慢游是好。''谓帝舜敕禹毋子不肖子也。重天命,恐禹私其子,故引丹硃以敕戒之。禹曰:``予娶若时,辛壬癸甲,开呱呱而泣,予弗子。''陈已行事以往推来,以见卜隐,效己不敢私不肖子也。不曰天厌之者,知俗人誓,好引天也。孔子为子路所疑,不引行事,效己不鄙,而云天厌之,是与俗人解嫌引天祝诅,何以异乎?

孔子曰:``凤鸟不至,河不出图,吾已矣夫。''夫子自伤不王也。己王,致太平;太平则凤鸟至,河出图矣。今不得王,故瑞应不至,悲心自伤,故曰``吾已矣夫''。

问曰:凤鸟、河图,审何据始起?始起之时,鸟、图未至;如据太平,太平之帝,未必常致凤鸟与河图也。五帝、三王,皆致太平。案其瑞应,不皆凤皇为必然之瑞;於太平,凤皇为未必然之应。孔子,圣人也,思未必然以自伤,终不应矣。

或曰:孔子不自伤不得王也,伤时无明王,故己不用也。凤鸟、河图,明王之瑞也。瑞应不至,时无明王;明王不存,己遂不用矣。

夫致瑞应,何以致之?任贤使能,治定功成;治定功成,则瑞应至矣。瑞应至後,亦不须孔子。孔子所望,何其末也!不思其本而望其末也。不相其主而名其物,治有未定,物有不至,以至而效明王,必失之矣。孝文皇帝可谓明矣,案其《本纪》,不见凤鸟与河图。使孔子在孝文之世,犹曰``吾已矣夫''。

子欲居九夷,或曰:``陋,如之何?''子曰:``君子居之,何陋之有!''孔子疾道不行於中国,志恨失意,故欲之九夷也。或人难之曰:``夷狄之鄙陋无礼义,如之何?''孔子曰:``君子居之,何陋之有?''言以君子之道,居而教之,何为陋乎?

问之曰:孔子欲之九夷者,何起乎?起道不行於中国,故欲之九夷。夫中国且不行,安能行於夷狄?``夷狄之有君,不若诸夏之亡''。言夷狄之难,诸夏之易也。不能行於易,能行於难乎?且孔子云:``以君子居之者,何谓陋邪?''谓修君子之道自容乎?谓以君子之道教之也?如修君子之道苟自容,中国亦可,何必之夷狄?如以君子之道教之,夷狄安可教乎?禹入裸国,裸入衣出,衣服之制不通於夷狄也。禹不能教裸国衣服,孔子何能使九夷为君子?或:``孔子实不欲往,患道不行,动发此言。或人难之,孔子知其陋,然而犹曰`何陋之有'者,欲遂已然,距或人之谏也。''

实不欲往,志动发言,是伪言也。君子於言无所苟矣。如知其陋,苟欲自遂,此子路对孔子以子羔也。子路使子羔为费宰,子曰:``贼夫人之子。''子路曰:
``有社稷焉,有民人焉,何必读书,然後为学?''子曰:``是故恶夫佞者。''子路知其不可,苟欲自遂,孔子恶之,比夫佞者。孔子亦知其不可,苟应或人。孔子、子路皆以佞也。

孔子曰:``赐不受命而货殖焉,亿则屡中。''何谓不受命乎?说曰:受当富之命,自以术知数亿中时也。

夫人富贵,在天命乎?在人知也?如在天命,知术求之不能得;如在人,孔子何为言``死生有命,富贵在天''?夫谓富不受命,而自知术得之,贵亦可不受命,而自以努力求之。世无不受贵命而自得贵,亦知无不受富命而自得富得者。成事,孔子不得富贵矣,周流应聘,行说诸侯,智穷策困,还定《诗》、《书》,望绝无翼,称``已矣夫''自知无贵命,周流无补益也。孔子知己不受贵命,周流求之不能得,而谓赐不受富命,而以术知得富,言行相违,未晓其故。

或曰:``欲攻子贡之短也。子贡不好道德而徒好货殖,故攻其短,欲令穷服而更其行节。''夫攻子贡之短,可言赐不好道德而货殖焉,何必立不受命,与前言富贵在天相违反也?

颜渊死,子曰:``噫!天丧予!''此言人将起,天与之辅;人将废,天夺其佑。孔子有四友,欲因而起,颜渊早夭,故曰``天丧予''。

问曰:颜渊之死,孔子不王,天夺之邪?不幸短命自为死也?如短命不幸,不得不死,孔子虽王,犹不得生。辅之於人,犹杖之扶疾也。人有病,须杖而行;如斩杖本得短,可谓天使病人不得行乎?如能起行,杖短能使之长乎?夫颜渊之短命,犹杖之短度也。且孔子言``天丧予''者,以颜渊贤也。案贤者在世,未必为辅也。夫贤者未必为辅,犹圣人未必受命也。为帝有不圣,为辅有不贤。何则?禄命骨法,与才异也。由此言之,颜渊生未必为辅,其死未必有丧。孔子云``天丧予'',何据见哉?且天不使孔子王者,本意如何?本禀性命之时,不使之王邪,将使之王,复中悔之也?如本不使之王,颜渊死,何丧?如本使之王,复中悔之,此王无骨法,便宜自在天也。且本何善所见,而使之王?後何恶所闻,中悔不命?天神论议,误不谛也?

孔子之卫,遇旧馆人之丧,入而哭之。出使子贡脱骖而赙之。子贡曰:``於门人之丧,未有所脱骖。脱骖於旧馆,毋乃已重乎?''孔子曰:``予乡者入而哭之,遇於一哀而出涕,予恶夫涕之无从也,小子行之。''

孔子脱骖以赙旧馆者,恶情不副礼也。副情而行礼,情起而恩动,礼情相应,君子行之。颜渊死,子哭之恸。门人曰:``子恸矣。''``吾非斯人之恸而为?''
夫恸,哀之至也。哭颜渊恸者,殊之众徒,哀痛之甚也。死有棺无椁,颜路请车以为之椁,孔子不予,为大夫不可以徒行也。吊旧馆,脱骖以赙,恶涕无从;哭颜渊恸,请车不与,使恸无副。岂涕与恸殊,马与车异邪?於彼则礼情相副,於此则恩义不称,未晓孔子为礼之意。

孔子曰:``鲤也死,有棺无椁,吾不徒行以为之椁。''鲤之恩深於颜渊,鲤死无椁,大夫之仪,不可徒行也。鲤,子也;颜渊,他姓也。子死且不礼,况其礼他姓之人乎?

曰:是盖孔子实恩之效也。副情於旧馆,不称恩於子,岂以前为士,後为大夫哉?如前为士,士乘二马;如为大夫,大夫乘三马。大夫不可去车徒行,何不截卖两马以为椁,乘其一乎?为士时乘二马,截一以赙旧馆,今亦何不截其二以副恩,乘一以解不徒行乎?不脱马以赙旧馆,未必乱制。葬子有棺无椁,废礼伤法。孔子重赙旧人之恩,轻废葬子之礼。此礼得於他人,制失〔於〕亲子也。然则孔子不粥车以为鲤椁,何以解於贪官好仕恐无车?而自云``君子杀身以成仁'',何难退位以成礼?

子贡问政,子曰:``足食,足兵,民信之矣。''曰:``必不得已而去,於斯三者何先?''曰:``去兵。''曰:``必不得已而去,於斯二者何先?''曰:``去食。自古皆有死,民无信不立。''信最重也。

问:使治国无食,民饿,弃礼义礼义弃,信安所立?传曰:``仓禀实,知礼节;衣食足,知荣辱。''让生於有余,争生於不足。今言去食,信安得成?春秋之时,战国饥饿,易子而食析,骸而炊,口饥不食,不暇顾恩义也。夫父子之恩,信矣。饥饿弃信,以子为食。孔子教子贡去食存信,如何?夫去信存食,虽不欲信,信自生矣;去食存信,虽欲为信,信不立矣。

子适卫,冉子仆,子曰:``庶矣哉!''曰:``既庶矣,又何加焉?''曰:``
富之。''曰:``既富矣,又何加焉?''曰:``教之。''语冉子先富而後教之,教子贡去食而存信。食与富何别?信与教何异?二子殊教,所尚不同,孔子为国,意何定哉?

蘧伯玉使人於孔子,孔子曰:``夫子何为乎?''对曰:``夫子欲寡其过而未能也。''使者出,孔子曰:``使乎!使乎!''非之也。说《论语》者,曰:``非之者,非其代人谦也。''

夫孔子之问使者曰:``夫子何为'',问所治为,非问操行也。如孔子之问也,使者宜对曰``夫子为某事,治某政'',今反言``欲寡其过而未能也'',何以知其对失指,孔子非之也?且实孔子何以非使者?非其代人谦之乎?其非乎对失指也?所非犹有一实,不明其过,而徒云``使乎使乎!''後世疑惑,不知使者所以为过。韩子曰:``书约则弟子辨。''孔子之言``使乎'',何其约也?

或曰:``《春秋》之义也,为贤者讳。蘧伯玉贤,故讳其使者。''夫欲知其子视其友,欲知其君,视其所使。伯玉不贤,故所使过也。《春秋》之义,为贤者讳,亦贬纤介之恶。今不非而讳,贬纤介安所施哉?使孔子为伯玉讳,宜默而已。扬言曰``使乎!使乎!'',时人皆知孔子之非也。出言如此,何益於讳?

佛肸召,子欲往。子路不说,曰:``昔者,由也闻诸夫子曰:`亲於其身为不善者,君子不入也。'佛肸以中牟畔,子之往也如之何?''子曰:``有是〔言〕也。不曰坚乎?磨而不磷;不曰白乎?涅而不淄。吾岂匏瓜也哉,焉能系而不食也?''

子路引孔子往时所言以非孔子也。往前孔子出此言,欲令弟子法而行之,子路引之以谏,孔子晓之,不曰``前言戏'',若非而不可行,而曰``有是言''者,审有当行之也。``不曰坚乎?磨而不磷;不曰白乎?涅而不淄'',孔子言此言者,能解子路难乎?``亲於其身为不善者,君子不入也'',解之,宜〔曰〕:佛肸未为不善,尚犹可入。而曰``坚磨而不磷,白涅而不淄''。如孔子之言,有坚白之行者可以入之,君子之行软而易污邪,何以独不入也?孔子不饮盗泉之水,曾子不入胜母之闾,避恶去污,不以义耻辱名也。盗泉、胜母有空名,而孔、曾耻之;佛肸有恶实,而子欲往。不饮盗泉是,则欲对佛肸非矣。``不义而富且贵,於我如浮云'',枉道食篡畔之禄,所谓``浮云''者非也?或:``权时欲行道也即权时行道,子路难之,当云``行道'',不〔当〕言食。有权时以行道,无权时以求食。
``吾岂匏瓜也哉,焉能系而不食''?自比以匏瓜者,言人当仕而食禄。我非匏瓜系而不食,非子路也。孔子之言,不解子路之难。子路难孔子,岂孔子不当仕也哉?当择善国而入之也。孔子自比匏瓜,孔子欲安食也。且孔之言,何其鄙也!何彼仕为食哉?君子不宜言也。匏瓜系而不食,亦系而不仕等也。距子路可云:
``吾岂匏瓜也哉,系而不仕也''?今吾``系而不食'',孔子之仕,不为行道,徒求食也。人之仕也,主贪禄也。礼义之言,为行道也。犹人之娶也,主为欲也,礼义之言,为供亲也。仕而直言食,娶可直言欲乎?孔子之言,解情而无依违之意,不假义理之名,是则俗人,非君子也。儒者说孔子周流应聘不济,闵道不行,失孔子情矣。

公山弗扰以费畔,召,子欲往。子路曰:``未如也已,何必公山氏之之也?''
子曰:``夫召我者,而岂徒哉?如用我,吾其为东周乎。''

为东周,欲行道也。公山、佛肸俱畔者,行道於公山,求食於佛肸,孔子之言无定趋也。言无定趋,则行无常务矣。周流不用,岂独有以乎?阳货欲见之,不见;呼之仕,不仕,何其清也?公山、佛肸召之欲往,何其浊也?公山不扰与阳虎俱畔,执季桓子,二人同恶,呼召礼等。独对公山,不见阳虎,岂公山尚可,阳虎不可乎?子路难公山之〔召〕,孔子宜解以尚及佛肸未甚恶之状也。

\hypertarget{header-n425}{%
\subsection{卷十}\label{header-n425}}

\hypertarget{header-n426}{%
\subsubsection{非韩篇}\label{header-n426}}

韩子之术,明法尚功。贤无益於国不加赏;不肖无害於治不施罚。责功重赏,任刑用诛。故其论儒也,谓之不耕而食,比之於一蠹;论有益与无益也,比之於鹿马。马之似鹿者千金,天下有千金之马,无千金之鹿,鹿无益,马有用也。儒者犹鹿,有用之吏犹马也。

夫韩子知以鹿马喻,不知以冠履譬。使韩子不冠,徒履而朝,吾将听其言也。加冠於首而立於朝,受无益之服,增无益之〔行〕,言与服相违,行与术相反,吾是以非其言而不用其法也。烦劳人体,无益於人身,莫过跪拜。使韩子逢人不拜,见君父不谒,未必有贼於身体也。然须拜谒以尊亲者,礼义至重,不可失也。故礼义在身,身未必肥;而礼义去身,身未必瘠而化衰。以谓有益,礼义不如饮食。使韩子赐食君父之前,不拜而用,肯为之乎?夫拜谒,礼义之效,非益身之实也,然而韩子终不失者,不废礼义以苟益也。夫儒生,礼义也;耕战,饮食也。贵耕战而贱儒生,是弃礼义求饮食也。使礼义废,纲纪败,上下乱而阴阳缪,水旱失时,五谷不登,万民饥死,农不得耕,士不得战也。子贡去告朔之饩羊,孔子曰:``赐也!尔爱其羊,我爱其礼。''子贡恶费羊,孔子重废礼也。故以旧防为无益而去之,必有水灾;以旧礼为无补而去之,必有乱患。

儒者之在世,礼义之旧防也,有之无益,无之有损。庠序之设,自古有之。重本尊始,故立官置吏。官不可废,道不可弃。儒生,道官之吏也,以为无益而废之,是弃道也。夫道无成效於人,成效者须道而成。然足蹈路而行,所蹈之路,须不蹈者。身须手足而动,待不动者。故事或无益,而益者须之;无效,而效者待之。儒生,耕战所须待也,弃而不存,如何也?

韩子非儒,谓之无益有损,盖谓俗儒无行操,举措不重礼,以儒名而俗行,以实学而伪说,贪官尊荣,故不足贵。夫志洁行显,不徇爵禄,去卿相之位若脱躧者,居位治职,功虽不立,此礼义为业者也。国之所以存者,礼义也。民无礼义,倾国危主。今儒者之操,重礼爱义,率无礼义士,激无义之人。人民为善,爱其主上,此亦有益也。闻伯夷风者,贪夫廉,懦夫有立志;闻柳下惠风者,薄夫敦,鄙夫宽。此上化也,非人所见。段干木阖门不出,魏文敬之,表式其闾,秦军闻之,卒不攻魏。使魏无干木,秦兵入境,境土危亡。秦,强国也,兵无不胜,兵加於魏,魏国必破,三军兵顿,流血千里。今魏文式阖门之士,却强秦之兵,全魏国之境,济三军之众,功莫大焉,赏莫先焉。齐有高节之士,曰狂谲、华士,二人昆弟也,义不降志,不仕非其主。太公封於齐,以此二子解沮齐众,开不为上用之路,同时诛之。韩子善之,以为二子无益而有损也。夫狂谲、华士,段干木之类也,太公诛之,无所却到;魏文侯式之,却强秦而全魏。功孰大者?使韩子善干木阖门高节,魏文式之,是也;狂谲、华士之操,干木之节也,善太公诛之,非也。使韩子非干木之行,下魏文之式,则干木以此行而有益,魏文用式之道为有功;是韩子不赏功尊有益也。

论者或曰:``魏文式段干木之闾,秦兵为之不至,非法度之功;一功特然,不可常行,虽全国有益,非所贵也。''夫法度之功者,谓何等也?养三军之士,明赏罚之命,严刑峻法,富国强兵,此法度也。案秦之强,肯为此乎?六国之亡,皆灭於秦兵。六国之兵非不锐,士众之力非不劲也,然而不胜,至於破亡者,强弱不敌,众寡不同,虽明法度,其何益哉?使童子变孟贲之意,孟贲怒之,童子操刃与孟贲战,童子必不胜,力不如也。孟贲怒,而童子修礼尽敬,孟贲不忍犯也。秦之与魏,孟贲之与童子也。魏有法度,秦必不畏,犹童子操刃,孟贲不避也。其尊士式贤者之闾,非徒童子修礼尽敬也。夫力少则修德,兵强则奋威。秦以兵强,威无不胜,却军还众,不犯魏境者,贤干木之操,高魏文之礼也。夫敬贤,弱国之法度,力少之强助也。谓之非法度之功,如何?

高皇帝议欲废太子,吕后患之,即召张子房而取策。子房教以敬迎四皓而厚礼之,高祖见之,心消意沮,太子遂安。使韩子为吕后议,进不过强谏,退不过劲力。以此自安,取诛之道也,岂徒易哉?夫太子敬厚四皓以消高帝之议,犹魏文式段干木之闾,却强秦之兵也。

治国之道,所养有二:一曰养德,二曰养力。养德者,养名高之人,以示能敬贤;养力者,养气力之士,以明能用兵。此所谓文武张设,德力具足者也,事或可以德怀,或可以力摧。外以德自立,内以力自备。慕德者不战而服,犯德者畏兵而却。徐偃王修行仁义,陆地朝者三十二国,强楚闻之,举兵而灭之。此有德守,无力备者也。夫德不可独任以治国,力不可直任以御敌也。韩子之术不养德,偃王之操不任力。二者偏驳,各有不足。偃王有无力之祸,知韩子必有无德之患。凡人禀性也,清浊贪廉,各有操行,犹草木异质,不可复变易也。狂谲、华士不仕於齐,犹段干木不仕於魏矣。性行清廉,不贪富贵,非时疾世,义不苟仕,虽不诛此人,此人行不可随也。太公诛之,韩子是之,是谓人无性行,草木无质也。太公诛二子,使齐有二子之类,必不为二子见诛之故,不清其身;使无二子之类,虽养之,终无其化。尧不诛许由,唐民不皆樔处;武王不诛伯夷,周民不皆隐饿;魏文侯式段干木之闾,魏国不皆阖门。由此言之,太公不诛二子,齐国亦不皆不仕。何则?清廉之行,人所不能为也。夫人所不能为,养使为之,不能使劝;人所能为,诛以禁之,不能使止。然则太公诛二子,无益於化,空杀无辜之民。赏无功,杀无辜,韩子所非也。太公杀无辜,韩子是之,以韩子之术杀无辜也。夫执不仕者,未必有正罪也,太公诛之。如出仕未有功,太公肯赏之乎?赏须功而加,罚待罪而施。使太公不赏出仕未有功之人,则其诛不仕未有罪之民,非也;而韩子是之,失误之言也。

且不仕之民,性廉寡欲;好仕之民,性贪多利。利欲不存於心,则视爵禄犹粪土矣。廉则约省无极,贪则奢泰不止;奢泰不止,则其所欲不避其主。案古篡畔之臣,希清白廉洁之人。贪,故能立功;骄,故能轻生。积功以取大赏,奢泰以贪主位。太公遗此法而去,故齐有陈氏劫杀之患。太公之术,致劫杀之法也;韩子善之,是韩子之术亦危亡也。

周公闻太公诛二子,非而不是,然而身执贽以下白屋之士。白屋之士,二子之类也,周公礼之,太公诛之,二子之操,孰为是者?宋人有御马者不进,拔剑刭而弃之於沟中;又驾一马,马又不进,又刭而弃之於沟。若是者三。以此威马,至矣,然非王良之法也。王良登车,马无罢驽。尧、舜治世,民无狂悖。王良驯马之心,尧、舜顺民之意。人同性,马殊类也。王良能调殊类之马,太公不能率同性之士。然则周公之所下白屋,王良之驯马也;太公之诛二子,宋人之刭马也。举王良之法与宋人之操,使韩子平之,韩子必是王良而非宋人矣。王良全马,宋人贼马也。马之贼,则不若其全;然则,民之死,不若其生。使韩子非王良,自同於宋人,贼善人矣。如非宋人,宋人之术与太公同。非宋人,是太公,韩子好恶无定矣。

治国犹治身也。治一身,省恩德之行,多伤害之操,则交党疏绝,耻辱至身。推治身以况治国,治国之道当任德也。韩子任刑独以治世,是则治身之人任伤害也。韩子岂不知任德之为善哉?以为世衰事变,民心靡薄,故作法术,专意於刑也。夫世不乏於德,犹岁不绝於春也。谓世衰难以德治,可谓岁乱不可以春生乎?人君治一国,犹天地生万物。天地不为乱岁去春,人君不以衰世屏德。孔子曰:
``斯民也,三代所以直道而行也。''

周穆王之世,可谓衰矣,任刑治政,乱而无功。甫侯谏之,穆王存德,享国久长,功传於世。夫穆王之治,初乱终治,非知昏於前,才妙於後也,前任蚩尤之刑,後用甫侯之言也。夫治人不能舍恩,治国不能废德,治物不能去春。韩子欲独任刑用诛,如何?

鲁缪公问於子思曰:``吾闻庞扪是子不孝,不孝其行奚如?''子思对曰:``
君子尊贤以崇德,举善以劝民。若夫过行,是细人之所识也,臣不知也。''子思出,子服厉伯见。君问庞是子,子服厉伯对以其过,皆君〔之〕所未曾闻。自是之後,君贵子思而贱子服厉伯。韩子闻之,以非缪公,以为明君求奸而诛之,子思不以奸闻,而厉伯以奸对,厉伯宜贵,子思宜贱。今缪公贵子思,贱厉伯,失贵贱之宜,故非之也。

夫韩子所尚者,法度也。人为善,法度赏之;恶,法度罚之。虽不闻善恶於外,善恶有所制矣。夫闻恶不可以行罚,犹闻善不可以行赏也。非人不举奸者,非韩子之术也。使韩子闻善,必将试之;试之有功,乃肯赏之。夫闻善不辄加赏,虚言未必可信也。若此,闻善与不闻,无以异也。夫闻善不辄赏,则闻恶不辄罚矣。闻善必试之,闻恶必考之。试有功乃加赏,考有验乃加罚。虚闻空见,实试未立,赏罚未加。赏罚未加,善恶未定,未定之事,须术乃立,则欲耳闻之,非也。

郑子产晨出,过东匠之宫,闻妇人之哭也,抚其仆之手而听之。有间,使吏执而问之;手杀其夫者也。翼日,其仆问曰:``夫子何以知之?''子产曰:``其声不恸。凡人於其所亲爱也,知病而忧,临死而惧,已死而哀。今哭夫已死,不哀而惧,是以知其有奸也。''韩子闻而非之曰:``子产不亦多事乎?奸必待耳目之所及而後知之,则郑国之得奸寡矣。不任典城之吏,察参伍之正,不明度量,待尽聪明、劳知虑而以知奸,不亦无术乎!''韩子之非子产,是也。其非缪公,非也。夫妇人之不哀,犹庞〔是〕子不孝也。非子产持耳目以知奸,独欲缪公须问以定邪。子产不任典城之吏,而以耳〔闻〕定实;缪公亦不任吏,而以口问立诚。夫耳闻口问,一实也,俱不任吏,皆不参伍。厉伯之对不可以立实,犹妇人之哭不可以定诚矣。不可定诚,使吏执而问之。不可以立实,不使吏考,独信厉伯口,以罪不考之奸,如何?

韩子曰:``子思不以过闻,缪公贵之。子服厉伯以奸闻,缪公贱之。人情皆喜贵而恶贱,故季氏之乱成而不上闻。此鲁君之所以劫也。''夫鲁君所以劫者,以不明法度邪,以不早闻奸也?夫法度明,虽不闻奸,奸无由生;法度不明,虽日求奸,决其源鄣之以掌也。御者无衔,见马且奔,无以制也。使王良持辔,马无欲奔之心,御之有数也。今不言鲁君无术,而曰``不闻奸'';不言〔不〕审法度,而曰``不通下情'',韩子之非缪公也,与术意而相违矣。

庞扪是子不孝,子思不言,缪公贵之。韩子非之,以为明君求善而赏之,求奸而诛之。夫不孝之人,下愚之才也。下愚无礼,顺情从欲,与鸟兽同,谓之恶,可也,谓奸,非也。奸人外善内恶,色厉内荏,作为操止象类贤行,以取升进,容媚於上,安肯作不孝、著身为恶,以取弃殉之咎乎?庞扪是子可谓不孝,不可谓奸。韩子谓之奸,失奸之实矣。

韩子曰:``布帛寻常,庸人不择;烁金百镒,盗跖不搏。''以此言之,法明,民不敢犯也。设明法於邦,有盗贼之心,不敢犯矣;不测之者,不敢发矣。奸心藏於胸中,不敢以犯罪法,罪法恐之也。明法恐之,则不须考奸求邪於下矣。使法峻,民无奸者;使法不峻,民多为奸。而不言明王之严刑峻法,而云求奸而诛之。言求奸,是法不峻,民或犯之也。世不专意於明法,而专心求奸。韩子之言,与法相违。

人之释沟渠也,知者必溺身。不塞沟渠而缮船楫者,知水之性不可阏,其势必溺人也。臣子之性欲奸君父,犹水之性溺人也。不教所以防奸,而非其不闻知,是犹不备水之具,而徒欲早知水之溺人也。溺於水,不责水而咎己者,己失防备也。然则人君劫於臣,己失法也。备溺不阏水源,防劫不求臣奸,韩子所宜用教己也。水之性胜火,如裹之以釜,水煎而不得胜,必矣。夫君犹火也,臣犹水也,法度釜也。火不求水之奸,君亦不宜求臣之罪也。

\hypertarget{header-n447}{%
\subsubsection{刺孟篇}\label{header-n447}}

孟子见梁惠王,王曰:``叟!不远千里而来,将何以利吾国乎?''孟子曰:
``仁义而已,何必曰利。''

夫利有二:有货财之利,有安吉之利。惠王曰``何以利吾国''?何以知不欲安吉之利,而孟於径难以货财之利也?《易》曰:``利见大人'',``利涉大川'',
``《乾》,元享利贞''。《尚书》曰:``黎民亦尚有利哉?''皆安吉之利也。行仁义,得安吉之利。孟子必〕且语问惠王:``何谓`利吾国''',惠王言货财之利,乃可答若设。令惠王之问未知何趣,孟子径答以货财之利。如惠王实问货财,孟子无以验效也;如问安吉之利,而孟子答以货财之利,失对上之指,违道理之实也。

齐王问时子:``我欲中国而授孟子室,养弟子以万钟,使诸大夫、国人皆有所矜式。子盍为我言之?''时子因陈子而以告孟子。孟子曰:``夫时子恶知其不可也?如使予欲富,辞十万而受万,是为欲富乎?''

夫孟子辞十万,失谦让之理也。夫富贵者,人之所欲也,不以其道得之,不居也。故君子之於爵禄也,有所辞,有所不辞。岂以己不贪富贵之故,而以距逆宜当受之赐乎?

陈臻问曰:``於齐,王馈兼金一百镒而不受;於宋,归七十镒而受;於薛,归五十镒而受取。前日之不受是,则今受之非也。今日之受是,则前日之不受非也。夫子必居一於此矣。''孟子曰:``皆是也。当在宋也,予将有远行,行者必以赆,辞曰归赆,予何为不受?当在薛也,予有戒心,辞曰`闻戒,故为兵戒归之备乎!'予何为不受?若於齐,则未有处也,无处而归之,是货之也,焉有君子而可以货取乎?''

夫金归或受或不受,皆有故。非受之时已贪,当不受之时己不贪也。金有受不受之义,而室亦宜有受不受之理。今不曰``己无功'',若``己致仕,受室非理,
''而曰``己不贪富'',引前辞十万以况後万。前当受十万之多,安得辞之?

彭更问曰:``後车数十乘,从者数百人,以传食於诸侯,不亦泰乎?''孟子曰:``非其道,则一箪食而不可受於人;如其道,则舜受尧之天下,不以为泰。''

受尧天下,孰与十万?舜不辞天下者,是其道也。今不曰受十万非其道,而曰己不贪富贵,失谦让也。安可以为戒乎?

沈同以其私问曰:``燕可伐与?''孟子曰:``可。子哙不得与人燕,子之不得受燕於子哙。有士於此,而子悦之,不告於王,而私与之子之爵禄。夫士也,亦无王命而私受之,於子,则可乎?何以异於是。''齐人伐燕,或问曰:``劝齐伐燕,有诸?''曰:``未也。沈同曰:`燕可伐与?'吾应之曰:`可。'彼然而伐之。如曰:`孰可以伐之?'则应之曰:`为天吏则可以伐之。'今有杀人者,或问之曰:`人可杀与?'则将应之曰:`可。'彼如曰:`孰可以杀之?'
则应之曰:``为士师则可以杀之。''今以燕伐燕,何为劝之也?''

夫或问孟子劝王伐燕,不诚是乎?沈同问``燕可伐与'',此挟私意欲自伐之也。知其意慊於是,宜曰:``燕虽可伐,须为天吏,乃可以伐之。''沈同意绝,则无伐燕之计矣。不知有此私意而径应之,不省其语,是不知言也。

公孙丑问曰:``敢问夫子恶乎长?''孟子曰:``我知言。''又问:``何谓知言?''曰:``诐辞知其所蔽,淫辞知其所陷,邪辞知其所离,遁辞知其所穷。生於其心,害於其政,发於其政;害於其事。虽圣人复起,必从吾言矣。''孟子知言者也,又知言之所起之祸,其极所致之〔害〕,见彼之问,则知其措辞所欲之矣。知其所之,则知其极所当害矣。

孟子有云:``民举安,王庶几改诸!予日望之。''孟子所去之王,岂前所不朝之王哉?而是,何其前轻之疾而後重之甚也?如非是前王,则不去,而於後去之,是後王不肖甚於前;而去三日宿,於前不甚,不朝而宿於景丑氏。何孟子之操,前後不同?所以为王,终始不一也?

且孟子在鲁,鲁平公欲见之。嬖人臧仓毁孟子,止平公。乐正子以告。曰:
``行,或使之;止,或尼之。行止非人所能也。予之不遇鲁侯,天也!''前不遇於鲁,後不遇於齐,无以异也。前归之天,今则归之於王。孟子论称竟何定哉?夫不行於齐,王不用,则若臧仓之徒毁谗之也。此亦止或尼之也,皆天命不遇,非人所能也。去,何以不径行而留三宿乎?天命不当遇於齐,王不用其言,天岂为三日之间易命使之遇乎?在鲁则归之於天,绝意无冀;在齐则归之於王,庶几有望。夫如是,不遇之议一在人也。

或曰:初去,未可以定天命也。冀三日之间,王复追之,天命或时在三日之间故可也。夫言如是,齐王初使之去者,非天命乎?如使天命在三日之间,鲁平公比三日亦时弃臧仓之议,更用乐正子之言,往见孟子,孟子归之於天,何其早乎?如三日之间,公见孟子,孟子奈前言何乎?

孟子去齐,充虞涂问曰:``夫子若不豫色然。前日,虞闻诸夫子曰:`君子不怨天,不尤人。'曰:``彼一时也,此一时也。五百年必有王者兴,其间必有名世者矣。由周以来,七百有余岁矣,以其数则过矣,以其时考之,则可矣。夫天未欲平治天下乎?如欲平治天下,当今之世,舍我而谁也?吾何为不豫哉?''

夫孟子言五百年有王者兴,何以见乎?帝喾王者,而尧又王天下;尧传於舜,舜又王天下;舜传於禹,禹又王天下。四圣之王天下也,断踵而兴。禹至汤且千岁,汤至周亦然,始於文王,而卒传於武王。武王崩,成王、周公共治天下。由周至孟子之时,又七百岁而无王者。五百岁必有王者之验,在何世乎?云``五百岁必有王者'',谁所言乎?论不实事考验,信浮淫之语;不遇去齐,有不豫之色;非孟子之贤效与俗儒无殊之验也?

``五百年''者,以为天出圣期也,又言以``天未欲平治天下也'',其意以为天欲平治天下,当以五百年之间生圣王也。如孟子之言,是谓天故生圣人也。然则五百岁者,天生圣人之期乎?如是其期,天何不生圣?圣王非其期故不生。孟子犹信之,孟子不知天也。

``自周已来,七百余岁矣,以其数则过矣;以其时考之,则可矣。''何谓数过?何谓``时可''乎?数则时,时则数矣。``数过'',过五百年也。从周到今七百余岁,逾二百岁矣。设或王者,生失时矣,又言``时可'',何谓也?云``五百年必有王者兴'',又言``其间必有名世'',与王者同乎?异也?如同,为再言之?如异,``名世''者,谓何等也?谓孔子之徒、孟子之辈,教授後生,觉悟顽愚乎?已有孔子,己又以生矣。如谓圣臣乎?当与圣〔王〕同时。圣王出,圣臣见矣。言五百年而已,何为言其间?如不谓五百年时,谓其中间乎?是谓二三百年之时也。〔人〕不与五百年时圣王相得。夫如是,孟子言其间必有名世者,竟谓谁也?
``夫天未欲平治天下也。如欲治天下,舍予而谁也?''言若此者,不自谓当为王者,有王者,若为王臣矣。为王者臣,皆天也。己命不当平治天下,不浩然安之於齐,怀恨有不豫之色,失之矣。

彭更问曰:``士无事而食,可乎?''孟子曰:``不通功易事,以羡补不足,则农有余粟,女有余布。子如通之,则梓匠轮舆,皆得食於子。於此有人焉,入则孝,出则悌,守先王之道,以待後世之学者,而不得食於子。子何尊梓匠轮舆,而轻为仁义者哉?''曰:``梓匠轮舆,其志将以求食也。君子之为道也,其志亦将以求食与?''孟子曰:``子何以其志为哉?其有功於子,可食而食之矣。且子食志乎?食功乎?''曰:``食志。''曰:``有人於此,毁瓦画墁,其志将以求食也,则子食之乎?''曰:``否。''曰:``然则子非食志,食功也。''

夫孟子引毁瓦画墁者,欲以诘彭更之言也。知毁瓦画墁无功而有志,彭更必不食也。虽然,引毁瓦画墁,非所以诘彭更也。何则?诸志欲求食者,毁瓦画墁者不在其中。不在其中,则难以诘人矣。夫人无故毁瓦画墁,此不痴狂则遨戏也。痴狂人之,志不求食,遨戏之人,亦不求食。求食者,皆多人所〔共〕得利之事,以作此鬻卖於市,得贾以归,乃得食焉。今毁瓦画墁,无利於人,何志之有?有知之人,知其无利,固不为也;无知之人,与痴狂比,固无其志。夫毁瓦画墁,犹比童子击壤於涂,何以异哉?击壤於涂者,其志亦欲求食乎?此尚童子,未有志也。巨人博戏,亦画墁之类也。博戏之人,其志复求食乎?博戏者尚有相夺钱财,钱财众多,己亦得食,或时有志。夫投石超距,亦画墁之类也。投石超距之人,其志有求食者乎?然则孟子之诘彭更也,未为尽之也。如彭更以孟子之言,可谓御人以口给矣。

匡章子曰:``陈仲子岂不诚廉士乎?居於於陵,三日不食,耳无闻、目无见也。井上有李,螬食实者过半,扶服往,将食之。三咽,然後耳有闻,目有见也。
''孟子曰:``於齐国之士,吾必以仲子为巨擘焉!虽然,仲子恶能廉?充仲子之操,则蚓而後可者也。夫蚓,上食槁壤,下饮黄泉。仲子之所居室,伯夷之所筑与?抑亦盗跖之所筑与?所食之粟,伯夷之所树与,抑亦盗跖之所树与?是未可知也。''曰:``是何伤哉?彼身织屦,妻辟纑,以易之也。''曰:``仲子,齐之世家,兄戴,盖禄万锺。以兄之禄为不义之禄,而不食也。以兄之室为不义之室,而弗居也。辟兄离母,处於於陵。他日归,则有馈其兄生鹅者也,己频蹙曰:恶用是鶂鶂者为哉?他日,其母杀是鹅也,与之食之。其兄自外至,曰:是鶂鶂之肉也。出而吐之。以母则不食,以妻则食之;以兄之室则不居,以於陵则居之。是尚能为充其类也乎?若仲子者,蚓而後充其操者也。''

夫孟子之非仲子也,不得仲子之短矣。仲子之怪鹅如吐之者,岂为在母不食乎?乃先谴鹅曰:``恶用鶂鶂者为哉?''他日,其母杀以食之,其兄曰:``是鶂鶂之肉。''仲子耻负前言,即吐而出之。而兄不告,则不吐;不吐,则是食於母也。谓之``在母则不食'',失其意矣。使仲子执不食於母,鹅膳至,不当食也。今既食之,知其为鹅,怪而吐之。故仲子之吐鹅也,耻食不合己志之物也,非负亲亲之恩,而欲勿母食也。

又``仲子恶能廉?充仲子之性,则蚓而後可者也。夫蚓,上食槁壤,下饮黄泉'',是谓蚓为至廉也。仲子如蚓,乃为廉洁耳。今所居之宅,伯夷之所筑;所食之粟,伯夷之所树。仲子居而食之,於廉洁可也。或时食盗跖之所树粟,居盗跖之所筑室,污廉洁之行矣。用此非仲子,亦复失之。室因人故,粟以屦纑易之,正使盗之所树筑,己不闻知。今兄之不义,有其操矣。操见於众,昭晰议论,故避於陵,不处其宅,织屦辟纑,不食其禄也。而欲使仲子处於陵之地,避若兄之宅,吐若兄之禄,耳闻目见,昭晰不疑,仲子不处不食,明矣。今於陵之宅,不见筑者为谁,粟,不知树者为谁,何得成室而居之?得成粟而食之?孟子非之,是为太备矣。仲子所居,或时盗之所筑,仲子不知而居之,谓之不充其操,唯蚓然後可者也。夫盗室之地中,亦有蚓焉,食盗宅中之槁壤,饮盗宅中之黄泉,蚓恶能为可乎?在仲子之操,满孟子之议,鱼然後乃可。夫鱼处江海之中,食江海之士,海非盗所凿,士非盗所聚也。

然则仲子有大非,孟子非之,不能得也。夫仲子之去母辟兄,与妻独处於陵,以兄之宅为不义之宅,以兄之禄为不义之禄,故不处不食,廉洁之至也,然则其徒于陵归候母也,宜自赍食而行。鹅膳之进也,必与饭俱。母之所为饭者,兄之禄也。母不自有私粟。以食仲子,明矣。仲子食兄禄也。伯夷不食周粟。饿死於首阳之下,岂一食周粟而以污其洁行哉?仲子之操,近不若伯夷,而孟子谓之若蚓乃可,失仲子之操所当比矣。

孟子曰:``莫非天命也,顺受其正。是故知命者,不立乎岩墙之下。''尽其道而死者,为正命也;桎梏而死者,非天命也。

夫孟子之言,是谓人无触值之命也。顺操行者得正命,妄行苟为得非正〔命〕,是天命於操行也。夫子不王,颜渊早夭,子夏失明,伯牛为疠。四者行不顺与?何以不受正命?比干剖,子胥烹,子路菹,天下极戮,非徒桎梏也。必以桎梏效非正命,则比干、子胥行不顺也。人禀性命,或当压溺兵烧,虽或慎操修行,其何益哉?窦广国与百人俱卧积炭之下,炭崩,百人皆死,广国独济,命当封侯也。积炭与岩墙何以异?命不压,虽岩崩,有广国之命者,犹将脱免。行,或使之;止,或尼之。命当压,犹或使之立於墙下。孔甲所入主人〔之〕子,天命当贱,虽载入宫,犹为守者。不立岩墙之下,与孔甲载子入宫,同一实也。

\hypertarget{header-n477}{%
\subsection{卷十一}\label{header-n477}}

\hypertarget{header-n478}{%
\subsubsection{谈天篇}\label{header-n478}}

儒书言:``共工与颛顼争为天子不胜,怒而触不周之山,使天柱折,地维绝。女娲销炼五色石以补苍天,断鰲足''以立四极。天不足西北,故日月移焉;地不足东南,故百川注焉。''此久远之文,世间是之言也。文雅之人,怪而无以非,若非而无以夺,又恐其实然,不敢正议。以天道人事论之,殆虚言也。

与人争为天子,不胜,怒触不周之山,使天柱折,地维绝,有力如此,天下无敌。以此之力,与三军战,则士卒蝼蚁也,兵革毫芒也,安得不胜之恨,怒触不周之山乎?且坚重莫如山,以万人之力,共推小山,不能动也。如不周之山,大山也,使是天柱乎,折之固难;使非柱乎?触不周山而使天柱折,是亦复难。信,颛顼与之争,举天下之兵,悉海内之众,不能当也,何不胜之有?且夫天者,气邪?体也?如气乎,云烟无异,安得柱而折之?女娲以石补之,是体也。如审然,天乃玉石之类也。石之质重,千里一柱,不能胜也。如五岳之巅,不能上极天乃为柱。如触不周,上极天乎?不周为共工所折,当此之时,天毁坏也。如审毁坏,何用举之?``断鰲之足,以立四极,''说者曰:``鳖,古之大兽也,四足长大,故断其足,以立四极。''夫不周,山也;鰲,兽也。夫天本以山为柱,共工折之,代以兽足,骨有腐朽,何能立之久?且鰲足可以柱天,体必长大,不容於天地,女娲虽圣,何能杀之?如能杀之,杀之何用?足可以柱天,则皮革如铁石,刀剑矛戟不能刺之,强弩利矢不能胜射也。

察当今天去地甚高,古天与今无异。当共工缺天之时,天非坠於地也。女娲,人也,人虽长,无及天者。夫其补天之时,何登缘阶据而得治之?岂古之天,若屋庑之形,去人不远,故共工得败之,女娲得补之乎?如审然者,女娲〔已〕前,齿为人者,人皇最先。人皇之时,天如盖乎?说《易》者曰:``元气未分,浑沌为一。''儒书又言:溟涬濛澒,气未分之类也。及其分离,清者为天,浊者为地。如说《易》之家、儒书之言,天地始分,形体尚小,相去近也。近则或枕於不周之山,共工得折之,女娲得补之也。含气之类,无有不长。天地,含气之自然也,从始立以来,年岁甚多,则天地相去,广狭远近,不可复计。儒书之言,殆有所见。然其言触不周山而折天柱,绝地维,消炼五石补苍天,断鰲之足以立四极,犹为虚也。何则?山虽动,共工之力不能折也。岂天地始分之时,山小而人反大乎?何以能触而折之?以五色石补天,尚可谓五石若药石治病之状。至其断鰲之足以立四极,难论言也。从女娲以来久矣,四极之立自若,鰲之足乎?

邹衍之书,言天下有九州,《禹贡》之上所谓九州也;《禹贡》九州,所谓一州也,若《禹贡》以上者九焉。《禹贡》九州,方今天下九州也,在东南隅,名曰赤县神州。复更有八州。每一州者四海环之,名曰裨海。九州之外,更有瀛海。此言诡异,闻者惊骇,然亦不能实然否,相随观读讽述以谈。故虚实之事,并传世间,真伪不别也。世人惑焉,是以难论。

案邹子之知不过禹。禹之治洪水,以益为佐。禹主治水,益〔主〕记物。极天之广,穷地之长,辨四海之外,竟四山之表,三十五国之地,鸟兽草木、金石水土,莫不毕载,不言复有九州。淮南王刘安,召术士伍被、左吴之辈,充满宫殿,作道术之书,论天下之事。《地形》之篇,道异类之物,外国之怪,列三十五国之异,不言更有九州。邹子行地不若禹、益,闻见不过被、吴,才非圣人,事非天授,安得此言?案禹之《山经》、淮南之《地形》,以察邹子之书,虚妄之言也。太史公曰:``《禹本纪》言河出昆仑,其高三千五百余里,日月所〔相〕辟隐为光明也,其上有玉泉、华池。今自张骞使大夏之後,穷河源,恶睹《本纪》所谓昆仑者乎?故言九州山川,《尚书》近之矣。至《禹本纪》、《山经》所有怪物,余不敢言也。''夫弗敢言者,谓之虚也。昆仑之高,玉泉、华池,世所共闻,张骞亲行无其实。案《禹贡》,九州山川,怪奇之物、金玉之珍,莫不悉载,不言昆仑山上有玉泉、华池。案太史公之言,《山经》、《禹纪》,虚妄之言。

凡事难知,是非难测。极为天中,方今天下,在极之南,则天极北,必高多民。《禹贡》``东渐於海,西被於流沙'',此则天地之极际也。日刺径千里,今从东海之上会,稽鄞、鄮,则察日之初出径二尺,尚远之验也。远则东方之地尚多。东方之地尚多,则天极之北,天地广长,不复訾矣。夫如是,邹衍之言未可非,《禹纪》、《山海》、《淮南地形》未可信也。邹衍曰:``方今天下,在地东南,名赤县神州。''天极为天中,如方今天下,在地东南,视极当在西北。今正在北,方今天下在极南也。以极言之,不在东南,邹衍之言非也。如在东南,近日所出,日如出时,其光宜大。今从东海上察日,及从流沙之地视日,小大同也。相去万里,小大不变,方今天下,得地之广,少矣。雒阳,九州之中也,从雒阳北顾,极正在北。东海之上,去雒阳三千里,视极亦在北。推此以度,从流沙之地视极,亦必复在北焉。东海、流沙,九州东西之际也,相去万里,视极犹在北者,地小居狭,未能辟离极也。日南之郡,去雒且万里。徙民还者,问之,言日中之时,所居之地,未能在日南也。度之复南万里,日在日〔南〕之南,是则去雒阳二万里,乃为日南也。今从雒地察日之去远近,非与极同也,极为远也。今欲北行三万里,未能至极下也。假令之至,是则名为距极下也。以至日南五万里,极北亦五万里也。极北亦五万里,极东西亦皆五万里焉。东西十万,南北十万,相承百万里。邹衍之言:``天地之间,有若天下者九。''案周时九州,东西五千里,南北亦五千里。五五二十五,一州者二万五千里。天下若此九之,乘二万五千里。二十二万五千里。如邹衍之书,若谓之多,计度验实,反为少焉。

儒者曰:``天,气也,故其去人不远。人有是非,阴为德害,天辄知之,又辄应之,近人之效也。''如实论之,天,体,非气也。人生於天,何嫌天无气?犹有体在上,与人相远。秘传或言:天之离天下,六万余里。数家计之,三百六十五度一周天。下有周度,高有里数。如天审气,气如云烟,安得里度?又以二十八宿效之,二十八宿为日月舍,犹地有邮亭为长吏廨矣。邮亭著地,亦如星舍著天也。案附书者,天有形体,所据不虚。〔由〕此考之,则无恍惚,明矣。

\hypertarget{header-n488}{%
\subsubsection{说日篇}\label{header-n488}}

儒者曰:``日朝见,出阴中;暮不见,入阴中。阴气晦冥,故没不见。''如实论之,不出入阴中。何以效之?夫夜,阴也,气亦晦冥,或夜举火者,光不灭焉。夜之阴,北方之阴也;朝出日,入所举之火也。火夜举,光不灭;日暮入,独不见,非气验也。夫观冬日之出入,朝出东南,暮入西南。东南、西南非阴,何故谓之出入阴中?且夫星小犹见,日大反灭,世儒之论,竟虚妄也。

儒者曰:``冬日短,夏日长,亦复以阴阳。夏时,阳气多,阴气少,阳气光明,与日同耀,故日出辄无鄣蔽。冬,阴气晦冥,掩日之光,日虽出,犹隐不见,故冬日日短,阴多阳少,与夏相反。''如实论之,日之长短,不以阴阳。何以验之?复以北方之星。北方之阴,日之阴也。北方之阴,不蔽星光,冬日之阴,何故〔独〕灭日明?由此言之,以阴阳说者,失其实矣。实者,夏时日在东井,冬时日在牵牛,牵牛去极远,故日道短,东井近极,故日道长。夏北至东井,冬南至牵牛,故冬夏节极,皆谓之至,春秋未至,故谓之分。或曰:``夏时阳气盛,阳气在南方,故天举而高;冬时阳气衰,天抑而下。高则日道多,故日长;下则日道少,故日短也。''夏日阳气盛,天南方举而日道长;月亦当复长。案夏日长之时,日出东北,而月出东南;冬日短之时,日出东南,月出东北。如夏时天举南方,日月当俱出东北,冬时天复下,日月亦当俱出东南。由此言之,夏时天不举南方,冬时天不抑下也。然则夏日之长也,其所出之星在北方也;冬日之短也,其所出之星在南方也。问曰:``当夏五月日长之时在东井,东井近极,故日道长。今案察五月之时,日出於寅,入於戌。日道长,去人远,何以得见其出於寅入於戌乎?''日东井之时,去人极近。夫东井近极,若极旋转,人常见之矣。使东井在极旁侧,得无夜常为昼乎?日昼行十六分,人常见之,不复出入焉。儒者或曰:
``日月有九道,故曰:``日行有近远,昼夜有长短也。''夫复五月之时,昼十一分,夜五分;六月,昼十分,夜六分;从六月往至十一月,月减一分:此则日行,月从一分道也,岁,日行天十六道也,岂徒九道?

或曰:``天高南方,下北方。日出高,故见;入下,故不见。天之居若倚盖矣,故极在人之北,是其效也。极其天下之中,今在人北,其若倚盖,明矣。''
日明既以倚盖喻,当若盖之形也。极星在上之北,若盖之葆矣;其下之南,有若盖之茎者,正何所乎?夫取盖倚於地不能运,立而树之,然後能转。今天运转,其北际不著地者,触碍何以能行?由此言之,天不若倚盖之状,日之出入不随天高下,明矣。或曰:``天北际下地中,日随天而入地,地密鄣隐,故人不见。''
然天地,夫妇也,合为一体。天在地中,地与天合,天地并气,故能生物。北方阴也,合体并气,故居北方。天运行於地中乎,不则,北方之地低下而不平也。如审运行地中,凿地一丈,转见水源,天行地中,出入水中乎,如北方低下不平,是则九川北注,不得盈满也。实者,天不在地中,日亦不随天隐,天平正,与地无异。然而日出上,日入下者,随天转运,视天若覆盆之状,故视日上下然,似若出入地中矣。然则日之出,近也;其入远,不复见,故谓之入,运见於东方近,故谓之出。何以验之?系明月之珠於车盖之,转而旋之,明月之珠旋邪?人望不过十里,天地合矣,远非合也。今视日入,非入也,亦远也。当日入西方之时,其下民亦将谓之日中。从日入之下,东望今之天下,或时亦天地合。如是方〔今〕天下在南方也,故日出於东方,入於北方之地,日出北方,入於南方。各於近者为出,远者为入。实者不入,远矣。临大泽之滨,望四边之际与天属;其实不属,远若属矣。日以远为入,泽以远为属,其实一也。泽际有陆,人望而不见,陆在,察之若〔亡〕,日亦在,视之若入,皆远之故也。太山之高,参天入云,去之百里,不见埵塊。夫去百里不见太山,况日去人以万里数乎?太山之验,则既明矣,试使一人把大炬火夜行於道,平易无险,去人不一里,火光灭矣,非灭也,远也。今日西转不复见者,非入也。问曰:``天平正与地无异,今仰视天,观日月之行,天高南方下北方,何也?''曰:方今天下在东南之上,视天若高,日月道在人之南,今天下在日月道下,故观日月之行,若高南下北也。何以验之?即天高南方,之星亦当高,今视南方之星低下,天复低南方乎?夫视天之居近者则高,远则下焉,极北方之民以为高,南方为下。极东极西,亦如此焉。皆以近者为高,远者为下。从北塞下,近仰视斗极,且在人上。匈奴之北,地之边陲,北上视天,天复高北下南,日月之道,亦在其上。立太山之上,太山高,去下十里,太山下。夫天之高下,犹人之察太山也。平正,四方中央高下皆同,今望天之四边若下者,非也,远也。非徒下,若合矣。

儒者或以旦暮日出入为近,日中为远;或以日中为近,日出入为远。其以日出入为近,日中为远者,见日出入时大,日中时小也。察物近则大,远则小,故日出入为近,日中为远也。其以日出入为远,日中时为近者,见日中时温,日出入时寒也。夫火光近人则温,远人则寒,故以日中为近,日出入为远也。二论各有所见,故是非曲直未有所定。如实论之,日中近而日出入远,何以验之?以植竿於屋下,夫屋高三丈,竿於屋栋之下,正而树之,上扣栋,下抵地,是以屋栋去地三丈。如旁邪倚之,则竿末旁跌,不得扣栋,是为去地过三丈也。日中时,日正在天上,犹竿之正树去地三丈也。日出入,邪在人旁,犹竿之旁跌去地过三丈也。夫如是,日中为近,出入为远,可知明矣。试复以屋中堂而坐一人,一人行於屋上,其行中屋之时,正在坐人之上,是为屋上之人,与屋下坐人相去三丈矣。如屋上人在东危若西危上,其与屋下坐人相去过三丈矣。日中时犹人正在屋上矣,其始出与入,犹人在东危与西危也。日中,去人近,故温,日出入,远,故寒。然则日中时日小,其出入时大者,日中光明故小,其出入时光暗故大,犹昼日察火光小,夜察之火光大也。既以火为效,又以星为验,昼日星不见者,光耀灭之也,夜无光耀,星乃见。夫日月,星之类也。平旦、日入光销,故视大也。

儒者论日旦出扶桑,暮入细柳。扶桑,东方地;细柳,西方野也。桑、柳,天地之际,日月常所出入之处。问曰:岁二月八月时,日出正东,日入正西,可谓日出於扶桑,入於细柳。今夏日长之时,日出於东北,入於西北;冬日短之时,日出东南,入於西南,冬与夏日之出入,在於四隅,扶桑、细柳,正在何所乎?所论之言,犹谓春秋,不谓冬与夏也。如实论之,日不出於扶桑,入於细柳。何以验之?随天而转,近则见,远则不见。当在扶桑、细柳之时,从扶桑、细柳之民,谓之日中之时,从扶桑、细柳察之,或时为日出入。〔皆〕以其上者为中,旁则为旦夕,安得出於扶桑,入细柳?儒者论曰:``天左旋,日月之行,不系於天,各自旋转''。难之曰:使日月自行,不系於天,日行一度,月行十三度,当日月出时,当进而东旋,何还始西转?系於天,随天四时转行也。其喻若蚁行於硙上,日月行迟天行疾,天持日月转,故日月实东行而反西旋也。

或问:``日、月、天皆行,行度不同,三者舒疾,验之人、物,〔何〕以〔为〕喻?''曰:天,日行一周。日行一度二千里,日昼行千里,夜行千里,〔骐骥〕昼日亦行千里。然则日行舒疾,与〔骐骥〕之步,相似类也。月行十三度,十度二万里,三度六千里,月一〔日〕〔一〕夜行二万六千里,与晨凫飞相类似也。天行三百六十五度,积凡七十三万里也,其行甚疾,无以为验,当与陶钧之运,弩矢之流,相类似乎?天行已疾,去人高远,视之若迟,盖望远物者,动若不动,行若不行。何以验之?乘船江海之中,顺风而驱,近岸则行疾,远岸则行迟,船行一实也,或疾或迟,远近之视使之然也。仰视天之运,不若〔骐骥〕负日而驰,〔比〕〔日〕暮,而日在其前,何则?〔骐骥〕近而日远也。远则若迟,近则若疾,六万里之程,难以得运行之实也。

儒者说曰:``日行一度,天一日一夜行三百六十五度,天左行,日月右行,与天相迎。''问:日月之行也,系著於天也,日月附天而行,不〔自〕行也。何以言之?《易》曰:``日月星辰丽乎天,百果草木丽於土。''丽者,附也。附天所行,若人附地而圆行,其取喻若蚁行於上焉。问曰:``何知不离天直自行也?''
如日能直自行,当自东行,无为随天而西转也。月行与日同,亦皆附天。何以验之?验之〔以〕云。云不附天,常止於所处,使不附天,亦当自止其处。由此言之,日行附天明矣。

问曰:``日,火也。火在地不行,日在天,何以为行?''曰:附天之气行,附地之气不行。火附地,地不行,故火不行。难曰:``附地之气不行,水何以行?
''曰:水之行也,东流入海也。西北方高,东南方下,水性归下,犹火性趋高也。使地不高西方,则水亦不东流。难曰:``附地之气不行,人附地,何以行?''曰:人之行,求有为也。人道有为,故行求。古者质朴,邻国接境,鸡犬之声相闻,终身不相往来焉。难曰:``附天之气行,列星亦何以不行?''曰:列星著天,天已行也,随天而转,是亦行也。难曰:``人道有为故行,天道无为何行?''曰:天之行也,施气自然也,施气则物自生,非故施气以生物也。不动,气不施,气不施,物不生,与人行异。日月五星之行,皆施气焉。

儒者曰:``日中有三足乌,月中有兔、蟾蜍。''夫日者,天之火也,与地之火无以异也。地火之中无生物,天火之中何故有乌?火中无生物,生物入火中,燋烂而死焉,乌安得立?.夫月者,水也水中有生物,非兔、蟾蜍也。
兔与蟾蜍久在水中,无不死者。日月毁於天,螺蚌汨於渊,同气审矣,所谓兔、蟾蜍者,岂反螺与蚌邪?且问儒者:乌、兔、蟾蜍死乎?生也?如死,久在日月,燋枯腐朽。如生,日蚀时既,月晦常尽,乌、兔、蟾蜍皆何在?夫乌、兔、蟾蜍,日月气也,若人之腹脏,万物之心膂也。月尚可察也,人之察日无不眩,不能知日审何气,通而见其中有物名曰乌乎?审日不能见乌之形,通而能见其足有三乎?此已非实。且听儒者之言,虫物非一,日中何为有乌,月中何为有``兔''、``蟾蜍''
?

儒者谓:``日蚀、月蚀也''。彼见日蚀常於晦朔,晦朔月与日合,故得蚀之。夫春秋之时,日蚀多矣。《经》曰:``某月朔,日有蚀之''。日有蚀之者,未必月也。知月蚀之,何讳不言月?说日蚀之变,阳弱阴强也,人物在世,气力劲强,乃能乘凌。案月晦光既,朔则如尽,微弱甚矣,安得胜日?夫日之蚀,月蚀也。日蚀,谓月蚀之,月谁蚀之者?无蚀月也,月自损也。以月论日,亦如日蚀,光自损也。大率四十一二月,日一食,百八十日,月一蚀,蚀之皆有时,非时为变,及其为变,气自然也。日时晦朔,月复为之乎?夫日当实满,以亏为变,必谓有蚀之者,山崩地动,蚀者谁也?或说:``日食者,月掩之也,日在上,月在下,障於〔月〕之形也。日月合相袭,月在上日在下者,不能掩日。日在上,月在日下,障於日,月光掩日光,故谓之食也,障於月也,若阴云蔽日月不见矣。其端合者,相食是也。其合相当如袭〔璧〕者,日既是也。''日月合於晦朔,.
天之常也。日食,月掩日光,非也。何以验之?使日月合,月掩日光,其初食崖当与旦复时易处。假令日在东,.月在西,月之行疾,东及日,掩日崖,
须臾过日而东,西崖初掩之处光当复,东崖未掩者当复食。今察日之食,西崖光缺,其复也,西崖光复,过掩东崖复西崖,谓之合袭相掩障,如何?

儒者谓:``日月之体皆至圆''。彼从下望见其形,若斗筐之状,状如正圆,不如望远光气,气不圆矣。夫日月不圆,视若圆者,〔去〕人远也。何以验之?夫日者,火之精也;月者,水之精也。在地,水火不圆;在天水火何故独圆?日月在天犹五星,五星犹列星,列星不圆,光耀若圆,去人远也。何以明之?春秋之时,星霣宋都,就而视之,石也,不圆。以星不圆,知日月五星亦不圆也。

儒者说日及工伎之家,皆以日为一。禹、〔益〕《山海经》言日有十,在海外东方有汤谷,上有扶桑,十日浴沐水中,有大木,九日居下枝,一日居上枝。《淮南书》又言:``烛十日。尧时十日并出,万物焦枯,尧上射十日。''以故不并一日见也。世俗又名甲乙为日,甲至癸凡十日,日之有十,犹星之有五也。通人谈士,归於难知,不肯辨明。是以文二传而不定,世两言而无主。诚实论之,且无十焉。何以验之?夫日犹月也,日而有十,月有十二乎?星有五,五行之精,金、木、水、火、土各异光色。如日有十,其气必异。今观日光无有异者,察其小大前後若一。如审气异,光色宜殊;如诚同气,宜合为一,无为十也。验日阳遂,火从天来,日者、大火也,察火在地,一气也,地无十火,天安得十日?然则所谓十日者,殆更自有他物,光质如日之状,居汤谷中水,时缘据扶桑,禹、益见之,则纪十日。数家度日之光,数日之质,刺径千里,假令日出是扶桑木上之日,扶桑木宜覆万里,乃能受之。何则?一日径千里,十日宜万里也。天之去人万里余也,仰察之,日光眩耀,火光盛明,不能堪也。使日出是扶桑木上之日,禹、益见之,不能知其为日也。何则?仰察一日,目犹眩耀,况察十日乎?当禹、益见之,若斗筐之状,故名之为日。夫火如斗筐,望六万之形,非就见之,即察之体也。由此言之,禹、益所见,意似日非日也。天地之间,物气相类,其实非者多。海外西南有珠树焉,察之是珠,然非鱼中之珠也。夫十日之日,犹珠树之珠也,珠树似珠非真珠,十日似日非实日也。淮南见《山海经》,则虚言真人烛十日,妄纪尧时十日并出。且日,火也;汤谷,水也。水火相贼,则十日浴於汤谷,当灭败焉。火燃木,扶桑,木也,十日处其上,宜燋枯焉。今浴汤谷而光不灭,登扶桑而枝不燋不枯,与今日出同,不验於五行,故知十日非真日也。且禹、益见十日之时,终不以夜,犹以昼也,则一日出,九日宜留,安得俱出十日?如平旦日未出,且天行有度数,日随天转行,安得留扶桑枝间,浴汤谷之水乎?留则失行度,行度差跌,不相应矣。如行出之日与十日异,是意似日而非日也。

《春秋》``庄公七年夏四月辛卯,夜中恆星不见,星如雨。''《公羊传》曰
``如雨者何?非雨也。非雨则曷为谓之如雨?不修《春秋》曰:雨星,不及地尺而复。君子修之曰:星霣如雨。''不修《春秋》者,未修《春秋》时《鲁史记》,曰``雨〔星〕,不及地尺而复''。君子者,孔子,孔子修之曰``星霣如雨''孔子之意,以为地有山陵楼台,云不及地尺,恐失其实,更正之曰如雨。如雨者,为从地上而下,星亦从天霣而复,与同,故曰如。夫孔子虽云不及地尺,但言如雨,其谓霣之者,皆是星也.孔子虽定其位,著其文,谓霣为星,与史同焉。
从平地望泰山之巅,鹤如乌,乌如爵者,泰山高远,物之小大失其实。天之去地六万余里,高远非直泰山之巅也;星著於天,人察之,失星之实,非直望鹤乌之类也。数等星之质百里,体大光盛,故能垂耀,人望见之,若凤卵之状,远失其实也。如星霣审者天之星霣而至地,人不知其为星也。何则?霣时小大,不与在天同也。今见星霣如在天时,是时星霣也;非星,则气为之也。人见鬼如死人之状,其实气象聚,非真死人。然则星之形,其实非星。孔子云正霣者非星,而徙,正言如雨非雨之文,盖俱失星之实矣。《春秋左氏传》:``四月辛卯,夜中恆星不见,夜明也;星霣如雨,与雨俱也。''其言夜明,故不见,与《易》之言日中见斗相依类也。日中见斗,幽不明也;夜中星不见,夜光明也。事异义同,盖其实也。其言与雨俱之集也。夫辛卯之夜明,故星不见,明则不雨之验也,雨气阴暗安得明?明则无雨,安得与雨俱?夫如是言与雨俱者非实,且言夜明不见,安得见星与雨俱?又僖公十六年正月戊申,霣石於宋五,《左氏传》曰:``星也。''夫谓霣石为星,则霣谓为石矣。辛卯之夜,星霣,为星,则实为石矣。辛卯之夜,星霣如是石,地有楼台,楼台崩坏。孔子虽不合言及地尺,虽地必有实数,鲁史目见,不空言者也,云与雨俱,雨集於地,石亦宜然。至地而楼台不坏,非星明矣。且左丘明谓石为星,何以审之?当时石霣轻然。何以其从天坠也?秦时三山亡,亡〔者〕不消散,有在其集下时必有声音,或时夷狄之山,从集於宋,宋闻石霣,则谓之星也。左丘明省,则谓之星。夫星,万物之精,与日月同。说五星者,谓五行之精之光也。五星众星同光耀,独谓列星为石,恐失其实。实者,辛卯之夜,霣星若雨而非星也,与彼汤谷之十日,若日而非日也。

儒者又曰:``雨从天下'',谓正从天坠也。如〔实〕论之,雨从地上,不从天下,见雨从上集,则谓从天下矣,其实地上也。然其出地起於山。何以明之?《春秋传》曰:``触石而出,肤寸而合,不崇朝而遍天下,惟太山也。''太山雨天下,小山雨一国,各以小大为近远差。雨之出山,或谓云载而行,云散水坠,名为雨矣。夫云则雨,雨则云矣,初出为云,云繁为雨。犹甚而泥露濡污衣服,若雨之状。非云与俱,云载行雨也。或曰:``《尚书》曰:`月之从星,则以风雨。'《诗》曰:``月丽於毕,俾滂沲矣。''二经咸言,所谓为之非天,如何?''
夫雨从山发,月经星丽毕之时,丽毕之时当雨也。时不雨,月不丽,山不云,天地上下自相应也。月丽於上,山烝於下,气体偶合,自然道也。云雾,雨之征也,夏则为露,冬则为霜,温则为雨,寒则为雪。雨露冻凝者,皆由地发,不从天降也。

\hypertarget{header-n505}{%
\subsubsection{答佞篇}\label{header-n505}}

或问曰:``贤者行道,得尊官厚禄;矣何心为佞,以取富贵?''曰:佞人知行道可以得富贵,必以佞取爵禄者,不能禁欲也;知力耕可以得谷,勉贸可以得货,然而必盗窃,情欲不能禁者也。以礼进退也,人莫之贵,然而违礼者众,尊义者希,心情贪欲,志虑乱溺也。夫佞与贤者同材,佞以情自败;偷盗与田商同知,偷盗以欲自劾也。

问曰:``佞与贤者同材,材行宜钧,而佞人曷为独以情自败?''曰:富贵皆人所欲也,虽有君子之行,犹有饥渴之情。君子则以礼防情,以义割欲,故得循道,循道则无祸;小人纵贪利之欲,逾礼犯义,故进得苟佞,苟佞则有罪。夫贤者,君子也;佞人,小人也。君子与小人本殊操异行,取舍不同。

问曰:``佞与谗者同道乎?有以异乎?''曰:谗与佞,俱小人也,同道异材,俱以嫉妒为性,而施行发动之异。谗以口害人,佞以事危人;谗人以直道不违,佞人依违匿端;谗人无诈虑,佞人有术数。故人君皆能远谗亲仁,莫能知贤别佞。难曰:``人君皆能远谗亲仁,而莫能知贤别佞,然则佞人意不可知乎?''曰:佞可知,人君不能知。庸庸之君,不能知贤,不能知贤,不能知佞。唯圣贤之人,以九德检其行,以事效考其言。行不合於九德,言不验於事效,人非贤则佞矣。夫知佞以知贤,知贤以知佞,知贤则贤智自觉,知贤则奸佞自得。贤佞异行,考之一验;情心不同,观之一实。

问曰:``九德之法,张设久矣,观读之者,莫不晓见,斗斛之量多少,权衡之县轻重也。然而居国有土之君,曷为常有邪佞之臣与常有欺惑之患?''〔曰〕:无患斗斛过,所量非其谷;不患无铨衡,所铨非其物故也。在人君位者,皆知九德之可以检行,事效可以知情,然而惑乱不能见者,则明不察之故也。人有不能行,行无不可检;人有不能考,情无不可知。

问曰:``行不合於九德,效不检於考功,进近非贤,非贤则佞。夫庸庸之材,无高之知不能及贤。贤功不效,贤行不应,可谓佞乎?''曰:材有不相及,行有不相追,功有不相袭。若知无相袭,人材相什百,取舍宜同。贤佞殊行,是是非非。实名俱立,而效有成败,是非之言俱当,功有正邪。言合行违,名盛行废。

问曰:``行合九德则贤,不合则佞。世人操行者可尽谓佞乎?''曰:诸非皆恶,恶中之逆者,谓之无道;恶中之巧者,谓之佞人。圣王刑宪,佞在恶中;圣王赏劝,贤在善中。纯洁之贤,善中殊高,贤中之圣也。〔恶〕中大佞,恶中之雄也。故曰:观贤由善,察佞由恶。善恶定成,贤佞形矣。

问曰:``聪明有蔽塞,推行有谬误,今以是者为贤,非者为佞,殆不得贤之实乎?''曰:聪明蔽塞,推行谬误,人之所歉也。故曰:刑故无小,宥过无大。圣君原心省意,故诛故贳误。故贼加增,过误减损,一狱吏所能定也,贤者见之不疑矣。

问曰:``言行无功效,可谓佞乎?''〔曰〕:苏秦约六国为从,强秦不敢窥兵於关外。张仪为横,六国不敢同攻於关内。六国约从,则秦畏而六国强;三秦称横,则秦强而天下弱。功著效明,载纪竹帛,虽贤何以加之?太史公叙言众贤,仪、秦有篇,无嫉恶之文,功钧名敌,不异於贤。夫功之不可以效贤,犹名之不可实也。仪、秦,排难之人也,处扰攘之世,行揣摩之术。当此之时,稷、契不能与之争计,禹、睾陶不能与之比效。若夫阴阳调和,风雨时适,五谷丰熟,盗贼衰息,人举廉让,家行道德之功,命禄贵美,术数所致,非道德之所成也。太史公记功,故高来祀,记录成则著效明验,揽载高卓,以仪、秦功美,故列其状。由此言之,佞人亦能以权说立功为效。无效,未可为佞也。难曰:``恶中立功者谓之佞。能为功者,材高知明。思虑远者,必傍义依仁,乱於大贤。故《觉佞》之篇曰:`人主好辨,佞人言利;人主好文,佞人辞丽。'心合意同,偶当人主,说而不见其非,何以知其伪而伺其奸乎?''曰:是谓庸庸之君也,材下知昬,蔽惑不见。〔若〕〔大〕贤之君,察之审明,若视俎上脯,指掌中之理,数局上之棋,摘辕中之马。鱼鳖匿渊,捕渔者知其源;禽兽藏山,畋猎者见其脉。佞人异行於世,世不能见,庸庸之主,无高材之人也。难曰:``人君好辨,佞人言利;人主好文,佞人辞丽。言操合同,何以觉之?''曰:《文王官人法》曰:推其往行,以揆其来言,听其来言,以省其往行,观其阳以考其阴,察其内以揆其外。是故诈善设节者可知,饰伪无情者可辨,质诚居善者可得,含忠守节者可见也。人之旧性不辨,人君好辨,佞人学求合於上也。人之故能不文,人君好文,佞人意欲称上。上奢,己丽服;上俭,己不饬。今操与古殊,朝行与家别。考乡里之迹,证朝庭之行,察共亲之节,明事君之操,外内不相称,名实不相副,际会发见、奸为觉露也。

问曰:``人操行无恆,权时制宜。信者欺人,直者曲挠。权变所设,前後异操,事有所应,左右异语。儒书所载,权变非一。今以素故考之,毋乃失实乎?''
曰:贤者有权,佞者有权。贤者之有权,後有应。佞人之有权,亦反经,後有恶。故贤人之权,为事为国;佞人之权,为身为家。观其所权,贤佞可论。察其发动,邪正可名。

问曰:``佞人好毁人,有诸?''曰:佞人不毁人。如毁人,是谗人也。何则?佞人求利,故不毁人。苟利於己,曷为毁之?苟不利於〔己〕,毁之无益。以计求便,以数取利,利则便得。妒人共事,然後危人。其危人也,非毁之;而其害人也,非泊之。誉而危之,故人不知;厚而害之,故人不疑。是故佞人危而不怨;害人,之败而不仇,隐情匿意为之功也。如毁人,人亦毁之,众不亲,士不附也,安能得容世取利於上?

问曰:``佞人不毁人於世间,毁人於将前乎?''曰:佞人以人欺将,不毁人於将。``然则佞人奈何?''曰:佞人毁人,誉之;危人,安之。``毁危奈何?''
假令甲有高行奇知,名声显闻,将恐人君召问,扶而胜己,欲故废不言,常腾誉之。荐之者众,将议欲用,问人,人必不对曰:``甲贤而宜召也。何则?甲意不欲留县,前闻其语矣,声望欲入府,在郡则望欲入州。志高则操与人异,望远则意不顾近。屈而用之,其心不满,不则卧病。贱而命之则伤贤,不则损威。故人君所以失名损誉者,好臣所常臣也。自耐下之,用之可也。自度不能下之,用之不便。夫用之不两相益,舍之不两相损。''人君畏其志,信佞人之言,遂置不用。

问曰:``佞人直以高才洪知考上世人乎?将有师学检也?''曰:人自有知以诈人,及其说人主,须术以动上,犹上人自有勇威人,及其战斗,须兵法以进众,术则从横,师则鬼谷也。传曰:``苏秦、张仪从横习之鬼谷先生,掘地为坑,曰:
`下,说令我泣出,则耐分人君之地。'苏秦下,说鬼谷先生泣下沾襟,张仪不若。苏秦相赵,并相六国。张仪贫贱往归,苏秦座之堂下,食以仆妾之食,数让激怒,欲令相秦。仪忿恨,遂西入秦。苏秦使人厚送。其後觉知,曰:此在其术中,吾不知也,此吾所不及苏君者。''知深有术,权变锋出,故身尊崇荣显,为世雄杰。深谋明术,深浅不能并行,明暗不能并知。

问曰:``佞人养名作高,有诸?''曰:佞人食利专权,不养名作高。贪权据凡,则高名自立矣。称於小人,不行於君子。何则?利义相伐,正邪相反。义动君子,利动小人。佞人贪利名之显,君子不安。下则身危。举世为佞者,皆以祸众。不能养其身,安能养其名?上世列传弃〔荣〕养身,违利赴名,竹帛所载,伯成子高委国而耕,於陵子辞位灌园。近世兰陵王仲子、东〔郡〕昔庐君阳,寝位久病,不应上征,可谓养名矣。夫不以道进,必不以道出身;不以义止,必不以义立名。佞人怀贪利之心,轻祸重身,倾死为矣,何名之养?义废德坏,操行随辱,何云作高?

问曰:``大佞易知乎?小佞易知也?''曰:大佞易知,小佞难知。何则?大佞材高,其迹易察;小佞知下,其效难省。何以明之?成事,小盗难觉,大盗易知也。攻城袭邑,剽劫虏掠,发则事觉,道路皆知盗也。穿凿垣墙,狸步鼠窃,莫知谓谁。曰:``大佞奸深惑乱其人如大盗易知,人君何难?''《书》曰:`知人则哲,惟帝难之。'虞舜大圣,驩兜大佞。大圣难知大佞,大佞不忧大圣。何易之有?''〔曰〕:是谓下知之,上知之。上知之,大难小易,下知之,大易小难。何则?佞人材高,论说丽美。因丽美之说,人主之威,人〔主〕心并不能责,知或不能觉。小佞材下,对乡失漏,际会不密,人君警悟,得知其故。大难小易也。屋漏在上,知者在下。漏大,下见之著;漏小,下见之微。或曰:``雍也仁而不佞。''孔子曰:``焉用佞?御人以口给,屡憎於民。''误设计数,烦扰农商,损下益上,愁民说主。损上益下,忠臣之说也;损下益上,佞人之义也。季氏富於周公,而求也为之聚敛而附益之。小子鸣鼓而攻之可也。聚敛,季氏不知其恶,不知百姓所共非也。

\hypertarget{header-n523}{%
\subsection{卷十二}\label{header-n523}}

\hypertarget{header-n524}{%
\subsubsection{程材篇}\label{header-n524}}

论者多谓儒生不及彼文吏,见文吏利便,而儒生陆落,则诋訾儒生以为浅短,称誉文吏谓之深长。是不知儒生,亦不知文吏也。

儒生、文吏皆有材智,非文吏材高而儒生智下也,文吏更事,儒生不习也。谓文吏更事,儒生不习,可也;谓文吏深长,儒生浅短,知妄矣。世俗共短儒生,儒生之徒,亦自相少。何则?并好仕学宦,用吏为绳表也。儒生有阙,俗共短之;文吏有过,俗不敢訾。归非於儒生,付是於文吏也。夫儒生材非下於文吏,又非所习之业非所当为也,然世俗共短之者,见将不好用也。将之不好用之者,事多己不能理,须文吏以领之也。夫论善谋材,施用累能,期於有益。文吏理烦,身役於职,职判功立,将尊其能。儒生栗栗,不能当剧;将有烦疑,不能效力,力无益於时,则官不及其身也。将以官课材,材以官为验,是故世俗常高文吏,贱下儒生。儒生之下,文吏之高,本由不能之将。世俗之论,缘将好恶。

今世之将,材高知深,通达众凡,举纲持领,事无不定。其置文吏也,备数满员,足以辅己志。志在修德,务在立化,则夫文吏瓦石,儒生珠玉也。夫文吏能破坚理烦,不能守身,〔不能守〕身,则亦不能辅将。儒生不习於职,长於匡救,将相倾侧,谏难不惧。案世间能建蹇蹇之节,成三谏之议,令将检身自敕,不敢邪曲者,率多儒生。阿意苟取容幸,将欲放失,低嘿不言者,率多文吏。文吏以事胜,以忠负;儒生以节优,以职劣。二者长短,各有所宜。世之将相,各有所取。取儒生者,必轨德立化者也;取文吏者,必优事理乱者也。材不自能则须助,须助则待劲。官之立佐,为力不足也;吏之取能,为材不及也。

日之照幽,不须灯烛;贲、育当敌,不待辅佐。使将相知力,若日之照幽,贲、育之难敌,则文吏之能无所用也。病作而医用,祸起而巫使。如自能案方和药,入室求祟,则医不售而巫不进矣。桥梁之设也,足不能越沟也;车马之用也,走不能追远也。足能越沟,走能追远,则桥梁不设、车马不用矣。天地事物,人所重敬,皆力劣知极,须仰以给足者也。今世之将相,不责己之不能,而贱儒生之不习;不原文吏之所得得用,而尊其材,谓之善吏。非文吏,忧不除;非文吏,患不救。是以选举取常故,案吏取无害。儒生无阀阅,所能不能任剧,故陋於选举,佚於朝庭。聪慧捷疾者,随时变化,学知吏事,则踵文吏之後,未得良善之名。守古循志,案礼修义,辄为将相所不任,文吏所毗戏。不见任则执欲息退,见毗戏则意不得。临职不劝,察事不精,遂为不能,斥落不习。有俗材而无雅度者,学知吏事,乱於文吏,观将所知,适时所急,转志易务,昼夜学问,无所羞耻,期於成能名文而已。其高志妙操之人,耻降意损崇,以称媚取进,深疾才能之儒,〔汨〕入文吏之科,坚守高志,不肯下学。亦时或精暗不及,意疏不密,临事不识;对向谬误;拜起不便,进退失度;秦记言事,蒙士解过,援引古义;割切将欲,直言一指,触讳犯忌;封蒙约缚,简绳检署,事不如法,文辞卓诡,辟刺离实,曲不应义。故世俗轻之,文吏薄之,将相贱之。

是以世俗学问者,不肯竟经明学,深知古今,急欲成一家章句,义理略具,同〔趋〕学史书,读律讽令,治作〔请〕奏,习对向,滑习跪拜,家成室就,召署辄能。徇今不顾古,趋仇不存志,竞进不案礼,废经不念学。是以古经废而不修,旧学暗而不明,儒者寂於空室,文吏哗於朝堂。材能之士,随世驱驰;节操之人,守隘屏窜。驱驰日以巧,屏窜日以拙。非材顿知不及也,希见阙为,不狎习也。盖足未尝行,尧、禹问曲折;目未尝见,孔、墨问形象。

齐部世刺绣,恆女无不能;襄邑俗织锦,钝妇无不巧。〔目〕见之,日为之,手狎也。使材士未尝见,巧女未尝为,异事诡手,暂为卒睹,显露易为者,犹愦愦焉。方今论事,不谓希更,而曰材不敏;不曰未尝为,而曰知不达。失其实也。儒生材无不能敏,业无不能达,志不〔肯〕为。今俗见不习,谓之不能,睹不为,谓之不达。科用累能,故文吏在前,儒生在後。是从朝庭谓之也。如从儒堂订之,则儒生在上,文吏在下矣。从农论田,田夫胜;从商讲贾,贾人贤;今从朝庭谓之文吏。朝庭之人也,幼为干吏,以朝庭为田亩,以刀笔为耒耜,以文书为农业,犹家人子弟,生长宅中,其知曲折,愈於宾客也。宾客暂至,虽孔、墨之材,不能分别。儒生犹宾客,文吏犹子弟也。以子弟论之,则文吏晓於儒生,儒生暗於文吏。今世之将相,知子弟以文吏为慧,不能知文吏以狎为能;知宾客以暂为固,不知儒生以希为拙:惑蔽暗昧,不知类也。一县佐史之材,任郡掾史。一郡修行之能,堪州从事。然而郡不召佐史,州不取修行者,巧习无害,文少德高也。五曹自有条品,簿书自有故事,勤力玩弄,成为巧吏,安足多矣?贤明之将,程吏取才,不求习论高,存志不顾文也。

称良吏曰忠,忠之所以为效,非簿书也。夫事可学而知,礼可习而善,忠节公行不可立也。文吏、儒生皆有所志,然而儒生务忠良,文吏趋理事。苟有忠良之业,疏拙於事,无损於高。论者以儒生不晓簿书,置之於下第。法令比例,吏断决也。文吏治事,必问法家。县官事务,莫大法令。必以吏职程高,是则法令之家宜最为上。或曰:``固然,法令,汉家之经,吏议决焉。事定於法,诚为明矣。''曰:夫《五经》亦汉家之所立,儒生善政,大义皆出其中。董仲舒表《春秋》之义,稽合於律,无乖异者。然则《春秋》,汉之经,孔子制作,垂遗於汉。论者徒尊法家,不高《春秋》,是暗蔽也。《春秋》、《五经》义相关穿,既是《春秋》,不大《五经》,是不通也。《五经》以道为务,事不如道,道行事立,无道不成。然则儒生所学者,道也;文吏所学者,事也。假使材同,当以道学。如比於文吏,洗泥者以水,燔腥生者用火。水火,道也,用之者,事也,事末於道。儒生治本,文吏理末,道本与事末比,定尊卑之高下,可得程矣。

尧以俊德,致黎民雍。孔子曰:``孝悌之至,通於神明。''张释之曰:``秦任刀笔小吏,陵迟至於二世,天下土崩。''张汤、赵禹,汉之惠吏,太吏公《序累》置於酷部而致土崩,孰与通於神明令人填膺也?将相知经学至道,而不尊经学之生,彼见经学之生,能不及治事之吏也。牛刀可以割鸡,鸡刀难以屠牛。刺绣之师能缝帷裳。纳缕之工不能织锦;儒生能为文吏之事,文吏不能立儒生之学。文吏之能,诚劣不及,儒生之不习,实优而不为。禹决江河,不秉锸;周公筑雒,不把筑杖。夫笔墨簿书,锸筑杖之类也,而欲合志大道者躬亲为之,是使将军战而大匠斫也。说一经之生,治一曹之事,旬月能之。典一曹之吏,学一经之业,一岁不能立也。何则?吏事易知,而经学难见也。儒生擿经,穷竟圣意;文吏摇笔,考迹民事。夫能知大圣之意,晓细民之情,孰者为难?以立难之材,含怀章句十万以上,行有余力。博学览古今,计胸中之颖,出溢十万。文吏所知,不过辨解簿书。富累千金,孰与赀直百十也?京禀知丘,孰与委聚如坻也?世名材为名器,器大者盈物多。然则儒生所怀,可谓多矣。

蓬生麻间,不扶自直;白纱入缁,不染自黑。此言所习善恶,变易质性也。儒生之性,非能皆善也,被服圣教,日夜讽咏,得圣人之操矣。文吏幼则笔墨,手习而行,无篇章之诵,不闻仁义之语。长大成吏,舞文巧法,徇私为己,勉赴权利;考事则受赂,临民则采渔,处右则弄权,幸上则卖将;一旦在位,鲜冠利剑。一岁典职,田宅并兼。性非皆恶,所习为者,违圣教也。故习善儒路,归化慕义,志操则励变从高明。明将见文,显用儒生:东海相宗叔犀,犀广召幽隐,春秋会飨,设置三科,以第补吏。一府员吏,儒生什九。陈留太守陈子瑀,开广儒路,列曹掾史,皆能教授。簿书之吏,什置一二。两将知道事之理,晓多少之量,故世称褒其名,书记纪累其行也。

\hypertarget{header-n536}{%
\subsubsection{量知篇}\label{header-n536}}

《程材》所论,论材能行操,未言学知之殊奇也。夫儒生之所以过文吏者,学问日多,简练其性,雕琢其材也。故夫学者所以反情治性,尽才成德也。材尽德成,其比於文吏,亦雕琢者,程量多矣。贫人与富人,俱赍钱百,并为赙礼死哀之家。知之者,知贫人劣能共百,以为富人饶羡有奇余也;不知之者,见钱俱百,以为财货贫富皆若一也。文吏、儒生有似於此。皆为掾吏,并典一曹,将知之者,知文吏、儒生笔同,而儒生胸中之藏,尚多奇余;不知之者,以为皆吏,深浅多少同一量,失实甚矣。地性生草,山性生木。如地种葵韭,山树枣栗,名曰美园茂林,不复与一恆地庸山比矣。文吏、儒生,有似於此,俱有材能,并用笔墨,而儒生奇有先王之道。先王之道,非徒葵韭枣栗之谓也。恆女之手,纺绩织经;如或奇能,织锦刺绣,名曰卓殊,不复与恆女科矣。夫儒生与文吏程材,而儒生侈有经传之学,犹女工织锦刺绣之奇也。

贫人好滥,而富人守节者,贫人不足而富人饶侈。儒生不为非,而文吏好为奸者,文吏少道德,而儒生多仁义也。贫人富人,并为宾客,受赐於主人,富人不惭而贫人常愧者,富人有以效,贫人无以复也。儒生、文吏,俱以长吏为主人者也。儒生受长吏之禄,报长吏以道;文吏空胸无仁义之学,居往食禄,终无以效,所谓尸位素餐者也。素者,空也;空虚无德,餐人之禄,故曰素餐。无道艺之业,不晓政治,默坐朝庭,不能言事,与尸无异,故曰尸位。然则文吏所谓尸位素餐者也。居右食嘉,见将倾邪,岂能举记陈言得失乎?一则不能见是非,二则畏罚不敢直言。《礼》曰:``情欲巧。''其能力言者,文丑不好,有骨无肉,脂腴不足,犯干将相指,遂取间郤。为地战者不能立功名,贪爵禄者不能谏於上。文吏贪爵禄,一日居位,辄欲图利,以当资用,侵渔徇身,不为将官显义。虽见太山之恶,安肯扬举毛发之言?事理如此,何用自解於尸位素餐乎?儒生学大义,以道事将,不可则止,有大臣之志,以经勉为公正之操,敢言者也,位又疏远。远而近谏,《礼》谓之谄,此则郡县之府庭所以常廓无人者也。

或曰:``文吏笔扎之能,而治定簿书,考理烦事,虽无道学,筋力材能尽於朝庭,此亦报上之效验也。''曰:此有似於贫人负官重责,贫无以偿,则身为官作,责乃毕竟。夫官之作,非屋庑则墙壁也。屋庑则用斧斤,墙壁则用筑锸。荷斤斧,把筑锸,与彼握刀持笔何以殊?苟谓治文书者报上之效验,此则治屋庑墙壁之人,亦报上也。俱为官作,刀笔斧斤筑锸钧也。抱布贸丝,交易有亡,各得所愿。儒生抱道贸禄,文吏无所抱,何用贸易?农商殊业,所畜之货,货不可同,计其精粗,量其多少,其出溢者名曰富人,富人在世,乡里愿之。夫先王之道,非徒农商之货也,其为长吏立功致化,非徒富多出溢之荣也。且儒生之业,岂徒出溢哉?其身简练,知虑光明,见是非审,审尤奇也。

蒸所与众山之材干同也,〔伐〕以为蒸,熏以火,烟热究浃,光色泽润,爇之於堂,其耀浩广,火灶之效加也。绣之未刺,锦之未织,恆丝庸帛,何以异哉?加五采之巧,施针缕之饰,文章炫耀,黼黻华虫,山龙日月。学士有文章,犹丝帛之有五色之巧也。本质不能相过,学业积聚,超逾多矣。物实无中核者谓之郁,无刀斧之断者谓之朴。文吏不学,世之教无核也,郁朴之人,孰与程哉?骨曰切,象曰瑳,玉曰琢,石曰磨,切琢磨,乃成宝器。人之学问知能成就,犹骨象玉石切瑳琢磨也。虽欲勿用,贤君其舍诸?孙武、阖庐,世之善用兵者也,知或学其法者,战必胜。不晓什伯之阵,不知击刺之术者,强使之军,军覆师败,无其法也。谷之始熟曰粟。舂之於臼,簸其粃糠;蒸之於甑,爨之以火,成熟为饭,乃甘可食。可食而食之,味生肌腴成也。粟未为米,米未成饭,气腥未熟,食之伤人。夫人之不学,犹谷未成粟,米未为饭也。知心乱少,犹食腥谷,气伤人也。学士简练於学,成熟於师,身之有益,犹谷成饭,食之生肌腴也。铜锡未采,在众石之间,工师凿掘,炉橐铸铄乃成器。未更炉橐,名曰积石,积石与彼路畔之瓦、山间之砾,一实也。故夫谷未舂蒸曰粟,铜未铸铄曰积石,人未学问曰矇。矇者,竹木之类也。夫竹生於山,木长於林,未知所入。截竹为筒,破以为牒,加笔墨之迹,乃成文字,大者为经,小者为传记。断木为椠,之为板,力加刮削,乃成奏牍。夫竹木,粗苴之物也,雕琢刻削,乃成为器用。况人含天地之性,最为贵者乎!

不入师门,无经传之教,以郁朴之实,不晓礼义,立之朝庭,植笮树表之类也,其何益哉?山野草茂,钩镰斩刈,乃成道路也。士未入道门,邪恶未除,犹山野草木未斩刈,不成路也。染练布帛,名之曰采,贵吉之服也。无染练之治,名縠粗,縠粗不吉,丧人服之。人无道学,仕宦朝庭,其不能招致也,犹丧人服粗,不能招吉也。能削柱梁,谓之木匠。能穿凿穴坎,谓之士匠;能雕琢文书,谓之史匠。夫文吏之学,学治文书也,当与木土之匠同科,安得程於儒生哉?御史之遇文书,不失分铢;有司之陈笾豆,不误行伍。其巧习者,亦先学之,人不贵者也,小贱之能,非尊大之职也。无经艺之本,有笔墨之末,大道未足而小伎过多,虽曰吾多学问,御史之知、有司之惠也。饭黍梁者餍,餐糟糠者饱,虽俱曰食,为腴不同。儒生文吏,学俱称习,其於朝庭,有益不钧。郑子皮使尹何为政,子产比於未能操刀使之割也。子路使子羔为费宰,孔子曰:``贼夫人之子。''
皆以未学,不见大道也。医无方术,云:``吾能治病。''问之曰:``何用治病?''
曰:``用心意。''病者必不信也。吏无经学,曰:``吾能治民。''问之曰:``何用治民?''曰:``以材能。''是医无方术,以心意治病也,百姓安肯信向,而人君任用使之乎?手中无钱,之市使货主问曰``钱何在'',对曰:``无钱'',货主必不与也。夫胸中不学,犹手中无钱也。欲人君任使之,百姓信向之,奈何也?

\hypertarget{header-n544}{%
\subsubsection{谢短篇}\label{header-n544}}

《程材》、《量知》,言儒生、文吏之材不能相过;以儒生修大道,以文吏晓簿书,道胜於事,故谓儒生颇愈文吏也。此职业外相程相量也,其内各有所以为短,未实谢也。夫儒生能说一经,自谓通大道,以骄文吏;文吏晓簿书,自谓文无害,以戏儒生。各持满而自〔臧〕,非彼而是我,不知所为短,不悟於己未足。《论衡》之,将使忄奭然各知所〔乏〕。训夫儒生所短,不徒以不晓簿书;文吏所劣,不徒以不通大道也,反以闭暗不览古今,不能各自知其所业之事未具足也。二家各短,不能自知也。世之论者,而亦不能训之,如何?夫儒生之业,《五经》也,南面为师,旦夕讲授章句,滑习义理,究备於《五经》可也。《五经》之後,秦、汉之事,不能知者,短也。夫知古不知今,谓之陆沉,然则儒生,所谓陆沉者也。《五经》之前,至於天地始开、帝王初立者,主名为谁,儒生又不知也。夫知今不知古,谓之盲瞽。《五经》比於上古,犹为今也。徒能说经,不晓上古,然则儒生,所谓盲瞽者也。

儒生犹曰:``上古久远,其事暗昧,故经不载而师不说也。''夫三王之事虽近矣,经虽不载,义所连及,《五经》所当共知,儒生所当审说也。夏自禹向国,几载而至於殷?殷自汤几祀而至於周?周自文王几年而至於秦?桀亡夏而纣弃殷,灭周者何王也?周犹为远,秦则汉之所伐也。夏始於禹,殷本於汤,周祖后稷,秦初为人者谁?秦燔《五经》,坑杀儒士,《五经》之家所共闻也。秦何起而燔《五经》,何感而坑儒生?秦则前代也。汉国自儒生之家也,从高祖至今朝几世?历年讫今几载?初受何命?复获何瑞?得天下难易孰与殷、周?家人子弟学问历几岁,人问之曰:``居宅几年?祖先何为?''不能知者,愚子弟也。然则儒生不能知汉事,世之愚蔽人也。``温故知新,可以为师。''古今不知,称师如何?彼人问曰:``二尺四寸,圣人文语,朝夕讲习,义类所及,故可务知。汉事未载於经,名为尺籍短书,比於小道,其能知,非儒者之贵也。''儒不能都晓古今,欲各别说其经;经事义类,乃以不知为贵也?

事不晓,不以为短!请复别问儒生,各以其经,旦夕之所讲说。先问《易》家:``《易》本何所起?造作之者为谁?''彼将应曰:``伏羲作八卦,文王演为六十四,孔子作《彖》、《象》、《系辞》。三圣重业,《易》乃具足。''问之曰:``《易》有三家,一曰《连山》,二曰《归藏》,三曰《周易》。伏羲所作,文王所造,《连山》乎?《归藏》、《周易》也?秦燔《五经》,《易》何以得脱?汉兴几年而复立?宣帝之时,河内女子坏老屋,得《易》一篇,名为何《易》?此时《易》具足未?''问《尚书》家曰:``今旦夕所授二十九篇,奇有百二篇,又有百篇。二十九篇何所起?百二篇何所造?秦焚诸书之时,《尚书》诸篇皆何在?汉兴,始录《尚书》者何帝?初受学者何人?''问《礼》家曰:``前孔子时,周已制礼,殷礼,夏礼,凡三王因时损益,篇有多少,文有增减。不知今《礼》,周乎?殷、夏也?''彼必以汉承周,将曰:``周礼。''夫周礼六典,又六转,六六三十六,三百六十,是以周官三百六十也。案今《礼》不见六典,无三百六十官,又不见天子。天子礼废何时?岂秦灭之哉?宣帝时,河内女子坏老屋,得佚《礼》一篇,六十篇中,是何篇是者?高祖诏叔孙通制作《仪品》,十〔二〕篇何在?而复定《仪礼》,见在十六篇,秦火之余也。更秦之时,篇凡有几?问《诗》家曰:``《诗》作何帝王时也?''彼将曰:``周衰而《诗》作,盖康王时也。康王德缺於房,大臣刺晏,故《诗》作。''夫文、武之隆贵在成、康,康王未衰,《诗》安得作?周非一王,何知其康王也?二王之末皆衰,夏、殷衰时,《诗》何不作?《尚书》曰``诗言志,歌咏言'',此时已有诗也,断取周以来,而谓兴於周。古者采诗,诗有文也,今《诗》无书,何知非秦燔《五经》,《诗》独无余〔札〕也?问《春秋》家曰:``孔子作《春秋》,周何王时也?自卫反鲁,然後乐正,《春秋》作矣。自卫反鲁,哀公时也。自卫,何君也?俟孔子以何礼,而孔子反鲁作《春秋》乎?孔子录《史记》以作《春秋》,《史记》本名《春秋》乎?制作以为经,乃归《春秋》也?

法律之家,亦为儒生。问曰:``《九章》,谁所作也?''彼闻皋陶作狱,必将曰:``皋陶山。''诘曰:``皋陶,唐、虞时,唐、虞之刑五刑,案今律无五刑之文。''或曰:``萧何也。''诘曰:``萧何,高祖时也,孝文之时,齐太仓令淳于〔意〕有罪,征诣长安,其女缇萦为父上书,言肉刑壹施,不得改悔。文帝痛其言,乃改肉刑。案今《九章》象刑,非肉刑也。文帝在萧何後,知时肉刑也。萧何所造,反具〔象〕刑也?而云《九章》萧何所造乎?''古礼三百,威仪三千,刑亦正刑三百,科条三千。出於礼,入於刑,礼之所去,刑之所取,故其多少同一数也。今《礼经》十六,萧何律有九章,不相应,又何?《五经》题篇,皆以事义别之,至礼与律独经也,题之,礼言昏礼,律言盗律何?

夫总问儒生以古今之义,儒生不能知,别〔各〕以其经事问之,又不能晓,斯则坐守〔信〕师法、不颇博览之咎也。文吏自谓知官事,晓簿书。问之曰:``
晓知其事,当能究达其义,通见其意否?''文吏必将罔然。问之曰:``古者封侯,各专国土,今置太守令长,何义?古人井田,民为公家耕,今量租刍,何意?一〔岁〕使民居更一月,何据?年二十三〔傅〕,十五赋,七岁头钱二十三,何缘?有臈,何帝王时?门户井灶,何立?社稷、先农、灵星,保祠?岁终逐疫,何驱?使立桃象人於门户,何旨?挂芦索於户上,画虎於门阑,何放?除墙壁书画厌火丈夫,何见?步之六尺,冠之六寸,何应?有尉史令史,无〔丞〕长史,何制?两郡移书,曰:``敢告卒人'',两县不言,何解?郡言事二府,曰``敢言之''
,司空曰``上'',何状?赐民爵八级,何法?名曰簪褭、上造,何谓?吏上功曰伐阅,名籍墨〔状〕,何指?七十赐王杖,何起?著鸠於杖末,不著爵,何杖?苟以鸠为善,不赐而赐鸠杖,而不爵,何说?日分六十,漏之尽〔百〕,鼓之致五,何故?吏衣黑衣,宫阙赤单,何慎?服革於腰,佩刀於右,〔带〕剑於左,何备?著钩於履,冠在於首,何象?吏居城郭,出乘车马,坐治文书,起城郭,何王?造车舆,何工?生马,何地?作书,何人?造城郭及马所生,难知也,远也。造车作书,易晓也,必将应曰:``仓颉作书,奚仲作车。''诘曰:``仓颉何感而作书?奚仲何起而作车?''又不知也。文吏所当知,然而不知,亦不博览之过也。夫儒生不览古今,〔所〕知不过守信经文,滑习章句,解剥互错,分明乖异。文吏不晓吏道,所能不过案狱考事,移书下记,对〔乡〕便给。准〔之〕无一阅备,皆浅略不及,偏驳不纯,俱有阙遗,何以相言?

\hypertarget{header-n553}{%
\subsection{卷十三}\label{header-n553}}

\hypertarget{header-n554}{%
\subsubsection{效力篇}\label{header-n554}}

《程才》、《量知》之篇,徒言知学,未言才力也。人有知学,则有力矣。文吏以理事为力,而儒生以学问为力。或问扬子云曰:``力能扛鸿鼎、揭华旗,知德亦有之乎?''答曰:``百人矣。''夫知德百人者,与彼扛鸿鼎、揭华旗者为料敌也。夫壮士力多者,扛鼎揭旗;儒生力多者,博达疏通。故博达疏通,儒生之力也;举重拔坚,壮士之力也。《梓材》曰:``强人有王开贤,厥率化民。''
此言贤人亦壮强於礼义,故能开贤,其率化民。化民须礼义,礼义须文章,``行有余力,则以学文''。能学文,有力之验也。问曰:``说一经之儒,可谓有力者?
''曰:非有力者也。陈留庞少都每荐诸生之吏,常曰:``王甲某子,才能百人。''
太守非其能,不答。少都更曰:``言之尚少,王甲某子,才能百万人。''太守怒曰:``亲吏妄言!''少都曰:``文吏不通一经一文,不调师一言;诸生能说百万章句,非才知百万人乎?''太守无以应。夫少都之言,实也,然犹未也。何则?诸生能传百万言,不能览古今,守信师法,虽辞说多,终不为博。殷、周以前,颇载《六经》,儒生所不能说也。秦、汉之事,儒生不见,力劣不能览也。周监二代,汉监周、秦,周、秦以来,儒生不知;汉欲观览,儒生无力。使儒生博观览,则为文儒。文儒者,力多於儒生,如少都之言,文儒才能千万人矣。

曾子曰:``士不可以不弘毅,任重而道远。仁以为己任,不亦重乎!死而後己,不亦远乎!''由此言之,儒者所怀,独己重矣,志所欲至,独己远矣。身载重任,至於终死,不倦不衰,力独多矣。夫曾子载於仁而儒生载於学,所载不同,轻重均也。夫一石之重,一人挈之,十石以上,二人不能举也。世多挈一石之任,寡有举十石之力。儒生所载,非徒十石之重也。地力盛者,草木暢茂。一亩之收,当中田五亩之分。苗田,人知出谷多者地力盛。不知出文多者才知茂,失事理之实矣。夫文儒之力过於儒生,况文吏乎?能举贤荐士,世谓之多力也。然能举贤荐士,上书〔占〕记也。能上书〔占〕记者,文儒也。文儒非必诸生也,贤达用文则是矣。谷子云、唐子高章奏百上,笔有余力,极言不讳,文不折乏,非夫才知之人不能为也。孔子,周世多力之人也。作《春秋》,删《五经》,秘书微文,无所不定。山大者云多,泰山不崇朝辩雨天下。夫然则贤者有云雨之知,故其吐文万牒以上,可谓多力矣。

世称力者,常褒乌获,然则董仲舒、扬子云,文之乌获也。秦武王与孟说举鼎不任,绝脉而死。少文之人,与董仲舒等涌胸中之思,必将不任,有绝脉之变。王莽之时,省《五经》章句皆为二十万,博士弟子郭路夜定旧说,死於烛下,精思不任,绝脉气减也。颜氏之子,已曾驰过孔子於涂矣,劣倦罢极,发白齿落。夫以庶几之材,犹有仆顿之祸,孔子力优,颜渊不任也。才力不相如,则其知〔惠〕不相及也。勉自什伯,鬲中呕血,失魂狂乱,遂至气绝。书五行之牍,〔奏〕十〔言〕之记,其才劣者,笔墨之力尤难,况乃连句结章,篇至十百哉!力独多矣!江河之水,驰涌滑漏,席地长远,无枯竭之流,本源盛矣。知江河之流远,地中之源盛,不知万牒之人,胸中之才茂,迷惑者也。故望见骥足,不异於众马之蹄,蹑平陆而驰骋,千里之迹,斯须可见。夫马足人手,同一实也,称骥之足,不荐文人之手,不知类也。夫能论筋力以见比类者,则能取文力之人立之朝庭。故夫文力之人,助有力之将,乃能以力为功。有力无助,以力为祸。何以验之?长巨之物,强力之人乃能举之。重任之车,强力之牛乃能挽之。是任车上阪,强牛引前,力人推後,乃能升逾。如牛羸人罢,任车退却,还堕坑谷,有破覆之败矣。文儒怀先王之道,含百家之言,其难推引,非徒任车之重也。荐致之者,罢羸无力,遂却退窜於岩穴矣。

河发昆仑,江起岷山,水力盛多,滂沛之流,浸下益盛,不得广岸低地,不能通流入乎东海。如岸狭地仰,沟洫决泆,散在丘墟矣。文儒之知,有似於此。文章滂沛,不遭有力之将援引荐举,亦将弃遗於衡门之下,固安得升陟圣主之庭,论说政事之务乎?火之光也,不举不明。有人於斯,其知如京,其德如山,力重不能自称,须人乃举,而莫之助,抱其盛高之力,窜於闾巷之深,何时得达?奡、育,古之多力者,身能负荷千钧,手能决角伸钩,使之自举,不能离地。智能满胸之人,宜在王阙,须三寸之舌,一尺之笔,然後自动,不能自进,进之又不能自安,须人能动,待人能安。道重知大,位地难适也。小石附於山,山力能得持之;在沙丘之间,小石轻微,亦能自安。至於大石,沙土不覆,山不能持,处危峭之际,则必崩坠於坑谷之间矣。大智之重,遭信之将,无左右沙土之助,虽在显位,将不能持,则有大石崩坠之难也。或伐薪於山,轻小之木,合能束之。至於大木十围以上,引之不能动,推之不能移,则委之於山林,收所束之小木而归。由斯以论,知能之大者,其犹十围以上木也,人力不能举荐,其犹薪者不能推引大木也。孔子周流,无所留止,非圣才不明,道大难行,人不能用也。故夫孔子,山中巨木之类也。

桓公九合诸侯,一匡天下,管仲之力。管仲有力,桓公能举之,可谓壮强矣。吴不能用子胥,楚不能用屈原,二子力重,两主不能举也。举物不胜,委地而去可也,时或恚怒,斧斫破败,此则子胥、屈原所取害也。渊中之鱼,递相吞食,度口所能容,然後咽之;口不能受,哽咽不能下。故夫商鞅三说孝公,後说者用,前二难用,後一易行也。观管仲之《明法》,察商鞅之《耕战》,固非弱劣之主所能用也。六国之时,贤才之臣,入楚楚重,出齐齐轻,为赵赵完,畔魏魏伤。韩用申不害,行其《三符》,兵不侵境,盖十五年。不能用之,又不察其书,兵挫军破,国并於秦。殷、周之世,乱迹相属,亡祸比肩,岂其心不欲为治乎?力弱智劣,不能纳至言也。是故碓重,一人之迹不能蹈也;大,一人之掌不能推也。贤臣有劲强之优,愚主有不堪之劣,以此相求,禽鱼相与游也。干将之刃,人不推顿,菰瓠不能伤;

筱\{辂\}之箭,机不动发,鲁缟不能穿。非无干将、筱\{辂\}之才也,无推顿发动之主。菰瓠、鲁缟不穿伤,焉望斩旗穿革之功乎?故引弓之力不能引强弩。弩力五石,引以三石,筋绝骨折,不能举也。故力不任强引,则有变恶折脊之祸;知不能用贤,则有伤德毁名之败。

论事者不曰才大道重,上不能用,而曰不肖不能自达。自达者带绝不抗,自衒者贾贱不仇。案诸为人用之物,须人用之,功力乃立。凿所以入木者,槌叩之也,锸所以能撅地者,跖蹈之也。诸有锋刃之器,所以能断斩割削者,手能把持之也,力能推引之也。韩信去楚入汉,项羽不能安,高祖能持之也。能用其善,能安其身,则能量其力、能别其功矣。樊、郦有攻城野战之功,高祖行封,先及萧何,则比萧何於猎人,同樊、郦於猎犬也。夫萧何安坐,樊、郦驰走,封不及驰走而先安坐者,萧何以知为力,而樊、郦以力为功也。萧何所以能使樊、郦者,以入秦收敛文书也。众将拾金,何独掇书,坐知秦之形势,是以能图其利害。众将驰走者,何驱之也。故叔孙通定仪,而高祖以尊;萧何造律,而汉室以宁。案仪律之功,重於野战,斩首之力,不及尊主。故夫垦草殖谷,农夫之力也;勇猛攻战,士卒之力也;构架斫削,工匠之力也;治书定簿,佐史之力也;论道议政,贤儒之力也。人生莫不有力,所以为力者,或尊或卑。孔子能举北门之关,不以力自章,知夫筋骨之力,不如仁义之力荣也。

\hypertarget{header-n564}{%
\subsubsection{别通篇}\label{header-n564}}

富人之宅,以一丈之地为内。内中所有,柙匮所〔赢〕,缣布丝〔帛〕也。贫人之宅,亦以一丈为内。内中空虚,徒四壁立,故名曰贫。夫通人犹富人,不通者犹贫人也。俱以七尺为形,通人胸中怀百家之言,不通者空腹无一牒之诵。贫人之内,徒四所壁立也。慕料贫富不相如,则夫通与不通不相及也。世人慕富不荣通,羞贫,不贱不贤,不推类以况之也。夫富人可慕者,货财多则饶裕,故人慕之。夫富人不如儒生,儒生不如通人。通人积文,十箧以上,圣人之言,贤者之语,上自黄帝,下至秦、汉,治国肥家之术,刺世讥俗之言,备矣。使人通明博见,其为可荣,非徒缣布丝〔帛〕也。萧何入秦,收拾文书,汉所以能制九州者,文书之力也。以文书御天下,天下之富,孰与家人之财?

人目不见青黄曰盲,耳不闻宫商曰聋,鼻不知香臭曰痈。痈聋与盲,不成人者也。人不博览者,不闻古今,不见事类,不知然否,犹目盲、耳聋、鼻痈者也。儒生不览,犹为闭暗,况庸人无篇章之业,不知是非,其为闭暗,甚矣!此则土木之人,耳目俱足,无闻见也。涉浅水者见虾,其颇深者察鱼鳖,其尤甚者观蛟龙。足行迹殊,故所见之物异也。入道浅深,其犹此也,浅者则见传记谐文,深者入圣室观秘书。故入道弥深,所见弥大。人之游也,必欲入都,都多奇观也。入都必欲见市,市多异货也。百家之言,古今行事,其为奇异,非徒都邑大市也。游於都邑者心厌,观於大市者意饱,况游於道艺之际哉?污大川旱不枯者,多所疏也。潢污兼日不雨,泥辄见者,无所通也。是故大川相间,小川相属,东流归海,故海大也。海不通於百川,安得巨大之名?夫人含百家之言,犹海怀百川之流也,不谓之大者,是谓海小於百川也。夫海大於百川也,人皆知之,通者明於不通,莫之能别也。润下作咸,水之滋味也。东海水咸,流广大也;西州盐井,源泉深也。人或无井而食,或穿井不得泉,有盐井之利乎?不与贤圣通业,望有高世之名,难哉!法令之家,不见行事,议罪不审。章句之生,不览古今,论事不实。或以说一经为〔足〕,何须博览。夫孔子之门,讲习《五经》。《五经》皆习,庶几之才也。

颜渊曰:``博我以文。''才智高者,能为博矣。颜渊之曰博者,岂徒一经哉?我不能博《五经》,又不能博众事,守信一学,不好广观,无温故知新之明,而有守愚不览之暗。其谓一经〔足〕者,其宜也。开户内日之光,日光不能照幽,凿窗启牖,以助户明也。夫一经之说,犹日明也,助以传书,犹窗牖也。百家之言令人晓明,非徒窗牖之开日光之照也。是故日光照室内,道术明胸中。开户内光,坐高堂之上,眇升楼台,窥四邻之廷,人之所愿也。闭户幽坐,向冥冥之内,穿圹穴卧,造黄泉之际,人之所恶也。夫闭心塞意,不高瞻览者,死人之徒也哉!孝武皇帝时,燕王旦在明光宫,欲入所卧,户三尽闭,使侍者二十人开户,户不开,其後旦坐谋反自杀。夫户闭,燕王旦死之状也。死者,凶事也,故以闭塞为占。齐庆封不通,六国大夫会而赋诗,庆封不晓,其後果有楚灵之祸也。夫不开通於学者,尸尚能行者也。亡国之社,屋其上、柴其下者,示绝於天地。《春秋》薄社,周以为城。夫经艺传书,人当览之,犹社当通气於天地也。故人之不通览者,薄社之类也。是故气不通者,强壮之人死,荣华之物枯。

东海之中,可食之物,集糅非一,以其大也。夫水精气渥盛,故其生物也众多奇异。故夫大人之胸怀非一,才高知大,故其於道术无所不包。学士同门高业之生,众共宗之。何则?知经指深,晓师言多也。夫古今之事,百家之言,其为深,多也,岂徒师门高业之生哉?甘酒醴不酤饴蜜,未为能知味也。耕夫多殖嘉谷,谓之上农夫;其少者,谓之下农夫。学士之才,农夫之力,一也。能多种谷,谓之上农,能博学问,〔不〕谓之上儒,是称牛之服重,不誉马速也。誉手毁足,孰谓之慧矣!县道不通於野,野路不达於邑,骑马乘舟者,必不由也。故血脉不通,人以甚病。夫不通者,恶事也,故其祸变致不善。是故盗贼宿於秽草,邪心生於无道,无道者,无道术也。医能治一病谓之巧,能治百病谓之良。是故良医服百病之方,治百人之疾;大才怀百家之言,故能治百族之乱。扁鹊之众方,孰若巧〔医〕之一伎?子贡曰:``不得其门而入,不见宗庙之美,百官之富。''盖以宗庙百官喻孔子道也。孔子道美,故譬以宗庙,众多非一,故喻以百官。由此言之,道达广博者,孔子之徒也。

殷、周之地,极五千里,荒服、要服,勤能牧之。汉氏廓土,牧万里之外,要荒之地,褒衣博带。夫德不优者,不能怀远,才不大者,不能博见。故多闻博识,无顽鄙之訾;深知道术,无浅暗之毁也。人好观图画者,图上所画,古之列人也。见列人之面,孰与观其言行?置之空壁,形容具存,人不激劝者,不见言行也。古贤之遗文,竹帛之所载粲然,岂徒墙壁之画哉?空器在厨,金银涂饰,其中无物益於饥,人不顾也。肴膳甘醢,土釜之盛,入者乡之。古贤文之美善可甘,非徒器中之物也,读观有益,非徒膳食有补也。故器空无实,饥者不顾,胸虚无怀,朝廷不御也。剑伎之家,斗战必胜者,得曲城、越女之学也。两敌相遭,一巧一拙,其必胜者,有术之家也。孔、墨之业,贤圣之书,非徒曲城、越女之功也。成人之操,益人之知,非徒战斗必胜之策也。故剑伎之术,有必胜之名;贤圣之书,有必尊之声。县邑之吏,召诸治下,将相问以政化,晓慧之吏,陈所闻见,将相觉悟,得以改政右文。圣贤言行,竹帛所传,练人之心,聪人之知,非徒县邑之吏对向之语也。

禹、益并治洪水,禹主治水,益主记异物,海外山表,无远不至,以所闻见作《山海经》。非禹、益不能行远,《山海》不造。然则《山海》之造,见物博也。董仲舒睹重常之鸟,刘子政晓贰负之尸,皆见《山海经》,故能立二事之说。使禹、益行地不远,不能作《山海经》;董、刘不读《山海经》,不能定二疑。实沉、台台,子产博物,故能言之;龙见绛郊,蔡墨晓占,故能御之。父兄在千里之外,且死,遗教戒之书,子弟贤者,求索观读,服臆不舍,重先敬长,谨慎之也;不肖者轻慢佚忽,无原察之意。古圣先贤,遗後人文字,其重非徒父兄之书也,或观读采取,或弃捐不录,二者之相高下也,行路之人,皆能论之,况辩照然否者不能别之乎?孔子病,商瞿卜期日中,孔子曰:``取书来,比至日中何事乎?''圣人之好学也,且死不休,念在经书,不以临死之故,弃忘道艺,其为百世之圣,师法祖修,盖不虚矣!自孔子以下,至汉之际,有才能之称者,非有饱食终日无所用心也,不说《五经》则读书传。书传文大,难以备之。卜卦占射凶吉,皆文、武之道。昔有商瞿能占爻卦;末有东方朔、翼少君,能达占射覆。道虽小,亦圣人之术也。曾又不知人生禀五常之性,好道乐学,故辨於物。今则不然,饱食快饮,虑深求卧,腹为饭坑,肠为酒襄,是则物也。倮虫三百,人为之长,``天地之性人为贵,贵其识知也。今闭暗脂塞,无所好欲,与三百倮虫何以异?而谓之为长而贵之乎!

诸夏之人所以贵於夷狄者,以其通仁义之文,知古今之学也。如徒〔任〕其胸中之知以取衣食,经厉年月,白首没齿,终无晓知,夷狄之次也。观夫蜘蛛之经丝以罔飞虫也,人之用作,安能过之?任胸中之知,舞权利之诈,以取富寿之乐,无古今之学,蜘蛛之类也。含血之虫,无饿死之患,皆能以知求索饮食也。人不通者,亦能自供,仕官为吏,亦得高官,将相长吏,犹吾大夫高子也,安能别之?随时积功,以命得官,不晓古今,以位为贤,与文〔人〕异术,安得识别通人,俟以不次乎?将相长吏不得若右扶风蔡伯偕、郁林太守张孟尝、东莱太守李季公之徒,心自通明,览达古今,故其敬通人也如见大宾。燕昭为邹衍拥彗,彼独受何性哉?东成令董仲绶知为儒枭,海内称通,故其接人,能别奇〔伟〕。是以锺离产公以编户之民,受圭璧之敬,知之明也。故夫能知之也,凡石生光气;不知之也,金玉无润色。自武帝以至今朝,数举贤良,令人射策甲乙之科,若董仲舒、唐子高、谷子云、丁伯玉,策既中实,文说美善,博览膏腴之所生也。使四者经徒能摘,笔徒能记疏,不见古今之书,安能建美善於圣王之庭乎?孝明之时,读《苏武传》,见武官名曰《栘中监》,以问百官,百官莫知。夫《仓颉》之章,小学之书,文字备具,至於无能对圣国之问者,是皆美命随牒之人多在官也。``木''旁``多''文字且不能知,其欲及若董仲舒之知重常,刘子政之知贰负,难哉!或曰:``通人之官,兰台令史,职校书定字,比夫太史、太柷,职在文书,无典民之用,不可施设。是以兰台之史,班固、贾逵、杨终、傅毅之徒,名香文美,委积不泄,大用於世。''曰:此不继。周世通览之人,邹衍之徒,孙卿之辈,受时王之宠,尊显於世。董仲舒虽无鼎足之位,知在公卿之上。周监二代,汉监周、秦然则兰台之官,国所监得失也。以心如丸卵,为体内藏;眸子如豆,为身光明。令史虽微,典国道藏,通人所由进,犹博士之官,儒生所由兴也。委积不绁,岂圣国微遇之哉,殆以书未定而职未毕也。

\hypertarget{header-n574}{%
\subsubsection{超奇篇}\label{header-n574}}

通书千篇以上,万卷以下,弘暢雅闲,审定文读,而以教授为人师者,通人也。杼其义旨,损益其文句,而以上书奏记,或兴论立说、结连篇章者,文人鸿儒也。好学勤力,博闻强识,世间多有;著书表文,论说古今,万不耐一。然则著书表文,博通所能用之者也。入山见木,长短无所不知;入野见草,大小无所不识。然而不能伐木以作室屋,采草以和方药,此知草木所不能用也。夫通人览见广博,不能掇以论说,此为匿生书主人,孔子所谓``诵《诗》三百,授之以政不达''者也,与彼草木不能伐采,一实也。孔子得《史记》以作《春秋》,及其立义创意,褒贬赏诛,不复因《史记》者,眇思自出於胸中也。凡贵通者,贵其能用之也,即徒诵读,读诗讽术虽千篇以上,鹦鹉能言之类也。衍传书之意,出膏腴之辞,非俶傥之才,不能任也。夫通览者,世间比有;著文者,历世希然。近世刘子政父子、扬子云、桓君山,其犹文、武、周公并出一时也;其余直有,往往而然,譬珠玉不可多得,以其珍也。故夫能说一经者为儒生,博览古今者为通人,采掇传书以上书奏记者为文人,能精思著文连结篇章者为鸿儒。故儒生过俗人,通人胜儒生,文人逾通人,鸿儒超文人。故夫鸿儒,所谓超而又超者也。以超之奇,退与儒生相料,文轩之比於敝车,锦绣之方於缊袍也,其相过,远矣。如与俗人相料,太山之巅墆,长狄之项跖,不足以喻。故夫丘山以土石为体,其有铜铁,山之奇也。铜铁既奇,或出金玉。然鸿儒,世之金玉也,奇而又奇矣。奇而又奇,才相超乘,皆有品差。

儒生说名於儒门,过俗人远也。或不能说一经,教诲後生。或带徒聚众,说论洞溢,称为经明。或不能成牍,治一说。或能陈得失,奏便宜,言应经传,文如星月。其高第若谷子云、唐子高者,说书於牍奏之上,不能连结篇章。或抽列古今,纪著行事,若司马子长、刘子政之徒,累积篇第,文以万数,其过子云、子高远矣。然而因成纪前,无胸中之造。若夫陆贾、董仲舒,论说世事,由意而出,不假取於外,然而浅露易见,观读之者,犹曰传记。阳成子长作《乐经》,扬子云作《太玄经》,造於〔眇〕思,极窅冥之深,非庶几之才,不能成也。孔子作《春秋》,二子作两经,所谓卓尔蹈孔子之迹,鸿茂参贰圣之才者也。王公问於桓君山以扬子云。君山对曰:``汉兴以来,未有此人。''君山差才,可谓得高下之实矣。采玉者心羡於玉,钻龟能知神於龟。能差众儒之才,累其高下,贤於所累。又作《新论》,论世间事,辩照然否,虚妄之言,伪饰之辞,莫不证定。彼子长、子云论说之徒,君山为甲。自君山以来,皆为鸿眇之才,故有嘉令之文。笔能著文,则心能谋论,文由胸中而出,心以文为表。观见其文,奇伟俶傥,可谓得论也。由此言之,繁文之人,人之杰也。

有根株於下,有荣叶於上;有实核於内,有皮壳於外。文墨辞说,士之荣叶、皮壳也。实诚在胸臆,文墨著竹帛,外内表里,自相副称。意奋而笔纵,故文见而实露也。人之有文也,犹禽之有毛也。毛有五色,皆生於体。苟有文无实,是则五色之禽,毛妄生也。选士以射,心平体正,执弓矢审固,然後射中。论说之出,犹弓矢之发也;论之应理,犹矢之中的。夫射以矢中效巧,论以文墨验奇。奇巧俱发於心,其实一也。文有深指巨略,君臣治术,身不得行,口不能〔泄〕,表著情心,以明己之必能为之也。孔子作《春秋》,以示王意。然则孔子之《春秋》,素王之业也;诸子之传书,素相之事也。观《春秋》以见王意,读诸子以睹相指。故曰:陈平割肉,丞相之端见;叔孙敖决期思,令〔尹〕之兆著。观读传书之文,治道政务,非徒割肉决水之占也。足不强则迹不远,锋不銛,则割不深。连结篇章,必大才智鸿懿之俊也。

或曰:著书之人,博览多闻,学问习熟,则能推类兴文。文由外而兴,未必实才学文相副也。且浅意於华叶之言,无根核之深,不见大道体要,故立功者希。安危之际,文人不与,无能建功之验,徒能笔说之效也。曰:此不然。周世著书之人皆权谋之臣,汉世直言之士皆通览之吏,岂谓文非华叶之生,根核推之也?心思为谋,集扎为文,情见於辞,意验於言。商鞅相秦,致功於霸,作《耕战》之书。虞卿为赵,决计定说,行退作春秋之思,起城中之议。《耕战》之书,秦堂上之计也。陆贾消吕氏之谋,与《新语》同一意。桓君山易晁错之策,与《新论》共一思。观谷永之陈说,唐林之宜言,刘向之切议,以知为本,笔墨之文,将而送之,岂徒雕文饰辞,苟为华叶之言哉?精诚由中,故其文语感动人深。是故鲁连飞书,燕将自杀;邹阳上疏,梁孝开牢。书疏文义,夺於肝心,非徒博览者所能造,习熟者所能为也。夫鸿儒希有,而文人比然,将相长吏,安可不贵?岂徒用其才力,游文於牒牍哉?州郡有忧,能治章上奏,解理结烦,使州郡连事,有如唐子高、谷子云之吏,出身尽思,竭笔牍之力,烦忧适有不解者哉?

古昔之远,四方辟匿,文墨之士,难得纪录,且近自以会稽言之,周长生者,文士之雄也,在州,为刺史任安举奏;在郡,为太守孟观上书,事解忧除,州郡无事,二将以全。长生之身不尊显,非其才知少、功力薄也,二将怀俗人之节,不能贵也。使遭前世燕昭,则长生已蒙邹衍之宠矣。长生死後,州郡遭忧,无举奏之吏,以故事结不解,征诣相属,文轨不尊,笔疏不续也。岂无忧上之吏哉?乃其中文笔不足类也。长生之才,非徒锐於牒牍也,作《洞历》十篇,上自黄帝,下至汉朝,锋芒毛发之事,莫不纪载,与太吏公《表》、《纪》相似类也。上通下达,故曰《洞历》。然则长生非徒文人,所谓鸿儒者也。前世有严夫子,後有吴君〔高〕,末有周长生。白雉贡於越,暢草献於宛,雍州出玉,荆、扬生金。珍物产於四远幽辽之地,未可言无奇人也。孔子曰:``文王既没,文不在兹乎!''
文王之文在孔子,孔子之文在仲舒。仲舒既死,岂在长生之徒与?何言之卓殊,文之美丽也!唐勒、宋玉,亦楚文人也,竹帛不纪者,屈原在其上也。会稽文才,岂独周长生哉?所以未论列者,长生尤逾出也。九州多山,而华、岱为岳,四方多川,而江、河为渎者,华、岱高而江、河大也。长生,州郡高大者也。同姓之伯贤,舍而誉他族之孟,未为得也。长生说文辞之伯,文人之所共宗,独纪录之,《春秋》记元於鲁之义也。俗好高古而称所闻,前人之业,菜果甘甜;後人新造,蜜酪辛苦。长生家在会稽,生在今世,文章虽奇,论者犹谓稚於前人。天禀元气,人受元精,岂为古今者差杀哉?优者为高,明者为上,实事之人,见然否之分者,睹非却前,退置於後,见是,推今进置於古,心明知昭,不惑於俗也。

班叔皮续《太史公书》百篇以上,记事详悉,义浅理备。观读之者以为甲,而太史公乙。子男孟坚为尚书郎,文比叔皮,非徒五百里也,乃夫周、召、鲁、卫之谓也。苟可高古,而班氏父子不足纪也。周有郁郁之文者,在百世之末也。汉在百世之後,文论辞说,安得不茂?喻大以小,推民家事,以睹王廷之义。庐宅始成,桑麻才有,居之历岁,子孙相续,桃李梅杏,〔奄〕丘蔽野。根茎众多,则华叶繁茂。汉氏治定久矣,土广民众,义兴事起,华叶之言,安得不繁?夫华与实,俱成者也,无华生实,物希有之。山之秃也,孰其茂也?地之泻也,孰其滋也?文章之人,滋茂汉朝者乃夫汉家炽盛之瑞也。天晏,列宿焕炳;阴雨,日月蔽匿。方今文人并出见者,乃夫汉朝明明之验也。高祖读陆贾之书,叹称万岁;徐乐、主父偃上疏,征拜郎中,方今未闻。膳无苦酸之肴,口所不甘味,手不举以啖人。诏书每下,文义经传四科,诏书斐然,郁郁好文之明验也。上书不实核,著书无义指,``万岁''之声,``征拜''之恩,何从发哉?饰面者皆欲为好,而运目者希;文音者皆欲为悲,而惊耳者寡。陆贾之书未奏,徐乐、主父之策未闻,群诸瞽言之徒,言事粗丑,文不美润,不指。所谓,文辞淫滑,不被涛沙之谪,幸矣!焉蒙征拜为郎中之宠乎?

\hypertarget{header-n584}{%
\subsection{卷十四}\label{header-n584}}

\hypertarget{header-n585}{%
\subsubsection{状留篇}\label{header-n585}}

论贤儒之才,既超程矣,世人怪其仕宦不进,官爵卑细。以贤才退在俗吏之後,信〔可〕怪也。夫如是,而适足以见贤不肖之分,睹高下多少之实也。龟生三百岁,大如钱,游於莲叶之上。三千岁青边缘,巨尺二寸。蓍生七十岁生一茎,七百岁生十茎。神灵之物也,故生迟留,历岁长久,故能明审。实贤儒之在世也,犹灵蓍、神龟也。计学问之日,固已尽年之半矣。锐意於道,遂无贪仕之心。及其仕也,纯特方正,无员锐之操。故世人迟取进难也。针锥所穿,无不暢达。使针锥末方,穿物无一分之深矣。贤儒方节而行,无针锥之锐,固安能自穿、取暢达之功乎?且骥一日行千里者,无所服也,使服任车舆,驽马同〔昔〕。骥曾以引盐车矣,垂头落汗,行不能进。伯乐顾之,王良御之,空身轻驰,故有千里之名。今贤儒怀古今之学,负荷礼义之重,内累於胸中之知,外劬於礼义之操,不敢妄进苟取,故有稽留之难。无伯乐之友,不遭王良之将,安得驰於清明之朝,立千里之迹乎?

且夫含血气物之生也,行则背在上而腹在下;其病若死,则背在下而腹在上。何则?背肉厚而重,腹肉薄而轻也。贤儒、俗吏,并在当世,有似於此。将明道行,则俗吏载贤儒,贤儒乘俗吏。将暗道废,则俗吏乘贤儒,贤儒处下位,犹物遇害,腹在上而背在下也。且背法天而腹法地,生行得其正,故腹背得其位;病死失其宜,故腹反而在背上。

非唯腹也,凡物仆僵者,足又在上。贤儒不遇,仆废於世,踝足之吏,皆在其上。东方朔曰:``目不在面而在於足,救昧不给,能何见乎?''汲黯谓武帝曰:
``陛下用吏如积薪矣,後来者居上。''原汲黯之言,察东方朔之语,独〔非〕以俗吏之得地,贤儒之失职哉?故夫仕宦,失地难以观德;得地难以察不肖。名生於高官,而毁起於卑位。卑位,固赏贤儒之所在也。遵礼蹈绳,修身守节,在下不汲汲,故有沉滞之留。沉滞在能自济,故有不拔之扼。其积学於身也多,故用心也固。俗吏无以自修,身虽拔进,利心摇动,则有下道侵渔之操矣。

枫桐之树,生而速长,故其皮肌不能坚刚。树檀以五月生叶,後彼春荣之木,其材强劲,车以为轴。殷之桑谷,七日大拱,长速大暴,故为变怪。大器晚成,宝货难售也。不崇一朝,辄成贾者,菜果之物也。是故湍濑之流,沙石转而大石不移。何者?大石重而沙石轻也。沙石转积於大石之上,大石没而不见。贤儒俗吏,并在世俗,有似於此。遇暗长吏,转移俗吏超在贤儒之上,贤儒处下,受驰走之使,至或岩居穴处,没身不见。咎在长吏不能知贤,而贤者道大,力劣不能拔举之故也。

夫手指之物器也,度力不能举,则不敢动。贤儒之道,非徒物器之重也。是故金铁在地,焱风不能动,毛芥在其间,飞扬千里。夫贤儒所怀,其犹水中大石、在地金铁也。其进不若俗吏速者,长吏力劣,不能用也。毛芥在铁石间也,一口之气,能吹毛芥,非必焱风。俗吏之易迁,犹毛芥之易吹也。故夫转沙石者,湍濑也;飞毛芥者,焱风也。活水洋风,毛芥不动。无道理之将,用心暴猥,察吏不详,遭以奸迁,妄授官爵,猛水之转沙石,焱风之飞毛芥也。是故毛芥因异风而飞,沙石遭猛流而转,俗吏遇悖将而迁。

且圆物投之於地,东西南北,无之不可,策杖叩动,才微辄停。方物集地,壹投而止;及其移徒,须人动举。贤儒,世之方物也,其难转移者,其动须人也。鸟轻便於人,趋远,人不如鸟,然而天地之性人为贵。蝗虫之飞,能至万里;麒麟须献,乃达阙下。然而蝗虫为灾,麒麟为瑞。麟有四足,尚不能自致,人有两足,安能自达?故曰:燕飞轻於凤皇,兔走疾於麒麟,\{圭黾\}跃躁於灵龟,蛇腾便於神龙。吕望之徒,白首乃显;百里奚之知,明於黄发:深为国谋,因为王辅,皆夫沉重难进之人也。轻躁早成,祸害暴疾。故曰:其进锐者,退速。阳温阴寒,历月乃至;灾变之气,一朝成怪。故夫河冰结合,非一日之寒;积土成山,非斯须之作。干将之剑,久在炉炭,銛锋利刃,百熟炼历。久销乃见作留,成迟故能割断。肉暴长者曰肿,泉暴出者曰涌,酒暴熟者易酸,醢暴酸者易臭。由此言之,贤儒迟留,皆有状故。状故云何?学多道重,为身累也。

草木之生者湿,湿者重;死者枯。枯而轻者易举,湿而重者难移也。然元气所在,在生不在枯。是故车行於陆,船行於沟,其满而重者行迟,空而轻者行疾。先王之道,载在胸腹之内,其重不徒船车之任也。任重,其取进疾速,难矣。窃人之物,其得非不速疾也,然而非其有,得之非己之力也。世人早得高官,非不有光荣也,而尸禄素餐之谤,喧哗甚矣。且贤儒之不进,将相长吏不开通也。农夫载谷奔都,贾人赍货赴远,皆欲得其愿也。如门郭闭而不通,津梁绝而不过,虽有勉力趋时之势,奚由早至以得盈利哉?长吏妒贤,不能容善,不被钳赭之刑,幸矣,焉敢望官位升举,道理之早成也?

\hypertarget{header-n595}{%
\subsubsection{寒温篇}\label{header-n595}}

说寒温者曰:人君喜则温,怒则寒。何则?喜怒发於胸中,然後行出於外,外成赏罚。赏罚,喜怒之效。故寒温渥盛,雕物伤人。夫寒温之代至也,在数日之间,人君未必有喜怒之气发胸中,然後渥盛於外。见外寒温,则知胸中之气也。当人君喜怒之时,胸中之气未必更寒温也。胸中之气,何以异於境内之气?胸中之气,不为喜怒变,境内寒温,何所生起?六国之时,秦、汉之际,诸侯相伐,兵革满道,国有相攻之怒,将有相胜之志,夫有相杀之气,当时天下未必常寒也。太平之世,唐、虞之时,政得民安,人君常喜,弦歌鼓舞,比屋而有,当时天下未必常温也。岂喜怒之气,为小发,不为大动邪?何其不与行事相中得也?

夫近水则寒,近火则温,远之渐微。何则?气之所加,远近有差也。成事,火位在南,水位在北,北边则寒,南极则热。火之在炉,水之在沟,气之在躯,其实一也。当人君喜怒之时,寒温之气,闺门宜甚,境外宜微。今案寒温,外内均等,殆非人君喜怒之所致。世儒说称,妄处之也。王者之变在天下,诸侯之变在境内,卿大夫之变在其位,庶人之变在其家。夫家人之能致变,则喜怒亦能致气。父子相怒,夫妻相督,若当怒反喜,纵过饰非,一室之中,宜有寒温。由此言之,变非喜怒所生,明矣。

或曰:``以类相招致也。喜者和温,和温赏赐,阳道施予,阳气温,故温气应之。怒者愠恚,愠恚诛杀。阴道肃杀,阴气寒,故寒气应之。虎啸而谷风至,龙兴而景云起。同气共类,动相招致。故曰:`以形逐影,以龙致雨'。雨应龙而来,影应形而去。天地之性,自然之道也。秋冬断刑,小狱微原,大辟盛寒,寒随刑至,相招审矣。''夫比寒温於风云,齐喜怒於龙虎,同气共类,动相招致,可矣。虎啸之时,风从谷中起;龙兴之时,云起百里内。他谷异境,无有风云。今寒温之变,并时皆然。百里用刑,千里皆寒,殆非其验。齐、鲁接境,赏罚同时,设齐赏鲁罚,所致宜殊,当时可齐国温、鲁地寒乎?

案前世用刑者,蚩尤、亡秦甚矣。蚩尤之民,湎湎纷纷;亡秦之路,赤衣比肩,当时天下未必常寒也。帝都之市,屠杀牛羊,日以百数,刑人杀牲,皆有贼心,帝都之市,气不能寒。或曰:``人贵於物,唯人动气。''夫用刑者动气乎?用受刑者为变也?如用刑者,刑人杀禽,同一心也。如用受刑者,人禽皆物也,俱为万物,百贱不能当一贵乎?或曰:``唯人君动气,众庶不能。''夫气感必须人君,世何称於邹衍?邹衍匹夫,一人感气,世又然之。刑一人而气辄寒,生一人而气辄温乎?赦令四下,万刑并除,当时岁月之气不温。往年,万户失火,烟焱参天;河决千里,四望无垠。火与温气同,水与寒气类。失火河决之时,不寒不温。然则寒温之至,殆非政治所致。然而寒温之至,遭与赏罚同时,变复之家,因缘名之矣。

春温夏暑,秋凉冬寒,人君无事,四时自然。夫四时非政所为,而谓寒温独应政治?正月之始,正月之后,立春之际,百刑皆断,囹圄空虚。然而一寒一温,当其寒也,何刑所断?当其温也,何赏所施?由此言之,寒温,天地节气,非人所为,明矣。

人有寒温之病,非操行之所及也。遭风逢气,身生寒温。变操易行,寒温不除。夫身近而犹不能变除其疾,国邑远矣,安能调和其气?人中於寒,饮药行解,所苦稍衰;转为温疾,吞发汗之丸而应愈。燕有寒谷,不生五谷。邹衍吹律,寒谷可种。燕人种黍其中,号曰黍谷。如审有之,寒温之灾,复以吹律之事,调和其气,变政易行,何能灭除?是故寒温之疾,非药不愈;黍谷之气,非律不调。尧遭洪水,使禹治之。寒温与尧之洪水,同一实也。尧不变政易行,知夫洪水非政行所致。洪水非政行所致,亦知寒温非政治所招。

或难曰:《洪范》庶征曰:``急,恆寒若;舒,恆燠若。''若,顺;燠,温;恆,常也。人君急,则常寒顺之;舒,则常温顺之。寒温应急舒,谓之非政,如何?夫岂谓急不寒、舒不温哉?人君急舒而寒温递至,偶适自然,若故相应,犹卜之得兆、筮之得数也。人谓天地应令问,其实适然。夫寒温之应急舒,犹兆数之应令问也。外若相应,其实偶然。何以验之?夫天道自然,自然无为。二令参偶,遭适逢会,人事始作,天气已有,故曰道也。使应政事,是有,非自然也。《易》京氏布六十卦於一岁中,六日七分,一卦用事。卦有阴阳,气有升降。阳升则温,阴升则寒。由此言之,寒温随卦而至,不应政治也。案《易》无妄之应,水旱之至,自有期节。百灾万变,殆同一曲。变复之家,疑且失实。何以为疑?夫大人与天地合德,先天而天不违,後天而奉天时。《洪范》曰:``急,恆寒若;舒,恆燠若。''如《洪范》之言,天气随人易徒,当先天而天不违耳,何故复言後天而奉天时乎?後者,天已寒温於前,而人赏罚於後也。由此言之,人言与《尚书》不合,一疑也。京氏占寒温以阴阳升降,变复之家以刑赏喜怒,两家乖迹,二疑也。民间占寒温,今日寒而明日温,朝有繁霜,夕有列光,旦雨气温,旦旸气寒。夫雨者阴,旸者阳也;寒者阴,而温者阳也。雨旦旸反寒,旸旦雨反温,不以类相应,三疑也。三疑不定,``自然''之说,亦未立也。

\hypertarget{header-n605}{%
\subsubsection{谴告篇}\label{header-n605}}

论灾异,谓古之人君为政失道,天用灾异谴告之也。灾异非一,复以寒温为之效。人君用刑非时则寒,施赏违节则温。天神谴告人君,犹人君责怒臣下也。故楚〔庄〕王曰:``天不下灾异,天其忘〔予〕乎!''灾异为谴告,故〔庄〕王惧而思之也。曰:此疑也。夫国之有灾异也,犹家人之有变怪也。有灾异,谓天谴人君;有变怪,天复谴告家人乎?家人既明,人之身中,亦将可以喻。身中病,犹天有灾异也。血脉不调,人生疾病;风气不和,岁生灾异。灾异谓天谴告国政,疾病天复谴告人乎?酿酒於罂,烹肉於鼎,皆欲其气味调得也。时或咸苦酸淡不应口者,犹人芍药失其和也。夫政治之有灾异也,犹烹酿之有恶味也。苟谓灾异为天谴告,是其烹酿之误,得见谴告也。占大以小,明物事之喻,足以审天。使〔庄〕王知如孔子,则其言可信。衰世霸者之才,犹夫变复之家也,言未必信,故疑之。

夫天道,自然也,无为。如谴告人,是有为,非自然也。黄、老之家,论说天道,得其实矣。且天审能谴告人君,宜变易其气以觉悟之。用刑非时,刑气寒,而天宜为温;施赏违节,赏气温,而天宜为寒。变其政而易其气,故君得以觉悟,知是非。今乃随寒从温,为寒为温,以谴告之意,欲令变更之且。太王父以王季之可立,故易名为历。历者,适也。太伯觉悟,之吴、越采药,以避王季。使太王不易季名,而复字之季,太伯岂觉悟以避之哉?今刑赏失法,天欲改易其政,宜为异气,若太王之易季名。今乃重为同气以谴告之,人君何时将能觉悟,以见刑赏之误哉?

鼓瑟者误於张弦设柱,宫商易声,其师知之,易其弦而复移其柱。夫天之见刑赏之误,犹瑟师之睹弦柱之非也。不更变气以悟人君,反增其气以渥其恶,则天无心意,苟随人君为误非也。纣为长夜之饮,文王朝夕曰:``祀兹酒。''齐奢於祀,晏子祭庙,豚不掩俎。何则?非疾之者,宜有以改易之也。子弟傲慢,父兄教以谨敬;吏民横悖,长吏示以和顺。是故康叔、伯禽失子弟之道,见於周公,拜起骄悖,三见三笞;往见商子,商子令观桥梓之树。二子见桥梓,心感觉悟,以知父子之礼。周公可随为骄,商子可顺为慢,必须加之捶杖,教观於物者,冀二人之见异,以奇自觉悟也。夫人君之失政,犹二子失道也,天不告以政道,令其觉悟,若二子观见桥梓,而顾随刑赏之误,为寒温之报,此则天与人君俱为非也。无相觉悟之感,有相随从之气,非皇天之意,爱下谴告之宜也。

凡物能相割截者,必异性者也;能相奉成者,必同气者也。是故《离》下、《兑》上曰革。革,更也。火金殊气,故能相革。如俱火而皆金,安能相成?屈原疾楚之臰洿,故称香洁之辞;渔父议以不随俗,故陈沐浴之言。凡相溷者,或教之熏隧,或令之负豕。二言之於除臰也,孰是孰非,非有不易,少有以益。夫用寒温,非刑赏也,能易之乎?

西门豹急,佩韦以自宽;董安於缓,带弦以自促。二贤知佩带变己之物,而以攻身之短。〔天〕至明矣,人君失政,不以他气谴告变易,反随其误,就起其气,此则皇天用意,不若二贤审也。楚庄王好猎,樊姬为之不食鸟兽之肉;秦缪公好淫乐,华阳後为之不听郑、卫之音。二姬非两主,拂其欲而不顺其行.
皇天非赏罚,而顺其操,而渥其气:此盖皇天之德,不若妇人贤也。

故谏之为言,``间''也,持善间恶,必谓之一乱。周缪王任刑,《甫刑篇》曰:``报虐用威。''威虐皆恶也,用恶报恶,乱莫甚焉。今刑失赏宽,恶也,〔天〕复为恶以应之,此则皇天之操,与缪王同也。故以善驳恶,以恶惧善,告人之理,劝厉为善之道也。舜戒禹曰:``毋若丹硃敖。''周公敕成王曰:``毋若殷王纣!''毋者,禁之也。丹硃、殷纣至恶,故曰``毋''以禁之。夫言``毋若'',孰与言必若哉?故毋必二辞,圣人审之。况肯谴非为非,顺人之过,以增其恶哉?天人同道,大人与天合德。圣贤以善反恶,皇天以恶随非,岂道同之效、合德之验哉?

孝武皇帝好仙,司马长卿献《大人赋》,上乃仙仙有凌云之气。孝成皇帝好广宫室,扬子云上《甘泉颂》,妙称神怪,若曰非人力所能为,鬼神力乃可成。皇帝不觉,为之不止。长卿之赋,如言仙无实效,子云之颂言奢有害,孝武岂有仙仙之气者,孝成岂有不觉之惑哉?然即天之不为他气以谴告人君,反顺人心以非应之,犹二子为赋颂,令两帝惑而不悟也。窦婴、灌夫疾时为邪,相与日引绳以纠缠之。心疾之甚,安肯从其欲?太伯教吴冠带,孰与随从其俗与之俱倮也?故吴之知礼义也,太伯改其俗也。苏武入匈奴,终不左衽;赵他入南越,箕踞椎髻。汉朝称苏武而毁赵他。之性习越土气,畔冠带之制,陆贾说之,夏服雅礼,风告以义,赵他觉悟,运心向内。如陆贾复越服夷谈,从其乱俗,安能令之觉悟,自变从汉制哉?三教之相违,文质之相反,政失,不相反袭也。谴告人君误,不变其失而袭其非,欲行谴告之教,不从如何?管、蔡篡畔,周公告教之至於再三。其所以告教之者,岂云当篡畔哉?人道善善恶恶,施善以赏,加恶以罪,天道宜然。刑赏失实,恶也,为恶气以应之,恶恶之义,安所施哉?汉正首匿之罪,制亡从之法,恶其随非而与恶人为群党也。如束罪人以诣吏,离恶人与异居,首匿亡从之法除矣。狄牙之调味也,酸则沃之以水,淡则加之以咸。水火相变易,故膳无咸淡之失也。今刑罚失实,不为异气以变其过,而又为寒於寒,为温於温,此犹憎酸而沃之以咸,恶淡而灌之以水也。由斯言之,谴告之言,疑乎?必信也?

今薪燃釜,火猛则汤热,火微则汤冷。夫政犹火,寒温犹热冷也。顾可言人君为政,赏罚失中也,逆乱阴阳,使气不和,乃言天为人君为寒为温以谴告之乎!儒者之说又言:``人君失政,天为异;不改,灾其人民;不改,乃灾其身也。先异後灾,先教後诛之义也。曰:此复疑也。以夏树物,物枯不生;以秋收谷,谷弃不藏。夫为政教,犹树物收谷也。顾可言政治失时,气物为灾;乃言天为异以谴告之,不改,为灾以诛伐之乎!儒者之说,俗人言也。盛夏阳气炽烈,阴气干之,激射\{敝衣\}裂,中杀人物。谓天罚阴过,外一闻若是,内实不然。夫谓灾异为谴告诛伐,犹为雷杀人罚阴过也。非谓之言,不然之说也。

或曰:谷子云上书陈言变异,明天之谴告,不改,後将复有,愿贯械待时。後竟复然。即不为谴告,何故复有?子云之言,故後有以示改也。曰:夫变异自有占候,阴阳物气自有终始。履霜以知坚冰必至,天之道也。子云识微,知後复然,借变复之说,以效其言,故愿贯械以待时也。犹齐晏子见钩星在房、心之间,则知地且动也。使子云见钩星,则将复曰:``天以钩星谴告政治,不改,将有地动之变矣。''然则子云之愿贯械待时,犹子韦之愿伏陛下,以俟荧惑徙,处必然之验,故谴告之言信也。予之谴告,何伤於义。损皇天之德,使自然无为转为人事,故难听之也。称天之谴告,誉天之聪察也,反以聪察伤损於天德。何以知其聋也?以其听之聪也。何以知其盲也?以其视之明也。何以知其狂也?以其言之当也。夫言当视听聪明,而道家谓之狂而盲聋。今言天之谴告,是谓天狂而盲聋也。

《易》曰:``大人与天地合其德。''故太伯曰:``天不言,殖其道於贤者之心。''夫大人之德,则天德也;贤者之言,则天言也。大人刺而贤者谏,是则天谴告也,而反归告於灾异,故疑之也。《六经》之文,圣人之语,动言天者,欲化无道、惧愚者。之言非独吾心,亦天意也。及其言天犹以人心,非谓上天苍苍之体也。变复之家,见诬言天,灾异时至,则生谴告之言矣。验古以〔今〕,知天以人。受终於文祖,不言受终於天。尧之心知天之意也。尧授之,天亦授之,百官臣子皆乡与舜。舜之授禹,禹之传启,皆以人心效天意。《诗》之``眷顾'',《洪范》之``震怒'',皆以人身效天之意。文、武之卒,成王幼少,周道未成,周公居摄,当时岂有上天之教哉?周公推心合天志也。上天之心,在圣人之胸;及其谴告,在圣人之口。不信圣人之言,反然灾异之气,求索上天之意,何其远哉?世无圣人,安所得圣人之言?贤人庶几之才,亦圣人之次也。

\hypertarget{header-n619}{%
\subsection{卷十五}\label{header-n619}}

\hypertarget{header-n620}{%
\subsubsection{变动篇}\label{header-n620}}

论灾异者,已疑於天用灾异谴告人矣。更说曰:``灾异之至,殆人君以政动天,天动气以应之。譬之以物击鼓,以椎扣锺,鼓犹天,椎犹政,锺鼓声犹天之应也。人主为於下,则天气随人而至矣。''曰:此又疑也。夫天能动物,物焉能动天?何则?人物系於天,天为人物主也。故曰:``王良策马,车骑盈野。''非车骑盈野,而乃王良策马也。天气变於上,人物应於下矣。故天且雨,商羊起舞,使天雨也。商羊者,知雨之物也,天且雨,屈其一足起舞矣。故天且雨,蝼蚁徙,丘蚓出,琴弦缓,固疾发,此物为天所动之验也。故在且风,巢居之虫动;且雨,穴处之物扰:风雨之气感虫物也。故人在天地之间,犹蚤虱之在衣裳之内,蝼蚁之在穴隙之中。蚤虱、蝼蚁为逆顺横从,能令衣裳穴隙之间气变动乎?蚤虱、蝼蚁不能,而独谓人能,不达物气之理也。

夫风至而树枝动,树枝不能致风。是故夏末蜻\{列虫\}鸣,寒螀啼,感阴气也。雷动而雉惊,〔蛰〕发而蛇出,起〔阳〕气也。夜及半而鹤唳,晨将旦而鸡鸣,此虽非变,天气动物,物应天气之验也。顾可言寒温感动人君,人君起气而以赏罚;乃言以赏罚感动皇天,天为寒温以应政治乎?六情风家言:``风至,为盗贼者感应之而起。''非盗贼之人精气感天,使风至也。风至怪不轨之心,而盗贼之操发矣。何以验之?盗贼之人,见物而取,睹敌而杀,皆在徙倚漏刻之间,未必宿日有其思也,而天风已以贪狼阴贼之日至矣。

以风占贵贱者,风从王相乡来则贵,从囚死地来则残。夫贵贱、多少,斗斛故也。风至而B谷之人贵贱其价,天气动怪人物者也。故谷价低昂,一贵一贱矣。《天官》之书,以正月朝占四方之风,风从南方来者旱,从北方来者湛,东方来者为疫,西方来者为兵。太史公实道言以风占水旱兵疫者,人物吉凶统於天也。使物生者,春也;物死者,冬也。春生而冬杀也。天〔也〕。如或欲春杀冬生,物终不死生,何也?物生统於阳,物死系於阴也。故以口气吹人,人不能寒;吁人,人不能温。使见吹吁之人,涉冬触夏,将有冻旸之患矣。寒温之气,系於天地,而统於阴阳。人事国政,安能动之?

且天本而人末也。登树怪其枝,不能动其株。如伐株,万茎枯矣。人事犹树枝,〔寒〕温犹根株也。生於天,含天之气,以天为主,犹耳目手足系於心矣。心有所为,耳目视听,手足动作,谓天应人,是谓心为耳目手足使乎?旌旗垂旒,旒缀於杆,杆东则旒随而西。苟谓寒温随刑罚而至,是以天气为缀旒也。钩星在房、心之间,地且动之占也。齐太卜知之,谓景公:``臣能动地。''景公信之。夫谓人君能致寒温,犹齐景公信太卜之能动地。夫人不能动地,而亦不能动天。

夫寒温,天气也。天至高大,人至卑小。篙不能鸣钟,而萤火不爨鼎者,何也?钟长而篙短,鼎大而萤小也。以七尺之细形,感皇天之大气,其无分铢之验,必也。占大将且入国邑,气寒,则将且怒,温则将喜。夫喜怒起事而发,未入界,未见吏民,是非未察,喜怒未发,而寒温之气已豫至矣。怒喜致寒温,怒喜之後,气乃当至。是竟寒温之气,使人君怒喜也。

或曰:``未至诚也。行事至诚,若邹衍之呼天而霜降,杞梁妻器而城崩,何天气之不能动乎?''夫至诚,犹以心意之好恶也。有果蓏之物,在人之前,去口一尺,心欲食之,口气吸之,不能取也;手掇送口,然後得之。夫以果之细,员圌易转,去口不远,至诚欲之,不能得也,况天去人高远,其气莽苍无端末乎!盛夏之时,当风而立,隆冬之月,向日而坐。其夏欲得寒而冬欲得温也,至诚极矣。欲之甚者,至或当风鼓C,向日燃炉,而天终不为冬夏易气,寒暑有节,不为人变改也。夫正欲得之而犹不能致,况自刑赏,意思不欲求寒温乎?

万人俱叹,未能动天,一邹衍之口,安能降霜?邹衍之状,孰与屈原?见拘之冤,孰与沉江?《离骚》《楚辞》凄怆,孰与一叹?屈原死时,楚国无霜,此怀、襄之世也。厉、武之时,卞和献玉,刖其两足,奉玉泣出,涕尽续之以血。夫邹衍之诚,孰与卞和?见拘之冤,孰与刖足?仰天而叹,孰与泣血?夫叹固不如泣,拘固不中刖,料计冤情,衍不如和,当时楚地不见霜。李斯、赵高谗杀太子扶苏,并及蒙恬、蒙骜。其时皆吐痛苦之言,与叹声同;又祸至死,非徒苟徙。而其死之地,寒气不生。秦坑赵卒於长平之下,四十万众,同时俱陷。当时啼号,非徒叹也。诚虽不及邹衍,四十万之冤,度当一贤臣之痛;入坑坎之啼,度过拘囚之呼。当时长平之下,不见陨霜。《甫刑》曰:``庶僇旁告无辜於天帝。''此言蚩尤之民被冤,旁告无罪於上天也。以众民之叫,不能致霜,邹衍之言,殆虚妄也。

南方至热,煎炒烂石,父子同水而浴。北方至寒,凝冰坼土,父子同穴而处。燕在北边,邹衍时,周之五月,正岁三月也。中州内正月二月霜雪时降。北边至寒,三月下霜,未为变也。此殆北边三月尚寒,霜适自降,而衍适呼,与霜逢会。传曰:``燕有寒谷,不生五谷。''邹衍吹律,寒谷复温,则能使气温,亦能使气复寒。何知衍不令时人知己之冤,以天气表己之诚,窃吹律於燕谷狱,令气寒而因呼天乎?即不然者,霜何故降?范雎为须贾所谗,魏齐僇之,折干摺胁。张仪游於楚,楚相掠之,被捶流血。二子冤屈,太史公列记其状。邹衍见拘,雎、仪之比也,且子长何讳不言?案《衍列传》,不言见拘而使霜降。伪书游言,犹太子丹使日再中、天雨粟也。由此言之,衍呼而降霜,虚矣!则杞梁之妻哭而崩城,妄也!

顿牟叛,赵襄子帅师攻之,军到城下,顿牟之城崩者十余丈,襄子击金而退之。夫以杞梁妻哭而城崩,襄子之军有哭者乎?秦之将灭,都门内崩;霍光家且败,第墙自坏。谁哭於秦宫,泣於霍光家者?然而门崩墙坏,秦、霍败亡之征也。或时杞国且圮,而杞梁之妻适哭城下,犹燕国适寒,而邹衍偶呼也。事以类而时相因,闻见之者或而然之。又城老墙朽,犹有崩坏。一妇之哭,崩五丈之城,是则一指摧三仞之楹也。春秋之时,山多变。山、城,一类也。哭能崩城,复能坏山乎?女然素缟而哭河,河流通。信哭城崩,固其宜也。案杞梁从军死,不归。其妇迎之,鲁君吊於途,妻不受吊,棺归於家,鲁君就吊,不言哭於城下。本从军死,从军死不在城中,妻向城哭,非其处也。然则杞梁之妻哭而崩城,复虚言也。

因类以及,荆轲〔刺〕秦王,白虹贯日;卫先生为秦画长平之计,太白食昴,复妄言也。夫豫子谋杀襄子,伏於桥下,襄子至桥心动。贯高欲杀高祖,藏人於壁中,高祖至柏人亦动心。二子欲刺两主,两主心动;綝实论之,尚谓非二子精神所能感也。而况荆轲欲刺秦王,秦王之心不动,而白虹贯日乎?然则白虹贯日,天变自成,非轲之精为虹而贯日也。钩星在房、心间,地且动之占也。地且动,钩星应房、心。夫太白食昴,犹钩星在房、心也。谓卫先生长平之议,令太白食昴,疑矣!岁星害鸟尾,周、楚恶之。然之气见,宋、卫、陈、郑灾。案时周、楚未有非,而宋、卫、陈、郑未有恶也。然而岁星先守尾,灾气署垂於天,其後周、楚有祸,宋、卫、陈、郑同时皆然。岁星之害周、楚,天气灾四国也。何知白虹贯日不致刺秦王,太白食昴〔不〕使长平计起也?

\hypertarget{header-n633}{%
\subsubsection{招致篇}\label{header-n633}}

已佚

\hypertarget{header-n637}{%
\subsubsection{明雩篇}\label{header-n637}}

变复之家,以久雨为湛,久旸为旱。旱应亢阳,湛应沈溺。或难曰:夫一岁之中,十日者一雨,五日者一风。雨颇留,湛之兆也。旸颇久,旱之渐也。湛之时,人君未必沈溺也;旱之时,未必亢阳也。人君为政,前後若一。然而一湛一早,时气也。范蠡计然曰:``太岁在〔於〕水,毁;金,穰;木,饥;火,旱。''
夫如是,水旱饥穰,有岁运也。岁直其运,气当其世,变复之家,指而名之。人君用其言,求过自改。旸久自雨,雨久自旸,变复之家,遂名其功;人君然之,遂信其术。试使人君恬居安处不求己过,天犹自雨,雨犹自旸。旸济雨济之时,人君无事,变复之家,犹名其术。是则阴阳之气,以人为主,不〔统〕於天也。夫人不能以行感天,天亦不随行而应人。《春秋》鲁大雩,旱求雨之祭也。旱久不雨,祷祭求福,若人之疾病祭神解祸矣。此变复也。《诗》云:``月离於毕,比滂沲矣。''《书》曰:``月之从星,则以风雨。''然则风雨随月所离从也。房星四表三道,日月之行,出入三道。出北则湛,出南则旱。或言出北则旱,南则湛。案月为天下占,房为九州候。月之南北,非独为鲁也。孔子出,使子路赍雨具。有顷,天果大雨。子路问其故,孔子曰:``昨暮月离於毕。''後日,月复离毕。孔子出,子路请赍雨具,孔子不听,出果无雨。子路问其故,孔子曰:``昔日,月离其阴,故雨。昨暮,月离其阳,故不雨。''夫如是,鲁雨自以月离,岂以政哉?如审以政令月,离於毕为雨占,天下共之。鲁雨,天下亦宜皆雨。六国之时,政治不同,人君所行赏罚异时,必以雨为应政令月,离六七毕星,然後足也。

鲁缪公之时,岁旱。缪公问县子:``天旱不雨,寡人欲暴巫,奚如?''县子不听。``欲徙市,奚如?''对曰:``天子崩,巷市七日;诸公薨,巷市五日。为之徙市,不亦可乎!''案县子之言,徙市得雨也。案《诗》、书之文,月离星得雨。日月之行,有常节度,肯为徙市故,离毕之阴乎?夫月毕,天下占。徙鲁之市,安耐移月?月之行天,三十日而周。一月之中,一过毕星,离阳则旸。假令徙市之感,能令月离毕〔阴〕,其时徙市而得雨乎。夫如县子言,未可用也。

董仲舒求雨,申《春秋》之义,设虚立祀,父不食於枝庶,天不食於下地。诸侯雩礼所祀,未知何神。如天神也,唯王者天乃歆,诸侯及今长吏,天不享也。神不歆享,安耐得神?如云雨者气也,云雨之气,何用歆享?触石而出,肤寸而合,不崇朝而辨雨天下,泰山也。泰山雨天下,小山雨国邑。然则大雩所祭,岂祭山乎?假令审然,而不得也。何以效之?水异川而居,相高分寸,不决不流,不凿不合。诚令人君祷祭水旁,能令高分寸之水流而合乎?夫见在之水,相差无几,人君请之,终不耐行。况雨无形兆,深藏高山,人君雩祭,安耐得之?

夫雨水在天地之间也,犹夫涕泣在人形中也。或赍酒食,请於惠人之前,〔求〕出其泣,惠人终不为之陨涕。夫泣不可请而出,雨安可求而得?雍门子悲哭,孟尝君为之流涕。苏秦、张仪悲说坑中,鬼谷先生泣下沾襟。或者傥可为雍门之声,出苏、张之说以感天乎!天又耳目高远,音气不通。杞梁之妻,又已悲哭,天不雨而城反崩。夫如是,竟当何以致雨?雩祭之家,何用感天?案月出北道,离毕之阴,希有不雨。由此言之,北道,毕星之所在也。北道星肯为雩祭之故下其雨乎?孔子出,使子路赍雨具之时,鲁未必雩祭也。不祭,沛然自雨;不求,旷然自旸。夫如是,天之旸雨,自有时也。一岁之中,旸雨连属。当其雨也,谁求之者?当其旸也,谁止之者?

人君听请,以安民施恩,必非贤也。天至贤矣,时未当雨,伪请求之,故妄下其雨,人君听请之类也。变复之家,不推类验之,空张法术,惑人君。或未当雨,而贤君求之而不得;或适当自雨,恶君求之,遭遇其时。是使贤君受空责,而恶君蒙虚名也。世称圣人纯而贤者驳,纯则行操无非,无非则政治无失。然而世之圣君,莫有如尧、汤。尧遭洪水,汤遭大旱。如谓政治所致,尧、汤恶君也;如非政治,是运气也。运气有时,安可请求?世之论者,犹谓尧、汤水旱。水旱者,时也;其小旱湛,皆政也。假令审然,何用致湛。审以政致之,不修所以失之,而从请求,安耐复之?世审称尧、汤水旱,天之运气,非政所致。夫天之运气,时当自然,虽雩祭请求,终无补益。而世又称汤以五过祷於桑林,时立得雨。夫言运气,则桑林之说绌;称桑林,则运气之论消。世之说称者,竟当何由?救水旱之术,审当何用?

夫灾变大抵有二:有政治之灾,有无妄之变。政治之灾,须耐求之,求之虽不耐得,而惠愍恻隐之恩,不得已之意也。慈父之於子,孝子之於亲,知病不祀神,疾痛不和药。又知病之必不可治,治之无益,然终不肯安坐待绝,犹卜筮求崇、召医和药者,恻痛殷勤,冀有验也。既死气绝,不可如何,升屋之危,以衣招复,悲恨思慕,冀其悟也。雩祭者之用心,慈父孝子之用意也。无妄之灾,百民不知,必归於主。为政治者慰民之望,故亦必雩。

问:``政治之灾,无妄之变,何以别之?''曰:德酆政得,灾犹至者,无妄也;德衰政失,变应来者,政治也。夫政治则外雩而内改,以复其亏;无妄则内守旧政,外修雩礼,以慰民心。故夫无妄之气,厉世时至,当固自一,不宜改政。何以验之?周公为成王陈《立政》之言曰:``时则物有间之。自一话一言,我则末,维成德之彦,以乂我受民。''周公立政,可谓得矣。知非常之物,不赈不至,故敕成王自一话一言,政事无非,毋敢变易。然则非常之变,无妄之气间而至也。水气间尧,旱气间汤。周宣以贤,遭遇久旱。建初孟〔年〕,北州连旱,牛死民乏,放流就贱。圣主宽明於上,百官共职於下,太平之明时也。政无细非,旱犹有,气间之也。圣主知之,不改政行,转谷赈赡,损酆济耗。斯见之审明,所以救赴之者得宜也。鲁文公间岁大旱,臧文仲曰:``修城郭,贬食省用,务啬劝分。
''文仲知非政,故徒修备,不改政治。变复之家,见变辄归於政,不揆政之无非,见异惧惑,变易操行,以不宜改而变,只取灾焉!

何以言必当雩也?曰:《春秋》大雩,传家〔左丘明〕、公羊、谷梁无讥之文,当雩明矣。曾晰对孔子言其志曰:``暮春者,春服既成,冠者五六人,童子六七人,浴乎沂,风乎舞雩,咏而归。''孔子曰:``吾与点也!''鲁设雩祭於沂水之上。暮者,晚也;春谓四月也。春服既成,谓四月之服成也。冠者、童子,雩祭乐人也。浴乎沂,涉沂水也,象龙之从水中出也。风乎舞雩,风,歌也。咏而馈,咏歌馈祭也,歌咏而祭也。说论之家,以为浴者,浴沂水中也,风干身也。周之四月,正岁二月也,尚寒,安得浴而风干身?由此言之,涉水不浴,雩祭审矣。

《春秋》《左氏传》曰:``启蛰而雩。''又曰:``龙见而雩。启蛰、龙见。''
皆二月也。春二月雩,秋八月亦雩。春祈谷雨,秋祈谷实。当今灵星,秋之雩也。春雩废,秋雩在。故灵星之祀,岁雩祭也。孔子曰:``吾与点也!''善点之言,欲以雩祭调和阴阳,故与之也。使雩失正,点欲为之,孔子宜非,不当与也。樊迟从游,感雩而问,刺鲁不能崇德而徒雩也。

夫雩,古而有之。故《礼》曰:``雩祭,祭水旱也。''故有雩礼,故孔子不讥,而仲舒申之。夫如是,雩祭,祀礼也。雩祭得礼,则大水鼓用牲於社,亦古礼也。得礼无非,当雩一也。礼祭〔地〕社,报生万物之功。土地广远,难得辨祭,故立社为位,主心事之。为水旱者,阴阳之气也,满六合难得尽祀,故修坛设位,敬恭祈求,效事社之义,复灾变之道也。推生事死,推人事鬼。阴阳精气,傥如生人能饮食乎?故共馨香,奉进旨嘉,区区惓惓,冀见答享。推祭社言之,当雩二也。岁气调和,灾害不生,尚犹而雩。今有灵星,古昔之礼也。况岁气有变,水旱不时,人君之惧,必痛甚矣。虽有灵星之祀,犹复雩,恐前不备,肜绎之义也。冀复灾变之亏,获酆穰之报,三也。礼之心悃,乐之意欢忻。悃愊以玉帛效心,欢忻以钟鼓验意。雩祭请祈,人君精诚也。精诚在内,无以效外。故雩祀尽己惶惧,关纳精心於雩祀之前,玉帛钟鼓之义,四也。臣得罪於君,子获过於父,比自改更,且当谢罪。惶惧於旱,如政治所致,臣子得罪获过之类也。默改政治,潜易操行,不彰於外,天怒不释。故必雩祭,惶惧之义,五也。汉立博士之官,师弟子相呵难,欲极道之深,形是非之理也。不出横难,不得从说;不发苦诘,不闻甘对。导才低仰,欲求裨也;砥石劘厉,欲求銛也。推《春秋》之义,求雩祭之说,实孔子之心,考仲舒之意,孔子既殁,仲舒已死,世之论者,孰当复问?唯若孔子之徒,仲舒之党,为能说之。

\hypertarget{header-n650}{%
\subsubsection{顺鼓篇}\label{header-n650}}

《春秋》之义,大水,鼓用牲於社。说者曰:``鼓者,攻之也。''或曰:``
胁之。''胁则攻矣。〔阴〕胜,攻社以救之。

或难曰:攻社谓得胜负之义,未可得顺义之节也。人君父事天,母事地。母之党类为害,可攻母以救之乎?以政令失道阴阳缪戾者,人君也。不自攻以复之,反逆节以犯尊,天地安肯济?使湛水害伤天,不以地害天,攻之可也。今湛水所伤,物也。万物於地,卑也。害犯至尊之体,於道违逆,论《春秋》者,曾不知难。案雨出於山,流入於川,湛水之类,山川是矣。大水之灾,不攻山川。社,土也。五行之性,水土不同。以水为害而攻土,土胜水。攻社之义,毋乃如今世工匠之用椎凿也?以椎击凿,令凿穿木。今傥攻土,令厌水乎?且夫攻社之义,以为攻阴之类也。甲为盗贼,伤害人民,甲在不亡,舍甲而攻乙之家,耐止甲乎?今雨者,水也。水在,不自攻水,而乃攻社。案天将雨,山先出云,云积为雨,雨流为水。然则山者,父母;水者子弟也。重罪刑及族属,罪父母子弟乎?罪其朋徒也?计山水与社,俱为雨类也,孰为亲者?社,土也。五行异气,相去远。

殷太戊桑谷俱生。或曰高宗。恐骇,侧身行道,思索先王之政,兴灭国,继绝世,举逸民,明养老之义,桑谷消亡,享国长久。''此说《春秋》〔者〕所共闻也。水灾与桑谷之变何以异?殷王改政,《春秋》攻社,道相违反,行之何从?周成王之时,天下雷雨,偃禾拔木,为害大矣。成王开金滕之书,求索行事周公之功,执书以泣遏,雨止风反,禾、大木复起。大雨久湛,其实一也。成王改过,《春秋》攻社,两经二义,行之如何?

月令之家,虫食谷稼,取虫所类象之吏,笞击僇辱以灭其变。实论者谓之未必真是,然而为之,厌合人意。今致雨者,政也、吏也,不变其政,不罪其吏,而徒攻社,能何复塞?苟以为当攻其类,众阴之精,月也,方诸乡月,水自下来,月离於毕,出房北道,希有不雨。月中之兽,兔、蟾蜍也。其类在地,螺与蚄也。月毁於天,螺、蚄舀缺,同类明矣。雨久不霁,攻阴之类,宜捕斩兔、蟾蜍,椎被螺、蚄,为其得实。蝗虫时至,或飞或集。所集之地,谷草枯索。吏卒部民,堑道作坎,榜驱内於堑坎,杷蝗积聚以千斛数。正攻蝗之身,蝗犹不止。况徒攻阴之类,雨安肯霁?

《尚书》《大传》曰:``烟氛郊社不修,出川不祝,风雨不时,霜雪不降,责於天公。臣多弑主,孽多杀宗,五品不训,责於人公。城郭不缮,沟池不修,水泉不隆,水为民害,责於地公。''王者三公,各有所主;诸侯卿大夫,各有分职。大水不责卿大夫而击鼓攻社,何〔如〕?不然,鲁国失礼,孔子作经,表以为戒也。公羊高不能实,董仲舒不能定,故攻社之义,至今复行之。使高尚生,仲舒未死,将难之曰:``久雨湛水溢,谁致之者?使人君也,宜改政易行以复塞之。如人臣也,宜罪其人以过解天。如非君臣,阴阳之气偶时运也,击鼓攻社,而何救止?《春秋》说曰:``人君亢阳致旱,沈溺致水。''夫如是,旱则为沈溺之行,水则为亢阳之操,何乃攻社?攻社不解,硃丝萦之,亦复未晓。说者以为社阴、硃阳也,水阴也,以阳色萦之,助鼓为救。夫大山失火,灌以壅水,众知不能救之者,何也?火盛水少,热不能胜也。今国湛水,犹大山失火也;以若绳之丝,萦社为救,犹以壅水灌大山也。

原天心以人意,状天治以人事。人相攻击,气不相兼,兵不相负,不能取胜。今一国水,使真欲攻阳,以绝其气,悉发国人操刀把杖以击之,若岁终逐疫,然後为可。楚、汉之际,六国之时,兵革战攻,力强则胜,弱劣则负。攻社一人击鼓,无兵革之威,安能救雨?夫一旸一雨,犹一昼一夜也;其遭若尧、汤之水旱,犹一冬一夏也。如或欲以人事祭祀复塞其变,冬求为夏,夜求为昼也。何以效之?久雨不霁,试使人君高枕安卧,雨犹自止。止久至於大旱,试使人君高枕安卧,旱犹自雨。何则?〔阳〕极反阴,阴极反〔阳〕。故夫天地之有湛也,何以知不如人之有水病也?其有旱也,何以知不如人有瘅疾也?祷请求福,终不能愈,变操易行,终不能救;使医食药,冀可得愈;命尽期至,医药无效。

尧遭洪水,《春秋》之大水也,圣君知之,不祷於神,不改乎政,使禹治之,百川东流。夫尧之使禹治水,犹病水者之使医也。然则尧之洪水,天地之水病也;禹之治水,洪水之良医也。说者何以易之?攻社之义,於事不得。雨不霁,祭女娲,於礼何见?伏羲、女娲,俱圣者也。舍伏羲而祭女娲,《春秋》不言。董仲舒之议,其故何哉?夫《春秋经》但言``鼓'',岂言攻哉?说者见有``鼓''文,则言攻矣。夫鼓未必为攻,说者用意异也。

季氏富於周公,而求也为之聚敛而附益之。孔子曰:``非吾徒也,小子鸣鼓攻之,可也。''攻者,责也,责让之也。六国兵革相攻,不得难此,此又非也。以卑而责尊,为逆矣。或据天责之也?王者母事地,母有过,子可据父以责之乎?下之於上,宜言谏。若事,臣子之礼也;责让,上文礼也。乖违礼意,行文如何?故警戒下也。必以伐鼓为攻此社,此则钟夫礼以鼓助号呼,明声响也。古者人君将出,撞钟击鼓,声鼓鸣攻击上也。

大水用鼓,或时再告社,阴之太盛,雨湛不霁。阴盛阳微,非道之宜,口祝不副,以鼓自助,与日食鼓用牲於社,同一义也。俱为告急,彰阴盛也。事大而急者用锺鼓,小而缓者用铃\{狄\},彰事告急,助口气也。大道难知,大水久湛,假令政治所致,犹先告急,乃斯政行。盗贼之发,与此同操。盗贼亦政所致,比求阙失,犹先发告。鼓用牲於社,发觉之也。社者,众阴之长,故伐鼓使社知之。说鼓者以为攻之,故攻母逆义之难,缘此而至。今言告以阴盛阳微,攻尊之难,奚从来哉?且告宜於用牲,用牲不宜於攻。告事用牲,礼也;攻之用牲,於礼何见?硃丝如绳,示在旸也。旸气实微,故用物微也。投一寸之针,布一丸之艾於血脉之蹊,笃病有瘳。硃丝如一寸之针、一丸之艾也?吴攻破楚,昭王亡走,申包胥间步赴秦,哭泣求救,卒得助兵,却吴而存楚。击鼓之人,〔诚〕如何耳;使诚若申包胥,一人击得。假令一人击鼓,将耐令社与秦王同感,以土胜水之威,却止云雨。云雨气得与吴同恐,消散入山,百姓被害者,得蒙霁晏,有楚国之安矣。迅雷风烈,君子必变,虽夜必兴,衣冠而坐,惧威变异也。

夫水旱,犹雷风也,虽运气无妄,欲令人君高枕幄卧,以俟其时,无恻怛忧民之心。尧不用牲,或时上世质也。仓颉作书,奚仲作车,可以前代之时无书、车之事,非後世为之乎?时同作殊,事乃可难;异世易俗,相非如何?俗图画女娲之象为妇人之形,又其号曰``女''。仲舒之意,殆谓女娲古妇人帝王者也。男阳而女阴,阴气为害,故祭女娲求福佑也。传又言:共工与颛顼争为天子,不胜,怒而触不周之山,使天柱折,地维绝。女娲消炼五色石以补苍天,断鰲之足以立四极。仲舒之祭女娲,殆见此传也。本有补苍天、立四极之神,天气不和,阳道不胜,傥女娲以精神助圣王止雨湛乎!

\hypertarget{header-n664}{%
\subsection{卷十六}\label{header-n664}}

\hypertarget{header-n665}{%
\subsubsection{乱龙篇}\label{header-n665}}

董仲舒申《春秋》之雩,设土龙以招雨,其意以云龙相致。《易》曰:``云从龙,风从虎。''以类求之,故设土龙。阴阳从类,云雨自至。儒者或问曰:夫《易》言``云从龙''者,谓真龙也,岂谓土哉?楚叶公好龙,墙壁盘盂皆画龙。必以象类为若真,是则叶公之国常有雨也。《易》又曰``风从虎'',谓虎啸而谷风至也。风之与虎,亦同气类。设为土虎,置之谷中,风能至乎?夫土虎不能而致风,土龙安能而致雨?古者畜龙,乘车驾龙,故有豢龙氏、御龙氏。夏后之庭,二龙常在,季年夏衰,二龙低伏。真龙在地,犹无云雨,况伪象乎?礼,画雷樽象雷之形,雷樽不闻能致雷,土龙安能而动雨?顿牟掇芥,磁石引针,皆以其真是,不假他类。他类肖似,不能掇取者,何也?气性异殊,不能相感动也。

刘子骏掌雩祭,典土龙事,桓君山亦难以顿牟、磁石不能真是,何能掇针取芥,子骏穷无以应。子骏,汉朝智襄,笔墨渊海,穷无以应者,是事非议误,不得道理实也。

曰:夫以非真难,是也;不以象类说,非也。夫东风至,酒湛溢。〔按酒味酸,从东方木也。其味酸,故酒湛溢也〕。

鲸鱼死,彗星出。天道自然,非人事也。事与彼云龙相从,同一实也。

日,火也;月,水也。水火感动,常以真气。今伎道之家,

铸阳燧取飞火於日,作方诸取水於月,非自然也,而天然之也。土龙亦非真,何为不能感天?一也。阳燧取火於天,五月丙午日中之时,消炼五石,铸以为器,乃能得火。今妄取刀剑偃月之钩,摩以向日,亦能感天。夫土龙既不得比於阳燧,当与刀剑偃月钩为比。二也。

齐孟常君夜出秦关,关未开,客为鸡鸣而真鸡鸣和之。夫鸡可以奸声感,则雨亦可以伪象致。三也。

李子长为政,欲知囚情,以梧桐为人,象囚之形。凿地为坎,以卢为椁,卧木囚其中。囚罪正,则木囚不动,囚冤侵夺,木囚动出。不知囚之精神着木人乎?将精神之气动木囚也?夫精神感动木囚,何为独不应从土龙?四也。

舜以圣德,入大麓之野,虎狼不犯,虫蛇不害。禹铸金鼎象百物,以入山林,亦辟凶殃。论者以为非实,然而上古久远,周鼎之神,不可无也。夫金与土,同五行也,使作土龙者如禹之德,则亦将有云雨之验。五也。

顿牟掇芥,磁石、钩象之石非顿牟也,皆能掇芥,土龙亦非真,当与磁石、钩象为类。六也。

楚叶公好龙,墙壁盂樽皆画龙象,真龙闻而下之。夫龙与云雨同气,故能感动,以类相从。叶公以为画致真龙,今独何以不能致云雨?七也。

神灵示人以象,不以实,故寝卧梦悟见事之象。将吉,吉象来;将凶,凶象至。神灵之气,云雨之类,神灵以象见实,土龙何独不能以伪致真?〔八〕也。

神灵以象见实,土龙何独不能以伪致真也?上古之人,有神荼、郁垒者,昆弟二人,性能执鬼,居东海度朔山上,立桃树下,简阅百鬼。鬼无道理,妄为人祸,荼与郁垒缚以卢索,执以食虎。故今县官斩桃为人,立之户侧;画虎之形,著之门阑。夫桃人,非荼、郁垒也;画虎,非食鬼之虎也,刻画效象,冀以御凶。今土龙亦非致雨之龙,独信桃人画虎,不知土龙。九也。

此尚因缘昔书,不见实验。鲁般、墨子刻木为鸢,蜚之三日而不集,为之巧也。使作土龙者若鲁般、墨子,则亦将有木鸢蜚不集之类。夫蜚鸢之气,云雨之气也。气而蜚木鸢,何独不能从土龙?十也。

夫云雨之气也,知於蜚鸢之气,未可以言。钓者以木为鱼,丹漆其身,近之水流而击之,起水动作,鱼以为真,并来聚会。夫丹木,非真鱼也,鱼含血而有知,犹为象至。云雨之知,不能过鱼。见土龙之象,何能疑之?十一也。

此尚鱼也,知不如人。匈奴敬畏郅都之威,刻木象都之状,交弓射之,莫能一中。不知都之精神在形象邪?亡将匈奴敬鬼精神在木也?如都之精神在形象,天龙之神亦在土龙。如匈奴精在於木人,则雩祭者之精亦在土龙。十二也。

金翁叔,休屠王之太子也,与父俱来降汉,父道死,与母俱来,拜为骑者尉。母死,武帝图其母於甘泉殿上,署曰``休屠王焉提''。翁叔从上上甘泉,拜谒起立,向之泣涕沾襟,久乃去。夫图画,非母之实身也,因见形象,涕泣辄下,思亲气感,不待实然也。夫土龙犹甘泉之图画也,云雨见之,何为不动?十三也。

此尚夷狄也。有若似孔子,孔子死,弟子思慕,共坐有若孔子之座。弟子知有若非孔子也,犹共坐而尊事之。云雨之知,使若诸弟子之知,虽知土龙非真,然犹感动,思类而至。十四也。

有若,孔子弟子疑其体象,则谓相似。孝武皇帝幸李夫人,夫人死,思见其形。道士以术为李夫人,夫人步入殿门,武帝望见,知其非也,然犹感动,喜乐近之。使云雨之气,如武帝之心,虽知土龙非真,然犹爱好感起而来。十五也。

既效验有十五,又亦有义四焉。

立春东耕,为土象人,男女各二人,秉耒把锄;或立土牛。未必能耕也。顺气应时,示率下也。今设土龙,虽知不能致雨,亦当夏时以类应变,与立土人土牛同义。〔一〕也。

礼,宗庙之主,以木为之,长尺二寸,以象先祖。孝子入庙,主心事之,虽知木主非亲,亦当尽敬。有所主事,土龙与木主同。虽知非真,示当感动,立意於象。二也。

涂车、刍灵,圣人知其无用,示象生存,不敢无也。夫设土龙,知其不能动雨也,示若涂车、刍灵而有致。三也。

天子射熊,诸侯射麋,卿大夫射虎豹,土射鹿豕,示服猛也。名布为侯,示射无道诸侯也。夫画布为熊麋之象,名布为侯,礼贵意象,示义取名也。土龙亦夫熊麋、布侯之类。四也。

夫以象类有十五验,以礼示意有四义。仲舒览见深鸿,立事不妄,设土龙之象,果有状也。龙暂出水,云雨乃至。古者畜龙、御龙,常存,无云雨。犹旧交相阔远,卒然相见,欢欣歌笑,或至悲泣涕,偃伏少久,则示行各恍忽矣。《易》曰:``云从龙。''非言龙从云也。云樽刻雷云之象,龙安肯来?夫如是,传之者何可解,则桓君山之难可说也,则刘子骏不能对,劣也,劣则董仲舒之龙说不终也。《论衡》终之。故曰``乱龙''。〔乱〕者,终也。

\hypertarget{header-n693}{%
\subsubsection{遭虎篇}\label{header-n693}}

变复之家,谓虎食人者,功曹为奸所致也。其意以为,功曹众吏之率,虎亦诸禽之雄也。功曹为奸,采渔於吏,故虎食人以象其意。

夫虎食人,人亦有杀虎。谓虎食人,功曹受取於吏,如人食虎,吏受於功曹也乎?案世清廉之士,百不能一。居功曹之官,皆有奸心,私旧故可以幸,苞苴赂遗,小大皆有。必谓虎应功曹,是野中之虎常害人也。夫虎出有时,犹龙见有期也。阴物以冬见,阳虫以夏出。出应其气,气动其类。参、伐以冬出,心、尾以夏见。参、伐则虎星,心、尾则龙象。象出而物见,气至而类动,天地之性也。动於林泽之中,遭虎搏噬之时,禀性狂勃,贪叨饥饿,触自来之人,安能不食?人之筋力,羸弱不适,巧便不知,故遇辄死。使孟贲登山,冯妇入林,亦无此害也。

孔子行鲁林中,妇人哭,甚哀,使子贡问之:``何以哭之哀也?''曰:``去年虎食吾夫,今年食吾子,是以哭哀也。''子贡曰:``若此,何不去也?''对曰:
``吾善其政之不苛、吏之不暴也。''子贡还报孔子。孔子曰:``弟子识诸!苛政暴吏,甚於虎也。''夫虎害人,古有之矣。政不苛,吏不暴,德化之足以却虎。然而二岁比食二人,林中兽不应善也。为廉不应,奸吏亦不应矣。

或曰:``虎应功曹之奸,所谓不苛政者,非功曹也。妇人,廉吏之部也,虽有善政,安耐化虎?''夫鲁无功曹之官,功曹之官,相国是也。鲁相者殆非孔、墨,必三家也。为相必无贤操,以不贤居权位,其恶,必不廉也。必以相国为奸,令虎食人,是则鲁野之虎常食人也。

水中之毒,不及陵上;陵上之气,不入水中;各以所近,罹殃取祸。是故渔者不死於山,猎者不溺於渊。好入山林,穷幽测深,涉虎窟寝,虎搏噬之,何以为变?鲁公牛哀病化为虎,搏食其兄,同变化者不以为怪。入山林草泽见害於虎,怪之非也。蝮蛇悍猛,亦能害入。行止泽中,〔害〕於蝮蛇,应何官吏?蜂虿害人,入毒气害人,入水火害人。人为蜂虿所螫,为毒气所中,为火所燔,为水所溺,又谁致之者?苟诸禽兽,乃应吏政。行山林中,麋鹿、野猪、牛象、熊罢、豺狼、蜼蠼,皆复杀人。苟谓食人乃应为变。蚤虱闽虻皆食人,人身强大,故不至死。仓卒之世,谷食之贵,百姓饥饿,自相啖食,厥变甚於虎。变复之家,不处苟政。

且虎所食,非独人也,含血之禽,有形之兽,虎皆食之。〔食〕人谓应功曹之奸,食他禽兽,应何官吏?夫虎,毛虫;人,倮虫。毛虫饥,食倮虫,何变之有?四夷之外,大人食小人,虎之与蛮夷,气性一也。平陆、广都,虎所不由也;山林、草泽,虎所生出也。必以虎食人应功曹之奸,是则平陆、广都之县,功曹常为贤,山林、草泽之邑功曹常伏诛也。

夫虎食人於野,应功曹之奸,虎时入邑行於民间,功曹游於闾巷之中乎?实说,虎害人於野不应政,其行都邑,乃为怪。

夫虎,山林之兽,不狎之物也,常在草野之中,不为驯畜,犹人家之有鼠也,伏匿希出,非可常见也。命吉居安,鼠不扰乱;禄衰居危,鼠为殃变。夫虎亦然也:邑县吉安,长吏无患,虎匿不见;长吏且危,则虎入邑,行於民间。何则?长吏光气已消,都邑之地与野均也。推此以论,虎所食人,亦命时也。命讫时衰,光气去身,视肉犹尸也,故虎食之。天道偶会,虎适食人,长吏遭恶,故谓为变,应上天矣。

古今凶验,非唯虎也,野物皆然。楚王英宫楼未成,鹿走上阶,其後果薨。鲁昭公且出,瞿鹆来巢,其後季氏逐昭公,昭公奔齐,遂死不还。贾谊为长沙王傅,鹏鸟集舍,发书占之,曰:``主人将去。''其後迁为梁王傅。怀王好骑,坠马而薨;贾谊伤之,亦病而死。昌邑王时,夷鸪鸟集宫殿下,王射杀之,以问郎中令龚遂,龚遂对曰:``夷鸪野鸟,入宫,亡之应也。''其後昌邑王竟亡。卢奴令田光与公孙弘等谋反,其且觉时,狐鸣光舍屋上,光心恶之。其後事觉坐诛。会稽东部都尉礼文伯时,羊伏下,其後迁为东莱太守。都尉王子凤时,麇入府中,其後迁丹阳太守。夫吉凶同占,迁免一验,俱象空亡,精气消去也。故人且亡也,野鸟入宅;城且空也,草虫入邑。等类众多,行事比肩,略举较著,以定实验也。

\hypertarget{header-n705}{%
\subsubsection{商虫篇}\label{header-n705}}

变复之家谓虫食谷者,部吏所致也。贪则侵渔,故虫食谷。身黑头赤,则谓武官;头黑身赤,则谓文官。使加罚於虫所象类之吏,则虫灭息,不复见矣。夫头赤则谓武吏,头黑则谓文吏所致也。时或头赤身白,头黑身黄,或头身皆黄,或头身皆青,或皆白若鱼肉之虫,应何官吏?时或白布豪民、猾吏被刑乞贷者,威胜於官,取多於吏,其虫形象何如状哉?虫之灭也,皆因风雨。案虫灭之时,则吏未必伏罚也。陆田之中时有鼠,水田之中时有鱼,虾蟹之类,皆为谷害,或时希出而暂为害,或常有而为灾,等类众多,应何官吏?

鲁宣公履亩而税,应时而有蝝生者,或言若蝗。蝗时至,蔽天如雨,集地食物,不择谷草。察其头身,象类何吏?变复之家,谓蝗何应?建武三十一年,蝗起太山郡,西南过陈留、河南,遂入夷狄,所集乡县以千百数。当时乡县之吏,未皆履亩,蝗食谷草,连日老极,或蜚徙去,或止枯死。当时乡县之吏,未必皆伏罪也。夫虫食谷,自有止期,犹蚕食桑,自有足时也。生出有日,死极有月,期尽变化,不常为虫。使人君不罪其吏,虫犹自亡。夫虫,风气所生,苍颉知之,故``凡''、``虫''为``风''之字,取气於风,故八日而化,生春夏之物,或食五谷,或食众草。食五谷,吏受钱谷也,其食他草,受人何物?

倮虫三百,人为之长。由此言之,人亦虫也。人食虫所食,虫亦食人所食,俱为虫而相食物,何为怪之?设虫有知,亦将非人曰:``女食天之所生,吾亦食之,谓我为变,不自谓为灾。''凡含气之类,所甘嗜者,口腹不异。人甘五谷,恶虫之食;自生天地之间,恶虫之出。设虫能言,以此非人,亦无以诘也。夫虫之在物间也,知者不怪,其食万物也不谓之灾。

甘香渥味之物,虫生常多,故谷之多虫者粢也。稻时有虫,麦与豆无虫。必以有虫责主者吏,是其粢乡部吏常伏罪也。神农、后稷藏种之方,煮马屎以汁渍种者,令禾不虫。如或以马屎渍种,其乡部吏鲍焦、陈仲子也。是故后稷、神农之术用,则其乡吏〔可〕免为奸。何则?虫无从生,上无以察也。

虫食他草,平事不怪,食五谷叶,乃谓之灾。桂有蠹,桑有蝎,桂中药而桑给蚕,其用亦急,与谷无异。蠹蝎不为怪,独谓虫为灾,不通物类之实,暗於灾变之情也。谷虫曰蛊,蛊若蛾矣。粟米饐热生蛊。夫蛊食粟米,不谓之灾,虫食苗叶,归之於政。如说虫之家,谓粟轻苗重也。

虫之种类,众多非一。鱼肉腐臭有虫,醯酱不闭有虫,饭温湿有虫,书卷不舒有虫,衣襞不悬有虫,蜗疽疮蝼症虾有虫。或白或黑,或长或短,大小鸿杀,不相似类,皆风气所生,并连以死。生不择日,若生日短促,见而辄灭。变复之家,见其希出,出又食物,则谓之灾。灾出当有所罪,则依所似类之吏,顺而说之。人腹中有三虫,下地之泽,其虫曰蛭,蛭食人足,三虫食肠。顺说之家,将谓三虫何似类乎?凡天地之间,阴阳所生,蛟蛲之类,蜫蠕之属,含气而生,开口而食。食有甘不,同心等欲,强大食细弱,知慧反顿愚。他物小大连相啮噬,不谓之灾,独谓虫食谷物为应政事,失道理之实,不达物气之性也。

然夫虫之生也,必依温湿。温湿之气,常在春夏。秋冬之气,寒而干燥,虫未曾生。若以虫生,罪乡部吏,是则乡部吏贪於春夏,廉於秋冬。虽盗跖之吏以秋冬署,蒙伯夷之举矣。夫春夏非一,而虫时生者,温湿甚也,甚则阴阳不和。阴阳不和,政也,徒当归於政治,而指谓部吏为奸,失事实矣。何知虫以温湿生也?以蛊虫知之。谷干燥者,虫不生;温湿饐餲,虫生不禁。藏宿麦之种,烈日干暴,投於燥器,则虫不生。如不干暴,闸喋之虫,生如云烟。以蛊闸喋,准况众虫,温湿所生,明矣。

《诗》云:``营营青蝇,止於籓。恺悌君子,无信谗言。''谗言伤善,青蝇污白,同一祸败,《诗》以为兴。昌邑王梦西阶下有积蝇矢,明旦召问郎中龚遂,遂对曰:``蝇者,谗人之象也。夫矢积於阶下,王将用谗臣之言也。''由此言之,蝇之为虫,应人君用谗。何故不谓蝇为灾乎?如蝇可以为灾,夫蝇岁生,世间人君常用谗乎?

案虫害人者,莫如蚊虻,蚊虻岁生。如以蚊虻应灾,世间常有害人之吏乎?必以食物乃为灾,人则物之最贵者也,蚊虻食人,尤当为灾。必以暴生害物乃为灾,夫岁生而食人,与时出而害物,灾孰为甚?人之病疥,亦希非常,疥虫何故不为灾?且天将雨,蚁出蚋蜚,为与气相应也。或时诸虫之生,自与时气相应,如何辄归罪於部吏乎?天道自然,吉凶偶会,非常之虫适生,贪吏遭署。人察贪吏之操,又见灾虫之生,则谓部吏之所为致也。

\hypertarget{header-n717}{%
\subsubsection{讲瑞篇}\label{header-n717}}

儒者之论,自说见凤皇骐驎而知之。何则?案凤皇骐驎之象。又《春秋》获麟文曰:``有麞而角。''麞而角者,则是骐驎矣。其见鸟而象凤皇者,则凤皇矣。黄帝、尧、舜、周之盛时皆致凤皇。孝宣帝之时,凤皇集於上林,後又於长乐之宫东门树上,高五尺,文章五色。周获麟,麟似麞而角。武帝之麟,亦如麞而角。如有大鸟,文章五色;兽状如麞,首戴一角:考以图象,验之古今,则凤、麟可得审也。

夫凤皇,鸟之圣者也;骐驎,兽之圣者也;五帝、三王、皋陶、孔子,人之圣也。十二圣相各不同,而欲以麞戴角则谓之骐,相与凤皇象合者谓之凤皇,如何?夫圣鸟兽毛色不同,犹十二圣骨体不均也。

戴角之相,犹戴午也。颛顼戴午,尧、舜必未然。今鲁所获麟戴角,即後所见麟未必戴角也。如用鲁所获麟求知世间之麟,则必不能知也。何则?毛羽骨角不合同也。假令不同,或时似类,未必真是。虞舜重瞳,王莽亦重瞳;晋文骈胁,张仪亦骈胁。如以骨体毛色比,则王莽,虞舜;而张仪,晋文也。有若在鲁,最似孔子。孔子死,弟子共坐有若,问以道事,有若不能对者,何也?体状似类,实性非也。今五色之鸟,一角之兽,或时似类凤皇、骐驎,其实非真,而说者欲以骨体毛色定凤皇、骐驎,误矣。是故颜渊庶几,不似孔子;有若恆庸,反类圣人。由是言之,或时真凤皇、骐驎,骨体不似,恆庸鸟兽,毛色类真,知之如何?

儒者自谓见凤皇、骐驎辄而知之,则是自谓见圣人辄而知之也。皋陶马口,孔子反宇,设後辄有知而绝殊,马口反宇,尚未可谓圣。何则?十二圣相不同,前圣之相,难以照後圣也。骨法不同,姓名不等,身形殊状,生出异土,虽复有圣,何如知之?

恆君山谓扬子云曰:``如後世复有圣人,徒知其才能之胜己,多不能知其圣与非圣人也。''子云曰:``诚然。''夫圣人难知,知能之美若桓、扬者,尚复不能知。世儒怀庸庸之知,赍无异之议,见圣不能知,可保必也。夫不能知圣,则不能知凤皇与骐驎。世人名凤皇、骐驎,何用自谓能之乎?夫上世之名凤皇、骐驎,闻其鸟兽之奇者耳。毛角有奇,又不妄翔苟游,与鸟兽争饱,则谓之凤皇、骐驎矣。

世人之知圣,亦犹此也。闻圣人人之奇者,身有奇骨,知能博达,则谓之圣矣。及其知之,非卒见暂闻而辄名之为圣也,与之偃伏,从〔之〕受学,然後知之。何以明之。子贡事孔子,一年自谓过孔子;二年,自谓与孔子同;三年,自知不及孔子。当一年、二年之时,未知孔子圣也;三年之後,然乃知之。以子贡知孔子,三年乃定。世儒无子贡之才,其见圣人不从之学,任仓卒之视,无三年之接,自谓知圣,误矣!少正卯在鲁,与孔子并。孔子之门,三盈三虚,唯颜渊不去,颜渊独知孔子圣也。夫门人去孔子归少正卯,不徒不能知孔子之圣,又不能知少正卯,门人皆惑。子贡曰:``夫少正卯,鲁之闻人也。子为政,何以先之?
''孔子曰:``赐退,非尔所及。''夫才能知佞若子贡,尚不能知圣。世儒见圣自谓能知之,妄也。

夫以不能知圣言之,则亦知其不能知凤皇与骐驎也。使凤皇羽翮长广,骐驎体高大,则见之者以为大鸟巨兽耳。何以别之?如必巨大别之,则其知圣人亦宜以巨大。春秋之时,鸟有爰居,不可以为凤皇;长狄来至,不可以为圣人。然则凤皇、骐与鸟兽等也,世人见之,何用知之?如以中国无有,从野外来而知之,则是瞿鹆同也。瞿鹆,非中国之禽也。凤皇、骐驎,亦非中国之禽兽也。皆非中国之物,儒者何以谓瞿鹆恶、凤皇骐驎善乎?

或曰:``孝宣之时,凤皇集於上林,群鸟从〔之〕以千万数。以其众鸟之长,圣神有异,故群鸟附从。''如见大鸟来集,群鸟附之,则是凤皇,凤皇审则定矣。夫凤皇与骐驎同性,凤皇见,群鸟从;骐驎见,众兽亦宜随。案《春秋》之麟,不言众兽随之。宣帝、武帝皆行骐驎,无众兽附从之文。如以骐驎为人所获,附从者散,凤皇人不获,自来蜚翔,附从可见。《书》曰:``《箫韶》九成,凤皇来仪。''《大传》曰:``凤皇在列树。''不言群鸟从也。岂宣帝所致者异哉?

或曰:``记事者失之。唐、虞之君,凤皇实有附从。上世久远,记事遗失,经书之文,未足以实也。''夫实有而记事者失之,亦有实无而记事者生之。夫如是,儒书之文,难以实事,案附从以知凤皇,未得实也。且人有佞猾而聚者,鸟亦有佼黠而从群者。当唐、虞之时,凤悫愿,宣帝之时佼黠乎?何其俱有圣人之德行,动作之操不均同也?

无鸟附从,或时是凤皇;群鸟附从,或时非也。君子在世,清节自守,不广结从,出入动作,人不附从。豪猾之人,任使用气,往来进退,士众云合。夫凤皇,君子也,必以随多者效凤皇,是豪黠为君子也。歌曲弥妙,和者弥寡;行操益清,交者益鲜。鸟兽亦然,必以附从效凤皇,是用和多为妙曲也。龙与凤皇为比类。宣帝之时,黄龙出於新丰,群蛇不随。神雀鸾鸟,皆众鸟之长也,其仁圣虽不及凤皇,然其从群鸟亦宜数十。信陵、孟尝,食客三千,称为贤君。汉将军卫青及将军霍去病,门无一客,亦称名将。太史公曰:``盗跖横行,聚党数千人。伯夷、叔齐,隐处首阳山。''鸟兽之操,与人相似。人之得众,不足以别贤。以鸟附从审凤皇,如何?

或曰:``凤皇、骐驎,太平之瑞也。太平之际,见来至也。然亦有未太平而来至也。鸟兽奇骨异毛,卓绝非常,则是矣,何为不可知?凤皇骐驎,通常以太平之时来至者,春秋之时,骐驎尝嫌於王孔子而至。光武皇帝生於济阳,凤皇来集。''夫光武始生之时,成、哀之际也,时未太平而凤皇至。如以自为光武有圣德而来,是则为圣王始生之瑞,不为太平应也。嘉瑞或应太平,或为始生,其实难知。独以太平之际验之,如何?

或曰:``凤皇骐驎,生有种类,若龟龙有种类矣。龟故生龟,龙故生龙,形色小大,不异於前者也。见之父,察其子孙,何为不可知?''夫恆物有种类,瑞物无种适生,故曰德应,龟龙然也。人见神龟、灵龙而别之乎?宋元王之时,渔者网得神龟焉,渔父不知其神也。方今世儒,渔父之类也。以渔父而不知神龟,则亦知夫世人而不知灵龙也。

龙或时似蛇,蛇或时似龙。韩子曰:``马之似鹿者千金。''良马似鹿,神龙或时似蛇。如审有类,形色不异。王莽时有大鸟如马,五色龙文,与众鸟数十集於沛国蕲县。宣帝时凤皇集於地,高五尺,与言如马身高同矣;文章五色,与言五色龙文,物色均矣;众鸟数十,与言俱集、附从等也。如以宣帝时凤皇体色众鸟附从,安知凤皇则王莽所致鸟凤皇也。如审是王莽致之,是非瑞也。如非凤皇,体色附从,何为均等?

且瑞物皆起和气而生,生於常类之中,而有诡异之性,则为瑞矣。故夫凤皇之圣也,犹赤乌之集也。谓凤皇有种,赤乌复有类乎?嘉禾、醴泉、甘露,嘉禾生於禾中,与禾中异穗,谓之嘉禾;醴泉、甘露,出而甘美也,皆泉、露生出,非天上有甘露之种,地下有醴泉之类,圣治公平而乃沾下产出也。蓂荚、硃草亦生在地,集於众草,无常本根,暂时产出,旬月枯折,故谓之瑞。夫凤皇骐驎,亦瑞也,何以有种类?

案周太平,越常献白雉。白雉,生短而白色耳,非有白雉之种也。鲁人得戴角之麞,谓之骐驎,亦或时生於麞,非有骐驎之类。由此言之,凤皇亦或时生於鹄鹊,毛奇羽殊,出异众鸟,则谓之凤皇耳,安得与众鸟殊种类也?有若曰:``
骐驎,之於走兽,凤皇之於飞鸟,太山之於丘垤,河海之於行潦,类也。''然则凤皇、骐驎,都与鸟兽同一类,体色诡耳!安得异种?同类而有奇,奇为不世,不世难审,识之如何?

尧生丹硃,舜生商均。商均、丹硃,尧、舜之类也,骨性诡耳。鲧生禹,瞽瞍生舜。舜、禹,鲧、瞽瞍之种也,知德殊矣。试种嘉禾之实,不能得嘉禾。恆见粢梁之粟,茎穗怪奇。人见叔梁纥,不知孔子父也;见伯鱼,不知孔子之子也。张汤之父五尺,汤长八尺,汤孙长六尺。孝宣凤皇高五尺,所从生鸟或时高二尺,後所生之鸟或时高一尺。安得常种?

种类无常,故曾皙生参,气性不世,颜路出回,古今卓绝。马有千里,不必骐〔骥〕之驹;鸟有仁圣,不必凤皇之雏。山顶之溪,不通江湖,然而有鱼,水精自为之也。废庭坏殿,基上草生,地气自出之也。按溪水之鱼,殿基上之草,无类而出。瑞应之自至,天地未必有种类也。

夫瑞应犹灾变也。瑞以应善,灾以应恶,善恶虽反,其应一也。灾变无种,瑞应亦无类也。阴阳之气,天地之气也,遭善而为和,遇恶而为变,岂天地为善恶之政,更生和变之气乎?然则瑞应之出,殆无种类,因善而起,气和而生。亦或时政平气和,众物变化,犹春则鹰变为鸠,秋则鸠化为鹰,蛇鼠之类辄为鱼鳖,虾蟆为鹑,雀为蜃蛤。物随气变,不可谓无。黄石为老父授张良书,去复为石也。儒知之。或时太平气和,麞为骐驎,鹄为凤皇。是故气性,随时变化,岂必有常类哉?褒姒,玄鼋之子,二龙漦也。晋之二卿,熊罴之裔也。吞燕子、薏苡、履大迹之语,世之人然之,独谓瑞有常类哉?以物无种计之,以人无类议之,以体变化论之,凤皇、骐驎生无常类,则形色何为当同?

案《礼记瑞命篇》云:``雄曰凤,雌曰皇。雄鸣曰即即,雌鸣足足。''《诗》云:``梧桐生矣,於彼高冈。凤皇鸣矣,於彼朝阳。菶々萋萋,噰々

喈喈。''《瑞命》与《诗》,俱言凤皇之鸣。《瑞命》之言``即即、足足'',《诗》云``噰々、喈喈'',此声异也。使声审,则形不同也;使审〔异〕同,《诗》与《礼》异。世传凤皇之鸣,故将疑焉。

案鲁之获麟云``有麞而角''。言``有麞''者,色如麞也。麞色有常,若鸟色有常矣。武王之时,火流为乌,云其色赤。赤非乌之色,故言其色赤。如似麞而色异,亦当言其色白若黑。今成事色同,故言``有麞''。麞无角,有异於故,故言``而角''也。夫如是,鲁之所得驎者,若麞之状也。武帝之时,西巡狩得白驎,一角而五趾。角或时同,言五趾者,足不同矣。鲁所得麟,云``有麞'',不言色者,麞无异色也。武帝云``得白驎'',色白不类麞,故〔不〕言有麞,正言白驎,色不同也。孝宣之时,九真贡,献驎,状如〔鹿〕而两角者。孝武言一,角不同矣。《春秋》之麟如麞,宣帝之驎言如鹿。鹿与麞小大相倍,体不同也。

夫三王之时,驎毛色、角趾、身体高大,不相似类。推此准後世,驎出必不与前同,明矣。夫骐驎,凤皇之类,骐驎前後体色不同,而欲以宣帝之时所见凤皇高五尺,文章五色,准前况後,当复出凤皇,谓与之同,误矣!後当复出见之凤皇、骐驎,必已不与前世见出者相似类。而世儒自谓见而辄知之,奈何?

案鲁人得驎,不敢正名驎,曰``有麞而角者'',时诚无以知也。武帝使谒者终军议之,终军曰:``野禽并角,明天下同本也。''不正名驎而言``野禽''者,终军亦疑无以审也。当今世儒之知,不能过鲁人与终军,其见凤皇、骐驎,必从而疑之非恆之鸟兽耳,何能审其凤皇、骐驎乎?

以体色言之,未必等;以鸟兽随从多者,未必善;以希见言之,有瞿鹆来;以相奇言之,圣人有奇骨体,贤者亦有奇骨。圣贤俱奇,人无以别。由贤圣言之,圣鸟、圣兽,亦与恆鸟庸兽俱有奇怪。圣人贤者,亦有知而绝殊,骨无异者;圣贤鸟兽,亦有仁善廉清,体无奇者。世或有富贵不圣,身有骨为富贵表,不为圣贤验。然则鸟亦有五采,兽有角而无仁圣者。夫如是,上世所见凤皇、骐驎,何知其非恆鸟兽?今之所见鹊、麞之属,安知非凤皇、骐驎也?

方今圣世,尧、舜之主,流布道化,仁圣之物,何为不生?或时以有凤皇、骐驎,乱於鹄鹊、麞鹿,世人不知。美玉隐在石中,楚王、令尹不能知,故有抱玉泣血之痛。今或时凤皇、骐驎,以仁圣之性,隐於恆毛庸羽,无一角五色表之,世人不之知,犹玉在石中也。何用审之?为此论草於永平之初,时来有瑞,其孝明宣惠,众瑞并至。至元和、章和之际,孝章耀德,天下和洽,嘉瑞奇物,同时俱应,凤皇、骐驎,连出重见,盛於五帝之时。此篇已成,故不得载。

或问曰:``《讲瑞》谓凤皇、骐驎难知,世瑞不能别。今孝章之所致凤皇、骐驎,不可得知乎?''曰:《五鸟》之记,四方中央,皆有大鸟,其出,众鸟皆从,小大毛色类凤皇,实难知也。故夫世瑞不能别,别之如何?以政治。时王之德,不及唐、虞之时,其凤皇、骐驎,目不亲见。然而唐、虞之瑞必真是者,尧之德明也。孝宣比尧、舜,天下太平,万里慕化,仁道施行,鸟兽仁者感动而来,瑞物小大、毛色、足翼必不同类。以政治之得失,主之明暗,准况众瑞,无非真者。事或难知而易晓,其此之谓也。又以甘露验之。甘露,和气所生也。露无故而甘,和气独已至矣。和气至,甘露降,德洽而众瑞凑。案永平以来,讫於章和,甘露常降,故知众瑞皆是,而凤凰、骐驎皆真也。

\hypertarget{header-n747}{%
\subsection{卷十七}\label{header-n747}}

\hypertarget{header-n748}{%
\subsubsection{指瑞篇}\label{header-n748}}

儒者说凤皇、骐驎为圣王来,以为凤皇、骐仁圣禽也,思虑深,避害远,中国有道则来,无道则隐。称凤皇、骐驎之仁知者,欲以褒圣人也,非圣人之德不能致凤皇、骐驎。此言妄也。夫凤皇、骐驎圣,圣人亦圣。圣人恓恓忧世,凤皇、骐驎亦宜率教。圣人游於世间,凤皇、骐亦宜与鸟兽会。何故远去中国,处於边外,岂圣人浊,凤皇、骐驎清哉?何其圣德俱而操不同也?如以圣人者当隐乎,十二圣宜隐;如以圣者当见,凤、驎亦宜见。如以仁圣之禽,思虑深,避害远,则文王拘於羑里,孔子厄於陈、蔡,非也。文王、孔子,仁圣之人,忧世悯民,不图利害,故其有仁圣之知,遭拘厄之患。凡人操行能修身正节,不能禁人加非於己。

案人操行莫能过圣人,圣人不能自免於厄,而凤、驎独能自全於世,是鸟兽之操,贤於圣人也。且鸟兽之知,不与人通,何以能知国有道与无道也?人同性类,好恶均等,尚不相知;鸟兽与人异性,何能知之?人不能知鸟兽,鸟兽亦不能知人,两不能相知;鸟兽为愚於人,何以反能知之?儒者咸称凤皇之德,欲以表明王之治,反令人有不及鸟兽,论事过情,使实不著。

且凤、驎岂独为圣王至哉?孝宣皇帝之时,凤皇五至,骐驎一至,神雀、黄龙,甘露、醴泉,莫不毕见,故有五凤、神雀、甘露、黄龙之纪。使凤、驎审为圣王见,则孝宣皇帝圣人也;如孝宣帝非圣,则凤、驎为贤来也。为贤来,则儒者称凤皇、骐驎,失其实也。凤皇、骐为尧、舜来,亦为宣帝来矣。夫如是,为圣且贤也。

儒者说圣太隆,则论凤、驎亦过其实。《春秋》曰:``西狩获死驎。''人以示孔子,孔子曰:``孰为来哉?孰为来哉?''反袂拭面,泣涕沾襟。儒者说之,以为天以命孔子,孔子不王之圣也。夫驎为圣王来,孔子自以不王,而时王鲁君无感驎之德,怪其来而不知所为,故曰:``孰为来哉?孰为来哉?''知其不为治平而至,为己道穷而来,望绝心感,故涕泣沾襟。以孔子言``孰为来哉'',知驎为圣王来也。曰:前孔子之时,世儒已传此说,孔子闻此说而希见其物也,见驎之至,怪所为来。实者,驎至,无所为来,常有之物也,行迈鲁泽之中,而鲁国见其物遭获之也。孔子见驎之获,获而又死,则自比於驎,自谓道绝不复行,将为小人所蹊获也。故孔子见驎而自泣者,据其见得而死也,非据其本所为来也。然则驎之至也,自与兽会聚也。其死,人杀之也。使驎有知,为圣王来,时无圣主,何为来乎?思虑深,避害远,何故为鲁所获杀乎?夫以时无圣王而驎至,知不为圣王来也;为鲁所获杀,知其避害不能远也。圣兽不能自免於难。圣人亦不能自免於祸。祸难之事,圣者所不能避,而云凤、驎思虑深,避害远,妄也。

且凤、驎非生外国也,中国有圣王乃来至也。生於中国,长於山林之间,性廉见希,人不得害也,则谓之思虑深,避害远矣。生与圣王同时,行与治平相遇,世间谓之圣王之瑞,为圣来矣。剥巢破卵,凤皇为之不翔;焚林而畋,漉池而渔,龟、龙为之不游。凤皇,龟、龙之类也,皆生中国,与人相近。巢剥卵破,屏窜不翔;林焚池漉,伏匿不游,无远去之文,何以知其在外国也?龟、龙、凤皇,同一类也。希见不害,谓在外国;龟、龙希见,亦在外国矣。

孝宣皇帝之时,凤皇、骐驎、黄龙、神雀皆至,其至同时,则其性行相似类,则其生出宜同处矣。龙不生於外国,外国亦有龙。凤、驎不生外国,外国亦有凤、驎。然则中国亦有,未必外国之凤、驎也。人见凤、驎希见,则曰在外国;见遇太平,则曰为圣王来。夫凤皇、骐驎之至也,犹醴泉之出、硃草之生也。谓凤皇在外国,闻有道而来,醴泉、硃草何知,而生於太平之时?醴泉、硃草,和气所生,然则凤皇、骐驎,亦和气所生也。和气生圣人,圣人生於衰世。物生为瑞,人生为圣,同时俱然,时其长大,相逢遇矣。衰世亦有和气,和气时生圣人。圣人生於衰世,衰世亦时有凤、驎也。孔子生於周之末世,骐驎见於鲁之西泽。光武皇帝生於成、哀之际,凤皇集於济阳之地。圣人圣物,生於衰世。圣王遭见圣物,犹吉命之人逢吉祥之类也,其实相遇,非相为出也。

夫凤、驎之来,与白鱼、赤乌之至,无以异也。鱼遭自跃,王舟逢之;火偶为乌,王仰见之。非鱼闻武王之德,而入其舟;乌知周家当起,集於王屋也。谓凤、驎为圣王来,是谓鱼、乌为武王至也。王者受富贵之命,故其动出见吉祥异物,见则谓之瑞。瑞有小大,各以所见,定德薄厚。若夫白鱼、赤乌小物,小安之兆也;凤皇、骐驎大物,太平之象也。故孔子曰:``凤鸟不至,河不出图,吾已矣夫!''不见太平之象,自知不遇太平之时矣。且凤皇、骐驎,何以为太平之象?凤皇、骐驎,仁圣之禽也,仁圣之物至,天下将为仁圣之行矣。《尚书大传》曰:``高宗祭成汤之庙,有雉升鼎耳而鸣。高宗问祖乙,祖乙曰:`远方君子殆有至者。'祖乙见雉有似君子之行,今从外来,则曰``远方君子将有至者''矣。

夫凤皇、骐驎犹雉也,其来之象,亦与雉同。孝武皇帝西巡狩,得白驎,一角而五趾,又有木,枝出复合於本。武帝议问群臣,谒者终军曰:``野禽并角,明同本也;众枝内附,示无外也。如此瑞者,外国宜有降者。若〔是〕应,殆且有解编发、削左衽、袭冠带而蒙化焉。''其後数月,越地有降者,匈奴名王亦将数千人来降,竟如终军之言。终军之言,得瑞应之实矣。推此以况白鱼、赤乌,犹此类也。鱼,〔水〕精;白者,殷之色也;乌者,孝鸟;赤者,周之应气也。先得白鱼,後得赤乌,殷之统绝,色移在周矣。据鱼、乌之见以占武王,则知周之必得天下也。

世见武王诛纣,出遇鱼、乌,则谓天用鱼、乌命使武王诛纣,事相似类,其实非也。春秋之时,瞿鹆来巢,占者以为凶。夫野鸟来巢,鲁国之都且为丘墟,昭公之身且出奔也。後昭公为季氏所攻,出奔於齐,死不归鲁。贾谊为长沙太傅,服鸟集舍,发书占之,云:``服鸟入室,主人当去''。其後贾谊竟去。野鸟虽殊,其占不异。夫凤、驎之来,与野鸟之巢、服鸟之集,无以异也。是瞿鹆之巢,服鸟之集,偶巢适集,占者因其野泽之物,巢集城宫之内,则见鲁国且凶、传〔主〕人不吉之瑞矣。非瞿鹆、服鸟知二国祸将至,而故为之巢集也。王者以天下为家,家人将有吉凶之事,而吉凶之兆豫见於人,知者占之,则知吉凶将至。非吉凶之物有知,故为吉凶之人来也。犹蓍龟之有兆数矣。龟兆蓍数,常有吉凶,吉人卜筮与吉相遇,凶人与凶相逢,非蓍龟神灵知人吉凶,出兆见数以告之也。虚居卜筮,前无过客,犹得吉凶。然则天地之间,常有吉凶,吉凶之物来至,自当与吉凶之人相逢遇矣。或言天使之所为也。夫巨大之天使,细小之物,音语不通,情指不达,何能使物?物亦不为天使,其来神怪,若天使之,则谓天使矣。

夏後孔甲畋於首山,天雨晦冥,入於民家,主人方乳。或曰:``後来,之子必大贵。''或曰:``不胜,之子必有殃。''夫孔甲之入民室也,偶遭雨而廕庇也,非知民家将生子,而其子必〔吉〕凶,为之至也。既至,人占则有吉凶矣。夫吉凶之物见於王朝,若入民家,犹孔甲遭雨入民室也。孔甲不知其将生子,为之故到。谓凤皇诸瑞有知,应吉而至,误矣。

\hypertarget{header-n752}{%
\subsubsection{是应篇}\label{header-n752}}

儒者论太平瑞应,皆言气物卓异,硃草、醴泉、翔〔风〕、甘露、景星、嘉禾、脯、蓂荚、屈轶之属;又言山出车,泽出舟,男女异路,市无二价,耕者让畔,行者让路,颁白不提挈,关梁不闭,道无虏掠,风不鸣条,雨不破塊,五日一风,十日一雨,其盛茂者,致黄龙、骐驎、凤皇。夫儒者之言,有溢美过实。瑞应之物,或有或无。夫言凤皇、骐驎之属,大瑞较然,不得增饰;其小瑞征应,恐多非是。夫风气雨露,本当和适。言其〔风〕翔、甘露,风不鸣条、雨不破塊,可也;言其五日一风、十日一雨,褒之也。风雨虽适,不能五日十日正如其数。言男女不相干,市价不相欺,可也;言其异路,无二价,褒之也。太平之时,岂更为男女各作道哉?不更作道,一路而行,安得异乎?太平之时,无商人则可,如有,必求便利以为业。买物安肯不求贱?卖货安肯不求贵?有求贵贱之心,必有二价之语。此皆有其事,而褒增过其实也。若夫脯、蓂荚、屈轶之属,殆无其物。何以验之?说以实者,太平无有此物。

儒者言脯生於庖厨者,言厨中自生肉脯,薄如形,摇鼓生风,寒凉食物,使之不臰。夫太平之气虽和,不能使厨生肉,以为寒凉。若能如此,则能使五谷自生,不须人为之也。能使厨自生肉,何不使饭自蒸於甑,火自燃於灶乎?凡生者,欲以风吹食物也,何不使食物自不臰,何必生以风之乎?厨中能自生,则冰室何事而复伐冰以寒物乎?人夏月操,须手摇之,然後生风,从手握持,以当疾风,不鼓动,言脯自鼓,可也?须风乃鼓,不风不动。从手风来,自足以寒厨中之物,何须脯?世言燕太子丹使日再中,天雨粟,乌白头,马生角,厨门〔木〕象生肉足。论之既虚,则脯之语,五应之类,恐无其实。

儒者又言:古者蓂荚夹阶而生,月朔日一荚生,至十五日而十五荚;於十六日,日一荚落,至月晦,荚尽,来月朔,一荚复生。王者南面视荚生落,则知日数多少,不须烦扰案日历以知之也。夫天既能生荚以为日数,何不使荚有日名,王者视荚之字则知今日名乎?徒知日数,不知日名,犹复案历然後知之,是则王者视日,则更烦扰不省蓂荚之生,安能为福?夫蓂,草之实也,犹豆之有荚也,春夏未生,其生必於秋末。冬月隆寒,霜雪陨零,万物皆枯,儒者敢谓蓂荚达冬独不死乎?如与万物俱生俱死,荚成而以秋末,是则季秋得察荚,春夏冬三时不得案也。且月十五日生十五荚,於十六日荚落,二十一日六荚落,落荚弃殒,不可得数,犹当计未落荚以知日数,是劳心苦意,非善佑也。使荚生於堂上,人君坐户牖间,望察荚生以知日数,匪谓善矣。今云``夹阶而生'',生於堂下也。王者之堂,墨子称尧、舜高三尺,儒家以为卑下。假使之然,高三尺之堂,蓂荚生於阶下,王者欲视其荚,不能从户牖之间见也,须临堂察之,乃知荚数。夫起视堂下之荚,孰与悬历日於扆坐,傍顾辄见之也?天之生瑞,欲以娱王者,须起察乃知日数,是生烦物以累之也。且荚,草也,王者之堂,旦夕所坐,古者虽质,宫室之中,草生辄耘,安得生荚而人得经月数之乎?且凡数日一二者,欲以纪识事也。古有史官典历主日,王者何事而自数荚?尧候四时之中,命曦、和察四星以占时气,四星至重,犹不躬视,而自察荚以数日也?

儒者又言:太平之时,屈轶生於庭之末,若草之状,主指佞人,佞人入朝,屈轶庭末以指之,圣王则知佞人所在。夫天能故生此物以指佞人,不使圣王性自知之,或佞人本不生出,必复更生一物以指明之,何天之不惮烦也?圣王莫过尧、舜,尧、舜之治,最为平矣。即屈轶已自生於庭之末,佞人来辄指知之,则舜何难於知佞人,而使皋陶陈知人之术?《经》曰:``知人则哲,惟帝难之。''人含五常,音气交通,且犹不能相知。屈轶,草也,安能知佞?如儒者之言,是则太平之时,草木逾贤圣也。狱讼有是非,人情有曲直,何不并令屈轶指其非而不直者,必苦心听讼,三人断狱乎?故夫屈轶之草,或时无有而空言生,或时实有而虚言能指,假令能指,或时草性见人而动。古者质朴,见草之动,则言能指,能指则言指佞人。司南之杓,投之於地,其柢指南。鱼肉之虫,集地北行,夫虫之性然也。今草能指,亦天性也。圣人因草能指,宣言曰:``庭末有屈轶能指佞人,
''百官臣子怀奸心者,则各变性易操,为忠正之行矣,犹今府廷画皋陶、\textless{}角圭
\textgreater{}\textless{}角虎\textgreater{}也。

儒者说云:觟\textless{}角虎\textgreater{}者,一角之羊也,性知有罪。皋陶治狱,其罪疑者令羊触之,有罪则触,无罪则不触。斯盖天生一角圣兽,助狱为验,故皋陶敬羊,起坐事之。此则神奇瑞应之类也。曰:夫觟\textless{}角虎\textgreater{}则复屈轶之语也。羊本二角,觟\textless{}角虎\textgreater{}一角,体损於群,不及众类,何以为奇?鳖三足曰能,龟三足曰贲。案能与贲,不能神於四足之龟鳖;一角之羊何能圣於两角之禽?

狌々知往,乾鹊知来,鹦鹉能言,天性能一,不能为二。或时觟\textless{}
角虎\textgreater{}之性,徒能触人,未必能知罪人,皋陶欲神事助政,恶受罪者之不厌服,因\textless{}
角圭\textgreater{}\textless{}角虎\textgreater{}触人则罪之,欲人畏之不犯,受罪之家,没齿无怨言也。
夫物性各自有所知,如以觟\textless{}角虎\textgreater{}能触谓之为神,则狌々之徒皆为神也。巫知吉凶,占人祸福,无不然者。如以觟\textless{}角虎\textgreater{}谓之巫类,则巫何奇而以为善?斯皆人欲神事立化也。师尚父为周司马,将师伐纣,到孟津之上,杖钺把旄,号其众曰:
``仓兕!仓兕!''仓〔兕〕者,水中之兽也,善覆人船,因神以化,欲令急渡,不急渡,仓〔兕〕害汝,则复觟\textless{}角虎\textgreater{}之类也。河中有此异物,时出浮扬,一身九头,人畏恶之,未必覆人之舟也,尚父缘河有此异物,因以威众。夫\textless{}
角圭\textgreater{}\textless{}角虎\textgreater{}之触罪人,犹仓〔兕〕之覆舟也,盖有虚名,无其实效也。人畏怪奇,故空褒增。

又言太平之时有景星。《尚书中候》曰:``尧时景星见於轸。''夫景星,或时五星也,大者岁星、太白也。彼或时岁星、太白行於轸度,古质不能推步五星,不知岁星、太白何如状,见大星则谓景星矣。《诗》又言:``东有启明,西有长庚。''亦或时复岁星、太白也。或时昏见於西,或时晨出於东,诗人不知,则名曰启明、长庚矣。然则长庚与景星同,皆五星也。太平之时,日月精明。五星,日月之类也,太平更有景星,可复更有日月乎?诗人,俗人也;《中候》之时,质世也。俱不知星。王莽之时,太白经天,精如半月,使不知星者见之,则亦复名之曰景星。《尔雅》《释四时章》曰:``春为发生,夏为长嬴,秋为收成,冬为安宁。四气和为景星。''夫如《尔雅》之言,景星乃四时气和之名也,恐非着天之大星。《尔雅》之书,《五经》之训,故儒者所共观察也,而不信从,更谓大星为景星,岂《尔雅》所言景星,与儒者之所说异哉?《尔雅》又言:``甘露时降,万物以嘉,谓之醴泉。''醴泉乃谓甘露也。今儒者说之,谓泉从地中出,其味甘若醴,故曰醴泉。二说相远,实未可知。案《尔雅》《释水》章:``泉一见一否曰瀸。槛泉正出,正出,涌出也;沃泉悬出,悬出,下出也。''是泉出之异,辄有异名。使太平之时,更有醴泉从地中出,当於此章中言之,何故反居《释四时章》中,言甘露为醴泉乎?若此,儒者之言醴泉从地中出,又言甘露其味甚甜,未可然也。

儒曰:``道至大者,日月精明,星辰不失其行,翔风起,甘露降。''雨〔霁〕而阴曀者谓之甘雨,非谓雨水之味甘也。推此以论,甘露必谓其降下时,适润养万物,未必露味甘也。亦有露甘味如饴蜜者,俱太平之应,非养万物之甘露也。何以明之?案甘露如饴蜜者,着於树木,不着五谷。彼露味不甘者,其下时,土地滋润流湿,万物洽沾濡溥。由此言之,《尔雅》且近得实。缘《尔雅》之言,验之於物,案味甘之露下着树木,察所着之树,不能茂於所不着之木。然今之甘露,殆异於《尔雅》之所谓甘露。欲验《尔雅》之甘露,以万物丰熟,灾害不生,此则甘露降下之验也。甘露下,是则醴泉矣。

\hypertarget{header-n756}{%
\subsubsection{治期篇}\label{header-n756}}

世谓古人君贤,则道德施行,施行则功成治安;人君不肖,则道德顿废,顿废则功败治乱。古今论者,莫谓不然。何则?见尧、舜贤圣致太平,桀、纣无道致乱得诛。如实论之,命期自然,非德化也。

吏百石以〔下〕,若〔斗〕食以〔上〕,居位治民,为政布教,教行与止,民治与乱,皆有命焉。或才高行洁,居位职废;或智浅操洿,治民而立。上古之黜陟幽明,考功,据有功而加赏,案无功而施罚。是考命而长禄,非实才而厚能也。论者因考功之法,据效而定贤,则谓民治国安者,贤君之所致;民乱国危者,无道之所为也。故危乱之变至,论者以责人君,归罪於为政不得其道。人君受以自责,愁神苦思,撼动形体,而危乱之变,终不减除。空愤人君人心,使明知之主,虚受之责,世论传称,使之然也。

夫贤君能治当安之民,不能化当乱之世。良医能行其针药,使方术验者,遇未死之人,得未死之病也。如命穷病困,则虽扁鹊末如之何。夫命穷病困之不可治,犹夫乱民之不可安也;药气之愈病,犹教导之安民也,皆有命时,不可令勉力也。公伯寮诉子路於季孙,子服景伯以告孔子。孔子曰:``道之将行也与,命也!道之将废也与,命也!''由此言之,教之行废,国之安危,皆在命时,非人力也。

夫世乱民逆,国之危殆,灾害系於上天,贤君之德,不能消却。《诗》道周宣王遭大旱矣。《诗》曰:``周余黎民,靡有孑遗。''言无有可遗一人不被害者。宣王贤者,嫌於德微。仁惠盛者,莫过尧、汤,尧遭洪水,汤遭大旱。水旱,灾害之甚者也,而二圣逢之,岂二圣政之所致哉?天地历数当然也。以尧、汤之水旱,准百王之灾害,非德所致,非德所致,则其福佑非德所为也。

贤君之治国也,犹慈父之治家。慈父耐平教明令,耐使子孙皆为孝善。子孙孝善,是家兴也;百姓平安,是国昌也。昌必有衰,兴必有废。兴昌非德所能成,然则衰废非德所能败也。昌衰兴废,皆天时也。此善恶之实,未言苦乐之效也。家安人乐,富饶财用足也。案富饶者命厚所致,非贤惠所获也。人皆知富饶居安乐者命禄厚,而不知国安治化行者历数吉也。故世治非贤圣之功,衰乱非无道之致。国当衰乱,贤圣不能盛;时当治,恶人不能乱。世之治乱,在时不在政;国之安危,在数不在教。贤不贤之君,明不明之政,无能损益。

世称五帝之时,天下太平,家有十年之蓄,人有君子之行。或时不然,世增其美,亦或时政致。何以审之?夫世之所以为乱者,不以贼盗众多,兵革并起,民弃礼义,负畔其上乎?若此者,由谷食乏绝,不能忍饥寒。夫饥寒并至而能无为非者寡,然则温饱并至而能不为善者希。传曰:``仓禀实,民知礼节;衣食足,民知荣辱。''让生於有余,争起於不足。谷足食多,礼义之心生;礼丰义重,平安之基立矣。故饥岁之春,不食亲戚,穰岁之秋,召及四邻。不食亲戚,恶行也;召及四邻,善义也。为善恶之行,不在人质性,在於岁之饥穰。由此言之,礼义之行,在谷足也。案谷成败,自有年岁。年岁水旱,五谷不成,非政所致,时数然也。必谓水旱政治所致,不能为政者莫过桀、纣,桀、纣之时,宜常水旱。案桀、纣之时,无饥耗之灾。灾至自有数,或时返在圣君之世。实事者说尧之洪水,汤之大旱,皆有遭遇,非政恶之所致。说百王之害,独谓为恶之应,此见尧、汤德优,百王劣也。审一足以见百,明恶足以照善。尧、汤证百王,至百王遭变,非政所致,以变见而明祸福。五帝致太平,非德所就,明矣。

人之温病而死也,先有凶色见於面部。其病,遇邪气也,其病不愈。至於身死,命寿讫也。国之乱亡,与此同验。有变见於天地,犹人温病而死,色见於面部也。有水旱之灾,犹人遇气而病也。灾祸不除,至於国亡,犹病不愈,至於身死也。论者谓变征政治,贤人温病色凶,可谓操行所生乎?谓水旱者无道所致,贤者遭病,可谓无状所得乎?谓亡者为恶极,贤者身死,可谓罪重乎?夫贤人有被病而早死,恶人有完强而老寿,人之病死,不在操行为恶也。然则国之乱亡,不在政之是非。恶人完强而老寿,非政平安而常存。由此言之,祸变不足以明恶,福瑞不足以表善,明矣。

在天之变,日月薄蚀,四十二月日一食,五六月月亦一食食有常数,不在政治,百变千灾,皆同一状,未必人君政教所致。岁害鸟帑,周、楚有祸;綝然之气见,宋、卫、陈、郑皆灾。当此之时,六国政教未必失误也。历阳之都,一夕沈而为湖,当时历阳长吏未必诳妄也。成败系於天,吉凶制於时。人事未为,天气已见,非时而何?五谷生地,一丰一耗;谷粜在市,一贵一贱。丰者未必贱,耗者未必贵。丰耗有岁,贵贱有时。时当贵,丰谷价增;时当贱,耗谷直减。夫谷之贵贱不在丰耗,犹国之治乱不在善恶。

贤君之立,偶在当治之世,德自明於上,民自善於下,世平民安,瑞佑并至,世则谓之贤君所致。无道之君,偶生於当乱之时,世扰俗乱,灾害不绝,遂以破国亡身灭嗣,世皆谓之为恶所致。若此,明於善恶之外形,不见祸福之内实也。祸福不在善恶,善恶之证不在祸福。长吏到官,未有所行,政教因前,无所改更。然而盗贼或多或寡,灾害或无或有,夫何故哉?长吏秩贵,当阶平安以升迁,或命贱不任,当由危乱以贬诎也。以今之长吏,况古之国君,安危存亡,可得论也。

\hypertarget{header-n761}{%
\subsection{卷十八}\label{header-n761}}

\hypertarget{header-n762}{%
\subsubsection{自然篇}\label{header-n762}}

天地合气,万物自生,犹夫妇合气,子自生矣。万物之生,含血之类,知饥知寒。见五谷可食,取而食之,见丝麻可衣,取而衣之。或说以为天生五谷以食人,生丝麻以衣人,此谓天为人作农夫桑女之徒也,不合自然,故其义疑,未可从也。试依道家论之。

天者,普施气万物之中,谷愈饥而丝麻救寒,故人食谷衣丝麻也。夫天之不故生五谷丝麻以衣食人,由其有灾变不欲以谴告人也。物自生,而人衣食之;气自变而人畏惧之。以若说论之,厌於人心矣。如天瑞为故,自然焉在?无为何居?

何以〔知〕天之自然也?以天无口目也。案有为者,口目之类也。口欲食而目欲视,有嗜欲於内,发之於外,口目求之,得以为利欲之为也。今无口目之欲,於物无所求索,夫何为乎?何以知天无口目也?以地知之。地以土为体,土本无口目。无地,夫妇也,地体无口目,亦知天口目也。使天体乎?宜与地同。使天气乎,气若云烟。云烟之属,安得口目?

或曰:``凡动行之类,皆本有为。有欲故动,动则有为。今天动行与人相似,安得无为?''曰:天之动行也,施气也,体动气乃出,物乃生矣。由人动气也,体动气乃出,子亦生也。夫人之施气也,非欲以生子,气施而子自生矣。天动不欲以生物,而物自生,此则自然也。施气不欲为物,而物自为,此则无为也。谓天自然无为者何?气也。恬淡无欲,无为无事者也,老聃得以寿矣。老聃禀之於天,使天无此气,老聃安所禀受此性!师无其说而弟子独言者,未之有也。或复於桓公,公曰:``以告仲父。''左右曰:``一则仲父,二则仲父,为君乃易乎?''
桓公曰:``吾未得仲父,故难;已得仲父,何为不易!''夫桓公得仲父,任之以事,委之以政,不复与知。皇天以至优之德,与王政〔随〕而谴告〔之〕,则天德不若桓公,而霸君之操过上帝也。

或曰:``桓公知管仲贤,故委任之;如非管仲,亦将谴告之矣。使天遭尧、舜,必无谴告之变。''曰:天能谴告人君,则亦能故命圣君。择才若尧、舜,受以王命,委以王事,勿复与知。今则不然,生庸庸之君,失道废德,随谴告之,何天不惮劳也!曹参为汉相,纵酒歌乐,不听政治,其子谏之,笞之二百。当时天下无扰乱之变。淮阳铸伪钱,吏不能禁,汲黯为太守,不坏一炉,不刑一人,高枕安卧,而淮阳政清。夫曹参为相若不为相,汲黯为太守若郡无人。然而汉朝无事,淮阳刑错者,参德优而黯威重也。计天之威德,孰与曹参、汲黯?而谓天与王政随而谴告之,是谓天德不若曹参厚,而威不若汲黯重也。蘧伯玉治卫,子贡使人问之:``何以治卫?''对曰:``以不治治之。''夫不治之治,无为之道也。

或曰:``太平之应,,河出图,洛出书。不画不就,不为不成。天地出之,有为之验也。张良游泗水之上,遇黄石公,授太公书,盖天佐汉诛秦,故命令神石为鬼书授人,复为有为之效也。''曰:此皆自然也。夫天安得以笔黑而为图书乎?天道自然,故图书自成。晋唐叔虞、鲁成季友生,文在其手,故叔曰``虞'',季曰``友''。宋仲子生,有文在其手,曰:``为鲁夫人。''三者在母之时,文字成矣,而谓天为文字,在母之时,天使神持锥笔墨刻其身乎?自然之化,固疑难知,外若有为,内实自然。是以太史公纪黄石事,疑而不能实也。赵简子梦上天,见一男子在帝之侧,後出,见人当道,则前所梦见在帝侧者也。论之以为赵国且昌之状也。黄石授书,亦汉且兴之象也。妖气为鬼,鬼象人形,自然之道,非或为之也。

草木之生,华叶青葱,皆有曲折,象类文章,谓天为文字,复为华叶乎?宋人或刻木为楮叶者,三年乃成。〔列〕子曰:``使〔天〕地三年乃成一叶,则万物之有叶者寡矣。''如〔列〕子之言,万物之叶自为生也。自为生也,故能并成。如天为之,其迟当若宋人刻楮叶矣。观鸟兽之毛羽,毛羽之采色,通可为乎?鸟兽未能尽实。春观万物之生,秋观其成,天地为之乎?物自然也。如谓天地为之,为之宜用手,天地安得万万千千手,并为万万千千物乎?诸物在天地之间也,犹子在母腹中也。母怀子气,十月而生,鼻、口、耳、目、发肤、毛理、血脉、脂腴、骨节、爪齿,自然成腹中乎?母为之也?偶人千万,不名为人者,何也?鼻口耳目非性自然也。武帝幸〔李〕夫人,〔李〕夫人死,思见其形。道士以方术作夫人形,形成,出入宫门,武帝大惊,立而迎之,忽不复见。盖非自然之真,方士巧妄之伪,故一见恍忽,消散灭亡。有为之化,其不可久行,犹〔李〕夫人形不可久见也。道家论自然,不知引物事以验其言行,故自然之说未见信也。

然虽自然,亦须有为辅助。耒耜耕耘,因春播种者,人为之也;及谷入地,日夜长〔大〕,人不能为也。或为之者,败之道也。宋人有闵其苗之不长者,就而揠之,明日枯死。夫欲为自然者,宋人之徒也。

问曰:``人生於天地,天地无为。人禀天性者,亦当无为,而有为,何也?''
曰:至德纯渥之人,禀天气多,故能则天,自然无为。禀气薄少,不遵道德,不似天地,故曰不肖。不肖者,不似也。不似天地,不类圣贤,故有为也。天地为炉,造化为工,禀气不一,安能皆贤?贤之纯者,黄、老是也。黄者,黄帝也;老者,老子也。黄、老之操,身中恬澹,其治无为。正身共己,而阴阳自和,无心於为而物自化,无意於生而物自成。

《易》曰:``黄帝、尧、舜垂衣裳而天下治。''垂衣裳者,垂拱无为也。孔子曰:``大哉,尧之为君也!惟天为大,惟尧则之。''又曰:``巍巍乎!舜、禹之有天下也,而不与焉。''周公曰:``上帝引佚。''上帝,谓〔虞〕舜也。〔虞〕舜承安继治,任贤使能,恭己无为而天下治。〔虞〕舜承尧之安,尧则天而行,不作功邀名,无为之化自成,故曰``荡荡乎,民无能名焉''。年五十者击壤於涂,不能知尧之德,盖自然之化也。《易》曰:``大人与天地合其德。''黄帝、尧、舜,大人也,其德与天地合,故知无为也。天道无为,故春不为生,而夏不为长,秋不为成,冬不为藏。阳气自出,物自生长;阴气自起,物自成藏。汲井决陂,灌溉园田,物亦生长,霈然而雨,物之茎叶根〔荄〕,莫不洽濡。程量澍泽,孰与汲井决陂哉!故无为之为大矣。本不求功,故其功立;本不求名,故其名成。沛然之雨,功名大矣,而天地不为也,气和而雨自集。

儒家说夫妇之道,取法於天地,知夫妇法天地,不知推夫妇之道,以论天地之性,可谓惑矣。夫天覆於上,地偃於下,下气烝上,上气降下,万物自生其中间矣。当其生也,天不须复与也,由子在母怀中,父不能知也。物自生,子自成,天地父母,何与知哉?及其生也,人道有教训之义。天道无为,听恣其性,故放鱼於川,纵兽於山,从其性命之欲也。不驱鱼令上陵,不逐兽令入渊者,何哉?拂诡其性,失其所宜也。夫百姓,鱼兽之类也。上德治之,若烹小鲜,与天地同操也。商鞅变秦法,欲为殊异之功,不听赵良之议,以取车裂之患,德薄多欲,君臣相憎怨也。道家德厚,下当其上,上安其下,纯蒙无为,何复谴告?故曰:
``政之适也,君臣相忘於治,鱼相忘於水,兽相忘於林,人相忘於世。故曰天也。
''孔子谓颜渊曰:``吾服汝,忘也;汝之服於我,亦忘也。''以孔子为君,颜渊为臣,尚不能谴告,况以老子为君,文子为臣乎?老子、文子,似天地者也。淳酒味甘,饮之者醉不相知。薄酒酸苦,宾主颦蹙。夫相谴告,道薄之验也。谓天谴告,曾谓天德不若淳酒乎?

礼者,忠信之薄,乱之首也。相讥以礼,故相谴告。三皇之时,坐者于于,行者居居,乍自以为马,乍自以为牛,纯德行而民瞳矇,晓惠之心未形生也。当时亦无灾异,如有灾异,不名曰谴告。何则?时人愚蠢,不知相绳责也。末世衰微,上下相非,灾异时至,则造谴告之言矣。夫今之天,古之天也,非古之天厚,而今之天薄也,谴告之言生於今者,人以心准况之也。诰誓不及五帝,要盟不及三王,交质子不及五伯。德弥薄者信弥衰。心险而行诐,则犯约而负教;教约不行,则相谴告;谴告不改,举兵相灭。由此言之,谴告之言,衰乱之语也,而谓之上天为之,斯盖所以疑也。

且凡言谴告者,以人道验之也。人道,君谴告臣,上天谴告君也,谓灾异为谴告。夫人道,臣亦有谏君,以灾异为谴告,而王者亦当时有谏上天之义,其效何在?苟谓天德优,人不能谏,优德亦宜玄默,不当谴告。万石君子有过,不言,对案不食,至优之验也。夫人之优者,犹能不言,皇天德大,而乃谓之谴告乎?夫天无为,故不言,灾变时至,气自为之。夫天地不能为,亦不能知也。腹中有寒,腹中疾痛,人不使也,气自为之。夫天地之间,犹人背腹之中也。谓天为灾变,凡诸怪异之类,无小大薄厚,皆天所为乎?牛生马,桃生李,如论者之言,天神入牛腹中为马,把李实提桃间乎?牢曰:``子云:`吾不试,故艺。'''又曰:``吾少也贱,故多能鄙事。''人之贱不用於大者,类多伎能。天尊贵高大,安能撰为灾变以谴告人?且吉凶蜚色见於面,人不能为,色自发也。天地犹人身,气变犹蜚色。人不能为蜚色,天地安能为气变!然则气变之见,殆自然也。变自见,色自发,占候之家,因以言也。

夫寒温、谴告、变动、招致,四疑皆已论矣。谴告於天道尤诡,故重论之,论之所以难别也。说合於人事,不入於道意。从道不随事,虽违儒家之说,合黄、老之义也。

\hypertarget{header-n766}{%
\subsubsection{感类篇}\label{header-n766}}

阴阳不和,灾变发起,或时先世遗咎,或时气自然。贤圣感类,慊惧自思,灾变恶徵,何为至乎?引过自责,恐有罪,畏慎恐惧之意,未必有其实事也。何以明之?以汤遭旱自责以五过也。圣人纯完,行无缺失矣,何自责有五过?然如《书》曰:``汤自责,天应以雨。''汤本无过,以五过自责,天何故雨?〔使〕以过致旱,〔不〕知自责,〔亦〕能得雨也。由此言之,旱不为汤至,雨不应自责。然而前旱後雨者,自然之气也。此言,《书》之语也。难之曰:《春秋》大雩,董仲舒设土龙,皆为一时间也。一时不雨,恐惧雩祭,求有请福,忧念百性也。汤遭旱七年,以五过自责,谓何时也?夫遭旱一时,辄自责乎?旱至七年,乃自责也?谓一时辄自责,七年乃雨,天应之诚,何其留也?〔如〕谓七年乃自责,忧念百姓,何其迟也?不合雩祭之法,不厌忧民之义。《书》之言未可信也。

由此论之,周成王之雷风发,亦此类也。《金滕》曰:``秋大熟未获。天大雷电以风,禾尽偃,大木斯拔,邦人大恐。''当此之时,周公死,儒者说之,以为成王狐疑於〔葬〕周公:欲以天子礼葬公,公人臣也;欲以人臣礼葬公,公有王功。狐疑於葬周公之间,天大雷雨,动怒示变,以彰圣功。古文家以武王崩,周公居摄,管、蔡流言,王意狐疑周公,周公奔楚,故天雷雨,以悟成王。夫一雷一雨之变,或以为葬疑,或以为信谗,二家未可审。且订葬疑之说,秋夏之际,阳气尚盛,未尝无雷雨也,顾其拔木偃禾,颇为〔壮〕耳。当雷雨时,成王感惧,开金滕之书,见周公之功,执书泣过,自责之深。自责适已,天偶反风,《书》家则谓天为周公怒也。千秋万夏,不绝雷雨。苟谓雷雨为天怒乎?是则皇天岁岁怒也。正月阳气发泄,雷声始动,秋夏阳至极而雷折。苟谓秋夏之雷,为天大怒,正月之雷天小怒乎?雷为天怒,雨为恩施。使天为周公怒,徒当雷,不当雨,今〔雷〕雨俱至,天怒且喜乎?``子於是日也,哭则不歌''。《周礼》``子卯稷食菜羹'',哀乐不并行。哀乐不并行,喜怒反并至乎?

秦始皇帝东封岱岳,雷雨暴至。刘媪息大泽,雷雨晦冥。始皇无道,自同前圣,治乱自谓太平,天怒可也。刘媪息大泽,梦与神遇,是生高祖,何怒於生圣人而为雷雨乎?尧时大风为害,尧〔缴〕大风於青丘之野。舜入大麓,烈风雷雨。尧、舜世之隆主,何过於天,天为风雨也?大旱,《春秋》雩祭,又董仲舒设土龙,以类招气,如天应雩龙,必为雷雨。何则?秋夏之雨,与雷俱也。必从《春秋》、仲舒之术,则大雩龙,求怒天乎?师旷奏《白雪之曲》,雷电下击,鼓《清角》之音,风雨暴至。苟为雷雨为天怒,天何憎於《白雪》《清角》,而怒师旷为之乎?此雷雨之难也。

又问之曰:``成王不以天子礼葬周公,天为雷风,偃禾拔木,成王觉悟,执书泣过,天乃反风,偃禾复起。何不为疾反风以立大木,必须国人起筑之乎?''
应曰:``天不能。''曰:``然则天有所不能乎?''应曰:``然。''难曰:``孟贲推人〔而〕人仆,接人而人立。天能拔木,不能复起,是则天力不如孟贲也。秦时三山亡,犹谓天所徒也。夫木之轻重,孰与三山?能徒三山,不能起大木,非天用力宜也。如谓三山非天所亡,然则雷雨独天所为乎?''问曰:``天之欲令成王以天子之礼葬周公,以公有圣德,以公有王功。《经》曰:`王乃得周公〔所〕自以为功代武王之说。'今天动威,以彰周公之德也。''

难之曰:``伊尹相汤伐夏,为民兴利除害,致天下太平;汤死,复相大甲,大甲佚豫,放之桐宫,摄政三年,乃退复位。周公曰:`伊尹格於皇天。'天所宜彰也。伊尹死时,天何以不为雷雨?''应曰:``以《百〔两〕篇》曰:`伊尹死,大雾三日。'大雾三日,乱气矣,非天怒之变也。东海张霸造《百〔两〕篇》,其言虽未可信,且假以问:``天为雷雨以悟成王,成王未开金匮雷止乎?已开金匮雷雨乃止也?''应曰:``未开金匮雷止也。开匮得书,见公之功,党悟泣过,决以天子孔葬公,出郊观变,天止雨反风,禾尽起。''由此言之,成王未觉悟,雷雨止矣。难曰:``伊尹〔死〕,雾三日。天何不三日雷雨,须成王觉悟乃止乎?太戊之时,桑谷生朝,七日大拱,太戊思政,桑谷消亡。宋景公时,荧〔惑〕守心,出三善言,荧惑徒舍。使太戊不思政,景公无三善言,桑谷不消,荧惑不徒。何则?灾变所以谴告也,所谴告未觉,灾变不除,天之至意也。今天怒为雷雨,以责成王,成王未觉,雨雷之息,何其早也?''

又问曰:``礼,诸侯之子称公子,诸侯之孙称公孙,皆食采地,殊之众庶。何则?公子公孙,亲而又尊,得体公称,又食采地,名实相副,犹文质相称也。天彰周公之功,令成王以天子礼葬,何不令成王号周公以周王,副天子之礼乎?''
应曰:``王者,名之尊号也,人臣不得名也。''难曰:``人臣犹得名王,礼乎?武王伐纣,下车追王大王、王季、文王。三人者,诸侯,亦人臣也,以王号加之。何为独可於三王,不可於周公?天意欲彰周公,岂能明乎?岂以王迹起於三人哉?然而王功亦成於周公。江起岷山,流为涛濑。相涛濑之流,孰与初起之源?秬鬯之所为到,白雉之所为来,三王乎?周公也?周公功德盛於三王,不加王号,岂天恶人妄称之哉?周衰,六国称王,齐、秦更为帝,当时天无禁怒之变。周公不以天子礼葬,天为雷雨以责成王,何天之好恶不纯一乎?''

又问曰:``鲁季孙赐曾子箦,曾子病而寝之。童子曰:`华而晥者,大夫之箦。'而曾子感惭,命元易箦。盖礼,大夫之箦,士不得寝也。今周公,人臣也,以天子礼葬,魂而有灵,将安之不也?''应曰:``成王所为,天之所予,何为不安?''难曰:``季孙所赐大夫之箦,岂曾子之所自制乎?何独不安乎?子疾病,子路遣门人为臣。病间曰:`久矣哉!由之行诈也!无臣而为有臣,吾谁欺,欺天乎?'孔子罪子路者也。己非人君,子路使门人为臣,非天之心而妄为之,是欺天也。周公亦非天子也,以孔子之心况周公,周公必不安也。季氏旅於太山,孔子曰:`曾谓泰山不如林放乎?'以曾子之细,犹却非礼;周公至圣,岂安天子之葬?曾谓周公不如曾子乎?由此原之,周公不安也。大人与天地合德,周公不安,天亦不安,何故为雷雨以责成王乎?''

又问曰:``死生有命,富贵在天。武王之命,何可代乎?''应曰:``九龄之梦,天夺文王年以益武王。克殷二年之时,九龄之年未尽,武王不豫,则请之矣。人命不可请,独武王可,非世常法,故藏於金滕;不可复为,故掩而不见。''难曰:``九龄之梦,武王已得文王之年未?''应曰:``已得之矣。''难曰:``已得文王之年,命当自延。克殷二年,虽病,犹将不死,周公何为请而代之?''应曰:
``人君爵人以官,议定,未之即与,曹下案目,然後可诺。天虽夺文王年以益武王,犹须周公请,乃能得之。命数精微,非一卧之梦所能得也。难曰:``九龄之梦,文王梦与武王九龄。武王梦帝予其九龄,其天已予之矣,武王已得之矣,何须复请?人且得官,先梦得爵,其後莫举,犹自得官。何则?兆象先见,其验必至也。古者谓年为龄,已得九龄,犹人梦得爵也。周公因必效之梦,请之於天,功安能大乎?''

又问曰:``功无大小,德无多少,人须仰恃赖之者,则为美矣。使周公不代武王,武王病死,周公与成王而致天下太平乎?''应曰:``成事,周公辅成王而天下不乱。使武王不见代,遂病至死,周公致太平何疑乎?''难曰:``若是,武王之生无益,其死无损,须周公功乃成也。周衰,诸侯背畔,管仲九合诸侯,一匡天下。孔子曰:`微管仲,吾其被发左衽矣。'使无管仲,不合诸侯,夷狄交侵,中国绝灭。此无管仲有所伤也。程量有益,管仲之功,偶於周公。管仲死,桓公不以诸侯礼葬,以周公况之,天亦宜怒,微雷薄雨不至,何哉?岂以周公圣而管仲贤乎?夫管仲为反坫,有三归,孔子讥之,以为不贤。反坫、三归,诸侯之礼;天子礼葬,王者之制,皆以人臣俱不得为。大人与天地合德,孔子,大人也,讥管仲之僭礼,皇天欲周公之侵制,非合德之验。《书》家之说,未可然也。
''

以见鸟迹而知为书,见蜚蓬而知为车。天非以鸟迹命仓颉,以蜚蓬使奚仲也,奚仲感蜚蓬,而仓颉起鸟迹也。晋文反国,命彻麋墨,舅犯心感,辞位归家。夫文公之彻麋墨,非欲去舅犯,舅犯感惭,自同於麋墨也。宋华臣弱其宗,使家贼六人,以铍杀华吴於宋命合左师之後。左师惧曰:``老夫无罪。''其後左师怨咎华臣,华臣备之。国人逐狗,狗入华臣之门,华臣以为左师来攻己也,逾墙而走。夫华臣自杀华吴而左师惧,国人自逐狗而华臣自走。成王之畏惧,犹此类也。心疑於不以天子礼葬公,卒遭雷雨之至,则惧而畏过矣。夫雷雨之至,天未必责成王也。雷雨至,成王惧以自责也。夫感则苍颉、奚仲之心,惧则左师、华臣之意也。怀嫌疑之计,遭暴至之气,以类之验见,则天怒之效成矣。见类验於寂漠,犹感动而畏惧,况雷雨扬〔軯〕盖之声,成王庶几能不怵惕乎?

迅雷风烈,孔子必变。礼,君子闻雷,虽夜,衣冠而坐,所以敬雷惧激气也。圣人君子,於道无嫌,然犹顺天变动,况成王有周公之疑,闻雷雨之变,安能不振惧乎?然则雷雨之至也,殆且自天气;成王畏惧,殆且感物类也。夫天道无为,如天以雷雨责怒人,则亦能以雷雨杀无道。古无道者多,可以雷雨诛杀其身,必命圣人兴师动军,顿兵伤士,难以一雷行诛,轻以三军克敌,何天之不惮烦也?

或曰:``纣父帝乙,射天殴地,游泾、渭之间,雷电击而杀之。斯天以雷电诛无道也。''帝乙之恶,孰与桀、纣?邹伯奇论桀、纣恶不如亡秦,亡秦不如王莽,然而桀、纣、秦、莽之〔死〕,不以雷电。孔子作《春秋》,采毫毛之善,贬纤介之恶,采善不逾其美,贬恶不溢其过。责小以大,夫人无之。成王小疑,天大雷雨。如定以臣葬公,其变何以过此?《洪范》稽疑,不悟灾变者,人之才不能尽晓,天不以疑责备於人也。成王心疑未决,天以大雷雨责之,殆非皇天之意。《书》家之说,恐失其实也。

\hypertarget{header-n770}{%
\subsubsection{齐世篇}\label{header-n770}}

语称上世之人,侗长佼好,坚强老寿,百岁左右;下世之人短小陋丑,夭折早死。何则?上世和气纯渥,婚姻以时,人民禀善气而生,生又不伤,骨节坚定,故长大老寿,状貌美好。下世反此,故短小夭折,形面丑恶。此言妄也。

夫上世治者,圣人也;下世治者,亦圣人也。圣人之德,前後不殊,则其治世,古今不异。上世之天,下世之天也。天不变易,气不改更。上世之民,下世之民也,俱禀元气。元气纯和,古今不异,则禀以为形体者,何故不同?夫禀气等则怀性均,怀性均,则体同;形体同,则丑好齐;丑好齐,则夭寿适。一天一地,并生万物。万物之生,俱得一气。气之薄渥,万世若一。帝王治世,百代同道。人民嫁娶,同时共礼。虽言男三十而娶,女二十而嫁,法制张设,未必奉行。何以效之?以今不奉行也。礼乐之制,存见於今,今之人民,肯行之乎?今人不肯行,古人亦不肯举。以今之人民,知古之人民也。

〔人,物也;〕物,亦物也。人生一世,寿至一百岁。生为十岁兒时,所见地上之物,生死改易者多。至於百岁,临且死时,所见诸物,与年十岁时所见,无以异也。使上世下世,民人无有异,则百岁之间,足以卜筮。六畜长短,五谷大小,昆虫草木,金石珠玉,蜎蜚蠕动,跂行喙息,无有异者,此形不异也。古之水火,今之水火也。今气为水火也,使气有异,则古之水清火热,而今水浊火寒乎?

人生长六七尺,大三四围,面有五色,寿至於百,万世不异。如以上世人民侗长佼好,坚强老寿,下世反此;则天地初立,始为人时,长可如防风之君,色如宋朝,寿如彭祖乎?从当今至千世之後,人可长如荚英,色如嫫母,寿如朝生乎?王莽之时,长人生长一丈,名曰霸出。建武年中,颖川张仲师长一丈二寸,张汤八尺有余,其父不满五尺,俱在今世,或长或短。儒者之言,竟〔大〕误也。语称上世使民以宜,伛者抱关,侏儒俳优。如皆侗长佼好,安得伛、侏之人乎?

语称上世之人,质朴易化;下世之人,文薄难治。故《易》曰:``上古之时,结绳以治,後世易之以书契。''先结绳,易化之故;後书契,难治之验也。故夫宓牺之前,人民至质朴,卧者居居,坐者于于,群居聚处,知其母不识其父。至宓牺时,人民颇文,知欲诈愚,勇欲恐怯,强欲凌弱,众欲暴寡,故宓牺作八卦以治之。至周之时,人民文薄,八卦难复因袭,故文王衍为六十四首,极其变,使民不倦。至周之时,人民〔文〕薄,故孔子作《春秋》,采毫毛之善,贬纤介之恶,称曰:``周监于二代,郁郁乎文哉!吾从周。''孔子知世浸弊,文薄难治,故加密致之罔,设纤微之禁,检〔押〕守持,备具悉极。此言妄也。

上世之人,所怀五常也;下世之人,亦所怀五常也。俱怀五常之道,共禀一气而生,上世何以质朴?下世何以文薄?彼见上世之民饮血茹毛,无五谷之食,後世穿地为井,耕土种谷,饮井食粟,有水火之调;又见上古岩居穴处,衣禽兽之皮,後世易以宫室,有布帛之饰,则谓上世质朴,下世文薄矣。

夫器业变易,性行不异。然而有质朴文薄之语者,世有盛衰,衰极久有弊也。譬犹衣食之於人也,初成鲜完,始熟香洁,少久穿败,连日臭茹矣。文质之法,古今所共。一质一文,一衰一盛,古而有之,非独今也。何以效之?传曰:``夏后氏之王教以忠。上教以忠,君子忠,其失也,小人野。救野莫如敬,殷〔之〕王教以敬。上教用敬,君子敬,其失也,小人鬼。救鬼莫如文,故周之王教以文。上教以文,君子文,其失也,小人薄。救薄莫如忠,承周而王者,当教以忠。''
夏所承唐、虞之教薄,故教以忠。唐、虞以文教,则其所承有鬼失矣。世人见当今之文薄也,狎侮非之,则谓上世朴质,下世文薄。犹家人子弟不谨,则谓他家子弟谨良矣。

语称上世之人重义轻身,遭忠义之事,得己所当赴死之分明也,则必赴汤趋锋,死不顾恨。故弘演之节,陈不占之义,行事比类,书籍所载,亡命捐身,众多非一。今世趋利苟生,弃义妄得,不相勉以义,不相激以行,义废身不以为累,行隳事不以相畏。此言妄也。

夫上世之士,今世之士也,俱含仁义之性,则其遭事并有奋身之节。古有无义之人,今有建节之士。善恶杂厕,何世无有。述事者好高古而下今,贵所闻而贱所见。辨士则谈其久者,文人则著其远者。近有奇而辨不称,今有异而笔不记。若夫琅邪兒子明,岁败之时,兄为饥人所食,自缚叩头,代兄为食,饿人美其义,两舍不食。兄死,收养其孤,爱不异於己之子,岁败谷尽,不能两活,饿杀其子,活兄之子。临淮许君叔亦养兄孤子,岁仓卒之时,饿其亲子,活兄之子,与子明同义。会稽孟章父英为郡决曹掾,郡将挝杀非辜,事至覆考,英引罪自予,卒代将死。章后复为郡功曹,从役攻贼,兵卒北败,为贼所射,以身代将,卒死不去。此弘演之节,陈不占之义何以异?当今著文书者,肯引以为比喻乎?比喻之证,上则求虞、夏,下则索殷、周。秦、汉之际,功奇行殊,犹以为后。又况当今在百代下,言事者目亲见之乎?

画工好画上代之人,秦、汉之士,功行谲奇,不肯图今世之士者,尊古卑今也。贵鹄贱鸡,鹄远而鸡近也。使当今说道深於孔、墨,名不得与之同;立行崇於曾、颜,声不得与之钧。何则?世俗之性,贱所见,贵所闻也。有人於此,立义建节,实核其操,古无以过。为文书者,肯载於篇籍,表以为行事乎?作奇论,造新文,不损於前人,好事者肯舍久远之书,而垂意观读之乎?扬子云作《太玄》,造《法言》,张伯松不肯壹观。与之并肩,故贱其言。使子云在伯松前,伯松以为《金匮》矣!

语称上世之时,圣人德优,而功治有奇。故孔子曰:``大哉,尧之为君也!唯天为大,唯尧则之。荡荡乎民无能名焉!巍巍乎其有成功也!焕乎其有文章也!
''舜承尧不堕洪业,禹袭舜不亏大功。其後至汤,举兵代桀,武王把钺讨纣,无巍巍荡荡之文,而有动兵讨伐之言。盖其德劣而兵试,武用而化薄。化薄,不能相逮之明验也。及至秦、汉,兵革云扰,战力角势,秦以得天下。既得在下,无嘉瑞之美,若``叶和万国''、``凤皇来仪''之类,非德劣不及,功被若之徵乎?此言妄也。

夫天地气和,即生圣人。圣人之治,即立大功。和气不独在古先,则圣人何故独优!世俗之性,好褒古而毁今,少所见而多所闻。又见经传增贤圣之美,孔子尤大尧、舜之功。又闻尧、舜禅而相让,汤、武伐而相夺。则谓古圣优於今,功化渥地後矣。夫经有褒增之文,世有空加之言,读经览书者所共见也。孔子曰:
``纣之不善,不若是之甚也。是以君子恶居下流,天下之恶皆归焉。''世常以桀、纣与尧、舜相反,称美则说尧、舜,言恶则举纣、桀。孔子曰``纣之不善,不若是之甚也'',则知尧、舜之德,不若是其盛也。

尧、舜之禅,汤、武之诛,皆有天命,非优劣所能为,人事所能成也。使汤、武在唐、虞,亦禅而不伐;尧、舜在殷、周,亦诛而不让。盖有天命之实,而世空生优劣之语。经言``叶和万国'',时亦有丹硃;``凤皇来仪'',时亦有有苗;兵皆动而并用,则知德亦何优劣而小大也?

世论桀、纣之恶,甚於亡秦。实事者谓亡秦恶甚於桀、纣。秦、汉善恶相反,犹尧、舜、桀、纣相违也。亡秦与汉皆在後世,亡秦恶甚於桀、纣,则亦知大汉之德不劣於唐、虞也。唐之``万国'',固增而非实者也。有虞之``凤皇'',宣帝贴已五致之矣。孝明帝符瑞并至。夫德优故有瑞,瑞钧则功不相下。宣帝、孝明如劣,不及尧、舜,何以能致尧、舜之瑞?光武皇帝龙兴凤举,取天下若拾遗,何以不及殷汤、周武?世称周之成、康不亏文王之隆,舜巍位亏尧之盛功也。方今圣朝,承光武,袭孝明,有浸酆溢美之化,无细小毫发之亏,上何以不逮舜、禹?下何以不若成、康?世见五帝、三王事在经传之上,而汉之记故尚为文书,则谓古圣优而功大,後世劣而化薄矣。

\hypertarget{header-n775}{%
\subsection{卷十九}\label{header-n775}}

\hypertarget{header-n776}{%
\subsubsection{宣汉篇}\label{header-n776}}

儒者称五帝、三王致天下太平,汉兴已来,未有太平。彼谓五帝、三王致太平,汉未有太平者,见五帝、三王圣人也,圣人之德能致太平;谓汉不太平者,汉无圣帝也,贤者之化,不能太平。又见孔子言``凤鸟不至,河不出图,吾已矣夫''!方今无凤鸟、河图,瑞颇未至悉具,故谓未太平。此言妄也。

夫太平以治定为效,百姓以安乐为符。孔子曰:``修己以安百姓,尧、舜其犹病诸!''百姓安者,太平之验也。夫治人以人为主,百姓安而阴阳和,阴阳和则万物育,万物育则奇瑞出。视今天下,安乎?危乎?安则平矣,瑞虽未具,无害於平。故夫王道定事以验,立实以效,效验不彰,实诚不见。时或实然,证验不具。是故王道立事以实,不必具验。圣主治世,期於平安,不须符瑞。

且夫太平之瑞,犹圣〔王〕之相也。圣王骨法未必同,太平之瑞何为当等?彼闻尧、舜之时,凤皇、景星皆见,河图、洛书皆出,以为後王治天下,当复若等之物,乃为太平。用心若此,犹谓尧当复比齿,舜当复八眉也。夫帝王圣相,前後不同,则得瑞古今不等。而今王无凤鸟、河图,〔谓〕未太平,妄矣。孔子言凤皇、《河图》者,假前瑞以为语也,未必谓世当复有凤皇与河图也。夫帝王之瑞,众多非一,或以凤鸟、麒麟,或以河图、洛书,或以甘露、醴泉,或以阴阳和调,或以百姓乂安。今瑞未必同於古,古应未必合於今,遭以所得,未必相袭。何以明之?以帝王兴起,命〔佑〕不同也。周则乌、鱼,汉斩大蛇。推论唐、虞,犹周、汉也。初兴始起,事效物气,无相袭者。太平瑞应,何故当钧?以已至之瑞,效方来之应,犹守株待兔之蹊,藏身破置之路也。

天下太平,瑞应各异,犹家人富殖,物不同也。或积米谷,或藏布帛,或畜牛马,或长田宅。夫乐米谷不爱布帛,欢牛马不美田宅,则谓米谷愈布帛,牛马胜田宅矣。今百姓安矣,符瑞至矣,终谓古瑞河图、凤皇不至,谓之未安,是犹食稻之人入饭稷之乡,不见稻米,谓稷为非谷也。实者,天下已太平矣,未有圣人,何以致之,未见凤皇,何以效实?问世儒不知圣,何以知今无圣人也?世人见凤皇,何以知之?既无以知之,何以知今无凤皇也?委不能知有圣与无,又不能别凤皇是凤与非,则必不能定今太平与未平也。

孔子曰:``如有王者,必世然後仁。''三十年而天下平〔也〕。汉兴,至文帝时二十余年,贾谊创议以为天下洽和,当改正朔、服色、制度,定官名,兴礼乐。文帝初即位,谦让未遑。夫如贾生之议,文帝时已太平矣。汉兴二十余年,应孔子之言``必世然後仁''也。汉一〔世〕之年数已满,太平立矣,贾生知之。况至今且三百年,谓未太平,误也。且孔子所谓一世,三十年也;汉家三百岁,十帝耀德,未平,如何?夫文帝之时,固已平矣,历世〔治〕平矣。至平帝时,前汉已灭,光武中兴,复致太平。

问曰:``文帝有瑞,可名太平,光武无瑞,谓之太平,如何?''曰:夫帝王瑞应,前後不同。虽无物瑞,百姓宁集,风气调和,是亦瑞也。何以明之?帝王治平,升封太山,告安也。秦始皇升封太山,遭雷雨之变,治未平,气未和。光武皇帝升封,天晏然无云,太平之应也,治平气应。光武之时,气和人安,物瑞等至,人气已验,论者犹疑。孝宣皇帝元康二年,凤皇集於太山,後又集於新平。四年,神雀集於长乐宫,或集於上林,九真献麟。神雀二年,凤皇、甘露降集京师。四年,凤皇下杜陵及上林。五凤三年,帝祭南郊,神光并见,或兴〔於〕谷,烛耀斋宫,十有余〔刻〕。明年,祭後土,灵光复至,至如南郊之时;甘露、神雀降集延寿万岁宫。其年三月,鸾凤集长乐宫东门中树上。甘露元年,黄龙至,见於新丰,醴泉滂流。彼凤皇虽五六至,或时一鸟而数来,或时异鸟而各至。麒麟、神雀、黄龙、鸾鸟、甘露、醴泉,祭後土、天地之时,神光灵耀,可谓繁盛累积矣。孝明时虽无凤皇,亦致〔麒〕麟、甘露、醴泉、神雀、白雉、紫芝、嘉禾,金出鼎见,离木复合。五帝、三王,经传所载瑞应,莫盛孝明。如以瑞应效太平,宣、明之年倍五帝、三王也。夫如是,孝宣、孝明可谓太平矣。

能致太平者,圣人也,世儒何以谓世未有圣人?天之禀气,岂为前世者渥,後世者泊哉!周有三圣,文王、武王、周公并时猥出。汉亦一代也,何以当少於周?周之圣王,何以当多於汉?汉之高祖、光武,周之文、武也。文帝、武帝、宣帝、孝明、今上,过周之成、康、宣王。非以身生汉世,可褒增颂叹,以求媚称也;核事理之情,定说者之实也。俗好褒远称古,讲瑞〔则〕上世为美,论治则古王为贤,睹奇於今,终不信然。使尧、舜更生,恐无圣名。猎者获禽,观者乐猎,不见渔者,心不顾也。是故观於齐不虞鲁,游於楚不欢宋。唐、虞、夏、殷同载在二尺四寸,儒者〔抽〕读,朝夕讲习,不见汉书,谓汉劣不若,亦观猎不见渔,游齐、楚不愿宋、鲁也。使汉有弘文之人,经传汉事,则《尚书》、《春秋》也,儒者宗之,学者习之,将袭旧六为七,今上、上王至高祖皆为圣帝矣。观杜抚、班固等所上《汉颂》,颂功德符瑞,汪濊深广,滂沛无量,逾唐、虞,入皇域。三代隘辟,厥深洿沮也。``殷监不远,在夏后之世。''且舍唐、虞、夏、殷,近与周家断量功德,实商优劣,周不如汉。

何以验之?周之受命者文、武也,汉则高祖、光武也。文、武受命之降怪,不及高祖、光武初起之佑;孝宣、〔孝〕明之瑞,美於周之成、康、宣王。孝宣、孝明符瑞,唐、虞以来,可谓盛矣。今上即命,奉成持满,四海混一,天下定宁。物瑞已极,人应〔斯〕隆。唐世黎民雍熙,今亦天下修仁,岁遭运气,谷颇不登,迥路无绝道之忧,深幽无屯聚之奸。周家越常献白雉,方今匈奴、善鄯、哀牢贡献牛马。周时仅治五千里内,汉氏廓土收荒服之外。牛马珍於白雉,近属不若远物。古之戎狄,今为中国;古之裸人,今被朝服;古之露首,今冠章甫;古之跣跗,今履〔高〕舄。以盘石为沃田,以桀暴为良民,夷坎坷为平均,化不宾为齐民,非太平而何?夫实德化则周不能过汉,论符瑞则汉盛於周,度土境则周狭於汉,汉何以不如周?独谓周多圣人,治致太平?儒者称圣泰隆,使圣卓而无迹;称治亦泰盛,使太平绝而无续也。

\hypertarget{header-n780}{%
\subsubsection{恢国篇}\label{header-n780}}

颜渊喟然叹曰:``仰之弥高,钻之弥坚。''此言颜渊学於孔子,积累岁月,见道弥深也。《宣汉》之篇,高汉於周,拟汉过周,论者未极也。恢而极之,弥见汉奇。夫经熟讲者,要妙乃见;国极论者,恢奇弥出。恢论汉国在百代之上,审矣。何以验之?黄帝有涿鹿之战;尧有丹水之师;舜时有苗不服;夏启有扈叛逆;高宗伐鬼方三年克之。周成王管、蔡悖乱,周公东征。前代皆然,汉不闻此。高祖之时,陈犭希反,彭越叛,治始安也。孝景之时,吴、楚兴兵,怨晃错也。匈奴时扰,正朔不及,无荒之地,王功不加兵,今皆内附,贡献牛马。此则汉之威盛,莫敢犯也。

纣为至恶,天下叛之。武王举兵,皆愿就战,八百诸侯,不期俱至。项羽恶微,号而用兵,与高祖俱起,威力轻重,未有所定,则项羽力劲。折铁难於摧木,高祖诛项羽,折铁;武王伐纣,摧木。然则汉力胜周多矣。凡克敌一则易,二则难。汤、武伐桀、纣,一敌也;高祖诛秦杀项,兼胜二家,力倍汤、武。武王为殷西伯,臣事於纣,以臣伐〔君〕,夷、齐耻之,扣马而谏,武王不听,不食周粟,饿死首阳。高祖不为秦臣,光武不仕王莽,诛恶伐无道,无伯夷之讥,可谓顺於周矣。

丘由易以起高,渊洿易以为深。起於微贱,无所因阶者难;袭爵乘位,尊祖统业者易。尧以唐侯入嗣帝位,舜以司徒因尧授禅,禹以司空缘功代舜,汤由七十里,文王百里,为西伯,武王袭文王位。三郊五代之起,皆有因缘,力易为也。高祖从亭长提三尺剑取天下,光武由白水奋威武,〔帝〕海内,无尺土所因,一位所乘,直奉天命,推自然。此则起高於渊洿,为深於丘山也。比方五代,孰者为优?

传书或称武王伐纣,太公《阴谋》,食小兒以丹,令身纯赤,长大,教言殷亡。殷民见兒身赤,以为天神,及言殷亡,皆谓商灭。兵至牧野,晨举脂烛,奸谋惑民,权掩不备,周之所讳也,世谓之虚。汉取天下,无此虚言。《武成》之篇言,周伐纣,血流浮杵。以《武成》言之,食兒以丹,晨举脂烛,殆且然矣。汉伐亡新,光武将五千人,王莽遣二公将〔百〕万人战於昆阳,雷雨晦冥,前後不相见。汉兵出昆阳城,击二公军,一而当十,二公兵散。天下以雷雨助汉威敌,孰与举脂烛以人事谲取殷哉?

或云:``武王伐纣,纣赴火死,武王就斩以钺,悬其首於大白之旌。''齐宣王怜衅钟之牛,睹其色之觳觫也。楚庄王赦郑伯之罪,见其肉袒而形暴也。君子恶〔恶〕,不恶其身。纣尸赴於火中,所见凄怆,非徒色之觳觫,袒之暴形也。就斩以钺,悬乎其首,何其忍哉!高祖入咸阳,阎乐诛二世,项羽杀子婴,高祖雍容入秦,不戮二尸。光武入长安,刘圣公已诛王莽,乘兵即害,不刃王莽之死。夫斩赴火之首,与贳被刃者之身,德虐孰大也?岂以羑里之恨哉?以人君拘人臣,其逆孰与秦夺周国,莽鸩平帝也?邹伯奇论桀、纣之恶不若亡秦,亡秦不若王莽。然则纣恶微而周诛之痛,秦、莽罪重而汉伐之轻,宽狭谁也?

高祖母妊之时,蛟龙在上,梦与神遇;好酒〔贳〕饮,酒舍负仇,及醉留卧,其上常有神怪;夜行斩蛇,蛇妪悲哭;与吕后俱之田庐,时自隐匿,光气暢见,吕后辄知;始皇望见东南有天子气。及起,五星聚於东井。楚望汉军,云气五色。光武且生,凤皇集於城,嘉禾滋於屋。皇妣之身,夜半无烛,空中光明。初者,苏伯阿望舂陵气,郁郁葱葱。光武起过旧庐,见气憧憧上属於天。五帝、三王初生始起,不闻此怪。尧母感於赤龙,及起,不闻奇佑。禹母吞薏苡,将生,得玄圭;契母咽燕子;汤起白狼衔钩;後稷母履大人之迹;文王起得赤雀;武王得鱼、乌:皆不及汉太平之瑞。黄帝、尧、舜凤皇一至。凡诸众瑞,重至者希。汉文帝黄龙、玉桮。武帝黄龙、麒麟、连木。宣帝凤皇五至,麒麟、神雀、甘露、醴泉、黄龙、神光。平帝白雉、黑雉。孝明麒麟、神雀、甘露、醴泉、白雉、黑雉、芝草、连木、嘉禾,与宣帝同,奇有神鼎黄金之怪。一代之瑞,累仍不绝。此则汉德丰茂,故瑞佑多也。孝明天崩,今上嗣位,元二之间,嘉德布流。三年,零陵生芝草五本。四年,甘露降五县。五年,芝复生六〔本〕,黄龙见,大小凡八。前世龙见不双,芝生无二,甘露一降。而今八龙并出,十一芝累生,甘露流五县。德惠盛炽,故瑞繁夥也。自古帝王,孰能致斯?

儒者论曰``王者推行道德,受命於天。''《论衡》《初〔禀〕》以为王者生禀天命,性命难审,且两论之。酒食之赐,一则为薄,再则为厚。如儒者之言,五代皆一受命,唯汉独再,此则天命於汉厚也。如审《论衡》之言,生禀自然,此亦汉家所禀厚也。绝而复属,死而复生。世有死而复生之人,人必谓之神。汉统绝而复属,江武存亡,可谓优矣。

武王伐纣,庸、蜀之夷佐战牧野。成王之时,越常献雉,倭人贡暢。幽、历衰微,戎狄攻周,平王东走,以避其难。至汉,四夷朝贡。孝平元始元年,越常重译,献白雉一、黑雉二。夫以成王之贤,辅以周公,越常献一,平帝得三。後至四年,金城塞外羌〔豪〕良愿等〔种〕献其鱼盐之地,愿内属汉,遂得西王母石室,因为西海郡。周时戎狄攻王,至汉内属,献其宝地。西王母国在绝极之外,而汉属之。德孰大?壤孰广?

方今哀牢、鄯善、诺降附归德,匈奴时扰,遣将攘讨,获虏生口千万数。夏禹倮入吴国,太伯采药,断发文身。唐、虞国界,吴为荒服,越在九夷,罽衣关头,今皆夏服、褒衣、履舄。巴、蜀、越■、郁林、日南、辽东、乐浪,周时被发椎髻,今戴皮弁;周时重译,今吟《诗》、《书》。

《春秋》之义,君亲无将,将而必诛。广陵王荆迷於\{薜女\}巫,楚王英惑於〔侠〕客,事情列见。孝明三宥,二王吞药,周诛管、蔡,违斯远矣!楚外家许氏与楚王谋议,孝明曰:``许〔氏〕有属於王,欲王尊贵,人情也。''圣心原之,不绳於法。隐强侯传悬书市里,诽谤圣政;今上海〔恩〕,犯夺爵土。恶其人者,憎其胥余。立二王之子,安楚、广陵,〔隐〕强弟员嗣祀阴氏。二王,帝族也,位为王侯,与管、蔡同。管、蔡灭嗣,二王立後,恩已褒矣。隐强,异姓也,尊重父祖,复存其祀。立武庚之义,继禄父之恩,方斯羸矣。何则?并为帝王,举兵相征,贪天下之大,绝成汤之统,非圣君之义,失承天之意也。隐强,臣子也。汉统自在,绝灭阴氏,无损於义,而犹存之,惠滂沛也。故夫雨露之施,内则注於骨肉,外则布於他施。唐之晏晏,舜之烝烝,岂能逾此!

欢兜之行,靖言庸回,共工私之,称荐於尧。三苗巧佞之人,或言有罪之国。鲧不能治水,知力极尽。罪皆在身,不加於上,唐、虞放流,死於不毛。怨恶谋上,怀挟叛逆。考事失实,误国杀将,罪恶重於四子。孝明加恩,则论徙边,今上宽惠,还归州里。开辟以来,因莫斯大。晏子曰:``钩星在房、心之间,地其动乎!''夫地动天时,非政所致。皇帝振畏,犹归於治,广征贤良,访求过阙。高宗之侧身,周成之开匮,\textless{}廑力\textgreater{}能逮此。谷登岁平,庸主因缘以建德政,颠沛危殆,圣哲优者,乃立功化。是故微病恆医皆巧,笃剧扁鹊乃良。建初孟年,无妄气至,岁之疾疫也,比旱不雨,牛死民流,可谓剧矣。皇帝敦德,俊乂在官,第五司空,股肱国维,转谷振赡,民不乏饿,天下慕德,虽危不乱。民饥於谷,饱於道德,身流在道,心回乡内。以故道路无盗贼之迹,深幽迥绝无劫夺之奸,以危为宁,以困为通,五帝、三王,孰能堪斯哉?

\hypertarget{header-n784}{%
\subsubsection{验符篇}\label{header-n784}}

永平十一年,庐江皖侯国际有湖。皖民小男曰陈爵、陈挺,年皆十岁以上,相与钓於湖涯。挺先钓,爵後往。爵问挺曰:``钓宁得乎?''挺曰:``得!。''
爵即归取竿纶,去挺四十步所,见湖涯有酒樽,色正黄,没水中。爵以为铜也,涉水取之,滑重不能举。挺望见,号曰:``何取?''爵曰:``是有铜,不能举也。
''挺往助之,涉水未持,樽顿衍更为盟盘,动行入深渊中,复不见。挺、爵留顾,见如钱等,正黄,数百千〔枚〕,即共掇〔摭〕,各得满手,走归示其家。爵父国,故免吏,字君贤,惊曰:``安所得此?''爵言其状,君贤曰:``此黄金也!。
''即驰与爵俱往,到金处,水中尚多,贤自涉水掇取。爵、挺邻伍并闻,俱竟采之,合得十余斤。贤自言於相,相言太守。太守遗吏收取,遣门下掾程躬奉献,具言得金状。诏书曰:``如章则可。不如章,有正法。''躬奉诏书,归示太守,太守以下,思省诏书,以为疑隐,言之不实,苟饰美也,即复因却上得黄金实状如前章。事寝。十二年,贤等上书曰:``贤等得金湖水中,郡牧献,讫今不得直。
''诏书下庐江,上不畀贤等金直状。郡上贤等所采金自官湖水,非贤等私渎,故不与直。''十二年,诏书曰:``视时金价,畀贤等金直。''汉瑞非一,金出奇怪,故独纪之。

金玉神宝,故出诡异。金物色先为酒樽,後为盟盘,动行入渊,岂不怪哉?夏之方盛,远方图物,贡金九牧,禹谓之瑞,铸以为鼎。周之九鼎,远方之金也。人来贡之,自出於渊者,其实一也。皆起盛德,为圣王瑞。金玉之世,故有金玉之应。文帝之时,玉棒见。金之与玉,瑞之最也。金声玉色,人之奇也。永昌郡中亦有金焉,纤靡大如黍粟,在水涯沙中。民采得,日重五铢之金,一色正黄。土生金,土色黄。汉,土德也,故金化出。金有三品,黄比见者,黄为瑞也。圯桥老父遗张良书,化为黄石。黄石之精,出为符也。夫石,金之类也,质异色钧,皆土瑞也。

建初三年,零陵泉陵女子傅宁宅,土中忽生芝草五本,长者尺四五寸,短者七八寸,茎叶紫色,盖紫芝也。太守沈酆遗门下掾衍盛奉献,皇帝悦怿,赐钱衣食。诏会公卿,郡国上计吏民皆在,以芝告示天下。天下并闻,吏民欢喜,咸知汉德丰雍,瑞应出也。四年,甘露下泉陵、零陵、洮阳、始安、冷道五县,榆柏梅李,叶皆洽薄,威委流漉,民嗽吮之,甘如饴蜜。五年,芝草复生泉陵男子周服宅〔土〕,六本,色状如三年芝,并前凡十一本。

湘水去泉陵城七里,水上聚石曰燕室丘,临水有侠山,其下岩淦,水深不测,二黄龙见,长出十六丈,身大於马,举头顾望,状如图中画龙,燕室丘民皆观见之。去龙可数十步,又见状如驹马,小大凡六,出水遨戏陵上,盖二龙之子也。并二龙为八,出移一时乃入。宣帝时,凤皇下彭城,彭城以闻。宣帝诏侍中宋翁一。翁一曰:``凤皇当下京师,集於天子之郊,乃远下彭城,不可收,与无下等。
''宣帝曰:``方今天下合为一家,下彭城与京师等耳,何令可与无下等乎?''令左右通经者论难翁一,翁一穷,免冠叩头谢。宣帝之时,与今无异。凤皇之集,黄龙之出,钧也。彭城、零陵,远近同也。帝宅长远,四表为界,零陵在内,犹为近矣。鲁人公孙臣,孝文时言汉土德,其符黄龙当见。其後,黄龙见於成纪。成纪之远,犹零陵也。孝武、孝宣时,黄龙皆出。黄龙比出,於兹为四。汉竟土德也。

贾谊创议於文帝之朝云:``汉色当尚黄,数以五为名。''贾谊,智襄之臣,云色黄数五,土德审矣。芝生於土,土气和,故芝生土。土爰稼穑,稼穑作甘,故甘露集。龙见,往世不双,唯夏盛时,二龙在庭,今龙双出,应夏之数,治谐偶也。龙出往世,其子希出,今小龙六头,并出遨戏,象乾坤六子,嗣後多也。唐、虞之时,百兽率舞,今亦八龙遨戏良久。芝草延年,仙者所食,往世生出不过一二,今并前後凡十一本,多获寿考之徵,生育松乔之粮也。甘露之降,往世一所,今流五县,应土之数,德布濩也。皇瑞比见,其出不空,必有象为,随德是应。

孔子曰:``知者乐,仁者寿。''皇帝圣人,故芝草寿徵生。黄为土色,位在中央,故轩辕德优,以黄为号。皇帝宽惠,德侔黄帝,故龙色黄,示德不异。东方曰仁,龙,东方之兽也,皇帝圣人,故仁瑞见。仁者,养育之味也,皇帝仁惠爱黎民,故甘露降。龙,潜藏之物也,阳见於外,皇帝圣明,招拔岩穴也。瑞出必由嘉士,佑至必依吉人也。天道自然,厥应偶合。圣主获瑞,亦出群贤。君明臣良,庶事以康。文、武受命,力亦周、邵也。

\hypertarget{header-n789}{%
\subsection{卷二十}\label{header-n789}}

\hypertarget{header-n790}{%
\subsubsection{须颂篇}\label{header-n790}}

古之帝王建鸿德者,须鸿笔之臣褒颂纪载,鸿德乃彰,万世乃闻。问说《书》者:```钦明文思'以下,谁所言也?''曰:``篇家也。''``篇家谁也?''``孔子也。''然则孔子鸿笔之人也。``自卫反鲁,然後乐正,《雅》、《颂》各得其所也。''鸿笔之奋,盖斯时也。或说《尚书》曰:``尚者,上也;上所为,下所书也。''``下者谁也?''曰:``臣子也。''然则臣子书上所为矣。问儒者:``礼言制,乐言作,何也?''曰:``礼者上所制,故曰制;乐者下所作,故曰作。天下太平,颂声作。''方今天下太平矣,颂诗乐声可以作未?传者不知也,故曰拘儒。卫孔悝之鼎铭,周臣劝行。孝宣皇帝称颍川太守黄霸有治状,赐金百斤,汉臣勉政。夫以人主颂称臣子,臣子当褒君父,於义较矣。虞氏天下太平,夔歌舜德;宣王惠周,《诗》颂其行;召伯述职,周歌棠树。是故《周颂》三十一,《殷颂》五,《鲁颂》四,凡《颂》四十篇,诗人所以嘉上也。由此言之,臣子当颂,明矣。

儒者谓汉无圣帝,治化未太平。《宣汉》之篇,论汉已有圣帝,治已太平;《恢国》之篇,极论汉德非常实然,乃在百代之上。表德颂功,宣褒主上,《诗》之颂言,右臣之典也。舍其家而观他人之室,忽其父而称异人之翁,未为德也。汉,今天下之家也;先帝、今上民臣之翁也。夫晓主德而颂其美,识国奇而恢其功,孰与疑暗不能也?孔子称``大哉!尧之为君也!唯天为大,唯尧则之。荡荡乎民无能名焉''!或年五十击壤於涂,或曰:``大哉!尧之德也。''击壤者曰:
``吾日出而作,日入而息,凿井而饮,耕田而食,尧何等力?''孔子乃言``大哉!尧之德''者,乃知尧者也。涉圣世不知圣主,是则盲者不能别青黄也;知圣主不能颂,是则暗者不能言是非也。然则方今盲喑之儒,与唐击壤之民,同一才矣。夫孔子及唐人言大哉者,知尧德,盖尧盛也;击壤之民云``尧何等力'',是不知尧德也。

夜举灯烛,光曜所及,可得度也;日照天下,远近广狭,难得量也。浮於淮、济,皆知曲折;入东海者,不晓南北。故夫广大从横难数,极深,揭历难测。汉德酆广,日光海外也。知者知之,不知者不知汉盛也。汉家著书,多上及殷、周,诸子并作,皆论他事,无褒颂之言,《论衡》有之。又《诗》颂国名《周颂》,杜抚、〔班〕固所上《汉颂》,相依类也。

宣帝之时,画图汉列士,或不在於画上者,子孙耻之。何则?父祖不贤,故不画图也。夫颂言,非徒画文也。如千世之後,读经书不见汉美,後世怪之。故夫古之通经之臣,纪主令功,记於竹帛;颂上令德,刻於鼎铭。文人涉世,以此自勉。汉德不及六代,论者不德之故也。

地有丘洿,故有高平,或以锸平而夷之,为平地矣。世见五帝、三王为经书,汉事不载,则谓五、三优於汉矣。或以论为锸,损五、〔三〕,少丰满汉家之下,并为平哉!汉将为丘,五、三转为洿矣。湖池非一,广狭同也,树竿测之,深浅可度。汉与百代俱为主也,实而论之,优劣可见。故不树长竿,不知深浅之度;无《论衡》之论,不知优劣之实。汉在百代之末,上与百代料德,湖池相与比也,无鸿笔之论,不免庸庸之名。论好称古而毁今,恐汉将在百代之下,岂徒同哉!

谥者,行之迹也。谥之美者,成、宣也;恶者,灵、历也。成汤遭旱,周宣亦然。然而成汤加``成'',宣王言``宣'',无妄之灾,不能亏政,臣子累谥,不失实也。由斯以论尧,尧亦美谥也,时亦有洪水,百姓不安,犹言尧者,得实考也。夫一字之谥,尚犹明主,况千言之论,万文之颂哉?

船车载人,孰与其徒多也?素车朴船,孰与加漆采画也?然则鸿笔之人,国之船车、采画也。农无〔强〕夫,谷粟不登;国无强文,德暗不彰。汉德不休,乱在百代之间,强笔之儒不著载也。高祖以来,著书非不讲论汉。司马长卿为《封禅书》,文约不具。司马子长纪黄帝以至孝武,扬子云录宣帝以至哀、平。陈平仲纪光武。班孟坚颂孝明。汉家功德,颇可观见。今上即命,未有褒载,《论衡》之人,为此毕精,故有《齐世》、《宣汉》、《恢国》、《验符》。

龙无云雨不能参天。鸿笔之人,国之云雨也。载国德於传书之上,宣昭名於万世之後,厥高非徒参天也。城墙之土,平地之壤也,人加筑蹈之力,树立临池。国之功德,崇於城墙,文人之笔,劲於筑蹈。圣主德盛功立,〔若〕不褒颂纪载,奚得传驰流去无疆乎?人有高行,或誉得其实,或欲称之不能言,或谓不善,不肯陈一。断此三者,孰者为贤?五、三之际,於斯为盛。孝明之时,众瑞并至,百官臣子,不为少矣,唯班固之徒,称颂国德,可谓誉得其实矣。颂文谲以奇,彰汉德於百代,使帝名如日月,孰与不能言,言之不美善哉?

秦始皇东南游,升会稽山,李斯刻石,纪颂帝德。至琅琊亦然。秦无道之国,刻石文世,观读之者见尧、舜之美。由此言之,须颂明矣。当今非无李斯之才也,无从升会稽历琅琊之阶也。弦歌为妙异之曲,坐者不曰善,弦歌之人,必怠不精。何则?妙异难为,观者不知善也。圣国扬妙异之政,众臣不颂,将顺其美,安得所施哉?今方〔技〕之书在竹帛,无主名所从生出,见者忽然,不卸服也。如题曰``〔某〕甲某子之方,''若言``已验尝试,''人争刻写,以为珍秘。上书於国,奏〔记〕於郡,誉荐士吏,称术行能,章下记出,士吏贤妙。何则?章表其行,记明其才也。国德溢炽,莫有宣褒,使圣国大汉有庸庸之名,咎在俗儒不实论也。

古今圣王不绝,则其符瑞亦宜累属。符瑞之出,不同於前,或时已有,世无以知,故有《讲瑞》。俗儒好长古而短今,言瑞则渥前而薄後。《是应》实而定之,汉不为少。汉有实事,儒者不称;古有虚美,诚心然之。信久远之伪,忽近今之实。斯盖三增九虚所以成也,《能圣》《实圣》,所以兴也。儒者称圣过实,稽合於汉,汉不能及。非不能及,儒者之说使难及也。〔如〕实论之,汉更难及。谷熟岁平,圣王因缘以立功化,故《治期》之篇,为汉激发。治有期,乱有时。能以乱为治者优,优者有之。建初孟年,无妄气至,圣世之期也。皇帝执德,救备其灾,故《顺鼓》、《明雩》,为汉应变。是故灾变之至,或在圣世。时旱祸湛,为汉论灾。是故《春秋》为汉制法,《论衡》为汉平说。从门应庭,听堂室之言,什而失九,如升堂窥室,百不失一。《论衡》之人在古荒流之地,其远非徒门庭也。

日刻径重千里,人不谓之广者,远也。望夜甚雨,月光不暗,人不睹曜者,隐也。圣者垂日月之明,处在中州。隐於百里,遥闻传授,不实。形耀不实,难论。得诏书到,计吏至,乃闻圣政。是以褒功失丘山之积,颂德遗膏腴之美。使至台阁之下,蹈班、贾之迹,论功德之实,不失毫厘之微。武王封比干之墓,孔子显三累之行。大汉之德,非直比干三累也。道立〔邮〕表,路出其下,望〔邮〕表者昭然知路。汉德明著,莫立邦表之言,故浩广之德未光於世也。

\hypertarget{header-n794}{%
\subsubsection{佚文篇}\label{header-n794}}

孝武皇帝封弟为鲁恭王。恭王坏孔子宅以为宫,得佚《尚书》百篇,《礼》三百,《春秋》三十篇,《论语》二十一篇,闻弦歌之声,俱复封涂,上言武帝。武帝遣吏发取,古经《论语》,此时皆出。经传也而有〔闻〕弦歌之声,文当兴於汉,喜乐得闻之祥也。当传於汉,寝藏墙壁之中,恭王〔闻〕之,圣王感动弦歌之象。此则古文不当掩,汉俟以为符也。孝成皇帝读百篇《尚书》,博士郎吏莫能晓知,征天下能为《尚书》者。东海张霸通《左氏春秋》,案百篇序,以《左氏》训诂造作百二篇,具成奏上。成帝出秘《尚书》以考校之,无一字相应者,成帝下霸於吏,吏当器辜大不谨敬。成帝奇霸之才,赦其辜,亦不〔灭〕其经,故百二《尚书》传在民间。孔子曰``才难'',能推精思,作经百篇,才高卓遹,希有之人也。成帝赦之,多其文也。虽奸非实,次序篇句,依倚事类,有似真是,故不烧灭之。疏一椟,相遣以书,书十数札,奏记长吏,文成可观,读之满意,百不能一。张霸推精思至於百篇,汉世〔寡〕类,成帝赦之,不亦宜乎?杨子山为郡上计吏,见三府为《哀牢传》不能成,归郡作上,孝明奇之,征在兰台。夫以三府掾吏,丛积成才,不能成一篇。子山成之,上览其文。子山之传,岂必审是?传闻依为之有状,会三府之士,终不能为,子山为之,斯须不难。成帝赦张霸,岂不有以哉?

孝武之时,诏百官对策,董仲舒策文最善。王莽时,使郎吏上奏,刘子骏章尤美。美善不空,才高知深之验也。《易》曰:``圣人之情见於辞。''文辞美恶,足以观才。永平中,神雀群集,孝明诏上《〔神〕爵颂》,百官颂上,文皆比瓦石,唯班固、贾逵、傅毅、杨终、侯讽五颂金玉,孝明览焉。夫以百官之众,郎吏非一,唯五人文善,非奇而何?孝武善《子虚》之赋,征司马长卿。孝成玩弄众书之多,善扬子云,出入游猎,子云乘从。使长卿、桓君山、子云作吏,书所不能盈牍,文所不能成句,则武帝何贪?成帝何欲?故曰:``玩扬子云之篇,乐於居千石之官;挟桓君山之书,富於积猗顿之财。''

韩非之书,传在秦庭,始皇叹曰:``独不得与此人同时!''陆贾《新语》,每奏一篇,高祖左右,称曰万岁。夫叹思其人,与喜称万岁,岂可空为哉?诚见其美,欢气发於内也。候气变者,於天不於地,天,文明也。衣裳在身,文着於衣,不在於裳,衣法天也。察掌理者左不观右,左文明也。占在右,不观左,右,文明也。《易》曰:``大人虎变其文炳,君子豹变其文蔚。''又曰:``观乎天文,观乎人文。''此言天人以文为观,大人君子以文为操也。高祖在母身之时,息於泽陂,蛟龙在上,龙觩炫耀;及起,楚望汉军,气成五采;将入咸阳,五星聚东井,星有五色。天或者憎秦,灭其文章;欲汉兴之,故先受命以文为瑞也。

恶人操意,前後乖违。始皇前叹韩非之书,後惑李斯之议;燔《五经》之文,设挟书之律。五经之儒,抱经隐匿,伏生之徒,窜藏土中。殄贤圣之文,厥辜深重,嗣之及孙。李斯创议,身伏五刑。汉兴,易亡秦之轨,削李斯之迹。高祖始令陆贾造书,未兴《五经》。惠、景以至元、成,经书并修。汉朝郁郁,厥语所闻,孰与亡秦?王莽无道,汉军云起,台阁废顿,文书弃散。光武中兴,修存未详。孝明世好文人,并征兰台之官,文雄会聚。今上即〔命〕,诏求亡失,购募以金,安得不有好文之声!唐、虞既远,所在书散;殷、周颇近,诸子存焉。汉兴以来,传文未远,以所闻见,伍唐、虞而什殷、周,焕炳郁郁,莫盛於斯!天晏,者星辰晓烂;人性奇者,掌文藻炳。汉今为盛,故文繁凑也。

孔子曰:``文王既殁,文不在兹乎!''文王之文,传在孔子。孔子为汉制文,传在汉也。受天之文。文人宜遵五经六艺为文,诸子传书为文,造论著说为文,上书奏记为文,文德之操为文。立五文在世,皆当贤也。造论著说之文,尤宜劳焉。何则?发胸中之思,论世俗之事,非徒讽古经、续故文也。论发胸臆,文成手中,非说经艺之人所能为也。周、秦之际,诸子并作,皆论他事,不颂主上,无益於国,无补於化。造论之人,颂上恢国,国业传在千载,主德参贰日月,非适诸子书传所能并也。上书陈便宜,奏记荐吏士,一则为身,二则为人。繁文丽辞,无上书文德之操。治身完行,徇利为私,无为主者。夫如是,五文之中,论者之文多矣。则可尊明矣。

孔子称周曰:``唐、虞之际,於斯为盛,周之德,其可谓至德已矣!''孔子,周之文人也,设生汉世,亦称汉之至德矣。赵他王南越,倍主灭使,不从汉制,箕踞椎髻,沉溺夷俗。陆贾说以汉德,惧以帝威,心觉醒悟,蹶然起坐。世儒之愚,有赵他之惑;鸿文之人,陈陆贾之说。观见之者,将有蹶然起坐,赵他之悟。汉氏浩烂,不有殊卓之声。文人之休,国之符也。

望丰屋知名家,睹乔木知旧都。鸿文在国,圣世之验也。孟子相人以眸子焉,心清则眸子了,了者,目文了也。夫候国占人,同一实也。国君圣而文人聚,人心惠而目多采。蹂蹈文锦於泥涂之中,闻见之者,莫不痛心。知文锦之可惜,不知文人之当尊,不通类也。天文人文,文岂徒调墨弄笔,为美丽之观哉?载人之行,传人之名也。善人愿载,思勉为善;邪人恶载,力自禁裁。然则文人之笔,劝善惩恶也。谥法所以章善,即以著恶也。加一字之谥,人犹劝惩,闻知之者,莫不自勉。况极笔墨之力,定善恶之实,言行毕载,文以千数,传流於世,成为丹青,故可尊也。

扬子云作《法言》,蜀富人赍钱千万,愿载於书。子云不听,``夫富无仁义之行,〔犹〕圈中之鹿,栏中之牛也,安得妄载?班叔皮续《太史公书》,载乡里人以为恶戒。邪人枉道,绳墨所弹,安得避讳?是故子云不为财劝,叔皮不为恩挠。文人之笔,独已公矣!贤圣定意於笔,笔集成文,文具情显,後人观之,以〔见〕正邪,安宜妄记?足蹈於地,迹有好丑;文集於礼,志有善恶。故夫占迹以睹足,观文以知情。《诗》三百,一言以蔽之,曰:``思无邪。''《论衡》篇以十数,亦一言也,曰:``疾虚妄。''

\hypertarget{header-n798}{%
\subsubsection{论死篇}\label{header-n798}}

世谓人〔死〕为鬼,有知,能害人。试以物类验之,人〔死〕不为鬼,无知,不能害人。何以验之?验之以物。

人,物也;物,亦物也。物死不为鬼,人死何故独能为鬼?世能别人物不能为鬼,则为鬼不为鬼尚难分明。如不能别,则亦无以知其能为鬼也。人之所以生者,精气也,死而精气灭,能为精气者,血脉也。人死血脉竭,竭而精气灭,灭而形体朽,朽而成灰土,何用为鬼?人无耳目则无所知,故聋盲之人,比於草木。夫精气去人,岂徒与无耳目同哉?朽则消亡,荒忽不见,故谓之鬼神。人见鬼神之形,故非死人之精也。何则?鬼神,荒忽不见之名也。人死精神升天,骸骨归土,故谓之鬼。鬼者,归也;神者,荒忽无形者也。或说:鬼神,阴阳之名也。阴气逆物而归,故谓之鬼;阳气导物而生,故谓之神。神者,〔申〕也。申复无已,终而复始。人用神气生,其死复归神气。阴阳称鬼神,人死亦称鬼神。气之生人,犹水之为冰也。水凝为冰,气凝为人;冰释为水,人死复神。其名为神也,犹冰释更名水也。人见名异,则谓有知,能为形而害人,无据以论之也。

人见鬼若生人之形。以其见若生人之形,故知非死人之精也。何以效之?以囊橐盈粟米,米在囊中,若粟在橐中,满盈坚强,立树可见。人瞻望之,则知其为粟米囊橐。何则?囊橐之形,若其容可察也。如囊穿米出,橐败粟弃,则囊橐委辟,人瞻望之,弗复见矣。人之精神藏於形体之内,犹粟米在囊橐之中也。死而形体朽,精气散,犹囊橐穿败,粟米弃出也。粟米弃出,囊橐无复有形,精气散亡,何能复有体,而人得见之乎!禽兽之死也,其肉尽索,皮毛尚在,制以为裘,人望见之,似禽兽之形。故世有衣狗裘为狗盗者,人不觉知,假狗之皮毛,故人不意疑也。今人死,皮毛朽败,虽精气尚在,神安能复假此形而以行见乎?夫死人不能假生人之形以见,犹生人不能假死人之魂以亡矣。六畜能变化象人之形者,其形尚生,精气尚在也。如死,其形腐朽,虽虎兕勇悍,不能复化。鲁公牛哀病化为虎,亦以未死也。世有以生形转为生类者矣,未有以死身化为生象者也。

天地开辟,人皇以来,随寿而死。若中年夭亡,以亿万数。计今人之数不若死者多,如人死辄为鬼,则道路之上,一步一鬼也。人且死见鬼,宜见数百千万,满堂盈廷,填塞巷路,不宜徒见一两人也。人之兵死也,世言其血为磷。血者,生时之精气也。人夜行见磷,不象人形,浑沌积聚,若火光之状。磷,死人之血也,其形不类生人之血也,其形不类生人之形。精气去人,何故象人之体?人见鬼也,皆象死人形,则可疑死人为鬼,或反象生人之形。病者见鬼,云甲来。甲时不死,气象甲形。如死人为鬼,病者何故见生人之体乎?

天地之性,能更生火,不能使灭火复燃;能更生人,不能令死人复见。能使灰更为燃火,吾乃颇疑死人能复为形。案火灭不能复燃以况之,死人不能复为鬼,明矣。夫为鬼者,人谓死人之精神。如审鬼者死人之精神,则人见之宜徒见裸袒之形,无为见衣带被服也。何则?衣服无精神,人死,与形体俱朽,何以得贯穿之乎?精神本以血气为主,血气常附形体。形体虽朽,精神尚在,能为鬼可也。今衣服,丝絮布帛也,生时血气不附着,而亦自无血气,败朽遂已,与形体等,安能自若为衣服之形?由此言之,见鬼衣服象〔人〕,则形体亦象〔人〕矣。象〔人〕,则知非死人之精神也。

夫死人不能为鬼,则亦无所知矣。何以验之?以未生之时无所知也。人未生,在元气之中;既死,复归元气。元气荒忽,人气在其中。人未生无所知,其死归无知之本,何能有知乎?人之所以聪明智惠者,以含五常之气也;五常之气所以在人者,以五藏在形中也。五藏不伤,则人智惠;五藏有病,则人荒忽。荒忽则愚痴矣。人死,五藏腐朽,腐朽则五常无所托矣,所用藏智者已败矣,所用为智者已去矣。形须气而成,气须形而知。天下无独燃之火,世间安得有无体独知之精?

人之死也,其犹梦也。梦者,殄之次也;殄者,死之比也。人殄不悟则死矣。案人殄复悟,死〔复〕来者,与梦相似,然则梦、殄、死,一实也。人梦不能知觉时所作,犹死不能识生时所为矣。人言谈有所作於卧人之旁,卧人不能知,犹对死人之棺,为善恶之事,死人不能复知也。夫卧,精气尚在,形体尚全,犹无所知,况死人精神消亡,形体朽败乎?

人为人所殴伤,诣吏告苦以语人,有知之故也。或为人所杀,则不知何人杀也,或家不知其尸所在。使死人有知,必恚人之杀己也,当能言於吏旁,告以贼主名;若能归语其家,告以尸之所在。今则不能,无知之效也。世间死者,〔令〕生人殄,而用其言,用巫叩元弦下死人魂,因巫口谈,皆夸诞之言也。知不夸诞,物之精神为之象也。或曰:不能言也。夫不能言,则亦不能知矣。知用气,言亦用气焉。人之未死也,智惠精神定矣,病则惛乱,精神扰也。夫死,病之甚者也。病,死之微,犹惛乱,况其甚乎!精神扰,自无所知,况其散也!

人之死,犹火之灭也。火灭而耀不照,人死而知不惠,二者宜同一实。论者犹谓死有知,惑也。人病且死,与火之且灭何以异?火灭光消而烛在,人死精亡而形存,谓人死有知,是谓火灭复有光也。隆冬之月,寒气用事,水凝为冰,逾春气温,冰释为水。人生於天地之间,其犹冰也。阴阳之气,凝而为人,年终寿尽,死还为气。夫春水不能复为冰,死魂安能复为形?

妒夫娼妻,同室而处,淫乱失行,忿怒斗讼,夫死,妻更嫁,妻死,夫更娶。以有知验之,宜大忿怒。今夫妻死者,寂寞无声,更嫁娶者,平忽无祸,无知之验也。

孔子葬母於防,既而雨甚至,防墓崩。孔子闻之,泫然流涕曰:``古者不修墓。''遂不复修。使死有知,必恚人不修也。孔子知之,宜辄修墓,以喜魂神。然而不修,圣人明审,晓其无知也。

枯骨在野,时鸣呼有声,若夜闻哭声,谓之死人音,非也。何以验之?生人所以言语吁呼者,气括口喉之中,动摇其舌,张歙其口,故能成言。譬犹吹箫笙,箫笙折破,气越不括,手无所弄,则不成音。夫箫笙之管,犹人之口喉也;手弄其孔,犹人之动舌也。人死口喉腐败,舌不复动,何能成言?然而枯骨时呻鸣者,人骨自有能呻鸣者焉,或以为秋〔气〕也,是与夜鬼哭无以异也。秋气为呻鸣之变,自有所为,依倚死骨之侧,人则谓之骨尚有知,呻鸣於野。草泽暴体以千万数,呻鸣之声,宜步属焉。

夫有能使不言者言,未有言者死能复使之言,言者亦不能复使之言。犹物生以青为〔色〕,或予之也,物死青者去,或夺之也。予之物青,夺之青去,去後不能复予之青,物亦不能复自青。声色俱通,并禀於天。青青之色,犹枭枭之声也,死物之色不能复青,独为死人之声能复自言,惑也。

人之所以能言语者,以有气力也,气力之盛,以能饮食也。饮食损减则气力衰,衰则声音嘶,困不能食,则口不能复言。夫死,困之甚,何能复言?或曰:
``死人歆肴食气,故能言。''夫死人之精,生人之精也。使生人不饮食,而徒以口歆肴食〔之〕气,不过三日则饿死矣。或曰:``死人之精,神於生人之精,故能歆气为音。''夫生人之精在於身中,死则在於身外,死之与生何以殊?身中身外何以异?取水实於大盎中,盎破水流地,地水能异於盎中之水乎?地水不异於盎中之水,身外之精,何故殊於身中之精?

人死不为鬼,无知,不能语言,则不能害人矣。何以验之?夫人之怒也用气,其害人用力,用力须筋骨而强,强则能害人。忿怒之人,呴呼於人之旁,口气喘射人之面,虽勇如贲、育,气不害人,使舒手而击,举足而蹶,则所击蹶无不破折。夫死,骨朽筋力绝,手足不举,虽精气尚在,犹呴吁之时无嗣助也,何以能害人也?凡人与物所以能害人者,手臂把刃,爪牙坚利之故也。今人死,手臂朽败,不能复持刃,爪牙堕落,不能复啮噬,安能害人?兒之始生也,手足具成,手不能搏,足不能蹶者,气适凝成,未能坚强也。由此言之,精气不能坚强,审矣。气为形体,形体微弱,犹未能害人,况死,气去精神绝。安能害人?寒骨谓能害人者邪?死人之气不去邪?何能害人?

鸡卵之未字也,澒溶於彀中,溃而视之,若水之形;良雌伛伏,体方就成,就成之後,能啄蹶之。夫人之死,犹澒溶之时,澒溶之气,安能害人?人之所以勇猛能害人者,以饮食也,饮食饱足则强壮勇猛,强壮勇猛则能害人矣。人病不能饮食,则身〔羸〕弱,〔羸〕弱困甚,故至於死。病困之时,仇在其旁,不能咄叱,人盗其物,不能禁夺,羸弱困劣之故也。夫死,羸弱困劣之甚者也,何能害人?有鸡犬之畜,为人所盗窃,虽怯无势之人,莫不忿怒,忿怒之极,至相贼灭。败乱之时,人相啖食者,使其神有知,宜能害人。身贵於鸡犬,己死重於见盗,忿怒於鸡犬,无怨於食己,不能害人之验也。蝉之未蜕也,为复育,已蜕也去复育之体,更为蝉之形。使死人精神去形体,若蝉之去复育乎!则夫为蝉者不能害为复育者。夫蝉不能害复育,死人之精神,何能害生人之身?梦者之义疑。〔或〕言:``梦者,精神自止身中,为吉凶之象。''或言:``精神行与人物相更。
''今其审止身中,死之精神,亦将复然。今其审行,人梦杀伤人,梦杀伤人,若为人所复杀,明日视彼之身,察己之体,无兵刃创伤之验。夫梦用精神,精神,死之精神也。梦之精神不能害人,死之精神安能为害?火炽而釜拂,沸止而气歇,以火为主也。精神之怒也,乃能害人;不怒,不能害人。火猛灶中,釜涌气蒸;精怒胸中,力盛身热。今人之将死,身体清凉,凉益清甚,遂以死亡。当死之时,精神不怒。身亡之後,犹汤之离釜也,安能害人?

物与人通,人有痴狂之病。如知其物然而理之,病则愈矣。夫物未死,精神依倚形体,故能变化,与人交通;已死,形体坏烂,精神散亡,无所复依,不能变化。夫人之精神,犹物之精神也。物生,精神为病;其死,精神消亡。人与物同,死而精神亦灭,安能为害祸!设谓人贵,精神有异,成事,物能变化,人则不能,是反人精神不若物,物精〔神〕奇於人也。

水火烧溺。凡能害人者,皆五行之物。金伤人,木殴人,土压人,水溺人,火烧人。使人死,精神为五行之物乎,害人;不为乎,不能害人。不为物,则为气矣。气之害人者,太阳之气为毒者也。使人死,其气为毒乎,害人;不为乎,不能害人。

夫论死不为鬼,无知,不能害人,则夫所见鬼者,非死人之精,其害人者,非其精所为,明矣。

\hypertarget{header-n803}{%
\subsection{卷二十一}\label{header-n803}}

\hypertarget{header-n804}{%
\subsubsection{死伪篇}\label{header-n804}}

传曰:``周宣王杀其臣杜伯而不辜,宣王将田於囿,杜伯起於道左,执彤弓而射宣王,宣王伏而死。赵简公杀其臣庄子义而不辜,简公将入於桓门,庄子义起於道左,执彤杖而捶之,毙於车下。''二者,死人为鬼之验;鬼之有知,能害人之效也。无之,奈何?曰:人生万物之中,物死不能为鬼,人死何故独能为鬼?如以人贵能为鬼,则死者皆当为鬼。杜伯、庄子义何独为鬼也?如以被非辜者能为鬼,世间臣子被非辜者多矣,比干、子胥之辈不为鬼。夫杜伯、庄子义无道忿恨,报杀其君。罪莫大於弑君,则夫死为鬼之尊者当复诛之,非杜伯、庄子义所敢为也。凡人相伤,憎其生,恶见其身,故杀而亡之。见杀之家,诣吏讼其仇,仇人亦恶见之。生死异路,人鬼殊处。如杜伯、庄子义怨宣王、简公,不宜杀也,当复为鬼,与己合会。人君之威,固严人臣,营卫卒使固多众,两臣杀二君,二君之死,亦当报之,非有知之深计,憎恶之所为也。如两臣神,宜知二君死当报己;如不知也,则亦不神。不神胡能害人?世多似是而非,虚伪类真,故杜伯、庄子义之语,往往而存。

晋惠公改葬太子申生。秋,其仆狐突适下国,遇太子。太子趋登仆车而告之曰:``夷吾无礼,余得请於帝矣,将以晋畀秦,秦将祀余。''狐突对曰:``臣闻之,神不歆非类,民不祀非族,君祀无乃殄乎!且民何罪,失刑乏祀,君其图之!''太子曰:``诺,吾将复请。七日,新城西偏,将有巫者,而见我焉。''许之,遂不见。及期,狐突之新城西偏巫者之舍,复与申生相见。申生告之曰:``帝许罚有罪矣,毙之於韩。''其後四年,惠公与秦穆公战於韩地,为穆公所获,竟如其言。非神而何?曰:此亦杜伯、庄子义之类。何以明之?夫改葬,私怨也;上帝,公神也。以私怨争於公神,何肯听之?帝许以晋畀秦,狐突以为不可,申生从狐突之言,是则上帝许申生非也。神为上帝,不若狐突,必非上帝,明矣。且臣不敢求私於君者,君尊臣卑,不敢以非干也。申生比於上帝,岂徒臣之与君哉!恨惠公之改葬,干上帝之尊命,非所得为也。骊姬谮杀其身,惠公改葬其尸。改葬之恶,微於杀人;惠公之罪,轻於骊姬。请罚惠公,不请杀骊姬,是则申生憎改葬,不怨见杀也。秦始皇用李斯之议,燔烧诗书,後又坑儒。博士之怨,不下申生;坑儒之恶,痛於改葬。然则秦之死儒,不请於帝,见形为鬼,〔诸生〕会告以始皇无道,李斯无状。

周武王有疾不豫,周公请命,设三坛同一墠,植璧秉圭,乃告於太王、王季、文王。史乃策祝,辞曰:``予仁若考,多才多艺,能事鬼神。乃元孙某,不若旦多才多艺,不能事鬼神。''鬼神者,谓三王也。即死人无知,不能为鬼神。周公,圣人也,圣人之言审,则得幽冥之实;得幽冥之实,则三王为鬼神,明矣。曰:实〔圣〕人能神乎?不能神也?如神,宜知三王之心,不宜徒审其为鬼也。周公请命,史策告祝,祝毕辞已,不知三王所以与不,乃卜三龟,三龟皆吉,然後乃喜。能知三王有知为鬼,不能知三王许己与不,须卜三龟,乃知其实。定其为鬼,须有所问,然後知之。死人有知无知,与其许人不许人,一实也。能知三王之必许己,则其谓三王为鬼,可信也;如不能知,谓三王为鬼,犹世俗人也;与世俗同知,则死人之实未可定也。且周公之请命,用何得之,以至诚得之乎?以辞正得之也?如以至诚,则其请〔命〕之说,精诚致鬼,不顾辞之是非也。董仲舒请雨之法,设土龙以感气。夫土龙非实,不能致雨,仲舒用之致精诚,不顾物之伪真也。然则周公之请命,犹仲舒之请雨也;三王之非鬼,犹聚土之非龙也。

晋荀偃伐齐,不卒事,而还。瘅疽生,疡於头,及著雍之地,病,目出,卒而视,不可唅。范宣子浣而抚之,曰:``事吴敢不如事主。''犹视。宣子睹其不瞑,以为恨其子吴也。人情所恨,莫不恨子,故言吴以抚之,犹视者,不得所恨也。栾怀子曰:``其为未卒事於齐故也乎?''乃复抚之,曰:``主苟死,所不嗣事於齐者,有如河。''乃瞑受唅。伐齐不卒,苟偃所恨也,怀子得之,故目瞑受含,宣子失之,目张口噤。曰:荀偃之病卒,苦目出。目出则口噤,口噤则不可含。新死气盛,本病苦目出,宣子抚之早,故目不瞑,口不阆。少久气衰,怀子抚之,故目瞑口受唅。此自荀偃之病,非死精神见恨於口目也。凡人之死,皆有所恨。志士则恨义事未立,学士则恨问多不及,农夫则恨耕未畜谷,商人则恨货财未殖,仕者则恨官位未极,勇者则恨材未优。天下各有所欲乎,然而各有所恨,必〔以〕目不瞑者为有所恨,夫天下之人,死皆不瞑也。且死者精魂消索,不复闻人之言。不能闻人之言,是谓死也。离形更自为鬼,立於人傍,虽〔闻〕人之言,已与形绝,安能复入身中,瞑目阆口乎?能入身中以尸示恨,则能不〔死〕,与形相守。案世人论死,谓其精神有若,能更以精魂立形见面,使尸若生人者,误矣。楚成王废太子商臣,欲立王子职。商臣闻之,以宫甲围王。王请食熊蹯而死,弗听。王缢而死。谥之曰``灵'',不瞑;曰``成'',乃瞑。夫为``灵''不瞑;为``成''乃瞑,成王有知之效也。谥之曰``灵'',心恨,故目不瞑;更谥曰``成'',心喜乃瞑。精神闻人之议,见人变易其谥,故喜目瞑。本不病目,人不抚慰,目自翕张,非神而何?曰:此复荀偃类也。虽不病目,亦不空张。成王於时缢死,气尚盛,新绝,目尚开,因谥曰``灵''。少久气衰,目适欲瞑,连更曰``成''。目之视瞑,与谥之为``灵'',偶应也。时人见其应``成''乃瞑,则谓成王之魂有所知。〔有所知,〕则宜终不瞑也。何则?太子杀己,大恶也;加谥为``灵'',小过也。不为大恶怀忿,反为小过有恨,非有神之效,见示告人之验也。夫恶谥非``灵''则``厉''也,纪於竹帛,为``灵''、``厉''者多矣,其尸未敛之时,未皆不暝也。岂世之死君不恶,而独成王憎之哉?何其为``灵''者众,不瞑者寡也?

郑伯有贪愎而多欲,子皙好在人上,二子不相得。子皙攻伯有,伯有出奔,驷带率国人以伐之,伯有死。其後九年,郑人相惊以伯有,曰:``伯有至矣。''则皆走,不知所往。後岁,人或梦见伯有介而行,曰:``壬子,余将杀带也。明年壬寅,余又将杀段也。''及壬子之日,驷带卒,国人益惧。後至壬寅日,公孙段又卒,国人愈惧。子产为之立後以抚之,乃止矣。其後子产适晋,赵景子问曰:``伯有犹能为鬼乎?''子产曰:``能。人生始化曰魄,既生魄,阳曰魂。用物精多,则魂魄强,是以有精爽至於神明。匹夫匹妇强死,其魂魄犹能凭依人以为淫厉。况伯有,我先君穆公之胄,子良子孙,子耳之子,弊邑之卿,从政三世矣。郑虽无腆,抑谚曰:``蕞尔小国,而三世执其政柄,其用物弘矣,取精多矣。其族又大,所凭厚矣。而强死,能为鬼,不亦宜乎!''伯有杀驷带、公孙段不失日期,神审之验也。子产立其後而止,知鬼神之操也。知其操,则知其实矣。实有不空,故对问不疑。子产,智人也,知物审矣。如死者无知,何以能杀带与段?如不能为鬼,子产何以不疑?曰:与伯有为怨者,子皙也。子皙攻之,伯有奔,驷带乃率国人遂伐伯有。公孙段随驷带,不造本〔仇〕,其恶微小。杀驷带不报子皙,公孙段恶微,与带俱死。是则伯有之魂无知,为鬼报仇,轻重失宜也。且子产言曰:``强死者能为鬼。''何谓强死?谓伯有命未当死而人杀之邪?将谓伯有无罪而人冤之也?如谓命未当死而人杀之,未当死而死者多。如谓无罪人冤之,被冤者亦非一。伯有强死能为鬼,比干、子胥不为鬼。春秋之时,弑君三十六。君为所弑,可谓强死矣。典长一国,用物之精可谓多矣。继体有土,非直三世也。贵为人君,非与卿位同也。始封之祖,必有穆公、子良之类也。以至尊之国君,受乱臣之弑祸,其魂魄为鬼,必明於伯有,报仇杀仇,祸繁於带、段。三十六君无为鬼者,三十六臣无见报者。如以伯有无道,其神有知,世间无道莫如桀、纣,桀、纣诛死,魄不能为鬼。然则子产之说,因成事者也。见伯有强死,则谓强死之人能为鬼。如有不强死为鬼者,则将云不强死之人能为鬼。子皙在郑,与伯有何异?死与伯有何殊?俱以无道为国所杀。伯有能为鬼,子皙不能。强死之说,通於伯有,塞於子皙。然则伯有之说,杜伯之语也。杜伯未可然,伯有亦未可是也。

秦桓公伐晋,次於辅氏。晋侯治兵於稷,以略翟土,立黎侯而还。及魏颗败秦师於辅氏,获杜回。杜回,秦之力人也。初,魏武子有嬖妾无子。武子疾,命颗曰:``必嫁是妾。''病困,则更曰:``必以是为殉。''及武子卒,颗不殉妾。人或难之,颗曰:``疾病则乱,吾从其治也。''及辅氏之役,魏颗见老人结草以亢杜回,杜回踬而颠,故获之;夜梦见老父曰:``余是所嫁妇人之父也。尔用先人之治命,是以报汝。''夫嬖妾之父知魏颗之德,故见体为鬼,结草助战,神晓有知之效验也。曰:夫妇人之父能知魏颗之德,为鬼见形以助其战,必能报其生时所善,杀其生时所恶矣。凡人交游必有厚薄,厚薄当报,犹〔嫁〕妇人之当谢也。今不能报其生时所厚,独能报其死後所善,非有知之验,能为鬼之效也。张良行泗水上,老父授书。光武困厄河北,老人教诲。命贵时吉,当遇福喜之应验也。魏颗当获杜回,战当有功,故老人妖象结草於路者也。

王季葬於滑山之尾,栾水击其墓,见棺之前和。文王曰:``嘻!先君必欲一见群臣百姓也夫!故使栾水见之於是也。''于是也而为之张朝,而百姓皆见之三日而後更葬。文王,圣人也,知道事之实。见王季棺见,知其精神欲见百姓,故出而见之。曰:古今帝王死,葬诸地中,有以千万数,无欲复出见百姓者,王季何为独然?河、泗之滨,立〔冢〕非一,水湍崩壤,棺椁露见,不可胜数,皆欲复见百姓者乎?栾水击滑山之尾,犹河、泗之流湍滨圻也。文王见棺和露,恻然悲恨,当先君欲复出乎,慈孝者之心,幸冀之意,贤圣恻怛,不暇思论。推生况死,故复改葬。世欲信贤圣之言,则谓王季欲见姓者也。

齐景公将伐宋,师过太山,公梦二丈人立而怒甚盛。公告晏子,晏子曰:``是宋之先,汤与伊尹也。''公疑以为泰山神。晏子曰:``公疑之,则婴请言汤、伊尹之状。汤晰,以长颐以髯,锐上而丰下,〔倨〕身而扬声。''公曰:``然,是已。''``伊尹黑而短,蓬而髯,丰上而锐下,偻身而下声。''公曰:``然,是已。今奈何?''晏子曰:``夫汤、太甲、武丁、祖己,天下之盛君也,不宜无後。今唯宋耳,而公伐之,故汤、伊尹怒。请散师和於宋。''公不用,终伐宋,军果败。夫汤、伊尹有知,恶景公之伐宋,故见梦盛怒以禁止之。景公不止,军果不吉。曰:夫景公亦曾梦见彗星,其时彗星不出,然而梦见之者,见彗星其实非。梦见汤、伊尹,实亦非也。或时景公军败不吉之象也。晏子信梦,明言汤、伊尹之形,景公顺晏子之言,然而是之。秦并天下,绝伊尹之後,遂至於今,汤、伊尹不祀,何以不怒乎?

郑子产聘於晋。晋侯有疾,韩宣子逆客,私焉,曰:``寡君寝疾,於今三月矣,并走群望,有加而无瘳。今梦黄熊入於寝门,其何厉鬼也?''对曰:``以君之明,子为大政,其何厉之有!昔尧殛鲧於羽山,其神为黄熊,以入於羽渊,实为夏郊,三代祀之。晋为盟主,其或者未之祀乎!''韩子祀夏郊,晋侯有间。黄熊,鲧之精神,晋侯不祀,故入寝门。晋知而祀之,故疾有间。非死人有知之验乎?夫鲧殛於羽山,人知也。神为黄熊,入於羽渊,人何以得知之?使若鲁公牛哀病化为虎,在,故可实也。今鲧远殛於羽山,人不与之处,何能知之?且文曰:``其神为熊。''是死也。死而魂神为黄熊,非人所得知也。人死世谓鬼,鬼象生人之形,见之与人无异,然犹非死人之神,况熊非人之形,不与人相似乎?审鲧死,其神为黄熊。''则熊之死,其神亦或时为人,人梦见之,何以知非死禽兽之神也?信黄熊谓之鲧神,又信所见之鬼以为死人精也,此人物之精未可定,黄熊为鲧之神未可审也。且梦,象也,吉凶且至,神明示象,熊罴之占,自有所为。使鲧死,其神审为黄熊,梦见黄熊,必鲧之神乎?诸侯祭山川,设晋侯梦见山川,〔可〕复〔不〕以祀山川,山川自见乎?人病,多或梦见先祖死人来立其侧,可复谓先祖死人求食,故来见形乎?人梦所见,更为他占,未必以所见为实也。何以验之?梦见生人,明日所梦见之人,不与己相见。夫所梦见之人不与己相见,则知鲧之黄熊不入寝门;不入,则鲧不求食;不求食,则晋侯之疾非废夏郊之祸;非废夏郊之祸,则晋侯有间,非祀夏郊之福也。无福之实,则无有知之验矣。亦犹淮南王刘安坐谋反而死,世传以为仙而升天。本传之虚,子产闻之,亦不能实。偶晋侯之疾适当自衰,子产遭言黄熊之占,则信黄熊鲧之神矣。

高皇帝以赵王如意为似我而欲立之,吕后恚恨,后鸩杀赵王。其後,吕后出,见苍犬,噬其左腋,怪而卜之,赵王如意为祟,遂病腋伤,不愈而死。盖以如意精神为苍犬,见变以报其仇也。愤曰:勇士忿怒,交刃而战,负者被创,仆地而死。目见彼之中己,死後其神尚不能报,吕后鸩如意时,身不自往,使人饮之,不知其为鸩毒,愤不知杀己者为谁,安能为祟以报吕后?使死人有知,恨者莫过高祖。高祖爱如意而吕后杀之,高祖魂怒,宜如雷霆,吕后之死,宜不旋日。岂高祖之精,不若如意之神,将死後憎如意,善吕后之杀也?

丞相武安侯田与故大将军灌夫怀酒之恨,事至上闻。灌夫系狱,窦婴救之,势不能免灌夫坐法,窦婴亦死。其後,田蚡病甚,号曰``诺诺'',使人视之,见灌夫、窦婴惧坐其侧,蚡病不衰,遂至死。曰:相杀不一人也,杀者後病,不见所杀,田蚡见所杀。田蚡独然者,心负愤恨,病乱妄见也。或时见他鬼,而占鬼之人,闻其往时与夫、婴争,欲见神审之名,见其狂``诺诺'',则言夫、婴坐其侧矣。

淮阳都尉尹齐,为吏酷虐,及死,怨家欲烧其尸,〔尸〕亡去归葬。夫有知,故人且烧之也;神,故能亡去。曰:尹齐亡,神也,有所应。秦时三山亡,周末九鼎沦,必以亡者为神,三山、九鼎有知也?或时吏知怨家之谋,窃举持亡,惧怨家怨己,云自去。凡人能亡,足能步行也。今死,血脉断绝,足不能复动,何用亡去?吴烹伍子胥,汉菹彭越。烧、菹,一僇也;胥、越,一勇也。子胥、彭越不能避烹亡菹,独谓尹齐能归葬,失实之言,不验之语也。

亡新改葬元帝傅后,发其棺,取玉柙印玺,送定陶,以民礼葬之。发棺时,臭憧於天,洛阳丞临棺,闻臭而死。又改葬定陶共王丁後,火从藏中出,烧杀吏士数百人。夫改葬礼卑,又损夺珍物,二恨怨,故为臭、出火,以中伤人。曰:臭闻於天,多藏食物,腐朽猥发,人不能堪毒愤,而未为怪也。火出於藏中者,怪也,非丁后之神也。何以验之?改葬之恨,孰与掘墓盗财物也?岁凶之时,掘丘墓取衣物者以千万物数,死人〔亡〕有知,人夺其衣物,倮其尸骸,时不能禁,後亦不能报。此尚微贱,未足以言。秦始皇葬於骊山,二世末,天下盗贼掘其墓,不能出臭、为火,以杀一人。贵为天子不能为神,丁、傅妇人,安能为怪?变神非一,发起殊处,见火闻臭,则谓丁、傅之神,误矣。

\hypertarget{header-n809}{%
\subsection{卷二十二}\label{header-n809}}

\hypertarget{header-n810}{%
\subsubsection{纪妖篇}\label{header-n810}}

卫灵公将之晋,至濮水之上,夜闻鼓新声者,说之,使人问之,左右皆报弗闻。召师涓而告之曰:``有鼓新声者,使人问左右,尽报弗闻其状似鬼,子为我听而写之。''师涓曰:``诺!''因静坐抚琴而写之。明日报曰:``臣得之矣,然而未习,请更宿而习之。''灵公曰:``诺!''因复宿。明日已习,遂去之晋。晋平公觞之施夷之台,酒酣,灵公起曰:``有新声,愿请奏以示公。''公曰:``善!''乃召师涓,令坐师旷之旁,援琴鼓之。未终,旷抚而止之,曰:``此亡国之声,不可遂也。''平公曰:``此何道出?''师旷曰:``此师延所作淫声,与纣为靡靡之乐也。武王诛纣,悬之白旄,师延东走,至濮水而自投,故闻此声者,必於濮水之上。先闻此声者,其国削,不可遂也。''平公曰:``寡人好者音也,子其使遂之。''师涓鼓究之。

平公曰:``此所谓何声也?''师旷曰:``此所谓清商。''公曰:``清商固最悲乎?''师旷曰:``不如清徵。''公曰:``清徵可得闻乎?''师旷曰:``不可!古之得听清徵者,皆有德义君也。今吾君德薄,不足以听之。''公曰:``寡人所好者音也,愿试听之。''师旷不得已,援琴鼓之。一奏,有玄鹤二八从南方来,集於郭门之上危;再奏而列;三奏,延颈而鸣,舒翼而舞。音中宫商之声,声彻於天。平公大悦,坐者皆喜。

平公提觞而起,为师旷寿,反坐而问曰:``乐莫悲於清徵乎?''师旷曰:``不如清角。''平公曰:``清角可得闻乎?''师旷曰:``不可!昔者黄帝合鬼神於西大山之上,驾象舆,六玄龙,毕方并辖,蚩尤居前,风伯进扫,雨师洒道,虎狼在前,鬼神在後,虫蛇伏地,白云覆上,大合鬼神,乃作为清角。今主君德薄,不足以听之。听之,将恐有败。''平公曰:``寡人老矣,所好者音也,愿遂听之。''师旷不得已而鼓之。一奏之,有云从西北起;再奏之,风至,大雨随之,裂帷幕,破俎豆,堕廊瓦,坐者散走。平公恐惧,伏於廊室。晋国大旱,赤地三年。平公之身遂癃病。何谓也?

曰:是非卫灵公国且削,则晋平公且病,若国且旱〔之〕妖也?师旷曰``先闻此声者国削''。二国先闻之矣。何知新声非师延所鼓也?曰:师延自投濮水,形体腐於水中,精气消於泥涂,安能复鼓琴?屈原自沉於江,屈原善著文,师延善鼓琴。如师延能鼓琴,则屈原能复书矣。杨子云吊屈原,屈原何不报?屈原生时,文无不作;不能报子云者,死为泥涂,手既朽,无用书也。屈原手朽无用书,则师延指败无用鼓琴矣。孔子当泗水而葬,泗水却流,世谓孔子神而能却泗水。孔子好教授,犹师延之好鼓琴也。师延能鼓琴於濮水之中,孔子何为不能教授於泗水之侧乎?

赵简子病,五日不知人。大夫皆俱,於是召进扁鹊。扁鹊入视病,出,董安於问扁鹊。扁鹊曰:``血脉治也,而〔何〕怪?昔秦缪公尝如此矣,七日悟。悟之日,告公孙支与子舆曰:`我之帝所,甚乐。吾所以久者,适有学也。帝告我晋国且大乱,五世不安,其〔後〕将霸,未老而死;霸者之子,且令而国男女无别。'公孙支书而藏之於箧。於是晋献公之乱,文公之霸,襄公败秦师於崤而归纵淫,此〔子〕之所〔闻〕。今主君之病与之同,不出三日,病必间,间必有言也。''

居二日半,简子悟,告大夫曰:``我之帝所,甚乐,与百神游於钧天,靡乐九奏万舞,不类三代之乐,其声动人心。有一熊欲〔援〕我,帝命我射之,中熊,熊死。有罴来,我又射之,中罴,罴死。帝甚喜,赐我〔二〕笥,皆有副。吾见兒在帝侧。帝属我一翟犬,曰:`及而子之长也,以赐之。'帝告我:`晋国且〔衰〕,〔七〕世而亡;嬴姓将大败周人於范魁之西,而亦不能有也。今余将思虞舜之勋,适余将以其胄女孟姚配而〔七〕世之孙。'''董安於受言而书藏之,以扁鹊言告简子,简子赐扁鹊田四万亩。

他日,简子出,有人当道,辟之不去。从者将拘之,当道者曰:``吾欲有谒於主君。''从者以闻,简子召之,曰:``嘻!吾有所见子游也。''当道者曰:``屏左右,愿有谒。''简子屏人。当道者曰:``日者主君之病,臣在帝侧。''简子曰:``然,有之。子见我何为?''当道者曰:``帝令主君射熊与罴皆死。''简子曰:``是何也?''当道者曰:``晋国且有大难,主君首之。帝令主君灭二卿,夫〔熊〕罴皆其祖也。''简子曰:``帝赐我二笥皆有副,何也?''当道者曰:``主君之子,将克二国於翟,皆子姓也。''简子曰:``吾见兒在帝侧,帝属我一翟犬,曰`及而子之长以赐之'夫兒何说以赐翟犬?''当道者曰:``兒,主君之子也。翟犬,代之先也。主君之子,且必有代。及主君之後嗣,且有革政而胡服,并二国〔於〕翟。''简子问其姓而延之以官。当道者曰:``臣野人,致帝命。''遂不见。是何谓也?曰:是皆妖也。其占皆如当道言,所见於帝前之事。所见当道之人,妖人也。其後晋二卿范氏、中行氏作乱,简子攻之,中行昭子、范文子败,出奔齐。

始,简子使姑布子卿相诸子,莫吉;至翟妇之子无恤,以为贵。简子与语,贤之。简子募诸子曰:``吾藏宝符於常山之上,先得者赏。''诸子皆上山,无所得。无恤还曰:``已得符矣。''简子问之,无恤曰:``从常山上临代,代可取也。''简子以为贤,乃废太子而立之。简子死,无恤代,是为襄子。襄子既立,诱杀代王而并其地。又并知氏之地。後取空同戎。自简子後,〔七〕世至武灵王,吴〔广〕入其〔女娃〕〔嬴〕孟姚。其後,武灵王遂取中山,并胡地。武灵王之十九年,更为胡服,国人化之。皆如其言,无不然者。盖妖祥见於兆,审矣,皆非实事。〔曰〕:吉凶之渐,若天告之。何以知天不实告之也?以当道之人在帝侧也。夫在天帝之侧,皆贵神也。致帝之命,是天使者也。人君之使,车骑备具,天帝之使,单身当道,非其状也。天官百二十,与地之王者以异也。地之王者,官属备具,法象天官,禀取制度。天地之官同,则其使者亦宜钧。官同人异者,未可然也。

何以知简子所见帝非实帝也?以梦占〔知〕之,楼台山陵,官位之象也。人梦上楼台,升山陵,辄得官位。实楼台山陵非官位也,则知简子所梦见帝者非天帝也。人臣梦出人君,人君必不见,又必不赐。以人臣梦占之,知帝赐二笥、翟犬者,非天帝也。非天帝,则其言与百鬼游於钧天,非天也。鲁叔孙穆子梦天压己者,审然是天下至地也。至地则有楼台之抗,不得及己,及己则楼台宜坏。楼台不坏,是天不至地。不至地则不得压己。不得压己则压己者非天也,则天之象也。叔孙穆子所梦压己之天非天,则知赵简子所游之天非天也。

或曰:``人亦有直梦。见甲,明日则见甲矣;梦见君,明日则见君矣。''曰:然。人有直梦,直梦皆象也,其象直耳。何以明之?直梦者梦见甲,梦见君,明日见甲与君,此直也。如问甲与君,甲与君则不见也。甲与君不见,所梦见甲与君者,象类之也。乃甲与君象类之,则知简子所见帝者象类帝也。且人之梦也,占者谓之魂行。梦见帝,是魂之上天也。上天犹上山也。梦上山,足登山,手引木,然後能升。升天无所缘,何能得上?天之去人以万里数。人之行,日百里。魂与体形俱,尚不能疾,况魂独行安能速乎?使魂行与形体等,则简子之上下天,宜数岁乃悟,七日辄觉,期何疾也!

夫魂者精气也,精气之行与云烟等。案云烟之行不能疾,使魂行若蜚鸟乎,行不能疾。人或梦蜚者用魂蜚也,其蜚不能疾於鸟。天地之气,尤疾速者,飘风也,飘风之发,不能终一日。使魂行若飘风乎,则其速不过一日之行,亦不能至天。人梦上天,一卧之顷也,其觉,或尚在天上,未终下也。若人梦行至雒阳,觉,因从雒阳悟矣。魂神蜚驰何疾也!疾则必非其状。必非其状,则其上天非实事也。非实事则为妖祥矣。夫当道之人,简子病,见於帝侧,後见当道象人而言,与相见帝侧之时无以异也。由此言之,卧梦为阴候,觉为阳占,审矣。

赵襄子既立。知伯益骄,请地韩、魏,韩、魏予之;请地於赵,赵不予。知伯益怒,遂率韩、魏攻赵襄子。襄子惧,用奔保晋阳。原过从,後,至於托平驿,见三人,自带以上可见,自带以下不可见,予原过竹二节,莫通,曰:``为我以是遗赵无恤。''既至,以告襄子。襄子齐三日,亲自割竹,有赤书曰:``赵无恤,余霍大山〔山〕阳侯,天〔使〕。三月丙戌,余将使汝灭知氏,汝亦祀我百邑,余将赐汝林胡之地。''襄子再拜,受神之命。是何谓也?

曰:是盖襄子且胜之祥也。三国攻晋阳岁余,引汾水灌其城,城不浸者三板。襄子惧,使相张孟谈私於韩、魏,韩、魏与合谋,竟以三月丙戌之日,〔反〕灭知氏,共分其地。盖妖祥之气。象人之形,称霍大山之神,犹夏庭之妖象龙,称褒之二君;赵简子之祥象人,称帝之使也。何以知非霍大山之神也?曰:大山,地之体,犹人有骨节,骨节安得神?如大山有神,宜象大山之形。何则?人谓鬼者死人之精,其象如生之形。今大山广长不与人同,而其精神不异於人。不异於人则鬼之类人。鬼之类人,则妖祥之气也。

秦始皇帝三十六年,荧惑守心,有星坠下,至地为石,〔民〕刻其石曰:``始皇死而地分。''始皇闻之,令御史逐问莫服,尽取石旁家人诛之,因燔其石。〔秋〕,使者从关东夜过华阴平〔舒〕,或有人持璧遮使者,曰:``为我遗镐池君。''因言曰:``今年祖龙死。''使者问之,因忽不见,置其璧去。使者奉璧,具以言闻,始皇帝默然良久,曰:``山鬼不过知一岁事,乃言曰`祖龙'者,人之先也。''使御府视璧,乃二十八年行渡江所沉璧也。明三十七年,梦与海神战,如人状。

是何谓也?曰:皆始皇且死之妖也。始皇梦与海神战,恚怒入海,候神射大鱼,自琅邪至劳、成山不见。至之罘山,还见巨鱼,射杀一鱼,遂旁海西至平原津而病,至沙丘而崩。当星坠之时,荧惑为妖,故石旁家人刻书其石,若或为之,文曰``始皇死'',或教之也。犹世间童谣,非童所为,气导之也。凡妖之发,或象人为鬼,或为人象鬼而使,其实一也。

晋公子重耳失国,乏食於道,从耕者乞饭。耕者奉塊土以赐公子。公子怒,咎犯曰:``此吉祥,天赐土地也。''其後公子得国复土,如咎犯之言。齐田单保即墨之城,欲诈燕军,云:``天神下助我。''有一人前曰:``我可以为神乎?''田单却走再拜事之,竟以神下之言闻於燕军。燕军信其有神,又见牛若五采之文,遂信畏惧,军破兵北。田单卒胜,复获侵地。此人象鬼之妖也。

使者过华阴,人持璧遮道,委璧而去,妖鬼象人之形也。夫沉璧於江,欲求福也。今还璧,示不受物,福不可得也。璧者象前所沉之璧,其实非也。何以明之?以鬼象人而见,非实人也。人见鬼象生存之人,定问生存之人,不与己相见,妖气象类人也。妖气象人之形,则其所赍持之物,非真物矣。``祖龙死'',谓始皇也。祖,人之本;龙,人君之象也。人、物类,则其言祸亦放矣。

汉高皇帝以秦始皇崩之岁,为泗上亭长,送徒至骊山。徒多道亡,因纵所将徒,遂行不还。被酒,夜经泽中,令一人居前,前者还报曰:``前有大蛇当道,愿还。''高祖醉,曰:``壮士行何畏!''乃前,拔剑击斩蛇,蛇遂分两,径开。行数里,醉因卧。高祖後人至蛇所,有一老妪夜哭之人曰:``妪何为哭?''妪曰:``人杀吾子。''人曰:``妪子为何见杀?''妪曰:``吾子白帝子,化为蛇当径。今者赤帝子斩之,故哭。''人以妪为妖言,因欲笞之。妪因忽不见。何谓也?曰:是高祖初起威胜之祥也。何以明之?以妪忽然不见也。不见,非人,非人则鬼妖矣。夫以妪非人,则知所斩之蛇非蛇也。云白帝子,何故为蛇夜而当道?谓蛇白帝子,高祖赤帝子;白帝子为蛇,赤帝子为人。五帝皆天之神也,子或为蛇,或为人。人与蛇异物,而其为帝同神,非天道也。且蛇为白帝子,则妪为白帝後乎?帝者之後,前後宜备,帝者之子,官属宜盛。今一蛇死於径,一妪哭於道。云白帝子,非实,明矣。夫非实则象,象则妖也,妖则所见之物皆非物也,非物则气也。高祖所杀之蛇非蛇也。则夫郑厉公将入郑之时,邑中之蛇与邑外之蛇斗者,非蛇也,厉公将入郑,妖气象蛇而斗也。郑国斗蛇非蛇,则知夏庭二龙为龙象,为龙象,则知郑子产之时龙战非龙也。天道难知,使非,妖也;使是,亦妖也。

留侯张良椎秦始皇,误中副车。始皇大怒,索求张良。张良变姓名,亡匿下邳,常闲从容步游下邳〔汜〕上,有一老父,衣褐至良所,直堕其履〔汜〕下,顾谓张良:``孺子下取履。''良愕然,欲殴之,以其老,为强忍下取履,因跪进履。父以足受履,笑去。良大惊。父去里所,复还,曰:``孺子可教矣。後五日平明,与我期此。''良怪之,因跪曰:``诺!''五日平明,良往。父已先在,怒曰:``与老人期,後,何也?去!後五日早会。''五日鸡鸣复往。父又已先在,复怒曰:``後,何也!去,後五日复早来。''五日,良夜未半往。有顷,父来,喜曰:``当如是矣。''出一篇书,曰:``读是则为帝者师。後十三年,子见我济北,谷成山下黄石即我也。''遂去,无他言,弗复见。旦日视其书,乃《太公兵法》也。良因异之,习读之。是何谓也?

曰:是高祖将起,张良为辅之祥也。良居下邳任侠,十年陈涉等起,沛公略地下邳,良从,遂为师将,封为留侯。後十三年,〔从〕高祖过济北界,得谷成山下黄石,取而葆祠之。及留侯死,并葬黄石。盖吉凶之象神矣,天地之化巧矣,使老父象黄石,黄石象老父,何其神邪?

问曰:``黄石审老父,老父审黄石耶?''曰:石不能为老父,老父不能为黄石。妖祥之气见,故验也。何以明之?晋平公之时,石言魏榆。平公问於师旷曰:``石何故言?''对曰:``石不能言,或凭依也。不然,民听偏也。''夫石不能人言,则亦不能人形矣。石言,与始皇时石坠〔东〕郡,民刻之,无异也。刻为文,言为辞。辞之与文,一实也。民刻文,气发言。民之与气,一性也。夫石不能自刻,则亦不能言。不能言,则亦不能为人矣。《太公兵法》,气象之也。何以知非实也?以老父非人,知书亦非太公之书也。气象生人之形,则亦能象太公之书。

问曰:气无刀笔,何以为文?曰:鲁惠公夫人仲子,生而有文在其掌,曰``为鲁夫人''。晋唐叔虞文在其手曰``虞''。鲁成季友文在其手曰``友''。三文之书,性自然;老父之书,气自成也。性自然,气自成,与夫童谣口自言,无以异也。当童之谣也,不知所受,口自言之。口自言,文自成,或为之也。推此以省太公钓得巨鱼,刳鱼得书,云``吕尚封齐'',及武王得白鱼,喉下文曰``以予发'',盖不虚矣。因此复原《河图》、《洛书》言光衰存亡、帝王际会,审有其文矣,皆妖祥之气,吉凶之瑞也。

\hypertarget{header-n814}{%
\subsubsection{订鬼篇}\label{header-n814}}

凡天地之间有鬼,非人死精神为之也,皆人思念存想之所致也。致之何由?由於疾病。人病则忧惧,忧惧见鬼出。凡人不病则不畏惧。故得病寝衽,畏惧鬼至;畏惧则存想,存想则目虚见。何以效之?传曰:``伯乐学相马,顾玩所见,无非马者。宋之庖丁学解牛,三年不见生牛,所见皆死牛也。''二者用精至矣。思念存想,自见异物也。人病见鬼,犹伯乐之见马,庖丁之见牛也。伯乐、庖丁所见非马与牛,则亦知夫病者所见非鬼也。病者困剧身体痛,则谓鬼持棰杖殴击之,若见鬼把椎锁绳纆立守其旁,病痛恐惧,妄见之也。初疾畏惊,见鬼之来;疾困恐死,见鬼之怒;身自疾痛,见鬼之击,皆存想虚致,未必有其实也。夫精念存想,或泄於目,或泄於口,或泄於耳。泄於目,目见其形;泄於耳,耳闻其声;泄於口,口言其事。昼日则鬼见,暮卧则梦闻。独卧空室之中,若有所畏惧,则梦见夫人据案其身哭矣。觉见卧闻,俱用精神;畏惧存想,同一实也。

一曰:人之见鬼,目光与卧乱也。人之昼也,气倦精尽,夜则欲卧,卧而目光反,反而精神见人物之象矣。人病亦气倦精尽,目虽不卧,光已乱於卧也,故亦见人物象。病者之见也,若卧若否,与梦相似。当其见也,其人能自知觉与梦,故其见物不能知其鬼与人,精尽气倦之效也。何以验之?以狂者见鬼也。狂痴独语,不与善人相得者,病困精乱也。夫病且死之时,亦与狂等。卧、病及狂,三者皆精衰倦,目光反照,故皆独见人物之象焉。

一曰:鬼者,人所见得病之气也。气不和者中人,中人为鬼,其气象人形而见。故病笃者气盛,气盛则象人而至,至则病者见其象矣。假令得病山林之中,其见鬼则见山林之精。人或病越地者,〔其见鬼〕〔则〕见越人坐其侧。由此言之,灌夫、窦婴之徒,或时气之形象也。凡天地之间气皆〔统〕於天,天文垂象於上,其气降而生物。气和者养生,不和者伤害。本有象於天,则其降下,有形於地矣。故鬼之见也,象气为之也。众〔气〕之体,为人与鸟兽,故其病人,则见人与鸟兽之形。

一曰:鬼者,老物精也。夫物之老者,其精为人;亦有未老,性能变化,象人形。人之受气,有与物同精者,则其物与之交;及病,精气衰劣也,则来犯陵之矣。何以效之?成事:俗间与物交者,见鬼之来也。夫病者所见之鬼,与彼病物何以异?人病见鬼来,象其墓中死人来迎呼之者,宅中之六畜也。及见他鬼,非是所素知者,他家若草野之中物为之也。

一曰:鬼者,本生於人,时不成人,变化而去。天地之性,本有此化,非道术之家所能论辩。与人相触犯者病,病人命当死,死者不离人。何以明之?《礼》曰:``颛顼氏有三子,生而亡去为疫鬼:一居江水,是为虐鬼;一居若水,是为魍魉鬼;一居人宫室区隅沤库,善惊人小兒。''前颛顼之世,生子必多,若颛顼之鬼神以百数也。诸鬼神有形体法,能立树与人相见者,皆生於善人,得善人之气,故能似类善人之形,能与善人相害。阴阳浮游之类,若云烟之气,不能为也。

一曰:鬼者,甲乙之神也。甲乙者,天之别气也,其形象人。人病且死,甲乙之神至矣。假令甲乙之日病,则死见庚辛之神矣。何则?甲乙鬼,庚辛报甲乙,故病人且死,杀鬼之至者,庚辛之神也。何以效之?以甲乙日病者,其死生之期,常在庚辛之日。此非论者所以为实也。天道难知,鬼神暗昧,故具载列,令世察之也。

一曰:鬼者,物也,与人无异。天地之间,有鬼之物,常在四边之外,时往来中国,与人杂〔厕〕,凶恶之类也,故人病且死者乃见之。天地生物也,有人如鸟兽。及其生凶物,亦有似人象鸟兽者。故凶祸之家,或见蜚尸,或见走凶,或见人形,三者皆鬼也。或谓之鬼,或谓之凶,或谓之魅,或谓之魑,皆生存实有,非虚无象类之也。何以明之?成事:俗间家人且凶,见流光集其室,或见其形若鸟之状,时流入堂室,察其不谓若鸟兽矣。夫物有形则能食,能食则便利。便利有验,则形体有实矣。《左氏春秋》曰:``投之四裔,以御魑魅。''《山海经》曰:``北方有鬼国。''说螭者谓之龙物也,而魅与龙相连,魅则龙之类矣。又言:国,人物之党也。《山海经》又曰:沧海之中,有度朔之山。上有大桃木,其屈蟠三千里,其枝间东北曰鬼门,万鬼所出入也。上有二神人,一曰神荼,一曰郁垒,主阅领万鬼。恶害之鬼,执以苇索,而以食虎。於是黄帝乃作礼以时驱之,立大桃人,门户画神荼、郁垒与虎,悬苇索以御凶魅。有形,故执以食虎。案可食之物,无空虚者。其物也性与人殊,时见时匿,与龙不常见,无以异也。

一曰:人且吉凶,妖祥先见。人之且死,见百怪,鬼在百怪之中。故妖怪之动,象人之形,或象人之声为应,故其妖动不离人形。天地之间,妖怪非一,言有妖,声有妖,文有妖,或妖气象人之形,或人含气为妖。〔妖气〕象人之形,诸所见鬼是也。人含气为妖,巫之类是也。是以实巫之辞,无所因据,其吉凶自从口出,若童之摇矣。童谣口自言,巫辞意自出。口自言,意自出,则其为人,与声气自立,音声自发,同一实也。

世称纣之时,夜郊鬼哭;及仓颉作书,鬼夜哭。气能象人声而哭,则亦能象人形而见,则人以为鬼矣。鬼之见也,人之妖也。天地之间,祸福之至,皆有兆象,有渐不卒然,有象不猥来。天地之道,入将亡,凶亦出;国将亡,妖亦见。犹人且吉,吉祥至;国且昌,昌瑞至矣。故夫瑞应妖祥,其实一也。而世独谓鬼者不在妖祥之中,谓鬼犹神而能害人,不通妖祥之道,不睹物气之变也。国将亡,妖见,其亡非妖也。人将死,鬼来,其死非鬼也。亡国者,兵也;杀人者,病也。何以明之?齐襄公将为贼所杀,游於姑棼,遂田於贝丘,见大豕。从者曰:``公子彭生也。''公怒曰:``彭生敢见!''引弓射之,豕人立而啼。公惧,坠於车,伤足丧履,而为贼杀之。夫杀襄公者,贼也。先见大豕於路,则襄公且死之妖也。人谓之彭生者,有似彭生之状也。世人皆知杀襄公者非豕,而独谓鬼能杀人,一惑也。

天地之气为妖者,太阳之气也。妖与毒同,气中伤人者谓之毒,气变化者谓之妖。世谓童谣,荧惑使之,彼言有所见也。荧惑火星,火有毒荧。故当荧惑守宿,国有祸败。火气恍惚,故妖象存亡。龙,阳物也,故时变化。鬼,阳气也,时藏时见。阳气赤,故世人尽见鬼,其色纯硃。蜚凶,阳也。阳,火也。故蜚凶之类为火光,火热焦物,故止集树木,枝叶枯死。《鸿范》五行二曰火,五事二曰言。言、火同气,故童谣、诗歌为妖言。言出文成,故世有文书之怪。世谓童子为阳,故妖言出於小童。童、巫含阳,故大雩之祭,舞童暴巫。雩祭之礼,倍阴合阳,故犹日食阴胜,攻社之阴也。日食阴胜,故攻阴之类。天旱阳胜,故愁阳之党。巫为阳党,故鲁僖遭旱,议欲焚巫。巫含阳气,以故阳地之民多为巫。巫党於鬼,故巫者为鬼巫。鬼巫比於童谣,故巫之审者,能处吉凶。吉凶能处,吉凶之徒也,故申生之妖见於巫。巫含阳,能见为妖也。申生为妖,则知杜伯、庄子义厉鬼之徒皆妖也。杜伯之厉为妖,则其弓、矢、投、措皆妖毒也。妖象人之形,其毒象人之兵。鬼、毒同色,故杜伯弓矢皆硃彤也。毒象人之兵,则其中人,人辄死也。中人微者即为腓,病者不即时死。何则?腓者,毒气所加也。妖或施其毒,不见其体;或见其形,不施其毒;或出其声,不成其言;或明其言,不知其音。若夫申生,见其体、成其言者也;杜伯之属,见其体、施其毒者也;诗妖、童谣、石言之属,明其言者也;濮水琴声、纣郊鬼哭,出其声者也。

妖之见出也,或且凶而豫见,或凶至而因出。因出,则妖与毒俱行。豫见,妖出不能毒。申生之见,豫见之妖也。杜伯、庄子义、厉鬼至,因出之妖也。周宣王、燕简公、宋夜姑时当死,故妖见毒因击。晋惠公身当获,命未死,故妖直见毒不射。然则杜伯、庄子义、厉鬼之见,周宣王、燕简、夜姑且死之妖也。申生之而出,晋惠公且见获之妖也。伯有之梦,驷带、公孙段且卒之妖也。老父结草,魏颗且胜之祥,亦或时杜回见获之妖也。苍犬噬吕后,吕后且死,妖象犬形也。武安且卒,妖象窦婴、灌夫之面也。故凡世间所谓妖祥、所谓鬼神者,皆太阳之气为之也。太阳之气,天气也。天能生人之体,故能象人之容。夫人所以生者,阴、阳气也。阴气主为骨肉,阳气主为精神。人之生也,阴、阳气具,故骨肉坚,精气盛。精气为知,骨肉为强,故精神言谈,形体固守。骨肉精神,合错相持,故能常见而不灭亡也。太阳之气,盛而无阴,故徒能为象,不能为形。无骨肉有精气,故一见恍惚,辄复灭亡也。

\hypertarget{header-n819}{%
\subsection{卷二十三}\label{header-n819}}

\hypertarget{header-n820}{%
\subsubsection{言毒篇}\label{header-n820}}

或问曰:``天地之间,万物之性,含血之虫,有蝮、蛇、蜂、虿,咸怀毒螫,犯中人身,〔谓〕获疾痛,当时不救,流遍一身;草木之中,有巴豆、野葛,食之凑懑,颇多杀人。不知此物,禀何气於天?万物之生,皆禀元气,元气之中,有毒螫乎?''

曰:夫毒,太阳之热气也,中人人毒。人食凑懑者,其不堪任也。不堪任,则谓之毒矣。太阳火气,常为毒螫,气热也。太阳之地,人民促急,促急之人,口舌为毒。故楚、越之人,促急捷疾,与人谈言,口唾射人,则人脣胎肿而为创。南郡极热之地,其人祝树树枯,唾鸟鸟坠。巫咸能以祝延人之疾、愈人之祸者,生於江南,含烈气也。夫毒,阳气也,故其中人,若火灼人。或为蝮所中,割肉置地焦沸,火气之验也。四方极皆为维边,唯东南隅有温烈气。温烈气发,常以春夏。春夏阳起。东南隅,阳位也。他物之气,入人鼻目,不能疾痛。火烟入鼻鼻疾,入目目痛,火气有烈也。物为靡屑者多,唯一火最烈,火气所燥也。食甘旨之食,无伤於人。食蜜少多,则令人毒。蜜为蜂液,蜂则阳物也。人行无所触犯,体无故痛,痛处若杖之迹。人腓,腓谓鬼殴之。鬼者,太阳之妖也。微者,疾谓之边,其治用蜜与丹。蜜丹阳物,以类治之也。夫治风用风,治热用热,治边用蜜丹。则知边者阳气所为,流毒所加也。天地之间,毒气流行,人当其冲,则面肿疾,世人谓之火流所刺也。

人见鬼者,言其色赤,太阳妖气,自如其色也。鬼为烈毒,犯人辄死,故杜伯射周宣立崩。鬼所赍物,阳火之类,杜伯弓矢,其色皆亦。南道名毒曰短狐。杜伯之象,执弓而射,阳气因而激,激而射,故其中人象弓矢之形。火困而气热,血毒盛,故食走马之肝杀人,气困为热也;盛夏暴行,暑暍而死,热极为毒也。人疾行汗出,对炉汗出,向日亦汗出,疾温病者亦汗出。四者异事而皆汗出,困同热等,火日之变也。天下万物,含太阳气而生者,皆有毒螫。毒螫渥者,在虫则为蝮蛇蜂虿,在草则为巴豆治葛,在鱼则为鲑与多、叔,故人食鲑肝而死,为多、叔螫有毒。鱼与鸟同类,故鸟蜚鱼亦蜚,鸟卵鱼亦卵,蝮蛇蜂虿皆卵,同性类也。

其在人也为小人,故小人之口,为祸天下。小人皆怀毒气,阳地小人,毒尤酷烈,故南越之人,祝誓辄效。谚曰:``众口烁金。''口者,火也。五行二曰火,五事二曰言。言与火直,故云烁金。道口舌之烁,不言拔木焰火,必云烁金,金制於火,火口同类也。

药生非一地,太伯〔采〕之吴。铸多非一工,世称楚棠溪。温气天下有,路畏入南海。鸩鸟生於南,人饮鸩死。辰为龙,巳为蛇,辰巳之位在东南。龙有毒,蛇有螫,故蝮有利牙,龙有逆鳞。木生火,火为毒,故苍龙之兽含火星。冶葛巴豆,皆有毒螫,故冶在东南,巴在西南。土地有燥湿,故毒物有多少。生出有处地,故毒有烈不烈。蝮蛇与鱼比,故生於草泽。蜂虿与鸟同,故产於屋树。江北地燥,故多蜂虿。江南地湿,故多蝮蛇。生高燥比阳,阳物悬垂,故蜂虿以尾刺。生下湿比阴,阴物柔伸,故蝮蛇以口齰。毒或藏於首尾,故螫有毒;或藏于体肤,故食之辄懑;或附於脣吻,故舌鼓为祸。

毒螫之生,皆同一气,发动虽异,内为一类。故人梦见火,占为口舌;梦见蝮蛇,亦口舌。火为口舌之象,口舌见於蝮蛇,同类共本,所禀一气也。故火为言,言为小人。小人为妖,由口舌。口舌之徵,由人感天,故五事二曰言。言之咎徵,``僭恆旸若''。僭者奢丽,故蝮蛇多文。文起於阳,故若致文。旸若则言从,故时有诗妖。

妖气生美好,故美好之人多邪恶。叔虎之母美,叔向之母知之,不使视寝。叔向谏其〔之〕,其母曰:``深山大泽,实生龙蛇。彼美,吾惧其生龙蛇以祸汝。汝弊族也,国多大宠,不仁之人间之,不亦难乎!余何爱焉!''使往视寝,生叔虎,美有勇力,嬖於栾怀子。及范宣子〔逐〕怀子,杀叔虎,祸及叔向。夫深山大泽,龙蛇所生也,比之叔虎之母者,美色之人怀毒螫也。生子叔虎,美有勇力,勇力所生,生於美色;祸难所发,由於勇力。火有光耀,木有容貌。龙蛇东方木,含火精,故美色貌丽。胆附於肝,故生勇力。火气猛,故多勇;木刚强,故多力也。生妖怪者,常由好色,为祸难者,常发勇力;为毒害者,皆在好色。

美酒为毒,酒难多饮;蜂液为蜜,蜜难益食。勇夫强国,勇夫难近。好女说心,好女难畜。辩士快意,辩士难信。故美味腐腹,好色惑心,勇夫招祸,辩口致殃。四者,世之毒也。辩口之毒,为害尤酷。何以明之?孔子见阳虎,却行,白汗交流。阳虎辩,有口舌。口舌之毒,中人病也。人中诸毒,一身死之;中于口舌,一国之贵乱。《诗》曰:``谗言罔极,交乱四国。''四国犹乱,况一人乎!故君子不畏虎,独畏谗夫之口。谗夫之口,为毒大矣。

\hypertarget{header-n824}{%
\subsubsection{薄葬篇}\label{header-n824}}

圣贤之业,皆以薄葬省用为务。然而世尚厚葬,有奢泰之失者,儒家论不明,墨家议之非故也。墨家之议右鬼,以为人死辄为神鬼而有知,能形而害人,故引杜伯之类以为效验。儒家不从,以为死人无知,不能为鬼,然而赙祭备物者,示不负死以观生也。陆贾依儒家而说,故其立语不肯明处。刘子政举薄葬之奏,务欲省用,不能极论。是以世俗内持狐疑之议,外闻杜伯之类,又见病且终者,墓中死人来与相见,故遂信是,谓死如生。闵死独葬,魂狐无副,丘墓闭藏,谷物乏匮,故作偶人以侍尸柩,多藏食物以歆精魂。积浸流至,或破家尽业,以充死棺;杀人以殉葬,以快生意。非知其内无益,而奢侈之心外相慕也。以为死人有知,与生人无以异。孔子非之而亦无以定实。然而陆贾之论两无所处。刘子政奏,亦不能明儒家无知之验,墨家有知之故。事莫明於有效,论莫定於有证。空言虚语,虽得道心,人犹不信。是以世俗轻愚信祸福者,畏死不惧义,重死不顾生,竭财以事神,空家以送终。辩士文人有效验,若墨家之以杜伯为据,则死无知之实可明,薄葬省财之教可立也。今墨家非儒,儒家非墨,各有所持,故乖不合,业难齐同,故二家争论。世无祭祀复生之人,故死生之义未有所定。实者死人暗昧,与人殊途,其实荒忽,难得深知。有知无知之情不可定,为鬼之实不可是。通人知士,虽博览古今,窥涉百家,条入叶贯,不能审知。唯圣心贤意,方比物类,为能实之。夫论不留精澄意,苟以外效立事是非,信闻见於外,不诠订於内,是用耳目论,不以心意议也。夫以耳目论,则以虚象为言;虚象效,则以实事为非。是故是非者不徒耳目,必开心意。墨议不以心而原物,苟信闻见,则虽效验章明,犹为失实。失实之议难以教,虽得愚民之欲,不合知者之心,丧物索用,无益於世。此盖墨术所以不传也。

鲁人将以玙敛,孔子闻之,径庭丽级而谏。夫径庭丽级,非礼也,孔子为救患也。患之所由,常由有所贪。

璠玙,宝物也,鲁人用敛,奸人间之,欲心生矣。奸人欲生,不畏罪法,不畏罪法,则丘墓抇矣。孔子睹微见著,故径庭丽级,以救患直谏。夫不明死人无知之义,而著丘墓必抇之谏,虽尽比干之执人,人必不听。何则?诸侯财多不忧贪,威强不惧抇。死人之议,狐疑未定,孝子之计,从其重者。如明死人无知,厚葬无益,论定议立,较著可闻,则璠之礼不行,径庭之谏不发矣。今不明其说而强其谏,此盖孔子所以不能立其教。孔子非不明死生之实,其意不分别者,亦陆贾之语指也。夫言死无知,则臣子倍其君父。故曰:''丧祭礼废,则臣子恩泊;臣子恩泊,则倍死亡先;倍死亡先,则不孝狱多。''圣人惧开不孝之源,故不明死无知之实。异道不相连,事生厚,化自生,虽事死泊,何损於化?使死者有知,倍之非也。如无所知,倍之何损?明其无知,未必有倍死之害。不明无知,成事已有贼生之费。

孝子之养亲病也,未死之时,求卜迎医,冀祸消、药有益也。既死之後,虽审如巫咸,良如扁鹊,终不复生。何则?知死气绝,终无补益。治死无益,厚葬何差乎!倍死恐伤化,绝卜拒医,独不伤义乎!亲之生也,坐之高堂之上,其死也,葬之黄泉之下。黄泉之下,非人所居,然而葬之不疑者,以死绝异处,不可同也。如当亦如生存,恐人倍之,宜葬於宅,与生同也。不明无知,为人倍其亲,独明葬黄泉,不为离其先乎?亲在狱中,罪疑未定,孝子驰走,以救其难。如罪定法立,终无门户,虽曾子、子骞,坐泣而已。何则?计动无益,空为烦也。今死亲之魂,定无所知,与拘亲之罪决不可救何以异?不明无知,恐人倍其先,独明罪定,不为忽其亲乎!圣人立义,有益於化,虽小弗除;无补於政,虽大弗与。今厚死人,何益於恩?倍之弗事,何损於义?

孔子又谓:为明器不成,示意有明,俑则偶人,象类生人。故鲁用偶人葬,孔子叹。睹用人殉之兆也,故叹以痛之。即如生当备物,不示如生,意悉其教,用偶人葬,恐後用生殉,用明器,独不为后用善器葬乎?绝用人之源,不防丧物之路,重人不爱用,痛人不忧国,传议之所失也。救漏防者,悉塞其穴,则水泄绝。穴不悉塞,水有所漏,漏则水为患害。论死不悉,则奢礼不绝,不绝则丧物索用。用索物丧,民贫耗〔乏〕,至,危亡之道也。

苏秦为燕使,使齐国之民高大丘冢,多藏财物,苏秦身弗以劝勉之,财尽民〔贫〕,国空兵弱,燕军卒至,无以自卫,国破城亡,主出民散。今不明死之无知,使民自竭以厚葬亲,与苏秦奸计同一败。墨家之议,自违其术,其薄葬而又右鬼,右鬼引效,以杜伯为验。杜伯死人,如谓杜伯为鬼,则夫死者审有知;如有知而薄葬之,是怒死人也。〔人〕情欲厚而恶薄,以薄受死者之责,虽右鬼,其何益哉?如以鬼非死人,则其信杜伯非也;如以鬼是死人,则其薄葬非也。术用乖错,首尾相违,故以为非。非与是不明,皆不可行。夫如是,世欲之人,可一详览。详览如斯,可一薄葬矣。

\hypertarget{header-n828}{%
\subsubsection{四讳篇}\label{header-n828}}

俗有大讳四:一曰讳西益宅。西益宅谓之不祥,不祥必有死亡。相惧以此,故世莫敢西益宅。防禁所从来者远矣。传曰:鲁哀公欲西益宅,史争以为不祥。哀公作色而怒,左右数谏而弗听,以问其傅宰质睢曰:``吾欲西益宅,史以为不祥,何如?''宰质睢曰:``天下有三不祥,西益宅不与焉。''哀公大说。有顷,复问曰:``何谓三不祥?''对曰:``不行礼义,一不祥也。嗜欲无止,二不祥也。不听规谏,三不祥也。''哀公缪然深惟,慨然自反,遂不益宅。令史与宰质睢止其益宅,徒为烦扰,则西益宅祥与不祥未可知也。令史、质睢以为西益宅审不祥,则史与质睢与今俗人等也。夫宅之四面皆地也,三面不谓之凶,益西面独谓不祥,何哉?西益宅,何伤於地体?何害於宅神?西益不祥,损之能善乎?西益不祥,东益能吉乎?夫不祥必有祥者,犹不吉必有吉矣。宅有形体,神有吉凶,动德致福,犯刑起祸。今言西益宅谓之不祥,何益而祥者?且恶人西益宅者谁也?如地恶之,益东家之西,损西家之东,何伤於地?如以宅神不欲西益,神犹人也,人这处宅,欲得广大,何故恶之?而以宅神恶烦扰,则四面益宅,皆当不祥。诸工技之家,说吉凶之占,皆有事状。宅家言治宅犯凶神,移徙言忌岁月,祭祀言触血忌,丧葬言犯刚柔,皆有鬼神凶恶之禁,人不忌避,有病死之祸。至於西益宅何害而谓之不祥?不祥之祸,何以为败?实说其义,``不祥''者义理之禁,非吉凶之忌也。夫西方,长老之地,尊者之位也。尊长在西,卑幼在东。尊长,主也;卑幼,助也。主少而助多,尊无二上,卑有百下也。西益主益,主不增助,二上不百下也,於义不善,故谓不祥。不祥者,不宜也,於义不宜,未有凶也。何以明之?夫墓,死人所藏;田,人所饮食;宅,人所居处。三者於人,吉凶宜等。西益宅不祥,西益墓与田,不言不祥。夫墓,死人所居,因忽不慎。田,非人所处,不设尊卑。宅者,长幼所共,加慎致意者,何可不之讳?义详於宅,略於墓与田也。

二曰讳被刑为徒,不上丘墓。但知不可,不能知其不可之意。问其禁之者,不能知其讳,受禁行者,亦不要其忌。连相放效,至或於被刑,父母死,不送葬;若至墓侧,不敢临葬;甚失至於不行吊伤、见佗之人柩。夫徒,〔辠〕人也,被刑谓之徒。丘墓之上,二亲也,死亡谓之先。宅与墓何别?亲与先何异?如以徒被刑,先人责之,则不宜入宅与亲相见;如徒不得与死人相见,则亲死在堂,不得哭柩;如以徒不得升丘墓,则徒不得上山陵,世俗禁之,执据何义?实说其意,徒不上丘墓有二义,义理之讳,非凶恶之忌也。徒用心以为先祖全而生之,子孙亦当全而归之。故曾子有疾,召门弟子曰:``开予足,开予手,而今而後,吾知免夫。小子!''曾子重慎,临绝效全,喜免毁伤之祸也。孔子曰:``身体发肤,受之父母,弗敢毁伤。''孝者怕入刑辟,刻画身体,毁伤发肤,少德泊行,不戒慎之所致也。愧负刑辱,深自刻责,故不升墓祀於先。古礼庙祭,今俗墓祀,故不升墓。惭负先人,一义也。墓者,鬼神所在,祭祀之处。祭祀之礼,齐戒洁清,重之至也。今已被刑,刑残之人,不宜与祭供侍先人,卑谦谨敬,退让自贱之意也。缘先祖之意,见子孙被刑,恻怛惨伤,恐其临祀,不忍歆享,故不上墓。二义也。昔太伯见王季有圣子文王,知太王意欲立之,入吴采药,断发文身,以随吴俗。太王薨,太伯还,王季辟主。太伯再让,王季不听,三让,曰:``吾之吴越,吴越之俗,断发文身,吾刑余之人,不可为宗庙社稷之主。''王季知不可,权而受之。夫徒不上丘墓,太伯不为主之义也。是谓祭祀不可,非谓柩当葬,身不送也。葬死人,先祖痛;见刑人,先祖哀。权可哀之身,送可痛之尸,使先祖有知,痛尸哀形,何愧之有?如使无知,丘墓,田野也,何惭之有?惭愧先者,谓身体刑残,与人异也。古者用刑,形毁不全,乃不可耳。方今象刑,象刑重者,髡钳之法也。若完城旦以下,施刑彩衣系躬,冠带与俗人殊,何为不可?世俗信而谓之皆凶,其失至於不吊乡党尸,不升佗人之丘,感也。

三曰讳妇人乳子,以为不吉。将举吉事,入山林,远行,度川泽者,皆不与之交通。乳子之家,亦忌恶之。丘墓庐道畔,逾月乃入,恶之甚也。暂卒见若为不吉,极原其事,何以为恶?夫妇人之乳子也,子含元气而出。元气,天地之精微也,何凶而恶之?人,物也;子,亦物也。子生与万物之生何以异?讳人之生谓之恶,万物之生又恶之乎?生与胞俱出,如以胞为不吉,人之有胞,犹木实之有核也,包〔裹〕兒身,因与俱出,若鸟卵之有壳,何妨谓之恶?如恶以为不吉,则诸生物有核壳者,宜皆恶之。万物广多,难以验事。人生何以异於六畜?皆含血气怀子,子生与人无异,独恶人而不憎畜,岂以人体大,气血盛乎?则夫牛马体大於人。凡可恶之事,无与钧等,独有一物,不见比类,乃可疑也。今六畜与人无异,其乳皆同一状。六畜与人无异,讳人不讳六畜,不晓其故也。世能别人之产与六畜之乳,吾将听其讳;如不能别,则吾谓世俗所讳妄矣。

且凡人所恶,莫有腐臭。腐臭之气,败伤人心。故鼻闻臭,口食腐,心损口恶,霍乱呕吐。夫更衣之室,可谓臭矣;鲍鱼之肉,可谓腐矣。然而有甘之更衣之室,不以为忌;肴食腐鱼之肉,不以为讳。意不存以为恶,故不计其可与不也。凡可憎恶者,若溅墨漆,附著人身。今目见鼻闻,一过则已,忽亡辄去,何故恶之?出见负豕於涂,腐澌於沟,不以为凶者,洿辱自在彼人,不著己之身也。今妇人乳子,自在其身,斋戒之人,何故忌之?

江北乳子,不出房室,知其无恶也。至於犬乳,置之宅外,此复惑也。江北讳犬不讳人,江南讳人不讳犬,谣俗防恶,各不同也。夫人与犬何以异?房室宅外何以殊,或恶或不恶,或讳或不讳,世俗防禁,竟无经也。月之晦也,日月合宿,纪为一月,犹八日,〔日〕月中分谓之弦;十五日,日月相望谓之望;三十日,日月合宿谓之晦。晦与弦望一实也,非月晦日月光气与月朔异也,何故逾月谓之吉乎?如实凶,逾月未可谓吉;如实吉,虽未逾月,犹为可也。实说讳忌产子、乳犬者,欲使人常自洁清,不欲使人被污辱也。夫自洁清则意精,意精则行清,行清而贞廉之节立矣。

四曰讳举正月、五月子。以为正月、五月子杀父与母,不得也举已举之,父母祸死,则信而谓之真矣。夫正月、五月子何故杀父与母?人之含气在腹肠之内,其生,十月而产,共一元气也。正与二月何殊?五与六月何异?而谓之凶也?世传此言久,拘数之人,莫敢犯之。弘识大材,实核事理,深睹吉凶之分者,然後见之。昔齐相田婴贱妾有子,名之曰文。文以五月生,婴告其母勿举也,其母窃举生之。及长,其母因兄弟而见其子文於婴,婴怒曰:``吾令女去此子,而敢生之,何也?''文顿首,因曰:``君所以不举五月子者,何故?''婴曰:``五月子者,长至户,将不利其父母。''文曰:``人生受命於天乎?将受命於户邪?''婴嘿然。文曰:``必受命於天,君何忧焉。如受命於户,即高其户,谁能至者?''婴善其言,曰:``子休矣!''其後使文主家,待宾客,宾客日进,名闻诸侯。文长过户而婴不死。以田文之说言之,以田婴不死效之,世俗所讳,虚妄之言也。夫田婴俗父,而田文雅子也。婴信忌不实义,文信命不辟讳。雅俗异材,举措殊操,故婴名暗而不明,文声驰而不灭。实说世俗讳之,亦有缘也。夫正月岁始,五月盛阳,子以生,精炽热烈,厌胜父母,父母不堪,将受其患。传相放效,莫谓不然。有空讳之言,无实凶之效,世俗惑之,误非之甚也。

夫忌讳非一,必托之神怪,若设以死亡,然後世人信用畏避。忌讳之语,四方不同,略举通语,令世观览。若夫曲俗微小之讳,众多非一,咸劝人为善,使人重慎,无鬼神之害,凶丑之祸。世讳作豆酱恶闻雷,一人不食,欲使人急作,不欲积家逾至春也。讳厉刀井上,恐刀堕井中也;或说以为刑之字,井与刀也,厉刀井上,井刀相见,恐被刑也。毋承屋檐而坐,恐瓦堕击人首也。毋反悬冠,为似死人服;或说恶其反而承尘溜也。毋偃寝,为其象尸也。毋以箸相受,为其不固也。毋相代扫,为修冢之人,冀人来代己也。诸言毋者,教人重慎,勉人为善。礼曰:``毋抟饭,毋流歠。''礼义之禁,未必吉凶之言也。

\hypertarget{header-n832}{%
\subsubsection{间时篇}\label{header-n832}}

世俗起土兴功,岁月有所食,所食之地,必有死者。假令太岁在子,岁食於酉,正月建寅,月食於巳,子、寅地兴功,则酉、巳之家见食矣。见食之家,作起厌胜,以五行之物,悬金木水火。假令岁月食西家,西家悬金;岁月食东家,东家悬炭。设祭祀以除其凶,或空亡徙以辟其殃。连相仿效,皆谓之然。如考实之,虚妄迷也。何以明之?

夫天地之神,用心等也。人民无状,加罪行罚,非有二心两意,前後相反也。移徙不避岁月,岁月恶其不避己之冲位,怒之也。今起功之家,亦动地体,无状之过,与移徙等。起功之家,当为岁所食,何故反令巳、酉之地受其咎乎?岂岁月之神怪移徙而〔不〕咎起功哉!用心措意,何其不平也。鬼神罪过人,犹县官谪罚民也。民犯刑罚多非一,小过宥罪,大恶犯辟,未有以无过受罪。无过而受罪,世谓之冤。今巳、酉之家,无过於月岁,子、〔寅〕起宅,空为见食,此则岁冤无罪也。且夫太岁在子,子宅直符,午宅为破,不须兴功起事,空居无为,犹被其害。今岁月所食,待子宅有为,巳、酉乃凶。太岁,岁月之神,用罚为害,动静殊致,非天从岁月神意之道也。

审论岁月之神,岁则太岁也,在天边际,立於子位。起室者在中国一州之内,假令扬州在东南,使如邹衍之言,天下为一州,又在东南,岁食於酉,食西羌之地,东南之地安得凶祸?假令岁在人民之间,西宅为酉地,则起功之家,宅中亦有酉地,何以不近食其宅中之酉地,而反食佗家乎!且食之者审谁也?如审岁月,岁月,天之从神,饮食与天同。天食不食人,故郊祭不以为牲。如非天神,亦不食人。天地之间,百神所食,圣人谓当与人等。推生事死,推人事鬼,故百神之祀皆用众物,无用人者。物食人者,虎与狼也。岁月之神,岂虎狼之精哉?仓卒之世,谷食乏匮,人民饥饿,自相啖食。岂其啖食死者,其精为岁月之神哉?岁月有神,日亦有神,岁食月食,日何不食?积日为月,积月为时,积时为岁,千五百三十九岁为一统,四千六百一十七岁为一元,增积相倍之数,分余终竟之名耳,安得鬼神之怪、祸福之验乎?如岁月终竟者宜有神,则四时有神,统元有神。月三日魄,八日弦,十五日望,与岁月终竟何异?岁月有神,魄与弦复有神也?一日之中,分为十二时,平旦寅,日出卯也。十二月建寅卯,则十二月时所加寅卯也。日加十二辰不食,月建十二辰独食,岂日加无神,月建独有哉?何故月建独食,日加不食乎!如日加无神,用时决事非也。如加时有神,独不食,非也。

神之口腹,与人等也。人饥则食,饱则止,不为起功乃一食也。岁月之神,起功乃食,一岁之中,兴功者希,岁月之神饥乎?仓卒之世,人民亡,室宅荒废,兴功者绝,岁月之神饿乎?且田与宅俱人所治,兴功用力,劳佚钧等。宅掘土而立木,田凿沟而起堤,堤与木俱立,掘与凿俱为。起宅,岁月食;治田,独不食。岂起宅时岁月饥,治田时饱乎?何事钧作同,饮食不等也?

说岁月食之家,必铨功之小大,立远近之步数。假令起三尺之功,食一步之内;起十丈之役,食一里之外。功有小大,祸有近远。蒙恬为秦筑长城,极天下之半,则其为祸宜以万数。案长城之造,秦民不多死。周公作雒,兴功至大,当时岁月宜多食。圣人知其审食,宜徙所食地,置於吉祥之位。如不知避,人民多凶,经传之文,贤圣宜有刺讥。今闻筑雒之民,四方和会,功成事毕,不闻多死。说岁月之家,殆虚非实也。且岁月审食,犹人口腹之饥必食也;且为巳、酉地有厌胜之故,畏一金刃,惧一死炭,同闭口不敢食哉!

如实畏惧,宜如其数。五行相胜,物气钧适。如〔泰〕山失火,沃以一杯之水;河决千里,塞以一掊之土,能胜之乎?非失五行之道,小大多少不能相当也。天地之性,人物之力,少不胜多,小不厌大。使三军持木杖,匹夫持一刃,伸力角气,匹夫必死。金性胜木,然而木胜金负者,木多而金寡也。积金如山,燃一炭火以燔烁之,金必不消。非失五行之道,金多火少,少多小大不钧也。五尺童子与孟贲争,童子不胜。非童子怯,力少之故也。狼众食人,人众食狼。敌力角气,能以小胜大者希;争强量功,能以寡胜众者鲜。天道人物,不能以小胜大者,少不能服多。以一刃之金,一炭之火,厌除凶咎,却岁之殃,如何也!

\hypertarget{header-n837}{%
\subsection{卷二十四}\label{header-n837}}

\hypertarget{header-n838}{%
\subsubsection{讥日篇}\label{header-n838}}

世俗既信岁时,而又信日。举事若病死灾患,大则谓之犯触岁月,小则谓之不避日禁。岁月之传既用,日禁之书亦行。世俗之人,委心信之;辩论之士,亦不能定。是以世人举事,不考於心而合於日,不参於义而致於时。时日之书,众多非一,略举较著,明其是非,使信天时之人,将一疑而倍之。夫祸福随盛衰而至,代谢而然。举事曰凶,人畏凶有效;曰吉,人冀吉有验。祸福自至,则述前之吉凶,以相戒惧此日禁所以累世不疑,惑者所以连年不悟也。

《葬历》曰:``葬避九空、地臽,及日之刚柔,月之奇耦。''日吉无害,刚柔相得,奇耦相应,乃为吉良。不合此历,转为凶恶。''夫葬,藏棺也;敛,藏尸也。初死藏尸於棺,少久藏棺於墓。墓与棺何别?敛与葬何异?敛於棺不避凶,葬於墓独求吉。如以墓为重,夫墓,土也,棺,木也,五行之性,木土钧也。治木以赢尸,穿土以埋棺,治与穿同事,尸与棺一实也。如以穿土贼地之体,凿沟耕园,亦宜择日。世人能异其事,吾将听其禁;不能异其事,吾不从其讳。日之不害,又求日之刚柔;刚柔既合,又索月之奇耦。夫日之刚柔,月之奇耦,合於《葬历》,验之於吉,无不相得。何以明之?春秋之时,天子、诸侯、卿、大夫死以千百数,案其葬日,未必合於历。

又曰:``雨不克葬,庚寅日中乃葬。''假令鲁小君以刚日死,至葬日己丑,刚柔等矣。刚柔合,善日也。不克葬者,避雨也。如善日,不当以雨之故,废而不用也。何则?雨不便事耳,不用刚柔,重凶不吉,欲便事而犯凶,非鲁人之意,臣子重慎之义也。今废刚柔,待庚寅日中,以为吉也。《礼》:``天子七月而葬,诸侯五月,卿、大夫、士三月。''假令天子正月崩,七月葬;二月崩,八月葬。诸侯、卿、大夫、士皆然。如验之《葬历》,则天子、诸侯葬月常奇常耦也。衰世好信禁,不肖君好求福。春秋之时,可谓衰矣!隐、哀之间,不肖甚矣。然而葬埋之日,不见所讳,无忌之故也。周文之世,法度备具,孔子意密,《春秋》义纤,如废吉得凶,妄举触祸,宜有微文小义,贬讥之辞。今不见其义,无《葬历》法也。

祭祀之历,亦有吉凶。假令血忌、月杀之日固凶,以杀牲设祭,必有患祸。夫祭者,供食鬼也;鬼者,死人之精也。若非死人之精,人未尝见鬼之饮食也。推生事死,推人事鬼,见生人有饮食,死为鬼当能复饮食,感物思亲,故祭祀也。及他神百鬼之祠,虽非死人,其事之礼,亦与死人同。盖以不见其形,但以生人之礼准况之也。生人饮食无日,鬼神何故有日?如鬼神审有知,与人无异,则祭不宜择日。如无知也,不能饮食,虽择日避忌,其何补益?实者,百祀无鬼,死人无知。百祀报功,示不忘德。死如事生,示不背亡。祭之无福,不祭无祸。祭与不祭,尚无祸福,况日之吉凶,何能损益?如以杀牲见血,避血忌、月杀,则生人食六畜亦宜辟之。海内屠肆,六畜死者,日数千头,不择吉凶,早死者,未必屠工也。天下死罪,冬月断囚亦数千人,其刑於市,不择吉日,受祸者,未必狱吏也。肉尽杀牲,狱具断囚。囚断牲杀,创血之实,何以异於祭祀之牲?独为祭祀设历,不为屠工、狱吏立见,世俗用意不实类也。祭非其鬼,又信非其讳,持二非往求一福,不能得也。

《沐书》曰:``子日沐,令人爱之。卯日沐,令人白头。''夫人之所爱憎,在容貌之好丑;头发白黑,在年岁之稚老。使丑如嫫母,以子日沐,能得爱乎?使十五女子以卯日沐,能白发乎?且沐者,去首垢也。洗去足垢,盥去手垢,浴去身垢,皆去一形之垢,其实等也。洗、盥、浴不择日,而沐独有日。如以首为最尊,则浴亦治面,面亦首也。如以发为最尊,则栉亦宜择日。栉用木,沐用水,水与木俱五行也。用木不避忌,用水独择日。如以水尊於木,则诸用水者宜皆择日。且水不若火尊,如必以尊卑,则用火者宜皆择日。且使子沐,人爱之;卯沐,其首白者,谁也?夫子之性,水也;卯,木也。水不可爱,木色不白。子之禽鼠,卯之兽兔也。鼠不可爱,兔毛不白。以子日沐,谁使可爱?卯日沐,谁使凝白者?夫如是,沐之日无吉凶,为沐立日历者,不可用也。

裁衣有书,书有吉凶。凶日制衣则有祸,吉日则有福。夫衣与食俱辅人体,食辅其内,衣卫其外。饮食不择日,制衣避忌日,岂以衣为於其身重哉?人道所重,莫如食急,故八政一曰食,二曰货。衣服,货也。如以加之於形为尊重,在身之物,莫大於冠。造冠无禁,裁衣有忌,是於尊者略,卑者详也。且夫沐去头垢,冠为首饰;浴除身垢,衣卫体寒。沐有忌,冠无讳;浴无吉凶,衣有利害。俱为一体,共为一身,或善或恶,所讳不均,欲人浅知,不能实也。且衣服不如车马。九锡之礼,一曰车马,二曰衣服。作车不求良辰,裁衣独求吉日,俗人所重,失轻重之实也。

工伎之书,起宅盖屋必择日。夫屋覆人形,宅居人体,何害於岁月而必择之?如以障蔽人身者神恶之,则夫装车、治船、着盖、施帽,亦当择日。如以动地穿土神恶之,则夫凿沟耕园亦宜择日。夫动土扰地神,地神能原人无有恶意,但欲居身自安,则神之圣心,必不忿怒。不忿怒,虽不择日,犹无祸也。如土地之神不能原人之意,苟恶人动扰之,则虽择日,何益哉?王法禁杀伤人,杀伤人皆伏其罪,虽择日犯法,终不免罪;如不禁也,虽妄杀伤,终不入法。县官之法,犹鬼神之制也;穿凿之过,犹杀伤之罪也。人杀伤,不在择日,缮治室宅何,故有忌?

又学书讳丙日,云:``仓颉以丙日死也。''礼不以子卯举乐,殷、夏以子卯日亡也。如以丙日书,子卯日举乐,未必有祸,重先王之亡日,凄怆感动,不忍以举事也。忌日之法,盖丙与子卯之类也,殆有所讳,未必有凶祸也。堪舆历,历上诸神非一,圣人不言,诸子不传,殆无其实。天道难知,假令有之,诸神用事之日也,忌之何福?不讳何祸?王者以甲子之日举事,民亦用之,王者闻之,不刑法也。夫王者不怒民不与己相避,天神何为独当责之?王法举事,以人事之可否,不问日之吉凶。孔子曰:``卜其宅兆而安厝之。''《春秋》祭祀,不言卜日。《礼》曰:``内事以柔日,外事以刚日。''刚柔以慎内外,不论吉凶以为祸福。

\hypertarget{header-n842}{%
\subsubsection{卜筮篇}\label{header-n842}}

俗信卜筮,谓卜者问天,筮者问地,蓍神龟灵,兆数报应,故舍人议而就卜筮,违可否而信吉凶。其意谓天地审告报,蓍龟真神也。如实论之,卜筮不问天地,蓍龟未必神灵。有神灵,问天地,俗儒所言也。何以明之?

子路问孔子曰:``猪肩羊膊,可以得兆,雚苇藁芼,可以得数,何必以蓍龟?''孔子曰:``不然!盖取其名也。夫蓍之为言耆也,龟之为言旧也,明狐疑之事,当问耆旧也。''由此言之,蓍不神,龟不灵,盖取其名,未必有实也。无其实,则知其无神灵,无神灵,则知不问天地也。且天地口耳何在,而得问之?天与人同道,欲知天,以人事。相问,不自对见其人,亲问其意,意不可知。欲问天,天高,耳与人相远。如天无耳,非形体也。非形体,则气也,气若云雾,何能告人?蓍以问地,地有形体,与人无异。问人,不近耳,则人不闻,人不闻,则口不告人。夫言问天,则天为气,不能为兆;问地,则地耳远,不闻人言。信谓天地告报人者,何据见哉?

人在天地之间,犹虮虱之着人身也。如虮虱欲知人意,鸣人耳傍,人犹人闻。何则?小大不均,音语不通也。今以微小之人,问巨大天地,安能通其声音?天地安能知其旨意?或曰:``人怀天地之气。天地之气,在形体之中,神明是矣。人将卜筮,告令蓍龟,则神以耳闻口言。若己思念,神明从胸腹之中闻知其旨。故钻龟揲蓍,兆见数著。''夫人用神思虑,思虑不决,故问蓍龟。蓍龟兆数,与意相应,则是神可谓明告之矣。时或意以为可,兆数不吉;或兆数则吉,意以为凶。夫思虑者,己之神也;为兆数者,亦己之神也。一身之神,在胸中为思虑,在胸外为兆数,犹人入户而坐,出门而行也。行坐不异意,出入不易情。如神明为兆数,不宜与思虑异。天地有体,故能摇动。摇动有生之类也。生,则与人同矣。问生人者,须以生人,乃能相报。如使死人问生人,则必不能相答。今天地生而蓍龟死,以死问生,安能得报?枯龟之骨,死蓍之茎,问生之天地,世人谓之天地报应,误矣。如蓍龟为若版牍,兆数为若书字,象类人君出教令乎?则天地口耳何在而有教令?孔子曰:``天何言哉?四时行焉,百物生焉。''天不言,则亦不听人之言。天道称自然无为,今人问天地,天地报应,是自然之有为以应人也。案《易》之文,观揲蓍之法,二分以象天地,四揲以象四时,归奇於扐,以象闰月。以象类相法,以立卦数耳。岂云天地〔告〕报人哉?

人道,相问则对,不问不应。无求,空扣人之门;无问,虚辨人之前,则主人笑而不应,或怒而不对。试使卜筮之人,空钻龟而卜,虚揲蓍而筮,戏弄天地,亦得兆数,天地妄应乎?又试使人骂天而卜,驱地而筮,无道至甚,亦得兆数。苟谓兆数天地之神,何不灭其火,灼其手,振其指而乱其数,使之身体疾痛,血气凑踊?而犹为之见兆出数,何天地之不惮劳,用心不恶也?由此言之,卜筮不问天地,兆数非天地之报,明矣。然则卜筮亦必有吉凶。论者或谓随人善恶之行也,犹瑞应应善而至,灾异随恶而到。治之善恶,善恶所致也,疑非天地故应之也。吉人钻龟,辄从善兆;凶人揲蓍,辄得逆数。何以明之?纣,至恶之君也,当时灾异繁多,七十卜而皆凶,故祖伊曰:``格人元龟,罔敢知吉。''贤者不举,大龟不兆,灾变亟至,周武受命。高祖龙兴,天人并佑,奇怪既多,丰、沛子弟,卜之又吉。故吉人之体,所致无不良;凶人之起,所招无不丑。卫石骀卒,无适子,有庶子六人,卜所以为后者,曰:``沐浴佩玉则兆。''五人皆沐浴佩玉。石祁子曰:``焉有执亲之丧而沐浴佩玉!''不沐浴佩玉,石祁子兆。卫人卜以龟为有知也。龟非有知,石祁子自知也。祁子行善政,有嘉言,言嘉政善,故有明瑞。使时不卜,谋之於众,亦犹称善。何则?人心神意同吉凶也。此言若然,然非卜筮之实也。

夫钻龟揲蓍,自有兆数,兆数之见,自有吉凶,而吉凶之人,适与相逢。吉人与善兆合,凶人与恶数遇,犹吉人行道逢吉事,顾睨见祥物,非吉事祥物为吉人瑞应也。凶人遭遇凶恶於道,亦如之。夫见善恶,非天应答,适与善恶相逢遇也。钻龟揲蓍有吉凶之兆者,逢吉遭凶之类也。何以明之?周武王不豫,周公卜三龟。公曰:``乃逢是吉。''鲁卿庄叔生子穆叔,以《周易》筮之,遇《明夷》之《谦》。夫卜曰逢,筮曰遇,实遭遇所得,非善恶所致也。善则逢吉,恶则遇凶,天道自然,非为人也。推此以论,人君治有吉凶之应,亦犹此也。君德遭贤,时适当平,嘉物奇瑞偶至。不肖之君,亦反此焉。

世人言卜筮者多,得实诚者寡。论者或谓蓍龟可以参事,不可纯用。夫钻龟揲蓍,兆数辄见。见无常占,占者生意。吉兆而占谓之凶,凶数而占谓之吉,吉凶不效,则谓卜筮不可信。周武王伐纣,卜筮之,逆,占曰:``大凶。''太公推蓍蹈龟而曰:``枯骨死草,何知而凶?''夫卜筮兆数,非吉凶误也,占之不审吉凶,吉凶变乱,变乱,故太公黜之。夫蓍筮龟卜,犹圣王治世;卜筮兆数,犹王治瑞应。瑞应无常,兆数诡异。诡异则占者惑,无常则议者疑。疑则谓〔世〕未治,惑则谓〔占〕不良。何以明之?夫吉兆数,吉人可遭也;治遇符瑞,圣德之验也。周王伐纣,遇乌鱼之瑞,其卜曷为逢不吉之兆?使武王不当起,出不宜逢瑞;使武王命当兴,卜不宜得凶。由此言之,武王之卜,不得凶占,谓之凶者,失其实也。鲁将伐越,筮之,得``鼎折足''。子贡占之以为凶。何则?鼎而折足,行用足,故谓之凶。孔子占之以为吉,曰:``越人水居,行用舟不用足,故谓之吉。''鲁伐越,果克之。夫子贡占鼎折足以为凶,犹周之占卜者谓之逆矣。逆中必有吉,犹折鼎足之占,宜以伐越矣。周多子贡直占之知,寡若孔子诡论之材,故睹非常之兆,不能审也。世因武王卜,无非而得凶,故谓卜筮不可纯用,略以助政,示有鬼神,明己不得专。

著书记者,采掇行事,若韩非《饰邪》之篇,明已效之验,毁卜訾筮,非世信用。夫卜筮非不可用,卜筮之人,占之误也。《洪范》稽疑,卜筮之变,必问天子卿士,或时审是。夫不能审占,兆数不验,则谓卜筮不可信用。晋文公与楚子战,梦与成王搏,成王在上而监其脑,占曰``凶''。咎犯曰:``吉!君得天,楚伏其罪。君之脑者,柔之也。''以战果胜,如咎犯占。夫占梦与占龟同。晋占梦者不见象指,犹周占龟者不见兆者为也。象无不然,兆无不审。人之知暗,论之失实也。传或言:武王伐纣,卜之而龟\textless{}兆昔\textgreater{}。占者曰``凶''。太公曰:``龟\textless{}兆昔\textgreater{},以祭则凶,以战则胜。''武王从之,卒克纣焉。审若此传,亦复孔子论卦,咎犯占梦之类也。盖兆数无不然,而吉凶失实者,占不巧工也。

\hypertarget{header-n846}{%
\subsubsection{辨祟篇}\label{header-n846}}

世俗信祸祟,以为人之疾病死亡,及更患被罪,戮辱欢笑,皆有所犯。起功、移徙、祭祀、丧葬、行作、入官、嫁娶,不择吉日,不避岁月,触鬼逢神,忌时相害。故发病生祸,絓法入罪,至於死亡,殚家灭门,皆不重慎,犯触忌讳之所致也。如实论之,乃妄言也。

凡人在世,不能不作事,作事之後,不能不有吉凶。见吉则指以为前时择日之福,见凶则刾以为往者触忌之祸。多或择日而得祸,触忌而获福。工伎射事者欲遂其术,见祸忌而不言,闻福匿而不达,积祸以惊不慎,列福以勉畏时。故世人无愚智、贤不肖、人君布衣,皆畏惧信向,不敢抵犯。归之久远,莫能分明,以为天地之书,贤圣之术也。人君惜其官,人民爱其身,相随信之,不复狐疑。故人君兴事,工伎满閤,人民有为,触伤问时。奸书伪文,由此滋生。巧惠生意,作知求利,惊惑愚暗,渔富偷贫,愈非古法度圣人之至意也。

圣人举事,先定於义。义已定立,决以卜筮,示不专己,明与鬼神同意共指,欲令众下信用不疑。故《书》列七卜,《易》载八卦,从之未必有福,违之未必有祸。然而祸福之至,时也;死生之到,命也。人命悬於天,吉凶存於时。命穷,操行善,天不能续。命长,操行恶,天不能夺。天,百神主也。道德仁义,天之道也;战粟恐惧,天之心也。废道灭德,贱天之道;险隘恣睢,悖天之意。世间不行道德,莫过桀、纣;妄行不轨,莫过幽、厉。桀、纣不早死,幽、厉不夭折。由此言之,逢福获喜,不在择日避时;涉患丽祸,不在触岁犯月,明矣。孔子曰:``死生有命,富贵在天。''苟有时日,诚有祸祟,圣人何惜不言?何畏不说?案古图籍,仕者安危,千君万臣,其得失吉凶,官位高下,位禄降升,各有差品。家人治产,贫富息耗,寿命长短,各有远近。非高大尊贵举事以吉日,下小卑贱以凶时也。以此论之,则亦知祸福死生不在遭逢吉祥、触犯凶忌也。然则人之生也,精气育也;人之死者,命穷绝也。人之生未必得吉逢喜,其死,独何为谓之犯凶触忌?以孔子证之,以死生论之,则亦知夫百祸千凶,非动作之所致也。孔子圣人,知府也;死生,大事也;大事,道效也。孔子云:``死生有命,富贵在天。''众文微言不能夺,俗人愚夫不能易,明矣。人之於世,祸福有命;人之操行,亦自致之。其安居无为,祸福自至,命也。其作事起功,吉凶至身,人也。人之疾病,希有不由风湿与饮食者。当风卧湿,握钱问祟;饱饭餍食,斋精解祸。而病不治,谓祟不得;命自绝,谓筮不审,欲人之知也。

夫倮虫三百六十,人为之长。人,物也,万物之中有知慧者也。其受命於天,禀气於元,与物无异。鸟有巢栖,兽有窟穴,虫鱼介鳞,各有区处,犹人之有室宅楼台也。能行之物,死伤病困,小大相害。或人捕取以给口腹,非作窠穿穴有所触,东西行徙有所犯也。人有死生,物亦有终始;人有起居,物亦有动作。血脉、首足、耳目、鼻口与人不别,惟好恶与人不同,故人不能晓其音,不见其指耳!及其游於党类,接於同品,其知去就,与人无异。共天同地,并仰日月,而鬼神之祸,独加於人,不加於物,未晓其故也。天地之性,人为贵,岂天祸为贵者作,不为贱者设哉!何其性类同而祸患别也?

刑不上大夫,圣王於贵者阔也。圣王刑贱不罚贵,鬼神祸贵不殃贱,非《易》所谓大人与鬼神合其吉凶也。〔或〕有所犯,抵触县官,罗丽刑法,不曰过所致,而曰家有负。居处不慎,饮食过节,不曰失调和,而曰徙触时。死者累属,葬棺至十,不曰气相污,而曰葬日凶。有事归之有犯,无为归之所居。居衰宅耗,蜚凶流尸,集人室居,又祷先祖,寝祸遗殃。疾病不请医,更患不修行,动归於祸,名曰犯触,用知浅略,原事不实,俗人之材也。犹系罪司空作徒,未必到吏日恶,系役时凶也。使杀人者求吉日出诣吏,剬罪〔者〕,推善时入狱系,宁能令事解,赦令至哉?人不触祸不被罪,不被罪不入狱。一旦令至,解械径出,未必解除其凶者也。天下千狱,狱中万囚,其举事未必触忌讳也。居位食禄,专城长邑,以千万数,其迁徙日未必逢吉时也。历阳之都,一夕沉而为湖,其民未必皆犯岁月也。高祖始起,丰、沛俱复,其民未必皆慎时日也。项羽攻襄安,襄安无噍类,未必不祷赛也。赵军为秦所坑於长平之下,四十万众同时俱死,其出家时,未必不择时也。辰日不哭,哭有重丧。戊己死者,复尸有随。一家灭门,先死之日,未必辰与戊己也。血忌下杀牲,屠肆不多祸,上朔不会众,沽沾舍不触殃。涂上之暴尸,未必出以往亡;室中之殡柩,未必还以归忌。由此言之,诸占射祸祟者,皆不可信用。信用之者,皆不可是。

夫使食口十人,居一宅之中,不动锤〔锸〕,不更居处,祠祀嫁娶,皆择吉日,从春至冬,不犯忌讳,则夫十人比至百年,能不死乎?占射事者必将复曰:``宅有盛衰,若岁破、直符,不知避也。''夫如是,令数问工伎之家,宅盛即留,衰则避之,及岁破、直符,辄举家移,比至百年,能不死乎?占射事者必将复曰:``移徙触时,往来不吉。''夫如是,复令辄问工伎之家,可徙则往,可还则来。比至百年,能不死乎?占射事者必将复曰:``泊命寿极。''夫如是,人之死生,竟自有命,非触岁月之所致,无负凶忌之所为也。

\hypertarget{header-n850}{%
\subsubsection{难岁篇}\label{header-n850}}

俗人险心,好信禁忌,知者亦疑,莫能实定。是以儒雅服从,工伎得胜。吉凶之书,伐经典之义;工伎之说,凌儒雅之论。今略实论,令〔观〕览,揔核是非,使世一悟。

《移徙法》曰:``徙抵太岁,凶;负太岁,亦凶。''抵太岁名曰岁下,负太岁名曰岁破,故皆凶也。假令太岁在甲子,天下之人皆不得南北徙,起宅嫁娶亦皆避之。其移东西,若徙四维,相之如者皆吉。何者?不与太岁相触,亦不抵太岁之冲也。实问:避太岁者,何意也?令太岁恶人徙乎?则徙者皆有祸。令太岁不禁人徙,恶人抵触之乎?则道上之人,南北行者皆有殃。太岁之意,犹长吏之心也。长吏在涂,人行触车马,干其吏从,长吏怒之,岂独抱器载物,去宅徙居触犯之者,而乃责之哉?昔文帝出,过霸陵桥,有一人行逢车驾,逃於桥下,以为文帝之车已过,疾走而出,惊乘舆马。文帝怒,以属廷尉张释之。释之当论。使太岁之神行若文帝出乎?则人犯之者,必有如桥下走出之人矣。方今行道路者,暴溺仆死,何以知非触遇太岁之出也?为移徙者,又不能处。不能处,则犯与不犯未可知。未可知,则其行与不行未可审也。

且太岁之神审行乎?则宜有曲折,不宜直南北也。长吏出舍,行有曲折。如天神直道不曲折乎?则从东西、四维徙者,犹干之也。若长吏之南北行,人从东如西,四维相之如,犹抵触之。如不正南北,南北之徙又何犯?如太岁不动行乎,则宜有宫室营堡,不与人相见,人安得而触之?如太岁无体,与长吏异,若烟云虹霓,直经天地,极子午南北陈乎?则东西徙,若四维徙者,亦干之。譬若今时人行触繁雾蜮气,无从横负乡皆中伤焉。如审如气,人当见之,虽不移徙,亦皆中伤。且太岁,天别神也,与青龙无异。龙之体不过数千丈,如令神者宜长大,饶之数万丈,令体掩北方,当言太岁在北方,不当言在子。其东有丑,其西有亥,明不专掩北方,极东西之广,明矣。令正言在子位,触土之中,直子午者不得南北徙耳,东边直丑巳之地,西边直亥、未之民,何为不得南北徙?丑与亥地之民,使太岁左右通,得南北徙及东西徙。何则?丑在子东,亥在子西,丑、亥之民东西徙,触岁之位;巳、未之民东西徙,忌岁所破。

儒者论天下九州,以为东西南北,尽地广长,九州之内五千里,竟三河土中。周公卜宅,《经》曰:``王来绍上帝,自服於土中。''雒则土之中也。邹衍论之,以为九州之内五千里,竟合为一州,在东〔南〕位,名曰赤县州。自有九州者九焉,九九八十一,凡八十一州。此言殆虚。地形难审,假令有之,亦一难也。使天下九州,如儒者之议,直雒邑以南,对三河以北,豫州、荆州、冀州之部有太岁耳。雍、梁之间,青、兗、徐、扬之地,安得有太岁?使如邹衍之论,则天下九州在东南位,不直子午,安得有太岁?如太岁不在天地极,分散在民间,则一家之宅,辄有太岁。虽不南北徙,犹抵触之。假令从东里徙西里,西里有太岁,从东宅徙西宅,西宅有太岁,或在人之东西,或在人之南北,犹行途上,东西南北皆逢触人。太岁位数千万亿,天下之民徙者皆凶,为移徙者何以审之?如审立於天地之际,犹王者之位在土中也。东方之民,张弓西射,人不谓之射王者,以不能至王者之都,自止射其处也。今徙岂能北至太岁位哉!自止徙百步之内,何为谓之伤太岁乎?且移徙之家禁南北徙者,以为岁在子位,子者破午,南北徙者抵触其冲,故谓之凶。夫破者须有以椎破之也。如审有所用,则不徙之民,皆被破害;如无所用,何能破之!

夫雷,天气也,盛夏击折,折木破山,时暴杀人。使太岁所破,若迅雷也,则声音宜疾,死者宜暴;如不若雷,亦无能破。如谓冲抵为破,冲抵安能相破?东西相与为冲,而南北相与为抵。如必以冲抵为凶,则东西常凶而南北常恶也。如以太岁神,其冲独凶,神莫过於天地,天地相与为冲,则天地之间无生人也。或上十二神,登明、从魁之辈,工伎家谓之皆天神也。常立子、丑之位,俱有冲抵之气,神虽不若太岁,宜有微败。移徙者虽避太岁之凶,犹触十二神之害,为移徙时者,何以不禁?冬气寒,水也,水位在北方。夏气热,火也,火位在南方。案秋冬寒,春夏热者,天下普然,非独南北之方水火冲也。今太岁位在子耳,天下皆为太岁,非独子、午冲也。审以所立者为主,则午可为大夏,子可为大冬。冬夏南北徙者,可复凶乎?立春,艮王、震相、巽胎、离没、坤死、兑囚、乾废、坎休。王之冲死,相之冲囚,王相冲位,有死囚之气。乾坤六子,天下正道,伏羲、文王象以治世。文为经所载,道为圣所信,明审於太岁矣。人或以立春东北徙,抵艮之下,不被凶害。太岁立於子,彼东北徙,坤卦近於午,犹艮以坤,徙触子位,何故独凶?正月建於寅,破於申,从寅、申徙,相之如者,无有凶害。太岁不指午,而空曰岁破;午实无凶祸,而虚禁南北,岂不妄哉!

十二月为一岁,四时节竟,阴阳气终,竟复为一岁,日月积聚之名耳,何故有神而谓之立於子位乎?积分为日,累日为月,连月为时,纪时为岁。岁则日、月、时之类也。岁而有神,日、月、时亦复有神乎?千五百三十九〔岁〕为一统,四千六百一十七岁为一元。岁犹统元也。岁有神,统元复有神乎?论之以为无。假令有之,何故害人?神莫过於天地,天地不害人。人谓百神,百神不害人。太岁之气,天地之气也,何憎於人,触而为害?且文曰:``甲子不徙。''言甲与子殊位,太岁立子不居甲,为移徙者,运之而复居甲。为之而复居甲,为移徙时者,亦宜复禁东西徙。甲与子钧,其凶宜同。不禁甲,而独忌子,为移徙时者,竟妄不可用也。人居不能不移徙,移徙不能不触岁,触岁不能不得时死。工伎之人,见今人之死,则归祸於往时之徙。俗心险危,死者不绝,故太岁之言,传世不灭。

\hypertarget{header-n855}{%
\subsection{卷二十五}\label{header-n855}}

\hypertarget{header-n856}{%
\subsubsection{诘术篇}\label{header-n856}}

图宅术曰``宅有八术,以六甲之名,数而第之,第定名立,宫商殊别。宅有五音,姓有五声。宅不宜其姓,姓与宅相贼,则疾病死亡,犯罪遇祸。''诘曰:夫人之在天地之间也,万物之贵者耳。其有宅也,犹鸟之有巢,兽之有穴也。谓宅有甲乙,巢穴复有甲乙乎?甲乙之神,独在民家,不在鸟兽何?夫人之有宅,犹有田也,以田饮食,以宅居处。人民所重,莫食最急,先田後宅,田重於宅也。田间阡陌,可以制八术,比土为田,可以数甲乙,甲乙之术独施於宅,不设於田,何也?府廷之内,吏舍比属,吏舍之形制,何殊於宅,吏之居处,何异於民,不以甲乙第舍,独以甲乙数宅,何也?民间之宅,与乡亭比屋相属,接界相连。不并数乡亭,独第民家。甲乙之神,何以独立於民家也?数宅之术行市亭,数巷街以第甲乙。入市门曲折,亦有巷街。人昼夜居家,朝夕坐市,其实一也,市肆户何以不第甲乙?州郡列居,县邑杂处,与街巷民家何以异?州郡县邑何以不数甲乙也?

天地开辟有甲乙邪?後王乃有甲乙。如天地开辟本有甲乙,则上古之时,巢居穴处,无屋宅之居、街巷之制,甲乙之神皆何在?数宅既以甲乙,五行之家数日亦当以甲乙。甲乙有支干,支干有加时。支干加时,专此者吉,相贼者凶。当其不举也,未必加忧辱也。事理有曲直,罪法有轻重,上官平心,原其狱状,未有支干吉凶之验,而有事理曲直之效,为支干者何以对此?武王以甲子日战胜,纣以甲子日战负,二家俱期,两军相当,旗帜相望,俱用一日,或存或亡。且甲与子专比,昧爽时加寅,寅与甲乙不相贼,武王终以破纣,何也?

日,火也,在天为日,在地为火。何以验之?阳燧乡日,火从天来。由此言之,火,日气也。日有甲乙,火无甲乙?何日十而辰十二,日辰相配,故甲与子连。所谓日十者,何等也?端端之日有十邪,而将一有十名也?如端端之日有十,甲乙是其名,何以不徒言甲乙,必言子丑?何日廷图甲乙有位,子丑亦有处,各有部署,列布五方,若王者营卫,常居不动今端端之日中行,旦出东方,夕入西方,行而不已,与日廷异,何谓甲乙为日之名乎?术家更说日甲乙者,自天地神也。日更用事,自用甲乙胜负为吉凶,非端端之日名也。夫如是,於五行之象,徒当用甲乙决吉凶而已,何为言加时乎?案加时者,端端之日加也。端端之日安得胜负?

五音之家,用口调姓名及字,用姓定其名,用名正其字。口有张歙,声有外内,以定五音宫商之实。夫人之有姓者,用禀於天。天得五行之气为姓邪?以口张歙、声外内为姓也?如以本所禀於天者为姓,若五谷万物禀气矣,何故用口张歙、声内外定正之乎?古者因生以赐姓,因其所生赐之姓也。若夏吞薏苡而生,则姓苡氏;商吞燕子而生,则姓为子氏;周履大人迹,则姬氏。其立名也,以信、以像、以假、以类。以生名为信,若鲁公子友生,文在其手曰``友''也。以德名为义,若文王为昌、武王为发也。以类名为像,若孔子名丘也。取於物为假,若宋公名杵臼也。取於父为类,有似类於父也。其立字也,展名取同义,名赐字子贡,名予字子我。其立姓则以本所生,置名则以信、义、像、假、类,字则展名取同义,不用口张歙、〔声〕外内。调宫商之义为五音术,何璩见而用?古有本姓,有氏姓,陶氏、田氏,事之氏姓也;上官氏、司马氏,吏之氏姓也;孟氏、仲氏,王父字之氏姓也。氏姓有三:事乎!吏乎!王父字乎!以本姓则用所生,以氏姓则用事、吏、王父字,用口张歙调姓之义何居?匈奴之俗,有名无姓字,无与相调谐,自以寿命终,祸福何在?礼:``买妾不知其姓则卜之。''不知者,不知本姓也。夫妾必有父母家姓,然而必卜之者,父母姓转易失实,礼重取同姓,故必卜之。姓徒用口调谐姓族,则礼买妾何故卜之?

图宅术曰:``商家门不宜南向,徵家门不宜北向。''则商金,南方火也;徵火,北方水也。水胜火,火贼金,五行之气不相得,故五姓之宅,门有宜向。向得其宜,富贵吉昌;向失其宜,贫贱衰耗。夫门之与堂何以异?五姓之门,各有五姓之堂,所向无宜何?门之掩地,不如堂庑,朝夕所处,於堂不於门。图吉凶者,宜皆以堂。如门,人所出入,则户亦宜然。孔子曰:``谁能出之由户?''言户不言门。五祀之祭,门与户均。如当以门正所向,则户何以不当与门相应乎?且今府廷之内,吏舍连属,门向有南北;长吏舍传,闾居有东西。长吏之姓,必有宫商,诸吏之舍必有徵羽。安官迁徙,未必角姓门南向也;失位贬黜,未必商姓门北出也。或安官迁徙,或失位贬黜何?姓有五音,人之性质亦有五行。五音之家,商家不宜南向门,则人禀金之性者,可复不宜南向坐、南行步乎?一曰:五音之门,有五行之人。假令商姓口食五人,五人中各有五色,木人青,火人赤,水人黑,金人白,土人黄。五色之人,俱出南向之门,或凶或吉,寿命或短或长。凶而短者未必色白,吉而长者,未必色黄也。五行之家,何以为决?南向之门,贱商姓家,其实如何?南方,火也,使火气之祸,若火延燔,径从南方来乎?则虽为北向门犹之凶也。火气之祸,若夏日之热,四方洽浃乎,则天地之间皆得其气,南向门家何以独凶?南方火者,火位南方,一曰其气布在四方,非必南方独有火,四方无有也,犹水位在北方,四方犹有水也。火满天下,水辨四方。火或在人之南,或在人之北。谓火常在南方,是则东方可无金,西方可无木乎?

\hypertarget{header-n860}{%
\subsubsection{解除篇}\label{header-n860}}

世信祭祀,谓祭祀必有福。又然解除,谓解除必去凶。解除初礼,先设祭祀。比夫祭祀,若生人相宾客矣。先为宾客设膳,食已,驱以刃杖。鬼神如有知,必恚与战,不肯径去,若怀恨,反而为祸;如无所知,不能为凶,解之无益,不解无损。且人谓鬼神何如状哉?如谓鬼有形象,形象生人,生人怀恨,必将害人。如无形象,与烟云同,驱逐云烟,亦不能除。形既不可知,心亦不可图,鬼神集止人宅,欲何求乎?如势欲杀人,当驱逐之时,避人隐匿;驱逐之止,则复还立故处。如不欲杀人,寄托人家,虽不驱逐,亦不为害。贵人之出也,万民并观,填街满巷,争进在前。士卒驱之,则走而却,士卒还去,即复其处;士卒立守,终日不离,仅能禁止。何则?欲在於观,不为壹躯还也。使鬼神与生人同,有欲於宅中,犹万民有欲於观也,士卒驱逐,不久立守,则观者不却也。然则驱逐鬼者,不极一岁,鬼神不去。今驱逐之,终食之间,则舍之矣。舍之鬼,复还来,何以禁之?

暴谷於庭,鸡雀啄之,主人驱弹则走,纵之则来,不终日立守,鸡雀不禁。使鬼神乎,不为驱逐去止;使鬼不神乎,与鸡雀等,不常驱逐,不能禁也。虎狼入都,弓弩巡之,虽杀虎狼,不能除虎狼所为来之患。盗贼攻城,官军击之,虽却盗贼,不能灭盗贼所为至之祸。虎狼之来,应政失也;盗贼之至,起世乱也。然则鬼神之集,为命绝也。杀虎狼,却盗贼,不能使政得世治。然则盛解除,驱鬼神,不能使凶去而命延。

病人困笃,见鬼之至,性猛刚者,挺剑操杖,与鬼战斗。战斗壹再,错指受服,知不服,必不终也。夫解除所驱逐鬼,与病人所见鬼无以殊也;其驱逐之,与战斗无以异也。病人战斗,鬼犹不去,宅主解除,鬼神必不离。由此言之,解除宅者,何益於事,信其凶去,不可用也。且夫所除,宅中客鬼也。宅中主神有十二焉,青龙白虎列十二位。龙虎猛神,天之正鬼也,飞尸流凶,安敢妄集,犹主人猛勇,奸客不敢窥也。有十二神舍之,宅主驱逐,名为去十二神之客,恨十二神之意,安能得吉?如无十二神,则亦无飞尸流凶。罻奚裎扌祝獬纻尾梗壳篸鸷稳嘒繝

解逐之法,缘古逐疫之礼也。昔颛顼氏有子三人,生而皆亡,一居江水为虐鬼,一居若水为魍魉,一居欧隅之间主疫病人。故岁终事毕,驱逐疫鬼,因以送陈、迎新、内吉也。世相仿效,故有解除。夫逐疫之法,亦礼之失也。行尧、舜之德,天下太平,百灾消灭,虽不逐疫,疫鬼不往。行桀、纣之行,海内扰乱,百祸并起,虽日逐疫,疫鬼犹来。衰世好信鬼,愚人好求福。周之季世,信鬼修祀。以求福助。愚主心惑,不顾自行,功犹之立,治犹不定。故在人不在鬼,在德不在祀。国期有远近,人命有长短,如祭祀可以得福,解除可以去凶,则王者可竭天下之财,以兴延期之祀;富家翁妪可求解除之福,以取逾世之寿。案天下人民,夭寿贵贱,皆有禄命;操行吉凶,皆有衰盛。祭祀不为福,福不由祭祀。世信鬼神,故好祭祀。祭祀无鬼神,故通人不务焉。祭祀,厚事鬼神之道也,犹无吉福之验,况盛力用威,驱逐鬼神,其何利哉!

祭祀之礼,解除之法,众多非一,且以一事效其非也。夫小祀足以况大祭,一鬼足以卜百神。世间缮治宅舍,凿地掘土,功成作毕,解谢土神,名曰:``解土''。为土偶人,以像鬼形,令巫祝延,以解土神。已祭之後,心快意喜,谓鬼神解谢,殃祸除去。如讨论之,乃虚妄也。何以验之?夫土地犹人之体也,普天之下皆为一体,头足相去,以万里数。人民居土上,犹蚤虱着人身也。蚤虱食人,贼人肌肤,犹人凿地,贼地之体也。蚤虱内知,有欲解人之心,相与聚会,解谢於所食之肉旁,人能知之乎?夫人不能知蚤虱之音,犹地不能晓人民之言也。胡、越之人,耳口相类,心意相似,对口交耳而谈,尚不相解;况人之与地相似,地之耳口与人相远乎!今所解者地乎?则地之耳远,不能闻也。所解一宅之土,则一宅之土犹人一分之肉也,安能晓之!如所解宅神乎,则此名曰``解宅'',不名曰``解土''。礼入宗庙,无所主意,斩尺二寸之木,名之曰主,主心事之,不为人像。今解土之祭,为土偶人,像鬼之形,何能解乎?神荒忽无形,出入无门,故谓之神。今作形象,与礼相违,失神之实,故知其非。象似布藉,不设鬼形。解土之礼,立土偶人,如祭山可为石形,祭门户可作木人乎?

晋中行寅将亡,召其太祝欲加罪焉,曰:``子为我祀,牺牲不肥泽也,且齐戒不敬也,使吾国亡,何也?''祝简对曰:``昔日吾先君中行密子,有车十乘,不忧其薄也,忧德义之不足也。今主君有革车百乘,不忧德义之薄也,唯患车之不足也。夫船车饰则赋敛厚,赋敛厚则民谤诅。君苟以祀为有益於国乎?诅亦将为亡矣。一人祝之,一国诅之,一祝不胜万诅,国亡,不亦宜乎?祝其何罪?''中行子乃惭。今世信祭祀,中行子之类也。不修其行而丰其祝,不敬其上而畏其鬼。身死祸至,归之於祟,谓祟未得;得祟修祀,祸繁不止,归之於祭,谓祭未敬。夫论解除,解除无益;论祭祀,祭祀无补;论巫祝,巫祝无力。竟在人不在鬼,在德不在祀,明矣哉!

\hypertarget{header-n864}{%
\subsubsection{祀义篇}\label{header-n864}}

世信祭祀,以为祭祀者必有福,不祭祀者必有祸。是以病作卜祟,祟得修祀,祀毕意解,意解病已,执意以为祭祀之助,勉奉不绝。谓死人有知,鬼神饮食,犹相宾客,宾客悦喜,报主人恩矣。其修祭祀,是也;信其享之,非也。实者,祭祀之意,主人自尽恩勤而已,鬼神未必欲享之也。何以明之?今所祭者报功,则缘生人为恩义耳,何歆享之有?今所祭死人,死人无知,不能饮食。何以审其不能歆享饮食也?夫天者,体也,与地同。天有列宿,地有宅舍。宅舍附地之体,列宿着天之形。形体具,则有口乃能食。使天地有口能食,祭食宜食尽;如无口,则无体,无体则气也,若云雾耳,亦无能食如。天地之精神,若人之有精神矣。以人之精神,何宜饮食?中人之体七八尺,身大四五围,食斗食,歠斗羹,乃能饱足,多者三四斗。天地之广大,以万里数,圜丘之上,一茧栗牛,粢饴大羹,不过数斛。以此食天地,天地安能饱?天地用心,犹人用意也。人食不饱足,则怨主人,不报以德矣。必谓天地审能饱食,则夫古之效者负天地。山,犹人之有骨节也;水,犹人之有血脉也。故人食肠满,则骨节与血脉因以盛矣。今祭天地,则山川随天地而饱。今别祭山川,以为异神,是人食已,更食骨节与血脉也。

社稷报生谷物之功。万民生於天地,犹毫毛生於体也。祭天地,则社稷设其中矣;人君重之,故复别祭。必以为有神,是人之肤肉当复食也。五祀初本在地。门户用木与土,土木生於地,井灶室中霤皆属於地。祭地,五祀设其中矣;人君重之,故复别祭。必以为有神,是人食己,当复食形体也。风伯、雨师、雷公,是群神也。风犹人之有吹煦也,雨犹人之有精液也,雷犹人之有腹鸣也,三者附於天地,祭天地,三者在矣;人君重之,故别祭。必以为有神,则人吹煦、精液、腹鸣当复食也。日月犹人之有目,星辰犹人之有发。三光附天,祭天,三光在矣;人君重之,故复别祭。必以为有神,则人之食已,复食目与发也。

宗庙,己之先也。生存之时,谨敬供养,死不敢不信,故修祭祀,缘生事死,示不忘先。五帝、三王郊宗黄帝、帝喾之属,报功坚力,不敢忘德,未必有鬼神审能歆享之也。夫不能歆享,则不能神;不能神,则不能为福,亦不能为祸。祸福之起,由於喜怒,喜怒之发,由於腹肠。有腹肠者辄能饮食,不能饮食则无腹肠,无腹肠则无用喜怒,无用喜怒则无用为祸福矣。

或曰:``歆气,不能食也。''夫歆之与饮食,一实也。用口食之,用口歆之。无腹肠则无口,无口无用食,则亦无用歆矣。何以验其不能歆也?以人祭祀有过,不能即时犯也。夫歆不用口则用鼻矣。口鼻能歆之,则目能见之,目能见之,则手能击之。今手不能击,则知口鼻不能歆之也。

或难曰:``宋公鲍之身有疾。祝曰夜姑,掌将事於历者。历鬼杖楫而与之言曰:`何而粢盛之不膏也?何而蒭牺之不肥硕也?何而珪璧之不中度量也?而罪欤?其鲍之罪欤?'夜姑顺色而对曰:`鲍身尚幼,在襁褓,不预知焉。审是掌之。'历鬼举楫而掊之,毙於坛下。此非能言用手之验乎?''

曰:夫夜姑之死,未必历鬼击之也,时命当死也。妖象历鬼,象鬼之形,则象鬼之言,象鬼之言,则象鬼而击矣。何以明之?夫鬼者,神也,神则先知。先知则宜自见粢盛之不膏、圭璧之失度、牺牲之臞小,则因以责让夜姑,以楫击之而已,无为先问。先问,不知之效也;不知,不神之验也。不知不神,则不能见体出言,以楫击人也。夜姑,义臣也,引罪自予己,故鬼击之。如无义而归之鲍身,则厉鬼将复以楫掊鲍之神矣。且祭祀不备,神怒见体,以杀掌祀。如礼备神喜,肯见体以食赐主祭乎?人有喜怒,鬼亦有喜怒。人不为怒者身存,不为喜者身亡。厉鬼之怒,见体而罚。宋国之祀,必时中礼,夫神何不见体以赏之乎?夫怒喜不与人同,则其赏罚不与人等。赏罚不与人等,则其掊夜姑,不可信也。

且夫歆者,内气也,言者,出气也。能歆则能言,犹能吸则能呼矣。如鬼神能歆,则宜言於祭祀之上。今不能言,知不能歆,一也。凡能歆者,口鼻通也。使鼻鼽不通,口钳不开,则不能歆矣。人之死也,口鼻腐朽,安能复歆?二也。《礼》曰:``人死也,斯恶之矣。''与人异类,故恶之也。为尸不动,朽败灭亡,其身不与生人同,则知不与生人通矣。身不同,知不通,其饮食不与人钧矣。胡、越异类,饮食殊味。死之与生,非直胡之与越也。由此言之,死人不歆,三也。当人之卧也,置食物其旁,不能知也。觉乃知之,知乃能食之。夫死,长卧不觉者也,安能知食?不能歆之,四也。

或难曰:```祭则鬼享之',何谓也?''曰:言其修具谨洁,粢牲肥香,人临见之,意饮食之。推己意以况鬼神,鬼神有知,必享此祭,故曰鬼享之也。难曰:``《易》曰:`东邻杀牛,不如西邻之祭。'夫言东邻不若西邻,言东邻牲大福少,西邻祭少福多也。今言鬼不享,何以知其福有多少也?''曰:此亦谓修具谨洁与不谨洁也。

纣杀牛祭,不致其礼。文王衤勺祭,竭尽其敬。夫礼不至,则人非之,礼敬尽,则人是之。是之则举事多助,非之则言行见畔。见畔,若祭,不见享之祸;多助,若祭,见歆之福:非鬼为祭祀之故有喜怒也。何以明之?苟鬼神,不当须人而食。须人而食,是不能神也。信鬼神,歆祭祀,祭祀为祸福,谓鬼神居处何如状哉?自有储偫邪,将以人食为饥饱也?如自有储偫,储偫必与人异,不当食人之物。如无储偫,则人朝夕祭乃可耳。壹祭壹否,则神壹饥壹饱。壹饥壹饱,则神壹怒壹喜矣。且病人见鬼,及卧梦与死人相见,如人之形,故其祭祀,如人之食。缘有饮食则宜有衣服,故复以缯制衣,以象生仪。其祭如生人之食,人欲食之,冀鬼飨之。其制衣也,广枞不过一尺若五六寸。以所见长大之神,贯一尺之衣,其肯喜而加福於人乎?以所见之鬼为审死人乎?则其制衣,宜若生人之服。如以所制之衣审鬼衣之乎?则所见之鬼宜如偶人之状。夫如是也,世所见鬼,非死人之神;或所衣之神非所见之鬼也。鬼神未定,厚礼事之,安得福佑而坚信之乎?

\hypertarget{header-n868}{%
\subsubsection{祭意篇}\label{header-n868}}

礼,王者祭天地,诸侯祭山川,卿大夫祭五祀,土庶人祭其先;宗庙社稷之祀,自天子达於庶人。《尚书》曰:``肆类於上帝,禋于六宗,望于山川,遍於群臣。''《礼》曰:``有虞氏禘黄帝而郊喾,祖颛顼而宗尧。夏後氏亦禘黄帝而郊鲧,祖颛顼而宗禹。殷人禘喾而郊冥,祖契而宗汤。周人喾而郊稷,祖文王而宗武王。燔柴於大坛,祭天也;瘗埋於大折,祭地也,用骍犊。埋少牢於大昭,祭时也;相近於坎坛,祭寒暑也。王宫,祭日也。夜明,祭月也;幽宗,祭星也。雩宗,祭水旱也。四坎坛,祭四方也。山林川谷丘陵能出云,为风雨,见怪物,皆曰神。有天下者祭百神。诸侯在其地则祭,亡其地则不祭。''此皆法度之祀,礼之常制也。

王者父事天,母事地,推人事父母之事,故亦有祭天地之祀。山川以下,报功之义也。缘生人有功得赏,鬼神有功亦祀之。山出云雨润万物,六宗居六合之间,助天地变化,王者尊而祭之。故曰六宗。社稷报生万物之功:社报万物,稷报五谷。五祀报门户井灶中霤之功:门户人所出入,井灶人所饮食,中霤人所托处。五者功钧,故俱祀之。

周弃曰:``少昊有四叔,曰重,曰该,曰修,曰熙,实能金木少及水。使重为句芒,该为蓐收,修及熙为玄冥,世不失职,遂济穷桑,此其三祀也。颛顼氏有子曰犁,为祝融。共工氏有子曰句龙,为後土,此其二祀也。後土为社。稷,田正也。有烈山氏之子曰柱,为稷。自夏以上祀之。周弃亦为稷,自商以来祀之。''《礼》曰:``列山氏之有天下也,其子曰柱,能殖百谷。夏之衰也,周弃继之,故祀以为稷。共工氏之霸九州也,其子曰後土,能平九土,故祀以为社。''传或曰:``炎帝作火,死而为灶。禹劳力天下,水死而为社。''《礼》曰:``王为群姓立七祀,曰司命,曰中霤,曰国门,曰国行,曰泰厉,曰户,曰灶。诸侯为国立五祀,曰司命,曰中霤,曰国门,曰国行,曰公厉。大夫立三祀,曰族厉,曰门,曰行。适士立二祀,曰门,曰行。庶人立二祀,或立户,或立灶。''社稷五祀之祭,未有所定,皆有思其德,不忘其功也。中心爱之,故饮食之。爱鬼神者祭祀之。自禹兴修社稷,祀後,稷其後绝废。高皇帝四年

诏天下祭灵星,七年,使天下祭社稷。灵星之祭,祭水旱也,於礼旧名曰雩。雩之礼,为民祈谷雨,祈谷实也。春求〔雨,秋求〕实,一岁再祀,盖重谷也。春以二月,秋以八月。故《论语》曰:``暮春者,春服既成,冠者五六人,童子六七人,浴乎沂,风乎舞雩,咏而归。''暮春,四月也。周之四月,正岁二月也。二月之时,龙星始出,故传曰:``龙见而雩''。龙星见时,岁己启蛰,□□□:``□□而雩。春雩之礼废,秋雩之礼存,故世常修灵星之祀,到今不绝。名变於旧,故世人不识;礼废不具,故儒者不知。世儒案礼,不知灵星何祀,其难晓而不识,说县官名曰明星。缘明星之名,说曰岁星,岁星东方也。东方主春,春主生物,故祭岁星求春之福也。四时皆有力於物,独求春者,重本尊始也。审如儒者之说,求春之福,及以秋祭,非求春也。《月令》祭户以春,祭门以秋,各宜其时。如或祭门以秋,谓之祭户,论者肯然之乎?不然,则明星非岁星也,乃龙星也。龙星二月见,则雩祈谷雨。龙星八月将入,则秋雩祈谷实。儒者或见其义,语不空生。春雩废,秋雩兴,故秋雩之名,自若为明星也,实曰灵星。灵星者,神也;神者,谓龙星也。群神谓风伯雨师雷公之属。风以摇之,雨以润之,雷以动之,四时生成,寒暑变化。日月星辰,人所瞻仰。水旱,人所忌恶。四方,气所由来。山林川谷,民所取材用。此鬼神之功也。

凡祭祀之义有二:一曰报功,二月修先。报功以勉力,修先以崇恩。力勉恩崇,功立化通,圣王之务也。是故圣王制祭祀也,法施於民则祀之,以死勤事则祀之,以劳定国则祀之,能御大灾则祀之,能捍大患则祀之。帝喾能序星辰以著众,尧能赏均刑法以义终,舜勤民事而野死,鲧勤洪水而殛死,禹能修鲧之功,黄帝正名百物以明民共财,颛顼能修之;契为司徒而民成,冥勤其官而水死,汤以宽治民而除其虐,文王以文治,武王以武功去民之灾。凡此功烈,施布於民,民赖其力,故祭报之。宗庙先祖,己之亲也,生时有养亲之道,死亡义不可背,故修祭祀,示如生存。推人事鬼神,缘生事死,人有赏功供养之道,故有报恩祀祖之义。

孔子之畜狗死,使子赣埋之,曰:``吾闻之也,弊帷不弃,为埋马也;弊盖不弃,为埋狗也。丘也贫,无盖,於其封也,亦与之席,毋使其首陷焉。''延陵季子过徐,徐君好其剑。季子以当使於上国,未之许与。季子使还,徐君已死,季子解剑带其冢树。御者曰:``徐君已死,尚谁为乎?''季子曰:``前已心许之矣。可以徐君死故负吾心乎?''遂带剑於冢树而去。祀为报功者,其用意犹孔子之埋畜狗也。祭为不背先者,其恩犹季子之带剑於冢树也。圣人知其若此,祭犹斋戒畏敬,若有鬼神,修兴弗绝,若有祸福。重恩尊功,殷勤厚恩,未必有鬼而享之者。何以明之?以饮食祭地也。人将饮食,谦退,示当有所先。孔子曰:``虽疏食菜羹,瓜祭必斋如也。''《礼》曰``侍食於君,君使之祭,然後饮食之。''祭,犹礼之诸祀也。饮食亦可毋祭,礼之诸神,亦可毋祀也。祭祀之实一也,用物之费同也。知祭地无神,犹谓诸祀有鬼,不知类也。经传所载,贤者所纪,尚无鬼神,况不著篇籍,世间淫祀非鬼之祭,信其有神为祸福矣?好道学仙者,绝谷不食,与人异食,欲为清洁也。鬼神清洁於仙人,如何与人同食乎?论之以为人死无知,其精不能为鬼。假使有之,与人异食。异食则不肯食人之食,不肯食人之食,则无求於人。无求於人,则不能为人祸福矣。

凡人之有喜怒也,有求得与不得。得则喜,不得则怒。喜则施恩而为福,怒则发怒而为祸。鬼神无喜怒,则虽常祭而不绝,久废而不修,其何祸福於人哉?

\hypertarget{header-n873}{%
\subsection{卷二十六}\label{header-n873}}

\hypertarget{header-n874}{%
\subsubsection{实知篇}\label{header-n874}}

儒者论圣人,以为前知千岁,後知万事,有独见之明,独听之聪,事来则名,不学自知,不问自晓,故称圣,则神矣。若蓍龟之知吉凶,蓍草称神,龟称灵矣。贤者才下不能及,智劣不能料,故谓之贤。夫名异则实殊,质同则称钧,以圣名论之,知圣人卓绝,与贤殊也。

孔子将死,遗谶书,曰:``不知何一男子,自谓秦始皇,上我之堂,踞我之床,颠倒我衣裳,至沙丘而亡。''其後,秦王兼吞天下,号始皇,巡狩至鲁,观孔子宅,乃至沙丘,道病而崩。又曰:``董仲舒乱我书。''其後,江都相董仲舒,论思《春秋》,造著传记。又书曰:``亡秦者,胡也。''其後,二世胡亥,竟亡天下。用三者论之,圣人後知万世之效也。孔子生不知其父,若母匿之,吹律自知殷宋大夫子氏之世也。不案图书,不闻人言,吹律精思,自知其世,圣人前知千岁之验也。

曰:此皆虚也。案神怪之言,皆在谶记,所表皆效图书。``亡秦者胡'',《河图》之文也。孔子条暢增益以表神怪,或後人诈记,以明效验。高皇帝封吴王,送之,拊其背曰:``汉後五十年,东南有反者,岂汝邪?''到景帝时,濞与七国通谋反汉。建此言者,或时观气见象,处其有反,不知主名。高祖见濞之勇,则谓之是。原此以论,孔子见始皇、仲舒,或时但言``将有观我之宅''、``乱我之书''者,後人见始皇入其宅,仲舒读其书,则增益其辞,著其主名。如孔子神而空见始皇、仲舒,则其自为殷後子氏之世,亦当默而知之,无为吹律以自定也。孔子不吹律,不能立其姓,及其见始皇,睹仲舒,亦复以吹律之类矣。案始皇本事,始皇不至鲁,安得上孔子之堂,踞孔子之床,颠倒孔子之衣裳乎?始皇三十七年十月癸丑出游,至云梦,望祀虞舜於九嶷。浮江下,观藉柯,度梅渚,过丹阳,至钱唐,临浙江,涛恶,乃西百二十里,从陕中度,上会稽,祭大禹,立石刊颂,望於南海。还过,从江乘,旁海上,北至琅邪。自琅邪北至劳、成山,因至之罘,遂并海,西至平原津而病,崩於沙丘平台。既不至鲁,谶记何见,而云始皇至鲁?至鲁未可知,其言孔子曰``不知何一男子''之言,亦未可用。``不知何一男子''之言不可用,则言``董仲舒乱我书''亦复不可信也。行事,文记谲常人言耳,非天地之书,则皆缘前因古,有所据状。如无闻见,则无所状。凡圣人见祸福也,亦揆端推类,原始见终,从闾巷论朝堂,由昭昭察冥冥。谶书秘文,远见未然,空虚暗昧,豫睹未有,达闻暂见,卓谲怪神,若非庸口所能言。

放象事类以见祸,推原往验以处来事,〔贤〕者亦能,非独圣也。周公治鲁,太公知其後世当有削弱之患;太公治齐,周公睹其後世当有劫弑之祸。见法术之极,睹祸乱之前矣。纣作象箸而箕子讥,鲁以偶人葬而孔子叹,缘象箸见龙干之患,偶人睹殉葬之祸也。太公、周公俱见未然,箕子、孔子并睹未有,所由见方来者,贤圣同也。鲁侯老,太子弱,次室之女倚柱而啸,由老弱之徵,见败乱之兆也。妇人之知,尚能推类以见方来,况圣人君子,才高智明者乎!秦始皇十年,庄襄王母夏太后薨,孝文王后曰华阳后,与文王葬寿陵,夏太后〔子〕〔庄〕襄王葬於〔芷阳〕,故夏太后别葬杜陵,曰:``东望吾子,西望吾夫,後百年,旁当有万家邑。''其後皆如其言。必以推类见方来为圣,次室、夏太后圣也。秦昭王十年,樗里子卒,葬於渭南章台之东,曰:``後百年,当有天子宫挟我墓。''至汉兴,长乐宫在其东,未央宫在其西,武库正值其墓,竟如其言。先知之效,见方来之验也。如以此效圣,樗里子圣人也。如非圣人,先知见方来不足以明圣。然则樗里子见天子宫挟其墓也,亦犹辛有知伊川之当戎。昔辛有过伊川,见被发而祭者,曰:``不及百年,此其戎乎!''其後百年,晋迁陆浑之戎於伊川焉,竟如〔其言〕。辛有之知当戎,见被发之兆也。樗里子之见天子〔宫〕挟其墓,亦见博平之〔基〕也。韩信葬其母,亦行营高敞地,令其旁可置万家。其後竟有万家处其墓旁。故樗里子之见博平〔土〕有宫台之兆,犹韩信之睹高敞万家之台也。先知之见,方来之事,无达视洞听之聪明,皆案兆察迹,推原事类。春秋之时,卿大夫相与会遇,见动作之变,听言谈之诡,善则明吉祥之福,恶则处凶妖之祸。明福处祸,远图未然,无神怪之知,皆由兆类。以今论之,故夫可知之事者,思虑所能见也;不可知之事,不学不问不能知也。不学自知,不问自晓,古今行事,未之有也。夫可知之事,推精思之,虽大无难;不可知之事,历心学问,虽小无易。故智能之士,不学不成,不问不知。

难曰:夫项托年七岁教孔子。案七岁未入小学而教孔子,性自知也。孔子曰:``生而知之,上也。学而知之,其次也。''夫言生而知之,不言学问,谓若项托之类也。王莽之时,勃海尹方年二十一,无所师友,性智开敏,明达六艺。魏都牧淳於仓奏:``方不学,得文能读诵,论义引《五经》文,文说议事,厌合人之心。''帝征方,使射蜚虫,筴射无〔弗〕知者,天下谓之圣人。夫无所师友,明达六艺,本不学书,得文能读,此圣人也。不学自能,无师自达,非神如何?

曰:虽无师友,亦已有所问受矣;不学书,已弄笔墨矣。兒始生产,耳目始开,虽有圣性,安能有知?项托七岁,其三四岁时,而受纳人言矣。尹方年二十一,其十四五时,多闻见矣。性敏才茂,独思无所据,不睹兆象,不见类验,却念百世之後,有马生牛,牛生驴,桃生李,李生梅,圣人能知之乎?臣弑君,子轼父,仁如颜渊,孝如曾参,勇如贲、育,辩如赐、予,圣人能见之乎?孔子曰:``其或继周者,虽百世可知也。''又曰:``後生可畏,焉知来者之不如今也?''论损益,言``可知'',称後生,言``焉知''。後生难处,损益易明也。此尚为远,非所听察也。使一人立於墙东,令之出声,使圣人听之墙西,能知其黑白、短长、乡里、姓字所自从出乎?沟有流澌,泽有枯骨,发首陋亡,肌肉腐绝,使〔圣〕人询之,能知其农商、老少、若所犯而坐死乎?非圣人无知,其知无以知也。知无以知,非问不能知也。不能知,则贤圣所共病也。

难曰:``詹何坐,弟子侍,有牛鸣於门外。弟子曰:`是黑牛也,而白蹄。'詹何曰:`然。是黑牛也'而白其蹄。使人视之,果黑牛而以布裹其蹄。詹何,贤者也,尚能听声而知其色。以圣人之智,反不能知乎?''

曰:能知黑牛白其蹄,能知此牛谁之牛乎?白其蹄者以何事乎?夫术数直见一端,不能尽其实。虽审一事,曲辩问之,辄不能尽知。何则?不目见口问,不能尽知也。鲁僖公二十九年,介葛卢来朝,舍於昌衍之上,闻牛鸣,曰:``是牛生三牺,皆已用矣。''或问:``何以知之?''曰:``其音云。''人问牛主,竟如其言。此复用术数,非知所能见也。广汉杨翁仲〔能〕听鸟兽之音,乘蹇马之野,田间有放眇马〔者〕,相去〔数里〕,鸣声相闻。翁仲谓其御曰:``彼放马目眇。''其御曰:``何以知之?''曰:``骂此辕中马蹇,此马亦骂之眇。''其御不信,往视之,目竟眇焉。翁仲之知马声,犹詹何、介葛卢之听牛鸣也。据术任数,相合其意,不达视听,遥见流目以察之也。夫听声有术,则察色有数矣。推用术数,若先闻见,众人不知,则谓神圣。若孔子之见兽,名之曰狌々,太史公之见张良,似妇人之形矣。案孔子未尝见狌々,至辄能名之,太史公与张良异世,而目见其形。使众人闻此言,则谓神而先知。然而孔子名狌々,闻《昭人之歌》;太史公之见张良,观宣室之画也。阴见默识,用思深秘。众人阔略,寡所意识,见贤圣之名物,则谓之神。推此以论,詹何见黑牛白蹄,犹此类也。彼不以术数,则先时闻见於外矣。方今占射事之工,据正术数,术数不中,集以人事。人事於术数而用之者,与神无异。詹何之徒,方今占射事者之类也。如以詹何之徒,性能知之,不用术数,是则巢居者先知风,穴处者先知雨。智明早成,项托、尹方其是也。

难曰:``黄帝生而神灵,弱而能言。帝喾生而自言其名。未有闻见於外,生辄能言,称其名,非神灵之效,生知之验乎?''

曰:黄帝生而言,然而母怀之二十月生,计其月数,亦已二岁在母身中矣。帝喾能自言其名,然不能言他人之名,虽有一能,未能遍通。所谓神而生知者,岂谓生而能言其名乎?乃谓不受而能知之,未得能见之也。黄帝、帝喾虽有神灵之验,亦皆早成之才也。人才早成,亦有晚就,虽未就师,家问室学。人见其幼成早就,称之过度。云项托七岁,是必十岁,云教孔子,是必孔子问之。云黄帝、帝喾生而能言,是亦数月。云尹方年二十一,是亦且三十。云无所师友,有不学书,是亦游学家习。世俗褒称过实,毁败愈恶。世俗传颜渊年十八岁升太山,望见吴昌门外有系白马。定考实,颜渊年三十不升太山,不望吴昌门。项托之称,尹方之誉,颜渊之类也。

人才有高下,知物由学。学之乃知,不问不识。子贡曰:``夫子焉不学,而亦何常师之有?''孔子曰:``吾十有五而志乎学。''五帝、三王,皆有所师。曰:``是欲为人法也''。曰:精思亦可为人法。何必以学者?事难空知,贤圣之才能立也。所谓神者,不学而知。所谓圣者,须学以圣。以圣人学,知其非圣。天地之间,含血之类,无性知者。

狌々知徃,鳱鹊知来,禀天之性,自然者也。如以圣人为若狌々乎?则夫狌々之类,鸟兽也。僮谣不学而知,可谓神而先知矣。如以圣人为若僮谣乎?则夫僮谣者,妖也。世间圣神,以为巫与?鬼神用巫之口告人。如以圣人为若巫乎?则夫为巫者,亦妖也。与妖同气,则与圣异类矣。巫与圣异,则圣不能神矣。不能神,则贤之党也。同党,则所知者无以异也。及其有异,以入道也。圣人疾,贤者迟;贤者才多,圣人智多。所知同业,多少异量;所道一途,步驺相过。

事有难知易晓,贤圣所共关思也。若夫文质之复,三教之重,正朔相缘,损益相因,贤圣所共知也。古之水火,今之水火也。今之声色,后世之声色也。鸟兽草木,人民好恶,以今而见古,以此而知来。千岁之前,万世之后,无以异也。追观上古,探察来世,文质之类,水火之辈,贤圣共之。见兆闻象,图画祸福,贤圣共之。见怪名物,无所疑惑,贤圣共之。事可知者,贤圣所共知也;不可知者,圣人亦不能知也。何以明之?使圣空坐先知雨也,性能一事知远道,孔窍不普,未足以论也。所论先知性达者,尽知万物之性,毕睹千道之要也。如知一不通二,达左不见右,偏驳不纯,踦校不具,非所谓圣也。如必谓之圣,是明圣人无以奇也。詹何之徒圣,孔子之党亦称圣,是圣无以异於贤,贤无以乏於圣也。贤圣皆能,何以称圣奇於贤乎?如俱任用术数,贤何以不及圣?

实者,圣贤不能性〔知〕,须任耳目以定情实。其任耳目也,可知之事,思之辄决;不可知之事,待问乃解。天下之事,世间之物,可思而〔知〕,愚夫能开精;不可思而知,上圣不能省。孔子曰:``吾尝终日不食,终夜不寝以思,无益,不如学也。''天下事有不可知,犹结有不可解也。见说善解结,结无有不可解。结有不可解,见说不能解也。非见说不能解也,结有不可解。及其解之,用不能也。圣人知事,事无不可知。事有不可知,圣人不能知,非圣人不能知,事有不可知。及其知之,用不知也。故夫难知之事,学问所能及也;不可知之事,问之学之,不能晓也。

\hypertarget{header-n878}{%
\subsubsection{知实篇}\label{header-n878}}

凡论事者,违实不引效验,则虽甘义繁说,众不见信。论圣人不能神而先知,先知之间,不能独见,非徒空说虚言,直以才智准况之工也。事有证验,以效实然。何以明之?

孔子问公叔文子於公明贾曰:``信乎,夫子不言、不笑、不取,有诸?''对曰:``以告者过也。夫子时然后言,人不厌其言;乐然后笑,人不厌其笑;义然后取,人不厌其取。''孔子曰:``岂其然乎?岂其然乎?''天下之人,有如伯夷之廉,不取一芥於人,未有不言、不笑者也。孔子既不能如心揣度,以决然否,心怪不信,又不能达视遥见,以审其实,问公明贾乃知其情。孔子不能先知,一也。

陈子禽问子贡曰:``夫子至於是邦也,必闻其政。求之与?抑与之与?''子贡曰:``夫子温良恭俭让以得之。''温良恭俭让,尊行也。有尊行於人,人亲附之。人亲附之,则人告语之矣。然则孔子闻政以人言,不神而自知之也。齐景公问子贡曰:``夫子贤乎?''子贡对曰:``夫子乃圣,岂徒贤哉!''景公不知孔子圣,子贡正其名。子禽亦不知孔子所以闻政,子贡定其实。对景公云``夫子圣,岂徒贤哉'',则其对子禽,亦当云``神而自知之,不闻人言''。以子贡对子禽言之,圣人不能先知,二也。

颜渊炊饭,尘落甑中,欲置之则不清,投地则弃饭,掇而食之。孔子望见以为窃食。圣人不能先知,三也。

涂有狂夫,投刃而候;泽有猛虎,厉牙而望。知见之者,不敢前进。如不知见,则遭狂夫之刃,犯猛虎之牙矣。匡人之围孔子,孔子如审先知,当早易道,以违其害。不知而触之,故遇其患。以孔子围言之,圣人不能先知,四也。

子畏於匡,颜渊后,孔子曰:``吾以汝为死矣。''如孔子先知,当知颜渊必不触害,匡人必不加悖。见颜渊之来,乃知不死;未来之时,谓以为死。圣人不能先知,五也。

阳货欲见孔子,孔子不见,馈孔子豚。孔子时其亡也,而往拜之,遇诸涂。孔子不欲见,既往,候时其亡,是势必不欲见也。反,遇於路。以孔子遇阳虎言之,圣人不能先知,六也。

长沮、桀溺偶而耕。孔子过之,使子路问津焉。如孔子知津,不当更问。论者曰:``欲观隐者之操''。则孔子先知,当自知之,无为观也。如不知而问之,是不能先知,七也。

孔子母死,不知其父墓,殡於五甫之衢。人见之者,以为葬也。盖以无所合葬,殡之谨,故人以为葬也。邻人邹曼甫之母告之,然後得合葬於防。有茔自在防,殡於衢路,圣人不能先知,八也。

既得合葬,孔子反,门人后,雨甚至。孔子问曰:``何迟也?''曰:``防墓崩。''孔子不应。孔子泫然流涕曰:``吾闻之,古不修墓。''如孔子先知,当先知防墓崩,比门人至,宜流涕以俟之。〔门〕人至乃知之,圣人不能先知,九也。子入太庙,每事问。不知故问,为人法也。孔子未尝入庙,庙中礼器,众多非一。孔子虽圣,何能知之?□□□:``以尝见,实已知,而复问,为人法。''孔子曰:``疑思问。''疑乃当问也?实已知,当复问,为人法。孔子知《五经》,门人从之学,当复行问,以为人法,何故专口授弟子乎?不以已知《五经》,复问为人法,独以已知太庙复问为人法,圣人用心,何其不一也?以孔子入太庙言之,圣人不能先知,十也。

主人请宾饮食,宾顿若舍。宾如闻其家有轻子〔泊〕孙,必教亲彻馔退膳,不得饮食;闭馆关舍,不得顿〔宾〕。宾之执计,则必不往。何则?知请呼无喜,空行劳辱也。如往无喜,劳辱复还,不知其家,不晓其实。人实难知,吉凶难图。如孔子先知,宜知诸侯惑於谗臣,必不能用,空劳辱己,聘召之到,宜寝不往。君子不为无益之事,不履辱身之行。无为周流应聘,以取削迹之辱;空说非主,以犯绝粮之厄。由此言之,近不能知。论者曰:``孔子自知不用,圣思闵道不行,民在涂炭之中,庶几欲佐诸侯,行道济民,故应聘周流,不避患耻。为道不为己,故逢患而不恶。为民不为名,故蒙谤而不避。''曰:此非实也。孔子曰:``吾自卫反鲁,然后乐正,雅颂各得其所。''是谓孔子自知时也。何以自知?鲁、卫,天下最贤之国也。鲁、卫不能用己,则天下莫能用己也,故退作《春秋》,删定《诗》、《书》。以自卫反鲁言之,知行应聘时,未自知也。何则?无兆象效验,圣人无以定也。鲁、卫不能用,自知极也;鲁人获麟,自知绝也。道极命绝,兆象著明,心怀望沮,退而幽思。夫周流不休,犹病未死,祷卜使痊也;死兆未见,冀得活也。然则应聘,未见绝证,冀得用也。死兆见舍,卜还医绝,揽笔定书。以应聘周流言之,圣人不能先知,十一也。

孔子曰:``游者可为纶。走这可为矰。至於龙,吾不知,其乘云风上升。今日见老子,其犹龙邪!''圣人知物知事。老子与龙,人、物也,所从上下,事也,何故不知?如老子神,龙亦神,圣人亦神。神者同道,精气交连,何故不知?以孔子不知龙与老子言之,圣人不能先知,十二也。

孔子曰:``孝哉,闵子骞!人不间於其父母昆弟之言。''虞舜大圣,隐藏骨肉之过,宜愈子骞。瞽叟与象,使舜治禀、浚井,意欲杀舜。当见杀己之情,早谏豫止。既无如何,宜避不行,若病不为。何故使父与弟得成杀己之恶,使人闻非父弟,万世不灭?以虞舜不豫见,圣人不能先知,十三也。

武王不豫,周公请命,坛墠既设,筴祝已毕,不知天之许己与不,乃卜三龟,三龟皆吉。如圣人先知,周公当知天已许之,无为顿复卜三龟。知圣人不以独见立法,则更请命,秘藏不见,天意难知,故卜而合兆,兆决心定,乃以从事。圣人不能先知,十四也。

晏子聘於鲁,堂上不趋,晏子趋;授玉不跪。晏子跪。门人怪而问於孔子。孔子不知,问於晏子。晏子解之,孔子乃晓。圣人不能先知,十五也。

陈贾问於孟子曰:``周公何人也?''曰:``圣人。''``使管叔监殷,管叔畔也。二者有诸?''曰:``然。''``周公知其畔而使,不知而使之与?''曰:``不知也。''``然则圣人且有过与?''曰:``周公,弟也,管叔,兄也。周公之过也,不亦宜乎!''孟子,实事之人也,言周公之圣,处其下,不能知管叔之畔。圣人不能先知,十六也。

孔子曰:``赐不受命,而货殖焉,亿则屡中。''罪子贡善居积,意贵贱之期,数得其时,故货殖多,富比陶硃。然则圣人先知也,子贡亿数中之类也。圣人据象兆,原物类,意而得之。其见变名物,博学而识之。巧商而善意,广见而多记,由微见较,若揆之今睹千载,所谓智如渊海。孔子见窍睹微,思虑洞达,材智兼倍,强力不倦,超逾伦等耳!目非有达视之明,知人所不知之状也。使圣人达视远见,洞听潜闻,与天地谈,与鬼神言,知天上地下之事,乃可谓神而先知,与人卓异。今耳目闻见,与人无别,遭事睹物,与人无异,差贤一等尔,何以谓神而卓绝?

夫圣犹贤也,人之殊者谓之圣,则圣贤差小大之称,非绝殊之名也。何以明之?齐桓公与管仲谋伐莒,谋未发而闻於国,桓公怪之,问管仲曰:``与仲甫谋伐莒,未发,闻於国,其故何也?''管仲曰:``国必有圣人也。''少顷,当东郭牙至。管仲曰:``此必是已。''乃令宾延而上之,分级而立。管〔仲〕曰:``子邪,言伐莒?''对曰:``然。''管仲曰:``我不伐莒,子何故言伐莒?''对曰:``臣闻君子善谋,小人善意。臣窃意之。''管仲曰:``我不言伐莒,子何以意之?''对曰:``臣闻君子有三色:驩然喜乐者,钟鼓之色;愁然清净者,衰绖之色;怫然充满手足者,兵革之色。君口垂不〔吟〕,所言莒也;君举臂而指,所当又莒也。臣窃虞国小诸侯不服者,其唯莒乎!臣故言之。''夫管仲,上智之人也,其别物审事矣。云``国必有圣人''者,至诚谓国必有也。东郭牙至,云``此必是已'',谓东郭牙圣也。如贤与圣绝辈,管仲知时无十二圣之党,当云``国必有贤者'',无为言圣也。谋未发而闻於国,管仲谓``国必有圣人'',是谓圣人先知也。及见东郭牙,云``此必是已'',谓贤者圣也。东郭牙知之审,是与圣人同也。

客有见淳于髡於梁惠王者,再见之,终无言也。惠王怪之,以让客曰:``子之称淳于生,言管、晏不及。及见寡人,寡人未有得也。寡人未足为言邪?''客谓髡。〔髡〕曰:``固也!吾前见王志在远,后见王志在音,吾是以默然。''客具报,王大骇曰:``嗟乎!淳于生诚圣人也?前淳于生之来,人有献龙马者,寡人未及视,会生至。后来,人有献讴者,为及试,亦会生至。寡人虽屏左右,私心在彼。''夫髡之见惠王在远与音也,虽汤、禹之察,不能过也。志在胸臆之中,藏匿不见,髡能知之。以髡等为圣,则髡圣人也。如以髡等非圣,则圣人之知,何以过髡之知惠王也?观色以窥心,皆有因缘以准的之。

楚灵王会诸侯,郑子产曰:``鲁、邾、宋、卫不来。''及诸侯会,四国果不至。赵尧为符玺御史,赵人方与公谓御史大夫周昌曰:``君之史赵尧且代君位。''其后尧果为御史大夫。然则四国不至,子产原其理也;赵尧之为御史大夫,方与公睹其状也。原理睹状,处著方来,有以审之也。鲁人公孙臣,孝文皇帝时,上书言汉土德,其符黄龙当见。後黄龙见成纪。然则公孙臣知黄龙将出,案律历以处之也。

贤圣之知,事宜验矣。贤圣之才,皆能先知;其先知也,任术用数,或善商而巧意,非圣人空知。神怪与圣贤,殊道异路也。圣贤知不逾,故用思相出入;遭事无神怪,故名号相贸易。故夫贤圣者,道德智能之号;神者,眇茫恍惚无形之实。实异,质不得同;实钧,效不得殊。圣神号不等,故谓圣者不神,神者不圣。东郭牙善意,以知国情,子贡善意,以得货利。圣人之先知,子贡、东郭牙之徒也。与子贡、东郭同,则子贡、东郭之徒亦圣也。夫如是,圣贤之实同而名号殊,未必才相悬绝,智相兼倍也。

太宰问於子贡曰:``夫子圣者欤?何其多能也!''子贡曰:``故天纵之将圣,又多能也。''将者,且也。不言已圣,言且圣者,以为孔子圣未就也。夫圣若为贤矣,治行厉操,操行未立,则谓且贤。今言且圣,圣可为之故也。孔子曰:``吾十有五而志於学,三十而立,四十而不惑,五十而知天命,六十而耳顺。''从知天命至耳顺,学就知明,成圣之验也。未五十、六十之时,未能知天命、至耳顺也,则谓之``且''矣。当子贡答太宰时,殆三十、四十之时也。

魏昭王问於田诎曰:``寡人在东宫之时,闻先生之议曰`为圣易'有之乎?''田诎对曰:``臣之所学也。''昭王曰:``然则先生圣乎?''田诎曰:``未有功而知其圣者,尧之知舜也。待其有功而後知圣者,市人之知舜也。今诎未有功,而王问诎曰:``若圣乎?敢问王亦其尧乎?''夫圣可学为,故田诎谓之易。如卓与人殊,禀天性而自然,焉可学?而为之安能成?田诎之言``为圣易'',未必能成,田诎之言为易,未必能是;言``臣之所学'',盖其实也。

〔圣〕可学,为劳佚殊,故贤圣之号,仁智共之。子贡问於孔子:``夫子圣矣乎?''孔子曰:``圣则吾不能。我学不餍,而教不倦。''子贡曰:``学不餍者,智也;教不倦者,仁也。仁且智,夫子既圣矣。''由此言之,仁智之人,可谓圣矣。孟子曰:``子夏、子游、子张得圣人之一体,冉牛、闵子骞、颜渊具体而微。''六子在其世,皆有圣人之才,或颇有而不具,或备有而不明,然皆称圣人,圣人可勉成也。孟子又曰:``非其君不事,非其民不使,治则进,乱则退,伯夷也。何事非君,何使非民,治亦进,乱亦进,伊尹也。可以仕则仕,可以已则已,可以久则久,可以速则速,孔子也。皆古之圣人也。''又曰:``圣人,百世之师也,伯夷、柳下惠是也。故闻伯夷之风者,顽夫廉,懦夫有立志;闻柳下惠之风者,薄夫敦,鄙夫宽。奋乎百世之上,百世之下闻之者,莫不兴起,非圣而若是乎?而况亲炙之乎?''夫伊尹、伯夷、柳下惠不及孔子,而孟子皆曰``圣人''者,贤圣同类,可以共一称也。宰予曰:``以予观夫子,贤於尧、舜远矣。''孔子圣,宜言圣於尧、舜,而言贤者,圣贤相出入,故其名称相贸易也。

\hypertarget{header-n883}{%
\subsection{卷二十七}\label{header-n883}}

\hypertarget{header-n884}{%
\subsubsection{定贤篇}\label{header-n884}}

圣人难知,贤者比於圣人为易知。世人且不能知贤,安能知圣乎?世人虽言知贤,此言妄也。知贤何用?知之如何?

以仕宦得高官身富贵为贤乎?则富贵者天命也。命富贵不为贤,命贫贱不为不肖。必以富贵效贤不肖,是则仕宦以才不以命也。

以事君调合寡过为贤乎?夫顺阿之臣,佞幸之徒是也。准主而说,适时而行,无廷逆之郄,则无斥退之患。或骨体嫺丽,面色称媚,上不憎而善生,恩泽洋溢过度,未可谓贤。

以朝庭选举皆归善为贤乎?则夫著见而人所知者举多,幽隐人所不识者荐少,虞舜是也。尧求,则咨於鲧、共工,则岳已不得。由此言之,选举多少,未可以知实。或德高而举之少,或才下而荐之多。明君求善察恶於多少之间,时得善恶之实矣。且广交多徒,求索众心者,人爱而称之;清直不容乡党,志洁不交非徒,失众心者,人憎而毁之。故名多生於知谢,毁多失於众意。齐威王以毁封即墨大夫,以誉烹阿大夫。即墨有功而无誉,阿无效而有名也。子贡问曰:``乡人皆好之,何如?''孔子曰:``未可也。''``乡人皆恶之,何如?''曰:``未可也,不若乡人之善者好之,其不善者恶之。''夫如是,称誉多而小大皆言善者,非贤也。善人称之,恶人毁之,毁誉者半,乃可有贤。

以善人所称,恶人所毁,可以知贤乎?夫如是,孔子之言可以知贤,不知誉此人者贤也?毁此人者恶也?或时称者,恶而毁者善也?人眩惑无别也。

以人众所归附、宾客云合者为贤乎?则夫人众所附归者,或亦广交多徒之人也,众爱而称之,则蚁附而归之矣。或尊贵而为利,或好士下客,折节俟贤。信陵、孟尝、平原、春申,食客数千,称为贤君。大将军卫青及霍去病门无一客,称为名将。故宾客之会,在好下之君。利害之贤,或不好士,不能为轻重,则众不归而士不附也。

以居位治人,得民心歌咏之为贤乎?则夫得民心者,与彼得士意者,无以异也。为虚恩拊循其民,民之欲得,即喜乐矣。何以效之?齐田成子、越王勾践是也。成子欲专齐政,以大斗贷、小斗收而民悦。句践欲雪会稽之耻,拊循其民,吊死问病而民喜。二者皆自有所欲为於他,而伪诱属其民,诚心不加,而民亦说。孟尝君夜出秦关,鸡未鸣而关不,下坐贱客,鼓臂为鸡鸣,而鸡皆和之,关即启,而孟尝得出。〔夫〕鸡可以奸声感,则人亦可以伪恩动也。人可以伪恩动,则天亦可巧诈应也。动致天气,宜以精神,而人用阳燧取火於天,消炼五石,五月盛夏铸以为器,乃能得火。今又但取刀剑铜钩之属,切磨以向日,亦得火焉。夫阳燧、刀、剑、钩能取火於日,恆非贤圣亦能动气於天。若董仲舒信土龙之能致云雨,盖亦有以也。夫如是,应天之治,尚未可谓贤,况徒得人心,即谓之贤,如何?

以居职有成功见效为贤乎?夫居职何以为功效?以人民附之,则人民可以伪恩说也。阴阳和、百姓安者,时也。时和,不肖遭其安;不和,虽圣逢其危。如以阴阳和而效贤不肖,则尧以洪水得黜,汤以大旱为殿下矣。如功效谓事也,身为之者,功著可见。以道为计者,效没不章。鼓无当於五音,五音非鼓不和。师无当於五服,五服非师不亲。水无当於五采,五采非水不章。道为功本,功为道效,据功谓之贤,是则道人之不肖也。高祖得天下,赏群臣之功,萧何为赏首。何则?高祖论功,比猎者之纵狗也。狗身获禽,功归於人。群臣手战,其犹狗也;萧何持重,其犹人也。必据成功谓之贤,是则萧何无功。功赏不可以效贤,一也。

夫圣贤之治世也有术,得其术则功成,失其术则事废。譬犹医之治病也,有方,笃剧犹治;无方,才微不愈。夫方犹术,病犹乱,医犹吏,药犹教也。方施而药行,术设而教从,教从而乱止,药行而病愈。治病之医,未必惠於不为医者。然而治国之吏,未必贤於不能治国者,偶得其方,遭晓其术也。治国须术以立功,亦有时当自乱,虽用术,功终不立者;亦有时当自安,虽无术,功犹成者。故夫治国之人,或得时而功成,或失时而无效。术人能因时以立功,不能逆时以致安。良医能治未当死之人命,如命穷寿尽,方用无验矣。故时当乱也,尧、舜用术,不能立功;命当死矣,扁鹊行方,不能愈病。射御巧技,百工之人,皆以法术,然后功成事立,效验可见。观治国,百工之类也;功立,犹事成也。谓有功者贤,是谓百工皆贤人也。赵人吾丘寿王,武帝时待诏,上使从董仲舒受《春秋》,高才,通明於事后为东郡都尉。上以寿王之贤,不置太守。时军发,民骚动,岁恶,盗贼不息。上赐寿王书曰:``子在朕前时,辐凑并至,以为天下少双,海内寡二,至连十余城之势,任四千石之重,而盗贼浮船行功取於库兵,甚不称在前时,何也?''寿王谢言难禁。复召为光禄大夫,常居左右,论事说议,无不是者,才高智深,通明多见。然其为东郡尉,岁恶,盗贼不息,人民骚动,不能禁止。不知寿王不得治东郡之术邪?亡将东郡适当复乱,而寿王之治偶逢其时也?夫以寿王之贤,治东郡不能立功,必以功观贤,则寿王弃而不选也。恐必世多如寿王之类,而论者以无功不察其贤。燕有谷,气寒不生五谷。邹衍吹律致气,既寒更为温,燕以种黍,黍生丰熟,到今名之曰``黍谷''。夫和阴阳,当以道德至诚。然而邹衍吹律,寒更为温,黍谷育生。推此以况诸有成功之类,有若邹衍吹律之法。故得其术也,不肖无不能;失其数也,贤圣有不治。此功不可以效贤,二也。

人之举事,或意至而功不成,事不立而势贯山。荆轲、医夏无且是矣。荆轲入秦之计,本欲劫秦王生致於燕,邂逅不偶,为秦所擒。当荆轲之逐秦王,秦王环柱而走,医夏无且以药囊提荆轲。既而天下名轲为烈士,秦王赐无且金二百镒。夫为秦所擒,生致之功不立,药囊提刺客,〔无〕益於救主,然犹称赏者,意至势盛也。天下之士不以荆轲功不成,不称其义,秦王不以无且无见效,不赏其志。志善不效成功,义至不谋就事。义有余,效不足,志巨大,而功细小,智者赏之,愚者罚之。必谋功不察志,论阳效不存阴计,是则豫让拔剑斩襄子之衣,不足识也;伍子胥鞭笞平王尸,不足载也;张良椎始皇误中副车,不足记也。三者道地不便,计画不得,有其势而无其功,怀其计而不得为其事,是功不可以效贤,三也。

以孝於父、弟於为兄贤乎?则夫孝弟之人,有父兄者也,父兄不慈,孝弟乃章。舜有瞽瞍,参有曾皙,孝立名成,众人称之。如无父兄,父兄慈良,无章显之效,孝弟之名,无所见矣。忠於君者,亦与此同。龙逢、比干忠著夏、殷,桀、纣恶也。稷、契、皋陶忠暗唐、虞,尧、舜贤也。故萤火之明,掩於日月之光;忠臣之声,蔽於贤君之名。死君之难,出命捐身,与此同。臣遭其时死其难,故立其义而获其名。大贤之涉世也,翔而有集,色斯而举;乱君之患,不累其身;危国之祸,不及其家,安得逢其祸而死其患乎?齐詹问於晏子曰:``忠臣之事其君也,若何?''对曰:``有难不死,出亡不送。''詹曰:``列地而予之,疏爵而贵之,君有难不死,出亡不送,可谓忠乎?''对曰:``言而见用,臣奚死焉?谏而见从,终身不亡,臣奚送焉?若言不见用,有难而死,是妄死也;谏而不见从,出亡而送,是诈伪也。故忠臣者能尽善於君,不能与陷於难。''案晏子之对,以求贤於世,死君之难、立忠节者,不应科矣。是故大贤寡可名之节,小贤多可称之行,可得箠者小,而可得量者少也。恶至大,箠弗能;数至多,升斛弗能。有小少易名之行,又发於衰乱易见之世,故节行显而名声闻也。浮於海者迷於东西,大也。行於沟,咸识舟楫之迹,小也。小而易见,衰乱亦易察。故世不危乱,奇行不见;主不悖惑,忠节不立。鸿卓之义,发於颠沛之朝;清高之行,显於衰乱之世。

以全身免害,不被刑戳,若南容惧白圭者为贤乎?则夫免於害者幸,而命禄吉也,非才智所能禁,推行所能却也。神蛇能断而复属,不能使人弗断。圣贤能困而复通,不能使人弗害。南容能自免於刑戳,公冶以非罪在缧絏,伯玉可怀於无道之国,文王拘羑里,孔子厄陈、蔡,非行所致之难,掩己而至,则有不得自免之患,累己而滞矣。夫不能自免於患者,犹不能延命於世也。命穷,贤不能自续;时厄,圣不能自免。

以委国去位,弃富贵,就贫贱为贤乎?则夫委国者,有所迫也。若伯夷之徒,昆弟相让以国,耻有分争之名;及大王甫重战其民,亶皆委国去位者,道不行而志不得也。如道行志得,亦不去位。故委国去位,皆有以也,谓之为贤,无以者,可谓不肖乎?且有国位者,故得委而去之,无国位者何委?夫割财用及让下受分,有此同实。无财何割?口饥何让?仓廪实,民知礼节,衣食足知荣辱。让生於有余,争生於不足。人或割财助用,袁将军再与兄子分家财,以为恩义。昆山之下,以玉为石;彭蠡之滨,以鱼食犬豕。使推让之人,财若昆山之玉、彭蠡之鱼,家财再分,不足为也。韩信寄食於南昌亭长,何财之割?颜渊箪食瓢饮,何财之让?管仲分财取多,无廉让之节,贫乏不足,志义废也。

以避世离俗,清身洁行为贤乎?是则委国去位之类也。富贵人情所贪,高官大位人之所欲去之而隐,生不遭遇,志气不得也。长沮、桀溺避世隐居,伯夷、於陵去贵取贱,非其志也。

〔以〕恬无欲,志不在於仕,苟欲全身养性为贤乎?是则老聃之徒也。道人与贤〔者〕殊科者,忧世济民於难。是以孔子栖栖,墨子遑遑。不进与孔、墨合务,而还与黄、老同操,非贤也。

以举义千里,师将朋友无废礼为贤乎?则夫家富财饶,筋力劲强者能堪之。匮乏无以举礼,赢弱不能奔远,不能任也。是故百金之家,境外无绝交;千乘之国,同盟无废赠,财多故也。使谷食如水火,虽贪吝之人,越境而布施矣。故财少则正礼不能举一,有余则妄施能於千,家贫无斗筲之储者,难责以交施矣。举檐千里之人,材筴越疆之士,手足胼胝,面目骊黑,无伤感不任之疾,筋力皮革必有与人异者矣。推此以况为君要证之吏,身被疾痛而口无一辞者,亦肌肉骨节坚强之故也。坚强则能隐事而立义,软弱则诬时而毁节。豫让自贼,妻不能识;贯高被箠,身无完肉。实体有不与人同者,则其节行有不与人钧者矣。

以经明带徒聚众为贤乎?则夫经明,儒者是也。儒者,学之所为也。儒者学,学,儒矣。传先师之业,习口说以教,无胸中之造,思定然否之论。邮人之过书、门者之传教也,封完书不遗,教审令不误者,则为善矣。〔儒〕者传学,不妄一言,先师古语,到今具存,虽带徒百人以上,位博士、文学,邮人、门者之类也。

以通览古今,秘隐传记无所不记为贤乎?是则〔儒〕者之次也。才高好事,勤学不舍,若专成之苗裔,有世祖遗文,得成其篇业,观览讽诵。若典官文书,若太史公及刘子政之徒,有主领书记之职,则有博览通达之名矣。

以权诈卓谲,能将兵御众为贤乎?是韩信之徒也。战国获其功,称为名将;世平能无所施,还入祸门矣。高鸟死,良弓藏;狡兔得,良犬烹。权诈之臣,高鸟之弓,狡兔之犬也。安平身无宜,则弓藏而犬烹。安平之主,非弃臣而贱士,世所用助上者,非其宜也。向令韩信用权变之才,为若叔孙通之事,安得谋反诛死之祸哉?有功强之权,无守平之智,晓将兵之计,不见已定之义,居平安之时,为反逆之谋,此其所以功灭国绝,不得名为贤也。

〔以〕辩於口,言甘辞巧为贤乎?则夫子贡之徒是也。子贡之辩胜颜渊,孔子序置於下。实才不能高,口辩机利,人决能称之。夫自文帝尚多虎圈啬夫,少上林尉,张释之称周勃、张相如,文帝乃悟。夫辩於口,虎圈啬夫之徒也,难以观贤。

以敏於笔,文墨〔雨〕集为贤乎?夫笔之与口,一实也。口出以为言,笔书以为文。口辩,才未必高;然则笔敏,知未必多也。且笔用何为敏?以敏於官曹事。事之难者莫过於狱,狱疑则有请谳。盖世优者,莫过张汤,张汤文深,在汉之朝,不称为贤。太史公《序累》以汤为酷,酷非贤者之行。鲁林中哭妇,虎食其夫,又食其子,不能去者,山政不苛,吏不暴也。夫酷,苛暴之党也,难以为贤。

以敏於赋颂,为弘丽之文为贤乎?则夫司马长卿、扬子云是也。文丽而务巨,言眇而趋深,然而不能处定是非,辩然否之实。虽文如锦锈,深如河、汉,民不觉知是非之分,无益於弥为崇实之化。

以清节自守,不降志辱身为贤乎?是则避世离俗,长沮、桀溺之类也。虽不离俗,节与离世者钧,清其身而不辅其主,守其节而不劳其民。大贤之在世也,时行则行,时止则止,铨可否之宜,以制清浊之行。子贡让而止善,子路受而观德。夫让,廉也;受则贪也。贪有益,廉有损,推行之节,不得常清眇也。伯夷无可,孔子谓之非,操违於圣,难以为贤矣。

或问於孔子曰:``颜渊何人也?''曰:``仁人也,丘不如也。''``子贡何人也?''曰:``辩人也,丘弗如也。''``子路何人也?''曰:``勇人也,丘弗如也。''客曰:``三子者皆贤於夫子,而为夫子服役,何也?''孔子曰:``丘能仁且忍,辩且诎,勇且怯。以三子之能,易丘之道,弗为也。''孔子知所设施之矣。有高才洁行,无知明以设施之,则与愚而无操者同一实也。夫如是,皆有非也。无一非者,可以为贤乎?是则乡原之人也。孟子曰:``非之无举也,刺之,无刺也。同於流俗,合於污世,居之似忠信,行之似廉洁,众皆说之,自以为是,而不可与入尧、舜之道。''故孔子曰:`乡原,德之贼也。'似之而非者,孔子恶之。夫如是,何以知实贤?知贤竟何用?世人之检,苟见才高能茂,有成功见效,则谓之贤。若此甚易,知贤何难!《书》曰:``知人则哲,惟帝难之。''据才高卓异者,则谓之贤耳,何难之有?然而难之,独有难者之故也。

夫虞舜不易知人,而世人自谓能知贤,误也。然则贤者竟不可知乎?曰:易知也。而称难者,不见所以知之则难,圣人不易知也;及见所以知之,中才而察之。譬犹工匠之作器也,晓之则无难,不晓则无易。贤者易知於作器。世无别,故真贤集於俗士之间。俗士以辩惠之能,据官爵之尊,望显盛之宠,遂专为贤之名。贤者还在闾巷之间,贫贱终老,被无验之谤。若此,何时可知乎?然而必欲知之,观善心也。夫贤者,才能未必高也而心明,智力未必多而举是。何以观心?必以言。有善心,则有善言。以言而察行,有善言则有善行矣。言行无非,治家亲戚有伦,治国则尊卑有序。无善心者,白黑不分,善恶同伦,政治错乱,法度失平。故心善,无不善也;心不善,无能善。心善则能辩然否。然否之义定,心善之效明,虽贫贱困穷,功不成而效不立,犹为贤矣。故治不谋功,要所用者是;行不责效,期所为者正。正是审明,则言不须繁,事不须多。故曰:``言不务多,务审所谓。行不务远,务审所由。''言得道理之心,口虽讷不辩,辩在胸臆之内矣。故人欲心辩,不欲口辩。心辩则言丑而不违,口辨则辞好而无成。

孔子称少正卯之恶曰:``言非而博,顺非而泽。''内非而外以才能饬之,众不能见则以为贤。夫内非外饬是,世以为贤,则夫内是外无以自表者,众亦以为不肖矣。是非乱而不治,圣人独知之。人言行多若少正卯之类,贤者独识之。世有是非错缪之言,亦有审误纷乱之事,决错缪之言,定纷乱之事,唯贤圣之人为能任之。圣心明而不暗,贤心理而不乱。用明察非,非无不见;用理铨疑,疑无不定。与世殊指,虽言正是,终不晓见。何则?沉溺俗言之日久,不能自还以从实也。是故正是之言为众所非,离俗之礼为世所讥。管子曰;``君子言堂满堂,言室满室。''怪此之言,何以得满?如正是之言出,堂之人皆有正是之知,然後乃满。如非正是,人之乖刺异,安得为满?夫歌曲妙者,和者则寡;言得实者,然者则鲜。和歌与听言,同一实也。曲妙人不能尽和,言是人不能皆信。鲁文公逆祀,去者三人;定公顺祀,畔者五人。贯於俗者,则谓礼为非。晓礼者寡,则知是者希。君子言之,堂室安能满?夫人不谓之满,世则不得见口谈之实语,笔墨之余迹,陈在简筴之上,乃可得知。故孔子不王,作《春秋》以明意。案《春秋》虚文业,以知孔子能王之德。孔子,圣人也。有若孔子之业者,虽非孔子之才,斯亦贤者之实验也。夫贤与同轨而殊名,贤可得定,则圣可得论也。问:``周道不弊,孔子不作《春秋》。《春秋》之作,起周道弊也。如周道不弊,孔子不作者,未必无孔子之才,无所起也。夫如是,孔子之作《春秋》,未可以观圣;有若孔子之业者,未可知贤也。曰:周道弊,孔子起而作之,文义褒贬是非,得道理之实,无非僻之误,以故见孔子之贤,实也。夫无言,则察之以文;无文,则察之以言。设孔子不作,犹有遗言,言必有起,犹文之必有为也。观文之是非,不顾作之所起,世间为文者众矣,是非不分,然否不定,桓君山论之,可谓得实矣。论文以察实,则君山汉之贤人也。陈平未仕,割肉闾里,分均若一,能为丞相之验也。夫割肉与割文,同一实也。如君山得执汉平,用心与为论不殊指矣。孔子不王,素王之业在於《春秋》。然则桓君山〔不相〕,素丞相之迹,存於《新论》者也。

\hypertarget{header-n889}{%
\subsection{卷二十八}\label{header-n889}}

\hypertarget{header-n890}{%
\subsubsection{正说篇}\label{header-n890}}

儒者说《五经》,多失其实。前儒不见本末,空生虚说。后儒信前师之言,随旧述故,滑习辞语。苟名一师之学,趋为师教授,及时蚤仕,汲汲竟进,不暇留精用心,考实根核。故虚说传而不绝,实事没而不见,《五经》并失其实。《尚书》、《春秋》事较易,略正题目粗粗之说,以照篇中微妙之文。

说《尚书》者,或以为本百两篇,后遭秦燔《诗》、《书》,遗在者二十九篇。夫言秦燔《诗》、《书》,是也;言本百两篇者,妄也。盖《尚书》本百篇,孔子以授也。遭秦用李斯之议,燔烧《五经》,济南伏生抱百篇藏於山中。孝景皇帝时,始存《尚书》。伏生已出山中,景帝遣晁错往从受《尚书》二十余篇。伏生老死,《书》残不竟,晁错传於倪宽。至孝宣皇帝之时,河内女子发老屋,得逸《易》、《礼》、《尚书》各一篇,奏之。宣帝下示博士,然後《易》、《礼》、《尚书》各益一篇,而《尚书》二十九篇始定矣。至孝〔武〕帝时,鲁共王坏孔子教授堂以为殿,得百篇《尚书》於墙壁中。武帝使使者取视,莫能读者,遂秘於中,外不得见。至孝成皇帝时,征为古文《尚书》学。东海张霸案百篇之序,空造百两之篇,献之成帝。帝出秘百篇以校之,皆不相应,於是下霸於吏。吏白霸罪当至死,成帝高其才而不诛,亦惜其文而不灭。故百两之篇,传在世间者,传见之人则谓《尚书》本有百两篇矣。或言秦燔诗书者,燔《诗经》之书也,其经不燔焉。夫《诗经》独燔其诗。书,《五经》》之总名也。传曰:``男子不读经,则有博戏之心。''子路使子羔为费宰,孔子曰``贼夫人之子。''子路曰:``有民人焉,有社稷焉,何必读书,然後为学。''《五经》总名为书。传者不知秦燔书所起,故不审燔书之实。秦始皇三十四年,置酒咸阳宫,博士七十人前为寿。仆射周青臣进颂秦始皇。齐人淳于越进谏,以为始皇不封子弟,卒有田常、六卿之难,无以救也,讥青臣之颂,谓之为谀。秦始皇下其议丞相府,丞相斯以为越言不可用,因此谓诸生之言惑乱黔首,乃令史官尽烧《五经》,有敢藏诸〔诗〕书百家语者刑,唯博士官乃得有之。《五经》皆燔,非独诸〔诗〕家之书也。传者信之,见言诗书则独谓《〔诗〕经》之书矣。

传者或知《尚书》为秦所燔,而谓二十九,篇其遗脱不烧者也。审若此言,《尚书》二十九篇,火之余也。七十一篇为炭灰,二十九篇独遗邪?夫伏生年老,晁错从之学时,适得二十余篇。伏生死矣,故二十九篇独见,七十一篇遗脱。遗脱者七十一篇,反谓二十九篇遗脱矣。

或说《尚书》二十九篇者,法曰斗〔四〕七宿也。四七二十八篇,其一曰斗矣,故二十九。夫《尚书》灭绝於秦,其见在者二十九篇,安得法乎?宣帝之时,得佚《尚书》及《易》、《礼》各一篇,《礼》、《易》篇数亦始足,焉得有法?案百篇之序,阙遗者七十一篇,独为二十九篇立法,如何?或说曰:``孔子更选二十九篇,二十九篇独有法也。''盖俗儒之说也,未必传记之明也。二十九篇残而不足,有传之者,因不足之数,立取法之说,失圣人之意,违古今之实。夫经之有篇也,犹有章句〔也〕。有章句,犹有文字也。文字有意以立句,句有数以连章,章有体以成篇,篇则章句之大者也。谓篇有所法,是谓章句复有所法也。《诗经》旧时亦数千篇,孔子删去复重,正而存三百篇,犹二十九篇也。谓二十九篇有法,是谓三百五篇复有法也。

或说《春秋》十二月也。《春秋》十二公,犹《尚书》之百篇。百篇无所法,十二公安得法?说《春秋》者曰:``二百四十二年,人道浃,王道备,善善恶恶,拨乱世,反诸正,莫近於《春秋》。''若此者,人道、王道适具足也。三军六师万二千人,足以陵敌伐寇,横行天下,令行禁止,未必有所法也。孔子作《春秋》,纪鲁十二公,犹三军之有六师也;士众万二千,犹年有二百四十二也。六师万二千人,足以成军;十二公二百四十二年,足以立义。说事者好神道恢义,不肖以遭祸。是故经传篇数,皆有所法。考实根本,论其文义,与彼贤者作书诗,无以异也。故圣人所经,贤者作书,义穷理竟,文辞备足,则为篇矣。其立篇也,种类相从,科条相附。殊种异类,论说不同,更别为篇。意异则文殊,事改则篇更。据事意作,安得法象之义乎?

或说《春秋》二百四十二年者,上寿九十,中寿八十,下寿七十。孔子据中寿三世而作,三八二十四,故二百四十年也。又说为赤制之中数也。又说二百四十二年,人道浃,王道备。夫据三世,则浃备之说非;言浃备之说为是,则据三世之论误。二者相伐,而立其义,圣人之意何定哉?凡纪事言年月日者,详悉重之也。《洪范》五纪,岁、月、日、星。纪事之文,非法象之言也。纪十二公享国之年,凡有二百四十二,凡此以立三世之说矣。实孔子纪十二公者,以为十二公事,适足以见王义邪?据三世,三世之数,适得十二公而足也?如据十二公,则二百四十二年不为三世见也。如据三世,取三八之数,二百四十年而已,何必取二?说者又曰:``欲合隐公之元也,不敢二年。隐公元年,不载於经。''夫《春秋》自据三世之数而作,何用隐公元年之事为始?须隐公元年之事为始,是竟以备足为义,据三世之说不复用矣。说隐公享国五十年,将尽纪元年以来邪?中断以备三八之数也?如尽纪元年以来,三八之数则中断;如中断以备三世之数,则隐公之元不合,何如?且年与月日,小大异耳,其所纪载,同一实也。二百四十二年谓之据三世,二百四十二年中之日月必有数矣。年据三世,月日多少何据哉?夫《春秋》之有年也,犹《尚书》之有章。章以首义,年以纪事。谓《春秋》之年有据,是谓《尚书》之章亦有据也。

说《易》者皆谓伏羲作八卦,文王演为六十四。夫圣王起,河出图,洛出书。伏羲王,《河图》从河水中出,《易》卦是也。禹之时,得《洛书》,书从洛水中出,《洪范》九章是也。故伏义以卦治天下,禹案《洪范》以治洪水。古者烈山氏之王得河图,夏後因之曰《连山》;〔归藏〕氏之王得河图,殷人因之曰《归藏》;伏羲氏之王得河图,周人曰《周易》。其经卦皆六十四,文王、周公因彖十八章究六爻。世之传说《易》者,言伏羲作八卦;不实其本,则谓伏羲真作八卦也。伏羲得八卦,非作之;文王得成六十四,非演之也。演作之言,生於俗传。苟信一文,使夫真是几灭不存。既不知《易》之为河图,又不知存於俗何家《易》也,或时《连山》、《归藏》,或时《周易》。案礼夏、殷、周三家相损益之制,较著不同。如以周家在後,论今为《周易》,则礼亦宜为周礼。六典不与今礼相应,今礼未必为周,则亦疑今《易》未必为周也。案左丘明之传,引周家以卦,与今《易》相应,殆《周易》也。

说《礼》者,皆知礼也,礼〔为〕何家礼也?孔子曰:``殷因於夏礼,所损益可知也。周因於殷礼,所损益可知也。''由此言之,夏、殷、周各自有礼。方今周礼邪?夏、殷也?谓之周礼,《周礼》六典。案今《礼经》不见六典,或时殷礼未绝,而六典之礼不传,世因谓此为周礼也?案周官之法不与今礼相应,然则《周礼》六典是也。其不传,犹古文《尚书》、《春秋》,《左氏》不兴矣。

说《论》者,皆知说文解语而已,不知《论语》本几何篇,但周以八寸为尺,不知《论语》所独一尺之意。夫《论语》者,弟子共纪孔子之言行,敕记之时甚多,数十百篇,以八寸为尺,纪之约省,怀持之便也。以其遗非经,传文纪识恐忘,故以但八寸尺,不二尺四寸也。汉兴失亡,至武帝发取孔子壁中古文,得二十一篇,齐、鲁二,河间九篇:三十篇。至昭帝女读二十一篇。宣帝下太常博士,时尚称书难晓,名之曰传,後更隶写以传诵。初孔子孙孔安国以教鲁人扶卿,官至荆州刺史,始曰《论语》。今时称《论语》二十篇,又失齐、鲁、河间九篇。本三十篇,分布亡失,或二十一篇。目或多或少,文赞或是或误。说《论语》者,但知以剥解之问,以纤微之难,不知存问本根篇数章目。温故知新,可以为师;今不知古,称师如何?

孟子曰:``王者之迹熄而《诗》亡,《诗》亡然後《春秋》作。晋之乘,楚之《杌》,鲁之《春秋》,一也。''若孟子之言,《春秋》者,鲁史记之名,《乘》、《檮杌》同。孔子因旧故之名,以号《春秋》之经,未必有奇说异意,深美之据也。今俗儒说之:``春者岁之始,秋者其终也。《春秋》之经,可以奉始养终,故号为《春秋》。''《春秋》之经,何以异《尚书》?《尚书》者,以为上古帝王之书,或以为上所为下所书,授事相实而为名,不依违作意以见奇。说《尚书》者得经之实,说《春秋》者失圣之意矣。《春秋左氏传》:``桓公十有七年冬十月朔,日有食之。不书日,官失之也。''谓官失之言,盖其实也。史官记事,若今时县官之书矣,其年月尚大难失,日者微小易忘也。盖纪以善恶为实,不以日月为意。若夫公羊、谷梁之传,日月不具,辄为意使。失平常之事,有怪异之说,径直之文,有曲折之义,非孔子之心。夫春秋实言〔冬〕夏,不言者,亦与不书日月,同一实也。

唐、虞、夏、殷、周者,土地之名。尧以唐侯嗣位,舜从虞地得达,禹由夏而起,汤因殷而兴,武王阶周而伐,皆本所兴昌之地,重本不忘始,故以为号,若人之有姓矣。说《尚书》谓之有天下之代号,唐、虞、夏、殷、周者,功德之名,盛隆之意也。故唐之为言荡荡也,虞者乐也,夏者大也,殷者中也,周者至也。尧则荡荡民无能名;舜则天下虞乐;禹承二帝之业,使道尚荡荡,民无能名;殷则道得中;周武则功德无不至。其立义美也,其褒五家大矣,然而违其正实,失其初意。唐、虞、夏、殷、周,犹秦之为秦,汉之为汉。秦起於秦,汉兴於汉中,故曰犹秦、汉;犹王莽从新都侯起,故曰亡新。使秦、汉在经传之上,说者将复为秦、汉作道德之说矣。

尧老求禅,四岳举舜。尧曰:``我其试哉!''说《尚书》曰:``试者,用也;我其用之为天子也。''文为天子也。文又曰:``女於时,观厥刑於二女。''观者,观尔虞舜於天下,不谓尧自观之也。若此者,高大尧、舜,以为圣人相见已审,不须观试,精耀相照,旷然相信。又曰:``四门穆穆,入於大麓,烈风雷雨不迷。''言大麓,三公之位也。居一公之位,大总录二公之事,众多并吉,若疾风大雨。夫圣人才高,未必相知也。圣成事,舜难知佞,使皋陶陈知人之法。佞难知,圣亦难别。尧之才,犹舜之知也。舜知佞,尧知圣。尧闻舜贤,四岳举之,心知其奇而未必知其能,故言``我其试〔哉〕!''试之於职,妻以二女,观其夫妇之法,职治修而不废,夫道正而不僻。复令人庶之野,而观其圣,逢烈风疾雨,终不迷惑。尧乃知其圣,授以天下。夫文言``观''``试'',观试其才也。说家以为譬喻增饰,使事失正是,诚而不存;曲折失意,使伪说传而不绝。造说之传,失之久矣。後生精者,苟欲明经,不原实,而原之者亦校古随旧,重是之文,以为说证。经之传不可从,《五经》皆多失实之说。《尚书》、《春秋》,行事成文,较著可见,故颇独论。

\hypertarget{header-n894}{%
\subsubsection{书解篇}\label{header-n894}}

或曰:``士之论高,何必以文?''

答曰:夫人有文质乃成。物有华而不实,有实而不华者。《易》曰:``圣人之情见乎辞。''出口为言,集札为文,文辞施设,实情敷烈。夫文德,世服也。空书为文,实行为德,著之於衣为服。故曰:德弥盛者文弥缛,德弥彰者人弥明。大人德扩其文炳。小人德炽其文斑。官尊而文繁,德高而文积。华而晥者,大夫之箦,曾子寝疾,命元起易。由此言之,衣服以品贤,贤以文为差。愚杰不别,须文以立折。非唯於人,物亦咸然。龙鳞有文,於蛇为神;凤羽五色,於鸟为君;虎猛,毛蚡蚖;龟知,背负文:四者体不质,於物为圣贤。且夫山无林,则为土山,地无毛,则为泻土;人无文,则为仆人。土山无麋鹿,泻土无五谷,人无文德,不为圣贤。上天多文而後土多理。二气协和,圣贤禀受,法象本类,故多文彩。瑞应符命,莫非文者。晋唐叔虞、鲁成季友、惠公夫人号曰仲子,生而怪奇,文在其手。张良当贵,出与神会,老父授书,卒封留侯。河神,故出图,洛灵,故出书。竹帛所记怪奇之物,不出潢洿。物以文为表,人以文为基。棘子成欲弥文,子贡讥之。谓文不足奇者,子成之徒也。

著作者为文儒,说经者为世儒。二儒在世,未知何者为优。或曰:``文儒不若世儒。世儒说圣人之经,解贤者之传,义理广博,无不实见,故在官常位,位最尊者为博士,门徒聚众,招会千里,身虽死亡,学传於後。文儒为华淫之说,於世无补,故无常官,弟子门徒不见一人,身死之後,莫有绍传,此其所以不如世儒者也。''

答曰:不然。夫世儒说圣情,□□□□,共起并验,俱追圣人。事殊而务同,言异而义钧。何以谓之文儒之说无补於世?世儒业易为,故世人学之多;非事可析第,故宫廷设其位。文儒之业,卓绝不循,人寡其书,业虽不讲,门虽无人,书文奇伟,世人亦传。彼虚说,此实篇。折累二者,孰者为贤?案古俊又著作辞说,自用其业,自明於世。世儒当时虽尊,不遭文儒之书,其迹不传。周公制礼乐,名垂而不灭。孔子作《春秋》,闻传而不绝。周公、孔子,难以论言。汉世文章之徒,陆贾、司马迁、刘子政、扬子云,其材能若奇,其称不由人。世传《诗》家鲁申公,《书》家千乘欧阳、公孙,不遭太史公,世人不闻。夫以业自显,孰与须人乃显?夫能纪百人,孰与廑能显其名?

或曰:``著作者,思虑间也,未必材知出异人也。居不幽,思不至。使著作之人,总众事之凡,典国境之职,汲汲忙忙,〔何〕暇著作?试使庸人积闲暇之思,亦能成篇八十数。文王日昃不暇食,周公一沐三握发,何暇优游为丽美之文於笔札?孔子作《春秋》,不用於周也。司马长卿不预公卿之事,故能作子虚之赋。扬子云存中郎之官,故能成《太玄经》,就《法言》。使孔子得王,《春秋》不作。〔籍〕长卿、子云为相,赋玄不工。''

答曰:文王日昃不暇食,此谓演《易》而益卦。周公一沐三握发,为周改法而制。周道不弊,孔子不作,休思虑间也!周法阔疏,不可因也。夫禀天地之文,发於胸臆,岂为间作不暇日哉?感伪起妄,源流气。管仲相桓公,致於九合。商鞅相孝公,为秦开帝业。然而二子之书,篇章数十。长卿、子云,二子之伦也。俱感,故才并;才同,故业钧。皆士而各著,不以思虑间也。问事弥多而见弥博,官弥剧而识弥泥。居不幽则思不至,思不至则笔不利。嚚顽之人,有幽室之思,虽无忧,不能著一字。盖人材有能,无有不暇。有无材而不能思,无有知而不能著。有鸿材欲作而无起,细知以问而能记。盖奇有无所因,无有不能言,两有无所睹,无不暇造作。

或曰:``凡作者精思已极,居位不能领职。盖人思有所倚着,则精有所尽索。著作之人,书言通奇,其材已极,其知已罢。案古作书者,多位布散盘解,辅倾宁危,非著作之人所能为也。夫有所逼,有所泥,则有所自,篇章数百。吕不韦作《春秋》举家徙蜀;淮南王作道书,祸至灭族;韩非著治术,身下秦狱。身且不全,安能辅国?夫有长於彼,安能不短於此?深於作文,安能不浅於政治?''

答曰:人有所优,固有所劣;人有所工,固有所拙。非劣也,志意不为也,非拙也,精诚不加也。志有所存,顾不见泰山;思有所至,有身不暇徇也。称干将之利,刺则不能击,击则不能刺,非刃不利,不能一旦二也。蛢弹雀则失鷃,射鹊则失雁,方员画不俱成,左右视不并见,人材有两为,不能成一。使干将寡刺而更击,舍鹊而射雁,则下射无失矣。人委其篇章,专为〔政〕治,则子产、子贱之迹不足侔也。古作书者,多立功不用也。管仲、晏婴,功书并作;商鞅、虞卿,篇治俱为。高祖既得天下,马上之计未败,陆贾造《新语》,高祖粗纳采。吕氏横逆,刘氏将倾,非陆贾之策,帝室不宁。盖材知无不能,在所遭遇,遇乱则知立功,有起则以其材著书者也。出口为言,著文为篇。古以言为功者多,以文为败者希。吕不韦、淮南王以他为过,不以书有非,使客作书,不身自为;如不作书,犹蒙此章章之祸。人古今违属,未必皆著作材知极也。邹阳举疏,免罪於梁。徐乐上书,身拜郎中。材能以其文为功於人,何嫌不能营卫其身?韩蚤信公子非,国不倾危。及非之死,李斯如奇,非以著作材极,不能复有为也。春物之伤,或死之也,残物不伤,秋亦不长。假令非不死,秦未可知。故才人能令其行可尊,不能使人必法己;能令其言可行,不能使人必采取之矣。

或曰:``古今作书者非一,各穿凿失经之实传,违圣人质,故谓之蕞残,比之玉屑。故曰:``蕞残满车,不成为道;玉屑满箧,不成为宝。''前人近圣,犹为蕞残,况远圣从後复重为者乎?其作必为妄,其言必不明,安可采用而施行?''

答曰:圣人作其经,贤者造其传,述作者之意,采圣人之志,故经须传也。俱贤所为,何以独谓经传是,他书记非?彼见经传,传经之文,经须而解,故谓之是。他书与书相违,更造端绪,故谓之非。若此者,韪是於《五经》。使言非《五经》,虽是不见听。使《五经》从孔门出,到今常令人不缺灭,谓之纯壹,信之可也。今《五经》遭亡秦之奢侈,触李斯之横议,燔烧禁防,伏生之休,抱经深藏。汉兴,收《五经》,经书缺灭而不明,篇章弃散而不具。晁错之辈,各以私意分拆文字,师徒相因相授,不知何者为是。亡秦无道,败乱之也。秦虽无道,不燔诸子。诸子尺书,文篇具在,可观读以正说,可采掇以示後人。後人复作,犹前人之造也。夫俱鸿而知,皆传记所称,文义与经相薄。何以独谓文书失经之实?由此言之,经缺而不完,书无佚本,经有遗篇。折累二者,孰与蕞残?《易》据事象,《诗》采民以为篇,《乐》须〔民〕欢,《礼》待民平。四经有据,篇章乃成。《尚书》、《春秋》,采掇史记。史记兴无异,以民事一意,《六经》之作皆有据。由此言之,书亦为本,经亦为末,末失事实,本得道质。折累二者,孰为玉屑?知屋漏者在宇下,知政失者在草野,知经误者在诸子。诸子尺书,文明实是。说章句者,终不求解扣明,师师相传,初为章句者,非通览之人也。

\hypertarget{header-n899}{%
\subsection{卷二十九}\label{header-n899}}

\hypertarget{header-n900}{%
\subsubsection{案书篇}\label{header-n900}}

儒家之宗,孔子也。墨家之祖,墨翟也。且案儒道传而墨法废者,儒之道义可为,而墨之法议难从也。何以验之?墨家薄葬、右鬼,道乖相反违其实,宜以难从也。乖违如何?使鬼非死人之精也,右之未可知。今墨家谓鬼审〔死〕人之精也,厚其精而薄其尸,此於其神厚而於其体薄也。薄厚不相胜,华实不相副,则怒而降祸,虽有其鬼,终以死恨。人情欲厚恶薄,神心犹然。用墨子之法,事鬼求福,福罕至而祸常来也。以一况百,而墨家为法,皆若此类也。废而不传,盖有以也。

《春秋左氏传》者,盖出孔子壁中。孝武皇帝时,鲁共王坏孔子教授堂以为宫,得佚《春秋》三十篇,《左氏传》也。公羊高、谷梁、胡母氏皆传《春秋》,各门异户,独《左氏传》为近得实。何以验之?《礼记》造於孔子之堂,太史公。汉之通人也,左氏之言与二书合,公羊高、谷梁寘、胡母氏不相合。又诸家去孔子远,远不如近,闻不如见。刘子政玩弄《左氏》,童仆妻子皆呻吟之。光武皇帝之时,陈元、范淑上书连属,条事是非,《左氏》遂立。范叔寻因罪罢。元、叔天下极才,讲论是非,有余力矣。陈元言讷,范叔章诎,左氏得实,明矣。言多怪,颇与孔子``不语怪力''相违返也。《吕氏春秋》亦如此焉。《国语》,《左氏》之外传也,左氏传经,辞语尚略,故复选录《国语》之辞以实。然则《左氏》《国语》,世儒之实书也。

公孙龙著坚白之论,析言剖辞,务折曲之言,无道理之较,无益於治。齐有三邹衍之书,瀇洋无涯,其文少验,多惊耳之言。案大才之人,率多侈纵,无实是之验;华虚夸诞,无审察之实。商鞅相秦,作耕战之术;管仲相齐,造轻重之篇。富民丰国,强主弱敌,公赏罚,与邹衍之书并言。

而太史公两纪,世人疑惑,不知所从。案张仪与苏秦同时,苏秦之死,仪固知之。仪知〔秦〕审,宜从仪言以定其实,而说不明,两传其文。东海张商亦作列传,岂苏秦商之所为邪?何文相违甚也?《三代世表》言五帝、三王皆黄帝子孙,自黄帝转相生,不更禀气於天。作《殷本纪》,言契母简狄浴於川,遇玄鸟坠卵,吞之,遂生契焉。及《周本纪》言後稷之母姜嫄野出,见大人迹,履之,则妊身,生後稷焉。夫观《世表》,则契与後稷,黄帝之子孙也;读《殷》、《周本纪》,则玄鸟、大人之精气也。二者不可两传,而太史公兼纪不别。案帝王之妃,不宜野出、浴於川水。今言浴於川,吞玄鸟之卵;出於野,履大人之迹:违尊贵之节,误是非之言也。

《新语》,陆贾所造,盖董仲舒相被服焉,皆言君臣政治得失,言可采行,事美足观。鸿知所言,参贰经传,虽古圣之言,不能过增。陆贾之言,未见遗阙,而仲舒之言雩祭可以应天,土龙可以致雨,颇难晓也。夫致旱者以雩祭,不夏郊之祀,岂晋候之过邪?以政失道,阴阳不和也。晋废夏郊之祀,晋侯寝疾,用郑子产之言,祀夏郊而疾愈。如审雩不修,龙不治,与晋同祸,为之再也。以政致旱,宜复以政。政亏而复修雩治龙,其何益哉!《春秋》公羊氏之说,亢阳之节,足以复政。阴阳相浑,旱湛相报,天道然也,何乃修雩设龙乎?雩祀神喜哉?或雨至,亢阳不改,旱祸不除,变复之义,安所施哉!且夫寒温与旱湛同,俱政所致,其咎在人。独为亢旱求福,不为寒温求佑,未晓其故。如当复报寒温,宜为雩、龙之事。鸿材巨识,第两疑焉!

董仲舒著书,不称子者,意殆自谓过诸子也。汉作书者多,司马子长、扬子云,河、汉也,其余泾、渭也。然而子长少臆中之说,子云无世俗之论。仲舒说道术奇矣,北方三家尚矣。谶书云``董仲舒乱我书'',盖孔子言也。读之者或为乱我书者,烦乱孔子之书也,或以为乱者,理也,理孔子之书也。共一``乱''字,理之与乱,相去甚远。然而读者用心不同,不省本实,故说误也。夫言``烦乱孔子之书,才高之语也。其言理孔子之书,亦知奇之言也。出入圣人之门,乱理孔子之书,子长、子云无此言焉。世俗用心不实,省事失情,二语不定,转侧不安。案仲舒之书不违儒家,不〔反〕孔子,其言``烦乱孔子之书者'',非也。孔子之书不乱,其言理孔子之书者,亦非也。孔子曰``师挚之始,《关雎》之乱,洋洋乎盈耳哉!''乱者,〔终〕孔子言也。孔子生周,始其本;仲舒在汉终其末。班叔皮续太史公书,盖其义也。赋颂篇下其有``乱曰''章,盖其类也。孔子终论,定於仲舒之言,其修雩始龙,必将有义,未可怪也。

颜渊曰:``舜何人也?予何人也?''五帝、三王,颜渊独慕舜者,知己步驺有同也。知德所慕,默识所追,同一实也。仲舒之言道德政治,可嘉美也。质定世事,论说世疑,桓君山莫上也。故仲舒之文可及,而君山之论难追也。骥与众马绝迹,或蹈骥哉?有马於此,足行千里,终不名骥者,与骥毛色异也。有人於此,文偶仲舒,论次君山,终不同於二子者,姓名殊也。故马效千里,不必骥;人期贤知,不必孔、墨。何以验之?君山之论难追也。两刃相割,利钝乃知;二论相订,是非乃见。是故韩非之《四难》,桓宽之《盐铁》,君山《新论》类也。世人或疑,言非是伪,论者实之,故难为也。卿决疑讼,狱定嫌罪,是非不决,曲直不立,世人必谓卿狱之吏才不任职。至於论,不务全疑,两传并纪,不宜明处,孰与剖破浑沌,解决乱丝,言无不可知,文无不可晓哉?案孔子作《春秋》,采毫毛之善,贬纤介之恶。可褒,则义以明其行善;可贬,则明其恶以讥其操。《新论》之义,与《春秋》会一也。

夫俗好珍古不贵今,谓今之文不如古书。夫古今一也,才有高下,言有是非,不论善恶而徒贵古,是谓古人贤今人也。案东番邹伯奇、临淮袁太伯、袁文术、会稽吴君高、周长生之辈,位虽不至公卿,诚能知之囊橐,文雅之英雄也。观伯奇之《元思》,太伯之《易〔章〕句》,文术之《咸铭》,君高之《越纽录》,长生之《洞历》,刘子政、扬子云不能过也。〔盖〕才有浅深,无有古今;文有伪真,无有故新。广陵陈子回、颜方,今尚书郎班固,兰台令杨终、傅毅之徒,虽无篇章,赋颂记奏,文辞斐炳,赋象屈原、贾生,奏象唐林、谷永,并比以观好,其美一也。当今未显,使在百世之後,则子政、子云之党也。韩非著书,李斯采以言事;扬子云作《太玄》,侯铺子随而宣之。非斯同门,云、铺共朝,睹奇见益,不为古今变心易意;实事贪善,不远为术并肩以迹相轻,好奇无已,故奇名无穷。扬子云反《离骚》之经,非能尽反,一篇文往往见非,反而夺之。《六略》之录,万三千篇,虽不尽见,指趣可知,略借不合义者,案而论之。

\hypertarget{header-n904}{%
\subsubsection{对作篇}\label{header-n904}}

或问曰:``贤圣之空生,必有以用其心。上自孔、墨之党,下至苟、孟之徒,教训必作垂文。何也?''

对曰:圣人作经,艺者传记,匡济薄俗,驱民使之归实诚也。案六略之书,万三千篇,增善消恶,割截横拓,驱役游慢,期便道善,归政道焉。孔子作《春秋》,周民弊也。故采求毫毛之善,贬纤介之恶,拨乱世,反诸正,人道浃,王道备,所以检押靡薄之俗者,悉具密致。夫防决不备,有水溢之害;网解不结,有兽失之患。是故周道不弊,则民不文薄;民不文薄,《春秋》不作。杨、墨之学不乱〔儒〕义,则孟子之传不造;韩国不小弱,法度不坏废,则韩非之书不为;高祖不辨得天下,马上之计未转,则陆贾之语不奏;众事不失实,凡论不坏乱,则桓谭之论不起。故夫贤圣之兴文也,起事不空为,因因不妄作。作有益於化,化有补於正。故汉立兰台之官,校审其书,以考其言。董仲舒作道术之书,颇言灾异政治所失,书成文具,表在汉室。主父偃嫉之,诬奏其书。天子下仲舒於吏,当谓之下愚。仲舒当死,天子赦之。夫仲舒言灾异之事,孝武犹不罪而尊其身,况所论无触忌之言,核道实之事,收故实之语乎!故夫贤人之在世也,进则尽忠宣化,以明朝廷;退则称论贬说,以觉失俗。俗也不知还,则立道轻为非;论者不追救,则迷乱不觉悟。

是故《论衡》之造也,起众书并失实,虚妄之言胜真美也。故虚妄之语不黜,则华文不见息;华文放流,则实事不见用。故《论衡》者,所以铨轻重之言,立真伪之平,非苟调文饰辞,为奇伟之观也。其本皆起人间有非,故尽思极心,以〔讥〕世俗。世俗之性,好奇怪之语,说虚妄之文。何则?实事不能快意,而华虚惊耳动心也。是故才能之士,好谈论者,增益实事,为美盛之语;用笔墨者,造生空文,为虚妄之传。听者以为真然,说而不舍;览者以为实事,传而不绝。不绝,则文载竹帛之上;不舍,则误入贤者之耳。至或南面称师,赋奸伪之说;典城佩紫,读虚妄之书。明辨然否,疾心伤之,安能不论?孟子伤杨、墨之议大夺儒家之论,引平直之说,褒是抑非,世人以为好辩。孟子曰:``予岂好辩哉?予不得已!''今吾不得已也!虚妄显於真,实诚乱於伪,世人不悟,是非不定,紫失杂厕,瓦玉集糅,以情言之,岂吾心所能忍哉!卫骖乘者越职而呼车,恻怛发心,恐〔上〕之危也。夫论说者闵世忧俗,与卫骖乘者同一心矣。愁精神而幽魂魄。动胸中之静气,贼年损寿,无益於性,祸重於颜回,违负黄、老之教,非人所贪,不得已,故为《论衡》。文露而旨直,辞奸而情实。其《政务》言治民之道。《论衡》诸篇,实俗间之凡人所能见,与彼作者无以异也。若夫九《虚》、三《增》、《论死》、《订鬼》,世俗所久惑,人所不能觉也。人君遭弊,改教於上;人臣愚惑,作论於下。〔下〕实得,则上教从矣。冀悟迷惑之心,使知虚实之分。实虚之分定,而华伪之文灭。华伪之文灭,则纯诚之化日以孽矣。

或曰:``圣人作,贤者述。以贤而作者,非也。《论衡》、《政务》,可谓作者。''曰:〔非〕作也,亦非述也,论也。论者,述之次也。《五经》之兴,可谓作矣。太史公《书》、刘子政《序》、班叔皮《传》,可谓述矣。桓君山《新论》、邹伯奇《检论》,可谓论矣。今观《论衡》、《政务》,桓、邹之二论也,非所谓作也。造端更为,前始未有,若仓颉作书,奚仲作车是也。《易》言伏羲作八卦,前是未有八卦,伏羲造之,故曰作也。文王图八,自演为六十四,故曰衍。谓《论衡》之成,犹六十四卦,而又非也。六十四卦以状衍增益,其卦溢,其数多。今《论衡》就世俗之书,订其真伪,辩其实虚,非造始更为,无本於前也。儒生就先师之说,诘而难之;文吏就狱之事,覆而考之,谓《论衡》为作,儒生、文吏谓作乎?

上书奏记,陈列便宜,皆欲辅政。今作书者,犹〔上〕书奏记,说发胸臆,文成手中,其实一也。夫上书谓之奏记,转易其名谓之书。建初孟年,中州颇歉,颍川、汝南民流四散,圣主忧怀,诏书数至。《论衡》之人,奏记郡守,宜禁奢侈,以备困乏。言不纳用,退题记草,名曰《备乏》。酒縻五谷,生起盗贼,沉湎饮酒,盗贼不绝,奏记郡守,禁民酒。退题记草,名曰《禁酒》。由此言之,夫作书者,上书奏记之文也。记谓之造作上书,上书奏记是作也?

晋之乘,而楚之檮杌,鲁之春秋,人事各不同也。《易》之乾坤,《春秋》之``元'',杨氏之``玄'',卜气号不均也。由此言之,唐林之奏,谷永之章,《论衡》、《政务》,同一趋也。汉家极笔墨之林,书论之造,汉家尤多。阳成子张作``乐'',扬子云造``玄'',二经发於台下,读於阙掖,卓绝惊耳,不述而作,材疑圣人,而汉朝不讥。况《论衡》细说微论,解释世俗之疑,辩照是非之理,使後进晓见然否之分,恐其废失,著之简牍,祖经章句之说,先师奇说之类也。其言伸绳,弹割俗传。俗传蔽惑,伪书放流,贤通之人,疾之无已。孔子曰:``诗人疾之不能默,丘疾之不能伏。''是以论也。玉乱於石,人不能别。或若楚之王尹以玉为石,卒使卞和受刖足之诛。是反为非,虚转为实,安能不言?俗传既过,俗书之伪。若夫邹衍谓今天下为一州,四海之外有若天下者九州。《淮南书》言共工与颛顼争为天子,不胜,怒而触不周之山,使天柱折,地维绝。尧时十日并出,尧上射九日;鲁阳战而日暮,援戈麾日,日为却还。世间书传,多若等类,浮妄虚伪,没夺正是。心渍涌,笔手扰,安能不论?论则考之以心,效之以事,浮虚之事,辄立证验。若太史公之书,据许由不隐,燕太子丹不使日再中。读见之者,莫不称善。

《政务》为郡国守相、县邑令长陈通政事所当尚务,欲令全民立化,奉称国恩。《论衡》九《虚》三《增》,所以使浴务实诚也;《论死》、《订鬼》,所以使浴薄丧葬也。孔子径庭丽级,被棺敛者不省。刘子政上薄葬,奉送藏者不约。光武皇帝草车茅马,为明器者不奸。何世书俗言不载?信死之语汶浊之也。今著《论死》及《死伪》之篇,明死无知,不能为鬼,冀观览者将一晓解约葬,更为节俭。斯盖《论衡》有益之验也。言苟有益,虽作何害?仓颉之书,世以纪事;奚仲之车,世以自载;伯余之衣,以辟寒暑;桀之瓦屋,以辟风雨。夫不论其利害,而徒讥其造作,是则仓颉之徒有非,《世本》十五家皆受责也。故夫有益也,虽作无害也。虽无害,何补?

古有命使采爵,欲观风俗知下情也。《诗》作民间,圣王可云``汝民也,何发作'',囚罪其身,殁灭其诗乎?今已不然,故《诗》传〔至〕今。《论衡》、《政务》,其犹《诗》也,冀望见采,而云有过。斯盖《论衡》之书所以兴也。且凡造作之过,意其言妄而谤诽也。《论衡》实事疾妄,《齐世》、《宣汉》、《恢国》、《验符》、《盛褒》、《须颂》之言,无诽谤之辞。造作如此,可以免於罪矣。

\hypertarget{header-n909}{%
\subsection{卷三十}\label{header-n909}}

\hypertarget{header-n910}{%
\subsubsection{自纪篇}\label{header-n910}}

王充者,会稽上虞人也,字仲任。其先本魏郡元城一姓。孙一几世尝从军有功,封会稽阳亭。一岁仓卒国绝,因家焉。以农桑为业。世祖勇任气,卒咸不揆於人。岁凶,横道伤杀,怨仇众多。会世扰乱,恐为怨仇所擒,祖父泛举家檐载,就安会稽,留钱唐县,以贾贩为事。生子二人,长曰蒙,少曰诵,诵即充父。祖世任气,至蒙、诵滋甚。故蒙、诵在钱唐,勇势凌人。末复与豪家丁伯等结怨,举家徙处上虞。

建武三年,充生。为小兒,与侪伦遨戏,不好狎侮。侪伦好掩雀、捕蝉、戏钱、林熙,充独不肯。诵奇之。六岁教书,恭愿仁顺,礼敬具备,矜庄寂寥,有臣人之志。父未尝笞,母未尝非,闾里未尝让。八岁出於书馆,书馆小僮百人以上,皆以过失袒谪,或以书丑得鞭。充书日进,又无过失。手书既成,辞师受《论语》、《尚书》,日讽千字。经明德就,谢师而专门,援笔而众奇。所读文书,亦日博多。才高而不尚苟作,口辩而不好谈对,非其人,终日之言。其论说始若诡於众,极听其终,众乃是之。以笔著文,亦如此焉;操行事上,亦如此焉。在县位至掾功曹,在都尉府位亦掾功曹,在太守为列掾五官功曹行事,入州为从事。不好徼名於世,不为利害见将。常言人长,希言人短。专荐未达,解已进者过。及所不善,亦弗誉;有过不解,亦弗复陷。能释人之不大过,亦悲夫人之细非。好自周,不肯自彰,勉以行操为基,耻以材能为名。众会乎坐,不问不言,赐见君将,不及不对。在乡里,慕蘧伯玉之节;在朝廷,贪史子鱼之行。见污伤,不肯自明;位不进,亦不怀恨。贫无一亩庇身,志佚於王公;贱无斗石之秩,意若食万锺。得官不欣,失位不恨。处逸乐而欲不放,居贫苦而志不倦。淫读古文,甘闻异言。世书俗说,多所不安,幽处独居,考论实虚。

充为人清重,游必择友,不好苟交。所友位虽微卑,年虽幼稚,行苟离俗,必与之友。好杰友雅徒,不氾结俗材。俗材因其微过,蜚条陷之,然终不自明,亦不非怨其人。或曰:``有良材奇文,无罪见陷,胡不自陈?羊胜之徒,摩口膏舌;邹阳自明,入狱复出。苟有全完之行,不宜为人所缺;既耐勉自伸,不宜为人所屈。''答曰:不清不见尘,不高不见危,不广不见削,不盈不见亏。士兹多口,为人所陷,盖亦其宜。好进故自明,憎退故自陈。吾无好憎,故默无言。羊胜为谗,或使之也;邹阳得免,或拔之也。孔子称命,孟子言天,吉凶安危,不在於人。昔人见之,故归之於命,委之於时,浩然恬忽,无所怨尤。福至不谓己所得,祸到不谓己所为。故时进意不为丰,时退志不为亏。不嫌亏以求盈,不违险以趋平,不鬻智以干禄,不辞爵以吊名,不贪进以自明,不恶退以怨人。同安危而齐死生,钧吉凶而一败成,遭十羊胜,谓之无伤。动归於天,故不自明。

充性恬淡,不贪富贵。为上所知,拔擢越次,不慕高官。不为上所知,贬黜抑屈,不恚下位。比为县吏,无所择避。或曰:``心难而行易,好友同志,仕不择地,浊操伤行,世何效放?''答曰:可效放者,莫过孔子。孔子之仕,无所避矣。为乘田委吏,无於邑之心;为司空相国,无说豫之色。舜耕历山,若终不免;及受尧禅,若卒自得。忧德之不丰,不患爵之不尊;耻名之不白,不恶位之不迁。垂棘与瓦同椟,明月与砾同囊,苟有二宝之质,不害为世所同。世能知善,虽贱犹显;不能别白,虽尊犹辱。处卑与尊齐操,位贱与贵比德,斯可矣。

俗性贪进忽退,收成弃败。充升擢在位之时,众人蚁附;废退穷居,旧故叛去。志俗人之寡恩,故闲居作《讥俗》、《节义》十二篇。冀俗人观书而自觉,故直露其文,集以俗言。或谴谓之浅。答曰:以圣典而示小雅,以雅言而说丘野,不得所晓,无不逆者。故苏秦精说於赵,而李兑不说;商鞅以王说秦,而孝公不用。夫不得心意所欲,虽尽尧、舜之言,犹饮牛以酒,啖马以脯也。故鸿丽深懿之言,关於大而不通於小。不得已而强听,入胸者少。孔子失马於野,野人闭不与,子贡妙称而怒,马圄谐说而懿。俗晓〔形〕露之言,勉以深鸿之文,犹和神仙之药以治鼽咳,制貂狐之裘以取薪菜也。且礼有所不彳侍,事有所不须。断决知辜,不必皋陶;调和葵韭,不俟狄牙;闾巷之乐,不用《韶》、《武》;里母之祀,不待太牢。既有不须,而又不宜。牛刀割鸡,舒戟采葵,鈇钺裁箸,盆盎酌卮,大小失宜,善之者希。何以为辩?喻深以浅。何以为智?喻难以易。贤圣铨材之所宜,故文能为深浅之差。

充既疾俗情,作《讥俗》之书;又闵人君之政,徒欲治人,不得其宜,不晓其务,愁精苦思,不睹所趋,故作《政务》之书。又伤伪书俗文多不实诚,故为《论衡》之书。夫贤圣殁而大义分,磋殊趋,各自开门。通人观览,不能钉铨。遥闻传授,笔写耳取,在百岁之前。历日弥久,以为昔古之事,所言近是,信之入骨,不可自解,故作《实论》。其文盛,其辩争,浮华虚伪之语,莫不澄定。没华虚之文,存敦庞之朴,拨流失之风,反宓戏之俗。

充书形露易观。或曰:``口辩者其言深,笔敏者其文沉。案经艺之文,贤圣之言,鸿重优雅,难卒晓睹。世读之者,训古乃下。盖贤圣之材鸿,故其文语与俗不通。玉隐石间,珠匿鱼腹,非玉工珠师,莫能采得。宝物以隐闭不见,实语亦宜深沉难测。《讥俗》之书,欲悟俗人,故形露其指,为分别之文。《论衡》之书,何为复然?岂材有浅极,不能为〔深〕覆?何文之察,与彼经艺殊轨辙也?''

答曰:玉隐石间,珠匿鱼腹,故为深覆。及玉色剖於石心,珠光出於鱼腹,其〔犹〕隐乎?吾文未集於简札之上,藏於胸臆之中,犹玉隐珠匿也;及出露,犹玉剖珠出乎,烂若天文之照,顺若地理之晓,嫌疑隐微,尽可名处。且名白,事自定也。

《论衡》者,论之平也。口则务在明言,笔则务在露文。高士之文雅,言无不可晓,指无不可睹。观读之者,晓然若盲之开目,聆然若聋之通耳。三年盲子,卒见父母,不察察相识,安肯说喜?道畔巨树,堑边长沟,所居昭察,人莫不知。使树不巨而隐,沟不长而匿,以斯示人,尧、舜犹惑。人面色部七十有余,颊肌明洁,五色分别,隐微忧喜,皆可得察,占射之者,十不失一。使面黝而黑丑,垢重袭而覆部,占射之者,十而失九。

夫文由语也,或浅露分别,或深迂优雅,孰为辩者?故口言以明志,言恐灭遗,故著之文字。文字与言同趋,何为犹当隐闭指意?狱当嫌辜,卿决疑事,浑沌难晓,与彼分明可知,孰为良吏?夫口论以分明为公,笔辩以荴露为通,吏文以昭察为良。深覆典雅,指意难睹,唯赋颂耳!经传之文,贤圣之语,古今言殊,四方谈异也。当言事时,非务难知,使指闭隐也。後人不晓,世相离远,此名曰语异,不名曰材鸿。浅文读之难晓,名曰不巧,不名曰知明。秦始皇读韩非之书,叹曰:``犹独不得此人同时。''其文可晓,故其事可思。如深鸿优雅,须师乃学,投之於地,何叹之有?夫笔著者,欲其易晓而难为,不贵难知而易造;口论务解分而可听,不务深迂而难睹。孟子相贤,以眸子明了者,察文,以义可晓。

充书违诡於俗。或难曰:``文贵夫顺合众心,不违人意,百人读之莫谴,千人闻之莫怪。故管子曰:`言室满室,言堂满堂。'今殆说不与世同,故文刺於俗,不合於众。''

答曰:论贵是而不务华,事尚然而不高合。论说辩然否,安得不谲常心、逆俗耳?众心非而不从,故丧黜其伪,而存定其真。如当从顺人心者,循旧守雅,讽习而已,何辩之有?孔子侍坐於鲁哀公,公赐桃与黍,孔子先食黍而后啖桃,可谓得食序矣,然左右皆掩口而笑,贯俗之日久也。今吾实犹孔子之序食也,俗人违之,犹左右之掩口也。善雅歌,於郑为人悲;礼舞,於赵为不好。尧、舜之典,伍伯不肯观;孔、墨之籍,季、孟不肯读。宁危之计,黜於闾巷;拨世之言,訾於品俗。有美味於斯,俗人不嗜,狄牙甘食。有宝玉於是,俗人投之,卞和佩服。孰是孰非,可信者谁?礼俗相背,何世不然?鲁文逆祀,畔者三人。盖独是之语,高士不舍,俗夫不好;惑众之书,贤者欣颂,愚者逃顿。

充书不能纯美。或曰:``口无择言,笔无择文。文必丽以好,言必辩以巧。言了於耳,则事味於心;文察於目,则篇留於手。故辩言无不听,丽文无不写。今新书既在论譬,说俗为戾,又不美好,於观不快。盖师旷调音,曲无不悲;狄牙和膳,肴无淡味。然则通人造书,文无暇秽。《吕氏》、《淮南》悬於市门,观读之者无訾一言。今无二书之美,文虽众盛,犹多谴毁。''答曰:夫养实者不育华,调行者不饰辞。丰草多华英,茂林多枯枝。为文欲显白其为,安能令文而无谴毁?救火拯溺,义不得好;辩论是非,言不得巧。入泽随龟,不暇调足;深渊捕蛟,不暇定手。言奸辞简,指趋妙远;语甘文峭,务意浅小。稻谷千锺,糠皮太半;阅钱满亿,穿决出万。大羹必有淡味,至宝必有瑕秽,大简必有大好,良工必有不巧。然则辩言必有所屈,通文犹有所黜。言金由贵家起,文粪自贱室出,《淮南》、《吕氏》之无累害,所由出者,家富官贵也。夫贵,故得悬於市,富,故有千金副。观读之者,惶恐畏忌,虽见乖不合,焉敢谴一字?

充书既成,或稽合於古,不类前人。或曰:``谓之饰岁偶辞,或径或迂,或屈或舒。谓之论道,实事委琐,文给甘酸,谐於经不验,集於传不合,稽之子长不当,内之子云不入。文不与前相似,安得名佳好,称工巧?''答曰:饰貌以强类者失形,调辞以务似者失情。百夫之子,不同父母,殊类而生,不必相似,各以所禀,自为佳好。文必有与合然後称善,是则代匠斫不伤手,然後称工巧也。文士之务,各有所从,或调辞以巧文,或辩伪以实事。必谋虑有合,文辞相袭,是则五帝不异事,三王不殊业也。美色不同面,皆佳於目;悲音不共声,皆快於耳。酒醴异气,饮之皆醉;百谷殊味,食之皆饱。谓文当与前合,是谓舜眉当复八采,禹目当复重瞳。

充书文重。或曰:``文贵约而指通,言尚省而趋明。辩士之言要而达,文人之辞寡而章。今所作新书,出万言,繁不省,则读者不能尽;篇非一,则传者不能领。被躁人之名,以多为不善。语约易言,文重难得。玉少石多,多者不为珍;龙少鱼众,少者固为神。''答曰:有是言也。盖〔要〕言无多,而华文无寡。为世用者,百篇无害;不为用者,一章无补。如皆为用,则多者为上,少者为下。累积千金,比於一百,孰为富者?盖文多胜寡,财寡愈贫。世无一卷,吾有百篇;人无一字,吾有万言,孰者为贤?今不曰所言非,而云泰多,不曰世不好善,而云不能领,斯盖吾书所以不得省也。夫宅舍多,土地不得小;户口众,簿籍不得少。今失实之事多,华虚之语众,指实定宜,辩争之言,安得约径?韩非之书,一条无异,篇以十第,文以万数。夫形大,衣不得褊;事众,文不得褊。事众文饶,水大鱼多。帝都谷多,王市肩磨。书虽文重,所论百种。按古太公望,近董仲舒,传作书篇百有余,吾书亦才出百,而云泰多,盖谓所以出者微,观读之者不能不谴呵也。河水沛沛,比夫众川,孰者为大?虫茧重厚,称其出丝,孰为多者?

充仕数不耦,而徒著书自纪。或〔戏〕曰:``所贵鸿材者,仕宦耦合,身容说纳,事得功立,故为高也。今吾子涉世落魄,仕数黜斥,材未练於事,力未尽於职,故徒幽思属文,著记美言,何补於身?众多欲以何移乎?''答曰:材鸿莫过孔子。孔子才不容,斥逐,伐树,接〔淅〕,见围,削迹,困饿陈、蔡,门徒菜色。今吾材不逮孔子,不偶之厄,未与之等,偏可轻乎?且达者未必知,穷者未必愚。遇者则得,不遇失之。故夫命厚禄善,庸人尊显;命薄禄恶,奇俊落魄。比以偶合称材量德,则夫专城食土者,材贤孔、墨。身贵而名贱,则居洁而行墨。食千锺之禄,无一长之德,乃可戏也。若夫德高而名白,官卑而禄泊,非才能之过,未足以为累也。士愿与宪共庐,不慕与赐同衡;乐与夷俱旅,不贪与蹠比迹。高士所贵,不与俗均,故其名称不与世同。身与草木俱朽,声与日月并彰,行与孔子比穷,文与杨雄为双,吾荣之。身通而知困,官大而德细,於彼为荣,於我为累。偶合容说,身尊体佚,百载之後,与物俱殁,名不流於一嗣,文不遗於一札,官虽倾仓,文德不丰,非吾所臧。德汪而渊懿,知滂沛而盈溢,笔泷漉而雨集,言溶氵窟而泉出,富材羡知,贵行尊志,体列於一世,名传於千载,乃吾所谓异也。

充细族孤门。或啁之曰:``宗祖无淑懿之基,文墨无篇籍之遗,虽著鸿丽之论,无所禀阶,终不为高。夫气无渐而卒至曰变,物无类而妄生曰异,不常有而忽见曰妖,诡於众而突出曰怪。吾子何祖?其先不载。况未尝履墨涂,出儒门,吐论数千万言,宜为妖变,安得宝斯文而多贤?''答曰:鸟无世凤皇,兽无种麒麟,人无祖圣贤,物无常嘉珍。才高见屈,遭时而然。士贵,故孤兴;物贵,故独产。文孰常在有以放贤,是则〔醴〕泉有故源,而嘉禾有旧根也。屈奇之士见,倜傥之辞生,度不与俗协,庸角不能程。是故罕发之迹,记於牒籍;希出之物,勒於鼎铭。五帝不一世而起,伊、望不同家而出。千里殊迹,百载异发。士贵雅材而慎兴,不因高据以显达。母骊犊骍,无害牺牲;祖浊裔清,不榜奇人。鲧恶禹圣,叟顽舜神。伯牛寝疾,仲弓洁全;颜路庸固,回杰超伦;孔、墨祖愚,丘、翟圣贤;扬家不通,卓有子云;桓氏稽可,谲出君山。更禀於元,故能著文。

充以元和三年徙家辟诣扬州部丹阳、九江、庐江。後入为治中,材小任大,职在刺割,笔札之思,历年寝废。章和二年,罢州家居。年渐七十,时可悬舆。仕路隔绝,志穷无如。事有否然,身有利害。发白齿落,日月逾迈,俦伦弥索,鲜所恃赖。贫无供养,志不娱快。历数冉冉,庚辛域际,虽惧终徂,愚犹沛沛,乃作《养性》之书,凡十六篇。养气自守,适时则酒,闭明塞聪,爱精自保,适辅服药引导,庶冀性命可延,斯须不老。既晚无还,垂书示後。惟人性命,长短有期,人亦虫物,生死一时。年历但记,孰使留之?犹入黄泉,消为土灰。上自黄、唐,下臻秦、汉而来,折衷以圣道,理於通材,如衡之平,如鉴之开,幼老生死古今,罔不详该。命以不延,吁叹悲哉!

\end{document}
