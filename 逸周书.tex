\PassOptionsToPackage{unicode=true}{hyperref} % options for packages loaded elsewhere
\PassOptionsToPackage{hyphens}{url}
%
\documentclass[]{article}
\usepackage{lmodern}
\usepackage{amssymb,amsmath}
\usepackage{ifxetex,ifluatex}
\usepackage{fixltx2e} % provides \textsubscript
\ifnum 0\ifxetex 1\fi\ifluatex 1\fi=0 % if pdftex
  \usepackage[T1]{fontenc}
  \usepackage[utf8]{inputenc}
  \usepackage{textcomp} % provides euro and other symbols
\else % if luatex or xelatex
  \usepackage{unicode-math}
  \defaultfontfeatures{Ligatures=TeX,Scale=MatchLowercase}
\fi
% use upquote if available, for straight quotes in verbatim environments
\IfFileExists{upquote.sty}{\usepackage{upquote}}{}
% use microtype if available
\IfFileExists{microtype.sty}{%
\usepackage[]{microtype}
\UseMicrotypeSet[protrusion]{basicmath} % disable protrusion for tt fonts
}{}
\IfFileExists{parskip.sty}{%
\usepackage{parskip}
}{% else
\setlength{\parindent}{0pt}
\setlength{\parskip}{6pt plus 2pt minus 1pt}
}
\usepackage{hyperref}
\hypersetup{
            pdfborder={0 0 0},
            breaklinks=true}
\urlstyle{same}  % don't use monospace font for urls
\setlength{\emergencystretch}{3em}  % prevent overfull lines
\providecommand{\tightlist}{%
  \setlength{\itemsep}{0pt}\setlength{\parskip}{0pt}}
\setcounter{secnumdepth}{0}
% Redefines (sub)paragraphs to behave more like sections
\ifx\paragraph\undefined\else
\let\oldparagraph\paragraph
\renewcommand{\paragraph}[1]{\oldparagraph{#1}\mbox{}}
\fi
\ifx\subparagraph\undefined\else
\let\oldsubparagraph\subparagraph
\renewcommand{\subparagraph}[1]{\oldsubparagraph{#1}\mbox{}}
\fi

% set default figure placement to htbp
\makeatletter
\def\fps@figure{htbp}
\makeatother


\date{}

\begin{document}

\hypertarget{header-n0}{%
\section{逸周书}\label{header-n0}}

\begin{center}\rule{0.5\linewidth}{\linethickness}\end{center}

\tableofcontents

\begin{center}\rule{0.5\linewidth}{\linethickness}\end{center}

\hypertarget{header-n6}{%
\subsection{卷一}\label{header-n6}}

\hypertarget{header-n7}{%
\subsubsection{度训解}\label{header-n7}}

天生民而制其度,度小大以整,权轻重以极,明本末以立中。立中以补损,补损以知足。爵以明等极,极以正民,正中外以成命。正上下以顺政。自迩弥兴自远,远迩备极终也。明王是以敬微而顺分,分次以知和,知和以知乐,知乐以知哀,知哀以知慧,内外以知人。

凡民生而有好有恶,小得其所好则喜,大得其所好则乐,小遭其所恶则忧,大遭其所恶则哀。然凡民之所好恶,生物是好,死物是恶,民至有好而不让,不从其所好,必犯法,无以事上。民至有恶不让,不去其所恶,必犯法,无以事上,边行于此,尚有顽民,而况□不去其所恶,而从其所好,民能居乎。

若不□力何以求之,力争则力政,力政则无让,无让则无礼,无礼,虽得所好,民乐乎?若不乐乃所恶也。凡民不忍好恶,不能分次,不次则夺,夺则战,战则何以养老幼,何以救痛疾葬丧,何以胥役也。

明王是以极等以断好恶,教民次分扬举力竞。任壮养老长幼有报,民是以胥役也。夫力竞,非众不克,众非和不众,和非中不立,中非礼不慎,礼非乐不履。明王是以无乐非人,无哀非人。人是以众,人众赏多罚少,政之美也。罚多赏少,政之恶也。罚多则困,赏多则乏,乏困无丑,教乃不至。是故民主明丑以长子孙,子孙习服,鸟兽仁德。土宜天时,百物和治,治之初,厉初哉。治化则顺,是故无顺非厉,长幼成而生,曰顺极。

\hypertarget{header-n14}{%
\subsubsection{命训解}\label{header-n14}}

天生民而成大命,命司德正之以祸扶。立明王以顺之,曰大命有常,小命日成,成则敬,有常则广,广以敬命则度至于极。夫司德司义而赐之福禄,福禄在,人能无惩乎?若惩而悔过,则度至于极。夫或司不义而降之祸,在人能无惩乎?若惩而悔过,则至于极。夫民生而丑不明,无以明之,能无丑乎?若有丑而竞行不丑,则度至于极。民生而乐,生无以谷之,能无劝乎,若劝之以忠,则度至于极。夫民生而恶性死,无意畏之,能无恐乎?若恐而承教,则度至于极。

六极既通,六闲具塞,通道通天以正人,正人莫如有极,道天莫如无极。道天有极则不威,不威则不昭;正人无极则不信,不信则不行。明王昭天信人,以度功地以利之,使信人畏天,则度,至于极。

夫天道三,人道三。天有命、有祸、有福,人有丑、有绋絻、有斧钺。以人之丑、当天之命,以不存,极命则民堕,民堕则旷命,旷命以戒其上,则殆于乱。极福则民禄,民禄则干善,干善则不行,极祸则民鬼,民鬼则淫祭,淫祭则罢家。极丑则民叛,民叛则伤人,伤人则不义。极赏则民贾其上,贾其上则民无让,无让则不顺。极罚则民所诈,多诈则不忠,不忠则无报。凡此六者,政之殆也,明王是故昭命以命之,曰:大命世,小命身。

福莫大于行义,祸莫大于淫祭,丑莫大于伤人,赏莫大于信义,让莫大于贾上,罚莫大于贪诈。古之明王奉此六者,以牧万民,民用而不失。

抚之以惠,和之以均,敛之以哀,娱之以乐,慎之以礼,教之以艺,震之以政,动之以事,劝之以赏,畏之以罚,临之以忠,行之以权。权不法,忠不忠,罚不服,赏不从,劳事不震,政不成,艺不淫,礼有时,乐不满,哀不至,均不壹,不忍人。凡此物攘之属也。

惠而不忍人,人不胜害,害不如死。□均一则不和,哀至则匮,乐满则荒,礼无时则不贵,艺淫则害于才,政成则不长,事震则寡功。以赏从劳,劳而不至,以法从中,则赏,赏不必中,以权从法则行,行不必以知权,权以知微,微以知始,始以知终。

\hypertarget{header-n23}{%
\subsubsection{常训解}\label{header-n23}}

天有常性,人有常顺,顺在可变,性在不改。不改可因,因在好恶,好恶生变,变习生常,常则生丑,丑命生德。明王于是生,政以正之。民生而有习有常,以习为常,以常为慎。民若生于中,习常为常。夫习民乃常,为自血气始。明王自血气耳目之习,以明之丑。丑明乃乐义,乐义乃至上,上贤而不穷。

哀乐不淫,民知其至,而至于子孙,民乃有古。古者因民以顺民。夫民群居而无选,为政以始之,始之以古,终之以古。行古志,今政之至也。政维今,法维古。顽贪以疑,疑意以两,平两以参,参伍以权,权数以多,多宁以允,允德以慎,慎微以始而敬。终乃不困。

困在坟,诱在王,民乃苟,苟乃不明,哀乐不时,四征不显,六极不服,把八政不顺,九德有奸,九奸不迁,万物不至。夫礼非克不承,非乐不竟,民是复生□,好恶有四征,喜乐忧哀,动之以则,发之以文,成之以民,行之以化。

命、丑、福、赏、祸、罚,六极不嬴,八政和平。八政:夫妻、父子、兄弟、君臣。八政不怩,九德纯恪。九德:忠、信、敬、刚、柔、和、固、贞、顺。顺言曰政,顺政曰遂,遂伪曰奸,奸物在目,奸声在耳,耳目有疑。疑言有枢,枢动有和,和意无等。万民无法,□□在赦,□复在古,古者,明王奉法以明幽,幽王奉幽以废法,则一人也,而绩功不同,明王是以敬微而顺分。

\hypertarget{header-n30}{%
\subsubsection{文酌解}\label{header-n30}}

民生而有欲、有恶、有乐、有哀、有德、有则,则有九聚,德有五宝,哀有四忍,乐有三丰,恶有二咎,欲有一极。极有七事,咎有三尼,丰有三频,忍有四教,宝有五大,聚有九酌。

九酌:一取允移人,二宗杰以亲,三发滞以正民,四贷官以属,我人□必礼,六往来取此,七商贾易资,八农人美利,九□宠可动。

五大:一大知率谋,我大武剑勇,三大工赋事,四大商行贿,五大农假贷。

四教:一守之以信,二因亲家年,三取戚免梏,四乐生身复。

三频:一频禄质,二阴福灵极,三留身散真。

三尼:一除戎咎丑,二申亲考疏,三假时权要。

七事:一滕咎信志,二援拔渎谋,三聚疑沮事,四滕属威众,五处宽身降,六陵塞胜备,七录兵免戎。

一极:惟事昌道,开蓄伐。

伐有三穆、七信、一幹、二御、三安、十二来。

三穆:一绝灵破城,二筮奇昌为,三龟从惟凶。

七信:一仁之慎散,二知之万巧,三勇之精富,四族之寡贿,五商之浅资,六农之少积,七贵之争宠。

一幹:胜权舆。

二御:一树惠不\textless{}疒悉\textgreater{},二既用兹忧。

三安:一定居安帑,二贡贵得布,三刑罪布财。

十二来:一弓二矢归射,三轮四舆归御,五剥六鱼归蓄,七陶八冶归灶,九柯十匠归林,十一竹十二苇归时。

三穆、七信、一幹、二御、三安、十二来,伐道咸布,物无不落,落物取配,维有用究。急哉急哉,后失时。

\hypertarget{header-n49}{%
\subsubsection{籴匡解}\label{header-n49}}

成年,年谷足宾祭,祭以盛。大驯锺绝,服美义淫。阜畜约制,余子务艺。宫室城廓修为备,供有嘉菜,于是日满。古

年俭谷不足,宾祭以中盛。乐唯锺鼓,不服美。三牧五库补摄,凡美不修,余子务穑,于是糺秩。

民饥则勤而不宾,举祭以薄。乐无锺鼓
。凡美禁,畜不阜群。车不雕攻,兵备不制,民利不淫。征当商旅,以救穷乏,问随乡,不鬻熟。分助有匡,以绥无者。

于是救困大荒,有祷无祭。国不称乐,企不满壑,刑法不修,舍用振穹,君亲迅方,卿参告籴,余子倅运,开□同食,民不藏粮,日有匡。俾民畜,唯牛羊,与民大疾惑,杀一人无赦。男守疆,戎禁不出,五库不膳,丧处无度,察以薄资。礼无乐,宫不帏,嫁娶不以时,宾旅设位有赐。

\hypertarget{header-n56}{%
\subsection{卷二}\label{header-n56}}

\hypertarget{header-n57}{%
\subsubsection{武称解}\label{header-n57}}

大国不失其威,小国不失其卑。敌国不失其权。岠险伐夷,并小夺乱,□强攻弱而袭不正,武之经也。伐乱伐疾伐疫,武之顺也。贤者辅之,乱者取之,作者劝之,怠者沮之,恐者惧之,欲者趣之,武之用也。美男破老,美女破舌少。淫图破□,淫巧破时,淫乐破正,淫言破义,武之毁也。

赦其众,遂其咎,抚其□,助其囊,武之闲也。饵敌以分,而照其储,以伐辅德,追时之权,武之尚也。春违其农,秋伐其穑,夏取其麦,冬寒其衣服,春秋欲舒,冬夏欲亟,武之时也。长胜短,轻胜重,直胜曲,众胜寡,强胜弱,饱胜觊,肃胜怒,先胜后,疾胜迟,武之胜也。

追戎追戎无恪,穷寇不格,力倦气竭,乃易克,武之追也。既胜人,举旗以号令,命吏禁掠,无取侵暴,爵位不谦,田宅不亏,各宁其亲,民服如合,武之抚也。百姓咸骨,偃兵兴德,夷厥险阻,以毁其服,四方畏服,奄有天下,武之定也。

\hypertarget{header-n63}{%
\subsubsection{允文解}\label{header-n63}}

思静振胜,允文维记。昭告周行,维旌所在。收武释贿,无迁厥里,官校属职,因其百吏。公货少多,振赐穷士,救瘠补病,赋均田布。命夫复服,用损忧耻,孤寡无告,获厚咸喜。咸问外戚,书其所在,迁同氏姓,位之宗子。率用十五,绥用□安,教用显允,若得父母。宽以政之,孰云不听,听言靡悔,遵养时晦。晦明遂语,于时允武,死思复生,生思复所。

人知不弃,爱守正户,上下和协,靡敌不下。执彼玉慓,以居其宇,庶民咸畊,童壮无辅,无拂其取,通其疆土。民之望兵,若待父母。是故天下,一旦而定有四海。

\hypertarget{header-n68}{%
\subsubsection{大武解}\label{header-n68}}

武有六制:政、攻、侵、伐、搏、战。善政不攻,善攻不侵,善侵不伐,善伐不搏,善搏不战。

政有四戚、五和;攻有四攻、五良;侵有四聚、三敛;伐有四时、三兴;搏有三哀、四赦;战有六厉、五卫、六庠、五虞。

四戚:一内姓,二外婚,三友朋,四同里。五和:一有天无恶,二有人无隙,三同好相固,四同恶相助,五远窄不薄。此九者,政之因也。

四攻者,一攻天时,二攻地宜,三攻人德,四攻行利。五良:一取仁,二取知,三取勇,四取材,五取艺。知

四聚:一酌之以仁,二怀之以乐,三旁聚封人,四设围以信。三敛:一男女比,二工次,三祗人死。此七者,侵之酌也。

四时:一春违其农,二夏食其谷,三秋取其刈,四冬冻其葆。三兴:一政以和时,三伐乱以治,三伐饥以饱。此七者,伐之机也。

四赦:一胜人必嬴,二取威信复,三人乐生身,四赦民所恶。此七者,搏之来也。六厉:一仁厉以行,二知厉以道,三武厉以勇,四师厉以士,五校正厉御,六射师厉伍。五卫:一明仁怀恕,二明知辅谋,三明武摄勇,四明材摄士,五明艺摄官。五虞:一鼓走疑,二备从来,三佐车举旗,四采虞人谋,五后动扌然之。无竞无害,有功无败。

\hypertarget{header-n78}{%
\subsubsection{大明武解}\label{header-n78}}

畏严大武,曰维四方畏威,乃宁。天作武,修戎兵,以助义正违。顺天行五官,官候厥政,谓有所亡。城郭沟渠,高厚是量。既践戎野,备慎其殃,敬其严君,乃战赦。十艺必明,加之以十因,靡敌不荒。阵若云布,侵若风行,轻车翼卫,在戎二方。我师之穷,靡人不刚。

十艺:一大援,二明从,三余子,四长兴,五伐人,六刑余,七三疑,八闲书,九用少,十兴怨。十因:一树仁,二胜欲,三宾客,四通旅,五亲戚,六无告,七同事,八程巧,九□能,十利事。

艺因伐用,是谓强转,应天顺时,时有寒暑,风雨饥疾,民乃不处,移散不败,农乃商贾,委以淫乐,赂以美女。主人若杖,□至城下,高堙临内,日夜不解。方阵并功,云何能御。虽易必敬,是谓明武。

城高城高难平,湮之以土,开之以走路,俄传器橹。因风行火障水,水下惠,用元元,文诲其寡。旁隧外权,堕城湮溪,老弱单处,其辩乃离。既克和服,使众咸宜,竟其金革,是谓大夷。

\hypertarget{header-n85}{%
\subsubsection{小明武解}\label{header-n85}}

凡攻之道,必得地势,以顺天时,观之以今,稽之以古,攻其逆政,毁其地阻,立之五教,以惠其下。矜寡无告,实为之主。五教允中,枝叶代兴。国为伪巧,后宫饰女,荒田逐兽,田猎之所,游观是崇,台泉池在下,淫乐无既,百姓辛苦。上有困令,乃有极□,上困下腾,戎迁其野,敦行王法,济用金鼓。降一列阵,无悗怒□。按道攻巷,无袭门户。无受货赂,攻用弓弩,上下祷祀,靡神不下,具行冲梯,振以长旗。怀戚思终,左右愤勇,无食六畜,无聚子女,群振若雷,造于城下,鼓行参呼,以正什伍。上有轩冕斧钺,在下胜国若化,故曰明武。

\hypertarget{header-n89}{%
\subsubsection{大匡解}\label{header-n89}}

维周王在酆,三年遭天之大荒,作《大匡》,以诏牧其方,三州之侯咸率。王乃召冢卿、三老、三吏大夫、百执事之人,朝于大庭。问罢病之故,政事之失,刑罚之戾,哀乐之尤,宾客之盛,用度之费,及关市之征,山林之匮,田宅之荒,沟渠之害,怠堕之过,骄顽之虐,水旱之灾。曰:``不谷不德,政事不时,国家罢病,不能胥匡,二三子不尚助不谷,官考厥职,乡问其人,因其耆老,及其总害,慎问其故,无隐乃情。''及某日以告于庙,有不用命,有常不赦。王既发命,八食不举,百官质方,□不食饔。

及期日质明,王麻衣以朝,朝中无采衣。官考其职,乡问其利,因谋其灾,旁匡于众,无敢有违。诘退骄顽,方收不服,慎惟怠堕,什伍相保,动劝游居,事节说茂,农夫任户,户尽夫出。农廪分乡,乡命受粮,程课物征,躬竞比藏,藏不粥籴,籴不加均,赋洒其币,乡正保贷。成年不偿,信诚匡之类,以辅殖财。财殖足食,克赋为征,数口以食,食均有赋。外食不赡,开关通粮,粮穷不转,孤寡不废。滞不转留,戍城不留,□足以守,出旅分均,驰车送逝,旦夕运粮。于是告四方游旅旁生,忻通所在,津济道宿,所至如归。币租轻,乃作母,以行其子。易资贵贱,以均游旅,使无滞。无粥熟,无室市,权内外,以立均,无蚤暮,闾次均行。均行众从,积而勿□,以罚助均,无使之穷,平均无乏,利民不淫。无播蔬,无食种,以数度多少省用。祈而不宾祭,服漱不制。国不乡射,乐不墙合。屋有补无作。资农不败务。非公卿不宾,宾不过具。哭不留日,登降一等。庶人不独,葬伍有植,送往迎来,亦如之。有不用命,有常不违。

\hypertarget{header-n94}{%
\subsubsection{程典解}\label{header-n94}}

维三月,既生魄,文王令六州之侯,奉勤于商。商王用宗谗,震怒无疆,诸侯不娱,逆诸文王。文王弗忍,乃作《程典》,以命三忠。曰:``余体民,无小不敬,若毛在躬,拔之痛,无不省。政失患作,作而无备,死亡不诫,诫在往事,备必慎,备思地,思地慎制,思制慎人,思人慎德,德开,开乃无患。慎德必躬恕,恕以明德,德当天而慎下。下为上贷,力竞以让,让德乃行。慎下必翼上,上中立而下比,争省,和而顺;携乃争,和乃比。比事无政,无政无选,无选民乃顽,顽乃害上。故选官以明训,顽民乃顺,慎守其教,小大有度,以备灾寇。习其武诫,依其山川,通其舟车,利其守务。士大夫不杂于工商,士之子不知义,不可以长幼。工不族居,不足以给官;族不乡别,不可以入惠。为上不明,为下不顺无丑。轻其行,多其愚不知,慎地必为之图,以举其物,物其善恶。度其高下,利其陂沟,爱其农时,修其等列,务其土实。差其施赋,设得其宜,宜协其务,务应其趣。慎用必爱,工攻其材,商通其财,百物鸟兽鱼鳖,无不顺时。生穑省用,不滥其度,津不行火,薮林不伐,牛羊不尽齿不屠。美不害用,用乃思慎,□备不敬,不意多□,用寡立亲,用胜怀远,远格而迩安。于安思危,与始思终,于迩思备,于远思近,与老思和,不备无违,无违严戒。''

\hypertarget{header-n98}{%
\subsubsection{程寤解}\label{header-n98}}

惟王元祀正月既生霸,大姒梦见商廷唯棘,乃小子发取周廷梓树于厥外,化为松柏棫柞。寤惊,告王。王弗敢占,诏太子发,俾灵名总祓。祝祈祓王,巫率祓大姒,宗丁祓太子发,敝告宗祊社稷,祈于六末山川,攻于商神,望,烝,占于明堂。王及太子发并拜吉梦,受商命于皇上帝。兴,曰:``发!汝敬听吉梦。朋棘雠梓松,梓松柏副,棫包柞,柞化为雘。呜呼!何警非朋?何戒非商?何用非树?树因欲,不违材。如天降疾,旨味既用,不可乐,时不远。惟商戚在周,周戚在商。欲惟柏梦,庶言□,□有勿亡秋。明武畏,如棫柞无根。呜呼,敬哉!朕闻周长不贰,务择用周,果拜不忍,绥用多福。惟梓敝,不义芃于商。俾行量亡乏,明明在向,惟容纳棘,抑亡勿用,不惎,思卑柔和顺,生民不灾,怀允。呜呼!何监非时?何务非和?何畏非文?何保非道?何爱非身?何力非人?人谋强,不可以藏。后戒,后戒,人用汝谋,爱日不足。''

\hypertarget{header-n102}{%
\subsubsection{秦阴解(亡)}\label{header-n102}}

已佚

\hypertarget{header-n106}{%
\subsubsection{九政解(亡)}\label{header-n106}}

已佚

\hypertarget{header-n110}{%
\subsubsection{九开解(亡)}\label{header-n110}}

已佚

\hypertarget{header-n114}{%
\subsubsection{刘法解(亡)}\label{header-n114}}

已佚

\hypertarget{header-n118}{%
\subsubsection{文开解(亡)}\label{header-n118}}

已佚

\hypertarget{header-n122}{%
\subsubsection{保开解(亡)}\label{header-n122}}

已佚

\hypertarget{header-n126}{%
\subsubsection{八繁解(亡)}\label{header-n126}}

已佚

\hypertarget{header-n130}{%
\subsection{卷三}\label{header-n130}}

\hypertarget{header-n131}{%
\subsubsection{酆保解}\label{header-n131}}

维二十三祀,庚子朔,九州之侯,咸格于周,王在酆,昧爽,立于少庭。王告周公旦曰:``呜呼,诸侯咸格来庆,辛苦役周,吾何保守,何用行?''

旦拜手稽首曰:``商无无道,弃德刑范,欺侮群臣,辛苦百姓,忍辱诸侯,莫大之纲,福其亡,亡人惟庸。王其祀德纯礼,明允无二,卑位柔色,金声以合之。''

王乃命三公九卿及百姓之人,曰:``恭敬齐洁,咸格而祀于上帝。''商馈始于王,因飨诸侯。重礼生存吏,出送于郊,树昏于崇,内备五祥、六卫、七厉、十败、四葛,外用四蠹、五落、六冗、七恶。

五祥:一君选择,二官得度,三务不舍,四不行赂,五察民困。六卫:一明仁怀恕,二明照设谋,三明戒摄勇,四明才摄士,五明德摄官,六明命摄政。七厉,一翼勤厉务,二动正厉民,三静兆厉武,四翼艺厉物,五翼言厉复,六翼敬厉众,七翼厉道。十败:一佞人败朴,二谄言毁积,三阴资自举,四女货速祸,五比党不拣,六佞说鬻狱,七神龟败卜,八宾祭推谷,九忿言自辱,十异姓乱族。四葛:一葛其农时不移,二费其土虑不化,三正赏罚、狱无矜奇,四葛其戎谋,族乃不罚。

四蠹:一美好怪奇以治之,三淫言流说以服之,三群巧仍兴以力之,四神巫灵宠以惑之。五落:一示吾贞以移其名,二微降霜雪以取松柏,三信蟜萌莫能安宅,四厚其祷巫,其谋乃获,五流德飘枉,以明其恶。六容:一游言,二行商工,三军旅之庸,四外风之所扬,五困失而亡,作事应时,时乃丧,六厚使以往来其所藏。七恶:一以物角兵,二令美其前而厚其伤,三闲于大国,安得吉凶,四交其所亲,静之以物,则以流其身,五率诸侯以朝贤人,而己犹不往,六令之有求遂以生尤,七见亲所亲,勿与深谋,命友人疑。''

旦拜曰:``呜呼,王孙其尊天下,适无见过,过适无好,自益以明而迹。呜呼,敬哉!视无祥、六卫、七厉、十败、四葛,不修国乃不固,务周四蠹、五落、六容、七恶,不使不允,不率不缓,反以自薄。呜呼,深念之哉!重维之哉!不深乃权不重,从权乃微,不从乃溃,溃不可复,戒后人其用汝谋。''

王曰:``允哉!''

\hypertarget{header-n135}{%
\subsubsection{大开解}\label{header-n135}}

维王二月,既生魄,王在酆,立于少庭,兆墓九开,开厥后人。八儆、王戒。八儆:一□旦于开,二躬修九过,三族修九禁,四无竞维义,五习用九教,六□用守备,七足用九利,八宁用怀□。五戒:一祗用谋宗,二经内戒工,三无远亲戚,四雕无薄□,五祷无忧玉,及为人尽不足。王拜儆无后人,谋竞不可以藏,戒后人其用汝谋,维宿不悉,日不足。

\hypertarget{header-n139}{%
\subsubsection{小开解}\label{header-n139}}

维三十有五祀,王念曰:``□所□,正月丙子,拜望食无时,汝开后嗣,谋曰:呜呼,于来,后之人,余闻在昔,曰明明非常维德,曰为明食无时。汝夜何修非躬,何慎非言,何择非德。呜呼,敬之哉。汝恭闻不命,贾粥不雠,谋念之哉。不索祸招,无曰不免不庸不茂不次,人灾不谋,迷弃非人。

``朕闻用人不以谋说,说恶謟言,色不知适,适不知谋,谋泄,汝躬不允。呜呼,敬之哉,后之人。朕闻曰:谋有共軵如乃而舍人之好。佚而无穷、贵而不傲、富而不骄、两而不争、闻而不遥、遥而不绝、穷而不匮者,鲜矣。汝辩斯何乡非翼,维有其枳,枳亡重大,害小不堪柯引,维德之用,用皆在国,谋大,鲜无害。

``呜呼!汝何敬非时,何择非德,德枳维大人,大人枳维公,公枳维卿,卿枳维大夫,大夫枳维士。登登皇皇,君枳维国,国枳维都,都枳维邑,邑枳维家,家枳维于无疆。动有三极,用有九因,因有四戚、五私。极:明与与,有畏劝,汝何畏非义,何畏非世,何劝非乐,谋获三极,无疆动获。九因:无限务用三德,顺攻奸□,言彼翼翼,在意,仞时德。春育素草肃疏数满,夏育长美,柯华务水潦,秋初艺不节落,冬大刘,倍信,何谋本□,时岁至天视。

``呜呼,汝何监非时,何务非德,何兴非因,何用非极。维周于民,人谋竞,不可以后戒,后戒宿,不悉日,不足。''

\hypertarget{header-n143}{%
\subsubsection{文儆解}\label{header-n143}}

维文王告梦,惧后嗣之无保,庚辰诏太子发曰:``汝敬之哉!民物多变,民何乡非利,利维生痛,痛维生乐,乐维生礼,礼维生义,义维生仁。呜呼,敬之哉!民之适败,上察下遂信,何乡非私,私维生抗,抗维生夺,夺维生乱,乱维生亡,亡维生死。呜呼,敬之哉!汝慎守勿失,以诏有司,夙夜勿忘,若民之向引。汝何慎非遂,遂时不远,非本非标,非微非煇壤非壤,不高水,非水不流。呜呼,敬之哉!倍本者槁。汝何葆非监,不维一保监顺时,维周于民之适,败无有时,盍后戒,后戒谋,念勿择。''

\hypertarget{header-n147}{%
\subsubsection{文传解}\label{header-n147}}

文王受命之九年,时维暮春,在鄗召太子发曰:``呜呼!我身老矣,吾语汝。我所保与我所守,传之子孙。吾厚德而广惠,忠信而志爱,人君之行。不为骄侈,不为泰靡,不淫于美,括柱茅茨,为民爱费。山林非时,不升斤斧,以成草木之长,川泽非时,不入网罟,以成鱼鳖之长。不鹿弭不卵,以成鸟兽之长,畋渔以时,童不夭胎,马不驰骛,土不失宜。土可犯材,可蓄润湿,不谷树之竹苇莞蒲,砾石不可谷,树之葛木,以为絺绤,以为材用。

``故凡土地之间者,胜任裁之,并为民利。是鱼鳖归其泉,鸟归其林,孤寡辛苦,咸赖其生,以遂其材。工匠以为其器,百物以平其利,商贾以通其货。工不失其务,农不失其时,是谓和德。土多民少,非其土也。土少人多,非其人也。是故土多发政,以漕四方,四方流之。土少安帑,而外其务方输。《夏箴》曰:中不容利,民乃外次。《开望》曰:土广无守,可袭伐;土狭无食,可围竭。二祸之来,不称之灾。天有四殃,水旱饥荒,其至无时,非务积聚,何以备之。《夏箴》曰:小人无兼年之食,遇天饥,妻子非其有也。大夫无兼年之食,遇天饥,臣妾舆马非其有也。戒之哉!弗思弗行,至无日矣。

``不明开塞禁舍者,其如天下何。各修其学而尊其名,圣人制之。故诸横生尽以养从,从生尽以养一丈夫。无杀夭胎,无伐不成材,无堕四时。如此者十年,有十年之积者王;有五年之积者霸,无一年之积者亡。生十杀一者,物十重,生一杀十者,物顿空。十重者王,顿空者亡。

``兵强胜人,人强胜天,能制其有者,则能制人之有。令行禁止,王始。出一曰神明,出二曰分光,出三曰无适异,出四曰无适与。无适与者亡。''

\hypertarget{header-n151}{%
\subsubsection{柔武解}\label{header-n151}}

维王元祀一月,既生魄,王召周公旦曰:``呜呼,维在王考之绪功,维周禁五戎,五戎不禁,厥民乃淫。一曰土观幸时,政匮不疑;二曰狱雠刑蔽,奸吏济贷;三曰声乐□□,饰女灭德;四曰维势是辅,维祷是怙;五曰盘游安居,枝叶维落。五者不距,自生戎旅。

``故必以德为本,以义为术,以信为动,以成为新,以决为计,以节为胜。务在审时,纪纲为序,和均□里以匡辛苦。见寇□戚,靡适无□。胜国若化,不动金鼓,善战不斗。故曰柔武,四方无拂,奄有天下。''

\hypertarget{header-n155}{%
\subsubsection{大开武解}\label{header-n155}}

维王一祀二月,王在酆,密命。访于周公旦,曰:``呜呼!余夙夜维商,密不显,谁知。告岁之有秋。今余不获其落,若何?''周公曰:``兹在德,敬在周,其维天命,王其敬命。远戚无十,和无再失,维明德无佚。佚不可还,维文考恪勤,战战何敬,何好何恶,时不敬,殆哉!''

王拜曰:``允哉!余闻国有四戚、五和、七失、九因、十淫。非不敬,不知。今而言维格,余非废善以自塞,维明戒是祗。''

周公拜曰:``兹顺天,天降寤于程,程降因于商。商今生葛,葛右有周。维王其明用老和之言,言孰敢不格。四戚:一内同外,二外婚姻,三官同师,四哀同劳。五和:一有天维国,二有地维义,三同好维乐,四同恶维哀,五远方不争。七失:一离在废,二废在祗,三比在门,四谄在内,五私在外,六私在公,七公不违。九因:一神有不飨,二德有不守,三才有不官,四事有不均,五两有必争,六富有别,七贪有匮,八好有遂,九敌有胜。十淫:一淫政破国,动不时,民不保;二淫好破义,言不协,民乃不和;三淫乐破德,德不纯,民乃失常;四淫动破丑,丑不足,民乃不让;五淫中破礼,礼不同,民乃不协;六淫采破服,服不度,民乃不顺;七淫文破典,典不式教,民乃不类;八淫权破故,故不法,官民乃无法;九淫贷破职,百官令不承;十淫巧破用,用不足,百意不成。呜呼,十淫不违,危哉!今商维兹,其唯第兹命不承,殆哉!若人之有政令,废令无赦。乃废天之命,讫文考之功绪,忍民之苦,不祥。若农之服田务耕而不耨,维草其宅,既秋而不获,维禽其飨之,人而获饥,云谁哀之。''

王拜曰:``格乃言。呜呼,夙夜战战,何畏非道,何恶非是。不敬,殆哉!''

\hypertarget{header-n159}{%
\subsubsection{小开武解}\label{header-n159}}

维四月乙未日,武王成辟,四方通殷,命有国。惟一月丙午,旁生魄,若翼日丁未,王乃步自于周,征伐商王纣。越若来二月既死魄,越五日,甲子朝,至接于商。则咸刘商王纣,执矢恶臣百人。太公望命御方来,丁卯至,告以馘俘。

戊辰,王遂御循追祀文王。时日王立政。吕他命伐越、戏、方,壬申荒新至,告以馘俘。侯来命伐,靡集于陈。辛巳,至,告以馘俘。甲申,百唶以虎贲誓命伐卫,告以亳俘。

辛亥,荐俘殷王鼎。武王乃翼,矢慓矢宪,告天宗上帝。王不革服,格于庙,秉语治庶国,籥入九终。王烈祖自太王、太伯、王季、虞公、文王、邑考以列升,维告殷罪,籥人造,王秉黄钺,正国伯。壬子,王服衮衣,矢琰格庙,籥人造王,秉黄钺,正邦君。

癸丑,荐殷俘王士百人。籥人造王矢琰、秉黄钺、执戈王奏庸,大享一终,王拜手,稽首。王定奏庸,大享三终。甲寅,谒戎殷于牧野,王佩赤白畤,籥人奏,武王入,进万献。明明三终。

乙卯,籥人奏崇禹生开三终,王定。庚子,陈本命,伐磨百韦,命伐宣方、新荒,命伐蜀。乙巳,陈本命新荒蜀磨,至告禽霍侯、艾侯,俘佚侯,小臣四十有六,禽御八百有三百两,告以馘俘。百谓至,告以禽宣方,禽御三十两,告以馘俘百韦,命伐厉,告以馘俘。武王狩,禽虎二十有二,猫二,糜五千二百三十五,犀十有二,氂七百二十有一熊百五十有一,罴百一十有八,豕三百五十有二,貉十有八,麈十有六,麝五十,糜三十,鹿三千五百有八。武王遂征四方,凡憝国九十有九国,馘磨亿有十万七千七百七十有九,俘人三亿万有二百三十。凡服国六百五十有二。

时四月,既旁生魄,越六日,庚戌,武王朝,至燎于周,维予冲子绥文。武王降自车,乃俾史佚繇书于天号。武王乃废于纣矢恶臣人百人,伐右厥甲孝子鼎大师。伐厥四十夫,家君、鼎帅、司徒、司马,初厥于郊号。武王乃夹于南门,用俘,皆施佩衣,衣先馘入。武王在祀,太师负商王纣,县首白畤,乃以先馘入燎于周庙。

若翼日辛亥,祀于位,用籥于天位。越五日乙卯,武王乃以庶祀馘于国周庙,翼予冲子,断牛六,断羊二。庶国乃竟,告于周庙,曰:``古朕闻文考修商人典,以斩纣身,告于天于稷。用小牲羊犬豕于百神水土、于誓社。''曰:``惟予冲子,绥文考,至于冲子,用牛于天、于稷,五百有四,用小牲羊豕于百神水土社三千七百有一。''

益之商王纣于南郊。时甲子夕,商王纣取天智玉琰五,环身,厚以自焚。凡厥有庶,告焚玉四千。五日,武王乃俾于千人,亲爱之四千庶玉,则销天智玉五,在火中不销。凡天智玉,武王则宝与同。凡武王俘商旧玉有百万。

\hypertarget{header-n163}{%
\subsubsection{宝典解}\label{header-n163}}

王若曰:``告尔伊旧何父□□□□几耿肃执,乃殷之旧官人序文□□□□,及太史比、小史昔,及百官里居献民□□□来尹师之敬诸戒疾听朕言,用胥生蠲尹王曰:嗟尔众,予言,若敢顾天命,予来致上帝之威命,明罚。今惟新诰命尔,敬诸朕话言,自一言至于十话,言其惟明命尔。王曰:在昔在昔后稷,惟上帝之言,克播百谷,登禹之绩,凡在天下之庶民,罔不维后稷之元谷用蒸享。在商先誓王明祀上帝□□□□亦维我后稷之元谷用告和用胥饮食。肆商先誓,王维厥故,斯用显我西土。今在商纣,昏忧天下,弗显上帝,昏虐百姓,奉天之命,上帝弗显,乃命朕文考,曰殪商之多罪。纣肆予小在发弗敢往天命,朕考胥翕稷政,肆上帝曰:必伐之。予惟甲子,克致天之大罚□帝之来革纣之□。予亦无敢违大命,敬诸。昔在我西土,我其有言,胥告商之百无罪。其维一夫。予既殛纣,承天命,予亦来休命,尔百姓里居君子,其周即命。□□□□□□□□□□□□□□□□□□□□□□□□□□□□□尔冢邦君无敢其有不告见互我有周,其比冢邦君,我无攸爱。上帝曰:必伐之。今予惟明告尔,予其往追□纣遂走秦集之于上帝。天王其有命尔百姓献民其有缀艿。夫自敬其有斯天命,不令尔百姓无告。西土疾勤,其斯有何重天维用重勤,兴起我,罪勤我。无克乃一心。尔多子其人,自敬之类天,永休于我西土,尔百姓其亦有安处在彼。宜在天命□及恻乱。予保奭其介,有斯勿用天命。若朕厣,在周曰:商百姓无罪,朕命在周,其乃先作我肆罪疾,予惟以先王之道御复正尔百姓,越则非朕负乱。惟尔在我王曰:百姓,我闻古商先誓王,成汤克辟上帝,保生商民,克用三德,疑商民,弗怀用辟厥辟。今纣弃成汤之典,肆上帝命我小国曰:革商国,肆予明命汝百姓。其斯弗用朕命,其斯尔冢邦君,商庶百姓,予则□刘灭之。王曰:靃予天命,维既咸汝克承天休于我有周,斯小国于有命不易,昔我盟津,帝休辨商其有何国命,予小子肆我殷戎,亦辨百度□□美左右予。予肆刘殷之命。今予维笃滪尔,予史、太史违我,黡视尔靖疑,胥敬,请其斯一话,敢逸僭予,则上帝之明命,予尔拜,拜□百姓,越尔庶义、庶刑,予维及西土,我乃其来,即刑乃,敬之哉!庶听朕言,罔胥告。

\hypertarget{header-n167}{%
\subsubsection{酆谋解}\label{header-n167}}

维王克殷,国君诸侯、乃厥献民征主、九牧之师见王与殷郊。王乃升汾之阜以望商邑,永叹曰:``呜呼,不淑,兑天对。''遂命一日,维显畏弗忘。王至于周,自鹿至于丘中。具明不寝,王小子御告叔旦,叔旦亟奔即王,曰:``久忧劳问,害不寝?''曰:``安予告汝。''

王曰:``呜呼,旦惟天不享于殷,发之未生,至于今六十年,夷羊在牧,飞鸿满野。天不享于殷,乃今有成。维天建殷,厥征天民,名三百六十夫,弗顾,亦不宾灭。用戾于今。呜呼于忧,兹难近饱于恤辰,是不室。我未定天保,何寝能欲。''

王曰:``旦,予克致天之明命,定天保,依天室,志我其恶,专从殷王纣,日夜劳来,定我于西土。我维显服,及德之方明。''

叔旦泣涕于常悲,不能对王。□□传于后王。王曰:``旦,汝维朕达弟,予有使汝,汝播食不遑食,矧其有乃室。今维天使子,惟二神授朕灵期,予未致,予休,予近怀子。朕室汝,维幼子大有知。昔皇祖厎于今,勖厥遗,得显义,告期付于朕身,肆若农服田,饥以望获。予有不显。朕卑皇祖不得高位于上帝。汝幼子庚厥心,庶乃来班,朕大肆环兹于有虞,意乃怀厥妻子,德不可追于上民,亦不可答于朕,下不宾在高祖,维天不嘉于降来省,汝其可瘳于兹,乃今我兄弟相后,我筮龟其何所即。今用建庶建。''

叔旦恐,泣涕其手。王曰:``呜呼,旦!我图夷,兹殷,其惟依天,其有宪命,求兹无远。天有求绎,相我不难。自洛汭延于伊汭,居阳无固,其有夏之居。我南望过于三途,北望过于有岳,鄙顾瞻过于河宛,瞻于伊洛。无远天室,其曰兹曰度邑。''

\hypertarget{header-n171}{%
\subsubsection{寤儆解}\label{header-n171}}

惟十有二祀四月,王告梦,丙辰,出金枝,郊宝,开和细书,命诏周公旦立后嗣,属小子诵文及宝典。王曰:``呜呼,敬之哉!汝勤之无盖□周未知所周不知商□无也。朕不敢望,敬守勿失,以诏宥。''小子曰:``允哉。''``汝夙夜勤性之无穷也。''

\hypertarget{header-n175}{%
\subsubsection{武顺解}\label{header-n175}}

维王不豫,于五日召周公旦,曰:``呜呼,敬之哉!昔天初降命于周,维在文考,克致天之命。汝维敬哉!先后小子,勤在维政之失。政有三机、五权,汝敬之哉。克中无苗,以保小子于位。

``三机:一疑家,二疑德,三质士。疑家无授众,疑德无举士,直士无远齐。吁,敬之哉!天命无常,敬在三机。五权:一曰地,地以权民;二曰物,物以权官;三曰鄙,鄙以权庶;四曰刑,刑以权常;五曰食,食以权爵。不承括食不宣,不宣授臣。极赏则淈,淈得不食。极刑则仇,仇至乃别,鄙庶则奴,奴乃不灭。国大则骄,骄乃不给,官庶则荷,荷至乃辛。物庶则爵,乃不和。地庶则荒,荒则聂。人庶则匮,匮乃匿。呜呼,敬之哉!汝慎和,称五权,维中是以,以长小子于位,实维永宁。''

\hypertarget{header-n179}{%
\subsubsection{武穆解}\label{header-n179}}

成王元年,大开告用。周公曰:``呜呼,余夙夜之勤,今商孽竞时,逋播以以辅。余何循,何循何慎,王其敬天命,无易天不虞。在昔文考躬修五典,勉兹九功,敬人畏天,教以六则、四守、五示、三极,祗应八方,立忠协义乃作。

``三极:一天有九列,表使阴阳;二地有九州,别处五行;三人有四佐,佐官维明,五示显允明所望。五示:一明位示士,二明惠示众,三明主示宁,四安宅示孥,五利用示产。产足不穷,家怀思终,主为之宗,德以抚众,众和乃同。四守:一政尽人材,材尽致死,二土守其城沟,三障水以御寇,四大有沙炭之政。六则:一和众,二发郁,三明怨,四转怒,五惧疑,六因欲。九功:一宾好在笥,三淫巧破制,三好危破事,四任利败功,五神巫动众,六尽哀民匮,七荒乐无别,八无制破教,九任谋生诈。

``和集,集以禁,实有离,莫遂通其五典。一言父典祭,祭祀昭天,百姓若敬;二显父登德,德降为则,则信民宁;三正父登过,过慎于武,设备无盈。四机父登失,修□□官,官无不敬;五□□□□,制哀节用,政治民怀。五典有常,政乃重开之守,内则顺意,外则顺敬,内外不爽,是曰明王。''

王拜曰:``允哉,维予闻曰何乡非怀,怀人惟思,思若不及,祸格无日。式皇敬哉。余小子思继厥常,以昭文祖,定武考之列。呜呼,余夙夜不宁。''

\hypertarget{header-n187}{%
\subsection{卷四}\label{header-n187}}

\hypertarget{header-n191}{%
\subsubsection{和寤解}\label{header-n191}}

武王克殷,乃立王子禄父,俾守商祀。建管叔于东,建蔡叔于殷,俾监殷臣。武王既归,乃岁十二月,崩,镐肂于岐周。周公立相,天子三叔及殷东徐奄及熊盈以略。

周公、召公内弭父兄,外抚诸侯。元年夏六月,葬武王于毕。二年,又作师旅,临卫政殷,殷大震溃。降辟三叔,王子禄父北奔,管叔经而卒,乃囚蔡叔于郭凌。凡所征熊盈族十有七国,俘维九邑。俘殷献民,迁于九毕。俾康帅宇于殷,俾中旄父宇于东。周公敬念于后,曰:``予畏同室克追,俾中天下。''及将致政,乃作大邑成周于中土。城方千七百二十丈,郛方七十里。南系于洛水,北因于郏山,以为天下之大凑。制郊甸,方六百里,国西土,为方千里。分以百县,县有四郡,郡有四鄙,大县城方王城三之一,小县立城,方王城九之一。都鄙不过百室,以便野事。农居鄙,得以庶士,士居国家,得以诸公大夫。凡工贾胥市臣仆州里,俾无交为。

乃设丘兆于南郊,以祀上帝,配以后稷,日月星辰先王皆与食。封人社遣,诸侯受命于周,乃建大社与国中,其遣东青土,南赤土,西白土,北骊土,中央亹以黄土,将建诸侯,凿取其方,一面之土,焘以黄土,苴以白茅,以为土封。故曰,受列土于周室。乃位五宫、大庙、宗宫、考宫、路寝、明堂,咸有四阿,反坫,亢重、郎、常累、复格藻棁,设移旅楹舂常画旅。内阶玄阶,堤唐山廧,应门库台玄阃。

\hypertarget{header-n195}{%
\subsubsection{武寤解}\label{header-n195}}

维正月庚午,周公格左闳门,会群门。曰:``呜呼!下邑小国,克有耇老,据屏位,建沈人,罔不用明刑。维其开告于予嘉德之说,命无辟王,小至于大。我闻在昔,有国誓王之不绥于恤。乃维其有大门宗子,势臣,罔不茂扬肃德,讫亦有孚,以助厥辟,勤王国王家。乃方求论择元圣武夫,羞于王所。其善臣,以至于有分私子,苟克有常,罔不允通,咸献言,在于王所。人斯是助,王恭明祀,敷明刑。王用有监明宪,朕命用克,和有成,用能承天嘏命。百姓兆民,用罔不茂在王庭。先用有劝,永有□于上下。人斯既助,厥勤劳王家。先人神祗,报职用休,俾嗣在厥家。王国用宁,选用格,□能稼穑,咸饲天神,戎兵克慎,军用克多。王用奄有四邻远士,丕承万子孙用末,被先王之灵光。至于厥后嗣,弗见先王之明刑,维时及胥学于非夷,一以家相厥室,弗恤王国王家,维德是用。以昏求臣,作威不详,不屑惠听无辜之乱辞,是羞于王。王阜良乃惟不顺之言,于是人斯乃非维直以应,维作诬以对,俾无依无助。譬如畋犬骄,用逐禽,其犹不克有获。是人斯乃谗贼媢嫉,以不利于厥家国。譬若匹夫之有婚妻曰:予独服在寝,以自露厥家。媚夫有迩无远,乃食,盖善夫,俾莫通在于王所。乃维有奉,狂夫是阳是绳,以为上是授司事于正长。命用迷乱,狱用无成,小民率穑。保用无用,寿亡以嗣,天用弗保。媚夫先受殄罚,国亦不宁。呜呼,敬哉!监于兹!朕维其及,朕荩臣夫,明尔德,以助予一人忧。无维乃身之暴,皆恤尔,假予德宪,资告予元。譬若众畋,常抚予险,乃而予于济。汝无作。

\hypertarget{header-n199}{%
\subsubsection{克殷解}\label{header-n199}}

维正月,既生魄,王访于周公曰:``呜呼,朕闻维时兆厥工,非不显,朕实不明。维士非不务,而不得助,大则骄,小则慑,慑谋不极。予重位与轻服。非其得福厚用遗。庸止生隙,庸行信贰,众辑群政,不辑自匿。呜呼,予夙勤之,无或告余。非不念,念不知。''

周公曰:``于敢称乃武考之言曰:微言入心,夙喻动众,大乃不骄,行惠于小,小乃不慑。连官集乘,同忧若一,谋有不行,予惟重告尔。庸厉□以饵士,权先申之,明约必遣之。其位不尊,其谋不阳。无不畏敬,材在四方。无擅于人,塞匿勿行,惠戚咸服,孝悌乃明,明立威耻乱。使众之道,抚之以惠,内姓无感,外姓无谪。人知其罪,上之明审教幼,乃勤贫贱,制□设九备,乃无乱谋。九备:一忠正不荒,美好乃不作恶,\ldots{}\ldots{}四□说声色忧乐盈匿,五硕信伤辩曰费□□,六出观好怪,内乃淫巧,七□□谋躁,内乃荒异,八□□好威,民众日逃,九富宠极足,是大极内,心其离。九备既明,我贵保之,应协以动,远尔同功。谋和适用,复以观之,上明任意,援贡有备。聚财多□,以援成功,克禁淫谋,众匿乃雍。顺得以动,人以立行。辑佐之道,上必尽其志,然后得其谋。无□其信,虽危不动。□□以昭,其乃得人。上危而转,下乃不亲。''

王拜曰:``允哉,允哉,敬行天道。''

\hypertarget{header-n203}{%
\subsubsection{大匡解}\label{header-n203}}

惟是有三祀,王在管,管叔自作殷之监,东隅之侯咸受赐于王,王乃旅之,以上东隅。

用大匡,顺九则、八宅、六位。宽俭恭敬,夙夜有严。昭质非朴,朴有不明,明执于私,私回不中,中忠于欲,思慧丑诈。昭信非展,展尽不伊,伊言于允,思复丑谮。昭让非背,背党雍德,德让于敬,思贤丑争。昭位非忿,忿非□直,直立于众,思直丑比。昭政非闲,闲非远节,节政于进,思止丑残。昭静非穷,穷居非意,意动于行,思静丑躁。昭洁非为,为穷非涓,涓洁于利,思义丑贪。昭因非疾疾非不贞,贞固于事,思任丑诞。昭明九则,九丑自齐,齐则曰知,悖则死勇。勇如害上,则不登于明堂。明堂所以明道,明道惟法。法人惟重老,重老惟宝。呜呼在昔,文考战战惟时,祗祗汝其。此有夺误夙夜济济,无竞惟人,惟允惟让,不远群正,不迩谗邪。入汝不时,行汝害于士,士惟都人,孝悌子孙。

不官则不长,官戒有敬。官□朝道,舍宾祭器,曰八宅。绥比新、故、外、内、贵、贱曰六位。大官作为武,小官承长。大匡封摄,外用和大。中匡用均,劳故礼心。小匡用惠,施舍静众。禁请无愿,顺生分杀,不忘不惮。俾若九则,生敬在国,国咸顺,顺维敬,敬维让,让维礼。辟不及宽,有永假。

\hypertarget{header-n207}{%
\subsubsection{文政解}\label{header-n207}}

惟十有三祀,桑在管,管蔡开宗循,禁九慝,昭九行,济九丑,尊九德,止九过,务九胜,倾九戒,固九守,顺九典。

九慝:一不类,二不服,三不则,四□务有不功,五外有内通,六幼不观国,七闾不通径,八家不开刑,九大禁不令路径。

九行:一仁,二行,三让,四信,五固,六治,七义,八意,九勇。

九丑:思勇丑忘,思意丑变,思义丑□,思治丑乱,思固丑专,思信丑奸,思让丑残,思行丑顽,思仁丑会亹。

九德:一忠,二慈,三禄,四赏,五民之利,六商工受资,七祗民之死,八无夺农,九足民之财。

九过:一视民傲,二听民暴,三远慎而近藐,四法令□乱,五仁善是诛,六不察而好杀,七不念□害行,八□思前后,九偷其身不路而助无渔。

九胜:一□□□□,二□□□□,三同恶潜某,四同好和因,五师□征恶,六迎旋便路,七明赂施舍,八幼子移成,九迪名书新。

九戒:一内有柔成,而示有危倾,三旅有罢置,四乱有立信,五教有康经,六合详毁成,七邑守维人,八饥有兆积,九劳休无期。

九守:一仁守以均,二知守以等,三固守以典,四信守维假,五城沟守立
,六廉守以名,七戒守以信,八竞守以备,九国守以谋。

九典:一祗道以明之,二称贤以赏之,三典师以教之,四四戚以劳之,五位长以遵之,六群长以老之,七群丑以移之,八什长以行之,九戒卒以将之。呜呼,虚为害,无由不通,无虚不败。

\hypertarget{header-n211}{%
\subsubsection{大聚解}\label{header-n211}}

维武王胜殷,抚国绥民,乃观于殷政,告周公旦曰:``呜呼,殷政总总,若风草有所积,有所虚,和此如何?''

周公曰:``闻之文考,来远宾,廉近者,道别其阴阳之利,相土地之宜,水土之便。营邑制命之曰:大聚先诱之以四郊,王亲在之。大夫免列以选,赦刑以宽,复亡解辱削赦□重,皆有数,此谓行风。乃令县鄙商旅曰:能来三室者,与之一室之禄。辟开修道,五里有郊,十里有井,二十里有舍,远旅来至关,人易资,舍有委。市有五均,早暮如一,送行逆来,振乏救穷。老弱疾病,孤子寡独,惟政所先。民有欲畜,发令。

``以国为邑,以邑为乡,以乡为闾,祸灾相恤,资丧比服,五户为伍,以首为长,十夫为什,以年为长。合闾立教,以威为长,合族同亲,以敬为长。饮食相约,兴弹相庸,耦耕□耘,男女有婚,坟墓相连,民乃有亲。六畜有群,室屋既完,民乃归之。

``乡立巫医,具百药,以备疾灾,畜五味,以备百草。立勤人以职孤,立正长以顺幼,立职丧以恤死,立大葬以正同。立君子以修礼略,立小人以教用兵。立乡射以习容,春和猎耕耘,以习迁行。教芋与树艺,比长立职,与田畴皆通。立祭祀,与岁谷登下厚薄,此之谓德教。

``若其凶土陋民,贱食贵货,是不知政。山林薮泽,以因其□工匠,役工以攻其材;商贾趣市,以合其用。外商资贵而来,贵物益贱,五出贵物,以通其器。夫然则关夷市平,财无郁废,商不乏资,百工不失其时,无愚不教,则无穷乏,此谓和德。

``若有不言,乃政其凶,陂沟道路,藂苴\ldots{},丘坟不可树谷者,树以材木。春发枯槁,夏发叶荣,秋发实蔬,冬发薪蒸。以匡穷困。揖其民力,相更为师。因其土宜,以为民资,则生无乏用,死无传尸。此谓仁德。

``旦闻禹之禁,春三月,山林不登斧,以成草木之长;三月遄不入网罟,以成鱼鳖之长。且以并农力执,成男女之功。夫然则有生而不失其宜,万物不失其性,人不失七事
,天不失其时,以成万财。既成,放此为人。此谓正德。

``泉深而鱼鳖归之,草木茂而鸟兽归之;称贤使能,官有材而士归之;关市平,商贾归之;分地薄敛,农民归之。水性归下,农民归利。王若欲求天下民,社设其利,而民自至,譬之若冬日之阳,夏日之阴,不召而民自来。此谓归德。

``五德既明,民乃知常。''

武王再拜曰:``呜呼,允哉!天民侧侧,余知其极有宜。''乃召昆吾冶而铭之,藏府而朔之。

\hypertarget{header-n215}{%
\subsubsection{世俘解}\label{header-n215}}

维四月乙未日,武王成辟,四方通殷,命有国。惟一月丙午,旁生魄,若翼日丁未,王乃步自于周,征伐商王纣。越若来二月既死魄,越五日,甲子朝,至接于商。则咸刘商王纣,执矢恶臣百人。太公望命御方来,丁卯至,告以馘俘。

戊辰,王遂御循追祀文王。时日王立政。吕他命伐越、戏、方,壬申荒新至,告以馘俘。侯来命伐,靡集于陈。辛巳,至,告以馘俘。甲申,百唶以虎贲誓命伐卫,告以亳俘。

辛亥,荐俘殷王鼎。武王乃翼,矢慓矢宪,告天宗上帝。王不革服,格于庙,秉语治庶国,籥入九终。王烈祖自太王、太伯、王季、虞公、文王、邑考以列升,维告殷罪,籥人造,王秉黄钺,正国伯。壬子,王服衮衣,矢琰格庙,籥人造王,秉黄钺,正邦君。

癸丑,荐殷俘王士百人。籥人造王矢琰、秉黄钺、执戈王奏庸,大享一终,王拜手,稽首。王定奏庸,大享三终。甲寅,谒戎殷于牧野,王佩赤白畤,籥人奏,武王入,进万献。明明三终。

乙卯,籥人奏崇禹生开三终,王定。庚子,陈本命,伐磨百韦,命伐宣方、新荒,命伐蜀。乙巳,陈本命新荒蜀磨,至告禽霍侯、艾侯,俘佚侯,小臣四十有六,禽御八百有三百两,告以馘俘。百谓至,告以禽宣方,禽御三十两,告以馘俘百韦,命伐厉,告以馘俘。武王狩,禽虎二十有二,猫二,糜五千二百三十五,犀十有二,氂七百二十有一熊百五十有一,罴百一十有八,豕三百五十有二,貉十有八,麈十有六,麝五十,糜三十,鹿三千五百有八。武王遂征四方,凡憝国九十有九国,馘磨亿有十万七千七百七十有九,俘人三亿万有二百三十。凡服国六百五十有二。

时四月,既旁生魄,越六日,庚戌,武王朝,至燎于周,维予冲子绥文。武王降自车,乃俾史佚繇书于天号。武王乃废于纣矢恶臣人百人,伐右厥甲孝子鼎大师。伐厥四十夫,家君、鼎帅、司徒、司马,初厥于郊号。武王乃夹于南门,用俘,皆施佩衣,衣先馘入。武王在祀,太师负商王纣,县首白畤,乃以先馘入燎于周庙。

若翼日辛亥,祀于位,用籥于天位。越五日乙卯,武王乃以庶祀馘于国周庙,翼予冲子,断牛六,断羊二。庶国乃竟,告于周庙,曰:``古朕闻文考修商人典,以斩纣身,告于天于稷。用小牲羊犬豕于百神水土、于誓社。''曰:``惟予冲子,绥文考,至于冲子,用牛于天、于稷,五百有四,用小牲羊豕于百神水土社三千七百有一。''

益之商王纣于南郊。时甲子夕,商王纣取天智玉琰五,环身,厚以自焚。凡厥有庶,告焚玉四千。五日,武王乃俾于千人,亲爱之四千庶玉,则销天智玉五,在火中不销。凡天智玉,武王则宝与同。凡武王俘商旧玉有百万。

\hypertarget{header-n219}{%
\subsubsection{箕子解(亡)}\label{header-n219}}

已佚

\hypertarget{header-n223}{%
\subsubsection{耆德解(亡)}\label{header-n223}}

已佚

\hypertarget{header-n231}{%
\subsection{卷五}\label{header-n231}}

\hypertarget{header-n235}{%
\subsubsection{商誓解}\label{header-n235}}

王若曰:``告尔伊旧何父□□□□几耿肃执,乃殷之旧官人序文□□□□,及太史比、小史昔,及百官里居献民□□□来尹师之敬诸戒疾听朕言,用胥生蠲尹王曰:嗟尔众,予言,若敢顾天命,予来致上帝之威命,明罚。今惟新诰命尔,敬诸朕话言,自一言至于十话,言其惟明命尔。王曰:在昔在昔后稷,惟上帝之言,克播百谷,登禹之绩,凡在天下之庶民,罔不维后稷之元谷用蒸享。在商先誓王明祀上帝□□□□亦维我后稷之元谷用告和用胥饮食。肆商先誓,王维厥故,斯用显我西土。今在商纣,昏忧天下,弗显上帝,昏虐百姓,奉天之命,上帝弗显,乃命朕文考,曰殪商之多罪。纣肆予小在发弗敢往天命,朕考胥翕稷政,肆上帝曰:必伐之。予惟甲子,克致天之大罚□帝之来革纣之□。予亦无敢违大命,敬诸。昔在我西土,我其有言,胥告商之百无罪。其维一夫。予既殛纣,承天命,予亦来休命,尔百姓里居君子,其周即命。□□□□□□□□□□□□□□□□□□□□□□□□□□□□□尔冢邦君无敢其有不告见互我有周,其比冢邦君,我无攸爱。上帝曰:必伐之。今予惟明告尔,予其往追□纣遂走秦集之于上帝。天王其有命尔百姓献民其有缀艿。夫自敬其有斯天命,不令尔百姓无告。西土疾勤,其斯有何重天维用重勤,兴起我,罪勤我。无克乃一心。尔多子其人,自敬之类天,永休于我西土,尔百姓其亦有安处在彼。宜在天命□及恻乱。予保奭其介,有斯勿用天命。若朕厣,在周曰:商百姓无罪,朕命在周,其乃先作我肆罪疾,予惟以先王之道御复正尔百姓,越则非朕负乱。惟尔在我王曰:百姓,我闻古商先誓王,成汤克辟上帝,保生商民,克用三德,疑商民,弗怀用辟厥辟。今纣弃成汤之典,肆上帝命我小国曰:革商国,肆予明命汝百姓。其斯弗用朕命,其斯尔冢邦君,商庶百姓,予则□刘灭之。王曰:靃予天命,维既咸汝克承天休于我有周,斯小国于有命不易,昔我盟津,帝休辨商其有何国命,予小子肆我殷戎,亦辨百度□□美左右予。予肆刘殷之命。今予维笃滪尔,予史、太史违我,黡视尔靖疑,胥敬,请其斯一话,敢逸僭予,则上帝之明命,予尔拜,拜□百姓,越尔庶义、庶刑,予维及西土,我乃其来,即刑乃,敬之哉!庶听朕言,罔胥告。

\hypertarget{header-n239}{%
\subsubsection{度邑解}\label{header-n239}}

维王克殷,国君诸侯、乃厥献民征主、九牧之师见王与殷郊。王乃升汾之阜以望商邑,永叹曰:``呜呼,不淑,兑天对。''遂命一日,维显畏弗忘。王至于周,自鹿至于丘中。具明不寝,王小子御告叔旦,叔旦亟奔即王,曰:``久忧劳问,害不寝?''曰:``安予告汝。''

王曰:``呜呼,旦惟天不享于殷,发之未生,至于今六十年,夷羊在牧,飞鸿满野。天不享于殷,乃今有成。维天建殷,厥征天民,名三百六十夫,弗顾,亦不宾灭。用戾于今。呜呼于忧,兹难近饱于恤辰,是不室。我未定天保,何寝能欲。''

王曰:``旦,予克致天之明命,定天保,依天室,志我其恶,专从殷王纣,日夜劳来,定我于西土。我维显服,及德之方明。''

叔旦泣涕于常悲,不能对王。□□传于后王。王曰:``旦,汝维朕达弟,予有使汝,汝播食不遑食,矧其有乃室。今维天使子,惟二神授朕灵期,予未致,予休,予近怀子。朕室汝,维幼子大有知。昔皇祖厎于今,勖厥遗,得显义,告期付于朕身,肆若农服田,饥以望获。予有不显。朕卑皇祖不得高位于上帝。汝幼子庚厥心,庶乃来班,朕大肆环兹于有虞,意乃怀厥妻子,德不可追于上民,亦不可答于朕,下不宾在高祖,维天不嘉于降来省,汝其可瘳于兹,乃今我兄弟相后,我筮龟其何所即。今用建庶建。''

叔旦恐,泣涕其手。王曰:``呜呼,旦!我图夷,兹殷,其惟依天,其有宪命,求兹无远。天有求绎,相我不难。自洛汭延于伊汭,居阳无固,其有夏之居。我南望过于三途,北望过于有岳,鄙顾瞻过于河宛,瞻于伊洛。无远天室,其曰兹曰度邑。''

\hypertarget{header-n243}{%
\subsubsection{武儆解}\label{header-n243}}

惟十有二祀四月,王告梦,丙辰,出金枝,郊宝,开和细书,命诏周公旦立后嗣,属小子诵文及宝典。王曰:``呜呼,敬之哉!汝勤之无盖□周未知所周不知商□无也。朕不敢望,敬守勿失,以诏宥。''小子曰:``允哉。''``汝夙夜勤性之无穷也。''

\hypertarget{header-n247}{%
\subsubsection{五权解}\label{header-n247}}

维王不豫,于五日召周公旦,曰:``呜呼,敬之哉!昔天初降命于周,维在文考,克致天之命。汝维敬哉!先后小子,勤在维政之失。政有三机、五权,汝敬之哉。克中无苗,以保小子于位。

``三机:一疑家,二疑德,三质士。疑家无授众,疑德无举士,直士无远齐。吁,敬之哉!天命无常,敬在三机。五权:一曰地,地以权民;二曰物,物以权官;三曰鄙,鄙以权庶;四曰刑,刑以权常;五曰食,食以权爵。不承括食不宣,不宣授臣。极赏则淈,淈得不食。极刑则仇,仇至乃别,鄙庶则奴,奴乃不灭。国大则骄,骄乃不给,官庶则荷,荷至乃辛。物庶则爵,乃不和。地庶则荒,荒则聂。人庶则匮,匮乃匿。呜呼,敬之哉!汝慎和,称五权,维中是以,以长小子于位,实维永宁。''

\hypertarget{header-n251}{%
\subsubsection{成开解}\label{header-n251}}

成王元年,大开告用。周公曰:``呜呼,余夙夜之勤,今商孽竞时,逋播以以辅。余何循,何循何慎,王其敬天命,无易天不虞。在昔文考躬修五典,勉兹九功,敬人畏天,教以六则、四守、五示、三极,祗应八方,立忠协义乃作。

``三极:一天有九列,表使阴阳;二地有九州,别处五行;三人有四佐,佐官维明,五示显允明所望。五示:一明位示士,二明惠示众,三明主示宁,四安宅示孥,五利用示产。产足不穷,家怀思终,主为之宗,德以抚众,众和乃同。四守:一政尽人材,材尽致死,二土守其城沟,三障水以御寇,四大有沙炭之政。六则:一和众,二发郁,三明怨,四转怒,五惧疑,六因欲。九功:一宾好在笥,三淫巧破制,三好危破事,四任利败功,五神巫动众,六尽哀民匮,七荒乐无别,八无制破教,九任谋生诈。

``和集,集以禁,实有离,莫遂通其五典。一言父典祭,祭祀昭天,百姓若敬;二显父登德,德降为则,则信民宁;三正父登过,过慎于武,设备无盈。四机父登失,修□□官,官无不敬;五□□□□,制哀节用,政治民怀。五典有常,政乃重开之守,内则顺意,外则顺敬,内外不爽,是曰明王。''

王拜曰:``允哉,维予闻曰何乡非怀,怀人惟思,思若不及,祸格无日。式皇敬哉。余小子思继厥常,以昭文祖,定武考之列。呜呼,余夙夜不宁。''

\hypertarget{header-n255}{%
\subsubsection{作雒解}\label{header-n255}}

武王克殷,乃立王子禄父,俾守商祀。建管叔于东,建蔡叔于殷,俾监殷臣。武王既归,乃岁十二月,崩,镐肂于岐周。周公立相,天子三叔及殷东徐奄及熊盈以略。

周公、召公内弭父兄,外抚诸侯。元年夏六月,葬武王于毕。二年,又作师旅,临卫政殷,殷大震溃。降辟三叔,王子禄父北奔,管叔经而卒,乃囚蔡叔于郭凌。凡所征熊盈族十有七国,俘维九邑。俘殷献民,迁于九毕。俾康帅宇于殷,俾中旄父宇于东。周公敬念于后,曰:``予畏同室克追,俾中天下。''及将致政,乃作大邑成周于中土。城方千七百二十丈,郛方七十里。南系于洛水,北因于郏山,以为天下之大凑。制郊甸,方六百里,国西土,为方千里。分以百县,县有四郡,郡有四鄙,大县城方王城三之一,小县立城,方王城九之一。都鄙不过百室,以便野事。农居鄙,得以庶士,士居国家,得以诸公大夫。凡工贾胥市臣仆州里,俾无交为。

乃设丘兆于南郊,以祀上帝,配以后稷,日月星辰先王皆与食。封人社遣,诸侯受命于周,乃建大社与国中,其遣东青土,南赤土,西白土,北骊土,中央亹以黄土,将建诸侯,凿取其方,一面之土,焘以黄土,苴以白茅,以为土封。故曰,受列土于周室。乃位五宫、大庙、宗宫、考宫、路寝、明堂,咸有四阿,反坫,亢重、郎、常累、复格藻棁,设移旅楹舂常画旅。内阶玄阶,堤唐山廧,应门库台玄阃。

\hypertarget{header-n259}{%
\subsubsection{皇门解}\label{header-n259}}

维正月庚午,周公格左闳门,会群门。曰:``呜呼!下邑小国,克有耇老,据屏位,建沈人,罔不用明刑。维其开告于予嘉德之说,命无辟王,小至于大。我闻在昔,有国誓王之不绥于恤。乃维其有大门宗子,势臣,罔不茂扬肃德,讫亦有孚,以助厥辟,勤王国王家。乃方求论择元圣武夫,羞于王所。其善臣,以至于有分私子,苟克有常,罔不允通,咸献言,在于王所。人斯是助,王恭明祀,敷明刑。王用有监明宪,朕命用克,和有成,用能承天嘏命。百姓兆民,用罔不茂在王庭。先用有劝,永有□于上下。人斯既助,厥勤劳王家。先人神祗,报职用休,俾嗣在厥家。王国用宁,选用格,□能稼穑,咸饲天神,戎兵克慎,军用克多。王用奄有四邻远士,丕承万子孙用末,被先王之灵光。至于厥后嗣,弗见先王之明刑,维时及胥学于非夷,一以家相厥室,弗恤王国王家,维德是用。以昏求臣,作威不详,不屑惠听无辜之乱辞,是羞于王。王阜良乃惟不顺之言,于是人斯乃非维直以应,维作诬以对,俾无依无助。譬如畋犬骄,用逐禽,其犹不克有获。是人斯乃谗贼媢嫉,以不利于厥家国。譬若匹夫之有婚妻曰:予独服在寝,以自露厥家。媚夫有迩无远,乃食,盖善夫,俾莫通在于王所。乃维有奉,狂夫是阳是绳,以为上是授司事于正长。命用迷乱,狱用无成,小民率穑。保用无用,寿亡以嗣,天用弗保。媚夫先受殄罚,国亦不宁。呜呼,敬哉!监于兹!朕维其及,朕荩臣夫,明尔德,以助予一人忧。无维乃身之暴,皆恤尔,假予德宪,资告予元。譬若众畋,常抚予险,乃而予于济。汝无作。

\hypertarget{header-n263}{%
\subsubsection{大戒解}\label{header-n263}}

维王二祀一月,既生魄,王召周公旦曰:``呜呼,余夙夜忌商,不知道极,敬听以勤天命。''

周公拜手稽首曰:``在我文考,顺明三极,躬是四察,循用五行,戒视七顺,顺道九纪。三极既明,五行乃常,四祭既是,七顺乃辨,明势天道,九纪咸当,顺德以谋,罔惟不行。三极:一维天九星,二维地九州,三维人四左。四察,
一目察维极,二耳察维声,三口察维言,四心察维念。五行:一黑位水,二赤位火,三苍位木,四白位金,五黄位土。七顺:一顺天得时,二顺地得助,三顺年得和,四顺利财足,五顺得助明,六顺仁无失,七顺道有功。九纪:一辰以纪日,二宿以纪月,三日以纪德,四月以纪刑,五春以纪生,六夏以纪长,七秋以纪杀,八冬以纪藏,九岁以纪终。时候天视,可监时,时不失以知吉凶。''

``王拜曰:``允哉!余闻在昔,训典中规,非使罔有恪言,日正余不足。''

\hypertarget{header-n271}{%
\subsection{卷六}\label{header-n271}}

\hypertarget{header-n275}{%
\subsubsection{周月解}\label{header-n275}}

惟一月,既南至,昏昴见,日短件,基践长,微阳动于黄泉,阴降惨于万物。是月,斗柄建子,始昏北指,阳气亏,草木萌荡。日月俱起于牵牛之初,右回而行月,周天进一次,而与日合宿。日行月一次,而周天历舍于十有二辰,终则复始,是谓日月权舆。周正岁首,数起于时,一而成于十次,一为首,其义则然。凡四时,成岁,有春夏秋冬,各有孟仲季以名,十有二月,中气以著时。应春三月中气,惊蛰、春分、清明。夏三月中气,小满、夏至、大暑;秋三曰中气,处暑、秋分、霜降;冬三月中气,小雪、冬至、大寒。闰无中气,指两辰之间。万物春生夏长,秋收、冬藏。天地之正,四时之极,不易之道。夏数得天,百王所同。其在商汤,用师于夏,除民之灾,顺天革命,改正朔,变服殊号,一文一质,示不相沿,以建丑之曰为正,易民之视。若天时大变,亦一代之事,亦越我周王致伐于商,改正异械,以垂三统,至于敬授民时,巡狩祭享,犹自夏焉。是谓周月,以纪于政。

\hypertarget{header-n279}{%
\subsubsection{时训解}\label{header-n279}}

立春之日,东风解冻。又五日,蛰虫始振。又五日,是对上冰,风不街冻,号令不行。蛰虫不振,阴奸阳。鱼不上冰,甲胄私藏。惊蛰之日,獭祭鱼。又五日,鸿雁来。又五日,草木萌动。獭不祭鱼,国多盗贼;鸿雁不来,远人不服。草木不萌,动,果蔬不熟。雨水之日,桃始华。又五日,仓庚鸣。又五日,鹰化为鸠。桃不始华,是谓阳否。仓庚不鸣,臣不□主。鹰不化鸠,寇戎数起。春分之日,玄鸟至。又五日,雷乃发声。又五日,始电。玄鸟不至,妇人不娠;雷不发声,诸侯失民。不始电,君无威震。谷雨之日,桐始华,又五日,田鼠化为鴽。又五日,虹始见。桐不华,岁有大寒。田鼠不化,鴽,若国贪残。虹不见,妇人苞乱。清明之日,萍始生。又五日,鸣鸠拂其羽。又五日,戴胜降于桑。萍不生,阴气愤盈。鸣鸠不拂其羽,国不治;戴胜不降于桑,政教不中。

立夏之日,蝼蝈鸣。又五日,蚯蚓出。又五日,王瓜生。蝼蝈不鸣,水潦淫漫;蚯蚓不出,嬖夺后;王瓜不生,困于百姓。小满之日,苦菜秀。又五日,靡草死。又五日,小暑至。苦菜不秀,贤人潜伏。靡草不死,国纵盗贼。小暑不至,是谓阴慝。芒种之日,螳螂生。又五日,臭始鸣。又五日,反舌无声。螳螂不生,是谓阴息。臭不始鸣,令奸壅偪。反舌有声,佞人在侧。夏至之日,鹿角解。又五日,蜩始鸣。又五日,半夏生。鹿角不解,兵革不息。蜩不鸣,贵臣放逸。半夏不生,民多厉疾。小暑之日,温风至。又五日,蟋蟀居辟。又五日,鹰乃学习。温风不至,国无宽教。蟋蟀不居辟,急迫之暴。鹰不学习,不备戎盗。大暑之日,腐草化为萤。又五日,土润溽暑。又五日,大雨时行。腐草不化为萤,谷实鲜落。土润不溽暑,物不应罚。大雨不时行,国无恩泽。

立秋之日,凉风至。又五日,白露降。又五日,寒蝉鸣。凉风不至,国无严政。白露不降,民多邪病。寒蝉不鸣,人皆力争。处暑之日,鹰乃祭鸟。又刷日,天地始肃。又五日,禾乃登。鹰不祭鸟,师旅无功。天地不肃,君臣乃□。农不登谷,暖气为灾。白露之日,鸿雁来。又五日,玄鸟归。又五日,群鸟养羞。鸿雁不来,远人背畔。玄鸟不归,室家离散。群鸟不养羞,下臣骄慢。秋分之日,雷始收声。又五日,蛰虫培户。又五日,水始涸。雷不始收声,诸侯淫佚。蛰虫不培户,民靡有赖。水不始涸,甲虫为害。寒露之日,鸿雁来宾。又五日,爵入大水,化为蛤。又五日,菊有黄华。鸿雁不来,小民不服。,爵不入大水,失时之极。较重无黄华,土不稼穑。霜降之日,豺乃祭兽。又五日,草木黄落。又五日,蛰虫咸俯。豺不祭兽,爪牙不良。草木不黄落,是为愆阳。蛰虫不咸附,民多流亡。

立冬之日,水始冰。又五日,地始冻。又五日,雉入大水为蜃。水不冰,是谓阴负地。不始冻,咎征之咎。雉不入大水,国多淫妇。小雪之日,虹藏不见。又五日,天气上腾。地气下降。又五日,闭塞而成冬。虹不藏,妇不专一。天气不上腾,地气不下降,君臣相嫉。不闭塞而成冬,母后淫佚。大雪之日,鴠鸟不鸣。又五日,虎始交。又五日,荔挺生。鴠鸟犹鸣,国有讹言。虎不始交,将帅不和。理睬挺不生,卿士专权。冬至之日,蚯蚓结。又五日,麋角解。又五日,水泉动。蚯蚓不结,君政不行。麋角不解,兵甲不藏。水泉不动,阴不承阳。小寒之日,雁北向。又五日,鹊始巢。又五日,雉始鸲。雁不北向,民不怀主。鹊不始巢,国不宁。雉不始雊,国大水。大寒之日,鸡始乳。又五日,鸷鸟厉疾。又五日,泽腹坚。鸡不始乳,淫女乱男。鸷鸟不厉,国不除兵。水泽不腹,坚言乃不从。

\hypertarget{header-n283}{%
\subsubsection{月令解}\label{header-n283}}

孟春之月,日在营室,昏参中,旦尾中。其日甲乙,其帝太昊,其神句芒,其虫鳞,其音角,律中太簇,其数八,其味酸,其臭膻,其祀户,祭先脾。东风解冻,蛰虫始振,鱼上冰,獭祭鱼,候雁北。天子居青阳左个,乘鸾辂,驾苍龙,载青畤。义青衣,服青玉,食麦与羊,其器疏以达。是月也,以立春。先立春三日,太史谒之天子,曰:某日立春,盛德在木。天子乃齐立春之日,天子亲率三公九卿诸侯大夫以迎春于东郊,还乃赏公卿诸侯大夫于朝。命相布德、和令、行庆、施惠,下及兆民,庆赐遂行,无有不当。乃命太史守典奉法,司天日月星辰之行,宿离不忒,无失经纪以初为常。是月也,天子乃以元日祈谷于上帝,乃择元辰,天子亲载耒耜,措之参于保介之御间。率三公九卿诸侯大夫躬耕,帝藉田。天子三推,三公五推,卿诸侯大夫九推。反执爵于太寝,三公九卿诸侯大夫皆御,命曰劳酒。是月也,天气下降,地气上腾,天地和同,草木繁动。王布农事,命田舍东郊,皆修封疆,审端经术,善相丘陵阪险原隰,土地所宜,五谷所殖,以教道民,必躬亲之。田事既饬,先定准直,农乃不惑。是月也,命乐正入学习舞,乃修祭典。命祀山林川泽,牺牲无用牝,禁止伐木,无覆巢,无杀孩虫胎夭飞鸟,无麛无卵,无聚大众,无置城郭。掩骼霾髊。是月也,不可以称兵,称兵必有天殃。兵戎不起,不可以从我始,无变天之道,无绝地之理,无乱人之纪。孟春行夏令,则风雨不时,草木旱槁,国乃有恐。行秋令则民大疫,疾风暴雨数至,藜莠蓬蒿并兴。行冬令则水潦为败,霜雪大挚,首种不入。

仲春之月,日在奎,昏弧中,旦建星中,其人甲乙,其帝太昊,其神句芒,其虫鳞,其音角,律中夹锺,其数八,其味酸,其臭膻,其祀户,祭先脾。始雨水,桃李华,苍庚鸣,鹰化为鸠。天子居青阳左个太庙,乘鸾辂,驾苍龙载青畤,衣青衣,服青玉,食麦与羊,其器疏以达。是月也,安萌牙,养幼少,存诸孤。择元日,命民社。命有司省囹圄,去桎梏,无肆掠,止狱讼。是月也,玄鸟至。至之日,以太牢祀于高禖。天子亲往,后妃率九嫔御,乃礼天子所御,带以弓韣,授以弓矢于高禖之前。是月也,日夜分,雷乃发声,始电。蛰虫咸动,启户始出。没雷三日,奋铎一令于兆民,曰:雷且发声,有不戒其容止者,生子不被。必有凶灾,日夜分则同度量钧衡石角斗桶,正权概。是月也,耕者少舍,乃修阖扇,寝庙必备。无作大事,以妨农功。是月也,无竭川泽,无漉陂池,无焚山林。天子乃献羔,开冰,先荐寝庙。上丁,命乐正入舞舍采。天子乃率三公九卿诸侯大夫亲往视之。中丁,又命乐正入学习乐。是月也,祀不用牺牲,用圭璧,更皮币。仲春行秋令,则其国大水,寒气匆至,寇戎了征。行冬令,则阳气不胜,麦乃不熟,民多相掠。行夏令则国乃大旱,暖气早来,虫螟为害。

季春之曰,日在胃,昏七星中,旦牵牛中,其日甲乙,其帝太昊,其神句芒,其虫鳞,其音角,律中姑洗,其数八,其味酸,其臭膻,其祀户,祭先脾。桐始华,田鼠化为鴽,虹始见,萍始生。天子居青阳右个,乘鸾辂,驾苍龙,载青畤,衣青衣,服青玉,食麦与羊。其器疏以达。是月也,天子乃荐鞠衣于先帝,命舟牧覆舟,五覆五反,乃告舟备,具于天子焉。天子焉始乘舟,荐鲔于寝庙。乃为麦祈实。是月也,生气方盛,阳气发泄,牙者毕出,萌者尽达,不可以内,天子布德行惠,命有司发仓窌,赐贫穷,振乏绝,开府库,出币帛,周天下,勉诸侯,聘名士,礼贤者。是月也,命司空曰:时雨将降,下水上腾,循行国邑,周视原野,修利堤防,道达沟渎,开通道路,无有障塞。田猎\textless{}罒毕\textgreater{}弋,罝罘罗网、兽之药,无出九门。是月也,命野虞无伐桑枳。鸣鸠拂其羽,戴任降于桑。具挟曲蒙筐。后妃齐戒,亲东乡,躬桑。禁妇女无观省,妇使劝蚕事。蚕事既登,分茧称丝,效其功。以其郊庙之服,无有敢堕。是月也,命工师令百工审五库之量,金铁皮革筋角齿羽箭干脂胶丹漆,无或不良。百工咸理,监工日号,无悖于时,无或作为淫巧以荡上心。是月之末,择吉日,大合乐,天子乃率三公九卿诸侯大夫,亲往视之。是月也,乃合缧牛腾马,游牝于牧。牺牲驹犊,举书其数,命国傩九门,磔禳于毕春气。行之是令而甘雨至三旬。季春行冬令,则寒气时发,草木皆肃,国有大恐。行夏令,则民多疾疫,时雨不降,山陵不收。行秋令,则天多沈阴,淫雨早降,兵革并起。

孟夏之月,日在毕,昏翼中,旦婺女中。其日丙丁,其帝炎帝,其神祝融,其虫羽,其音徵,律中仲吕,其数七,其性礼,其事视,其味苦,其臭焦,其祀灶,祭先肺。蝼蝈鸣,丘蚓出,王瓜生,苦菜秀。天子居明堂左个,乘硃辂,驾赤□,载赤畤,衣赤衣,服赤玉,余菽与鸡,其器高以粗。是月也,以立夏。先立夏三日,太史谒之天子,曰:某日立夏,盛德在火,天子乃齐。立夏之日,天子亲率三公九卿大夫,以迎夏于南郊。还乃行赏,封侯庆赐,无不欣说。乃命乐师习礼乐,命司马赞杰隽,遂贤良,举长大。行爵出禄,必当其位。是月也,继长增高,无有坏堕,无起土功,无发大众,无伐大树。是月也,天子始絺,命野虞出行田原,劳农劝民,无或失时。命司徒循行县鄙,命农勉作,无伏于都。是月也,驱兽无害五谷,无大田猎,农乃升麦。天子乃以彘尝麦,先荐寝庙。是月也,聚蓄百药,靡草死麦,秋至断薄刑,决次罪,出轻系。蚕事既毕,后妃献茧,乃收茧税,以桑为均,贵贱少长如一,以给郊庙之祭服。是月也,天子饮酎,用礼乐,行之是令,而甘雨至三旬。孟夏行秋令,则苦雨数来,五谷不滋。四鄙入保;行冬令,则草木早枯,后乃大水,败其城郭。行春令则虫蝗为败,暴风来格,秀草不实。

仲夏之月,日在东井,昏亢中,旦危中,其日丙丁,其帝炎帝,其神祝融,其虫羽,其音徵,率中蕤宾,其数七,其味苦,其臭焦,其祀灶,祭先肺。小暑至,螳螂生,臭始鸣,反舌无声。天子居明堂太庙,乘硃辂,驾赤□,载赤畤,衣赤衣,服赤玉,食菽与鸡,其器高以粗,养壮狡。是月也,命乐师修鞀鞞鼓,均琴瑟管箫,执干戚戈羽,调竽笙埙篪,饬锺磬柷敔,命有司为民气祀山川百原,大雩。帝用盛乐,乃命百县雩祭祀,百辟卿士有益于民者,以祈谷实。是月也,农乃登黍,天子以雏尚黍,羞以含桃,先荐寝庙,令民无刈蓝以染,无烧炭,无暴布,门闾无闭,关市无索,挺重囚,益其食,游牝别其群,则絷腾驹,班马正。是月也,日长至,阴阳争,死生分,君子齐戒,处必掩身,欲静无躁,止声色,无或进薄,滋味无致,和退嗜欲,定心志,百官静事无刑,以定晏阴之所成。鹿角解,蝉始鸣,半夏生,木槿荣。是月也,无用火。南方可以居高明,可以远眺望,可以登山陵,可以处台榭。仲夏行冬令,则雹霰伤谷,道路不通,暴兵来至。行春令,则五谷晚熟,百滕时起,其国乃饥。行秋令,则草木零落,果实早成,民殃于疫。

季夏之月,日在柳,昏心中,但奎中,其日丙丁,其帝炎帝,其神祝融,其虫羽,其音徵,律中林锺,其数七,其味苦,其臭焦,其祀灶,祭先肺。凉风始至,蟋蟀居,宇鹰乃学习,腐草化为萤□。天子居明堂右个,乘硃辂,驾赤□,载赤畤,衣硃衣,服赤玉,食菽与鸡,其器高以牴。是月也,命渔师伐蛟,取□升龟取鼋。乃命虞人入材苇。是月也,命四监大夫合百县之秩刍,以养牺牲。令民无不咸出其力,以供皇天上帝名山大川四方之神,以祀宗庙社稷之灵,为民所福。是月也,命妇官染采,黼黻文章必以法,故无或差忒。黑黄苍赤,莫不质良,勿敢位诈,以给郊庙祭祀之服,以为旗章,以别贵贱等级之度。是月也,树木方盛,乃命虞人入山,行木,无或斩伐,不可以兴土功,不可以合诸侯,不可以起兵动众,无举大事,以摇荡于气。无发令而干时,以妨神农之事。水潦盛昌,命神农将巡功,举大事则有天殃。是月也,土润溽暑,大雨时,行烧剃,行水利,以杀草,如以热汤,可以粪田畴,又美图疆,行之是令。是月,甘雨三至,三旬二日。季夏行春令,则谷实解落,国多风咳,民乃迁徙。行秋令,则丘隰水潦,禾稼不熟,乃多女灾。行冬令,则寒气不时,鹰隼早鸷,四鄙入保。

中央土黄,其日戊己,其帝黄帝,其神后土,其虫倮,其音宫,律中黄锺之宫,其数五,其味甘,其臭香,其祀中霤祭先心,天子居太庙太室,乘大辂,驾黄□,载黄畤,衣黄衣,服黄玉,食稷与牛,其气圜以掩。

孟秋之月,日在翼,昏斗中,旦毕中,其日庚辛,其帝少昊,其神蓐收,其虫毛,其音商,律中夷则,其数九,其味辛,其臭腥,其祀门,祭先肝,。凉风至,白露降,寒蝉鸣,鹰乃祭鸟,始用行戮。天子居总章左个,乘戎路,驾白骆,载白畤,衣白衣,服白玉,食麻与犬,其器廉以深。是月也,以立秋。先立秋三日,太史谒之天子,曰:某日立秋,盛德在金,天子乃齐。立秋之日,天子亲率三公九卿诸侯大夫以迎秋于西郊,还,乃赏军,率武人于朝,天子乃命将帅,选士厉兵,简练杰隽,专任有功,以征不义,诘诛暴慢,以明好恶,巡彼远方。是月也,命有司修法制,缮囹圄,具桎梏,禁止奸,慎罪邪,务搏执,命理瞻伤察创,视折审断。决狱讼必正平,戮有罪,严断刑。天地始肃,不可以赢。是月也,农乃升谷,天子尚新,先荐寝庙。命百官始收敛完堤防,谨壅塞,以并未水潦,修宫室,坿墙垣,补城郭。是月也,无以封诸侯、立大官,无割土地、行重币、出大使。行之是令。而凉风至三旬。孟秋秋行冬令,则阴气大胜,介虫败谷,戎兵乃来。行春令,则其国乃旱,阳气复还,五谷不实。行夏令,则国多火灾,寒热不节,民多疟疾。

仲秋之月,日在角,昏牵牛中,旦觜□崔瞆中。其日庚辛,其帝少昊,其神蓐收,其虫买,其音商,律中南吕,其数九,其味辛,其臭腥,其祀门,祭先肝。盲风至,候雁来,玄鸟归,群鸟养羞。天子居总章太庙,乘戎路,驾白辂,载白畤衣白衣,服白玉,食麻与犬,其器廉以深。是月也,养衰老,授几杖,行麋粥饮食。乃命司服具饬衣裳,文绣有恆,制有小大,度有短长,衣服有量,必循其故,冠带有常。乃命有司,申严百刑,斩杀必当,无或枉桡;不当,反受其殃。是月也,乃命宰巡行牺牲,视全具,案刍豢,朝肥瘠,察物色,必比类,量小大,视长短,皆中度,五者备当,上帝其享。天子乃傩,御佐疾,以通秋气。以犬尝麻,先祭寝庙。是月也,可以筑城郭,建都邑,穿窦窌,修□仓。乃命有司趣民收敛,务蓄菜,多积聚,乃劝种麦,若或失时,行罪无疑。是月也,日夜分雷,乃始收声,蛰虫俯户,杀气浸盛,阳气日衰,水始涸,日夜分,则,则一度量,平权衡,正钧石,齐斗桶。是月也,易关市,来商旅,入货贿,以便民事。四方来杂,远乡皆至,则财务不匮。上无乏用,百事乃遂。凡举事,无逆天数,必顺其时,乃因其类,行之是令。白露降三旬。仲秋行春令,则秋雨不降,草木生溶剂,国乃有恐。行夏令,则其国乃旱,蛰虫不藏,五谷复生。行冬令则风灾数起,收雷先行,草木早死。

季秋之月,日在房,昏虚中,旦柳中,其日庚辛,其神少昊,其神蓐收,其虫毛,其音商,律中无射,其数九,其味辛,其臭腥,其祀门,祭先肝。候雁来宾,爵入大水为蛤,菊有黄华,豺则祭兽戮禽。天子居总章右个,乘戎路,驾白骆,载白畤,衣白衣,服白玉,食麻与犬,其器廉以深。是月也,申严号令,命百官贵贱无不务入,以会天地之藏,无有宣出。乃命冢宰,农事备收,举五种之要,藏帝籍之收于神仓,祗敬比饬。是月也,霜始降,则百工休。乃命有司曰,寒气总至,民力不堪,其皆入室。上丁,命乐正入学习吹。是月也,大飨于帝,尝牺牲,告备于天子,合诸侯,制百县。为来岁受朔日,与诸侯所税于民轻重之法,数以远近土地所宜为度。以给郊庙之事,无有所私。是月也,天子乃教于田猎,以习五戎獀马,命仆及七驺,咸驾载旍□,舆受车,以纪整设于屏外,司徒缙扑北向以誓之,命主祠祭禽于四方。是月也,草木黄落,乃伐薪为炭,蛰虫咸俯在穴,皆??其户。乃趣狱刑,无留有罪,收禄秩之不当者,共养之不宜者。是月也,天子乃以犬尝稻,先荐寝庙。季秋行夏令,则其国大水,冬藏殃败,民多??鼻九窒。行冬令则国多盗贼,边境不宁,土地分裂。行春令,则暖风来至,民气解堕,师旅必兴。

孟冬之月,日在尾,昏危中,旦七星中,其日壬癸,其帝颛顼,其神玄冥,其虫介,其音羽,律中应锺,其数六,其味咸,其臭朽,其祀行,祭先肾。水始冰,地始冻,雉入大水为蜃,虹藏不见。天子居玄堂左个,乘玄辂,驾铁骊,载玄畤,衣黑衣,服玄玉,食黍与彘。其器宏以□。是月也,以立冬。先立冬三日,太史谒之天子,曰:某日立冬,盛德在水。天子乃齐。立冬之日,天子亲率三公九卿大夫以迎冬于北郊,还,乃赏死事,恤孤寡。是月也,命太卜祷祠,龟策占兆,审卦吉凶。于是察阿上乱法者,则罪之,无有掩蔽。是月也,天子始裘,命有司曰:天子天气上腾,地气下降,天地不通,闭塞而成冬。命百官谨盖藏,命司徒循行积聚,无有不敛,坿城郭,戒门闾,修楗闭,慎关钥,固封玺,备边境,备要塞,谨关梁,塞蹊径,饬丧纪,,办衣裳,审棺椁之厚薄,营丘垄之小大高卑薄厚之度,贵贱之等级。是月也,工师效功,陈祭器,案度程,无或作为淫巧,以荡上心,必功致为上,物勒工名,以考其诚,工有不当,必行其罪,以穷其情。是月也,大饮蒸,天子乃祈来年于天宗,大割祠于公社,及门闾,飨先祖五祀,劳农夫以休息之。天子乃命将率讲武,肄射御角力。是月也,乃命水虞渔师收水泉池泽之赋,无或敢侵削众庶兆民,以为天子取怨于下,其有若此者,行罪无赦。孟冬行春令,则冻闭不密,地气发泄,民多流亡。行夏令,则国多暴风,方冬不寒,蛰虫复出,行秋令则雪霜不时,小兵时起,土地侵削。

仲冬之月,日在斗,昏东壁中,旦轸中,其日壬癸,其帝颛顼,其神玄冥,其虫介,其音羽,律中黄锺,其数六,其味咸,其臭朽,其祀行,祭先肾。,冰益壮,地始坼,鹖鴠不鸣,虎始交。天子居玄堂太庙,乘玄辂,驾铁骊,载玄畤,衣黑衣,服玄玉,食黍与彘,其器宏以□。饬死事,命有司曰:土事无作,无发盖藏,无起大众,以固而闭。发盖藏,起大众,地气且泄,是谓发天地之房,诸蛰则死,民多疾疫,又随以丧。命之曰没暢月。是月也,命阉尹深宫令,审门闾可,谨房室,必重闭。省妇事,无得淫。虽有贵戚近习,无有不察。乃命大酋,秫稻必齐,麹蘖必时,湛饎必洁,水泉必香,陶气必良,火齐必得,兼用六物,大酋监之,无有差忒。天子乃命有司祈祀四海大川名原渊泽井泉。是月也,农有不收藏积聚者,牛马畜首有放佚者,取之不诘。山林薮泽有能取疏食田猎禽兽者,野虞教导之。其有相侵,夺者罪之不赦。是月也,日短至,阴阳争。诸生荡。君子齐戒,处必掩,身欲宁,去声色,禁嗜欲,安形性,事欲静,以待阴阳之所定,芸始生,荔挺出,蚯蚓结,麋角解,水泉动。日短至,则伐林木,取竹箭。是月也,可以罢官之无事者,去器之无用者,途阙庭门闾,筑囹圄,此所以助天地之闭藏也。仲冬行夏令,则其国乃旱,氛雾冥冥,雷乃发声。行秋令,则天时雨汁,瓜瓠不成,国有大兵。行春令,则虫螟为败,水泉减竭,民多疾疠。

季冬之月,日在婺内,昏娄中,旦氐中,其日壬癸,日其间颛顼,其神玄冥,其虫介,其音羽,律中大吕,其数六,其味咸,其臭朽,其祀行,祭先胜。雁北乡,鹊始巢,雊鸡乳。天子居玄堂右个,成玄辂,驾铁骊,载玄畤,衣黑衣,服玄玉,食黍与彘,其器宏,以掩命。有司大傩,旁磔出土牛,以宋寒气,征鸟厉疾,乃毕行山川之祀,及帝之大臣天地之神祗。是月也,命渔师始渔,天子亲往,乃尝鱼,先荐寝庙。冰方盛,水泽腹坚,命取冰,冰已入,令告民出五种,命司农计耦耕事,修耒耜,具田器。命乐师大合吹而罢。乃命四检收秩薪柴,以供寝庙及百祀之薪燎。是月也,日穷于次,月穷于纪,星回于天,数将几终,岁将更始,专于农民,无有所使。天子乃与公顷大夫饬国典,困时令以待来岁之宜。乃命大事次诸侯之列赋之牺牲,以供皇天上帝社稷之享,乃元同姓之邦,供寝庙之刍豢。命宰历卿大夫至于庶民土田之数,而赋之牺牲,以供山林名川之祀。凡在天下九州之民者,无不咸献其力,以供皇天上帝社稷寝庙山林名川之祀。行之是令。此谓一终,三旬二日。季冬行秋令,则白露蚤降,介虫为妖,四邻入保。行春令,则胎夭多伤,国多固疾,命之曰逆。行夏令,则水潦败国,时雪不降,冰冻消释。

\hypertarget{header-n287}{%
\subsubsection{谥法解}\label{header-n287}}

维周公旦、太公望,开嗣王业,建功于牧之野,终将葬,乃制谥。遂叙谥法。谥者,行之迹也。号者,功之表也。车服者,位之章也。是以大行受大名,细行受细名,行出于己,名生于人。

民无能名曰神。称善赋简曰圣,敬宾厚礼曰圣。德象天地曰帝。静民则法曰皇。仁义所在曰王。赏庆刑威曰君,从之成群曰君。立制及众曰公。执应八方曰侯。壹德不解曰简。平易不疵曰简。

经纬天地曰文,道德博闻曰文,学勤好问曰文,慈惠爱民曰文,愍民惠礼曰文。锡民爵位曰文。刚强理直曰武,威强澼德曰武,克定祸乱曰武,刑民克服曰武,夸志多穷曰武。敬事供上曰恭,尊贤贵义曰恭,尊贤敬让曰恭,既过能改曰恭,执事坚固曰恭,爱民长弟曰恭,执礼御宾曰恭,芘亲之阙曰恭,尊贤让善曰恭,渊源流通曰恭。照临四方曰明,谮诉不行曰明。

威仪悉备曰钦。大虑静民曰定,安民大虑曰定,安民法古曰定,纯行不二曰定。谏争不威曰德。辟地有德曰襄,甲胄有劳曰襄。有伐而还曰厘,质渊受谏曰厘。博闻多能曰宪。聪明澼哲曰献。温柔圣善曰懿。五宗安之曰孝,慈惠爱亲曰孝,协时肇享曰孝,秉德不回曰孝。大虑行节曰考。执心克庄曰齐,资辅供就曰齐。丰年好乐曰康,安乐抚民曰康,令民安乐曰康。安民立政曰成。布德执以曰穆,中情见貌曰穆。敏以敬顺曰顷。

昭德有劳曰昭,容仪恭美听昭,圣闻周达曰昭。保民耆艾曰胡,弥年寿考曰胡。强毅果敢曰刚,追补前过曰刚。柔德考众曰静,恭己鲜言曰静,宽乐令终曰静。治而无眚曰平,执事有制曰平,布纲治纪曰平。由义而济曰景,布义行刚曰景,耆意大虑曰景。清白守节曰贞,大虑克就曰贞,不隐无屈曰贞。猛以刚果曰威,猛以强果曰威,强毅信正曰威。辟屠服远曰桓,克敬勤民曰桓,辟土兼国曰桓。道德纯一曰思,大省兆民曰思,外内思索曰思。追悔前过曰思。柔质慈民曰惠,爱民好与曰惠。柔质受谏曰慧。能思辩众曰元,行义说民曰元,始建国都曰元,主义行德曰元。

兵甲亟作曰庄,澼圉克服曰庄,胜敌志强曰庄,死于原野曰庄,屡征杀伐曰庄,武而不遂曰庄。克杀秉政曰夷,安心好静曰夷。执义扬善曰怀,慈仁短折曰怀。夙夜警戒曰敬,夙夜恭事曰敬,象方益平曰敬,善合法典曰敬。述义不克曰丁,迷而不悌曰丁。有功安民曰烈,秉德遵业曰烈。刚克为伐曰翼,思虑深远曰翼。执心决断曰肃。爱民好治曰戴,典礼不忄寒曰戴。

死而志成曰灵,乱而不损曰灵,极知鬼神曰灵,不勤成名曰灵,死见神能曰灵,好祭鬼神曰灵。短折不成曰殇,未家短折曰殇。不显尸国曰隐,隐拂不成曰隐。年中早夭曰悼,肆行劳祀曰悼,恐惧从处曰悼。不思忘爱曰刺,愎狠遂过曰刺。外内从乱曰荒,好乐怠政曰荒。在国逢难曰愍,使民折伤曰愍,在国连忧曰愍,祸乱方作曰愍。蚤孤短折曰哀,恭仁短折曰哀。蚤孤铺位曰幽,壅遏不通曰幽,动祭乱常曰幽。克威捷行曰魏,克威惠礼曰魏。去礼远众曰炀,好内远礼曰炀,好内怠政曰炀。

甄心动惧曰顷。威德刚武曰圉。圣善周闻曰宣。治民克尽曰使。行见中外曰悫。胜敌壮志曰勇。昭功宁民曰商。状古述今曰誉。心能制义曰度,好和不争曰安。外内贞复曰白。不生其国曰声。杀戮无辜曰厉。官人应实曰知。凶年无谷曰糠。名实不爽曰质。不悔前过曰戾。温良好乐曰良。怙威肆行曰丑。德正应和曰莫。勤施无私曰类。好变动民曰躁。慈和便服曰顺。满志多穷曰感。危身奉上曰忠。果虑果远曰趕。息政外交曰携。疏远继位曰绍。彰义掩过曰坚。肇敏行成曰直。内外宾服曰正。华言无实曰夸。教诲不倦曰长。爱民在刑曰克。逆天虐民曰抗。好廉自克曰节。择善而从曰比。好更改旧曰易。名与实爽曰缪。思厚不爽曰愿。贞心大度曰匡。

隐哀之方,景武之方也,施为文也,除为武也,辟地为襄,服远为桓,刚克为发,柔克为懿,履正为庄,有过为僖,施而不成为宣,惠无内德为平。失志无转,则以其明,余皆象也。和,会也;勤,劳也。遵,循也;爽,伤也;肇,始也;憹,治也;康,安也;怙,恃也。享,祀也;胡,大也;服,败也。秉,顺也;就,会也;忄寒,过也;锡,与也;典,常也;肆,放也;穅,虚也;澼圣也;惠,爱也;绥,安也;坚,长也;耆,强也;考,成也;周,至也;怀,思也;式,法也;布,施也;敏,疾也;捷,克也;载,事也;弥,久也。

\hypertarget{header-n291}{%
\subsubsection{明堂解}\label{header-n291}}

大维商纣暴虐,脯鬼侯以享诸侯,天下患之,四海兆民欣戴文武,是以周公相武王以伐纣,夷定天下,既克纣六年,而武王崩,成王嗣,幼弱,未能践天子之位。

周公摄政君天下,弭乱六年,而天下大治,乃会方国诸侯于宗周,大朝诸侯明堂之位。天子之位,负斧依,南面立。率公卿士,侍于左右。三公之位,中阶之前。北面东上,诸侯之位。西阶之西,东面北上,诸子之位。门内之东,北面东上,诸男之位。门内之西,北面东上,九夷之国。东门之外,西面北上,八蛮之国。南门之外,北面东上,六戎之国。西门之外,难免南上,五狄之国。北门之外,难免东上,四塞九采之国。世告至者,应门之外,北而东上,宗周明堂之位也。

明堂,明诸侯之尊卑也,故周公建焉,而朝诸侯于明堂之位。制礼作乐,颁度量,而天下大服,万国各致其方贿。七年,致政于成王。

明堂方百一十二尺,高四尺,阶广六尺三寸。室居中方百尺,室中方六十尺,户高八尺,广四尺。东应门,南库门,西皋门,北雉门。东方曰青阳,南方曰明堂,西方曰总章,北方曰玄堂,中央曰太庙。左为左介,右为右介。

\hypertarget{header-n295}{%
\subsubsection{尝麦解}\label{header-n295}}

维四年孟夏,王初祈祷于宗庙,乃尝麦于太祖。是月,王命大正正刑书。爽明,仆告既驾,少祝导王,祝亚迎王降阶,即假于太宗、少宗、少秘于社,各牡羊一、牡豕三。史导王于北阶,王陟阶,在东序。

乃命太史尚大正,即居于户,西南向。九州□伯咸进在中,西向。宰乃承王中升自客阶,作策执策,从中宰坐尊中于大正之前,太祝以王命作策,策告太宗。王命□□秘,作策许诺,乃北向繇书于两楹之间。

王若曰:``宗揜大正,昔天之初,□作二后,乃设建典命,赤帝分正二卿,命蚩尤于宇,少昊以临四方,司□□上天末成之庆。蚩尤乃逐帝,争于涿鹿之河,九隅无遗。赤帝大慑,乃说于黄帝,执蚩尤,杀之于中冀,以甲兵释怒,用大正顺天思序,纪于大帝。用名之曰:绝辔之野。乃命少昊清司马、鸟师,以正五帝之官,故名曰质。天用大成,至于今不乱。

``其在殷当作夏之五子,往伯禹之名,假国无正,用胥兴作乱,遂凶厥国,皇天哀禹,赐以彭寿,思正夏略。

``今予小子闻有古遗训,亦述朕文考之言,不易,予用,皇威不忘,祗天之明典,令□我大治,用我九宗正州伯,教告于我,相在大国,有殷之□,辟自其作□于古,是威厥邑,无类于冀州。嘉我小国,小国其命余克长国王。

``呜呼,敬之哉!如木既颠厥巢,其犹有枝叶作休,尔弗敬恤,尔执以屏助予一人,集天之显,亦尔子孙其能常忧恤乃事,勿畏多宠,无爱乃嚚,亦无或刑于鳏寡非罪。惠乃其常无别于民。''

众臣咸兴,受大正书。太史策刑书九篇,以升授大正,乃左还自两柱之间。□箴大正曰:``钦之哉,诸正。敬功尔颂,审三节,无思民,因顺尔临狱,无颇正刑有掇,夫循乃德,式监不远,以有此人保宁尔国,克戒尔服,世世是其不殆,维公咸若。''太史乃降,大正坐,举书,乃中降,再拜稽首。王命太史、正升拜于上,王则退。

是曰,士师乃命太宗序于天时,祠大暑;乃命少宗祠风雨百享。士师用受其胾,以为之资。邑乃命百姓遂享于富,无思民疾,供百享。归祭,闾帅里君以为之资。野宰乃命家邑县都祠于太祠,乃风雨也。宰用受其职胾,以为之资。采君乃命天御,丰穑享祠为施,大夫以为资。箴太史乃藏之于盟府,以为岁典。

\hypertarget{header-n299}{%
\subsubsection{本典解}\label{header-n299}}

维四月既生魄,王在东宫,告周公曰:``呜呼,朕闻武考不知,乃闻不得,乃学俾资不肖,永无惑矣。今朕不知明德所则,政教所行,字民之道,礼乐所生,非不念而知,故闻伯父。''

周公再拜稽首曰:``臣闻之文考,能求士□者,智也;与民利者,仁也;能收民狱者,义也;能督民过者,德也;为民犯难者,武也。智能亲智,仁能亲仁,义能亲义,德能亲德,武能亲武。五者昌于国,曰明,明能见物,高能致物,物备咸至,曰帝,帝乡在地,曰本;本生万物,曰世;世可则□,曰至;至德照天,百姓□径备有好丑,民无不戒,显父登德,德降则信;信则民宁,为畏为极。民无淫慝,生民知常利之道,则国强序明,好丑□必固,其务均分以利之则民安。□用以资之,则民乐;明德以师之,则民让;生之乐之,则母之,礼也。政之教之、遂以成之,则父之,礼也。父母之礼,以加于民,其慈□□古之圣王乐体其政。士有九等,皆得其宜,曰材。人有八政,皆得其则,曰礼。服士乐其生,而务其宜,是故奏鼓以章乐,奏舞以观礼,奏歌以观和、礼乐既和,其上乃不危。''

王拜曰:``允哉,幼愚,敬守以为本典。''

\hypertarget{header-n307}{%
\subsection{卷七}\label{header-n307}}

\hypertarget{header-n311}{%
\subsubsection{官人解}\label{header-n311}}

王曰:``呜呼,大师,朕维民务官,论用有征,观诚考言,视声观色,观隐揆德,可得闻乎?''

周公曰:``亦有六征。呜呼,乃齐以揆之。一曰富贵者观其有礼施,贫贱者观其有德守,嬖宠者观其不骄奢,隐约观其不慑惧,其少者观其恭敬好学而能悌,其壮者观其连接务行而胜私,其老者观其思慎,强其所不足而不逾。父子之间观其孝慈,兄弟之间观其和友,君臣之间观其忠惠,乡党之间观其信诚。省其居处观其义方,省其丧哀观其贞良,省其出入观其交并以,省其交友观其任廉。设之以谋以观其智,示之以难以观其勇,烦之以事以观其治,临之以利以观其不贪。滥之以乐以观其不荒,喜之以观其轻,怒之以观其重,醉之以观其恭,从之色以观其常,远之以观其不二,昵之以观其不狎。复征其言以观其精,曲省其行以观其备。此之谓观诚。``二曰方与之言以观其志,志殷以渊其气,宽以柔,其色俭而不谄,其处仙,其言后人,见其所不足。曰日益者也。好临人以色,高人以气,贤人以言,防其所不足,发其所能,曰日损者也。其貌直而不止其言,正而不私,不饰其美,不隐其恶,不防其过,曰有质者也。其貌曲媒,其言工巧,饰其见物,务其小证,以故自说,曰无质者也。喜怒以物而色不变,烦乱一事而志不营,深道以利而心不移,临慑以威而气惵惧,曰鄙心而假气者也。设之以物而数决,敬之以卒而度应,不文而辩,曰有虑者也。难决以物,难说以守,一而不可变,困而不知止,曰愚依人也。营之以物而不误,犯之以卒而不惧,置义而不可迁,临之货色而不过,曰果敢者也。移易以言,志不能固,已诺无决,曰弱志者也。顺予之弗为喜,非夺之弗为怒,沈静而寡言,多稽而险貌,曰质静者也。屏言而弗顾,自顺而弗让,非是而强之,曰妒诬者也。微而能发,察而能深,宽顺而恭俭,温柔而能断,果敢而能屈,曰志治者也。华废而诬,巧言令色,皆以无为有者也。此之谓考言。

``三曰,诚在其中,必见诸外,以其声,处其实,气初生物,物生有声,声有刚柔,清浊好恶,咸发于声。新气华诞者,其声流散,心气顺信者,其声顺节。心气鄙戾者,其声醒丑,心气宽柔者,其声温和。信气中易,义气时舒,和气简备,勇气壮力。听其声,处其气,考其所为,观其所由,以其前观其后,以其隐观其显,以其小占其大,此之谓视声。

``四曰:民有五气,喜怒欲惧忧。喜气内蓄,虽欲隐之,阳喜必见。怒气内蓄,虽欲隐之,阳怒必见。欲气、惧气,忧悲之气,皆隐之,阳气必见。五气诚于中,发形于外,民情不可隐也。喜色犹然,以出怒色,荐然以侮,欲色妪然。以愉惧色,薄然以下,忧悲之色,瞿然以静。诚智必有难局之色,诚仁必有可尊之色,诚勇必有难慎之色,诚忠必有可新之色,诚洁必有难污之色,诚静必有可信之色。质浩然固以安,伪蔓然乱以烦,虽欲改之中色,弗听。此之谓观色。

``五曰:民生则有阴有阳,人多隐其情、饰其伪,以攻其名。有隐于仁贤者,有隐于智理者,有隐于文艺者,有隐于廉勇者,有隐于忠孝者,有隐于交友者。如此不可不察也。小施而好德,小让而争大,言愿以为质,伪爱以为忠,尊其得以改其名,如此,隐于仁贤者也。前总唱功,虑诚弗及,佯为不言,诚不足,色示有余。自顺而不让,措辞而不遂,此隐于智理者也。动人以言,竭而弗终,问则不对,佯为不穷,□貌而有余,假道而自顺,因之□初穷则托深,如此者,隐于文艺者也。□言以廉,矫厉以为勇,内恐外夸,亟称其说,以诈临人,如此,隐于廉勇者也。自事其亲而好以告人,饰其见物,不诚于内,发名以事亲,自以名私其身,如此,隐于忠孝者也。比周以相誉,智贤可征而左右,不同而交,交必重己,心说而身弗近身,进而实不至,惧不尽见于众,而貌克,如此,隐于交友者也。

``六曰:言行不类,终始相悖,外怎能不合,虽有假节,见行,曰非成质者也。言忠行夷,靡及私□,弗求及,情忠而宽,貌庄而安,曰有仁者也。事变而能治,效穷而能达,措身立方而能遂,曰有知者也。少言以行恭俭,以让有知而言弗发,有施而□弗德,曰谦良者也。微忽之言,久而可复,幽间之行,独而弗克,其行亡如存,曰顺信者也。贵富恭俭而能施,严威有礼而不骄,曰有德者也。隐约而不慎,安乐而不奢,勤劳而不变,喜怒而有度,曰有守者也。直方而不毁,廉洁而不戾,强立而无私,曰有经者也。虚以待命,不报刊不至,不问不言,言不过行,行不过道,曰沈静者也。中忠爱以事亲,欢以尽力而不回,敬以尽力而不□,曰忠孝者也。合志而同方,共其忧而任其难,行忠信而不疑□,隐远而不舍,曰交友者也。志色辞气,其人甚偷,进退多巧,就人甚数,辞不至,少其所不足,谋而不已,曰伪诈者也。言行亟变,从容克易,好恶无常,行身不笃,曰无诚者也。少知而不大决,少能而不大成,规小物而不知大伦,曰华诞者也。规谏而不类道,行而不平,曰窃名者也。

``故曰事阻者不夷,时□者不回,面誉者不忠,饰貌者不静,假节者不平,多私者不义,扬言者卦按信,此之谓揆德。''

\hypertarget{header-n315}{%
\subsubsection{王会解}\label{header-n315}}

成周之会,墠上张赤帟阴羽,天子南面立,絻无繁露,朝服,八十物缙珽。唐叔、荀叔、周公在左,太公望在右,皆絻,亦无繁露,朝服,七十物,缙笏,旁天子而立于堂上。

堂下之右,唐公虞公南面立焉。堂下之左,尹公、夏公立焉,皆南面,絻有繁露,朝服,五十物,皆缙笏。为诸侯之有疾病者,阼阶之南,祝淮氏、荣氏,次之慓瓚,次之皆西面,弥宗旁之。为诸侯有疾病者之医药所居。

相者,太史鱼、大行人,皆朝服,有繁露。堂下之东面,郭叔掌为天子菉币焉,絻有繁露。内台西面正北方,应侯、曹叔伯舅、中舅。比服次之,要服次之,荒服次之。西方东面正北方,伯父中子次之。

方千里之内为比服,方二千里之内为要服,方三千里之内为荒服,是皆朝于内者。

堂后东北为赤帟焉,浴盆在其中。其西天子车立马乘,六青阴羽凫旌。中台之外,其右泰士,台右弥士。受贽者八人,东面者四人,西面四人也。陈币当外台,天玄\textless{}曷毛\textgreater{}宗马十二,王元缭璧綦十二,参方玄缭璧、豹虎皮十二,四方玄缭璧琰十二。外台之四隅,张赤帟,为诸侯,欲息者皆息焉,命之曰爻闾。

周公旦主东方,所之青马、黑\textless{}葛毛\textgreater{},谓之母兒,其守营墙者,衣青操弓执矛。西面者,正北方,稷慎大麈。秽人前兒,前兒若弥猴,立行似小兒。良夷在子,在子□身人首,脂其腹,炙之霍,则鸣曰在子。扬州禺禺,鱼名,解隃冠,发人麃麃者,若鹿迅走。俞人虽马,青丘狐九尾,周头煇互,煇互者,羊也。黑齿白鹿白马,白民乘黄,乘黄者似骐,背有两角。东越海蛤,欧人蝉蛇,蝉蛇顺食之美。遇越纳□。姑妹珍,

且瓯文蜃,其人玄贝,海阳大蟹。自深桂。会稽以\textless{}单黾\textgreater{},皆面向。正北方义渠,以兹白,兹白者,若白马,锯牙食虎豹。央林以酋耳,酋耳者,身若虎豹,尾长,参其身,食虎豹。北唐以闾,闾似隃冠。渠叟以\textless{}鼠勺\textgreater{}犬,\textless{}鼠勺\textgreater{}犬者,露犬也,能飞食虎豹。楼烦以星施,星施者,珥旌。卜庐以纨牛,纨牛者,牛之小者也。区阳以鳖封,鳖封者,若彘,前后有首。规规以麟,麟者仁兽也。西申以凤鸟,凤鸟者,戴仁抱义掖信。氐羌以鸾鸟。巴人以比翼鸟。反炀以皇鸟,蜀人以文翰,文翰者,若皋鸡。方人以孔鸟,卜人以丹沙,夷用\textless{}门焦\textgreater{}木。康民以桴苡,苡者,其实如李,食之宜子。周靡费费,其形人身反踵,自笑,笑则上脣翕其目,食人,北方谓之吐喽。都郭生生,欺羽生生,若黄狗人面能言。奇干善芳,善芳者,头若雄鸡,佩之令人不昧。皆东向。

北方台正东,高夷嗛羊,嗛羊者,羊而四角。独鹿邛邛,距虚善走也。孤竹距虚,不令支玄獏,不屠何青熊。东胡黄罴,山戎戎菽。其西般吾,白虎。屠州黑豹,禺氏騊駼。大夏兹白牛,兹白牛野兽也,牛形而象齿。犬戎文马,文马赤鬣缟身,目若黄金,名古黄之乘。数楚每牛,每牛者,牛之小者也。匈奴狡犬,狡犬者,巨身四足果。皆北向。

权扶玉目。白州比闾,比闾者,其华若羽,伐其木以为车,终行不败。禽人菅,路人大竹,长沙鳖。其西鱼复,鼓锺,锺牛。蛮杨之翟。仓吾翡翠,翡翠者,所以取羽。其余皆可知。自古之政,南人至,众皆北向。

伊尹朝,献商书,汤问伊尹曰:``诸侯来献,或无马牛之所生,而献远方之物事实相反不利。今吾欲因其地势,所有献之,必易得而不贵,其为四方献令。''伊殷受命,于是为四方令曰:``臣请正东,符娄、仇州、伊虑、沤深、十蛮、越沤,剪发文身,请令以鱼皮之鞞,乌鰂之酱,鲛鼥利剑为献。正南,瓯邓、桂国、损子、产里、百濮、九菌,请令以珠玑、玳瑁、象齿、文犀、翠羽、菌鹤、短狗为献。正西,昆仑、狗国、鬼亲、枳巳、闟耳贯胸、雕题、离卿、漆齿,请令以丹青、白旄、纟比罽、江历、龙角、神龟为献。正北空同、大夏、莎车、姑他、旦略、豹胡、代翟,匈奴、楼烦、月氏、韯犁、其龙、东胡,请令以橐驼、白玉、野马、騊駼、駃騠、良弓为献。''

汤曰:``善。''

\hypertarget{header-n323}{%
\subsection{卷八}\label{header-n323}}

\hypertarget{header-n327}{%
\subsubsection{祭公解}\label{header-n327}}

王若曰:``祖祭公,予小子虔虔在位,昊天疾威,予多时溥愆。我闻祖不豫,有加予,维敬省不吊,田降疾病,予畏天威,公其告予懿德。''

祭公拜手稽首曰:``天子,谋复疾,维不瘳,朕身尚在兹,朕魄在于天。昭王之所勖宅天命。''

王曰:``呜呼,公,朕皇祖文王、烈祖武王,度下国,作陈周,维皇皇上帝,度其心,置之明德。付俾于四方,用应受天命,敷文在下。我亦维有若文祖周公,暨列祖召公,兹申予小子,追学于文武之蔑。用克龛绍成康之业,以将天命,用夷居之大商之众。我亦维有若祖祭公之执,和周国,保憹王家。''

王曰:``公称丕显之德,以予小子扬文武大勋,弘成康昭考之烈。''

王曰:``公无困我哉,俾百僚,乃心率辅弼予一人。''

祭公拜,稽首曰:``允乃诏,毕桓于黎民般。''

公曰:``天子谋父,疾维维不瘳,敢告天子皇天,改大殷之命,。维文王受之,维武王大克之,咸茂厥功。维天贞文王之董用威,亦尚宽壮,厥心康受憹之式用休。亦先王茂绥厥心敬恭承之,维武王申大肆命戡厥敌。''

公曰:``天子自三公上下辟于文武。文武之子孙,大开方封于下土。天之所锡,武王使,疆土丕维周之基。丕维后稷之受命,是永宅之。维我后嗣旁建宗子,丕维周之始并。呜呼,天子三公监于夏商之既败,丕则无遗后难,至于万亿年,守序终之。既毕丕,乃有利宗,丕维文王由之。''

公曰:``呜呼,天子,我不则寅哉,寅哉。汝无以戾反罪,疾丧时二王大功。汝无以嬖御固庄后。汝无以小谋败大作,汝无以嬖御士疾大夫卿士,汝无以家相乱王室而莫恤其外。尚皆以时中憹万国。呜呼,三公,汝念哉!汝无泯泯芬芬,厚颜忍丑,时维大不吊哉。昔在先王,我亦维丕,以我辟险于难,不失于正,我亦以免没我世。呜呼,三公!予维不起,朕疾。汝其皇敬哉!兹皆保之。''

曰:``康子之攸保勖教诲之,世祀无绝,不,我周有常刑。''

王拜手稽首党言。

\hypertarget{header-n331}{%
\subsubsection{史记解}\label{header-n331}}

维正月王在成周,昧爽,召三公、左史戎夫,曰:``今夕朕寤,遂事惊予。''乃取遂事之要戒,俾戎夫主之,朔望以闻。

信不行、义不立,则哲士凌君政,禁而生乱,皮氏以亡。

谄谀日近,方正日远,则邪人专国政。禁而生乱,华氏以亡。好货财珍怪,则邪人进,邪人进,则贤良日蔽而远。赏罚无位,随财而行,夏后失以亡。

严兵而不□者,其臣慑;其臣慑,则不敢忠;不敢忠,则民不亲其吏。刑始于亲,远者寒心,殷商以亡。

乐专于君者,权专于臣,权专于臣则刑专于民。君娱于乐,臣争于权,民尽于刑,有虞氏以亡。

奉孤以专命者,谋主比畏其威,而疑其前事。挟德而责是,日疏位均而争,平林以亡。

大臣有锢职哗诛者,危。昔者,质沙三卿,朝而无礼,君怒而久据之,哗而弗加。哗卿谋变,质沙以亡。

外内相间,下挠其民,民无所附,三苗以亡。

弱小在强大之间,存亡将由之,则无天命矣。不知命者死。有夏之方兴也,扈失弱而不恭,身死国亡。

嬖子两重者亡。昔者,义渠氏有两子异母,皆重。君疾大臣,分党而争,义渠以亡。

功大不赏者危。昔平州之臣,功大而不赏,谗臣日贵,功臣日怒而生变,平州之君以走出。

召远不亲者,危。昔有林失召离戎之君而朝之,至而不礼,留而弗亲,离戎逃而去之,林失诛之,天下叛林氏。

昔者曲集之君,伐智而专事强力,而不信其臣,忠良皆伏。愉州氏伐之,君孤而无使,曲集以亡。

昔者有巢氏,有乱臣而贵任之,以国假之,以权擅国而主断,君已而夺之,臣怒而生变,有巢以亡。

斧小不胜柯者,亡。昔有会阝君啬俭,减爵损禄,群臣卑让,上下不临,后□小弱禁罚不行,重氏伐之,会阝君以亡。

久空重位者危。昔有共工,自贤,自以无臣,久空大官,下官交乱,民无所附,唐氏伐之,共工以亡。

犯难争权,疑者死。昔有林氏,上衡氏争权,林氏再战而胜,上衡氏伪义,弗克,俱身死国亡。

知能均而不亲,并重事君者危。昔有南氏,有二臣,贵宠,力钧势底,竞进争权,下争朋党,君弗能禁,南氏以分。

昔有果氏,好以新易故,故者疾怨,新故不和,内争朋党,阴事外权,有果氏以亡。

爵重禄轻,比□不成者亡。昔有毕程氏,损顺增爵,群臣貌匮,比而戾民,毕程氏以亡。好变故易常者,亡。昔阳氏之君,自伐而好变,事无故业,官无定位,民运于下,阳氏以亡。

业形而愎者,危。昔谷平之君,愎类无亲,破国弗克,业形用国,外内相援,谷平以亡。

武不止者,亡。昔阪泉氏用兵无已,诛战不休,并兼无亲,文无所立,智士寒心,徙居至于独鹿,诸侯畔之,阪泉以亡。

狠而无亲者,亡。昔者县宗之君,狠而无听,执事不从,宗职者疑,发大事,群臣解体,国无立功,县宗以亡。

昔者玄都贤鬼道,废人事天,谋臣不用,龟策是从,神巫用国,哲士在外,玄都以亡。

文武不行者,亡。昔者西夏性仁非兵,城郭不修,武士无位,惠而好赏,屈而无以赏,唐氏伐之,城郭不守,武士不用,西夏以亡。美女破国。昔者绩阳强力四征,重丘遗之美女,绩阳之君悦之,荧惑不治,大臣争权,远近不相听,国分为二。

宫室破国。昔者有洛氏宫室无常,池囿大,工功日进,以后更前,民不得休,农失其时,饥馑无食,成商伐之,有洛以亡。''

\hypertarget{header-n335}{%
\subsubsection{职方解}\label{header-n335}}

职方氏掌天下之图,辩其邦国、都鄙、四夷、八蛮、七闽、九貉、五戎、六狄之人民,与其财用九谷六畜之数,周知其利害,乃辩九州之国,使同贯利。

东南曰扬州,其山镇曰会稽,其泽薮曰具区,其川三江,其浸五湖,其利金锡竹箭,其民二男五女,其畜宜鸡狗鸟兽,其谷宜稻。

正南曰荆州,其山镇曰衡山,其泽薮曰云梦,其川江汉,其浸颍湛,其利丹银齿革,其民一男二女,其畜宜鸟兽,其谷宜稻。

河南曰豫州,其山镇曰华山,其泽薮曰圃田,其川荧雒,其浸陂溠,其利林漆丝枲,其民二男三女,其畜宜六扰,其谷宜五种。

正东曰青州,其山镇曰沂山,其泽薮曰望诸,其川淮泗,其浸沂沭,其利蒲鱼,其年二男三女,其畜宜鸡犬,其谷宜稻麦。

河东曰兗州,其山镇曰岱山,其泽薮曰大野,其川河,其浸庐濰,其利蒲鱼,其民二男三女,其畜宜六扰,其谷宜四种。

正西曰雍州,其山镇曰岳山,其泽薮曰强蒲,其川泾汭,其浸渭洛,其利玉石,其民三男二女,其畜宜牛马,其谷宜黍稷。

东北曰幽州,其山镇曰医无闾,其泽薮曰貕养,其川河,其浸菑时,其利鱼盐,其民一男三女,其畜宜四扰,其谷宜三种。

河内曰冀州,其山镇曰化验山,其泽薮曰扬纡,其川漳,其浸汾露,其利松柏,其民五男三女,其畜宜牛羊,其谷宜黍稷。

正北曰并州,其山镇曰恆山,其泽薮曰昭余祁,其川虖池,呕夷,其浸涞易,其利布帛,其民二男三女,其畜宜五扰,其谷宜五种。

乃辩九服之国,方千里曰王圻,其外方五百里为侯服,又其外方五百里为甸服,又其外方五百传统为甸服,又其外方五百里我男服,又其外方五百里为采服,又其外方五百里为卫服,又其外方五百里为蛮服,又其外方五百里为夷服,又其外方五百里为镇服,又其外方五百里为籓服。

凡国公侯伯子男,以周知天下。凡邦国大小相维,王设其牧,制其职,各以其所能,制其贡,各以其所有。王将巡狩,则戒于四方,曰各修平乃守,考乃职事,无敢不敬戒,国有大刑。及王者之所行道,率其属而巡戒命,王殷国亦如之。

\hypertarget{header-n343}{%
\subsection{卷九}\label{header-n343}}

\hypertarget{header-n347}{%
\subsubsection{芮良夫解}\label{header-n347}}

芮伯若曰:``予小臣良夫,稽道谋告,天子惟民父母,致厥道,无远不服,无道,左右臣妾乃违。民归于德,德则民戴,否则民雠。兹言允效与前不远。商纣不道,夏桀之虐肆无有家。呜呼,惟尔天子嗣文武业,惟尔执政小子同先王之臣昏行□顾道,王不若,专利作威,佐乱进祸,民将弗堪。治乱信乎,其行惟王,暨而执政小子攸闻。古人求多闻以监戒,不闻是惟弗知。除民害不惟民害,害民乃非后,惟其雠。后作类,后弗类,民不知后,惟其怨。民至亿兆,后一而已,寡不敌众,后其危哉。

``呜呼!□□□如之。今尔执政小子,惟以贪谀为事,不勤德以备难。下民胥怨,财力单竭,手足靡措,弗堪上,不其乱而。以予小臣良夫,观天下有土之君,厥德不远,罔有代德。时为王之患,其惟国人。呜呼!惟尔执政朋友小子其惟洗尔心、改尔行,克忧往愆,以保尔居。尔乃聩祸玩烖,遂弗悛,余未知王之所定,矧乃□□。惟祸发于人之攸忽,于人之攸轻,□不存焉。变之攸伏。尔执政小子不图善,偷生苟安,爵以贿成,贤智箝口,小人鼓舌,逃害要利,并得厥求,唯曰哀哉。

``我闻曰,以言取人,人饰其言;以行取人,人竭其行。饰言无庸,竭行有成。惟尔小子,饰言事王,黡蕃有徒。王貌受之,终弗获用,面相诬蒙,及尔颠覆。尔自谓有余,予谓尔弗足。敬思以德,备乃祸难。难至而悔,悔将安及,无曰予为惟尔之祸也。''

\hypertarget{header-n351}{%
\subsubsection{太子晋解}\label{header-n351}}

晋平公使叔誉于周,见太子晋而与之言。五称而三穷,逡巡而退,其言不遂。归告公曰:``太子晋行年十五,而臣弗能与言。君请归,声就复与田,若不反,及有天下,将以为诛。''平公将归之,师旷不可曰:``请使瞑臣往与之言,若能幪予,反而复之。''

师旷见太子,称曰:``吾闻王子之语,高于泰山,夜寝不寐,昼居不安,不远长道,而求一言。''

王子应之曰:``吾闻太师将来,甚喜。热又惧吾年臣少,见子而慎,尽忘吾其度。''

师旷曰:``吾闻王子,古之君子,甚成不骄,自晋始如周,行不知劳。''

王子应之曰:``声之君子,其行至慎,委积施关,道路无限,百姓悦之,相将而远,远人来欢,视道如咫。''

师旷告善,又称曰:``古之君子,其行可则,由舜而下,其孰有广德?''

王子应之曰:``如舜者天,舜居其所以利天下,奉翼远人,皆得己仁,此之谓天。如禹者,圣劳而不居,以利天下,好取不好与,必度其正,是谓之圣。如文王者,其大道仁,其小道惠。三分天下而有其二,敬人无方,服事于商,既有其众,而返失其身,此之谓仁。如武王者义,杀一人而以利天下,异姓同姓各得其所,是之谓仪。''

师旷告善。又称曰:``宣辨名命,异姓恶之。王侯君公,何以为尊,何以为上?''

王子应之曰:``人生而重丈夫,谓之胄子;胄子成人能治上官,谓之士;士率众时作,谓之曰伯;伯能移善于众,与百姓同,谓之公;公能树名生物,与天道俱,谓之侯,侯能成群,谓之君。君有广德,分任诸侯而敦信,曰予一人;善至于四海,曰天子,达于四荒曰天王。四荒至,莫有怨訾,乃登为帝。''

师旷罄然。又称曰:``温恭敦敏,方德不改,闻物□□,下学以起,尚登帝臣,乃参天子,自古谁?''

王子应之曰:``穆穆虞舜,明明赫赫,立义治律,万物皆作,分均天财,万物熙熙,非舜而谁能?''

师旷东躅其足,曰:``善哉,善哉!''王子曰:``太师何举足骤?''师旷曰:``天寒足跔,是以数也。''

王子曰:``请入坐。''遂敷席注瑟。师旷歌无射,曰:``国诚宁矣,远人来观,修义经矣,好乐无荒。''乃注瑟于王子,王子歌峤曰:``何自南极,至于北极,绝境越国,弗愁道远。''

师旷蹶然起,曰:``瞑臣请归。''王子赐之乘车四马,曰:``太师亦善御之。''师旷对曰:``御吾未之学也。''王子曰:``汝不为夫《诗》,《诗》云:`马之刚矣,辔之柔矣,马亦不刚,辔亦不柔,志气镳镳,取予不疑。'以是御之。''师旷对曰:``瞑臣无见,为人辩也,唯耳之恃,而耳又寡闻而易穷。王子,汝将为天下宗乎?''

王子曰:``太师何汝戏我乎?自太昊以下,至于尧舜禹,未有一姓而再有天下者,夫大当时而不伐,天何可得?吾闻汝知人年之长短,告吾。''

师旷对曰:``汝声清汗,汝色赤白,火色不寿。''

王子曰:``然。吾后三年,将上宾于帝所,汝慎无言,殃将及汝。''

师旷归,未及三年,告死者至。

\hypertarget{header-n355}{%
\subsubsection{王佩解}\label{header-n355}}

王者所佩在德,德在利民,民在顺上。合为在因时,应事则易成。谋成在周,长有功在力多。昌大在自克,不过在数惩。不困在豫慎,见祸在未形。除害在能断,安民在知过,用兵在知时,胜大患在合人心。殃毒在信疑,孽子在听内,化行在知和,施舍在平心。不幸在不闻其过,福在受谏,基在爱民,固在亲贤。祸福在所密,利害在所近,存亡在所用,离合在出命。尊在慎,威安在恭己,危亡在不知时。见善而怠,时至而疑,亡正处邪,弗能居此,得失之方也,不可不察。

\hypertarget{header-n359}{%
\subsubsection{殷祝解}\label{header-n359}}

汤将放桀于中野,士民闻汤在野,皆委货扶老携幼奔,国中虚。桀请汤曰:``国所以为国者,以有家;家所以为家者,以有人也。今国无家无人矣,君有人,请致国君之有也。''

汤曰:``否。昔大帝作道,明教士民。今君王灭道残政,士民惑矣,吾为王明之。''士民复致于桀,曰:``以薄之居,济民之贱,何必君更?''桀与其属五百人南徙千里,止于不齐,民往奔汤于中野。桀复请汤,言:``君之有也。''汤曰:``否。无为君王明之,士民复重请之。''桀与其属五百人徙于鲁,鲁士民复奔汤。

桀又曰:``国君之有也,吾则外。人有言,彼以吾道是邪,我将为之。''汤曰:``此君王之士也,君王之民也,委之何?''汤不能止桀。汤曰:``欲从者,从君。''桀与其属五百人去居南巢。

汤放桀,而复薄三千诸侯大会,汤退,,再拜,从诸侯之位。汤曰:``此太子位,有道者可以处之,天下非一家之有也,有道者之有也。故天下者,唯有道者理之,唯有道者纪之,唯有道者宜久处之。''

汤以此让,三千诸侯莫敢即位,然后汤即天子之位。与诸侯誓曰:``阴胜阳即谓之变,而天弗施。雌胜雄即谓之乱,而人弗行。''故诸侯之治,政在诸侯之大夫,治与从。

\hypertarget{header-n363}{%
\subsubsection{周祝解}\label{header-n363}}

曰:维哉其时,告汝□□道,恐为深灾,欢哉,民乎,朕则生汝,朕则刑汝。朕则经汝,朕则亡汝,朕则寿汝,朕则名汝。故曰:文之美而以身剥,自谓智也者,故不足。角之美,杀其牛,荣华之言,后有茅。凡彼济者,必不怠。观彼圣人,必趣时。石有玉伤其山,万民之患在□言及。时之行也,勤以徙,辟召道者,福为祸。时之从也,勤以行,不知道者以福亡。故曰:费豕必烹,甘泉必竭,直木必伐。

地出物而圣人是时,鸡鸣而人为时,观彼万物,且何为求?故他有时,人以为正;地出利,而民是争。人出谋,圣人是经,陈五刑,民乃敬。教之以礼,民不争,被之以刑,民始听。因其能,民乃静。故狐有牙而不敢以噬,豲有蚤而不敢以撅。势居小者,不能为大。特欲正中,不贪其害。凡势道者,不可以不大。故木之伐也,而木为斧贼,难之起,自近者。二人同术,谁昭谁暝;二虎同穴,谁死谁生。故虎之猛也,而陷于获;人之智也,而陷于诈。曰之美也,解其柯;柯之美也,离其枝;枝之美也,拔其本。俨矢将至,不可以无盾。

故泽有兽而焚其草木,大威将至,不可为巧。焚其草木则无种,大威将至,不可以为勇。故天之生也,固有度;国家之患,离之以故。地之生也,国有植,国家之患,离之以谋。故时之还也,无私貌;日之出也,无私照。时之行也,顺至无逆。为天下者,用大略。火之燀也,固定上。为天下者,用牧。水之流也,固走下。不善,故有桴。故福之起也,恶别之;祸之起也,恶别之。

故平国若之何?须国覆国事国孤国屠,皆若之何?故日之中也,仄月之望也。食威之失也,阴食阳。善为国者,使之有行。是彼万物,必有常。国君而无道,以微亡。故天为盖,地为轸。善用道者,终无尽。地为轸,天为盖,善用道者,终无害。天地之间,有沧热,善用道者,终不竭。陈彼五行,必有胜,天之所覆,尽可称。故万物之所生也,性于从;万物之所反也,性于同。故恶姑幽,恶姑明,恶姑阴阳,恶姑短长,恶姑刚柔。

故海之大也,而鱼何为可得?山之深也,虎豹貔貅何为可服?人智之邃也,奚何为可测?跂动哕息,而奚为可牧?玉石之坚也,奚可刻?阴阳之号也,孰使之?牝牡之合也,孰交之?君子不察,福不来。故忌而不得,是生事;故欲而不得,是生诈。欲伐而不得,深斧柯;欲鸟而不得,生网罗;欲彼天下,是生为。维彼幽心,是生包;维彼大心,是生雄;维彼忌心,是生胜。

故天为高,地为下,察汝躬,奚为喜怒。天为古,地为久,察彼万物,名于始。左名左,右名右。视彼万物,数为纪。纪之行也,利而无方,行而无止,以观人情。利有等维,彼大道成而弗改,用彼大道知其极,加诸事,则万物服。用其则必有群,加诸物则为之君,举其修则有理,加诸物则为天子。

\hypertarget{header-n371}{%
\subsection{卷十}\label{header-n371}}

\hypertarget{header-n375}{%
\subsubsection{武纪解}\label{header-n375}}

币帛之间,有巧言令色,事不成。车甲之间,有巧言令色,事不捷。克□事而有武色,必失其德。临权而疑,必离其灾。□□不捷,智不可□,□于不足,并于不几,则始而施,几而弗免,无功。

国有三守,卑辞重币以服之,弱国之守也;伐服不祥,伐战危,伐险难,故善反而者不伐三守。伐国有六时、五动、四顺,间其疏薄其疑,推其危扶其弱,乘其衰,暴其约,此谓六时。扶之而不让,振之而不动,是之而不服,暴之而不革,威之而不恐,未可伐也,此谓五动。立之害,毁之利,克之易,并之能,以时伐之,此谓四顺。立之不害,毁之不利,唯克之易,并之不能,可伐也。立之害,毁之未利,克之难,并之不能,可动也。静以待众,力不与争,权弗果据,德不肆国,若是,而可毁也。地荒而不振,得衰而氏与,无苦而危矣,求之以其道,□□无不得,为之以其事,而时无不成。有利备无患,事时至而不迎,大禄乃迁。延之不道,行事乃困,不作小□,动大殃。

谋有不足者三:仁废,则文谋不足;勇废,武谋不足;备废,则事谋不足。国有本有干,有权有伦。质有枢体,土地本也,人民干也,敌国侔交权也。政教顺,成伦质也。君臣和,□枢体也。土地未削,人民未散,国权未倾,伦质未移,虽有昏乱之君,国未亡也。国有几失,居之不可阻,体之小也。不果邻家,难复饰也。封疆侵凌,难复振也。服国从失,难复扶也。大国之无养,小国之畏事,不可以本权,失□家之交,不可以枉绳。失邻家之交,不据直以约,不亏体以阴,不可虞而夺也,不可策而服也,不可亲而侵也,不可摩而测也,不可求而循也。

施度于体,不虑费事;利于国,不计劳。失德丧服于邻家,则不顾难矣。交体侵凌,则不顾权矣。封疆不时得其所,无为养民矣。合同不得其位,无畏患矣。百姓屈急,无藏蓄矣。挤社稷、失宗庙、离坟墓、困鬼神、残宗族,无为爱死矣。卑辞而不听,□财而无枝,计战而□足,近告而无顾,告过而不悔,请服而不得,然后绝好于闭门,循险近,说外援以天命,无为是定亡矣。

凡有事君民,守社稷宗庙,而先衰亡者,皆失礼也。大事不法,弗可作;法而不时,弗可行;说而失礼,弗可长;得礼而无备,弗可成;举物不备,而欲□大功于天下者,未之有也。势不求周流,举而不几其成,亡。薄其事而求厚其功,亡。内无文,道外无武,迹往不复,来者有悔,而求合者,亡。不难不费,而致大功,古今未有。

据名而不辱,应行而不困,唯礼;得之而无逆,复之而无咎,唯敬;成事而不难,序功而不费,唯时;劳而有成,费而不亡,唯当;施而不拂,成而有权,久之而能□,唯义。不知所取之量,不知所施之度,不知动静之时,不知吉凶之事,不知困达之谋,疑此五者,未可以动大事。恃名不久,恃功不立,虚愿不至,妄为不祥。太上敬而服,其次欲而得,其次夺而得,其次争而克,其下动而上资其力。凡建国君民,内事文而和。外事武而义,其形慎而杀,其政直而公,本之以礼,动之以时,正之以度,师之以法,成之以仁。此之道也。

\hypertarget{header-n379}{%
\subsubsection{铨法解}\label{header-n379}}

有三不远,有三不近,有三不畜。敬谋、祗德、亲同,三不远也。听谗自乱、听谀自欺,近憝自恶,三不近也。有如忠言竭,亲以为信;有如同好,以谋易寇;有如同恶,合计掬虑,虑泄事败;是谓好害,三不畜也。

\hypertarget{header-n383}{%
\subsubsection{器服解}\label{header-n383}}

明器因外,有三疲二用。器服:数犊四棓禁丰一\textless{}角豪\textgreater{}天韦独食器甒迤膏侯屑侯乐铋枼参冠一竿,皆素。独二丸弇焚菜脍五,昔纁里桃枝素独蒲箪席,皆素。斧独巾玄缋緌缟冠、素纰、玄冠、组武、缨象、□□、瑱絺、绅带、象玦、硃极、韦素,独簟、籥捍、次车、羔冒、□纯、载枉线丧勤焚缨一给器,因名有三:几玄菌纁里桃枝独蒲席,皆素,布独巾,玄象玄纯。

\hypertarget{header-n391}{%
\subsection{附录}\label{header-n391}}

\hypertarget{header-n395}{%
\subsubsection{序}\label{header-n395}}

昔在文王,商纣并立,困于虐政,将弘道以弼无道,作《度训》。殷人作,教民不知极,将明道极,以移其俗,作《命训》。纣作淫乱,民散无性习常,文王惠和,化服之,作《常训》。上失其道,民散无纪,西伯修仁,明耻示教,作《文酌》。上失其道,民失其业,□□凶年,作《籴匡》。文王立,西距昆夷,北备猃狁,谋武以昭威怀,作《武称》。武以禁暴,文以绥德,大圣允兼,作《允文》。武有七德,□王作《大武》、《大明武》、《小明武》三篇。穆王遭大荒,谋救患分灾,作《大匡》。□□□□□□□□□□□□□□□□□□□□□□□□□□□□作《九开》。文王唯庶邦之多难,论典以匡谬,作《刘法》。文王卿士谂发教禁戒,作《文开》。维美公命于文王,修身观天以谋商难,作《保开》。文王训乎武王以繁害之戒,作《八繁》。文王在酆,命周公谋商难,作《酆保》。文启谋乎后嗣,以修身敬戒,作《大开》、《小开》二篇。文王有疾,告武王以没之多变,作《文儆》。文王告武王以序德之行,作《文传》。文王既没,武王嗣位,告周公禁五戎,作《柔武》。武王忌商,周公勤天下,作《大小开武》二篇。武王评周公,维道以为宝,作《宝典》。商谋启平周,周人将兴师以承之,作《酆谋》。武王将起师伐商,寤有商儆,作《寤儆》。周将伐商,顺天革命,申喻武义,以训乎民,作《武顺》、《武穆》二篇。武王将行大事乎商郊,乃明德□众,作《和寤》、《武寤》二篇。武王率六州之兵,车三百五十乘,以灭殷,作《克殷》。武王既克商,建三监以救其民,为之训范,□□□□□□□□□作《大聚》。□□□□□□□□□□□武王既释箕子囚,俾民辟宁之以王,作《箕子》。武王秉天下,论德施□,而□位以官,作《考德》。武王命商王之诸侯绥定厥邦,申义告之,作《商誓》。武王平商,维定保天室,规拟伊洛,作《度邑》。武王有疾,□□□□□□□□□□命周公辅小子,告以正要,作《五权》。武王既没,成王元年,周公忌商之孽,训敬命,作《成开》。周公既诛三监,乃谁武王之志,建都伊洛,作《作洛》。周公会群臣于闳门,以辅主之格言,作《皇门》。周公陈武王之言及,以赞己言,戒乎成王,作《大戒》。周公正三统之义,作《周月》,辩二十四气之应,以明天时,作《时训》。周公制十二月赋政之法,作《月令》。周公肇制文王之谥义,以垂于后,作《谥法》。周公将致政成王,朝诸侯于明堂,作《明堂》。成王近即政,因尝麦以语众臣,而求助,作《尝麦》。周公为太师,告成王以五则,作《本典》。成王访周公以民事,周公陈六征以观察之,作《官人》。周石板嘲弄宁,八方会同,各以其职来献,欲垂法厥后,作《王会》。周公云殁,王制将衰,穆王因祭祖不豫,询某守位,作《祭公》。穆王思保位惟难,恐贻世羞,欲自警悟,作《史记》。王化虽弛,天命方永,四夷八蛮,攸尊王政,作《职方》。芮伯稽古,作《训纳》。王于善暨,执政小臣咸省厥躬,作《芮良夫》。晋侯尚力,侵我王略,叔向闻储幼而果贤,□复王位,作《太王晋》。王者德以饰躬,用为所佩,作《王佩》。夏多罪,汤将放之,征前事以戒后王也,作《殷祝》。民非后罔义,后非民罔与,为邦慎政在微,作《周祝》。武以靖乱,非直不克,作《武纪》。积习生常,不可不慎,作《铨法》。车服制度,明不苟逾,作《器服》。周道于是乎大备。

\end{document}
