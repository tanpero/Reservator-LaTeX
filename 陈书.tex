\PassOptionsToPackage{unicode=true}{hyperref} % options for packages loaded elsewhere
\PassOptionsToPackage{hyphens}{url}
%
\documentclass[]{article}
\usepackage{lmodern}
\usepackage{amssymb,amsmath}
\usepackage{ifxetex,ifluatex}
\usepackage{fixltx2e} % provides \textsubscript
\ifnum 0\ifxetex 1\fi\ifluatex 1\fi=0 % if pdftex
  \usepackage[T1]{fontenc}
  \usepackage[utf8]{inputenc}
  \usepackage{textcomp} % provides euro and other symbols
\else % if luatex or xelatex
  \usepackage{unicode-math}
  \defaultfontfeatures{Ligatures=TeX,Scale=MatchLowercase}
\fi
% use upquote if available, for straight quotes in verbatim environments
\IfFileExists{upquote.sty}{\usepackage{upquote}}{}
% use microtype if available
\IfFileExists{microtype.sty}{%
\usepackage[]{microtype}
\UseMicrotypeSet[protrusion]{basicmath} % disable protrusion for tt fonts
}{}
\IfFileExists{parskip.sty}{%
\usepackage{parskip}
}{% else
\setlength{\parindent}{0pt}
\setlength{\parskip}{6pt plus 2pt minus 1pt}
}
\usepackage{hyperref}
\hypersetup{
            pdfborder={0 0 0},
            breaklinks=true}
\urlstyle{same}  % don't use monospace font for urls
\setlength{\emergencystretch}{3em}  % prevent overfull lines
\providecommand{\tightlist}{%
  \setlength{\itemsep}{0pt}\setlength{\parskip}{0pt}}
\setcounter{secnumdepth}{0}
% Redefines (sub)paragraphs to behave more like sections
\ifx\paragraph\undefined\else
\let\oldparagraph\paragraph
\renewcommand{\paragraph}[1]{\oldparagraph{#1}\mbox{}}
\fi
\ifx\subparagraph\undefined\else
\let\oldsubparagraph\subparagraph
\renewcommand{\subparagraph}[1]{\oldsubparagraph{#1}\mbox{}}
\fi

% set default figure placement to htbp
\makeatletter
\def\fps@figure{htbp}
\makeatother


\date{}

\begin{document}

\hypertarget{header-n1065}{%
\section{陈书}\label{header-n1065}}

\begin{center}\rule{0.5\linewidth}{\linethickness}\end{center}

\tableofcontents

\begin{center}\rule{0.5\linewidth}{\linethickness}\end{center}

\hypertarget{header-n4159}{%
\subsection{本纪}\label{header-n4159}}

\hypertarget{header-n4160}{%
\subsubsection{卷一}\label{header-n4160}}

高祖上

高祖武皇帝,讳霸先,字兴国,小字法生,吴兴长城下若里人,汉太丘长陈实之后也。世居颍川。实玄孙准,晋太尉。准生匡,匡生达,永嘉南迁,为丞相掾,历太子洗马,出为长城令,悦其山水,遂家焉。尝谓所亲曰:``此地山川秀丽,当有王者兴,二百年后,我子孙必钟斯运。''达生康,复为丞相掾,咸和中土断,故为长城人。康生盱眙太守英,英生尚书郎公弼,公弼生步兵校尉鼎,鼎生散骑侍郎高,高生怀安令咏,咏生安成太守猛,猛生太常卿道巨,道巨生皇考文赞。高祖以梁天监二年癸未岁生。少倜傥有大志,不治生产。既长,读兵书,多武艺,明达果断,为当时所推服。身长七尺五寸,日角龙颜,垂手过膝。尝游义兴,馆于许氏,夜梦天开数丈,有四人硃衣捧日而至,令高祖开口纳焉。及觉,腹中犹热,高祖心独负之。

大同初,新喻侯萧映为吴兴太守,甚重高祖,尝目高祖谓僚佐曰:``此人方将远大。''及映为广州刺史,高祖为中直兵参军,随府之镇。映令高祖招集士马,众至千人,仍命高祖监宋隆郡。所部安化二县元不宾,高祖讨平之。寻监西江督护、高要郡守。先是,武林侯萧谘为交州刺史,以裒刻失众心,土人李贲连结数州豪杰同时反,台遣高州刺史孙冏、新州刺史卢子雄将兵击之,冏等不时进,皆于广州伏诛。子雄弟子略与冏子侄及其主帅杜天合、杜僧明共举兵,执南江督护沈顗,进寇广州,昼夜苦攻,州中震恐。高祖率精兵三千,卷甲兼行以救之,频战屡捷,天合中流矢死,贼众大溃。僧明遂降。梁武帝深叹异焉,授直阁将军,封新安子,邑三百户,仍遣画工图高祖容貌而观之。

其年冬,萧映卒。明年,高祖送丧还都,至大庾岭,会有诏高祖为交州司马,领武平太守,与刺史杨蒨南讨。高祖益招勇敢,器械精利。蒨喜曰:``能克贼者,必陈司武也。''委以经略。高祖与众军发自番禺。是时萧勃为定州刺史,于西江相会,勃知军士惮远役,阴购诱之,因诡说蒨。蒨集诸将问计,高祖对曰:``交趾叛涣,罪由宗室,遂使僭乱数州,弥历年稔。定州复欲昧利目前,不顾大计,节下奉辞伐罪,故当生死以之。岂可畏惮宗室,轻于国宪?今若夺人沮众,何必交州讨贼,问罪之师,即回有所指矣。''于是勒兵鼓行而进。十一年六月,军至交州,贲众数万于苏历江口立城栅以拒官军。蒨推高祖为前锋,所向摧陷,贲走典澈湖,于屈獠界立砦,大造船舰,充塞湖中,众军惮之,顿湖口不敢进。高祖谓诸将曰:``我师已老,将士疲劳,历岁相持,恐非良计,且孤军无援,入人心腹,若一战不捷,岂望生全。今藉其屡奔,人情未固,夷獠乌合,易为摧殄,正当共出百死,决力取之,无故停留,时事去矣。''诸将皆默然,莫有应者。是夜江水暴起七丈,注湖中,奔流迅激。高祖勒所部兵,乘流先进,众军鼓噪俱前,贼众大溃。贲窜入屈獠洞中,屈獠斩贲,传首京师,是岁太清元年也。贲兄天宝遁入九真,与劫帅李绍隆收余兵二万,杀德州刺史陈文戒,进围爱州,高祖仍率众讨平之。除振远将军、西江督护、高要太守、督七郡诸军事。

二年冬,侯景寇京师,高祖将率兵赴援,广州刺史元景仲阴有异志,将图高祖。高祖知其计,与成州刺史王怀明、行台选郎殷外臣等密议戒严。三年七月,集义兵于南海,驰檄以讨景仲。景仲穷蹙,缢于阁下,高祖迎萧勃镇广州。是时临贺内史欧阳頠监衡州,兰裕、兰京礼扇诱始兴等十郡,共举兵攻頠,頠请援于勃。勃令高祖率众救之,悉擒裕等,仍监始兴郡。

十一月,高祖遣杜僧明、胡颖将二千人顿于岭上,并厚结始兴豪杰同谋义举,侯安都、张亻思等率千余人来附。萧勃闻之,遣钟休悦说高祖曰:``侯景骁雄,天下无敌,前者援军十万,士马精强,然而莫敢当锋,遂令羯贼得志。君以区区之众,将何所之?如闻岭北王侯又皆鼎沸,河东、桂阳相次屠戮,邵陵、开建亲寻干戈,李迁仕许身当阳,便夺马仗,以君疏外,讵可暗投?未若且住始兴,遥张声势,保此太山,自求多福。''高祖泣谓休悦曰:``仆本庸虚,蒙国成造。往闻侯景渡江,即欲赴援,遭值元、兰,梗我中道。今京都覆没,主上蒙尘,君辱臣死,谁敢爱命!君侯体则皇枝,任重方岳,不能摧锋万里,雪此冤痛,见遣一军。犹贤乎已,乃降后旨,使人慨然。仆行计决矣,凭为披述。''乃遣使间道往江陵,禀承军期节度。时蔡路养起兵据南康,勃遣腹心谭世远为曲江令,与路养相结,同遏义军。大宝元年正月,高祖发自始兴,次大庾岭。路养出军顿南野,依山水立四城以拒高祖。高祖与战,大破之,路养脱身窜走,高祖进顿南康。湘东王承制授高祖员外散骑常侍、持节、明威将军、交州刺史,改封南野县伯。

六月,高祖修崎头古城,徙居焉。高州刺史李迁仕据大皋,遣主帅杜平虏率千人入赣石、鱼梁。高祖命周文育将兵击走之,迁仕奔宁都。承制授高祖通直散骑常侍、使持节、信威将军、豫州刺史,领豫章内史,改封长城县侯。寻授散骑常侍、使持节、都督六郡诸军事、军师将军、南江州刺史,余如故。时宁都人刘蔼等资迁仕舟舰兵仗,将袭南康,高祖遣杜僧明等率二万人据白口,筑城以御之,迁仕亦立城以相对。二年三月,僧明等攻拔其城,生擒迁仕送南康,高祖斩之。承制命高祖进兵定江州,仍授江州刺史,余如故。

六月,高祖发自南康。南康赣石旧有二十四滩,滩多巨石,行旅者以为难。高祖之发也,水暴起数丈,三百里间巨石皆没。进军顿西昌,有龙见于水滨,高五丈许,五采鲜耀,军民观者数万人。是时承制遣征东将军王僧辩督众军讨侯景。八月,僧辩军次湓城,高祖率杜僧明等众军及南川豪帅合三万人将会焉。时西军乏食,高祖先贮军粮五十万石,至是分三十万以资之,仍顿巴丘。会侯景废简文帝,立豫章嗣王栋,高祖遣兼长史沈衮奉表于江陵劝进。十一月,承制授高祖使持节、都督会稽东阳新安临海永嘉五郡诸军事、平东将军、东扬州刺史,领会稽太守、豫章内史,余并如故。三年正月,高祖率甲士三万人、强弩五千张、舟舰二千乘,发自豫章。二月,次桑落洲,遣中记室参军江元礼以事表江陵,承制加高祖鼓吹一部。是时僧辩已发湓城,会高祖于白茅湾,乃登岸结坛,刑牲盟约。进军次芜湖,侯景城主张黑弃城走。三月,高祖与诸军进克姑孰,仍次蔡洲。侯景登石头城观望形势,意甚不悦,谓左右曰:``此军上有紫气,不易可当。''乃以叉步贮石沈塞淮口,缘淮作城,自石头迄青溪十余里中,楼雉相接。诸将未有所决,僧辩遣杜崱问计于高祖,高祖曰:``前柳仲礼数十万兵隔水而坐,韦粲之在青溪,竟不渡岸,贼乃登高望之,表里俱尽,肆其凶虐,覆我王师。今围石头,须渡北岸。诸将若不能当锋,请先往立栅。''高祖即于石头城西横陇筑栅,众军次连八城,直出东北。贼恐西州路断,亦于东北果林作五城以遏大路。景率众万余人、铁骑八百余匹,结阵而进。高祖曰:``军志有之,善用兵者,如常山之蛇,首尾相应。今我师既众,贼徒甚寡,应分贼兵势,以弱制强,何故聚其锋锐,令必死于我?''乃命诸将分处置兵。贼直冲王僧志,僧志小缩,高祖遣徐度领弩手二千横截其后,贼乃却。高祖与王琳、杜龛等以铁骑悉力乘之,贼退据其栅。景仪同卢辉略开石头北门来降。荡主戴冕、曹宣等攻拔果林一城,众军又克其四城。贼复还,殊死战,又尽夺所得城栅。高祖大怒,亲率攻之,士卒腾栅而入,贼复散走。景与百余骑弃槊执刀,左右冲阵,阵不动,景众大溃,逐北至西明门。景至阙下,不敢入台,遣腹心取其二子而遁。高祖率众出广陵应接,会景将郭元建奔齐,高祖纳其部曲三千人而还。僧辩启高祖镇京口。

五月,齐遣辛术围严超达于秦郡,高祖命徐度领兵助其固守。齐众七万,填堑,起土山,穿地道,攻之甚急。高祖乃自率万人解其围,纵兵四面击齐军,弓弩乱发,齐平秦王中流矢死,斩首数百级,齐人收兵而退。高祖振旅南归,遣记室参军刘本仁献捷于江陵。

七月,广陵侨民硃盛、张象潜结兵袭齐刺史温仲邕,遣使来告,高祖率众济江以应之。会齐人来聘,求割广陵之地,王僧辩许焉,仍豹高祖,高祖于是引军还南徐州,江北人随军而南者万余口。承制授高祖使持节、散骑常侍、都督南徐州诸军事、征北大将军、开府仪同三司、南徐州刺史,余并如故。及王僧辩率众征陆纳于湘州,承制命高祖代镇扬州。十一月,湘东王即位于江陵,改大宝三年为承圣元年。湘州平,高祖旋镇京口。三年三月,进高祖位司空,余如故。

十一月,西魏攻陷江陵,高祖与王僧辩等进启江州,请晋安王以太宰承制,又遣长史谢哲奉笺劝进。十二月,晋安王至自寻阳,入居朝堂,给高祖班剑二十人。四年五月,齐送贞阳侯深明还主社稷,王僧辩纳之,即位,改元曰天成,以晋安王为皇太子。初,齐之请纳贞阳也,高祖以为不可,遣使诣僧辩苦争之,往返数四,僧辩竟不从。高祖居常愤叹,密谓所亲曰:``武皇虽磐石之宗,远布四海,至于克雪仇耻,宁济艰难,唯孝元而已,功业茂盛,前代未闻。我与王公俱受重寄,语未绝音,声犹在耳,岂期一旦便有异图。嗣主高祖之孙,元皇之子,海内属目,天下宅心,竟有何辜,坐致废黜,远求夷狄,假立非次,观其此情,亦可知矣。''乃密具袍数千领,及锦彩金银,以为赏赐之具。九月壬寅,高祖召徐度、侯安都、周文育等谋之,仍部列将士,分赏金帛,水陆俱进。是夜发南徐诌讨王僧辩。甲辰,高祖步军至石头前,遣勇士自城北逾入。时僧辩方视事,外白有兵。俄而兵自内出,僧辩遽走,与其第三子頠相遇,俱出阁,左右尚数十人,苦战。高祖大兵寻至,僧辩众寡不敌,走登城南门楼。高祖因风纵火,僧辩穷迫,乃就擒。是夜缢僧辩及頠。丙午,贞阳侯逊位,百僚奉晋安王上表劝进。十月己酉,晋安王即位,改承圣四年为绍泰元年。壬子,诏授高祖侍中、大都督中外诸军事、车骑将军、扬南徐二州刺史,持节、司空、班剑、鼓吹并如故。仍诏高祖甲仗百人,出入殿省。

震州刺史杜龛据吴兴,与义兴太守韦载同举兵反。高祖命周文育率众攻载于义兴,龛遣其从弟北叟将兵拒战,北叟败归义兴。辛未,高祖表自东讨,留高州刺史侯安都、石州刺史杜棱宿卫台省。甲戌,军至义兴。丙子,拔其水栅。秦州刺史徐嗣徽据其城以入齐,又要南豫州刺史任约共举兵应龛、载,齐人资其兵食。嗣徽等以京师空虚,率精兵五千奄至阙下,侯安都领骁勇五百人出战,嗣徽等退据石头。丁丑,载及北叟来降,高祖抚而释之。以嗣徽寇逼,卷甲还都,命周文育进讨杜龛。十一月己卯,齐遣兵五千济渡据姑孰。高祖命合州刺史徐度于冶城寺立栅,南抵淮渚。齐又遣安州刺史翟子崇、楚州刺史刘仕荣、淮州刺史柳达摩领兵万人,于胡墅渡米粟三万石、马千匹,入于石头。癸未,高祖遣侯安都领水军夜袭胡墅,烧齐船千余艘,周铁虎率舟师断齐运输,擒其北徐州刺史张领州,获运舫米数千石。仍遣韦载于大航筑城,使杜棱据守。齐人又于仓门水南立二栅以拒官军。甲辰,嗣徽等攻冶城栅,高祖领铁骑精甲,出自西明门袭击之,贼众大溃。嗣徽留柳达摩等守城,自率亲属腹心,往南州采石,以迎齐援。十二月癸丑,高祖遣侯安都领舟师,袭嗣徽家口于秦州,俘获数百人。官军连舰塞淮口,断贼水路。先是太白自十一月丙戌不见。乙卯出于东方。丙辰,高祖尽命众军分部甲卒,对冶城立航渡兵,攻其水南二栅。柳达摩等渡淮置阵,高祖督兵疾战,纵火烧栅,烟尘张天。贼溃,争舟相排挤,溺死者以千数。时百姓夹淮观战,呼声震天地。军士乘胜,无不一当百,尽收其船舰,贼军慑气。是日嗣徽、约等领齐兵水步万余人,还据石头,高祖遣兵往江宁。据要险以断贼路。贼水步不敢进,顿江宁浦口,高祖遣侯安都领水军袭破之,嗣徽等乘单舸脱走,尽收其军资器械。己未,官军四面攻城,自辰讫酉,得其东北小城,及夜兵不解。庚申,达摩遣使侯子钦、刘仕荣等诣高祖请和,高祖许之,乃于城门外刑牲盟约,其将士部曲一无所问,恣其南北。辛酉,高祖出石头南门,陈兵数万,送齐人归北者。

壬戌,齐和州长史乌丸远自南州奔还历阳。江宁令陈嗣、黄门侍郎曹朗据姑孰反,高祖命侯安都、徐度等讨平之,斩首数千级,聚为京观。石头、采石、南州悉平,收获马仗船米不可胜计。是月杜龛以城降。二年正月癸未,诛杜龛于吴兴,龛从弟北叟、司马沈孝敦并赐死。

二月庚申,高祖遣侯安都、周铁虎率舸舰备江州,仍顿梁山起栅。甲子,敕司空有军旅之事,可骑马出入城内。戊辰,前宁远石城公外兵参军王位于石头沙际获玉玺四纽,高祖表以送台。

三月戊戌,齐遣水军仪同萧轨、厍狄伏连、尧难宗、东方老、侍中裴英起、东广州刺史独孤辟恶、洛州刺史李希光,并任约、徐嗣徽等,率众十万出栅口,向梁山,帐内荡主黄丛逆击,败之,烧其前军船舰,齐顿军保芜湖。高祖遣定州史沈泰、吴郡太守裴忌就侯安都,共据梁山以御之。

自去冬至是,甘露频降于钟山、梅岗、南涧及京口、江宁县境,或至三数升,大如弈棋子,高祖表以献台。

四月丁巳,高祖诣梁山军巡抚。五月甲申,齐兵发自芜湖,丙申,至秣陵故治。高祖遣周文育屯方山,徐度顿马牧,杜棱顿大航南。己亥,高祖率宗室王侯及朝臣将帅,于大司马门外白虎阙下刑牲告天,以齐人背约,发言慷慨,涕泗交流,同盟皆莫能仰视,士卒观者益奋。辛丑,齐军于秣陵故县跨淮立桥栅,引渡兵马。其夜至方山。侯安都、周文育、徐度等各引还京师。癸卯,齐兵自方山进及兒塘,游骑至台。周文育、侯安都顿白土岗,旗鼓相望,都邑震骇。高祖潜撤精卒三千配沈泰,渡江袭齐行台赵彦深于瓜步,获舟舰百余艘,陈粟万斛。即日天子总羽林禁兵,顿于长乐寺。六月甲辰,齐兵潜至钟山龙尾。丁未,进至莫府山。高祖遣钱明领水军出江乘,要击齐人粮运,尽获其船米,齐军于是大馁,杀马驴而食之。庚戌,齐军逾钟山,高祖众军分顿乐游苑东及覆舟山北,断其冲要。壬子,齐军至玄武湖西北莫府山南,将据北郊坛。众军自覆舟东移,顿郊坛北,与齐人相对。其夜大雨震电,暴风拔木,平地水丈余,齐军尽夜坐立泥中,悬鬲以爨,而台中及潮沟北水退路燥,官军每得番易。甲寅,少霁,高祖命众军秣马蓐食,迟明攻之。乙卯旦,自率帐内麾下出莫府山南,吴明彻、沈泰等众军首尾齐举,纵兵大战,侯安都自白下引兵横出其后,齐师大溃,斩获数千人,相蹂藉而死者不可胜计,生执徐嗣徽及其弟嗣宗,斩之以徇。追奔至于临沂。其江乘、摄山、钟山等诸军相次克捷,虏萧轨、东方老、王敬宝、李希光、裴英起等将帅凡四十六人。其军士得窜至江者,缚荻筏以济,中江而溺,流尸至京口,翳水弥岸。丁巳,众军出南州,烧贼舟舰。己未,斩刘归义、徐嗣彦、傅野猪于建康市。是日解严。庚申,萧轨、东方老、王敬宝、李希光、裴英起皆伏诛。高祖表解南徐州以授侯安都。七月丙子,诏授高祖中书监、司徒、扬州刺史,进爵为公,增邑并前五千户,侍中、使持节、都督中外诸军事、将军、尚书令、班剑、鼓吹、甲仗并如故,并给油幢皁轮车。是月侯瑱以江州入附。遣侯安都镇上流,定南中诸郡。

八月癸卯,太府卿何敱、新州刺史华志各上玉玺一。高祖表以送台,诏归之高祖。是日诏高祖食安吉、武康二县,合五千户。九月壬寅,改年曰太平元年。进高祖位丞相、录尚书事、镇卫大将军,改刺史为牧,进封义兴郡公,侍中、司徒、都督、班剑、鼓吹、甲仗、皁轮车并如故。丁未,中散大夫王彭笺称今月五日平旦于御路见龙迹,自大社至象阙,亘三四里。庚申,诏追赠高祖考侍中、光禄大夫,加金章紫绶,封义兴郡公,谥曰恭。十月甲戌,敕丞相自今入问讯,可施别榻以近扆坐。二年正月壬寅,天子朝万国于太极东堂,加高祖班剑十人,并前三十人,余如故。丁未,诏赠高祖兄道谭散骑常侍、使持节、平北将军、南兗州刺史、长城县公,谥曰昭烈;弟休先侍中、使持节、骠骑将军、南徐州刺史、武康县侯,谥曰忠壮,食邑各二千户。甲寅,遣兼侍中谒者仆射陆缮策拜长城县夫人章氏为义兴国夫人。丁卯,诏赠高祖祖侍中、太常卿,谥曰孝。追封高祖祖母许氏吴郡嘉兴县君,谥曰敬;妣张氏义兴国太夫人,谥曰宣。

二月庚午,萧勃举兵,自广州渡岭,顿南康,遣其将欧阳頠、傅泰及其子孜为前军,至于豫章,分屯要险,南江州刺史余孝顷起兵应勃,高祖命周文育、侯安都率众讨平之。

八月甲午,进高祖位太傅,加黄钺,剑履上殿,入朝不趋,赞拜不名,并给羽葆鼓吹一部,其侍中、都督、录尚书、镇卫大将军、扬州牧、义兴郡公、班剑、甲仗、油幢皁轮车并如故。丙申,加高祖前后部羽葆鼓吹。是时,湘州刺史王琳拥兵不应命,高祖遣周文育、侯安都率众讨之。

九月辛丑,诏曰:

肇昔元胎剖判,太素氤氲,崇建人皇,必凭洪宰。故贤哲之后,牧伯征于四方,神武之君,大监治乎万国。又有一匡九合,渠门之赐以隆,戮带围温,行宫之宠斯茂,时危所以贞固,运泰所以光熙,斯乃千载同风,百王不刊之道也。太傅义兴公,允文允武,乃圣乃神,固天生德,康济黔首。昔在休期,早隆朝寄,远逾沧海,大拯交、越。皇运不造,书契未闻,中国其亡,兵凶总至。哀哀噍类,譬彼穷牢,悠悠上天,莫云斯极。否终则泰,元辅应期,救此将崩,援兹已溺,乘舟履輂,架险浮深,经略中途,毕歼群丑。洎乎石头、姑孰,流髓履肠,一朝指捴,六合清晏。是用光昭下武,翼亮中都,雪三后之勍仇,夷三灵之巨慝。尧台禹佐,未始能阶,殷相周师,固非云拟。重之以屯剥余象,荆楚大崩,天地无心,乘舆委御,五胡荐食,竞谋诸夏,八方棋跱,莫有匡救,强臣放命,黜我冲人,顾影于荼孺之魂,甘心于宁卿之辱。却按下髻,求哀之路莫从,窃鈇逃责,容身之地无所。公神兵奄至,不日清澄,惟是孱蒙,再膺天录。斯又巍巍荡荡,无德而称焉。加以仗兹忠义,屠彼逆,震部夷氛,稽山罢昆,番禺、蠡泽,北鄙西郊,歼厥凶徒,罄无遗种。斯则兆民之命,修短所县,率土之基,兴亡是赖。于是刑礼兼训,沿革有章,中外成平,遐迩宁一,用能使阳光合魄,曜象呈晖,栖阁游庭,抱仁含信,宏勋该于厚地,大道格于玄天。羲、农、炎、昊以来,卷领垂衣之世,圣人济物,未有如斯者也。夫备物典策,桓、文是膺,助理阴阳,萧、曹不让,未有功高于宇县,而赏薄于伊、周,凡厥人祇,固怀延伫,是由公谦捴自牧,降损为怀,嘉数迟回,永言增叹。岂可申兹雅尚,久废朝猷,宜戒司勋,敬升鸿典。且重华大圣。妫汭惟贤,盛德之祀无忘,公侯之门必复。是以殷嘉亶甫,继后稷之官,尧命羲和,纂重黎之位。况其本枝攸建,宜誓山河者乎?其进公位相国,总百揆,封十郡为陈公,备九锡之礼,加玺绂,远游冠、绿綟绶,位在诸侯王上,其镇卫大将军、扬州牧如故。

策曰:

大哉乾元,资日月以贞观,至哉坤元,凭山川以载物。故惟天为大,陟配者钦明,惟王建国,翼辅者齐圣。是以文、武之佐,磻溪蕴其玉璜,尧、舜之臣,荣河镂其金版。况乎体得一之鸿姿,宁阳九之危厄,拯横流于碣石,扑燎火于昆岑,驱驭于韦、彭,跨弩于齐、晋,神功行而靡用,圣道运而无名者乎?今将授公典策,其敬听朕命:日者昊天不吊,钟乱于我国家,网漏吞舟,强胡内赑,茫茫宇宙,惵々黎元,方足圆颅,万不遗一,太清否亢,桥山之痛已深,大宝屯如,平阳之祸相继。上宰膺运,康救兆民,鞠旅于滇池之南,扬旌于桂岭之北,悬三光于已坠,谧四海于群飞,屠猰窳于中原,斩鲸鲵于蒙汜。荡宁上国,光启中兴。此则公之大造于皇家者也。既而天未悔祸,夷丑荐臻,南夏崩腾,西京荡覆,群胡孔炽,藉乱乘间,推纳籓枝,盗假神器,冢司昏摐,旁引寇雠,既见贬于桐宫,方谋危于汉阁。皇运已殆,何殊赘旒,中国摇然,非徒如线。公赫然投袂,匡救本朝,复莒齐都,平戎王室。朕所以还膺宝历,重履辰居,挹建武之风猷,歌宣王之雅颂。此又公之再造于皇家者也。公应务之初,登庸惟始,三川五岭,莫不窥临,银洞珠宫,所在宁谧。孙、卢肇衅,越貊为灾,番部阽危,势将沦殄。公赤旗所指,祅垒洞开,白羽才捴,凶徒粉溃。非其神武,久丧南籓。此又公之功也。大同之末,边政不修,李贲狂迷,窃我交、爱,敢称大号,骄恣甚于尉他,据有连州,雄豪炽于梁硕。公英谟雄算,电扫风行,驰御楼船,直跨沧海,新昌、典澈,备履艰难,苏历、嘉宁,尽为京观。三山獠洞,八角蛮陬,逖矣水寓之乡,悠哉火山之国,马援之所不届,陶璜之所未闻,莫不惧我王灵,争朝边候,归賝天府,献状鸿胪。此又公之功也。自寇虏陵江,宫闱幽辱,公枕戈尝胆,提剑拊心,气涌青霄,神飞紫闼。而番禺连率,本自诸夷,言得其朋,是怀同恶。公仗此忠诚,乘机剿定,执沛令而衅鼓,平新野而据鞍。此又公之功也。世道初艰,方隅多难,勋门桀黠,作乱衡嶷,兵切池隍,众兼夷獠。公以国盗边警,知无不为,恤是同盟,诛其丑类,莫不鱼惊鸟散,面缚头悬。南土黔黎,重保苏息。此又公之功也。长驱岭峤,梦想京畿,缘道酋豪,递为榛梗,路养渠率,全据大都,蓄聚逋逃,方谋阻乱,百楼不战,云梯之所未窥,万驽齐张,高輣之所非敌。公龙骧虎步,啸吒风云,山靡坚城,野无强阵,清氛于赣石,灭沴气于雩都。此又公之功也。迁仕凶慝,屯据大皋,乞活类马腾之军,流民多杜弢之众,推锋转斗,自北徂南,频岁稽诛,实惟勍虏。公坐挥三略,遥制六奇,义勇同心,貔貅骋力,雷奔电击,谷静山空,列郡无犬吠之惊,丛祠罢狐鸣之盗。此又公之功也。王师讨虏,次届沦波,兵乏兼储,士有饥色。公回麾蠡泽,积谷巴丘,亿庾之咏斯豊,壶浆之迎是众,军民转漕,曾无砥柱之难,舻舳相望,如运敖仓之府,犀渠贝胄,顾蔑雷霆,高舰层楼,仰扪霄汉,故使三军勇锐,百战无前,承此兵粮,遂殄凶逆。此又公之功也。若夫英图迈俗,义旅如云,湓垒猜携,用淹戎略。公志唯同奖,师克在和,鹄塞非虞,鸿门是会,若晋侯之誓白水,如萧王之推赤心,屈礼交盟,人祗感咽,故能使舟师并路,远迩朋心。此又公之功也。姑孰襟要,崤函阻凭,寇虏据其关梁,大盗负其扃鐍。公一校裁捴,三雄并奋,左贤、右角,沙溃土崩,木甲殪于中原,氈裘赴于江水,他他藉藉,万计千群,鄂坂之隘斯开,夷庚之道无塞。此又公之功也。义军大众,俱集帝京,逆竖凶徒,犹屯皇邑。若夫表里山河,金汤险固,疏龙首以抗殿,揃华岳以为城,杂虏凭焉,强兵自若。公回兹地轴,抗此天罗,曾不崇朝,俾无遗噍,军容甚穆,国政方修,物重睹于衣冠,民还瞻于礼乐,楚人满道,争睹于叶公,汉老衔悲,俱欢于司隶。此又公之功也。内难初静,诸侯出关,外郡传烽,鲜卑犯塞,莫非且渠、当户,中贵名王,冀马迾于淮南,胡笳动于徐北。公舟师步甲,亘野横江,歼厥群羝,遂殚封豨,莫不絓木而止,戎车靡遗,遇泞而旋,归骖尽殪。此又公之功也。公克黜祸难,劬劳皇室,而孙宁之党,翻启狄心,伊、洛之间,咸为虏戍,虽金陵佳气,石垒天严,朝暗戎尘,夜喧胡鼓。公三筹既画,八阵斯张,裁举灵鉟,亦抽金仆,咸俘丑类,悉反高墉,异李广之皆诛,同庞元之尽赦。此又公之功也。任约叛涣,枭声不悛,戎羯贪婪,狼心无改,穹庐氈幕,抵北阙而为营,乌孙天马,指东都而成阵。公左甄右落,箕张翼舒,扫是搀枪,驱其猃狁,长狄之种埋于国门,椎髻之酋烹于军市,投秦坑而尽沸,噎濉水而不流。此又公之功也。一相居中,自折彝鼎,五湖小守,妄怀同恶。公夙驾兼道,秉羽杖戈,玉斧将挥,金钲且戒,妖酋震慑,遽请灰钉,爇榇以表其含弘,焚书以安其反侧。此又公之功也。贼龛凶横,陵虐具区,阻兵安忍,凭灾怙乱,自古虫言鸟迹,浑沌洪荒,凡或虔刘,未此残酷。公虽宗居汝颍,世寓东南,育圣诞贤之乡,含章挺生之地,眷言桑梓,公私愤切,卓尔英状,丞规奉算,戮此大憝,如烹小鲜。此又公之功也。乱离永久,群盗孔多,浙左凶渠,连兵构逆,岂止千兵、五校、白雀、黄龙而已哉!公以中军无率,选是亲贤,奸寇途穷,涔然冰泮,刑溏之所,文命动其大威,雷门之间,句践行其严戮,英规圣迹,异代同风。此又公之功也。同姓有扈,顽凶不宾,凭藉宗盟,图危社稷,观兵汇泽,势震京师,驱率南蛮,已为东帝。公论兵于朝堂之上,决胜于樽俎之间,寇、贾、樊、滕,浮江下濑,一朝揃扑,无待甸师,万里澄清,非劳新息。此又公之功也。豫章妖寇,依凭山泽,缮甲完聚,多历岁时,结从连横,爰洎交、广。吕嘉既获,吴濞已鏦,命我还师,征其不恪,连营尽拔,伪党斯擒,曜圣武于匡山,回神旌于蠡派。此又公之功也。自八纮九野,瓜剖豆分,窃帝偷王,连州比县。公武灵已畅,文德又宣,折简驰书,风猷斯远,至于苍苍浴日,杳杳无雷,北洎丈夫之乡,南逾女子之国,莫不屈膝膜拜,求吏款关。此又公之功也。京师祸乱,亟积寒暄,双阙低昂,九门寥豁。宁秦宫之可顾,岂鲁殿之犹存!五都簪弁,百僚卿士,胡服缦缨,咸为戎俗,高冠厚履,希复华风,宋微子《麦遂》之歌,周大夫《黍离》之叹,方之于斯,未足为悲矣。公求衣昧旦,昃食高舂,兴构宫闱,具瞻遐迩,郊痒稷宗之典,六符十等之章,还闻太始之风流,重睹永平之遗事。此又公之功也。公有济天下之勋,重之以明德,凝神体道,合德符天,用百姓以为心,随万机而成务,耻一物非唐、虞之民,归含灵于仁寿之域,上德不德,无为以为,夏长春生,显仁藏用,忠信为宝,风雨弗愆,仁惠为基,牛羊勿践,功成治定,乐奏《咸》、《云》,安上治民,礼兼文质,物色丘园,衣裾里巷,朝多君子,野无遗贤,菽粟同水火之饶,工商富猗顿之旅。是以天无蕴宝,地有呈祥,潏露卿云,朝团晓映,山车泽马,服驭登闲,既景焕于图书,方葳蕤于史谍。高勋逾于象纬,积德冠于嵩、华,固无德而称者矣。朕又闻之,前王宰世,茂赏尊贤,式树籓长,总征群伯,《二南》崇绝,四履遐旷,泱泱表海,祚土维齐,岩岩泰山,俾侯于鲁;抑又勤王反郑,夹辅迁周,召伯之命斯隆,河阳之礼咸备;况复经营宇宙,宁唯断鰲足之功,弘济苍生,非直凿龙门之险;而畴庸报德,寂尔无闻,朕所以垂拱当宁,载怀惭悸者也。今授公相国,以南豫州之陈留、南丹阳、宣城,扬州之吴兴、东阳、新安、新宁,南徐州之义兴,江州之鄱阳、临川十郡,封公为陈公。锡兹青土,苴以白茅,爰定尔邦,用建冢社。昔旦、奭分陕,俱为保师,晋、郑诸侯,咸作卿士,兼其内外,礼实攸宜。今命使持节兼太尉王通授相国印绶、陈公玺绂。使持节兼司空王瑒授陈公茅土,金虎符第一至第五左,竹使符第一至第十左。相国秩逾三铉,任总百司,位绝朝班,礼由事革。其以相国总百揆,除录尚书之号,上所假节侍中貂蝉、中书监印章、中外都督太傅印绶、义兴公印策,其镇卫大将军、扬州牧如故。又加公九锡,其敬听后命:以公礼为桢干,律等衔策,四维皆举,八柄有章,是用锡公大辂、戎辂各一,玄牡二驷。以公贱宝崇谷,疏爵待农,室富京坻,民知荣辱,是用锡公衮冕之服,赤舄副焉。以公调理阴阳,燮谐风雅,三灵允降,万国同和,是用锡公轩县之乐,六佾之舞。以公宣导王猷,弘阐风教,光景所照,鞮象必通,是用锡公硃户以居。以公抑扬清浊,褒德进贤,髦士盈朝,幽人虚谷,是用锡公纳陛以登。以公嶷然廊庙,为世镕范,折冲四表,临御八荒,是用锡公武贲之士三百人。以公执兹明罚,期在刑措,象恭无赦,干纪必诛,是用锡公斧、钺各一。以公英猷远量,跨厉嵩溟,包一车书,括囊寰宇,是用锡公彤弓一、彤矢百、甗弓十、甗矢千。以公天经地义,贯彻幽明,春露秋霜,允恭粢盛,是用锡公秬鬯一卣,圭瓒副焉。陈国置丞相已下,一遵旧式。往钦哉!其恭循朕命,克相皇天,弘建邦家,允兴洪业,以光我高祖之休命!

十月戊辰,进高祖爵为王,以扬州之会稽、临海、永嘉、建安,南徐州之晋陵、信义,江州之寻阳、豫章、安成、庐陵并前为二十郡,益封陈国。其相国、扬州牧、镇卫大将军并如故。又命陈王冕十有二旒,建天子旌旗,出警入跸,乘金根车,驾六马,备五时副车,置旄头云罕,乐舞《八佾》,设钟虡宫县。王妃、王子、王女爵命之号,陈台百官,一依旧典。辛未,梁帝禅位于陈,诏曰:

五运更始,三正迭代,司牧黎庶,是属圣贤,用能经纬乾坤,弥纶区宇,大庇黔首,阐扬鸿烈。革晦以明,积代同轨,百王踵武,咸由此则。梁德湮微,祸乱荐发,太清云始,见困长蛇,承圣之季,又罹封豕。爰至天成,重窃神器,三光亟沈,七庙乏祀,含生已泯,鼎命斯坠,我武、元之祚,有如缀旒,静惟屯剥,夕惕载怀。相国陈王,有命自天,降神惟狱,天地合德,晷曜齐明,拯社稷之横流,提亿兆之涂炭,东诛逆叛,北歼獯丑,威加四海,仁渐万国,复张崩乐,重兴绝礼,儒馆聿修,戎亭虚候,大功在舜,盛绩惟禹,巍巍荡荡,无得而称。来献白环,岂直皇虞之世,入贡素雉,非止隆周之日。固以效珍川陆,表瑞烟云,甘露醴泉,旦夕凝涌,嘉禾硃草,孳植郊甸。道昭于悠代,勋格于皇穹,明明上天,光华日月,革故著于玄象,代德彰于图谶,狱讼有归,讴歌爰适,天之历数,实有攸在。朕虽庸貌,暗于古昔,永稽崇替,为日已久,敢忘列代之遗典,人祇之至愿乎。今便逊位别宫,敬禅于陈,一依唐、虞、宋、齐故事。

策曰:

咨尔陈王:惟昔上古,厥初生民,骊连、栗陆之前,容成、大庭之代,并结绳写鸟,杳冥慌忽,故靡得而详焉。自羲、农、轩、昊之君,陶唐、有虞之主,或垂衣而御四海,或无为而子万姓,居之如驭朽索,去之如脱敝屣。裁遇许由,便能舍帝,暂逢善卷,即以让王。故知玄扈璇玑,非关尊贵,金根玉辂,示表君临。及南观河渚,东沈刻璧,精华既竭,耄勤已倦,则抗首而笑,唯贤是与,讠劳然作歌,简能斯授,遗风余烈,昭晰图书。汉、魏因循,是为故实。宋、齐授受,又弘斯义。我高祖应期抚运,握枢御宇,三后重光,祖宗齐圣。及时属阳九,封豕荐食,西都失驭,夷狄交侵,乃皋天成,轻弄龟鼎,喋喋黔首,若崩厥角,徽徽皇极,将甚缀旒。惟王乃圣乃神,钦明文思,二仪并运,四时合序,天锡智勇,人挺雄杰,珠庭日角,龙行武步,爰初投袂,日乃勤王,电扫番禺,云撤彭蠡,揃其元恶,定我京畿。及王贺帝弘,贸兹冠屦,既行伊、霍,用保冲人。震泽、稽阴,并怀叛逆,獯羯丑虏,三乱皇都,裁命偏师,二邦自殄,薄伐猃狁,六戎尽殪。岭南叛涣,湘、郢结连,贼帅既擒,凶渠传首,用能百揆时序,四门允穆,无思不服,无远不届,上达穹昊,下漏渊泉,蛟鱼并见,讴歌攸属。况乎长彗横天,已徵布新之兆,璧日斯既,实表更姓之符。是以始创义师,紫云曜彩,肇惟尊主,黄龙负舟。苦矢素翚,梯山以至,白环玉玦,慕德而臻。若夫安国字萌,本因万物之志,时乘御宇,良会乐推之心。七百无常期,皇王非一族,昔木德既季,而传祚于我有梁,天之历数,允集明哲。式遵前典,广询群议,王公卿尹,莫不攸属,敬从人祇之愿,授帝位于尔躬。四海困穷,天禄永终,王其允执厥中,轨仪前式,以副溥天之望。禋祀上帝,时膺大礼,永固洪业,岂不盛欤!

又玺书曰:

君子者自昭明德,达人者先天弗违,故能进退咸亨,动静元吉。朕虽蒙寡,庶乎景行。何则?三才剖判,九有区分,情性相乖,乱离云起,是以建彼司牧,推乎圣贤,授受者任其时来,皇王者本非一族,人谋是与,屈己从万物之心,天意斯归,鞠躬奉百灵之命。讴歌所往,则攘袂以膺之,菁华已竭,乃褰裳而去之。昔在唐、虞,鉴于天道,举其黎献,授彼明哲,虽复质文殊轨,沿革不同,历代因循,斯风靡替。我大梁所以考庸太室,接礼贰宫,月正元日,受终文祖。但运不常夷,道无恒泰,山岳倾偃,河海沸腾,电目雷声之禽,钩爪锯牙之兽,咀啮含生,不知纪极。二后英圣,相仍在天,六夷贪狡,争侵中国,县王都帝,人怀干纪,一民尺土,皆非梁地。朕以不造,幼罹闵凶,仰凭衡佐,亟移年序。周成、汉惠,邈矣无阶,惟是童蒙,必贻颠蹶。若使时无圣哲,世靡艰难,犹当高蹈于沧洲,自求于泰伯者矣。惟王应期诞秀,开箓握图,性道故其难闻,嘉庸已其被物,乾行同其焘覆,日御比其贞明,登承圣于复禹之功,树鞠子于兴周之业,灭陆浑于伊、洛,歼骊戎于镐京,大小二震之骁徒,东南两越之勍寇,遽行天讨,无遗神策。于是祖述尧舜,宪章文武,大乐与天地同和,大礼与天地同节,鼓之以雷霆,润之以风雨,仁沾葭苇,信及豚鱼,殷牖斯空,夏台虚设,民惟大畜,野有同人,升平颂平,无偏无党,固以云飞紫盖,水跃黄龙,东伐西征,晻映川陆。荣光暧暧,已冒郊廛,甘露瀼瀼,亟流庭苑。车辙马迹,谁不率从?蟠水流沙,谁不怀德?祥图远至,非唯赤伏之符,灵命昭然,何止黄星之气。海口河目,贤圣之表既彰,握旄执钺,君人之状斯伟。且自摄提无纪,孟陬殄灭,枉矢宵飞,天弧晓映,久矣夷羊之在牧,时哉蛟龙之出泉。革运之兆咸徵,惟新之符并集,朕所以钦若勋、华,屡回星琯。昔者水运斯尽,予高祖受焉。今历去炎精,神归枢纽,敬以火德,传于尔陈。远鉴前王,近谋群辟,明灵有悦,率土同心。今遣使持节兼太保侍中尚书左仆射平乐亭侯王通,兼太尉司徒左长史王锡奉皇帝玺绶。受终之礼,一依唐、虞故事。王其时陟元后,宁育兆民,光阐洪猷,以承昊天之休命!

是日,梁帝逊于别宫。高祖谦让再三,群臣固请,乃许。

\hypertarget{header-n4194}{%
\subsubsection{卷二}\label{header-n4194}}

高祖下

永定元年冬十月乙亥,高祖即皇帝位于南郊,柴燎告天曰:``皇帝臣霸先,敢用玄牡昭告于皇皇后帝:梁氏以圮剥荐臻,历运有极,钦若天应,以命于霸先。夫肇有烝民,乃树司牧,选贤与能,未常厥姓。放勋、重华之世,咸无意于受终,当涂、典午之君,虽有心于揖让,皆以英才处万乘,高勋御四海,故能大庇黔首,光宅区县。有梁末运,仍叶遘屯,獯丑凭陵,久移神器,承圣在外,非能祀夏,天未悔祸,复罹寇逆,嫡嗣废黜,宗枝僭诈,天地荡覆,纪纲泯绝。霸先爰初投袂,大拯横流,重举义兵,实戡多难,废王立帝,实有厥功,安国定社,用尽其力。是谓小康,方期大道。既而烟云表色,日月呈瑞,纬聚东井,龙见谯邦,除旧布新,即彰玄象,迁虞事夏,且协讴歌,九域八荒,同布衷款,百神群祀,皆有诚愿。梁帝高谢万邦,授以大宝,霸先自惟菲薄,让德不嗣,至于再三,辞弗获许。佥以百姓须主,万机难旷,皇灵眷命,非可谦拒。畏天之威,用膺嘉祚,永言夙志,能无惭德。敬简元辰,升坛受禅,告类上帝,用答民心,永保于我有陈。惟明灵是飨!''先是氛雾,昼夜晦冥,至于是日,景气清晏,识者知有天道焉。礼毕,舆驾还宫,临太极前殿。诏曰:``五德更运,帝王所以御天,三正相因,夏、殷所以宰世,虽色分辞翰,时异文质,揖让征伐,迄用参差,而育德振民,义归一揆。朕以寡昧,时属艰危,国步屡屯,天维三绝,肆勤先后,拯厥横流,藉将帅之功,兼猛士之力,一匡天下,再造黔黎。梁氏以天禄永终,历数攸在,遵与能之典,集大命于朕躬。顾惟菲德,辞不获亮,式从天眷,俯协民心,受终文祖,升禋上帝,继迹百王,君临万宇,若涉川水,罔知攸济。宝业初建,皇祚惟新,思俾惠泽,覃被亿兆。可大赦天下,改梁太平二年为永定元年。赐民爵二级,文武二等。鳏寡孤独不能自存者人谷五斛。逋租宿债,皆勿复收。其有犯乡里清议赃污淫盗者,皆洗除先注,与之更始。长徒敕系,特皆原之。亡官失爵,禁锢夺劳,一依旧典。''又诏曰:``《礼》陈杞、宋,《诗》咏二客,弗臣之重,历代斯敦。梁氏钦若人祇,宪章在昔,济河沈璧,高谢万邦,茅赋所加,宜遵旧典。其以江阴郡奉梁主为江阴王,行梁正朔,车旗服色,一依前准,宫馆资待,务尽优隆。''又诏梁皇太后为江阴国太妃,皇后为江阴国妃。又诏百司依位摄职。丙子,舆驾幸钟山祠帝庙。戊寅,舆驾幸华林园,亲览词讼,临赦囚徒。己卯,分遣大使宣劳四方,下玺书敕州郡曰:``夫四王革代,商、周所以应天,五胜相推,轩、羲所以当运。梁德不造,丧乱积年,东夏崩腾,西都荡覆。萧勃干纪,非唯赵伦,侯景滔天,逾于刘载。贞阳反篡,贼约连兵,江左累属于鲜卑,金陵久非于梁国。自有氤氲混沌之世,龙图凤纪之前,东汉兴平之,西朝永嘉之乱,天下分崩,未有若于梁朝者也。朕以虚薄,属当兴运,自昔登庸,首清诸越,徐门浪泊,靡不征行,浮海乘山,所在戡定。冒朔风尘,骋驰师旅,六延梁祀,十翦强寇,岂曰人谋,皆由天启。梁氏以天禄斯改,期运永终,钦若唐、虞,推其鼎玉,朕东西退让,拜手陈辞,避舜子于箕山之阳,求支伯于沧洲之野,而公卿敦逼,率土翘惶,天命难稽,遂享嘉祚。今月乙亥,升礼太坛,言念迁桐,但有惭德。自梁氏将末,频月亢阳,火运斯终,秋霖奄降。翌日成礼,圆丘宿设,埃云晚霁,星象夜张。朝景重轮,泫三危之膏露,晨光合璧,带五色之卿云。顾惟寡薄,弥惭休祉,昧旦丕显,方思至治。卿等拥旄方岳,相任股肱,剖符名守,方寄恤隐。王历惟新,念有欣庆,想深求民瘼,务在廉平,爱惠以抚孤贫,威刑以御强猾。若有萑蒲之盗,或犯戎商,山谷之酋,擅强幽险,皆从肆赦,咸使知闻。如或迷途,俾在无贷。今遣使人具宣往旨,念思善政,副此虚怀。''庚辰,诏出佛牙于杜姥宅,集四部设无遮大会,高祖亲出阙前礼拜。初,齐故僧统法献于乌缠国得之,常在定林上寺,梁天监末,为摄山庆云寺沙门慧兴保藏,慧兴将终,以属弟慧志,承圣末,慧志密送于高祖,至是乃出。辛巳,追尊皇考曰景皇帝,庙号太祖;皇妣董太夫人曰安皇后。追谥前夫人钱氏号为昭皇后,世子克为孝怀太子。立夫人章氏为皇后。癸未,尊景帝陵曰瑞陵,昭皇后陵曰嘉陵,依梁初园陵故事。立删定郎,治定律令。戊子,迁景皇帝神主祔于太庙。辛卯,以中权将军、开府仪同三司、丹阳尹王冲为左光禄大夫。癸巳,追赠皇兄梁故散骑常侍、平北将军、兗州刺史长城县公道谭骠骑大将军、太尉,封始兴郡王;弟梁故侍中、骠骑将军、南徐州刺史武康县侯休先车骑大将军、司徒,封南康郡王。是月,西讨都督周文育、侯安都于郢州败绩,囚于王琳。十一月丙申,诏曰:``东都齐国,义乃亲贤,西汉城阳,事兼功烈。散骑常侍、使持节、都督会稽等十郡诸军事、宣毅将军、会稽太守长城县侯蒨,学尚清优,神宇凝正,文参礼乐,武定妖氛,心力谋猷,为家治国,拥旄作守,期月有成,辟彼关河,功逾萧、寇,萑蒲之盗,自反耕农,篁竹之豪,用禀声朔。朕以虚寡,属当兴运,提彼三尺,宾于四门,王业艰难,赖乎此子,宜隆上爵,称是元功。可封临川郡王,邑二千户。兄子梁中书侍郎顼袭封始兴王,弟子梁中书侍郎昙朗袭封南康王,礼秩一同正王。''己亥,甘露降于钟山松林,弥满岩谷。庚子,开善寺沙门采之以献,敕颁赐群臣。丙辰,以镇西将军、南豫州刺史徐度为镇右将军、领军将军。庚申,京师大火。十二月庚辰,皇后谒太庙。

二年春正月乙未,诏曰:``夫设官分职,因事重轻,羽仪车马,随时隆替,晋之五校,鸣笳启途,汉之九卿,传呼并迾,虞官夏礼,岂曰同科,殷朴周文,固无恒格。朕膺兹宝历,代是天工,留念官方,庶允时衷。梁天监中,左右骁骑领硃衣直阁,并给仪从,北徐州刺史昌义之初,首为此职。乱离岁久,朝典不存,后生年少,希闻旧则。今去左右骁骑,宜通文武,文官则用腹心,武官则用功臣,所给仪从,同太子二卫率。此外众官,尚书详为条制。''车骑将军、开府仪同三司侯瑱进位司空,中权将军、开府仪同三司、新除左光禄大夫王冲为太子少傅。左卫将军徐世谱为护军将军,南兗州刺史吴明彻进号安南将军,衡州刺史欧阳頠进号镇南将军。辛丑,舆驾亲祠南郊。诏曰:``朕受命君临,初移星琯,孟陬嘉月,备礼泰坛,景候昭华,人祗允庆,思令亿兆,咸与惟新。且往代祅氛,于今犹梗,军机未息,征赋咸繁,事不获已,久知下弊,言念黔黎,无忘寝食。夫罪无轻重,已发觉未发觉,在今昧爽以前,皆赦除之。西寇自王琳以下,并许返迷,一无所问。近所募义军,本拟西寇,并宜解遣,留家附业。晚订军资未送者并停,元年军粮逋馀者原其半。州郡县军戍并不得辄遣使民间,务存优养。若有侵扰,严为法制。''乙巳,舆驾亲祠北郊。甲辰,振远将军、梁州刺史张立表称去乙亥岁八月,丹徒、兰陵二县界遗山侧,一旦因涛水涌生,沙涨,周旋千馀顷,并膏腴,堪垦植。戊午,舆驾亲祠明堂。二月壬申,南豫州刺史沈泰奔于齐。辛卯,诏车骑将军、司空侯瑱总督水步众军以遏齐寇。三月甲午,诏曰:``罚不及嗣,自古通典,罪疑惟轻,布在方策。沈泰反覆无行,遐迩所知。昔有微功,仍荷朝寄,剖符名郡,推毂累籓,汉口班师,还居方岳,良田有逾于四百,食客不止于三千,富贵显荣,政当如此。鬼害其盈,天夺之魄,无故猖狂,自投獯丑。虽复知人则哲,惟帝其难,光武有蔽于庞萌,魏武不知于于禁,但令朝廷,无我负人。其部曲妻儿,各令复业,所在及军人若有恐胁侵掠者,皆以劫论。若有男女口为人所藏,并许诣台申诉。若乐随临川王及节将立效者,悉皆听许。''乙卯,高祖幸后堂听讼,还于桥上观山水,赋诗示群臣。是月,王琳立梁永嘉王萧庄于郢州。夏四月甲子,舆驾亲祠太庙。乙丑,江絮王薨,诏遣太宰吊祭,司空监护丧事,凶礼所须,随由备办。以梁武林侯萧谘息季卿嗣为江阴王。丙寅,舆驾幸石头,饯司空侯瑱。戊辰,重云殿东鸱尾有紫烟属天。五月乙未,京师地震。癸丑,齐广陵南城主张显和、长史张僧那各率其所部入附。辛酉,舆驾幸大庄岩寺舍身。壬戌,群臣表请还宫。六月己巳,诏司空侯瑱、领军将军徐度率舟师为前军,以讨王琳。秋七月戊戌,舆驾幸石头,亲送瑱等。己亥,江州刺史周迪擒王琳将李孝钦、樊猛、余孝顷于工塘。甲辰,遣吏部尚书谢哲谕王琳。甲寅,嘉禾一穗六岐生五城。初,侯景之平也,火焚太极殿,承圣中议欲营之,独阙一柱,至是有樟木大十八围,长四丈五尺,流泊陶家后渚,监军邹子度以闻。诏中书令沈众兼起部尚书,少府卿蔡俦兼将作大匠,起太极殿。八月丙寅,以广梁郡为陈留郡。辛未,诏临川王蒨西讨,以舟师五万发自京师,舆驾幸冶城寺亲送焉。前开府仪同三司、南豫州刺史周文育,前镇北将军、南徐州刺史、新除开府仪同三司侯安都等于王琳所逃归,自劾廷尉,即日引见,并宥之。戊寅,诏复文育等本官。壬午,追封皇子立为豫章王,谥曰献;权为长沙王,谥曰思;长女为永世公主,谥曰懿。谢哲反命,王琳请还镇湘川,诏追众军缓其伐。癸未,西讨众军至自大雷。丁亥,以信威将军、江州刺史周迪为开府仪同三司,进号平南将军。改南徐州所领南兰陵郡复为东海郡。冬十月庚午,遣镇南将军、开府仪同三司周文育都督众军出豫章,讨余孝劢。乙亥,舆驾幸庄严寺,发《金光明经》题。丁酉,以仁威将军、高州刺史黄法抃为开府仪同三司,进号镇南将军。甲寅,太极殿成,匠各给复。十二月庚申,侍中、安东将军临川王蒨率百僚朝前殿,拜上牛酒。甲子,舆驾幸大庄严寺,设无珝大会,舍乘舆法物。群臣备法驾奉迎,即日舆驾还宫。丙寅,高祖于太极殿东堂宴群臣,设金石之乐,以路寝告成也。壬申,割吴郡盐官、海盐、前京三县置海宁郡,属扬州。以安成所部广兴六洞置安乐郡。丙戌,以宁远将军、北江州刺史熊昙朗为开府仪同三司,进号平西将军。丁亥,诏曰:``梁时旧仕,乱离播越,始还朝廷,多未铨序。又起兵已来,军勋甚众。选曹即条文武簿及节将应九流者,量其所拟。''于是随材擢用者五十馀人。

三年春正月己丑,青龙见于东方。丁酉,以镇南将军、广州刺史欧阳頠即本号开府仪同三司。是夜大雪,及旦,太极殿前有龙迹见。甲午,广州刺史欧阳頠表称白龙见于州江南岸,长数十丈,大可八九围,历州城西道入天井岗。仙人见于罗浮山寺小石楼,长三丈所,通身洁白,衣服楚丽。辛丑,诏曰:``南康、始兴王诸妹,已有封爵,依礼止是籓主。此二王者,有殊恒情,宜隆礼数。诸主仪秩及尚主,可并同皇女。''戊申,诏临川王蒨省扬、徐二州辞讼。二月辛酉,以平西将军、桂州刺史淳于量为开府仪同三司,进号镇西大将军。壬午,司空侯瑱督众军自江入合州,焚齐舟舰。三月丙申,侯瑱至自合肥,众军献捷。夏闰四月庚寅,诏曰:``开廪赈绝,育民之大惠,巡方恤患,前王之令典。朕当斯季俗,膺此乐推,君德未孚,民瘼犹甚,重兹多垒,弥疚纳隍。良由四聪弗达,千里勿应。博施之仁,何其或爽?残弊之轨,致此未康。吴州、缙州,去岁蝗旱,郢田虽疏,郑渠终涸,室靡盈积之望,家有填壑之嗟。百姓不足,兆民何赖?近已遣中书舍人江德藻衔命东阳,与令长二千石问民疾苦,仍以入台仓见米分恤。虽德非既饱,庶微慰阻饥。''甲午,诏依前代置西省学士,兼以伎术者预焉。丁酉,遣镇北将军徐度率众城南皖口。是时久不雨,丙午,舆驾幸钟山祠蒋帝庙,是日降雨,迄于月晦。五月丙辰朔,日有食之,有司奏:旧仪,御前殿,服硃纱袍、通天冠。诏曰:``此乃前代承用,意有未同。合朔仰助太阳,宜备衮冕之服。自今已去,永可为准。''丙寅,扶南国遣使献方物。乙酉,北江州刺史熊昙朗杀都督周文育于军,举兵反。王琳遣其将常众爱、曹庆率兵援余孝劢。六月戊子,仪同侯安都败众爱等于左里,获琳从弟袭、主帅羊暕等三十馀人,众爱遁走,庚寅,庐山民斩之,传首京师。甲午,众师凯归。诏曰:``昙朗噬逆,罪不容诛,分命众军,仍事掩讨,方加枭磔,以明刑宪。''徵临川王装往皖口置城栅,以钱道戢守焉。丁酉,高祖不豫,遣兼太宰、尚书左仆射王通以疾告太庙,兼太宰、中书令谢哲告大社、南北郊。辛丑,高祖疾小瘳。故司空周文育之柩至自建昌。壬寅,高祖素服哭于东堂,哀甚。癸卯,高祖临讯狱讼。是夜,荧惑在天尊。高祖疾甚。丙午,崩于璿玑殿,时年五十七。遗诏追临川王蒨入纂。甲寅,大行皇帝迁殡于太极殿西阶。秋八月甲午,群臣上谥曰武皇帝,庙号高祖。丙申,葬万安陵。

高祖智以绥物,武以宁乱,英谋独运,人皆莫及,故能征伐四克,静难夷凶。至升大麓之日,居阿衡之任,恒崇宽政,爱育为本。有须发调军储,皆出于事不可息。加以俭素自率,常膳不过数品,私飨曲宴,皆瓦器蚌盘,肴核庶羞,裁令充足而已,不为虚费。初平侯景,及立绍泰,子女玉帛,皆班将士。其充闱房者,衣不重彩,饰无金翠,哥钟女乐,不列于前。及乎践祚,弥厉恭俭。故隆功茂德,光有天下焉。

陈吏部尚书姚察曰:高祖英略大度,应变无方,盖汉高、魏武之亚矣。及西都荡覆,诚贯天人。王僧辩阙伊尹之才,空结桐宫之愤,贞阳假秦兵之送,不思穆嬴之泣。高祖乃蹈玄机而抚末运,乘势隙而拯横流,王迹所基,始自于此,何至戡黎升陑之捷而已焉。故于慎徽时序之世,变声改物之辰,兆庶归以讴歌,炎灵去如释负,方之前代,何其美乎!

\hypertarget{header-n4203}{%
\subsubsection{卷三}\label{header-n4203}}

世祖

世祖文皇帝,讳蒨,字子华,始兴昭烈王长子也。少沈敏有识量,美容仪,留意经史,举动方雅,造次必遵礼法。高祖甚爱之,常称``此儿吾宗之英秀也''。梁太清初,梦两日斗,一大一小,大者光灭坠地,色正黄,其大如斗,世祖因三分取一而怀之。侯景之乱,乡人多依山湖寇抄,世祖独保家无所犯。时乱日甚,乃避地临安。及高祖举义兵,侯景遣使收世祖及衡阳献王,世祖乃密袖小刀,冀因入见而害景。至便属吏,故其事不行。高祖大军围石头,景欲加害者数矣。会景败,世祖乃得出赴高祖营。起家为吴兴太守。时宣城劫帅纪机、郝仲等各聚众千馀人,侵暴郡境,世祖讨平之。承圣二年,授信武将军,监南徐州。三年,高祖北征广陵,使世祖为前军,每战克捷。高祖之将讨王僧辩也,先召世祖与谋。时僧辩女婿杜龛据吴兴,兵众甚盛,高祖密令世祖还长城,立栅以备龛。世祖收兵才数百人,战备又少,龛遣其将杜泰领精兵五千,乘虚奄至。将士相视失色,而世祖言笑自若,部分益明,于是众心乃定。泰知栅内人少,日夜苦攻。世祖激厉将士,身当矢石,相持数旬,泰乃退走。及高祖遣周文育率兵讨龛,世祖与并军往吴兴。时龛兵尚众,断据冲要,水步连阵相结,世祖命将军刘澄、蒋元举率众攻龛,龛军大败,窘急,因请降。东扬州刺史张彪起兵围临海太守王怀振,怀振遣使求救,世祖与周文育轻兵往会稽以掩彪。后彪将沈泰开门纳世祖,世祖尽收其部曲家累,彪至,又破走,若邪村民斩彪,传其首。以功授持节、都督会稽等十郡诸军事、宣毅将军、会稽太守。山越深险,皆不宾附,世祖分命讨击,悉平之,威惠大振。高祖受禅,立为临川郡王,邑二千户,拜侍中、安东将军。及周文育、侯安都败于沌口,高祖诏世祖入卫,军储戎备,皆以委焉。寻命率兵城南皖。

永定三年六月丙午,高祖崩,遗诏征世祖入纂。甲寅,至自南皖,入居中书省。皇后令曰:``昊天不吊,上玄降祸。大行皇帝奄捐万国,率土哀号,普天如丧,穷酷烦冤,无所迨及。诸孤藐尔,反国无期,须立长主,以宁宇县。侍中、安东将军、临川王蒨,体自景皇,属惟犹子。建殊功于牧野,敷盛业于戡黎,纳麓时叙之辰,负扆乘机之日,并佐时雍,是同草创,祧祏所系,遐迩宅心,宜奉大宗,嗣膺宝录,使七庙有奉,兆民宁晏。未亡人假延馀息,婴此百罹,寻绎缠绵,兴言感绝。''世祖固让,至于再三,群公卿士固请,其日即皇帝位于太极前殿。诏曰:``上天降祸,奄集邦家,大行皇帝背离万国,率土崩心,若丧考妣。龙图宝历,眇属朕躬,运钟扰攘,事切机务,南面须主,西让礼轻,令便式膺景命,光宅四海。可大赦天下,罪无轻重,悉皆荡涤。逋租宿债,吏民愆负,可勿复收。文武内外,量加爵叙。孝悌力田为父后者,赐爵一级。庶祗畏在心,公卿毕力,胜残去杀,无待百年。兴言号哽,深增恸绝。''又诏州郡悉停奔赴。秋七月丙辰,尊皇后为皇太后,己未,以镇南将军、开府仪同三司、广州刺史欧阳頠进号征南将军,平南将军、开府仪同三司周迪进号镇南将军,平南将军、开府仪同三司、高州刺史黄法抃进号安南将军。庚申,以镇南大将军、开府仪同三司、桂州刺史淳于量进号征南大将军。辛酉,以侍中、车骑将军、司空侯瑱为太尉,镇西将军、开府仪同三司、南豫州刺史侯安都为司空,侍中、中权将军、开府仪同三司王冲为特进、左光禄大夫,镇北将军、南徐州刺史徐度为侍中、中抚军将军、开府仪同三司。壬戌,以侍中、护军将军徐世谱为特进、安右将军;侍中、忠武将军杜棱为领军将军。乙丑,重云殿灾。八月癸巳,以平北将军、南徐州刺史留异为安南将军、缙州刺史,平南将军、北江州刺史鲁悉达进号安左将军。庚戌,封皇子伯茂为始兴王,奉昭烈王后。徙封始兴嗣王顼为安成王。九月辛酉,立皇子伯宗为皇太子,王公以下赐帛各有差。乙亥,立妃沈氏为皇后。冬十一月乙卯,王琳寇大雷,诏遣太尉侯瑱、司空侯安都、仪同徐度率众以御之。

天嘉元年春正月癸丑,诏曰:``朕以寡昧,嗣纂洪业,哀茕在疚,治道弗昭,仰惟前德,幽显遐畅,恭己不言,庶几无改。虽宏图懋轨,日月方弘,而清庙廓然,圣灵浸远,感寻永往,瞻言罔极。今四象运周,三元告献,华夷胥洎,玉帛骏奔,思覃遗泽,播之亿兆。其大赦天下。改永定四年为天嘉元年。鳏寡孤独不能自存立者,赐谷人五斛。孝悌力田殊行异等,加爵一级。''甲寅,分遣使者宣劳四方。辛酉,舆驾亲祠南郊,诏曰:``朕式飨上玄,虔奉牲玉,高禋礼毕,诚敬兼弘。且阴霾浃辰,褰霁在日,云物韶朗,风景清和,庆动人祇,忭流庶俗,思俾黎元,同此多祐。可赐民爵一级。''辛未,舆驾亲祠北郊。日有冠。二月辛卯,老人星见。乙未,高州刺史纪机自军叛还宣城,据郡以应王琳,泾令贺当迁讨平之。丙申,太尉侯瑱败王琳于梁山,攻齐兵于博望,生擒齐将刘伯球,尽收其资储船舰,俘馘以万计,王琳及其主萧庄奔于齐。戊戌,诏曰:``夫五运递来,三灵眷命,皇王因之改创,殷、周所以乐推。朕统历承基,丕隆鼎运,期理攸属,数祚斯在,岂侥幸所至,宁卜祝可求。故知神器之重,必在符命。是以逐鹿贻讥,断蛇定业,乱臣贼子,异世同尤。王琳识暗挈瓶,智惭卫足,干纪乱常,自贻颠沛,而缙绅君子,多被絷维,虽泾渭合流,兰鲍同肆,求之厥理,或有胁从。今九罭既设,八纮斯掩,天网恢恢,吞舟是漏。至如伏波游说,永作汉蕃,延寿脱归,终为魏守,器改秦、虞,材通晋、楚,行藏用舍,亦岂有恒,宜加宽仁,以彰雷作。其衣冠士族,预在凶党,悉皆原宥;将帅战兵,亦同肆眚,并随才铨引,庶收力用。''又诏师旅以来,将士死王事者,并加赠谥。己亥,诏曰:``日者凶渠肆虐,众军进讨,舟舰输积,权倩民丁,师出经时,役劳日久。今气昆廓清,宜有甄被。可蠲复丁身。夫妻三年,于役不幸者,复其妻子。''庚子,分遣使者赍玺书宣劳四方。乙巳,遣太尉侯瑱镇湓城。庚戌,以高祖第六子昌为骠骑将军、湘州牧,立为衡阳王。三月丙辰,诏曰:``自丧乱以来,十有馀载,编户凋亡,万不遗一,中原氓庶,盖云无几。顷者寇难仍接,算敛繁多,且兴师已来,千金日费,府藏虚竭,杼轴岁空。近所置军资,本充戎备,今元恶克殄,八表已康,兵戈静戢,息肩方在,思俾馀黎,陶此宽赋,今岁军粮通减三分之一。尚书申下四方,称朕哀矜之意。守宰明加劝课,务急农桑,庶鼓腹含哺,复在兹日。''萧庄所署郢州刺史孙瑒举州内附。丁巳,江州刺史周迪平南中,斩贼率熊昙朗,传首京师。先是,齐军守鲁山城,戊午,齐军弃城走,诏南豫州刺史程灵洗守之。甲子,分荆州之天门、义阳、南平,郢州之武陵四郡,置武州。其刺史督沅州,领武陵太守,治武陵郡。其都尉所部六县为沅州。别置通宁郡,以刺史领太守,治都尉城,省旧都尉。以安南将军、南兗州刺史、新除右卫将军吴明彻为安西将军、武州刺史,伪郢州刺史孙瑒为安南将军、湘州刺史。丙子,衡阳王昌薨。丁丑,诏曰:``萧庄伪署文武官属还朝者,量加录序。''夏四月丁亥,立皇子伯信为衡阳王,奉献王后。乙未,以安南将军荀朗为安北将军、合州刺史。五月乙卯,改桂阳之汝城县为庐阳郡。分衡州之始兴、安远二郡,置东衡州。六月辛巳,改谥皇祖妣景安皇后曰景文皇后。壬辰,诏曰:``梁孝元遭离多难,灵榇播越,朕昔经北面,有异常伦,遣使迎接,以次近路。江宁既有旧茔,宜即安卜,车旗礼章,悉用梁典,依魏葬汉献帝故事。''甲午,追策故始兴昭烈王妃曰孝妃。丁酉,以开府仪同三司徐度为侍中、中军将军。辛丑,国哀周忌,上临于太极前殿,百僚陪哭。赦京师殊死已下。是月,葬梁元帝于江宁。秋七月甲寅,诏曰:``朕以眇身,属当大宝,负荷至重,忧责实深,而庶绩未康,胥怨犹结,伫咨贤良,发于梦想,每有一言入听,片善可求,何尝不褒奖抽扬,缄书绅带。而傅岩虚往,穹谷尚淹,蒲币空陈,旌弓不至。岂当有乖则哲,使草泽遗才?将时运浇流,今不逮古?侧食长怀,寝兴增叹。新安太守陆山才有启,荐梁前征西从事中郎萧策,梁前尚书中兵郎王暹,并世胄清华,羽仪著族,或文史足用,或孝德可称,并宜登之朝序,擢以不次。王公已下,其各进举贤良,申荐沦屈,庶众才必萃,大厦可成,使《棫朴》载歌,《由庚》在咏。''乙卯,诏曰:``自顷丧乱,编户播迁,言念馀黎,良可哀惕。其亡乡失土,逐食流移者,今年内随其适乐,来岁不问侨旧,悉令著籍,同土断之例。''丙辰,立皇子伯山为鄱阳王。八月庚辰,老人星见。壬午,诏曰:``菽粟之贵,重于珠玉。自顷寇戎,游手者众,民失分地之业,士有佩犊之讥。朕哀矜黔庶,念康弊俗,思俾阻饥,方存富教。麦之为用,要切斯甚,今九秋在节,万实可收,其班宣远近,并令播种。守宰亲临劝课,务使及时。其有尤贫,量给种子。''癸未,世祖临景阳殿听讼。戊子,诏曰:``汙罇土鼓,诚则难追,画卵雕薪,或可易革。梁氏末运,奢丽已甚,刍豢厌于胥史,歌钟列于管库,土木被硃丹之采,车马饰金玉之珍,逐欲浇流,迁讹遂远。朕自诸生,颇为内足,而家敦朴素,室靡浮华,观览时俗,常所扼腕。今妄假时乘,临驭区极,属当沦季,思闻治道,菲食卑宫,自安俭陋,俾兹薄俗,获反淳风。维雕镂淫饰,非兵器及国容所须,金银珠玉,衣服杂玩,悉皆禁断。''甲午,周将贺若敦率马步一万,奄至武陵,武州刺史吴明彻不能拒,引军还巴陵。丁酉,上幸正阳堂阅武。九月癸丑,彗星见。乙卯,周将独孤盛领水军将趣巴、湘,与贺若敦水陆俱进,太尉侯瑱自寻阳往御之。辛酉,遣仪同徐度率众会瑱于巴丘。丙子,太白昼见。丁丑,诏侯瑱众军进讨巴、湘。十月癸巳,侯瑱袭破独孤盛于杨叶洲,尽获其船舰,盛收兵登岸,筑城以保之。丁酉,诏司空侯安都率众会侯瑱南讨。十二月乙未,诏曰:``古者春夏二气,不决重罪。盖以阳和布泽,天秩是弘,宽网省刑,义符含育,前王所以则天象地,立法垂训者也。朕属当浇季,思求民瘼,哀矜恻隐,念甚纳隍,常欲式遵旧轨,用长风化。自今孟春讫于夏首,罪人大辟事已款者,宜且申停。''己亥,周巴陵城主尉迟宪降,遣巴州刺史侯安鼎守之。庚子,独孤盛将馀众自杨叶州潜遁。

二年春正月庚戌,大赦天下。以云麾将军、晋陵太守杜棱为侍中、领军将军。辛亥,以始兴王伯茂为宣惠将军、扬州刺史。乙卯,合州刺史裴景徽奔于齐。辛未,周湘州城主殷亮降,湘州平。二月丙戌,以太尉侯瑱为车骑将军、湘州刺史。庚寅,曲赦、湘州诸郡。三月乙卯,太尉、车骑将军、湘州刺史侯瑱薨。丁丑,以镇东将军、会稽太守徐度为镇南将军、湘州刺史。夏四月,分荆州之南平、宜都、罗、河东四郡,置南荆州,镇河东郡。以安西将军、武州刺史吴明彻为南荆州刺史。庚寅,以安左将军鲁悉达为安南将军、吴州刺史。辛卯,老人星见。秋七月丙午,周将贺若敦自拔遁归,人畜死者十七八。武陵、天门、南平、义阳、河东、宜都郡悉平。九月甲寅,诏曰:``姬业方阐,望载渭滨,汉历既融,道通圮上。若乃摛精辰宿,降灵惟岳,风云有感,梦寐是求,斯固舟楫盐梅,递相表里,长世建国,罔或不然。至于铭德太常,从祀清庙,以贻厥后来,垂诸不朽者也。前皇经济区宇,裁成品物,灵贶式甄,光膺宝命,虽谟明浚发,幽显协从,亦文武贤能,翼宣王业。故大司马、骠骑大将军瑱,故司空文育,故平北将军、开府仪同三司僧明,故中护军颖,故领军将军拟,或缔构艰难,经纶夷险;或摧锋冒刃,殉义遗生;或宣哲协规,绸缪帷幄;或披荆汗马,终始勤劬;莫不罄诚悉力,屯泰以之。朕以寡昧,嗣膺丕绪,永言勋烈,思弘典训,便可式遵故实,载扬盛轨,可并配食高祖庙庭,俾兹大猷,永传宗祏。''丙辰,以侍中、中权将军、特进、左光禄大夫、开府仪同三司王冲为丹阳尹;丹阳尹沈君理为左民尚书,领步兵校尉。冬十月乙巳,霍州西山蛮率部落内属。十一月乙卯,高丽国遣使献方物。甲子,以武昌、国川为竟陵郡,以安流民。十二月辛巳,以安东将军、吴郡太守孙瑒为中护军。甲申,立始兴国庙于京师,用王者之礼。太子中庶子虞荔、御史中丞孔奂以国用不足,奏立煮海盐赋及榷酤之科,诏并施行。先是,缙州刺史留异应于王琳等反,丙戌,诏司空侯安都率众讨之。

三年春正月庚戌,设帷宫于南郊,币告胡公以配天。辛亥,舆驾亲祠南郊。诏曰:``朕负荷宝图,亟回星琯,兢兢业业,庶几治定,而德化不孚,俗弊滋甚,永言念之,无忘日夜。阳和布气,昭事上玄,躬奉牺玉,诚兼飨敬,思与黎元,被斯宽惠,可普赐民爵一级,其孝悌力田,别加一等。''辛酉,舆驾亲祠北郊。闰二月己酉,以百济王馀明为抚东大将军,高句骊王高汤为宁东将军。江州刺史周迪举兵应留异,袭湓城,攻豫章郡,并不克。辛亥,以南荆州刺史吴明彻为安右将军。甲子,改铸五铢钱。三月丙子,安成王顼至自周,诏授侍中、中书监中卫将军,置佐史。丁丑,以安右将军吴明彻为安南将军、江州刺史,督众军南讨。甲申,大赦天下。庚寅,司空侯安都破留异于桃支岭,异脱身奔晋安,东阳郡平。夏四月癸卯,曲赦东阳郡。乙巳,齐遣使来聘。六月丙辰,以侍中、中卫将军安成王顼为骠骑将军、扬州刺史。以会稽、东阳、临海、永嘉、新安、新宁、晋安、建安八郡置东扬州。以扬州刺史始兴王伯茂为镇东将军、东扬州刺史,征北将军、司空、南徐州刺史侯安都为侍中、征北大将军。秋七月己丑,皇太子纳妃王氏。在位文武赐帛各有差,孝悌力田为父后者赐爵二级。九月戊辰朔,日有食之。以侍中、都官尚书到仲举为尚书右仆射、丹阳尹。丁亥,周迪请降,诏安成王顼督众军以招纳之。是岁,周所立梁王萧詧死,子岿代立。

四年春正月丙子,干陀利国遣使献方物。甲申,周迪弃城走,闽州刺史陈宝应纳之,临川郡平。壬辰,以平西将军、郢州刺史章昭达为护军将军,仁武将军、新州刺史华皎进号平南将军,镇南将军、开府仪同三司、高州刺史黄法抃为镇北大将军、南徐州刺史,安西将军、领临川太守周敷为南豫州刺史,中护军孙瑒为镇右将军。罢高州隶入江州。二月戊戌,征南将军、开府仪同三司、广州刺史欧阳頠进号征南大将军。庚戌,以侍中、司空、征北大将军侯安都为征南大将军、江州刺史。庚申,以平南将军华皎为南湘州刺史。三月辛未,以镇南将军、开府仪同三司徐度为侍中、中军大将军。辛巳,诏赠讨周迪将士死王事者。夏四月辛丑,设无珝大会于太极前殿。乙卯,以侍中、中书监、中卫将军、骠骑将军、扬州刺史安成王顼为开府仪同三司。五月丁卯,安前将军、右光禄大夫徐世谱卒。六月癸巳,太白昼见。司空侯安都赐死。七月丁丑,以镇北大将军、开府仪同三司、南徐州刺史黄法甗为镇南大将军、江州刺史。九月壬戌,开府仪同三司、广州刺史欧阳頠薨。癸亥,曲赦京师。辛未,周迪复寇临川,诏护军章昭达率众讨之。十一月辛酉,章昭达大破周迪,悉擒其党与,迪脱身潜窜。十二月丙申,大赦天下。诏护军将军章昭达进军建安,以讨陈宝应。信威将军、益州刺史余孝顷督会稽、东阳、临海、永嘉诸军自东道会之。癸丑,以前安南将军、江州刺史吴明彻为镇前将军。

五年春正月庚辰,以吏部尚书、领右军将军袁枢为丹阳尹。辛巳,舆驾亲祠北郊。乙酉,江州湓城火,烧死者二百馀人。三月丁丑,以征南大将军、开府仪同三司、桂州刺史淳于量为中抚军大将军。壬午,诏以故护军将军周铁虎配食高祖庙庭。夏四月庚子,周遣使来聘。五月庚午,罢南丹阳郡。是月,周、齐并遣使来聘。六月丁未,夜,有白气两道,出于北斗东南,属地。秋七月丁丑,诏曰:``朕以寡昧,属当负重,星籥亟改,冕旒弗旷,不能仰协璇衡,用调玉烛,傍慰苍生,以安黔首。兵无宁岁,民乏有年,移风之道未弘,习俗之患犹在,致令氓多触网,吏繁笔削,狱犴滋章,虽由物犯,囹圄淹滞,亦或有冤。念俾纳隍,载劳负扆,加以肤凑不适,摄卫有亏,比获微痊,思覃宽惠,可曲赦京师。''九月,城西城。冬十一月丁亥,以左卫将军程灵洗为中护军。己丑,章昭达破陈宝应于建安,擒宝应、留异,送京师,晋安郡平。甲辰,以护军将军章昭达为镇前将军、开府仪同三司。十二月甲子,曲赦建安、晋安二郡。讨陈宝应将士死王事者,并给棺槥,送还本乡,并复其家。疮痍未瘳者,给其医药。癸未,齐遣使来聘。

六年春正月甲午,皇太子加元服,王公以下赐帛各有差,孝悌力田为父后者赐爵一级,鳏寡孤独不能自存者谷人五斛。庚戌,以领军将军杜棱为翊左将军、丹阳尹,丹阳尹袁枢为吏部尚书,卫尉卿沈钦为中领军。三月乙未,诏侯景以来遭乱移在建安、晋安、义安郡者,并许还本土,其被略为奴婢者,释为良民。夏四月甲寅,以侍中、中书监、中卫将军、骠骑将军、开府仪同三司、扬州刺史安成王顼为司空。辛酉,有彗星见。周遣使来聘。秋七月癸未,大风至自西南,广百馀步,激坏灵台候楼。甲申,仪贤堂无故自坏。丙戌,临川太守骆文牙斩周迪,传首京师,枭于硃雀航。丁酉,太白昼见。八月丁丑,诏曰:``梁室多故,祸乱相寻,兵甲纷纭,十年不解,不逞之徒虐流生气,无赖之属暴及徂魂。江左肇基,王者攸宅,金行水位之主,木运火德之君,时更四代,岁逾二百。若其经纶王业,缙绅民望,忠臣孝子,何世无才,而零落山丘,变移陵谷,或皆剪伐,莫不侵残。玉杯得于民间,漆简传于世载,无复五株之树,罕见千年之表。自大祚光启,恭惟揖让,爰暨朕躬,聿修祖武,虽复旂旗服色,犹行杞、宋之邦,每车驾巡游,眇瞻河、雒之路,故乔山之祀,蘋藻弗亏,骊山之坟,松柏恒守。唯戚籓旧垄,士子故茔,掩殣未周,樵牧犹众。或亲属流隶,负土无期,子孙冥灭,手植何寄。汉高留连于无忌,宋祖惆怅于子房,丘墓生哀,性灵共恻者也。朕所以兴言永日,思慰幽泉。维前代王侯,自古忠烈,坟冢被发绝无后者,可检行修治,墓中树木,勿得樵采,庶幽显咸畅,称朕意焉。''己卯,立皇子伯固为新安郡王,伯恭为晋安王,伯仁为庐陵王,伯义为江夏王。九月癸未,罢豫章郡。是月,新作大航。冬十月辛亥,齐遣使来聘。十二月乙卯,立皇子伯礼为武陵王。丁巳,以镇前将军、开府仪同三司章昭达为镇南将军、江州刺史,镇南大将军、江州刺史黄法抃为中卫大将军,中护军程灵洗为宣毅将军、郢州刺史,军师将军、郢州刺史沈恪为中护军,镇东将军、吴兴太守吴明彻为中领军。戊午,以东中郎将、吴郡太守鄱阳王伯山为平北将军、南徐州刺史。癸亥,诏曰:``朕自居民牧之重,托在王公之上,顾其寡昧,郁于治道。加以屡亏听览,事多壅积,冤滞靡申,幽枉弗鉴。念兹罪隶,有甚纳隍。而惠泽未流,愆阳累月,今岁序云暮,元正向肇,欲使幽圄之内,同被时和,可曲赦京师。''

天康元年春二月丙子,诏曰:``朕以寡德,纂承洪绪,日昃劬劳,思弘景业,而政道多昧,黎庶未康,兼疹患淹时,亢阳累月,百姓何咎,实由朕躬,念兹在兹,痛如疾首。可大赦天下,改天嘉七年为天康元年。三月己卯,以骠骑将军、开府仪同三司、扬州刺史、司空安成王顼为尚书令。夏四月乙卯,皇孙至泽生,在位文武赐绢帛各有差,为父后者赐爵一级。癸酉,世祖疾甚。是日,崩于有觉殿。遗诏曰:``朕疾苦弥留,遂至不救,修短有命,夫复何言。但王业艰难,频岁军旅,生民多弊,无忘愧惕。今方隅乃定,俗教未弘,便及大渐,以为遗恨。社稷任重,太子可即君临,王侯将相,善相辅翊,内外协和,勿违朕意!山陵务存俭速。大敛竟,群臣三日一临,公除之制,率依旧典。''六月甲子,群臣上谥曰文皇帝,庙号世祖。丙寅,葬永宁陵。

世祖起自艰难,知百姓疾苦。国家资用,务从俭约。常所调敛,事不获已者,必咨嗟改色,若在诸身。主者奏决,妙识真伪,下不容奸,人知自励矣。一夜内刺闺取外事分判者,前后相续。每鸡人伺漏,传更签于殿中,乃敕送者必投签于阶石之上,令枪然有声,云``吾虽眠,亦令惊觉也''。始终梗概,若此者多焉。

陈吏部尚书姚察曰:世称继体守文,宗枝承统,得失之间,盖亦祥矣。大抵以奉而勿坠为贤能,挠而易之为不肖;其有光扬前轨,克荷曾构,固以少焉。世祖自初发迹,功庸显著,宁乱静寇,首佐大业。及国祸奄臻,入承宝祚,兢兢业业,其若驭朽,加以崇尚儒术,爱悦文义,见善如弗及,用人如由己,恭俭以御身,勤劳以济物,自昔允文允武之君,东征西怨之后,宾实之迹,可为联类。至于杖聪明,用鉴识,斯则永平之政,前史其论诸。

\hypertarget{header-n4218}{%
\subsubsection{卷四}\label{header-n4218}}

废帝

废帝,讳伯宗,字奉业,小字药王,世祖嫡长子也。梁承圣三年五月庚寅生。永定二年二月戊辰,拜临川王世子。三年,世祖嗣位,八月庚戌,立为皇太子。自梁室乱离,东宫焚烬,太子居于永福省。

天康元年四月癸酉,世祖崩,其日,太子即皇帝位于太极前殿,诏曰:``上天降祸,大行皇帝奄弃万国,攀号靡及,五内崩殒。朕以寡德,嗣膺宝命,茕茕在疚,惧甚缀旒,方赖宰辅,匡其不逮。可大赦天下。''又诏内外文武,各复其职,远方悉停奔赴。五月己卯,尊皇太后曰太皇太后,皇后曰皇太后。庚寅,以骠骑将军、司空、扬州刺史、新除尚书令安成王顼为骠骑大将军,进位司徒、录尚书、都督中外诸军事。丁酉,中军大将军、开府仪同三司徐度进位司空;镇南将军、开府仪同三司、江州刺史章昭达为侍中,进号征南将军;镇东将军、东扬州刺史始兴王伯茂进号征东将军、开府仪同三司;平北将军、南徐州刺史鄱阳王伯山进号镇北将军;吏部尚书袁枢为尚书左仆射;云麾将军、吴兴太守沈钦为尚书右仆射;新除中领军吴明彻为领军将军;新除中护军沈恪为护军将军;平南将军、湘州刺史华皎进号安南将军;散骑常侍、御史中丞徐陵为吏部尚书。六月辛亥,翊右将军、右光禄大夫王通进号安右将军。秋七月丁酉,立妃王氏为皇后。冬十月庚申,舆驾奉祠太庙。十一月乙亥,周遣使来吊。十二月甲子,高丽国遣使献方物。

光大元年春正月癸酉,尚书左仆射袁枢卒。乙亥,诏曰:``昔昊天成命,降集宝图,二后重光,九区咸乂。闵余冲薄,王道未昭,荷兹神器,如涉灵海,庶亲贤并建,牧伯惟良,天下雍熙,缅同刑措。今三元改历,万国充庭,清庙无追,具僚斯在,言瞻宁位,触感崩心。思播遗恩,俾覃黎献。可大赦天下。改天康二年为光大元年。孝悌力田赐爵一级。''己卯,以领军将军吴明彻为丹阳尹。辛卯,舆驾亲祠南郊。二月辛亥,宣毅将军、南豫州刺史余孝顷谋反伏诛。癸丑,以征东将军、开府仪同三司、东扬州刺史始兴王伯茂为中卫大将军,开府仪同三司黄法抃为镇北将军、南徐州刺史,镇北将军、南徐州刺史鄱阳王伯山为镇东将军、东阳州刺史。三月甲午,以尚书右仆射沈钦为侍中、尚书左仆射。夏四月乙卯,太白昼见。五月癸巳,以领军将军、丹阳尹吴明彻为安南将军、湘州刺史。乙未,以镇右将军杜棱为领军将军。安南将军、湘州刺史华皎谋反,丙申,以中抚大将军淳于量为使持节、征南大将军,总率舟师以讨之。六月壬寅,以中军大将军、司空徐度进号车骑将军,总督京邑众军,步道袭湘州。闰月癸巳,以云麾将军新安王伯固为丹阳尹。秋七月戊申,立皇子至泽为皇太子,赐天下为父后者爵一级,王公卿士已下赉帛各有差。九月乙巳,诏曰:``逆贼华皎,极恶穷凶,遂树立萧岿,谋危社稷。弃亲即仇,人神愤惋,王师电迈,水陆争前,枭剪之期,匪朝伊暮。其家口在北里尚方,宜从诛戮,用明国宪。''丙辰,百济国遣使献方物。是月,周将长胡公拓跋定率步骑二万入郢州,与华皎水陆俱进,都督淳于量、吴明彻等与战,大破之。皎单舸奔江陵,擒拓跋定,俘获万馀人,马四千馀匹,送京师。冬十月辛巳,赦湘、巴二州为皎所诖误者。甲申,舆驾亲祠太庙。十一月己未,以护军将军沈恪为平西将军、荆州刺史。甲子,侍中、中权将军、开府仪同三司、特进、左光禄大夫王冲薨。十二月庚寅,以兼从事中郎孔英哲为奉圣亭侯,奉孔子祀。

二年春正月己亥,侍中、都督中外诸军事、骠骑大将军、司徒、录尚书、扬州刺史安成王顼进位太傅,领司徒,加殊礼,剑履上殿;侍中、征南将军、开府仪同三司、江州刺史章昭达进号征南大将军;中抚大将军、新除征南大将军淳于量为侍中、中军大将军、开府仪同三司;安南将军、湘州刺史吴明彻即本号开府仪同三司,进号镇南将军;云麾将军、郢州刺史程灵洗进号安西将军。庚子,诏讨华皎军人死王事者并给棺槥,送还本乡,仍复其家。甲子,罢吴州,以鄱阳郡还属江州。侍中、司空、车骑将军徐度薨。夏四月辛巳,太白昼见。丁亥,割东扬州晋安郡为豊州。五月丙辰,太傅安成王顼献玉玺一。六月丁卯,彗星见。秋七月丙午,舆驾亲祠太庙。戊申,新罗国遣使献方物。壬戌,立皇弟伯智为永阳王,伯谋为桂阳王。九月甲辰,林邑国遣使献方物。丙午,狼牙修国遣使献方物。以侍中、征南大将军、开府仪同三司、江州刺史章昭达为中抚大将军。戊午,太白昼见。冬十月庚午,舆驾亲祠太庙。十一月丙午,以前平西将军、荆州刺史沈恪为护军将军。壬子,以镇北将军、开府仪同三司、南徐州刺史黄法抃为镇西将军、郢州刺史,新除中军大将军、开府仪同三司淳于量为镇北将军、南徐州刺史。甲寅,慈训太后集群臣于朝堂,令曰:

中军仪同、镇北仪同、镇右将军、护军将军、八座卿士:昔梁运季末,海内沸腾,天下苍生,殆无遗噍。高祖武皇帝拨乱反正,膺图御箓,重悬三象,还补二仪;世祖文皇帝克嗣洪基,光宣宝业,惠养中国,绥宁外荒;并战战兢兢,劬劳缔构,庶几鼎运,方隆殷、夏。\#\#\$\$伯宗昔在储宫,本无令问,及居崇极,遂骋凶淫。居处谅闇,固不哀戚,嫔嫱弗隔,就馆相仍,岂但衣车所纳,是讥宗正,衰绖生子,得诮右师。七百之祚何凭,三千之罪为大。且费引金帛,令充椒阃,内府中藏,军备国储,未盈期稔,皆已空竭。太傅亲承顾托,镇守宫闱,遗诰绸缪,义深垣屏,而欑涂未御,翌日无淹,仍遣刘师知、殷不佞等显言排斥。韩子高小竖轻佻,推心委仗,阴谋祸乱,决起萧墙。元相虽持,但除君侧。又以余孝顷密迩京师,便相征召,殃慝之咎,凶徒自擒,宗社之灵,祅氛是灭。于是密诏华皎,称兵上流,国祚忧惶,几移丑类。乃至要招远近,叶力巴、湘,支党纵横,寇扰黟、歙。又别敕欧阳纥等攻逼衡州,岭表纷纭,殊淹弦望。岂止罪浮于昌邑,非唯声丑于太和。但贼竖皆亡,妖徒已散,日望惩改,犹加掩抑,而悖礼忘德,情性不悛,乐祸思乱,昏慝无已。张安国蕞尔凶狡,穷为小盗,仍遣使人蒋裕钩出上京,即置行台,分选凶党。贼皎妻吕,舂徒为戮,纳自奚官,藏诸永巷,使其结引亲旧,规图戕祸。荡主侯法喜等,太傅麾下,恒游府朝,啖以深利,谋兴肘腋。适又荡主孙泰等潜相连结,大有交通,兵力殊强,指期挺乱。皇家有庆,历数遐长,天诱其衷,同然开发。此诸文迹,今以相示,是而可忍,谁则不容?祖宗基业,将惧倾陨,岂可复肃恭禋祀,临御兆民?式稽故实,宜在流放,今可特降为临海郡王,送还籓邸。太傅安成王固天生德,齐圣广深,二后钟心,三灵伫眷。自前朝不悆,任总邦家,威惠相宣,刑礼兼设,指挥啸诧,湘、郢廓清,辟地开疆,荆、益风靡,若太戊之承殷历,中都之奉汉家,校以功名,曾何仿佛。且地彰灵玺,天表长彗,布新除旧,祯祥咸显。文皇知子之鉴,事甚帝尧,传弟之怀,又符太伯。今可还申曩志,崇立贤君,方固宗祧,载贞辰象。中外宜依旧典,奉迎舆驾。未亡人不幸属此殷忧,不有崇替,容危社稷,何以拜祠高寝,归祔武园?揽笔潸然,兼怀悲庆。

是日,出居别第。太建二年四月薨,时年十九。

帝仁弱无人君之器,世祖每虑不堪继业。既居冢嫡,废立事重,是以依违积载。及疾将大渐,召高宗谓曰:``吾欲遵太伯之事。''高宗初未达旨,后寤,乃拜伏涕泣,固辞。其后宣太后依诏废帝焉。

史臣曰:临海虽继体之重,仁厚懦弱,混一是非,不惊得丧,盖帝挚、汉惠之流也。世祖知神器之重,谅难负荷,深鉴尧旨,弗传宝祚焉。

\hypertarget{header-n4230}{%
\subsubsection{卷五}\label{header-n4230}}

宣帝

高宗孝宣皇帝讳顼,字绍世,小字师利,始兴昭烈王第二子也。梁中大通二年七月辛酉生,有赤光满堂室。少宽大,多智略。及长,美容仪,身长八尺三寸,手垂过膝。有勇力,善骑射。高祖平侯景,镇京口,梁元帝征高祖子侄入侍,高祖遣高宗赴江陵,累官为直阁将军、中书侍郎。时有马军主李总与高宗有旧,每同游处。高宗尝夜被酒,张灯而寐,总适出,寻返,乃见高宗身是大龙,总便惊骇,走避他室。及江陵陷,高宗迁于关右。永定元年,遥袭封始兴郡王,邑二千户。三年,世祖嗣位,改封安成王。天嘉三年,自周还,授侍中、中书监、中卫将军,置佐史。寻授使持节、都督扬南徐东扬南豫北江五州诸军事、扬州刺史,进号骠骑将军,馀如故。四年,加开府仪同三司。六年,迁司空。天康元年,授尚书令,馀并如故。废帝即位,拜司徒,进号骠骑大将军,录尚书,都督中外诸军事,给班剑三十人。光大二年正月,进位太傅,领司徒,加殊礼,剑履上殿,增邑并前三千户,馀并如故。十一月甲寅,慈训太后令废帝为临海王,以高宗入纂。

太建元年春正月甲午,即皇帝位于太极前殿,诏曰:``夫圣人受命,王者中兴,并由懿德,方作元后。高祖武皇帝揖拜尧图,经纶禹迹,配天之业,光辰象而利贞,格地之功,侔川岳而长远。世祖文皇帝体上圣之姿,当下武之运,筑宫示俭,所务唯德,定鼎初基,厥谋斯在。朕以寡薄,才非圣贤,夙荷前规,方传景祚。虽复亲承训诲,志守籓维,咏季子之高风,思城阳之远托,自元储绍国,正位君临,无道非几,伫闻刑措。岂图王室不造,频谋乱阶,天步艰难,将倾宝历,仰惟嘉命,爰集朕躬。我心贞确,坚誓苍昊,而群辟启请,相喧渭桥,文母尊严,悬心长乐,对扬玺绂,非止殷汤之三辞,履涉春冬,何但代王之五让。今便肃奉天策,钦承介圭。若据沧溟,逾增兢业。思所以云行雨施,品物咸亨,当与黔黎,普同斯庆。可改光大三年为太建元年。大赦天下。在位文武赐位一阶,孝悌力田及为父后者赐爵一级,异等殊才,并加策序。鳏寡孤独不能自存者,人赐谷五斛。''复太皇太后尊号曰皇太后。立妃柳氏为皇后,世子叔宝为皇太子,皇子南中郎将、江州刺史康乐侯叔陵为始兴王,奉昭烈王祀。乙未,舆驾谒太庙。丁酉,分命大使巡行四方,观省风俗。征南大将军、开府仪同三司、新除中抚大将军章昭达进号车骑大将军,新除中军大将军、开府仪同三司、南徐州刺史淳于量为征北大将军,镇北将军、开府仪同三司、南徐州刺史、新除镇西将军、郢州刺史黄法抃进号征西大将军,新除安南将军、开府仪同三司、湘州刺史吴明彻进号镇南将军,镇东将军、扬州刺史、鄱阳王伯山进号中卫将军,尚书仆射沈钦为尚书左仆射,度支尚书王劢为尚书右仆射,护军将军沈恪为镇南将军、广州刺史。辛丑,舆驾亲祠南郊。壬寅,以皇子建安侯叔英为宣惠将军、东扬州刺史,改封豫章王。豊城侯叔坚改封长沙王。癸卯,以明威将军周弘正为特进。戊午,舆驾亲祠太庙。二月庚午,皇后谒太庙。辛未,皇太子谒太庙。乙亥,舆驾亲耕藉田。夏五月甲午,齐遣使来聘。丁巳,以吏部尚书、领大著作徐陵为尚书右仆射,太子詹事、驸马都尉沈君理为吏部尚书。秋七月辛卯,皇太子纳妃沈氏,王公已下赐帛各有差。丁酉,以平东将军、吴郡太守晋安王伯恭为中护军,进号安南将军。九月甲辰,以新除中护军晋安王伯恭为中领军。冬十月,新除左卫将军欧阳纥据广州举兵反。辛未,遣车骑将军、开府仪同三司章昭达率众讨之。壬午,舆驾亲祠太庙。

二年春正月乙酉,以征西大将军、开府仪同三司、郢州刺史黄法抃为中权大将军。丙午,舆驾亲祠太庙。二月癸未,仪同章昭达擒欧阳纥送都,斩于建康市,广州平。三月丙申,皇太后崩。丙午,曲赦广、衡二州。丁未,大赦天下。又诏自讨周迪、华皎已来,兵交之所有死亡者,并令收敛,并给棺槥,送还本乡;疮痍未瘳者,各给医药。夏四月乙卯,临海王伯宗薨。戊寅,皇太后祔葬万安陵。闰月戊申,舆驾谒太庙。己酉,太白昼见。五月乙卯,仪同黄法抃献瑞璧一。壬午,齐遣使来吊。六月戊子,新罗国遣使献方物。辛卯,大雨雹。乙巳,分遣大使巡行州郡,省理冤屈。戊申,车骑将军、开府仪同三司章昭达进号车骑大将军,安南将军、广州刺史沈恪进号镇南将军。秋八月甲申,诏曰:``怀远以德,抑惟恒典,去戎即华,民之本志。顷年江介襁负相随,崎岖归化,亭候不绝,宜加恤养,答其诚心。维是荒境自皞,有在都邑及诸州镇,不问远近,并蠲课役。若克平旧土,反我侵地,皆许还乡,一无拘限。州郡县长明加甄别,良田废村,随便安处。若辄有课订,即以扰民论。''又诏曰:``民惟邦本,著在典谟,治国爱民,抑又通训。朕听朝晏罢,日昃劬劳,方流惠泽,覃被亿兆。有梁之季,政刑废缺,条纲弛紊,僭盗荐兴,役赋征徭,尤为烦刻。大陈御宇,拯兹馀弊,灭扈戡黎,弗遑创改,年代弥流,将及成俗,如弗解张,物无与厝,夕惕疚怀,有同首疾。思从卑菲,约己济民,虽府帑末充,君孰与足,便可删革,去其甚泰,冀永为定准,令简而易从。自今维作田,值水旱失收,即列在所,言上折除。军士年登六十,悉许放还。巧手于役死亡及与老疾,不劳订补。其籍有巧隐,并王公百司辄受民为程廕,解还本属,开恩听首。在职治事之身,须递相检示,有失不推,当局任罪。令长代换,具条解舍户数,付度后人。户有增进,即加擢赏;若致减散,依事准结。有能垦起荒田,不问顷亩少多,依旧蠲税。''戊子,太白昼见。九月乙丑,以散骑常侍镇东将军吴兴太守杜棱为特进、护军将军。冬十月乙酉,舆驾亲祠太庙。十一月辛酉,高丽国遣使献方物。十二月癸巳夜,西北有雷声。

三年春正月癸丑,以尚书右仆射、领大著作徐陵为尚书仆射。辛酉,舆驾亲祠南郊。辛未,亲祠北郊。二月辛巳,舆驾亲祠明堂。丁酉,亲耕籍田。三月丁丑,大赦天下。自天康元年讫太建元年,逋馀军粮、禄秩、夏调未入者,悉原之。又诏犯逆子弟支属逃亡异境者,悉听归首;见絷系者,量可散释;其有居宅,并追还。夏四月壬辰,齐遣使来聘。五月戊申,太白昼见。辛亥,辽东、新罗、丹丹、天竺、盘盘等国并遣使献方物。六月丁亥,江阴王萧季卿以罪免。甲辰,封东中郎将长沙王府谘议参军萧彝为江阴王。秋八月辛丑,皇太子亲释奠于太学,二傅、祭酒以下赉帛各有差。九月癸酉,太白昼见。冬十月甲申,舆驾亲祠太庙。乙酉,周遣使来聘。己亥,丹丹国遣使献方物。十二月壬辰,车骑大将军、司空章昭达薨。

四年春正月丙午,以云麾将军、江州刺史始兴王叔陵为湘州刺史,进号平南将军;东中郎将、吴郡太守长沙王叔坚为宣毅将军、江州刺史;尚书仆射、领大著作徐陵为尚书左仆射;中书监王劢为尚书右仆射。庚申,以丹阳尹衡阳王伯信为信威将军、中护军。庚午,舆驾亲祠太庙。二月乙酉,立皇子叔卿为建安王,授东中郎将、东扬州刺史。三月壬子,以散骑常侍孙瑒为安西将军、荆州刺史。乙丑,扶南、林邑国并遣使来献方物。夏四月戊子,以中权大将军、开府仪同三司黄法抃为征南大将军、南豫州刺史。五月癸卯,尚书右仆射王劢卒。六月辛巳,侍中、镇右将军、右光禄大夫杜棱卒。秋八月辛未,周遣使来聘。丁丑,景云见。戊寅,诏曰:``国之大事,受赈兴戎。师出以律,禀策于庙,所以乂安九有,克成七德。自顷扫涤群秽,廓清诸夏,乃貔貅之戮力,亦帷幄之运筹。虽左衽已戡,干戈载戢,呼韩来谒,亭鄣无警,但不教民战,是谓弃之,仁必有勇,无忘武备。磻溪之传韬诀,谷城之授神符,文叔悬制戎规,孟德颇言兵略。朕既惭暗合,良皆披览。兼昔经督戎,备尝行阵,齐以七步,肃之三鼓,得自胸襟,指掌可述。今并条制,凡十三科,宜即班宣,以为永准。''乙未,诏停督湘、江二州逋租,无锡等十五县流民,并蠲其徭赋。九月庚子朔,日有蚀之。辛亥,大赦天下。又诏曰:``举善从谏,在上之明规;进贤谒言,为臣之令范。朕以寡德,嗣守宝图,虽世袭隆平,治非宁一。辨方分职,旰食早衣;傍阙争臣,下无贡士。何其阙尔,鲜能抗直。岂余独运,匪荐谠言。置鼓公车,罕论得失;施石象魏,莫陈可否。硃云摧槛,良所不逢;禽息触楹,又为难值。至如衣褐以见,檐簦以游,或耆艾绝伦,或妙年异等,干时而不偶,左右莫之誉,黑貂改弊,黄金且殚,终其滞淹,可为太息。又贵为百辟,贱有十品,工拙并骛,劝沮莫分,街谣徒拥,廷议斯阙。实朕之弗明,而时无献替。永言至治,何乃爽欤?外可通示文武:凡厥在位,风化乖殊,朝政纟比蠹,正色直辞,有犯无隐。兼各举所知,随才明试。其莅政廉秽,在职能否,分别矢言,俟兹黜陟。''丙寅,以故太尉徐度、仪同杜棱、仪同程灵洗配食高祖庙庭,故车骑将军章昭达配食世祖庙庭。冬十月乙酉,舆驾亲祠太庙。戊戌,以镇南将军、广州刺史沈恪为领军将军。十一月己亥夜地震。闰月辛未,诏曰:``姑熟饶旷,荆河斯拟,博望关畿,天限严峻,龙山南指,牛渚北临,对熊绎之馀城,迩全琮之故垒,良畴美柘,畦畎相望,连宇高甍,阡陌如绣。自梁末兵灾,凋残略尽,比虽务优宽,犹未克复,咫尺封畿,宜须殷阜。且众将部下,多寄上下,军民杂俗,极为蠹秏。自今有罢任之徒,许分留部下;其已在江外,亦令迎还,悉住南州津里安置。有无交货,不责市估;莱荒垦辟,亦停租税。台遣镇监一人,共刺史、津主分明检押,给地赋田,各立顿舍。''十二月壬寅,甘露降乐游苑。甲辰,舆驾幸乐游苑,采甘露,宴群臣。丁卯,诏曰:``梁氏之季,兵火荐臻,承华焚荡,顿无遗构。宝命惟新,迄将二纪,频事戎旅,未遑修缮。今工役差闲,椽楹有拟,来岁开肇,创筑东宫,可权置起部尚书、将作大匠,用主监作。''

五年春正月癸酉,以征北大将军、开府仪同三司、南徐州刺史淳于量为中权大将军;宣惠将军、豫章王叔英为南徐州刺史,进号平北将军;吏部尚书、驸马都尉沈君理为尚书右仆射,领吏部。辛巳,舆驾亲祠南郊。甲午,舆驾亲祠太庙。二月辛丑,舆驾亲祠明堂。乙卯,夜有白气如虹,自北方贯北斗紫宫。三月壬午,分命众军北伐,以镇前将军、开府仪同三司吴明彻都督征讨诸军事。丙戌,西衡州献马生角。己丑,皇孙胤生,内外文武赐帛各有差,为父后者爵一级。北讨大都督吴明彻统众十万,发自白下。夏四月癸卯,前巴州刺史鲁广达克齐大岘城。辛亥,吴明彻克秦州水栅。庚申,齐遣兵十万援历阳,仪同黄法抃破之。辛酉,齐军救秦州,吴明彻又破之。癸亥,诏北伐众军所杀齐兵,并令埋掩。甲子,南谯太守徐槾克石梁城。五月己巳,瓦梁城降。癸酉,阳平郡城降。甲戌,徐槾克庐江郡城。丙子,黄法抃克历阳城。己卯,北高唐郡城降。辛巳,诏征南大将军、开府仪同三司、南豫州刺史黄法抃徙镇历阳,齐改县为郡者并复之。乙酉,南齐昌太守黄咏克齐昌外城。丙戌,庐陵内史任忠军次东关,克其东西二城,进克蕲城。戊子,又克谯郡城,秦州城降。癸巳,瓜步、胡墅二城降。六月庚子,郢州刺史李综克滠口城。乙巳,任忠克合州外城。庚戌,淮阳、沭阳郡并弃城走。癸丑,景云见。豫章内史程文季克泾州城。乙卯,宣毅司马湛陀克新蔡城。癸亥,周遣使来聘。黄法抃克合州城。吴明彻师次仁州,甲子,克其州城。是月,治明堂。秋七月乙丑,镇前将军、开府仪同三司吴明彻进号征北大将军。戊辰,齐遣众二万援齐昌,西阳太守周炅破之。己巳,吴明彻军次峡口,克其北岸城,南岸守者弃城走。周炅克巴州城。淮北绛城及谷阳士民,并诛其渠帅,以城降。丙戌,吴明彻克寿阳外城。八月乙未,山阳城降。壬寅,盱眙城降。戊申,罢南齐昌郡。壬子,戎昭将军徐敬辩克海安城。青州东海城降。戊午,平固侯陈敬泰等克晋州城。九月甲子,阳平城降。壬申,高唐太守沈善度克马头城。甲戌,齐安城降。丙子,左卫将军樊毅克广陵楚子城。癸未,尚书右仆射、领吏部、驸马都尉沈君理卒。丁亥,前鄱阳内史鲁天念克黄城小城,齐军退保大城。戊子,割南兗州之盱眙郡属谯州。壬辰晦,夜明。黄城大城降。冬十月甲午,郭默城降。戊戌,以中书令王瑒为吏部尚书。己亥,以特进、领国子祭酒周弘正为尚书右仆射。乙巳,吴明彻克寿阳城,斩王琳,传首京师,枭于硃雀航。丁未,齐兵万人至颍口。樊毅击走之。辛亥,齐遣兵援苍陵,又破之。丙辰,诏曰:``梁末得悬瓠,以寿阳为南豫州,今者克复,可还为豫州。以黄城为司州,治下为安昌郡,鳷湍为汉阳郡,三城依梁为义阳郡,并属司州。''以征北大将军、开府仪同三司吴明彻为豫州刺史,进号车骑大将军;征南大将军、开府仪同三司、南豫州刺史黄法抃为征西大将军、合州刺史。戊午,湛陀克齐昌城。十一月甲戌,淮阴城降。庚辰,威虏将军刘桃根克朐山城。辛巳,樊毅克济阴城。己丑,鲁广达等克北徐州。十二月壬辰朔,诏曰:``古者反噬叛逆,尽族诛夷,所以藏其首级,诫之后世。比者所戮止在一身,子胤或存,枭悬自足,不容久归武库,长比月支。恻隐之怀,有仁不忍。维熊昙朗、留异、陈宝应、周迪、邓绪等及今者王琳首,并还亲属,以弘广宥。''乙未,谯城降。乙巳,立皇子叔明为宜都王,叔献为河东王。壬午,任忠克霍州城。

六年春正月壬戌朔,诏曰:``王者以四海为家,万姓为子,一物乖方,夕惕犹厉,六合未混,旰食弥忧。朕嗣纂鸿基,思弘经略,上符景宿,下叶人谋,命将兴师,大拯沦溺。灰琯未周,凯捷相继,拓地数千,连城将百。蠢彼馀黎,毒兹异境,江淮年少,犹有剽掠,乡闾无赖,摘出阴私,将帅军人,罔顾刑典,今使苛法蠲除,仁声载路。且肇元告庆,边服来荒,始睹皇风,宜覃曲泽,可赦江右淮北南司、定、霍、光、建、朔、合、豫、北徐、仁、北兗、青、冀,南谯、南兗十五州,郢州之齐安、西阳,江州之齐昌、新蔡、高唐,南豫州之历阳、临江郡土民,罪无轻重,悉皆原宥。将帅职司,军人犯法,自依常科。''以翊前将军新安王伯固为中领军,进号安前将军;安前将军、中领军晋安王伯恭为安南将军、南豫州刺史。壬午,舆驾亲祠太庙。甲申,广陵金城降。周遣使来聘。高丽国遣使献方物。二月壬辰朔,日有蚀之。辛亥,舆驾亲耕籍田。丙辰,以中权大将军、开府仪同三司淳于量为征西大将军、郢州刺史。三月癸亥,诏曰:``去岁南川颇言失稔,所督田租于今未即。豫章等六郡太建五年田租,可申半至秋。豫章又逋太建四年检首田税,亦申至秋。南康一郡,岭下应接,民间尤弊,太建四年田租未入者,可特原除。庶修垦无废,岁取方实。''夏四月庚子,彗星见。辛丑,诏曰:``戢情怀善,有国之令图,拯弊救危,圣范之通训。近命师薄伐,义在济民,青、齐旧隶,胶、光部落,久患凶戎,争归有道,弃彼农桑,忘其衣食。而大军未接,中途止憩,朐山、黄郭,车营布满,扶老携幼,蓬流草跋,既丧其本业,咸事游手,饥馑疾疫,不免流离。可遣大使精加慰抚,仍出阳平仓谷,拯其悬磬,并充粮种。劝课士女,随近耕种。石鳖等屯,适意修垦。''六月壬辰,尚书右仆射、领国子祭酒周弘正卒。乙巳,以中卫将军、扬州刺史鄱阳王伯山为征北将军、南徐州刺史,中护军衡阳王伯信为宣毅将军、扬州刺史。冬十一月乙亥,诏北讨行军之所,并给复十年。十二月癸巳,平南将军、湘州刺史始兴王叔陵进号镇南将军。戊戌,以吏部尚书王瑒为尚书右仆射,度支尚书孔奂为吏部尚书。丙午,安右将军、左光禄大夫王通加特进。

七年春正月辛未,舆驾亲祠南郊。乙亥,左卫将军樊毅克潼州城。辛巳,舆驾亲祠北郊。二月戊申,樊毅克下邳、高栅等六城。三月辛未,诏豫、二兗、谯、徐、合、霍、南司、定九州及南豫、江、郢所部在江北诸郡置云旗义士,往大军及诸镇备防。戊寅,以新除征西大将军、合州刺史、开府仪同三司黄法抃为豫州刺史。改梁东徐州为安州,武州为沅州。移谯州镇于新昌郡,以秦郡属之。盱眙、神农二郡还隶南兗州。夏四月丙戌,有星孛于大角。庚寅,监豫州陈桃根于所部得青牛,献之,诏遣还民。甲午,舆驾亲祠太庙。乙未,陈桃根又表上织成罗文锦被各二,诏于云龙门外焚之。壬子,郢州献瑞钟六。五月乙卯,割谯州之秦郡还隶南兗州。分北谯县置北谯郡,领阳平所属北谯、西谯二县。合州之南梁郡,隶入谯州。六月丙戌,为北讨将士死王事者克日举哀。壬辰,以尚书右仆射王瑒为尚书仆射。己酉,改作云龙、神虎门。秋八月壬寅,移西阳郡治保城。癸卯,周遣使来聘。闰九月壬辰,都督吴明彻大破齐军于吕梁。是月,甘露频降乐游苑。丁未,舆驾幸乐游苑,采甘露,宴群臣,诏于苑龙舟山立甘露亭。冬十月戊午,以征北将军、南徐州刺史鄱阳王伯山为征南将军、江州刺史;安前将军、中领军新安王伯固为南徐州刺史,进号镇北将军;信威将军、江州刺史长沙王叔坚为云麾将军、中领军。己巳,立皇子叔齐为新蔡王,叔文为晋熙王。十一月庚戌,以征西大将军、开府仪同三司、郢州刺史淳于量为中军大将军。十二月丙辰,以新除云麾将军、郢州刺史长沙王叔坚为平越中郎将、广州刺史,东中郎将、东扬州刺史建安王叔卿为云麾将军、郢州刺史,宣惠将军宜都王叔明为东扬州刺史。壬戌,以尚书仆射王瑒为尚书左仆射,太子詹事、扬州大中正陆缮为尚书右仆射,国子祭酒徐陵为领军将军。甲子,南康郡献瑞钟。

八年春正月庚辰,西南有紫云见。二月壬申,车骑大将军、开府仪同三司吴明彻进位司空。丁丑,诏江东道太建五年以前租税夏调逋在民间者,皆原之。夏西月甲寅,诏曰:``元戎凯旋,群师振旅,旌功策赏,宜有飨宴。今月十七日,可幸乐游苑,设丝竹之乐,大会文武。''己未,舆驾亲祠太庙。五月庚寅,尚书左仆射王瑒卒。六月癸丑,以云麾将军、广州刺史长沙王叔坚为合州刺史,进号平北将军。甲寅,以尚书右仆射陆缮为尚书左仆射,新除晋陵太守王克为尚书右仆射。秋八月丁卯,以车骑大将军、司空吴明彻为南兗州刺史。九月戊戌,以皇子叔彪为淮南王。冬十一月乙酉,以平南将军、湘州刺史长沙王叔坚为平西将军、郢州刺史。丁酉,分江州晋熙、高唐、新蔡三郡为晋州。辛丑,以冠军将军庐陵王伯仁为中领军。十二月丁卯,以新除太子詹事徐陵为右光禄大夫。

九年春正月辛卯,舆驾亲祠北郊。壬寅,以湘州刺史、新除中卫将军始兴王叔陵为扬州刺史;云麾将军建安王叔卿为湘州刺史,进号平南将军。二月壬子,舆驾亲耕藉田。夏五月丙子,诏曰:``朕昧旦求衣,日旰方食,思弘亿兆,用臻俾乂,而牧守莅民,廉平未洽,年常租赋,多致逋馀,即此务农,宜弘宽省。可起太建已来讫八年流移叛户所带租调,七年八年叛义丁、五年讫八年叛军丁、六年七年逋租田米粟夏调绵绢丝布麦等,五年讫七年逋赀绢,皆悉原之。''秋七月乙亥,以轻车将军、丹阳尹江夏王伯义为合州刺史。己卯,百济国遣使献方物。庚辰,大雨,震万安陵华表。己丑,震慧日寺刹及瓦官寺重门,一女子于门下震死。冬十月戊午,司空吴明彻破周将梁士彦众数万于吕梁。十二月戊申,东宫成,皇太子移于新宫。

十年春正月己巳朔,以中领军庐陵王伯仁为平北将军、南徐州刺史,翊左将军、右光禄大夫、领太子詹事徐陵为领军将军。二月甲子,北讨众军败绩于吕梁,司空吴明彻及将卒已下,并为周军所获。三月辛未,震武库。丙子,分命众军以备周:中军大将军、开府仪同三司淳于量为大都督,总水陆诸军事;明威将军孙瑒都督荆、郢水陆诸军事,进号镇西将军;左卫将军樊毅为大都督,督硃沛、清口上至荆山缘淮众军,进号平北将军;武毅将军任忠都督寿阳、新蔡、霍州等众军,进号宁远将军。乙酉,大赦天下。丁酉,以中军大将军、开府仪同三司、护军将军淳于量为南兗州刺史,进号车骑将军。夏四月庚戌,诏曰:``懋赏之言,明于训诰,挟纩之美,著在抚巡。近岁薄伐,廓清淮、泗,摧锋致果,文武毕力,栉风沐雨,寒暑亟离,念功在兹,无忘终食。宜班荣赏,用酬厥劳。应在军者可并赐爵二级,并加赉恤,付选即便量处。''又诏曰:``惟尧葛衣鹿裘,则天为大,伯禹弊衣菲食,夫子曰`无间然',故俭德之恭,约失者鲜。朕君临宇宙,十变年籥,旰日勿休,乙夜忘寝,跂予思治,若济巨川,念兹在兹,懔同驭朽。非贪四海之富,非念黄屋之尊,导仁寿以置群生,宁劳役以奉诸己。但承梁季,乱离斯瘼,宫室禾黍,有名亡处,虽轮奂未睹,颇事经营,去泰去甚,犹为劳费。加以戎车屡出,千金日损,府帑未充,民疲征赋。百姓不足,君孰与足?兴言静念,夕惕怀抱,垂训立法,良所多惭。斫雕为朴,庶几可慕,雉头之服既焚,弋绨之衣方袭,损撤之制,前自朕躬,草偃风行,冀以变俗。应御府堂署所营造礼乐仪服军器之外,其馀悉皆停息;掖庭常供、王侯妃主诸有俸恤,并各量减。''丁巳,以新除镇右将军新安王伯固为护军将军。戊午,樊毅遣军度淮北对清口筑城。庚申,大雨雹。壬戌,清口城不守。五月甲申,太白昼见。六月丁卯,大雨,震大皇寺刹、庄严寺露盘、重阳阁东楼、千秋门内槐树、鸿胪府门。秋七月戊戌,新罗国遣使献方物。乙巳,以散骑常侍、兼吏部尚书袁宪为吏部尚书。八月乙丑朔,改秦郡为义州。戊寅,陨霜,杀稻菽。九月壬寅,以平北将军樊毅为中领军。乙巳,立方明坛于娄湖。戊申,以中卫将军、扬州刺史始兴王叔陵兼王官伯临盟。甲寅,舆驾幸娄湖临誓。乙卯,分遣大使以盟誓班下四方,上下相警戒也。壬戌,以宣惠将军江夏王伯义为东扬州刺史。冬十月戊寅,罢义州及琅邪、彭城二郡。立建兴,领建安、同夏、乌山、江乘、临沂、湖熟等六县,属扬州。戊子,以尚书左仆射陆缮为尚书仆射。十一月辛丑,以镇西将军孙瑒为郢州刺史。十二月乙亥,合州庐江蛮田伯兴出寇枞阳,刺史鲁广达讨平之。

十一年春正月丁酉,龙见于南兗州永宁楼侧池中。二月癸亥,舆驾亲耕藉田。三月丁未,诏淮北义人率户口归国者,建其本属旧名,置立郡县,即隶近州,赋给田宅,唤订一无所预。夏五月乙巳,诏曰:``昔轩辕命于风后、力牧,放勋咨尔稷、契、硃武,冕旒垂拱,化致隆平。爰逮汉列五曹,周分六职,设官理务,各有攸司,亦几期刑措,卜世弥永,并赖群才,用康庶绩。朕日昃劬劳,思弘治要,而机事尚拥,政道未凝,夕惕于怀,罔知攸济。方欲仗兹舟楫,委成股肱,征名责实,取宁多士。自今应尚书曹、府、寺、内省监、司文案,悉付局参议分判。其军国兴造、征发、选序、三狱等事,前须详断,然后启闻。凡诸辩决,务令清乂,约法守制,较若画一,不得前后舛互,自相矛盾,致有枉滞。纡意舞文,纠听所知,靡有攸赦。''甲寅,诏曰:``旧律以枉法受财为坐虽重,直法容贿其制甚轻,岂不长彼贪残,生其舞弄?事涉货财,宁不尤切?今可改不枉法受财者,科同正盗。''六月庚辰,以镇前将军豫章王叔英为镇南将军、江州刺史。丙戌,以征南将军、江州刺史鄱阳王伯山为中权将军、护军将军。秋七月辛卯,初用大货六铢钱。八月甲子,青州义主硃显宗等率所领七百户入附。丁卯,舆驾幸大壮观阅武。戊寅,舆驾还宫。冬十月甲戌,以安前将军、祠部尚书晋安王伯恭为军师将军,尚书仆射陆缮为尚书左仆射。十一月辛卯,诏曰:``画冠弗犯,革此浇风,孥戮是蹈,化于薄俗。朕肃膺宝命,迄将一纪,思经邦济治,忧国爱民,日仄劬劳,夜分辍寝,而还淳反朴,其道靡阶,雍熙盛美,莫云能致。遂乃鞫讯之牒,盈于听览,舂釱之人,烦于牢犴。周成刑措,汉文断狱,杼轴空劳,邈焉既远。加以蕞尔丑徒,轶我彭、汴,淮、汝氓庶,企踵王略,治兵誓旅,义存拯救。飞刍挽粟,征赋颇烦,暑雨祁寒,宁忘咨怨。兼宿度乖舛,次舍违方,若曰之诚,责归元首,愧心斯积,驭朽非惧。即建子令月,微阳初动,应此嘉辰,宜播宽泽,可大赦天下。''甲午,周遣柱国梁士彦率众至肥口。戊戌,周军进围寿阳。辛丑,以车骑将军、开府仪同三司、南兗州刺史淳于量为上流水军都督;中领军樊毅都督北讨诸军事,加安北将军;散骑常侍、左卫将军任忠都督北讨前军事,加平北将军;前豊州刺史皋文奏率步骑三千趣阳平郡。癸卯,任忠率步骑七千趣秦郡。丙午,新除仁威将军、右卫将军鲁广达率众入淮。是日,樊毅领水军二万自东关入焦湖,武毅将军萧摩诃率步骑趣历阳。戊申,豫州陷。辛亥,霍州又陷。癸丑,以新除中卫大将军、扬州刺史始兴王叔陵为大都督,总督水步众军。十二月乙丑,南北兗、晋三州,及盱眙、山阳、阳平、马头、秦、历阳、沛、北谯、南梁等九州,并自拔还京师。谯、北徐州又陷。自是淮南之地尽没于周矣。己巳,诏曰:``昔尧、舜在上,茅屋土阶,汤、禹为君,藜杖韦带。至如甲帐珠络,华榱璧珰,未能雍熙,徒闻侈欲。朕企仰前圣,思求讼平,正道多违,浇风靡乂。至今贵里豪家,金铺玉舄,贫居陋巷,彘食牛衣,称物平施,何其辽远。爟烽未息,役赋兼劳,文吏奸贪,妄动科格。重以旗亭关市,税敛繁多,不广都内之钱,非供水衡之费,逼遏商贾,营谋私蓄。靖怀众弊,宜事改张。弗弘王道,安拯民蠹?今可宣勒主衣、尚方诸堂署等,自非军国资须,不得缮造众物。后宫僚列,若有游长,掖庭启奏,即皆量遣。大予秘戏,非会礼经,乐府倡优,不合雅正,并可删改。市估津税,军令国章,更须详定,唯务平允。别观离宫,郊间野外,非恒飨宴,勿复修治。并勒内外文武车马宅舍,皆循俭约,勿尚奢华。违我严规,抑有刑宪。所由具为条格,标榜宣示,令喻朕心焉。''癸酉,遣平北将军沈恪、电威将军裴子烈镇南徐州,开远将军徐道奴镇栅口,前信州刺史杨宝安镇白下。戊寅,以中领军樊毅为镇西将军、都督荆郢巴武四州水陆诸军事。

十二年春正月戊戌,以散骑常侍、左卫将军任忠为平南将军、南豫州刺史,督缘江军防事。三月壬辰,以平北将军庐陵王伯仁为翊左将军、中领军。夏四月癸亥,尚书左仆射陆缮卒。乙丑,以宣毅将军河东王叔献为南徐州刺史。己卯,大雩。壬午,雨。五月癸巳,以军师将军、尚书右仆射晋安王伯恭为尚书仆射。六月壬戌,大风坏皋门中闼。秋八月己未,周使持节、上柱国、郧州总管荥阳郡公司马消难以郧、随、温、应、土、顺、沔、儇、岳等九州,鲁山、甑山、沌阳、应城、平靖、武阳、上明、涢水等八镇内附。诏以消难为使持节、侍中、大都督、总督安随等九州八镇诸军事、车骑将军、司空,封随郡公,给鼓吹、女乐各一部。庚申,诏镇西将军樊毅进督沔、汉诸军事。遣平南将军、南豫州刺史任忠率众趣历阳;通直散骑常侍、超武将军陈慧纪为前军都督,趣南兗州。戊辰,以新除司空司马消难为大都督水陆诸军事。庚午,通直散骑常侍淳于陵克临江郡。癸酉,智武将军鲁广达克郭默城。甲戌,大雨霖。丙子,淳于陵克祐州城。九月癸未,周临江太守刘显光率众内附。是夜,天东南有声,如风水相击,三夜乃止。丙戌,改安陆郡为南司州。丁亥,周将王延贵率众援历阳,任忠击破之,生擒延贵等。己酉,周广陵义主曹药率众入附。冬十月癸丑,大雨雹震。十一月己丑,诏曰:``朕君临四海,日旰劬劳,思弘至治,未臻斯道。而兵车骤出,军费尤烦,刍漕控引,不能征赋。夏中亢旱伤农,畿内为甚,民失所资,岁取无托。此则政刑未理,阴阳舛度,黎元阻饥,君孰与足?靖言兴念,余责在躬,宜布惠泽,溥沾氓庶。其丹阳、吴兴、晋陵、建兴、义兴、东海、信义、陈留、江陵等十郡,并诸署即年田税、禄秩,并各原半,其丁租半申至来岁秋登。''十二月庚辰,宣毅将军、南徐州刺史河东王叔献薨。

十三年春正月壬午,以车骑将军、开府仪同三司淳于量为左光禄大夫;中权将军、护军将军鄱阳王伯山即本号开府仪同三司;镇右将军、国子祭酒新安王伯固为扬州刺史;军师将军、尚书仆射晋安王伯恭为尚书左仆射;安右将军、丹阳尹徐陵为中书监,领太子詹事;吏部尚书袁宪为尚书右仆射。庚寅,以轻车将军、卫尉卿宜都王叔明为南徐州刺史。二月甲寅,诏赐司马消难所部周大将军田广等封爵各有差。乙亥,舆驾亲耕藉田。夏四月乙巳,分衡州始兴郡为东衡州,衡州为西衡州。五月丙辰,以前镇西将军樊毅为中护军。六月辛卯,以新除中护军樊毅为护军将军。秋九月癸亥,夜,大风至自西北,发屋拔树,大雷震雹。冬十月癸未,以散骑常侍、丹阳尹毛喜为吏部尚书,护军将军樊毅为镇西将军、荆州刺史。改鄱阳郡为吴州。壬寅,丹丹国遣使献方物。十二月辛巳,彗星见。己亥,以翊右将军、卫尉卿沈恪为护军将军。

十四年春正月己酉,高宗弗豫。甲寅,崩于宣福殿,时年五十三。遗诏曰:``朕爰自遘疾,曾未浃旬,医药不瘳,便属大渐,终始定分,夫复奚言。但君临寰宇,十有四载,诚则虽休勿休,日慎一日,知宗庙之负重,识王业之艰难。而边鄙多虞,生民未乂,方欲荡清四海,包吞八荒,有志莫从,遗恨幽壤。皇太子叔宝继体正嫡,年业韶茂,纂统洪基,社稷有主。群公卿士,文武内外,俱罄心力,同竭股肱,送往事居,尽忠诚之节,当官奉职,引翼亮之功。务在叶和,无违朕意。凡厥终制,事从省约。金银之饰,不须入圹,明器之具,皆令用瓦。唯使俭而合礼,勿得奢而乖度。以日易月,既有通规,公除之制,悉依旧准。在位百司,三日一临,四方州镇,五等诸侯,各守所职,并停奔赴。''二月辛卯,上谥孝宣皇帝,庙号高宗。癸巳,葬显宁陵。

高宗在田之日,有大度干略,及乎登庸,实允天人之望。梁室丧乱,淮南地并入齐,高宗太建初,志复旧境,乃运神略,授律出师,至于战胜攻取,献捷相继,遂获反侵地,功实懋焉。及周灭齐,乘胜略地,还达江际矣。

史臣曰:高宗器度弘厚,亦有人君之量焉。世祖知冢嗣仁弱,弗可传于宝位,高宗地居姬旦,世祖情存太伯,及乎弗悆,大事咸委焉。至于纂业,万机平理,命将出师,克淮南之地,开拓土宇,静谧封疆。享国十馀年,志大意逸,吕梁覆军,大丧师徒矣。江左削弱,抑此之由。呜呼!盖德不逮文,智不及武,虽得失自我,无御敌之略焉。

\hypertarget{header-n4251}{%
\subsubsection{卷六}\label{header-n4251}}

后主

后主,讳叔宝,字元秀,小字黄奴,高宗嫡长子也。梁承圣二年十一月戊寅生于江陵。明年,江陵陷,高宗迁关右,留后主于穰城。天嘉三年,归京师,立为安成王世子。天康元年,授宁远将军,置佐史。光大二年,为太子中庶子,寻迁侍中,馀如故。太建元年正月甲午,立为皇太子。

十四年正月甲寅,高宗崩。乙卯,始兴王叔陵作逆,伏诛。丁巳,太子即皇帝位于太极前殿。诏曰:``上天降祸,大行皇帝奄弃万国,攀号擗踊,无所迨及。朕以哀茕,嗣膺宝历,若涉巨川,罔知攸济,方赖群公,用匡寡薄。思播遗德,覃被亿兆,凡厥遐迩,咸与惟新。可大赦天下。在位文武及孝悌力田为父后者,并赐爵一级。孤老鳏寡不能自存者,赐谷人五斛、帛二匹。''癸亥,以侍中、翊前将军、丹阳尹长沙王叔坚为骠骑将军、开府仪同三司、扬州刺史,右卫将军萧摩诃为车骑将军、南徐州刺史,镇西将军、荆州刺史樊毅进号征西将军,平南将军、豫州刺史任忠进号镇南将军,护军将军沈恪为特进、金紫光禄大夫,平西将军鲁广达进号安西将军,仁武将军、豊州刺史章大宝为中护军。乙丑,尊皇后为皇太后,宫曰弘范。丙寅,以冠军将军晋熙王叔文为宣惠将军、丹阳尹。丁卯,立弟叔重为始兴王,奉昭烈王祀,己巳,立妃沈氏为皇后。辛未,立皇弟叔俨为寻阳王,皇弟叔慎为岳阳王,皇弟叔达为义阳王,皇弟叔熊为巴山王,皇弟叔虞为武昌王。壬申,侍中、中权将军、开府仪同三司鄱阳王伯山进号中权大将军,军师将军、尚书左仆射晋安王伯恭进号翊前将军、侍中,翊右将军、中领军庐陵王伯仁进号安前将军,镇南将军、江州刺史豫章王叔英进号征南将军,平南将军、湘州刺史建安王叔卿进号安南将军。以侍中、中书监、安右将军徐陵为左光禄大夫,领太子少傅。甲戌,设无珝大会于太极前殿。三月辛亥,诏曰:``躬推为劝,义显前经,力农见赏,事昭往诰。斯乃国储是资,民命攸属,豊俭隆替,靡不由之。夫入赋自古,输藁惟旧,沃饶贵于十金,硗确至于三易,腴脊既异,盈缩不同。诈伪日兴,簿书岁改。稻田使者,著自西京,不实峻刑,闻诸东汉。老农惧于祗应,俗吏因以侮文。辍耒成群,游手为伍,永言妨蠹,良可太息。今阳和在节,膏泽润下,宜展春耨,以望秋坻。其有新辟塍畎,进垦蒿莱,广袤勿得度量,征租悉皆停免。私业久废,咸许占作,公田荒纵,亦随肆勤。傥良守教耕,淳民载酒,有兹督课,议以赏擢。外可为格班下,称朕意焉。''癸亥,诏曰:``夫体国经野,长世字氓,虽因革傥殊,驰张或异,至于旁求俊乂,爰逮侧微,用适和羹,是隆大厦,上智中主,咸由此术。朕以寡薄,嗣膺景祚,虽哀疚在躬,情虑鋋舛,而宗社任重,黎庶务殷,无由自安拱默,敢忘康济,思所以登显髦彦,式备周行。但空劳宵梦,屡勤史卜,五就莫来,八能不至。是用申旦凝虑,丙夜损怀。岂以食玉炊桂,无因自达?将怀宝迷邦,咸思独善?应内外众官九品已上,可各荐一人,以会汇征之旨。且取备实难,举长或易,小大之用,明言所施,勿得南箕北斗,名而非实。其有负能仗气,摈压当时,著《宾戏》以自怜,草《客嘲》以慰志,人生一世,逢遇诚难,亦宜去此幽谷,翔兹天路,趋铜驼以观国,望金马而来庭,便当随彼方圆,饬之矩矱。''又诏曰:``昔睿后宰民,哲王御宇,虽德称汪濊,明能普烛,犹复纡己乞言,降情访道,高咨岳牧,下听舆台,故能政若神明,事无悔吝。朕纂承丕绪,思隆大业,常惧九重已邃,四聪未广,欲听昌言,不疲痺足,若逢廷折,无惮批鳞。而口柔之辞,傥闻于在位,腹诽之意,或隐于具僚,非所以弘理至公,缉熙帝载者也。内外卿士文武众司,若有智周政术,心练治体,救民俗之疾苦,辩禁网之疏密者,各进忠谠,无所隐讳。朕将虚己听受,择善而行,庶深鉴物情,匡我王度。''己巳,以侍中、尚书左仆射、新除翊前将军晋安王伯恭为安南将军、湘州刺史,新除翊左将军、永阳王伯智为尚书仆射,中护军章大宝为豊州刺史。夏四月丙申,立皇子永康公胤为皇太子,赐天下为父后者爵一级,王公已下赉帛各有差。庚子,诏曰:``朕临御区宇,抚育黔黎,方欲康济浇薄,蠲省繁费,奢僭乖衷,实宜防断。应镂金银薄及庶物化生土木人彩花之属,及布帛幅尺短狭轻疏者,并伤财废业,尤成蠹患。又僧尼道士,挟邪左道,不依经律,民间淫祀妖书诸珍怪事,详为条制,并皆禁绝。''癸卯,诏曰:``中岁克定淮、泗,爰涉青、徐,彼土酋豪,并输罄诚款,分遣亲戚,以为质任。今旧土沦陷,复成异域,南北阻远,未得会同,念其分乖,殊有爱恋。夷狄吾民,斯事一也,何独讥禁,使彼离析?外可即检任子馆及东馆并带保任在外者,并赐衣粮,颁之酒食,遂其乡路,所之阻远,便发遣船仗卫送,必令安达。若已预仕宦及别有事义不欲去者,亦随其意。''六月癸酉朔,以明威将军、通直散骑常侍孙瑒为中护军。秋七月辛未,大赦天下。是月,江水色赤如血,自京师至于荆州。八月癸未夜,天有声如风水相击。乙酉夜亦如之。丙戌,以使持节、都督缘江诸军事、安西将军鲁广达为安左将军。九月丙午,设无珝大会于太极殿,舍身及乘舆御服,大赦天下。辛亥夜,天东北有声如虫飞,渐移西北。乙卯,太白昼见。丙寅,以骠骑将军、开府仪同三司、扬州刺史长沙王叔坚为司空,征南将军、江州刺史豫章王叔英即本号开府仪同三司。

至德元年春正月壬寅,诏曰:``朕以寡薄,嗣守鸿基,哀惸切虑,疹恙缠织,训俗少方,临下靡算,惧甚践冰,忄栗同驭朽。而四气易流,三光遄至,缨绂列陛,玉帛充庭,具物匪新,节序疑旧,缅思前德,永慕昔辰,对轩闼而哽心,顾枿筵而慓气。思所以仰遵遗构,俯励薄躬,陶铸九流,休息百姓,用弘宽简,取叶阳和。可大赦天下,改太建十五年为至德元年。''以征南将军、江州刺史、新除开府仪同三司豫章王叔英为中卫大将军,骠骑将军、开府仪同三司、扬州刺史长沙王叔坚为江州刺史,征东将军、开府仪同三司、东扬州刺史司马消难进号车骑将军,宣惠将军、丹阳尹晋熙王叔文为扬州刺史,镇南将军、南豫州刺史任忠为领军将军,安左将军鲁广达为平南将军、南豫州刺史,祠部尚书江总为吏部尚书。癸卯,立皇子深为始安王。二月丁丑,以始兴王叔重为扬州刺史。夏四月戊辰,交州刺史李幼荣献驯象。己丑,以前轻车将军、扬州刺史晋熙王叔文为江州刺史。秋八月丁卯,以骠骑将军、开府仪同三司长沙王叔坚为司空。九月丁巳,天东南有声如虫飞。冬十月丁酉,立皇弟叔平为湘东王,叔敖为临贺王,叔宣为阳山王,叔穆为西阳王。戊戌,侍中、安右将军、左光禄大夫、太子少傅徐陵卒。癸丑,立皇弟叔俭为南安王,叔澄为南郡王,叔兴为沅陵王,叔韶为岳山王,叔纯为新兴王。十二月丙辰,头和国遣使献方物。司空长沙王叔坚有罪免。戊午夜,天开自西北至东南,其内有青黄色,隆隆若雷声。

二年春正月丁卯,分遣大使巡省风俗。平南将军、豫州刺史鲁广达进号安南将军。癸巳,大赦天下。夏五月戊子,以尚书仆射永阳王伯智为平东将军、东扬州刺史,轻车将军、江州刺史晋熙王叔文为信威将军、湘州刺史,仁威将军、扬州刺史始兴王叔重为江州刺史,信武将军、南琅邪彭城二郡太守南平王嶷为扬州刺史,吏部尚书江总为尚书仆射。秋七月戊辰,以长沙王叔坚为侍中、镇左将军。壬午,太子加元服,在位文武赐帛各有差,孝悌力田为父后者各赐一级,鳏寡癃老不能自存者人谷五斛。九月癸未,太白昼见。冬十月己酉,诏曰:``耕凿自足,乃曰淳风,贡赋之兴,其来尚矣。盖由庚极务,不获已而行焉。但法令滋章,奸盗多有,俗尚浇诈,政鲜惟良。朕日旰夜分,矜一物之失所,泣辜罪己,愧三千之未措。望订初下,使强廕兼出,如闻贫富均起,单弱重弊,斯岂振穷扇曷之意欤?是乃下吏箕敛之苛也。故云`百姓不足,君孰与足'。自太建十四年望订租调逋未入者,并悉原除。在事百僚,辩断庶务,必去取平允,无得便公害民,为己声绩,妨紊政道。''十一月丙寅,大赦天下。壬申,盘盘国遣使献方物。戊寅,百济国遣使献方物。

三年春正月戊午朔,日有蚀之。庚午,以镇左将军长沙王叔坚即本号开府仪同三司,征西将军、荆州刺史樊毅为护军将军,守吏部尚书、领著作陆琼为吏部尚书,金紫光禄大夫袁敬加特进。三月辛酉,前豊州刺史章大宝举兵反。夏四月庚戌,豊州义军主陈景详斩大宝,传首京师。秋八月戊子夜,老人星见。己酉,以左民尚书谢伷为吏部尚书。九月甲戌,特进、金紫光禄大夫袁敬卒。冬十月己丑,丹丹国遣使献方物。十一月己未,诏曰:``宣尼诞膺上哲,体资至圣,祖述宪章之典,并天地而合德,乐正雅颂之奥,与日月而偕明,垂后昆之训范,开生民之耳目。梁季湮微,灵寝忘处,鞠为茂草,三十馀年,敬仰如在,永惟忾息。今《雅道》雍熙,《由庚》得所,断琴故履,零落不追,阅笥开书,无因循复。外可详之礼典,改筑旧庙,蕙房桂栋,咸使惟新,芳繁洁潦,以时飨奠。''辛巳,舆驾幸长干寺,大赦天下。十二月丙戌,太白昼见。辛卯,皇太子出太学,讲《孝经》,戊戌,讲毕。辛丑,释奠于先师,礼毕,设金石之乐,会宴王公卿士。癸卯,高丽国遣使献方物。是岁,萧岿死,子琮代立。

四年春正月甲寅,诏曰:``尧施谏鼓,禹拜昌言,求之异等,久著前徽,举以淹滞,复闻昔典,斯乃治道之深规,帝王之切务。朕以寡昧,丕承鸿绪,未明虚己,日旰兴怀,万机多紊,四聪弗达,思闻蹇谔,采其谋计。王公已下,各荐所知,旁询管库,爰及舆皁,一介有能,片言可用,朕亲加听览,伫于启沃。''中权大将军、开府仪同三司鄱阳王伯山进号镇卫将军,中卫大将军、开府仪同三司豫章王叔英进号骠骑大将军,镇左将军、开府仪同三司长沙王叔坚进号中军大将军,安南将军晋安王伯恭进号镇右将军,翊右将军宜都王叔明进号安右将军。二月丙戌,以镇右将军晋安王伯恭为特进。丙申,立皇弟叔谟为巴东王,叔显为临江王,叔坦为新会王,叔隆为新宁王。夏五月丁巳,立皇子庄为会稽王。秋九月甲午,舆驾幸玄武湖,肆舻舰阅武,宴群臣赋诗。戊戌,以镇卫将军、开府仪同三司鄱阳王伯山为东扬州刺史,智武将军岳阳王叔慎为丹阳尹。丁未,百济国遣使献方物。冬十月癸亥,尚书仆射江总为尚书令,吏部尚书谢伷为尚书仆射。十一月己卯,诏曰:``惟刑止暴,惟德成物,三才是资,百王不改。而世无抵角,时鲜犯鳞,渭桥惊马,弗闻廷争,桃林逸牛,未见其旨。虽剽悍轻侮,理从钳棨,蠢愚杜默,宜肆矜弘,政乏良哉,明惭则哲,求诸刑措,安可得乎?是用属寤寐以轸怀,负黼扆而於邑。复兹合璧轮缺,连珠纬舛,黄钟献吕,和气始萌,玄英告中,履长在御,因时宥过,抑乃斯得。可大赦天下。''

祯明元年春正月丙子,以安前将军衡阳王伯信进号镇前将军,安东将军、吴兴太守庐陵王伯仁为特进,智武将军、丹阳尹岳阳王叔慎为湘州刺史,仁武将军义阳王叔达为丹阳尹。戊寅,诏曰:``柏皇、大庭,鼓淳和于曩日,姬王、嬴后,被浇风于末载,刑书已铸,善化匪融,礼义既乖,奸宄斯作。何其淳朴不反,浮华竞扇者欤?朕居中御物,纳隍在眷,频恢天网,屡绝三边,元元黔庶,终罹五辟。盖乃康哉寡薄,抑焉法令滋章。是用当宁弗怡,矜此向隅之意。今三元具序,万国朝辰,灵芝献于始阳,膏露凝于聿岁,从春施令,仰乾布德,思与九有,惟新七政。可大赦天下,改至德五年为祯明元年。''乙未,地震。癸卯,以镇前将军衡阳王伯信为镇南将军、西衡州刺史。二月丁未,以特进、镇右将军晋安王伯恭进号中卫将军,中书令建安王叔卿为中书监。丁卯,诏至德元年望订租调逋未入者,并原之。秋八月癸卯,老人星见。丁未,以车骑将军萧摩诃为骠骑将军。九月乙亥,以骠骑将军、开府仪同三司豫章王叔英为骠骑大将军。庚寅,萧琮所署尚书令、太傅安平王萧岩,中军将军、荆州刺史义兴王萧献,遣其都官尚书沈君公,诣荆州刺史陈纪请降。辛卯,岩等率文武男女十万馀口济江。甲午,大赦天下。冬十一月乙亥,割扬州吴郡置吴州,割钱塘县为郡,属焉。丙子,以萧岩为平东将军、开府仪同三司、东扬州刺史,萧献为安东将军、吴州刺史。丁亥,以骠骑大将军、开府仪同三司豫章王叔英兼司徒。十二月丙辰,以前镇卫将军、开府仪同三司、东扬州刺史鄱阳王伯山为镇卫大将军、开府仪同三司,前中卫将军晋安王伯恭为中卫将军、右光禄大夫。

二年春正月辛巳,立皇子恮为东阳王,恬为钱塘王。是月,遣散骑常侍周罗嵒帅兵屯峡口。夏四月戊申,有群鼠无数,自蔡洲岸入石头渡淮,至于青塘两岸,数日死,随流出江。戊午,以左民尚书蔡徵为吏部尚书。是月,郢州南浦水黑如墨。五月壬午,以安前将军庐陵王伯仁为特进。甲午,东冶铸铁,有物赤色如数斗,自天坠熔所,有声隆隆如雷,铁飞出墙外,烧民家。六月戊戌,扶南国遣使献方物。庚子,废皇太子胤为吴兴王,立军师将军、扬州刺史始安王深为皇太子。辛丑,平南将军、江州刺史南平王嶷进号镇南将军;忠武将军、南徐州刺史永嘉王彦进号安北将军;会稽王庄为翊前将军、扬州刺史;宣惠将军、尚书令江总进号中权将军;云麾将军、太子詹事袁宪为尚书仆射;尚书仆射谢伷为特进;宁远将军、新除吏部尚书蔡徵进号安右将军。甲辰,以安右将军鲁广达为中领军。丁巳,大风至自西北激涛水入石头城,淮渚暴益,漂没舟乘。冬十月己亥,立皇子蕃为吴郡王。辛丑,以度支尚书、领大著作姚察为吏部尚书。己酉,舆驾幸莫府山,大校猎。十一月丁卯,诏曰:``夫议狱缓刑,皇王之所垂范,胜残去杀,仁人之所用心。自画冠既息,刻吏斯起,法令滋章,手足无措。朕君临区宇,属当浇末,轻重之典,在政未康,小大之情,兴言多愧。眷兹狴犴,有轸哀矜,可克日于大政殿讯狱。''壬申,以镇南将军、江州刺史南平王嶷为征西将军、郢州刺史,安北将军、南徐州刺史永嘉王彦为安南将军、江州刺史,军师将军南海王虔为安北将军、南徐州刺史。丙子,立皇弟叔荣为新昌王,叔匡为太原王。是月,隋遣晋王广众军来伐,自巴、蜀、沔、汉下流至广陵,数十道俱入,缘江镇戍,相继奏闻。时新除湘州刺史施文庆、中书舍人沈客卿掌机密用事,并抑而不言,故无备御。

三年春正月乙丑朔,雾气四塞。是日,隋总管贺若弼自北道广陵济京口,总管韩擒虎趋横江,济采石,自南道将会弼军。丙寅,采石戍主徐子建驰启告变。丁卯,召公卿入议军旅。戊辰,内外戒严,以骠骑将军萧摩诃、护军将军樊毅、中领军鲁广达并为都督,遣南豫州刺史樊猛帅舟师出白下,散骑常侍皋文奏将兵镇南豫州。庚午,贺若弼攻陷南徐州。辛未,韩擒虎又陷南豫州,文奏败还。至是隋军南北道并进。后主遣骠骑大将军、司徒豫章王叔英屯朝堂,萧摩诃屯乐游苑,樊毅屯耆阇寺,鲁广达屯白土冈,忠武将军孔范屯宝田寺。己卯,镇东大将军任忠自吴兴入赴,仍屯硃雀门。辛巳,贺若弼进据钟山,顿白土冈之东南。甲申,后主遣众军与弼合战,众军败绩。弼乘胜至乐游苑,鲁广达犹督散兵力战,不能拒。弼进攻宫城,烧北掖门。是时韩擒虎率众自新林至于石子冈,任忠出降于擒虎,仍引擒虎经硃雀航趣宫城,自南掖门而入。于是城内文武百司皆遁出,唯尚书仆射袁宪在殿内。尚书令江总、吏部尚书姚察、度支尚书袁权、前度支尚书王瑗、侍中王宽在省中。后主闻兵至,从宫人十馀出后堂景阳殿,将自投于井。袁宪侍侧,苦谏不从,后阁舍人夏侯公韵又以身蔽井,后主与争久之,方得入焉。及夜,为隋军所执。丙戌,晋王广入据京城。三月己巳,后主与王公百司发自建鄴,入于长安。隋仁寿四年十一月壬子,薨于洛阳,时年五十二。追赠大将军,封长城县公,谥曰炀,葬河南洛阳之芒山。

史臣侍中郑国公魏徵曰:高祖拔起垅亩,有雄桀之姿。始佐下籓,奋英奇之略,弭节南海,职思静乱。援旗北迈,义在勤王,扫侯景于既成,拯梁室于已坠。天网绝而复续,国步屯而更康,百神有主,不失旧物。魏王之延汉鼎祚,宋武之反晋乘舆,懋绩鸿勋,无以尚也。于时内难未弭,外邻勍敌,王琳作梗于上流,周、齐摇荡于江、汉,畏首畏尾,若存若亡,此之不图,遽移天历,虽皇灵有眷,何其速也?然志度弘远,怀抱豁如,或取士于仇雠,或擢才于亡命,掩其受金之过,宥其吠尧之罪,委以心腹爪牙,咸能得其死力,故乃决机百胜,成此三分,方诸鼎峙之雄,足以无惭权、备矣。世祖天姿睿哲,清明在躬,早预经纶,知民疾苦,思择令典,庶几至治。德刑并用,戡济艰虞,群凶授首,强邻震慑。虽忠厚之化未能及远,恭俭之风足以垂训,若不尚明察,则守文之良主也。临川年长于成王,过微于太甲,宣帝有周公之亲,无伊尹之志,明避不复,桐宫遂往,欲加之罪,其无辞乎!高宗爰自在田,雅量宏廓,登庸御极,民归其厚,惠以使下,宽以容众。智勇争奋,师出有名,扬旆分麾,风行电扫,辟土千里,奄有淮、泗,战胜攻取之势,近古未之有也。既而君侈民劳,将骄卒堕,帑藏空竭,折衄师徒,于是秦人方强,遂窥兵于江上矣。李克以为吴之先亡,由乎数战数胜,数战则民疲,数胜则主骄,以骄主御疲民,未有不亡者也。信哉言乎!高宗始以宽大得人,终以骄侈致败,文、武之业,坠于兹矣。后主生深宫之中,长妇人之手,既属邦国殄瘁,不知稼穑艰难。初惧阽危,屡有哀矜之诏,后稍安集,复扇淫侈之风。宾礼诸公,唯寄情于文酒,昵近群小,皆委之以衡轴。谋谟所及,遂无骨鲠之臣,权要所在,莫匪侵渔之吏。政刑日紊,尸素盈朝,躭荒为长夜之饮,嬖宠同艳妻之孽。危亡弗恤,上下相蒙,众叛亲离,临机不寤,自投于井,冀以苟生,视其以此求全,抑亦民斯下矣。遐观列辟,纂武嗣兴,其始也皆欲齐明日月,合德天地,高视五帝,俯协三王,然而靡不有初,克终盖寡,其故何哉?并以中庸之才,怀可移之性,口存于仁义,心怵于嗜欲。仁义利物而道远,嗜欲遂性而便身。便身不可久违,道远难以固志。佞谄之伦,承颜候色,因其所好,以悦导之,若下坂以走丸,譬顺流而决壅。非夫感灵辰象,降生明德,孰能遗其所乐,而以百姓为心哉?此所以成、康、文、景千载而罕遇,癸、辛、幽、厉靡代而不有,毒被宗社,身婴戮辱,为天下笑,可不痛乎!古人有言,亡国之主,多有才艺,考之梁、陈及隋,信非虚论。然则不崇教义之本,偏尚淫丽之文,徒长浇伪之风,无救乱亡之祸矣。

史臣曰:后主昔在储宫,早标令德,及南面继业,实允天人之望矣。至于礼乐刑政,咸遵故典,加以深弘六艺,广辟四门,是以待诏之徒,争趋金马,稽古之秀,云集石渠。且梯山航海,朝贡者往往岁至矣。自魏正始、晋中朝以来,贵臣虽有识治者,皆以文学相处,罕关庶务,朝章大典,方参议焉。文案簿领,咸委小吏,浸以成俗,迄至于陈。后主因循,未遑改革,故施文庆、沈客卿之徒,专掌军国要务,奸黠左道,以裒刻为功,自取身荣,不存国计。是以朝经堕废,祸生邻国。斯亦运钟百六,鼎玉迁变,非唯人事不昌,盖天意然也。

\hypertarget{header-n4266}{%
\subsection{列传}\label{header-n4266}}

\hypertarget{header-n4267}{%
\subsubsection{卷一}\label{header-n4267}}

高祖章皇后世祖沈皇后废帝王皇后高宗柳皇后后主沈皇后张贵妃

周礼,王者立后,六宫,三夫人,九嫔,二十七世妇,八十一御妻,以听天下之内治。然受命继体之主,非独外相佐也,盖亦有内德助焉。汉魏已来,六宫之职,因袭增置,代不同矣。高祖承微接乱,光膺天历,以朴素自处,故后宫员位多阙。世祖天嘉初,诏立后宫员数,始置贵妃、贵嫔、贵姬三人,以拟古之三夫人。又置淑媛、淑仪、淑容、昭华、昭容、昭仪、修华、修仪、修容九人,以拟古之九嫔。又置婕妤、容华、充华、承徽、列荣五人,谓之五职,亚于九嫔。又置美人、才人、良人三职,其职无员数,号为散位。世祖性恭俭,而嫔嫱多阙,高宗、后主内职无所改作。今之所缀,略备此篇。

高祖宣皇后章氏,讳要儿,吴兴乌程人也。本姓钮,父景明为章氏所养,因改焉。景明,梁代官至散骑侍郎。后母苏,尝遇道士以小龟遗己,光采五色,曰:``三年有徵。''及期后生,而紫光照室,因失龟所在。少聪慧,美容仪,手爪长五寸,色并红白,每有期功之服,则一爪先折。高祖先娶同郡钱仲方女,早卒,後乃聘后。后善书计,能诵《诗》及《楚辞》。

高祖自广州南征交止,命后与衡阳王昌随世祖由海道归于长城。侯景之乱,高祖下至豫章,后为景所囚。景平,而高祖为长城县公,后拜夫人。及高祖践祚,永定元年立为皇后。追赠后父景明特进、金紫光禄大夫,加金章紫绶,拜后母苏安吉县君。二年,安吉君卒,与后父合葬吴兴。明年,追封后父为广德县侯,邑五百户,谥曰温。高祖崩,后与中书舍人蔡景历定计,秘不发丧,召世祖入纂,事在蔡景历及侯安都传。世祖即位,尊后为皇太后,宫曰慈训。废帝即位,尊后为太皇太后。光大二年,后下令黜废帝为临海王,命高宗嗣位。太建元年,尊后为皇太后。二年三月丙申,崩于紫极殿,时年六十五。遗令丧事所须,并从俭约,诸有馈奠,不得用牲牢。其年四月,群臣上谥曰宣太后,祔葬万安陵。

后亲属无在朝者,唯族兄钮洽官至中散大夫。

世祖沈皇后,讳妙容,吴兴武康人也。父法深,梁安前中录事参军。后年十馀岁,以梁大同中归于世祖。高祖之讨侯景,世祖时在吴兴,景遣使收世祖及后。景平,乃获免。高祖践祚,永定元年,后为临川王妃。世祖即位,为皇后。追赠后父法深光禄大夫,加金章紫绶,封建城县侯,邑五百户,谥曰恭,追赠后母高绥安县君,谥曰定。废帝即位,尊后为皇太后,宫曰安德。

时高宗与仆射到仲举、舍人刘师知等并受遗辅政,师知与仲举恒居禁中参决众事,而高宗为扬州刺史,与左右三百人入居尚书省。师知见高宗权重,阴忌之,乃矫敕谓高宗曰:``今四方无事,王可还东府,经理州务。''高宗将出,而谘议毛喜止之曰:``今若出外,便受制于人,譬如曹爽,愿作富家翁不可得也。''高宗乃称疾,召师知留之与语,使毛喜先入言之于后。后曰:``今伯宗年幼,政事并委二郎,此非我意。''喜又言于废帝,帝曰:``此自师知等所为,非朕意也。''喜出以报高宗,高宗因囚师知,自入见后及帝,极陈师知之短,仍自草敕请画,以师知付廷尉治罪。其夜,于狱中赐死。自是政无大小,尽归高宗。后忧闷,计无所出,乃密赂宦者蒋裕,令诱建安人张安国,使据郡反,冀因此以图高宗。安国事觉,并为高宗所诛。时后左右近侍颇知其事,后恐连逮党与,并杀之。高宗即位,以后为文皇后。陈亡入隋,大业初,自长安归于江南,顷之,卒。

后兄钦,随世祖征伐,以功至贞威将军、安州刺史。世祖即位,袭爵建城侯,加通直散骑常侍、持节、会稽等九郡诸军事、明威将军、会稽太守,入为侍中、左卫将军、卫尉卿。光大中,为尚书右仆射,寻迁左仆射。钦素无技能,奉己而已。高宗即位,出为云麾将军、义兴太守,秩中二千石。太建元年卒,时年六十七,赠侍中、特进、翊左将军,谥曰成。子观嗣,颇有学识,官至御史中丞。

废帝王皇后,金紫光禄大夫固之女也。天嘉元年,为皇太子妃,废帝即位,立为皇后。废帝为临海王,后为临海王妃。至德中薨。

后生临海嗣王至泽。至泽以光大元年为皇太子。太建元年,袭封临海嗣王。寻为宣惠将军,置佐史。陈亡入长安。

高宗柳皇后,讳敬言,河东解人也。曾祖世隆,齐侍中、司空、尚书令、贞阳忠武公。祖恽,有重名于梁代,官至秘书监,赠侍中、中护军。父偃,尚梁武帝女长城公主,拜驸马都尉,大宝中,为鄱阳太守,卒官。后时年九岁,干理家事,有若成人。侯景之乱,后与弟盼往江陵依梁元帝,元帝以长城公主之故,待遇甚厚。及高宗赴江陵,元帝以后配焉。承圣二年,后生后主于江陵。明年,江陵陷,高宗迁于关右,后与后主俱留穰城。天嘉二年,与后主还朝,后为安成王妃。高宗即位,立为皇后。

后美姿容,身长七尺二寸,手垂过膝。初,高宗居乡里,先娶吴兴钱氏女,及即位,拜为贵妃,甚有宠,后倾心下之。每尚方供奉之物,其上者皆推于贵妃,而己御其次焉。高宗崩,始兴王叔陵为乱,后主赖后与乐安君吴氏救而获免,事在叔陵传。后主即位,尊后为皇太后,宫曰弘范。当是之时,新失淮南之地,隋师临江,又国遭大丧,后主病疮,不能听政,其诛叔陵、供大行丧事、边境防守及百司众务,虽假以后主之命,实皆决之于后。后主疮愈,乃归政焉。陈亡入长安,大业十一年薨于东都,年八十三,葬洛阳之邙山。

后性谦谨,未尝以宗族为请,虽衣食亦无所分遗。

弟盼,太建中尚世祖女富阳公主,拜驸马都尉。后主即位,以帝舅加散骑常侍。盼性愚戆,使酒,常因醉乘马入殿门,为有司所劾,坐免官,卒于家。赠侍中、中护军。

后从祖弟庄,清警有鉴识,太建末,为太子洗马,掌东宫管记。后主即位,稍迁至散骑常侍、卫尉卿。祯明元年,转右卫将军,兼中书舍人,领雍州大中正。自盼卒后,太后宗属唯庄为近,兼素有名望,犹是深被恩遇。寻迁度支尚书。陈亡入隋,为岐州司马。

后主沈皇后,讳婺华,仪同三司望蔡贞宪侯君理女也。母即高祖女会稽穆公主。主早亡,时后尚幼,而毁瘠过甚。及服毕,每至岁时朔望,恒独坐涕泣,哀动左右,内外咸敬异焉。太建三年,纳为皇太子妃。后主即位,立为皇后。

后性端静,寡嗜欲,聪敏强记,涉猎经史,工书翰。初,后主在东宫,而后父君理卒,后居忧,处于别殿,哀毁逾礼。后主遇后既薄,而张贵妃宠倾后宫,后宫之政并归之,后澹然未尝有所忌怨。而居处俭约,衣服无锦绣之饰,左右近侍才百许人,唯寻阅图史、诵佛经为事。陈亡,与后主俱入长安。及后主薨,后自为哀辞,文甚酸切。隋炀帝每所巡幸,恒令从驾。及炀帝为宇文化及所害,后自广陵过江还乡里,不知所终。

后无子,养孙姬子胤为己子。后宗族多有显官,事在君理传。

后叔君公,自梁元帝败后,常在江陵。祯明中,与萧献、萧岩率众叛隋归朝,后主擢为太子詹事。君公博学有才辩,善谈论,后主深器之。陈亡,隋文帝以其叛己,命斩于建康。

后主张贵妃,名丽华,兵家女也。家贫,父兄以织席为事。后主为太子,以选入宫。是时龚贵嫔为良娣,贵妃年十岁,为之给使,后主见而说焉,因得幸,遂有娠,生太子深。后主即位,拜为贵妃。性聪惠,甚被宠遇。后主每引贵妃与宾客游宴,贵妃荐诸宫女预焉,后宫等咸德之,兢言贵妃之善,由是爱倾后宫。又好厌魅之术,假鬼道以惑后主,置淫祀于宫中,聚诸妖巫使之鼓舞。因参访外事,人间有一言一事,妃必先知之,以白后主。由是益重妃,内外宗族,多被引用。及隋军陷台城,妃与后主俱入于井,隋军出之,晋王广命斩贵妃,榜于青溪中桥。

史臣侍中郑国公魏徵考览记书,参详故老,云:后主初即位,以始兴王叔陵之乱,被伤卧于承香阁下,时诸姬并不得进,唯张贵妃侍焉。而柳太后犹居柏梁殿,即皇后之正殿也。后主沈皇后素无宠,不得侍疾,别居求贤殿。至德二年,乃于光照殿前起临春、结绮、望仙三阁。阁高数丈,并数十间,其窗牖、壁带、悬楣、栏槛之类,并以沈檀香木为之,又饰以金玉,间以珠翠,外施珠廉,内有宝床、宝帐、其服玩之属,瑰奇珍丽,近古所未有。每微风暂至,香闻数里,朝日初照,光映后庭。其下积石为山,引水为池,植以奇树,杂以花药。后主自居临春阁,张贵妃居结绮阁,龚、孔二贵嫔居望仙阁,并复道交相往来。又有王、李二美人、张、薛二淑媛、袁昭仪、何婕妤、江修容等七人,并有宠,递代以游其上。以宫人有文学者袁大舍等为女学士。后主每引宾客对贵妃等游宴,则使诸贵人及女学士与狎客共赋新诗,互相赠答,采其尤艳丽者以为曲词,被以新声,选宫女有容色者以千百数,令习而歌之,分部迭进,持以相乐。其曲有《玉树后庭花》、《临春乐》等,大指所归,皆美张贵妃、孔贵嫔之容色也。其略曰:``璧月夜夜满,琼树朝朝新。''而张贵妃发长七尺,鬒黑如漆,其光可鉴。特聪惠,有神采,进止闲暇,容色端丽。每瞻视盼睐,光采溢目,照映左右。常于阁上靓妆,临于轩槛,宫中遥望,飘若神仙。才辩强记,善候人主颜色。是时后主怠于政事,百司启奏,并因宦者蔡脱儿、李善度进请,后主置张贵妃于膝上共决之。李、蔡所不能记者,贵妃并为条疏,无所遗脱。由是益加宠异,冠绝后庭。而后宫之家,不遵法度,有挂于理者,但求哀于贵妃,贵妃则令李、蔡先启其事,而后从容为言之。大臣有不从者,亦因而谮之,所言无不听。于是张、孔之势,薰灼四方,大臣执政,亦从风而靡。阉宦便佞之徒,内外交结,转相引进,贿赂公行,赏罚无常,纲纪瞀乱矣。

史臣曰:《诗》表《关雎》之德,《易》著《乾坤》之基,然夫妇之际,人道之大伦也。若夫作俪天则,燮赞王化,则宣太后有其懿焉。

\hypertarget{header-n4292}{%
\subsubsection{卷二}\label{header-n4292}}

杜僧明周文育子宝安侯安都

杜僧明,字弘照,广陵临泽人也。形貌眇小,而胆气过人,有勇力,善骑射。梁大同中,卢安兴为广州南江督护,僧明与兄天合及周文育并为安兴所启,请与俱行。频征俚獠有功,为新州助防。天合亦有材干,预在征伐。安兴死,僧明复副其子子雄。及交州土豪李贲反,逐刺史萧谘,谘奔广州,台遣子雄与高州刺史孙冏讨贲。时春草已生,瘴疠方起,子雄请待秋讨之,广州刺史新渝侯萧映不听,萧谘又促之,子雄等不得已,遂行。至合浦,死者十六七,众并惮役溃散,禁之不可,乃引其馀兵退还。萧谘启子雄及冏与贼交通,逗留不进,梁武帝敕于广州赐死。子雄弟子略、子烈并雄豪任侠,家属在南江。天合谋于众曰:``卢公累代待遇我等亦甚厚矣,今见枉而死,不能为报,非丈夫也。我弟僧明万人之敌,若围州城,召百姓,谁敢不从。城破,斩二侯祭孙、卢,然后待台使至,束手诣廷尉,死犹胜生。纵其不捷,亦无恨矣。''众咸慷慨曰:``是愿也,唯足下命之。''乃与周文育等率众结盟,奉子雄弟子略为主,以攻刺史萧映。子略顿城南,天合顿城北,僧明、文育分据东西,吏人并应之,一日之中,众至数万。高祖时在高要,闻事起,率众来讨,大破之,杀天合,生擒僧明及文育等,高祖并释之,引为主帅。

高祖征交止及讨元景仲,僧明、文育并有功。侯景之乱,俱随高祖入援京师。高祖于始兴破兰裕,僧明为前锋,擒裕斩之。又与蔡路养战于南野,僧明马被伤,高祖驰往救之,以所乘马授僧明,僧明乘马与数十人复进,众皆披靡,因而乘之,大败路养。高州刺史李迁仕又据大皋,入灨石,以逼高祖,高祖遣周文育为前军,与僧明击走之。迁仕与宁都人刘孝尚并力将袭南康,高祖又令僧明与文育等拒之,相持连战百馀日,卒擒迁仕,送于高祖军。及高祖下南康,留僧明顿西昌,督安成、庐陵二郡军事。元帝承制授假节、清野将军、新州刺史,临江县子,邑三百户。

侯景遣于庆等寇南江,高祖顿豫章,命僧明为前驱,所向克捷。高祖表僧明为长史,仍随东讨。军至蔡洲,僧明率麾下烧贼水门大舰。及景平,以功除员外散骑常侍、明威将军、南兗州刺史,进爵为侯,增邑并前五百户,仍领晋陵太守。承圣二年,从高祖北围广陵,加使持节,迁通直散骑常侍、平北将军、馀如故。荆州陷,高祖使僧明率吴明彻等随侯瑱西援,于江州病卒,时年四十六。赠散骑常侍,谥曰威。世祖即位,追赠开府仪同三司。天嘉二年,配享高祖庙庭。子晋嗣。

周文育,字景德,义兴阳羡人也。少孤贫,本居新安寿昌县,姓项氏,名猛奴。年十一,能反覆游水中数里,跳高五六尺,与群儿聚戏,众莫能及。义兴人周荟为寿昌浦口戍主,见而奇之,因召与语。文育对曰:``母老家贫,兄姊并长大,困于赋役。''荟哀之,乃随文育至家,就其母请文育养为己子,母遂与之。及荟秩满,与文育还都,见于太子詹事周舍,请制名字,舍因为立名文育,字景德。命兄子弘让教之书计。弘让善隶书,写蔡邕《劝学》及古诗以遗文育,文育不之省也,谓弘让曰:``谁能学此,取富贵但有大槊耳。''弘让壮之,教之骑射,文育大悦。

司州刺史陈庆之与荟同郡,素相善,启荟为前军军主。庆之使荟将五百人往新蔡悬瓠,慰劳白水蛮,蛮谋执荟以入魏,事觉,荟与文育拒之。时贼徒甚盛,一日之中战数十合,文育前锋陷阵,勇冠军中。荟于阵战死,文育驰取其尸,贼不敢逼。及夕,各引去。文育身被九创,创愈,辞请还葬,庆之壮其节,厚加灊遗而遣之。葬讫,会庐安兴为南江督护,启文育同行。累征俚獠,所在有功,除南海令。安兴死后,文育与杜僧明攻广州,为高祖所败,高祖赦之,语在僧明传。

后监州王劢以文育为长流,深被委任。劢被代,文育欲与劢俱下,至大庾岭,诣卜者,卜者曰:``君北下不过作令长,南入则为公侯。''文育曰:``足钱便可,谁望公侯。''卜人又曰:``君须臾当暴得银至二千两,若不见信,以此为验。''其夕,宿逆旅,有贾人求与文育博,文育胜之,得银二千两。旦日辞劢,劢问其故,文育以告,劢乃遣之。高祖在高要,闻其还也,大喜,遣人迎之,厚加赏赐,分麾下配焉。

高祖之讨侯景,文育与杜僧明为前军,克兰裕,援欧阳頠,皆有功。高祖破蔡路养于南野,文育为路养所围,四面数重,矢石雨下,所乘马死,文育右手搏战,左手解鞍,溃围而出,因与杜僧明等相得,并力复进,遂大败之。高祖乃表文育为府司马。

李迁仕之据大皋,遣其将杜平虏入灨石鱼梁作城,高祖命文育击之,平虏弃城走,文育据其城。迁仕闻平虏败,留老弱于大皋,悉选精兵自将,以攻文育,其锋甚锐,军人惮之。文育与战,迁仕稍却,相持未解,会高祖遣杜僧明来援,别破迁仕水军,迁仕众溃,不敢过大皋,直走新淦。梁元帝授文育假节、雄信将军、义州刺史。迁仕又与刘孝尚谋拒义军,高祖遣文育与侯安都、杜僧明、徐度、杜棱筑城于白口拒之。文育频出与战,遂擒迁仕。

高祖发自南康,遣文育将兵五千,开通江路。侯景将王伯丑据豫章,文育击走之,遂据其城。累前后功,除游骑将军、员外散骑常侍,封东迁县侯,邑五百户。

高祖军至白茅湾,命文育与杜僧明常为军锋,平南陵、鹊头诸城。及至姑熟,与景将侯子鉴战,破之。景平,授通直散骑常侍,改封南移县侯,邑一千户,拜信义太守。累迁南丹阳、兰陵、晋陵太守、智武将军、散骑常侍。

高祖诛王僧辩,命文育督众军会世祖于吴兴,围杜龛,克之。又济江袭会稽太守张彪,得其郡城。及世祖为彪所袭,文育时顿城北香岩寺,世祖夜往趋之,因共立栅。顷之,彪又来攻,文育悉力苦战,彪不能克,遂破平彪。

高祖以侯瑱拥据江州,命文育讨之,仍除都督南豫州诸军事、武威将军、南豫州刺史,率兵袭湓城。未克,徐嗣徽引齐寇渡江据芜湖,诏征文育还京。嗣徽等列舰于青墩,至于七矶,以断文育归路。及夕,文育鼓噪而发,嗣徽等不能制。至旦,反攻嗣徽,嗣徽骁将鲍砰独以小舰殿军,文育乘单舴艋与战,跳入舰,斩砰,仍牵其舰而还。贼众大骇,因留船芜湖,自丹阳步上。时高祖拒嗣徽于白城,适与文育大会。将战,风急,高祖曰:``兵不逆风。''文育曰:``事急矣,当决之,何用古法。''抽槊上马,驰而进,众军从之,风亦寻转,杀伤数百人。嗣微等移营莫府山,文育徙顿对之。频战功最,加平西将军,进爵寿昌县公,并给鼓吹一部。

广州刺史萧勃举兵逾岭,诏文育督众军讨之。时新吴洞主余孝顷举兵应勃,遣其弟孝劢守郡城,自出豫章,据于石头。勃使其子孜将兵与孝顷相会,又遣其别将欧阳騑顿军苦竹滩,傅泰据墌口城,以拒官军。官军船少,孝顷有舴艋三百艘、舰百馀乘在上牢,文育遣军主焦僧度、羊柬潜军袭之,悉取而归,仍于豫章立栅。时官军食尽,并欲退还,文育不许。乃使人间行遗周迪书,约为兄弟,并陈利害。迪得书甚喜,许馈粮饷。于是文育分遣老小乘故船舫,沿流俱下,烧豫章郡所立栅,伪退。孝顷望之,大喜,因不设备。文育由间道兼行,信宿达芊韶。芊韶上流则欧阳頠、萧勃,下流则傅泰、余孝顷,文育据其中间,筑城飨士,贼徒大骇。欧阳頠乃退入泥溪,作城自守。文育遣严威将军周铁虎与长史陆山才袭頠,擒之。于是盛陈兵甲,与頠乘舟而,以巡傅泰城下,因而攻泰,克之。萧勃在南康闻之,众皆股栗,莫能自固。其将谭世远斩勃欲降,为人所害。世远军主夏侯明彻持勃首以降。萧孜、余孝顷犹据石头,高祖遣侯安都助文育攻之,孜降文育,孝顷退走新吴,文州平,广育还顿豫章。以功授镇南将军、开府仪同三司、都督江广衡交等州诸军事、江州刺史。

王琳拥据上流,诏命侯安都为西道都督,文育为南道都督,同会武昌。与王琳战于沌口,为琳所执,后得逃归,语在安都传。寻授使持节、散骑常侍、镇南将军、开府仪同三司,寿昌县公,给鼓吹一部。

及周迪破余孝顷,孝顷子公飏、弟孝劢犹据旧栅,扇动南土,高祖复遣文育及周迪、黄法抃等讨之。豫章内史熊昙朗亦率军来会,众且万人。文育遣吴明彻为水军,配周迪运粮,自率众军入象牙江,城于金口。公飏领五百人伪降,谋执文育,事觉,文育囚之,送于京师,以其部曲分隶众军。乃舍舟为步军,进据三陂。王琳遣将曹庆帅兵二千人以救孝劢,庆分遣主帅常众爱与文育相拒,自帅所领径攻周迪、吴明彻军。迪等败绩,文育退据金口。熊昙朗因其失利,谋害文育,以应众爱。文育监军孙白象颇知其事,劝令先之。文育曰:``不可,我旧兵少,客军多,若取昙朗,人人惊惧,亡立至矣,不如推心以抚之。''初,周迪之败也,弃船走,莫知所在,及得迪书,文育喜,赍示昙朗,昙朗害之于座,时年五十一。高祖闻之,即日举哀,赠侍中、司空、谥曰忠愍。

初,文育之据三陂,有流星坠地,其声如雷,地陷方一丈,中有碎炭数斗。又军市中忽闻小儿啼,一市并惊,听之在土下,军人掘得棺长三尺,文育恶之。俄而迪败,文育见杀。天嘉二年,有诏配享高祖庙庭。子宝安嗣。文育本族兄景曜,因文育官至新安太守。

宝安字安民。年十馀岁,便习骑射,以贵公子骄蹇游逸,好狗马,乐驰骋,靡衣媮食。文育之为晋陵,以征讨不遑之郡,令宝安监知郡事,尤聚恶少年,高祖患之。及文育西征败绩,絷于王琳,宝安便折节读书,与士君子游,绥御文育士卒,甚有威惠。除员外散骑侍郎。文育归,复除贞威将军、吴兴太守。文育为熊昙朗所害,征宝安还。起为猛烈将军,领其旧兵,仍令南讨。

世祖即位,深器重之,寄以心膂,精卒利兵多配焉。及平王琳,颇有功。周迪之破熊昙朗,宝安南入,穷其馀烬。天嘉二年,重除雄信将军、吴兴太守,袭封寿昌县公。三年,征留异,为侯安都前军。异平,除给事黄门侍郎、卫尉卿。四年,授持节、都督南徐州诸军事、贞毅将军、南徐州刺史。征为左卫将军,加信武将军。寻以本官领卫尉卿,又进号仁威将军。天康元年卒,时年二十九。赠侍中、左卫将军,谥曰成。

子嗣。宝安卒后,亦为偏将。征欧阳纥,平定淮南,并有功,封江安县伯,邑四百户。历晋陵、定远二郡太守。太建九年卒,时年二十四,赠电威将军。

侯安都,字成师,始兴曲江人也。世为郡著姓。父文捍,少仕州郡,以忠谨称,安都贵后,官至光禄大夫、始兴内史,秩中二千石。

安都工隶书,能鼓琴,涉猎书传,为五言诗,亦颇清靡,兼善骑射,为邑里雄豪。梁始兴内史萧子范辟为主簿。侯景之乱,招集兵甲,至三千人。高祖入援京邑,安都引兵从高祖,攻蔡路养,破李迁仕,克平侯景,并力战有功。元帝授猛烈将军、通直散骑常侍,富川县子,邑三百户。随高祖镇京口,除兰陵太守。高祖谋袭王僧辩,诸将莫有知者,唯与安都定计,仍使安都率水军自京口趋石头,高祖自率马步从江乘罗落会之。安都至石头北,弃舟登岸,僧辩弗之觉也。石头城北接岗阜,雉堞不甚危峻,安都被甲带长刀,军人捧之投于女垣内,众随而入,进逼僧辩卧室。高祖大军亦至,与僧辩战于听事前,安都自内阁出,腹背击之,遂擒僧辩。

绍泰元年,以功授使持节、散骑常侍、都督南徐州诸军事、仁威将军、南徐州刺史。高祖东讨杜龛,安都留台居守。徐嗣徽、任约等引齐寇入据石头,游骑至于阙下。安都闭门偃旗帜,示之以弱,令城中曰:``登陴看贼者斩。''及夕,贼收军还石头,安都夜令士卒密营御敌之具。将旦,贼骑又至,安都率甲士三百人,开东西掖门与战,大败之,贼乃退还石头,不敢复逼台城。及高祖至,以安都为水军,于中流断贼粮运。又袭秦郡,破嗣徽栅,收其家口并马驴辎重。得嗣徽所弹琵琶及所养鹰,遣信饷之曰:``昨至弟住处得此,今以相还。''嗣徽等见之大惧,寻而请和,高祖听其还北。及嗣徽等济江,齐之馀军犹据采石,守备甚严,又遣安都攻之,多所俘获。

明年春,诏安都率兵镇梁山,以备齐。徐嗣徽等复入丹阳,至湖熟,高祖追安都还,率马步拒之于高桥。又战于耕坛南,安都率十二骑,突其阵,破之,生擒齐仪同乞伏无劳。又刺齐将东方老堕马,会贼骑至,救老获免。贼北渡蒋山,安都又与齐将王敬宝战于龙尾,使从弟晓、军主张纂前犯其阵。晓被枪坠马,张纂死之。安都驰往救晓,斩其骑士十一人,因取纂尸而还,齐军不敢逼。高祖与齐军战于莫府山,命安都领步骑千馀人,自白下横击其后,齐军大败。安都又率所部追至摄山,俘获首虏不可胜计。以功进爵为侯,增邑五百户,给鼓吹一部。又进号平南将军,改封西江县公。

仍都督水军出豫章,助豫州刺史周文育讨萧勃。安都未至,文育已斩勃,并擒其将欧阳頠、傅泰等。唯余孝顷与勃子孜犹据豫章之石头,作两城,孝顷与孜各据其一,又多设船舰,夹水而阵。安都至,乃衔枚夜烧其舰。文育率水军,安都领步骑,登岸结阵。孝顷俄断后路,安都乃令军士多伐松木,竖栅,列营渐进,频战屡克,孜乃降。孝顷奔归新吴,请入子为质,许之。师还,以功进号镇北将军,加开府仪同三司。

仍率众会于武昌,与周文育西讨王琳。将发,王公已下饯于新林,安都跃马渡桥,人马俱堕水中,又坐絺内坠于橹井,时以为不祥。至武昌,琳将樊猛弃城走。文育亦自豫章至。时两将俱行,不相统摄,因部下交争,稍不平。军至郢州,琳将潘纯陀于城中遥射官军,安都怒,进军围之,未能克。而王琳至于弇口,安都乃释郢州,悉众往沌口以御之,遇风不得进。琳据东岸,官军据西岸,相持数日,乃合战,安都等败绩。安都与周文育、徐敬成并为琳所囚。琳总以一长锁系之,置于絺下,令所亲宦者王子晋掌视之。琳下至湓城白水浦,安都等甘言许厚赂子晋。子晋乃伪以小船依絺而钓,夜载安都、文育、敬成上岸,入深草中,步投官军。还都自劾,诏并赦之,复其官爵。

寻为丹阳尹,出为都督南豫州诸军事、镇西将军、南豫州刺史。令继周文育攻余孝劢及王琳将曹庆、常众爱等。安都自宫亭湖出松门,蹑众爱后。文育为熊昙朗所害,安都回取大舰,值琳将周炅、周协南归,与战,破之,生擒炅、协。孝劢弟孝猷率部下四千家欲就王琳,遇炅、协败,乃诣安都降。安都又进军于禽奇洲,破曹庆、常众爱等,焚其船舰。众爱奔于庐山,为村人所杀,馀众悉平。

还军至南皖,而高祖崩,安都随世祖还朝,仍与群臣定议,翼奉世祖。时世祖谦让弗敢当,太后又以衡阳王故,未肯下令,群臣犹豫不能决。安都曰:``今四方未定,何暇及远,临川王有功天下,须共立之。今日之事,后应者斩。''便按剑上殿,白太后出玺,又手解世祖发,推就丧次。世祖即位,迁司空,仍为都督南徐州诸军事、征北将军、南徐州刺史,给扶。

王琳下至栅口,大军出顿芜湖,时侯瑱为大都督,而指麾经略,多出安都。天嘉元年,增邑千户。及王琳败走入齐,安都进军湓城,讨琳馀党,所向皆下。

仍别奉中旨,迎衡阳献王昌。初,昌之将入也,致书于世祖,辞甚不逊,世祖不怿,乃召安都从容而言曰:``太子将至,须别求一蕃,吾其老焉。''安都对曰:``自古岂有被代天子?臣愚不敢奉诏。''因请自迎昌,昌济汉而薨。以功进爵清远郡公,邑四千户。自是威名甚重,群臣无出其右。

安都父文捍,为始兴内史,卒于官。世祖征安都还京师,为发丧。寻起复本官,赠其父散骑常侍、金紫光禄大夫,拜其母为清远国太夫人。仍迎还都,母固求停乡里,上乃下诏,改桂阳之汝城县为卢阳郡,分衡州之始兴、安远二郡,合三郡为东衡州,以安都从弟晓为刺史,安都第三子秘年九岁,上以为始兴内史,并令在乡侍养。其年,改封安都桂阳郡公。

王琳败后,周兵入据巴、湘,安都奉诏西讨。及留异拥据东阳,又奉诏东讨。异本谓台军由钱塘江而上,安都乃步由会稽之诸暨,出于永康。异大恐,奔桃枝岭,处岭谷间,于岩口坚栅,以拒王师。安都作连城攻异,躬自接战,为流矢所中,血流至踝,安都乘舆麾军,容止不变。因其山垅之势,迮而为堰。天嘉三年夏,潦,水涨满,安都引船入堰,起楼舰与异城等,放拍碎其楼雉。异与第二子忠臣脱身奔晋安,安都虏其妻子,尽收其人马甲仗,振旅而归。以功加侍中、征北大将军,增邑并前五千户,仍还本镇。其年,吏民诣阙表请立碑,颂美安都功绩,诏许之。

自王琳平后,安都勋庸转大,又自以功安社稷,渐用骄矜,数招聚文武之士,或射驭驰骋,或命以诗赋,第其高下,以差次赏赐之。文士则褚玠、马枢、阴铿、张正见、徐伯阳,刘删、祖孙登,武士则萧摩诃、裴子烈等,并为之宾客,斋内动至千人。部下将帅,多不遵法度,检问收摄,则奔归安都。世祖性严察,深衔之。安都弗之改,日益骄横。每有表启,封讫,有事未尽,乃开封自书之,云又启某事。及侍宴酒酣,或箕踞倾倚。尝陪乐游禊饮,乃白帝曰:``何如作临川王时?''帝不应。安都再三言之,帝曰:``此虽天命,抑亦明公之力。''宴讫,又启便借供帐水饰,将载妻妾于御堂欢会,世祖虽许其请,甚不怿。明日,安都坐于御坐,宾客居群臣位,称觞上寿。初,重云殿灾,安都率将士带甲入殿,帝甚恶之,自是阴为之备。又周迪之反,朝望当使安都讨之,帝乃使吴明彻讨迪,又频遣台使案问安都部下,检括亡叛,安都内不自安。三年冬,遣其别驾周弘实自托于舍人蔡景历,并问省中事。景历录其状具奏之,希旨称安都谋反。世祖虑其不受制,明年春,乃除安都为都督江吴二州诸军事、征南大将军、江州刺史。自京口还都,部伍入于石头,世祖引安都宴于嘉德殿,又集其部下将帅会于尚书朝堂,于坐收安都,囚于嘉德西省,又收其将帅,尽夺马仗而释之。因出舍人蔡景历表以示于朝。乃诏曰:``昔汉厚功臣,韩、彭肇乱,晋倚蕃牧,敦、约称兵。托六尺于庞萌,野心窃发;寄股肱于霍禹,凶谋潜构。追惟往代,挻逆一揆,永言自古,患难同规。侯安都素乏遥图,本惭令德,幸属兴运,预奉经纶,拔迹行间,假之毛羽,推于偏帅,委以驰逐。位极三槐,任居四狱,名器隆赫,礼数莫俦。而志唯矜己,气在陵上,招聚逋逃,穷极轻狡,无赖无行,不畏不恭。受脤专征,剽掠一逞,推毂所镇,裒敛无厌。寄以徐蕃,接邻齐境,贸迁禁货,鬻卖居民,椎埋发掘,毒流泉壤,睚眦僵尸,罔顾彝宪。朕以爰初缔构,颇著功绩,飞骖代邸,预定嘉谋,所以淹抑有司,每怀遵养,杜绝百辟,日望自新。款襟期于话言,推丹赤于造次,策马甲第,羽林息警,置酒高堂,陛戟无卫。何尝内隐片嫌,去柏人而勿宿,外协猜防,入成皋而不留?而勃戾不悛,骄暴滋甚,招诱文武,密怀异图。去年十二月十一日,获中书舍人蔡景历启,称侯安都去月十日遣别驾周弘实来景历私省宿,访问禁中,具陈反计,朕犹加隐忍,待之如初。爰自北门,迁授南服,受命经停,奸谋益露。今者欲因初镇,将行不轨。此而可忍,孰不可容?赖社稷之灵,近侍诚悫,丑情彰暴,逆节显闻。外可详案旧典,速正刑书,止在同谋,馀无所问。''明日,于西省赐死,时年四十四。寻有诏,宥其妻子家口,葬以士礼,丧事所须,务加资给。

初,高祖在京城,尝与诸将宴,杜僧明、周文育、侯安都为寿,各称功伐。高祖曰:``卿等悉良将也,而并有所短。杜公志大而识暗,狎于下而骄于尊,矜其功不收其拙。周侯交不择人,而推心过差,居危履险,猜防不设。侯郎傲诞而无厌,轻佻而肆志。并非全身之道。''卒皆如其言。

安都长子敦,年十二,为员外散骑侍郎,天嘉二年堕马卒,追谥桂阳国愍世子。太建三年,高宗追封安都为陈集县侯,邑五百户,子亶为嗣。

安都从弟晓,累从安都征讨有功,官至员外散骑常侍、明威将军、东衡州刺史,怀化县侯,邑五百户。天嘉三年卒,年四十一。

史臣曰:杜僧明、周文育并树功业,成于兴运,颇、牧、韩、彭,足可连类矣。侯安都情异向时,权逾曩日,因之以侵暴,加之以纵诞,苟曰非夫逆乱,奚用免于亡灭!昔汉高醢之为赐,宋武拉于坐右,良有以而然也。

\hypertarget{header-n4332}{%
\subsubsection{卷三}\label{header-n4332}}

侯瑱欧阳頠子纥吴明彻裴子烈

侯瑱,字伯玉,巴西充国人也。父弘远,世为西蜀酋豪。蜀贼张文萼据白崖山,有众万人,梁益州刺史鄱阳王萧范命弘远讨之。弘远战死,瑱固请复仇,每战必先锋陷阵,遂斩文萼,由是知名。因事范,范委以将帅之任,山谷夷獠不宾附者,并遣瑱征之。累功授轻车府中兵参军、晋康太守。范为雍州刺史。瑱除超武将军、冯翊太守。范迁镇合肥,瑱又随之。

侯景围台城,范乃遣瑱辅其世子嗣,入援京邑。京城陷,瑱与嗣退还合肥,仍随范徙镇湓城。俄而范及嗣皆卒,瑱领其众,依于豫章太守庄铁。铁疑之,瑱惧不自安,诈引铁谋事,因而刃之,据有豫章之地。侯景将于庆南略地至豫章,城邑皆下,瑱穷蹙,乃降于庆。庆送瑱于景,景以瑱与己同姓,托为宗族,待之甚厚,留其妻子及弟为质。遣瑱随庆平定蠡南诸郡。及景败于巴陵,景将宋子仙、任约等并为西军所获,瑱乃诛景党与,以应义军,景亦尽诛其弟及妻子。梁元帝授瑱武臣将军、南兗州刺史,郫县侯,邑一千户。仍随都督王僧辩讨景,恒为前锋,每战却敌。既复台城,景奔吴郡,僧辩使瑱率兵追之,与景战于吴松江,大败景,尽获其军实。进兵钱塘,景将谢答仁、吕子荣等皆降。以功除南豫州刺史,镇于姑熟。

承圣二年,齐遣郭元建出自濡须,僧辩遣瑱领甲士三千,筑垒于东关以捍之,大败元建。除使持节、镇北将军,给鼓吹一部,增邑二千户。西魏来寇荆州,王僧辩以瑱为前军,赴援,未至而荆州陷。瑱之九江,因卫晋安王还都。承制以瑱为侍中、使持节、都督江晋吴齐四州诸军事、江州刺史,改封康乐县公,邑五千户,进号车骑将军。司徒陆法和据郢州,引齐兵来寇,乃使瑱都督众军西讨,未至,法和率其部北度入齐。齐遣慕容恃德镇于夏首,瑱控引西还,水陆攻之,恃德食尽,请和,瑱还镇豫章。僧辩使其弟僧忄音率兵与瑱共讨萧勃,及高祖诛僧辩,僧忄音阴欲图瑱而夺其军,瑱知之,尽收僧忄音徒党,僧忄音奔齐。

绍泰二年,以本号加开府仪同三司,馀并如故。是时,瑱据中流,兵甚强盛,又以本事王僧辩,虽外示臣节,未有入朝意。初,余孝顷为豫章太守,及瑱镇豫章,乃于新吴县别立城栅,与瑱相拒。瑱留军人妻子于豫章,令从弟奫知后事,悉众以攻孝顷。自夏及冬,弗能克,乃长围守之,尽收其禾稼。奫与其部下侯方儿不协,方儿怒,率所部攻奫,虏掠瑱军府妓妾金玉,归于高祖。瑱既失根本,兵众皆溃,轻归豫章,豫章人拒之,乃趋湓城,投其将焦僧度。僧度劝瑱投齐,瑱以高祖有大量,必能容己,乃诣阙请罪,高祖复其爵位。

永定元年,授侍中、车骑将军。二年,进位司空。王琳至于沌口,周文育、侯安都并没,乃以瑱为都督西讨诸军事。瑱至于梁山。世祖即位,进授太尉,增邑千户。王琳至于栅口,又以瑱为都督,侯安都等并隶焉。瑱与琳相持百馀日,未决。天嘉元年二月,东关春水稍长,舟舰得通,琳引合肥漅湖之众,舳舻相次而下,其势甚盛。瑱率军进兽槛洲,琳亦出船列于江西,隔洲而泊。明日合战,琳军少却,退保西岸。及夕,东北风大起,吹其舟舰,舟舰并坏,没于沙中,溺死者数十百人。浪大不得还浦,夜中又有流星坠于贼营。及旦风静,琳入浦治船,以荻船塞于浦口,又以鹿角绕岸,不敢复出。是时,西魏遣大将军史宁蹑其上流,瑱闻之,知琳不能持久,收军却据湖浦,以待其敝。及史宁至,围郢州,琳恐众溃,乃率船舰来下,去芜湖十里而泊,击柝闻于军中。明日,齐人遣兵数万助琳,琳引众向梁山,欲越官军以屯险要。齐仪同刘伯球率兵万馀人助琳水战,行台慕容恃德子子会领铁骑二千,在芜湖西岸博望山南,为其声势。瑱令军中晨炊蓐食,分搥荡顿芜湖洲尾以待之。将战,有微风至自东南,众军施拍纵火。定州刺史章昭达乘平虏大舰,中江而进,发拍中于贼舰,其馀冒突、青龙,各相当值。又以牛皮冒蒙冲小船,以触贼舰,并熔铁洒之。琳军大败。其步兵在西岸者,自相蹂践,马骑并淖于芦荻中,弃马脱走以免者十二三。尽获其舟舰器械,并禽齐将刘伯球、慕容子会,自馀俘馘以万计。琳与其党潘纯陀等乘单舴艋冒阵走至湓城,犹欲收合离散,众无附者,乃与妻妾左右十馀人入齐。

其年,诏以瑱为都督湘、巴、郢、江、吴等五州诸军事,镇湓城。周将贺若敦、独孤盛等寇巴、湘,又以瑱为西讨都督,与盛战于西江口,大败盛军,虏其人马器械,不可胜数。以功授使持节、都督湘、桂、郢、巴、武、沅六州诸军事、湘州刺史,改封零陵郡公,邑七千户,馀如故。二年,以疾表求还朝。三月,于道薨,时年五十二。赠侍中、骠骑大将军、大司马,加羽葆、鼓吹、班剑二十人,给东园秘器,谥曰壮肃。其年九月,配享高祖庙庭。子净藏嗣。

净藏尚世祖第二女富阳公主,以公主除员外散骑侍郎。太建三年卒,赠司徒主簿。净藏无子,弟就袭封。

欧阳頠,字靖世,长沙临湘人也。为郡豪族。祖景达,梁代为本州治中。父僧宝,屯骑校尉。頠少质直有思理,以言行笃信著闻于岭表。父丧毁瘠甚至。家产累积,悉让诸兄。州郡频辟不应,乃庐于麓山寺傍,专精习业,博通经史。年三十,其兄逼令从宦,起家信武府中兵参军,迁平西邵陵王中兵参军事。

梁左卫将军兰钦之少也,与頠相善,故頠常随钦征讨。钦为衡州,仍除清远太守。钦南征夷獠,擒陈文彻,所获不可胜计,献大铜鼓,累代所无,頠预其功。还为直阁将军,仍除天门太守,伐蛮左有功。刺史庐陵王萧续深嘉之,引为宾客。钦征交州,复启頠同行。钦度岭以疾终,頠除临贺内史,启乞送钦丧还都,然后之任。时湘衡之界五十馀洞不宾,敕令衡州刺史韦粲讨之,粲委頠为都督,悉皆平殄。粲启梁武,称頠诚干,降诏褒赏,仍加超武将军,征讨广、衡二州山贼。

侯景构逆,粲自解还都征景,以頠监衡州。京城陷后,岭南互相吞并,兰钦弟前高州刺史裕攻始兴内史萧绍基,夺其郡。裕以兄钦与頠有旧,遣招之,頠不从。乃谓使云:``高州昆季隆显,莫非国恩,今应赴难援都,岂可自为跋扈。''及高祖入援京邑,将至始兴,頠乃深自结托。裕遣兵攻頠,高祖援之,裕败,高祖以王怀明为衡州刺史,迁頠为始兴内史。高祖之讨蔡路养、李迁仕也,頠率兵度岭,以助高祖。及路养等平,頠有功,梁元帝承制以始兴郡为东衡州,以頠为持节、通直散骑常侍、都督东衡州诸军事、云麾将军、东衡州刺史,新豊县伯,邑四百户。

侯景平,元帝遍问朝宰:``今天下始定,极须良才,卿各举所知。''群臣未有对者。帝曰:``吾已得一人。''侍中王褒进曰:``未审为谁?''帝云:``欧阳頠公正有匡济之才,恐萧广州不肯致之。''乃授武州刺史,寻授郢州刺史,欲令出岭,萧勃留之,不获拜命。寻授使持节、散骑常侍、都督衡州诸军事、忠武将军、衡州刺史,进封始兴县侯。

时萧勃在广州,兵强位重,元帝深患之,遣王琳代为刺史。琳已至小桂岭,勃遣其将孙信监州,尽率部下至始兴,避琳兵锋。頠别据一城,不往谒勃,闭门高垒,亦不拒战。勃怒,遣兵袭頠,尽收其此赀财马仗。寻赦之,还复其所,复与结盟。荆州陷,頠委质于勃。及勃度岭出南康,以頠为前军都督,顿豫章之苦竹滩,周文育击破之,擒送于高祖,高祖释之,深加接待。萧勃死后,岭南扰乱,頠有声南土,且与高祖有旧,乃授頠使持节、通直散骑常侍、都督衡州诸军事、安南将军、衡州刺史,始兴县侯。未至岭南,頠子纥已克定始兴。及頠至岭南,皆慑伏,仍进广州,尽有越地。改授都督广、交、越、成、定、明、新、高、合、罗、爱、建、德、宜、黄、利、安、石、双十九州诸军事、镇南将军、平越中郎将、广州刺史,持节、常侍、侯并如故。王琳据有中流,頠自海道及东岭奉使不绝。永定三年,进授散骑常侍,增都督衡州诸军事,即本号开府仪同三司。世祖嗣位,进号征南将军,改封阳山郡公,邑一千五百户,又给鼓吹一部。

初,交州刺史袁昙缓密以金五百两寄頠,令以百两还合浦太守龚翙,四百两付儿智矩,馀人弗之知也。頠寻为萧勃所破,赀财并尽,唯所寄金独在。昙缓亦寻卒,至是頠并依信还之,时人莫不叹伏。其重然诺如此。

时頠弟盛为交州刺史,次弟邃为衡州刺史,合门显贵,名振南土。又多致铜鼓、生口,献奉珍异,前后委积,颇有助于军国焉。頠以天嘉四年薨,时年六十六。赠侍中、车骑大将军、司空、广州刺史,谥曰穆。子纥嗣。

纥字奉圣,颇有干略。天嘉中,除黄门侍郎、员外散骑常侍。累迁安远将军、衡州刺史。袭封阳山郡公,都督交、广等十九州诸军事、广州刺史。在州十馀年,威惠著于百越,进号轻车将军。

光大中,上流蕃镇并多怀贰,高宗以纥久在南服,颇疑之。太建元年,下诏征纥为左卫将军。纥惧,未欲就征,其部下多劝之反,遂举兵攻衡州刺史钱道戢。道戢告变,乃遣仪同章昭达讨纥,屡战兵败,执送京师,伏诛,时年三十三。家口籍没。子询以年幼免。

吴明彻,字通昭,秦郡人也。祖景安,齐南谯太守。父树,梁右军将军。明彻幼孤,性至孝,年十四,感坟茔未备,家贫无以取给,乃勤力耕种。时天下亢旱,苗稼焦枯,明彻哀愤,每之田中,号泣,仰天自诉。居数日,有自田还者,云苗已更生。明彻疑之,谓为绐己,及往田所,竟如其言。秋而大获,足充葬用。时有伊氏者,善占墓,谓其兄曰:``君葬之日,必有乘白马逐鹿者来经坟所,此是最小孝子大贵之徵。''至时果有此应,明彻即树之最小子也。

起家梁东宫直后。及侯景寇京师,天下大乱,明彻有粟麦三千馀斛,而邻里饥餧,乃白诸兄曰:``当今草窃,人不图久,柰何有此而不与乡家共之?''于是计口平分,同其豊俭,群盗闻而避焉,赖以存者甚众。

及高祖镇京口,深相要结,明彻乃诣高祖,高祖为之降阶,执手即席,与论当世之务。明彻亦微涉书史经传,就汝南周弘正学天文、孤虚、遁甲,略通其妙,颇以英雄自许,高祖深奇之。

承圣三年,授戎昭将军、安州刺史。绍泰初,随周文育讨杜龛、张彪等。东道平,授使持节、散骑常侍、安东将军、南兗州刺史,封安吴县侯。高祖受禅,拜安南将军,仍与侯安都、周文育将兵讨王琳。及众军败没,明彻自拔还京。世祖即位,诏以本官加右卫将军。王琳败,授都督武沅二州诸军事、安西将军、武州刺史,馀并如故。周遣大将军贺若敦率马步万馀人奄至武陵,明彻众寡不敌,引军巴陵,仍破周别军于双林。

天嘉三年,授安西将军。及周迪反临川,诏以明彻为安南将军、江州刺史,领豫章太守,总督众军,以讨迪。明彻雅性刚直,统内不甚和,世祖闻之,遣安成王顼慰晓明彻,令以本号还朝。寻授镇前将军。五年,迁镇东将军、吴兴太守。及引辞之郡,世祖谓明彻曰:``吴兴虽郡,帝乡之重,故以相授。君其勉之!''及世祖弗豫,征拜中领军。

废帝即位,授领军将军,寻迁丹阳尹,仍诏明彻以甲仗四十人出入殿省。到仲举之矫令出高宗也,毛喜知其谋,高宗疑惧,遣喜与明彻筹焉。明彻谓喜曰:``嗣君谅闇,万机多阙,外邻强敌,内有大丧。殿下亲实周、邵,德冠伊、霍,社稷至重,愿留中深计,慎勿致疑。''

及湘州刺史华皎阴有异志,诏授明彻使持节、散骑常侍、都督湘、桂、武三州诸军事、安南将军、湘州刺史,给鼓吹一部,仍与征南大将军淳于量等率兵讨皎。皎平,授开府仪同三司,进爵为公。太建元年,授镇南将军。四年,征为侍中、镇前将军,馀并如故。

会朝议北伐,公卿互有异同,明彻决策请行。五年,诏加侍中、都督征讨诸军事,仍赐女乐一部。明彻总统众军十馀万,发自京师,缘江城镇,相续降款。军至秦郡,克其水栅。齐遣大将尉破胡将兵为援,明彻破走之,斩获不可胜计,秦郡乃降。高宗以秦郡明彻旧邑,诏具太牢,令拜祠上冢,文武羽仪甚盛,乡里以为荣。

进克仁州,授征北大将军,进爵南平郡公,增邑并前二千五百户。次平峡石岸二城。进逼寿阳,齐遣王琳将兵拒守。琳至,与刺史王贵显保其外郭。明彻以琳初入,众心未附,乘夜攻之,中宵而溃,齐兵退据相国城及金城。明彻令军中益修治攻具,又迮肥水以灌城。城中苦湿,多腹疾,手足皆肿,死者十六七。会齐遣大将军皮景和率兵数十万来援,去寿春三十里,顿军不进。诸将咸曰:``坚城未拔,大援在近,不审明公计将安出?''明彻曰:``兵贵在速,而彼结营不进,自挫其锋,吾知其不敢战明矣。''于是躬擐甲胄,四面疾攻,城中震恐,一鼓而克,生禽王琳、王贵显、扶风王可硃浑孝裕、尚书庐潜、左丞李騊駼,送京师。景和惶惧遁走,尽收其驼马辎重。琳之获也,其旧部曲多在军中,琳素得士卒心,见者皆歔欷不能仰视。明彻虑其有变,遣左右追杀琳,传其首。诏曰:``寿春者古之都会,襟带淮、汝,控引河、洛,得之者安,是称要害。侍中、使持节、都督征讨诸军事、征北大将军、开府仪同三司南平郡开国公明彻,雄图克举,宏略盖世。在昔屯夷,缔构皇业,乃掩衡、岳,用清氛沴,实吞云梦,即叙上游。今兹荡定,恢我王略,风行电扫,貔虎争驰,月阵云梯,金汤夺险,威陵殊俗,惠渐边氓。惟功与能,元戎是属,崇麾广赋,茂典恒宜,可都督、豫、合、建、光、朔、北徐六州诸军事、车骑大将军、豫州刺史,增封并前三千五百户,馀如故。''诏遣谒者萧淳风就寿阳册明彻,于城南设坛,士卒二十万,陈旗鼓戈甲,明彻登坛拜受,成礼而退,将卒莫不踊跃焉。

初,秦郡属南兗州,后隶谯州,至是,诏以谯之秦、盱眙、神农三郡还属南兗州,以明彻故也。

六年,自寿阳入朝,舆驾幸其第,赐钟磬一部,米一万斛,绢布二千匹。

七年,进攻彭城。军至吕梁,齐遣援兵前后至者数万,明彻又大破之。八年,进位司空,馀如故。又诏曰:``昔者军事建旌,交锋作鼓,顷日讹替,多乖旧章,至于行阵,不相甄别。今可给司空、大都督泬钺龙麾,其次将各有差。''寻授都督南北兗、南北青谯五州诸军事、南兗州刺史。

会周氏灭齐,高宗交事徐、兗,九年,诏明彻进军北伐,令其世子戎昭将军、员外散骑侍郎惠觉摄行州事。明彻军至吕梁,周徐州总管梁士彦率众拒战,明彻频破之,因退兵守城,不复敢出。明彻仍迮清水以灌其城,环列舟舰于城下,攻之甚急。周遣上大将军王轨将兵救之。轨轻行自清水入淮口,横流竖木,以铁锁贯车轮,遏断船路。诸将闻之,甚惶恐,议欲破堰拔军,以舫载马。马主裴子烈议曰:若决堰下船,船必倾倒,岂可得乎?不如前遣马出,于事为允。''适会明彻苦背疾甚笃,知事不济,遂从之,乃遣萧摩诃帅马军数千前还。明彻仍自决其堰,乘水势以退军,冀其获济。及至清口,水势渐微,舟舰并不得渡,众军皆溃,明彻穷蹙,乃就执。寻以忧愤遘疾,卒于长安,时年六十七。

至德元年诏曰:``李陵矢竭,不免请降,于禁水涨,犹且生获,固知用兵上术,世罕其人。故侍中、司空南平郡公明彻,爰初蹑足,迄届元戎,百战百胜之奇,决机决死之勇,斯亦侔于古焉。及拓定淮、肥,长驱彭、汴,覆勍寇如举毛,扫锐帅同沃雪,风威慴于异俗,功郊著于同文。方欲息驾阴山,解鞍浣海,既而师出已老,数亦终奇,不就结缨之功,无辞入褚之屈,望封崤之为易,冀平翟之非难,虽志在屈伸,而奄中霜露,埋恨绝域,甚可嗟伤。斯事已往,累逢肆赦,凡厥罪戾,皆蒙洒濯,独此孤魂,未沾宽惠,遂使爵土湮没,飨醊无主。弃瑕录用,宜在兹辰,可追封邵陵县开国侯,食邑一千户,以其息惠觉为嗣。''

惠觉历黄门侍郎,以平章大宝功,授豊州刺史。

明彻兄子超,字逸世。少倜傥,以干略知名。随明彻征伐,有战功,官至忠毅将军、散骑常侍、桂州刺史,封汝南县侯,邑一千户。卒,赠广州刺史,谥曰节。

裴子烈,字大士,河东闻喜人,梁员外散骑常侍猗之子。子烈少孤,有志气。遇梁末丧乱,因习武艺,以骁勇闻。频从明彻征讨,所向必先登陷阵。官至电威将军、北谯太守、岳阳内史,海安县伯,邑三百户。至德四年卒。

史臣曰:高祖拨乱创基,光启天历,侯瑱、欧阳頠并归身有道,位贵鼎司,美矣。吴明彻居将帅之任,初有军功,及吕梁败绩,为失算也。斯以勇非韩、白,识异孙、吴,遂使蹙境丧师,金陵虚弱,祯明沦覆,盖由其渐焉。

\hypertarget{header-n4370}{%
\subsubsection{卷四}\label{header-n4370}}

周铁虎程灵洗子文季

周铁虎,不知何许人也,梁世南渡。语音伧重,膂力过人,便马槊,事梁河东王萧誉,以勇敢闻,誉板为府中兵参军。誉为广州刺史,以铁虎为兴宁令。誉迁湘州,又为临蒸令。侯景之乱,元帝于荆州遣世子方等代誉,且以兵临之。誉拒战,大捷,方等死,铁虎功最,誉委遇甚重。及王僧辩讨誉,于阵获铁虎,僧辩命烹之,铁虎呼曰:``侯景未灭,柰何杀壮士!''僧辩奇其言,乃宥之,还其麾下。

及侯景西上,铁虎从僧辩克任约,获宋子仙,每战皆有功。元帝承制授仁威将军、潼州刺史,封沌阳县子,邑三百户。又从僧辩克定京邑,降谢答仁,平陆纳于湘州。承圣二年,以前后战功,进爵为侯,增邑并前五百户。仍为散骑常侍,领信义太守,将军如故。高祖诛僧辩,铁虎率所部降,因复其本职。

徐嗣徽引齐寇渡江,铁虎于板桥浦破其水军,尽获甲仗船舸。又攻历阳,袭齐寇步营,并皆克捷。嗣徽平,绍泰二年,迁散骑常侍、严威将军、太子左卫率。

寻随周文育于南江拒萧勃,恒为前军。文育又命铁虎偏军,于苦竹滩袭勃前军欧阳頠。又随文育西征王琳,于沌口败绩,铁虎与文育、侯安都并为琳所擒。琳引见诸将,与之语,唯铁虎辞气不屈,故琳尽宥文育之徒,独铁虎见害,时年四十九。高祖闻之,下诏曰:``天地之宝,所贵曰生,形魄之徒,所重唯命。至如捐生立节,效命酬恩,追远怀昔,信宜加等。散骑常侍、严威将军、太子左卫率、潼州刺史、领信义太守沌阳县开国侯铁虎,器局沈厚,风力勇壮,北讨南征,竭忠尽力。推锋江夏,致陷凶徒,神气弥雄,肆言无挠。岂直温序见害,方其理须,庞德临危,犹能瞋目。忠贞如此,恻怆兼深,可赠侍中、护军将军、青、冀二州刺史,加封一千户,并给鼓吹一部,侯如故。''天嘉五年,世祖又诏曰:``汉室功臣,形写宫观,魏朝猛将,名配宗祧,功烈所以长存,世代因之不朽。故侍中、护军将军、青、冀二州刺史沌阳县开国侯铁虎,诚节梗亮,力用雄敢,王业初基,行间累及,垂翅贼垒,正色寇庭,古之遗烈,有识同壮。陨身不屈,虽隆荣等,营魂易远,言追嘉惜。宜仰陪需寝,恭颁飨奠,可配食高祖庙庭。''子瑜嗣。

时有盱眙马明,字世朗,梁世事鄱阳嗣王萧范。侯景之乱,据庐江之东界,拒贼临城栅。元帝授散骑常侍、平北将军、北兗州刺史,领庐江太守。荆州陷没,归于高祖。绍泰中,复官位,封西华县侯,邑二千户。亦随文育西征王琳,于沌口军败,明力战死之,赠使持节、征西将军、郢州刺史。

程灵洗,字玄涤,新安海宁人也。少以勇力闻,步行日二百馀里,便骑善游。梁末,海宁、黟、歙等县及鄱阳、宣城郡界多盗贼,近县苦之。灵洗素为乡里所畏伏,前后守长恒使召募少年,逐捕劫盗。

侯景之乱,灵洗聚徒据黟、歙以拒景。景军据有新安,新安太守湘西乡侯萧隐奔依灵洗,灵洗奉以主盟。梁元帝于荆州承制,又遣使间道奉表。刘神茂自东阳建义拒贼,灵洗攻下新安,与神茂相应。元帝授持节、通直散骑常侍、都督新安郡诸军事、云麾将军、谯州刺史资,领新安太守,封巴丘县侯,邑五百户。神茂为景所破,景偏帅吕子荣进攻新安,灵洗退保黟、歙。及景败,子荣退走,灵洗复据新安。进军建德,擒贼帅赵桑乾。以功授持节、散骑常侍、都督青、冀二州诸军事、青州刺史,增邑并前一千户,将军、太守如故。

仍令灵洗率所部下扬州,助王僧辩镇防。迁吴兴太守,未行,僧辩命灵洗从侯瑱西援荆州。荆州陷,还都。高祖诛僧辩,灵洗率所领来援,其徒力战于石头西门,军不利,遣使招谕,久之乃降,高祖深义之。绍泰元年,授使持节、信武将军、兰陵太守,常侍如故,助防京口。及平徐嗣徽,灵洗有功,除南丹阳太守,封遂安县侯,增邑并前一千五百户,仍镇采石。

随周文育西讨王琳,于沌口败绩,为琳所拘。明年,与侯安都等逃归。兼丹阳尹,出为高唐、太原二郡太守,仍镇南陵。迁太子左卫率。高祖崩,王琳前军东下,灵洗于南陵破之,虏其兵士,并获青龙十馀乘。以功授持节、都督南豫州缘江诸军事、信武将军、南豫州刺史。侯瑱等败王琳于栅口,灵洗乘胜逐北,据有鲁山。征为左卫将军,馀如故。

天嘉四年,周迪重寇临川,以灵洗为都督,自鄱阳别道击之,迪又走山谷间。五年,迁中护军,常侍如故。出为使持节、都督郢、巴、武三州诸军事、宣毅将军、郢州刺史。废帝即位,进号云麾将军。

华皎之反也,遣使招诱灵洗,灵洗斩皎使,以状闻。朝廷深嘉其忠,增其守备,给鼓吹一部,因推心待之,使其子文季领水军助防。是时周遣其将长胡公拓跋定率步骑二万助皎攻围灵洗,灵洗婴城固守。及皎退,乃出军蹑定,定不获济江,以其众降。因进攻周沔州,克之,擒其刺史裴宽。以功进号安西将军,改封重安县公,增邑并前二千户。

灵洗性严急,御下甚苛刻,士卒有小罪,必以军法诛之,造次之间,便加捶挞,而号令分明,与士卒同甘苦,众亦以此依附。性好播植,躬勤耕稼,至于水陆所宜,刈获早晚,虽老农不能及也。伎妾无游手,并督之纺绩。至于散用赀财,亦弗俭吝。光大二年,卒于州,时年五十五。赠镇西将军、开府仪同三司,谥曰忠壮。太建四年,诏配享高祖庙庭。子文季嗣。

文季字少卿。幼习骑射,多干略,果决有父风。弱冠从灵洗征讨,必前登陷阵。灵洗与周文育、侯安都等败于沌口,为王琳所执,高祖召陷贼诸将子弟厚遇之,文季最有礼容,深为高祖所赏。永定中,累迁通直散骑侍郎、句容令。世祖嗣位,除宣惠始兴王府限内中直兵参军。是时王为扬州刺史,镇冶城,府中军事,悉以委之。

天嘉二年,除贞毅将军、新安太守,仍随侯安都东讨留异。异党向文政据有新安,文季率精甲三百,轻往攻之。文政遣其兄子瓒来拒,文季与战,大破瓒军,文政乃降。

三年,始兴王伯茂出镇东州,复以文季为镇东府中兵参军,带剡令。

四年,陈宝应与留异连结,又遣兵随周迪更出临川,世祖遣信义太守余孝顷自海道袭晋安,文季为之前军,所向克捷。陈宝应平,文季战功居多,还,转府谘议参军,领中直兵。出为临海太守。寻乘金翅助父镇郢城。华皎平,灵洗及文季并有捍御之功。及灵洗卒,文季尽领其众,起为超武将军,仍助防郢州。文季性至孝,虽军旅夺礼,而毁瘠甚至。

太建二年,为豫章内史,将军如故。服阕,袭封重安县公。随都督章昭达率军往荆州征萧岿。岿与周军多造舟舰,置于青泥水中。时水长漂疾,昭达乃遣文季共钱道戢轻舟袭之,尽焚其舟舰。昭达因萧岿等兵稍怠,又遣文季夜入其外城,杀伤甚众。既而周兵大出,巴陵内史雷道勤拒战死之,文季仅以身免。以功加通直散骑常侍、安远将军,增邑五百户。

五年,都督吴明彻北讨秦郡,秦郡前江浦通涂水,齐人并下大柱为杙,栅水中,乃前遣文季领骁勇拔开其栅,明彻率大军自后而至,攻秦郡克之。又别遣文季围泾州,屠其城,进攻盱眙,拔之。仍随明彻围寿阳。

文季临事谨急,御下严整,前后所克城垒,率皆迮水为堰,土木之功,动逾数万。每置阵役人,文季必先诸将,夜则早起,迄暮不休,军中莫不服其勤干。每战恒为前锋,齐军深惮之,谓为程虎。以功除散骑常侍、明威将军,增邑五百户。又带新安内史,进号武毅将军。

八年,为持节、都督谯州诸军事、安远将军、谯州刺史。其年,又督北徐仁州诸军事、北徐州刺史,馀并如故。九年,又随明彻北讨,于吕梁作堰,事见明彻传。十年春,败绩,为周所囚,仍授开府仪同三司。十一年,自周逃归,至涡阳,为边吏所执,还送长安,死于狱中。后主是时既与周绝,不之知也。至德元年,后主始知之,追赠散骑常侍。寻又诏曰:``故散骑常侍、前重安县开国公文季,纂承门绪,克荷家声。早岁出军,虽非元帅,启行为最,致果有闻,而覆丧车徒,允从黜削。但灵洗之立功捍御,久而见思,文季之埋魂异域,有足可悯。言念劳旧,伤兹废绝,宜存庙食,无使馁而。可降封重安县侯,邑一千户,以子飨袭封。''

史臣曰:程灵洗父子并御下严苛,治兵整肃,然与众同其劳苦,匪私财利,士多依焉,故临戎克办矣。

\hypertarget{header-n4395}{%
\subsubsection{卷五}\label{header-n4395}}

黄法抃淳于量章昭达

黄法抃,字仲昭,巴山新建人也。少劲捷有胆力,步行日三百里,距跃三丈。颇便书疏,闲明簿领,出入郡中,为乡闾所惮。侯景之乱,于乡里合徒众。太守贺诩下江州,法抃监知郡事。高祖将逾岭入援建业,李迁仕作梗中途,高祖命周文育屯于西昌,法抃遣兵助文育。时法抃出顿新淦县,景遣行台于庆至豫章,庆分兵来袭新淦,法抃拒战,败之。高祖亦遣文育进军讨庆,文育疑庆兵强,未敢进,法抃率众会之,因进克笙屯,俘获甚众。

梁元帝承制授超猛将军、交州刺史资,领新淦县令,封巴山县子,邑三百户。承圣三年,除明威将军、游骑将军,进爵为侯,邑五百户。贞阳侯僭位,除左骁骑将军。敬帝即位,改封新建县侯,邑如前。太平元年,割江州四郡置高州,以法抃为使持节、散骑常侍、都督高州诸军事、信武将军、高州刺史,镇于巴山。萧勃遣欧阳頠攻法抃,法抃与战,破之。

永定二年,王琳遣李孝钦、樊猛、余孝顷攻周迪,且谋取法抃,法抃率兵援迪,擒孝顷等三将。进号宣毅将军,增邑并前一千户,给鼓吹一部。又以拒王琳功,授平南将军、开府仪同三司。熊昙朗于金口反,害周文育,法抃共周迪讨平之,语在昙朗传。

世祖嗣位,进号安南将军。天嘉二年,周迪反,法抃率兵会都督吴明彻,讨迪于工塘。迪平,法抃功居多,征为使持节、散骑常侍、都督南徐州诸军事、镇北大将军、南徐州刺史,仪同、鼓吹并如故。未拜,寻又改授都督江、吴二州诸军事、镇南大将军、江州刺史。六年,征为中卫大将军。废帝即位,进爵为公,给扶。光大元年,出为使持节、都督南徐州诸军事、镇北将军、南徐州刺史。二年,徙为都督郢、巴、武三州诸军事、镇西将军、郢州刺史,持节如故。

太建元年,进号征西大将军。二年,征为侍中、中权大将军。四年,出为使持节、散骑常侍、都督南豫州诸军事、征南大将军、南豫州刺史。五年,大举北伐,都督吴明彻出秦郡,以法抃为都督,出历阳。齐遣其历阳王步骑五万来援,于小岘筑城。法抃遣左卫将军樊毅分兵于大岘御之,大破齐军,尽获人马器械。于是乃为拍车及步舰,竖拍以逼历阳。历阳人窘蹴乞降,法抃缓之,则又坚守,法抃怒,亲率士卒攻城,施拍加其楼堞。时又大雨,城崩,克之,尽诛戍卒。进兵合肥,望旗降款,法抃不令军士侵掠,躬自抚劳,而与之盟,并放还北。以功加侍中,改封义阳郡公,邑二千户。其年,迁都督合、霍二州诸军事、征西大将军、合州刺史,增邑五百户。七年,徙都督豫、建、光、朔、合、北徐六州诸军事、豫州刺史,镇寿阳,侍中、散骑常侍、持节、将军、仪同、鼓吹、扶并如故。八年十月,薨,时年五十九。赠侍中、中权大将军、司空,谥曰威。子玩嗣。

淳于量,字思明。其先济北人也,世居京师。父文成,仕梁为将帅,官至光烈将军、梁州刺史。量少善自居处,伟姿容,有干略,便弓马。梁元帝为荆州刺史,文成分量人马,令往事焉。起家湘东王国常侍,兼西中郎府中兵参军。累迁府佐、常兼中兵、直兵者十馀载,兵甲士卒,盛于府中。

荆、雍之界,蛮左数反,山帅文道期积为边患,中兵王僧辩征之,频战不利,遣量助之。量至,与僧辩并力,大破道期,斩其酋长,俘虏万计。以功封广晋县男,邑三百户,授涪陵太守。历为新兴、武宁二郡太守。

侯景之乱,梁元帝凡遣五军入援京邑,量预其一。台城陷,量还荆州。元帝承制以量为假节、通直散骑常侍、都督巴州诸军事、信威将军、巴州刺史。侯景西上攻巴州,元帝使都督王僧辩入据巴陵。量与僧辩并力拒景,大败景军,擒其将任约。进攻郢州,获宋子仙。仍随僧辩克平侯景。承圣元年,以功授左卫将军,封谢沐县侯,邑五百户。寻出为持节、都督桂、定、东、西宁等四州诸军事、信威将军、安远护军、桂州刺史。

荆州陷,量保据桂州。王琳拥割湘、郢,累遣召量,量外虽与琳往来,而别遣使从间道归于高祖。高祖受禅,授持节、散骑常侍、平西大将军,给鼓吹一部,都督、刺史并如故。寻进号镇南将军。仍授都督、镇西大将军、开府仪同三司。世祖嗣位,进号征南大将军。王琳平后,频请入朝,天嘉五年,征为中抚大将军,常侍、仪同、鼓吹并如故。量所部将帅,多恋本土,并欲逃入山谷,不愿入朝。世祖使湘州刺史华皎征衡州界黄洞,且以兵迎量。天康元年,至都,以在道淹留,为有司所奏,免仪同,馀并如故。光大元年,给鼓吹一部。华皎构逆,以量为使持节、征南大将军、西讨大都督,总率大舰,自郢州樊浦拒之。皎平,并降周将长胡公拓跋定等。以功授侍中、中军大将军、开府仪同三司,进封醴陵县公,增邑一千户。未拜,出为使持节、都督南徐州诸军事、镇北将军、南徐州刺史,侍中、仪同、鼓吹并如故。

太建元年,进号征北大将军,给扶。三年,坐就江阴王萧季卿买梁陵中树,季卿坐免,量免侍中。寻复加侍中。五年,征为中护大将军,侍中、仪同、鼓吹、扶并如故。

吴明彻之西伐也,量赞成其事,遣第六子岑率所领从军。淮南克定,量改封始安郡公,增邑一千五百户。六年,出为使持节、都督郢、巴、南司、定四州诸军事、征西大将军、郢州刺史,侍中、仪同、鼓吹、扶并如故。七年,征为中军大将军、护军将军。九年,以公事免侍中。寻复加侍中。十年,吴明彻陷没,加量使持节、都督水陆诸军事,仍授散骑常侍、都督南北兗、谯三州诸军事、车骑将军、南兗州刺史,馀并如故。十三年,加左光禄大夫,增邑五百户,馀并如故。十四年四月薨,时年七十二。赠司空。

章昭达,字伯通,吴兴武康人也。祖道盖,齐广平太守。父法尚,梁扬州议曹从事。昭达性倜傥,轻财尚气。少时,尝遇相者,谓昭达曰:``卿容貌甚善,须小亏损,则当富贵。''梁大同中,昭达为东宫直后,因醉坠马,鬓角小伤,昭达喜之,相者曰:``未也。''及侯景之乱,昭达率募乡人援台城,为流矢所中,眇其一目,相者见之,曰:``卿相善矣,不久当贵。''

京城陷,昭达还乡里,与世祖游,因结君臣之分。侯景平,世祖为吴兴太守,昭达杖策来谒世祖。世祖见之大喜,因委以将帅,恩宠优渥,超于侪等。及高祖讨王僧辩,令世祖还长城招聚兵众,以备杜龛,频使昭达往京口,禀承计划。僧辩诛后,龛遣其将杜泰来攻长城,世祖拒之,命昭达总知城内兵事。及杜泰退走,因从世祖东进,军吴兴,以讨杜龛。龛平,又从世祖东讨张彪于会稽,克之。累功除明威将军、定州刺史。

是时留异拥据东阳,私署守宰,高祖患之,乃使昭达为长山县令,居其心腹。永定二年,除武康令。世祖嗣位,除员外散骑常侍。天嘉元年,追论长城之功,封欣乐县侯,邑一千户。寻随侯安都等拒王琳于栅口,战于芜湖,昭达乘平虏大舰,中流而进,先锋发拍中于贼舰。王琳平,昭达册勋第一。二年,除使持节、散骑常侍、都督郢、巴、武沅四州诸军事、智武将军、郢州刺史,增邑并前千五百户。寻进号平西将军。

周迪据临川反,诏令昭达便道征之。及迪败走,征为护军将军,给鼓吹一部,改封邵武县侯,增邑并前二千户,常侍如故。四年,陈宝应纳周迪,复共寇临川,又以昭达为都督讨迪。至东兴岭,而迪又退走。昭达仍逾岭,顿于建安,以讨陈宝应。宝应据建安、晋安二郡之界,水陆为栅,以拒官军。昭达与战不利,因据其上流,命军士伐木带枝叶为筏,施拍于其上,缀以大索,相次列营,夹于两岸。宝应数挑战,昭达按甲不动。俄而暴雨,江水大长,昭达放筏冲突宝应水栅,水栅尽破。又出兵攻其步军。方大合战,会世祖遣余孝顷出自海道。适至,因并力乘之,宝应大溃,遂克定闽中,尽擒留异、宝应等。以功授镇前将军、开府仪同三司。

初,世祖尝梦昭达升于台铉,及旦,以梦告之。至是侍宴,世祖顾诏达曰:``卿忆梦不?何以偿梦?''昭达对曰:``当效犬马之用,以尽臣节,自馀无以奉偿。''寻又出为使持节、都督江、郢、吴三州诸军事、镇南将军、江州刺史,常侍、仪同、鼓吹如故。

废帝即位,迁侍中、征南将军,改封邵陵郡公。华皎之反也,其移书文檄,并假以昭达为辞,又频遣使招之,昭达尽执其使,送于京师。皎平,进号征南大将军,增邑并前二千五百户。秩满,征为中抚大将军,侍中、仪同、鼓吹如故。高宗即位,进号车骑大将军,以还朝迟留,为有司所劾,降号车骑将军。

欧阳纥据有岭南反,诏昭达都督众军讨之。昭达倍道兼行,达于始兴。纥闻昭达奄至,恇扰不知所为,乃出顿洭口,多聚沙石,盛以竹笼,置于水栅之外,用遏舟舰。昭达居其上流,装造拍,以临贼栅。又令军人衔刀,潜行水中,以斫竹笼,笼篾皆解。因纵大舰随流突之,贼众大败,因而擒纥,送于京师,广州平。以功进车骑大将军,迁司空,馀并如故。

太建二年,率师征萧岿于江陵。时萧岿与周军大蓄舟舰于青泥中,昭达分遣偏将钱道戢、程文季等,乘轻舟袭之,焚其舟舰。周兵又于峡下南岸筑垒,名曰安蜀城,于江上横引大索,编苇为桥,以度军粮。昭达乃命军士为长戟,施于楼船之上,仰割其索,索断粮绝,因纵兵以攻其城,降之。三年,遘疾,薨,时年五十四。赠大将军,增邑五百户,给班剑二十人。

昭达性严刻,每奉命出征,必昼夜倍道;然有所克捷,必推功将帅,厨膳饮食,并同于群下,将士亦以此附之。每饮会,必盛设女伎杂乐,备尽羌胡之声,音律姿容,并一时之妙,虽临对寇敌,旗鼓相望,弗之废也。四年,配享世祖庙庭。

子大宝,袭封邵陵郡公,累官至散骑常侍、护军。出为豊州刺史,在州贪纵,百姓怨酷,后主以太仆卿李晕代之。至德三年四月,晕将到州,大宝乃袭杀晕,举兵反,遣其将杨通寇建安。建安内史吴慧觉据郡城拒之,通累攻不克。官军稍近,人情离异,大宝计穷,乃与通俱逃。台军主陈景详率兵追蹑大宝。大宝既入山,山路阴险,不复能行,通背负之,稍进。寻为追兵所及,生擒送都,于路死,传首枭于硃雀航,夷三族。

史臣曰:黄法抃、淳于量值梁末丧乱,刘、项未分,其有辩明暗见是非者盖鲜,二公达向背之理,位至鼎司,亦其智也。昭达与世祖乡壤惟旧,义等邓、萧,世祖纂历,委任隆重,至于战胜攻取,累平寇难,斯亦良臣良将,一代之吴、耿矣。

\hypertarget{header-n4421}{%
\subsubsection{卷六}\label{header-n4421}}

胡颖徐度子敬成杜棱沈恪

胡颖,字方秀,吴兴东迁人也。其先寓居吴兴,土断为民。颖伟姿容,性宽厚。梁世仕至武陵国侍郎,东宫直前。出番禺,征讨俚洞,广州西江督护。高祖在广州,颖仍自结高祖,高祖与其同郡,接遇甚隆。及南征交趾,颖从行役,馀诸将帅皆出其下。及平李贲,高祖旋师,颖隶在西江,出兵多以颖留守。

侯景之乱,高祖克元景仲,仍渡岭援台,平蔡路养、李迁仕,颖皆有功。历平固、遂兴二县令。高祖进军顿西昌,以颖为巴丘县令,镇大皋,督粮运。下至豫章,以颖监豫章郡。高祖率众与王僧辩会于白茅湾,同讨侯景,以颖知留府事。

梁承圣初,元帝授颖假节、铁骑将军、罗州刺史,封汉阳县侯,邑五百户。寻除豫章内史,随高祖镇京口。齐遣郭元建出关,都督侯瑱率师御之。高祖选府内骁勇三千人配颖,令随瑱,于东关大破之。三年,高祖围广陵,齐人东方光据宿预请降,以颖为五原太守,随杜僧明援光,不克,退还,除曲阿令。寻领马军,从高祖袭王僧辩。又随周文育于吴兴讨杜龛。绍泰元年,除假节、都督南豫州诸军事、轻车将军、南豫州刺史。太平元年,除持节、散骑常侍、仁威将军。寻兼丹阳尹。

高祖受禅,兼左卫将军,馀如故。永定三年,随侯安都征王琳,于宫亭破贼帅常众爱等。世祖嗣位,除侍中、都督吴州诸军事、宣惠将军、吴州刺史。不行,寻为义兴太守,将军如故。天嘉元年,除散骑常侍、吴兴太守。其年六月卒,时年五十四。赠侍中、中护军,谥曰壮。二年,配享高祖庙庭。子六同嗣。

颖弟铄,亦随颖将军。颖卒,铄统其众。历东海、豫章二郡守,迁员外散骑常侍。随章昭达南平欧阳纥,为广州东江督护。还预北伐,除雄信将军、历阳太守。太建六年卒,赠桂州刺史。

徐度,字孝节,安陆人也。世居京师。少倜傥,不拘小节。及长,姿貌瑰伟,嗜酒好博。恒使僮仆屠酤为事。梁始兴内史萧介之郡,度从之,将领士卒,征诸山洞,以骁勇闻。高祖征交趾,厚礼招之,度乃委质。

侯景之乱,高祖克定广州,平蔡路养,破李迁仕,计划多出于度。兼统兵甲,每战有功。归至白茅湾,梁元帝授宁朔将军、合州刺史。侯景平后,追录前后战功,加通直散骑常侍,封广德县侯,邑五百户。迁散骑常侍。高祖镇硃方,除信武将军、兰陵太守。高祖遣衡阳献王往荆州,度率所领从焉。江陵陷,间行东归。高祖平王僧辩,度与侯安都为水军。绍泰元年,高祖东讨杜龛,奉敬帝幸京口,以度领宿卫,并知留府事。

徐嗣徽、任约等来寇,高祖与敬帝还都。时贼已据石头城,市廛阝居民,并在南路,去台遥远,恐为贼所乘,乃使度将兵镇于冶城寺,筑垒以断之。贼悉众来攻,不能克。高祖寻亦救之,大败约等。明年,嗣徽等又引齐寇济江,度随众军破之于北郊坛。以功除信威将军、郢州刺史,兼领吴兴太守。寻迁镇右将军、领军将军、徐州缘江诸军事、镇北将军、南徐州刺史,给鼓吹一部。

周文育、侯安都等西讨王琳,败绩,为琳所拘,乃以度为前军都督,镇于南陵。世祖嗣位,迁侍中、中抚军将军、开府仪同三司,进爵为公。未拜,出为使持节、散骑常侍、镇东将军、吴郡太守。天嘉元年,增邑千户。以平王琳功,改封湘东郡公,邑四千户。秩满,为侍中、中军将军。出为使持节、都督会稽、东阳、临海、永嘉、新安、新宁、信安、晋安、建安九郡诸军事、镇东将军、会稽太守。未行而太尉侯瑱薨于湘州,乃以度代瑱为都督湘、沅、武、巴、郢、桂六州诸军事、镇南将军、湘州刺史。秩满,为侍中、中军大将军,仪同、鼓吹并如故。

世祖崩,度预顾命,以甲仗五十人入殿省。废帝即位,进位司空。华皎据湘州反,引周兵下至沌口,与王师相持,乃加度使持节、车骑将军,总督步军,自安成郡由岭路出于湘东,以袭湘州,尽获其所留军人家口以归。光大二年薨,时年六十。赠太尉,给班剑二十人,谥曰忠肃。太建四年,配享高祖庙庭。子敬成嗣。

敬成幼聪慧,好读书,少机警,善占对,结交文义之士,以识鉴知名。起家著作郎。永定元年,领度所部士卒,随周文育、侯安都征王琳,于沌口败绩,为琳所絷。二年,随文育、安都得归,除太子舍人,迁洗马。度为吴郡太守,以敬成监郡。天嘉二年,迁太子中舍人,拜湘东郡公世子。四年,度自湘州还朝,士马精锐,敬成尽领其众。随章昭达征陈宝应,晋安平,除贞威将军、豫章太守。光大元年,华皎谋反,以敬成为假节、都督巴州诸军事、云旗将军、巴州刺史。寻诏为水军,随吴明彻征华皎,皎平还州。二年,以父忧去职。寻起为持节、都督南豫州诸军事、壮武将军、南豫州刺史。四年,袭爵湘东郡公,授太子右卫率。

五年,除贞威将军、吴兴太守。其年随都督吴明彻北讨,出秦郡,别遣敬成为都督,乘金翅自欧阳引埭上溯江由广陵。齐人皆城守,弗敢出。自繁梁湖下淮,围淮阴城。仍监北兗州。淮、泗义兵相率响应,一二日间,众至数万,遂克淮阴、山阳、盐城三郡,并连口、朐山二戍。仍进攻郁州,克之。以功加通直散骑常侍、云旗将军,增邑五百户。又进号壮武将军,镇朐山。坐于军中辄科订,并诛新附,免官。寻复为持节、都督安、元、潼三州诸军事、安州刺史,将军如故,镇宿预。七年卒,时年三十六。赠散骑常侍,谥曰思。子敞嗣。

杜棱,字雄盛,吴郡钱塘人也。世为县大姓。棱颇涉书传,少落泊,不为当世所知。遂游岭南,事梁广州刺史新渝侯萧映。映卒,从高祖,恒典书记。侯景之乱,命棱将领,平蔡路养、李迁仕皆有功。军至豫章,梁元帝承制授棱仁威将军、石州刺史,上陌县侯,邑八百户。

侯景平,高祖镇硃方,棱监义兴、琅邪二郡。高祖诛王僧辩,引棱与侯安都等共议,棱难之。高祖惧其泄己,乃以手巾绞棱,棱闷绝于地,因闭于别室。军发,召与同行。及僧辩平后,高祖东征杜龛等,留棱与安都居守。徐嗣徽、任约引齐寇济江,攻台城,安都与棱随方抗拒,棱昼夜巡警,绥抚士卒,未常解带。贼平,以功除通直散骑常侍、右卫将军、丹阳尹。永定元年,加侍中、忠武将军。寻迁中领军,侍中,将军如故。

三年,高祖崩,世祖在南皖。时内无嫡嗣,外有强敌,侯瑱、侯安都、徐度等并在军中,朝廷宿将,唯棱在都,独典禁兵,乃与蔡景历等秘不发丧,奉迎世祖,事见景历传。世祖即位,迁领军将军。天嘉元年,以预建立之功,改封永城县侯,增邑五百户。出为云麾将军,晋陵太守,加秩中二千石。二年,征为侍中、领军将军。寻迁翊左将军、丹阳尹。废帝即位,迁镇右将军、特进,侍中、尹如故。光大元年,解尹,量置佐史,给扶,重授领军将军。

太建元年,出为散骑常侍、镇东将军、吴兴太守,秩中二千石。二年,征为侍中、镇右将军。寻加特进、护军将军。三年,以公事免侍中、护军。四年,复为侍中、右光禄大夫,并给鼓吹一部,将军、佐史、扶并如故。

棱历事三帝,并见恩宠。末年不预征役,优游京师,赏赐优洽。顷之卒于官,时年七十。赠开府仪同三司,丧事所须,并令资给,谥曰成。其年配享高祖庙庭。子安世嗣。

沈恪,字子恭,吴兴武康人也。深沈有干局。梁新渝侯萧映为郡将,召为主簿。映迁北徐州,恪随映之镇。映迁广州,以恪兼府中兵参军,常领兵讨伐俚洞。卢子略之反也。恪拒战有功,除中兵参军。高祖与恪同郡,情好甚昵,萧映卒后,高祖南讨李贲,仍遣妻子附恪还乡。寻补东宫直后,以岭南勋除员外散骑侍郎,仍令招集宗从子弟。

侯景围台城,恪率所领入台,随例加右军将军。贼起东西二土山以逼城,城内亦作土山以应之,恪为东土山主,昼夜拒战。以功封东兴县侯,邑五百户。迁员外散骑常侍。京城陷,恪间行归乡里。高祖之讨侯景,遣使报恪,乃于东起兵相应。贼平,恪谒高祖于京口,即日授都军副。寻为府司马。

及高祖谋讨王僧辩,恪预其谋。时僧辩女婿杜龛镇吴兴,高祖乃使世祖还长城,立栅备龛,又使恪还武康,招集兵众。及僧辩诛,龛果遣副将杜泰率众袭世祖于长城。恪时已率兵士出县诛龛党与,高祖寻遣周文育来援长城,文育至,泰乃遁走。世祖仍与文育进军出郡,恪军亦至,屯于郡南。及龛平,世祖袭东扬州刺史张彪,以恪监吴兴郡。太平元年,除宣猛将军、交州刺史。其年迁永嘉太守。不拜,复令监吴兴郡。自吴兴入朝。高祖受禅,使中书舍人刘师知引恪,令勒兵入,因卫敬帝如别宫。恪乃排闼入见高祖,叩头谢曰:``恪身经事萧家来,今日不忍见许事,分受死耳,决不奉命。''高祖嘉其意,乃不复逼,更以荡主王僧志代之。

高祖践祚,除吴兴太守。永定二年,徙监会稽郡。会余孝顷谋应王琳,出兵临川攻周迪,以恪为壮武将军,率兵逾岭以救迪。余孝顷闻恪至,退走。三年,迁使持节、通直散骑常侍、智武将军、吴州刺史,便道之鄱阳。寻有诏追还,行会稽郡事。其年,除散骑常侍、忠武将军、会稽太守。

世祖嗣位,进督会稽、东阳、新安、临海、永嘉、建安、晋安、新宁、信安九郡诸军事,将军、太守如故。天嘉元年,增邑五百户。二年,征为左卫将军。俄出为都督郢、武、巴定四州诸军事、军师将军、郢州刺史。六年,征为中护军。寻迁护军将军。光大二年,迁使持节、都督荆武右三州诸军事、平西将军、荆州刺史。未之镇,改为护军将军。

高宗即位,加散骑常侍、都督广、衡、东衡、交、越、成、定、新、合、罗、爱、德、宜、黄、利、安、石、双等十八州诸军事、镇南将军、平越中郎将、广州刺史。恪未至岭,前刺史欧阳纥举兵拒险,恪不得进,朝廷遣司空章昭达督众军讨纥,纥平,乃得入州。州罹兵荒,所在残毁,恪绥怀安缉,被以恩惠,岭表赖之。

太建四年,征为领军将军。及代还,以途还不时至,为有司所奏免。十一年,起为散骑常侍、卫尉卿。其年授平北将军、假节,监南兗州。十二年,改授散骑常侍、翊右将军,监南徐州。又遣电威将军裴子烈领马五百匹,助恪缘江防戍。明年,入为卫尉卿,常侍、将军如故。寻加侍中,迁护军将军。后主即位,以疾改授散骑常侍、特进、金紫光禄大夫。其年卒,时年七十四。赠翊左将军,诏给东园秘器,仍出举哀,丧事所须,并令资给,谥曰元。子法兴嗣。

史臣曰:胡颖、徐度、杜棱、沈恪并附骐骥而腾跃,依日月之光辉,始觏王佐之才,方悟公辅之量,生则肉食,终以配飨。盛矣哉!

\hypertarget{header-n4450}{%
\subsubsection{卷七}\label{header-n4450}}

徐世谱鲁悉达周敷荀朗子法尚周炅

徐世谱,字兴宗,巴东鱼复人也。世居荆州,为主帅,征伐蛮、蜒。至世谱,尤敢勇有膂力,善水战。梁元帝之为荆州刺史,世谱将领乡人事焉。

侯景之乱,因预征讨,累迁至员外散骑常侍。寻领水军,从司徒陆法和讨景,与景战于赤亭湖。时景军甚盛,世谱乃别造楼船、拍舰、火舫、水车以益军势。将战,又乘大舰居前,大败景军,生擒景将任约,景退走。因随王僧辩攻郢州,世谱复乘大舰临其仓门,贼将宋子仙据城降。以功除使持节、信武将军、信州刺史,封鱼复县侯,邑五百户。仍随僧辩东下,恒为军锋。又破景将侯子鉴于湖熟。侯景平后,以功除通直散骑常侍、衡州刺史资,领河东太守,增邑并前一千户。

西魏来寇荆州,世谱镇马头岸,据有龙洲,元帝授侍中、使持节、都督江南诸军事、镇南将军、护军将军,给鼓吹一部。江陵陷没,世谱东下依侯瑱。绍泰元年,征为侍中、左卫将军。高祖之拒王琳,其水战之具,悉委世谱。世谱性机巧,谙解旧法,所造器械,并随机损益,妙思出入。

永定二年,迁护军将军。世祖嗣位,加特进,进号安右将军。天嘉元年,增邑五百户。二年,出为使持节、散骑常侍、都督宣城郡诸军事、安西将军、宣城太守,秩中二千石。还为安前将军、右光禄大夫。寻以疾失明,谢病不朝。四年卒,时年五十五。赠本官,谥曰桓侯。

世谱从弟世休,随世谱自梁征讨,亦有战功。官至员外散骑常侍、安远将军,枳县侯,邑八百户。光大二年,隶都督淳于量征华皎。卒,赠通直散骑常侍,谥曰壮。

鲁悉达,字志通,扶风郿人也。祖斐,齐通直散骑常侍、安远将军、衡州刺史,阳塘侯。父益之,梁云麾将军、新蔡、义阳二郡太守。悉达幼以孝闻,起家为梁南平嗣王中兵参军。侯景之乱,悉达纠合乡人,保新蔡,力田蓄谷。时兵荒饥馑,京都及上川饿死者十八九,有得存者,皆携老幼以归焉。悉达分给粮廪,其所济活者甚众,仍于新蔡置顿以居之。招集晋熙等五郡,尽有其地。使其弟广达领兵随王僧辩讨侯景。景平,梁元帝授持节、仁威将军、散骑常侍、北江州刺史。

敬帝即位,王琳据有上流,留异、余孝顷、周迪等所在蜂起,悉达抚绥五郡,甚得民和,士卒皆乐为之用。琳授悉达镇北将军,高祖亦遣赵知礼授征西将军、江州刺史,各送鼓吹女乐,悉达两受之,迁延顾望,皆不就。高祖遣安西将军沈泰潜师袭之,不能克。齐遣行台慕容绍宗以众三万来攻郁口诸镇,兵甲甚盛,悉达与战,败齐军,绍宗仅以身免。

王琳欲图东下,以悉达制其中流,恐为己患,频遣使招诱,悉达终不从。琳不得下,乃连结于齐,共为表里,齐遣清河王高岳助之。相持岁馀,会裨将梅天养等惧罪,乃引齐军入城。悉达勒麾下数千人,济江而归高祖。高祖见之,甚喜,曰:``来何迟也。''悉达对曰:``臣镇抚上流,愿为蕃屏,陛下授臣以官,恩至厚矣,沈泰袭臣,威亦深矣,然臣所以自归于陛下者,诚以陛下豁达大度,同符汉祖故也。''高祖叹曰:``卿言得之矣。''授平南将军、散骑常侍、北江州刺史,封彭泽县侯。世祖即位,进号安左将军。

悉达虽仗气任侠,不以富贵骄人,雅好词赋,招礼才贤,与之赏会。迁安南将军、吴州刺史。遭母忧,哀毁过礼,因遘疾卒,时年三十八。赠安左将军、江州刺史,谥曰孝侯。子览嗣。弟广达,别有传。

周敷,字仲远,临川人也。为郡豪族。敷形貌眇小,如不胜衣,而胆力劲果,超出时辈。性豪侠,轻财重士,乡党少年任气者咸归之。

侯景之乱,乡人周续合徒众以讨贼为名,梁内史始兴王毅以郡让续,续所部内有欲侵掠于毅,敷拥护之,亲率其党捍卫,送至豫章。时观宁侯萧永、长乐侯萧基、豊城侯萧泰避难流寓,闻敷信义,皆往依之。敷愍其危惧,屈体崇敬,厚加给恤,送之西上。俄而续部下将帅争权,复反,杀续以降周迪。迪素无簿阀,恐失众心,倚敷族望,深求交结。敷未能自固,事迪甚恭,迪大凭仗之,渐有兵众。迪据临川之工塘,敷镇临川故郡。侯景平,梁元帝授敷使持节、通直散骑常侍、信武将军、宁州刺史,封西豊县侯,邑一千户。

高祖受禅,王琳据有上流,余孝顷与琳党李孝钦等共图周迪,敷大致人马以助于迪。迪擒孝顷等,敷功居多。熊昙朗之杀周文育,据豫章,将兵万馀人袭敷,径至城下,敷与战,大败之,追奔五十馀里,昙朗单马获免,尽收其军实。昙朗走巴山郡,收合馀党,敷因与周迪、黄法等进兵围昙朗,屠之。王琳平,授散骑常侍、平西将军、豫章太守。是时南江酋帅并顾恋巢窟,私署令长,不受召,朝廷未遑致讨,但羁縻之,唯敷独先入朝。天嘉二年,诣阙,进号安西将军,给鼓吹一部,赐以女乐一部,令还镇豫章。

周迪以敷素出己下,超致显贵,深不平,乃举兵反,遣弟方兴以兵袭敷。敷与战,大破方兴。仍率众从都督吴明彻攻迪,破之,擒其弟方兴并诸渠帅。诏以敷为安西将军、临川太守,馀并如故。寻征为使持节、都督南豫、北江二州诸军事、镇南将军、南豫州刺史,增邑五百户,常侍、鼓吹如故。五年,迪又收合馀众,还袭东兴。世祖遣都督章昭达征迪,敷又从军。至定川县,与迪相对。迪绐敷曰:``吾昔与弟戮力同心,宗从匪他,岂规相害。今愿伏罪还朝,因弟披露心腑,先乞挺身共立盟誓。''敷许之,方登坛,为迪所害,时年三十五。诏曰:``使持节、散骑常侍、都督南豫州缘江诸军事、镇南将军、南豫州刺史西豊县开国侯敷,受任遐征,淹时违律,虚衿奸诡,遂贻丧仆。但夙著勤诚,亟劳戎旅,犹深恻怆,愍悼于怀。可存其茅赋,量所赙恤,还葬京邑。''谥曰脱。子智安嗣。

敷兄彖,共敷据本乡,亦授临川太守。

荀朗,字深明,颍川颍阴人也。祖延祖,梁颍川太守,父伯道,卫尉卿。朗少慷慨,有将帅大略,起家梁庐陵王行参军。侯景之乱,朗招率徒旅,据巢湖间,无所属。台城陷后,简文帝密诏授朗云麾将军、豫州刺史,令与外籓讨景。景使仪同宋子仙、任约等频往征之,朗据山立砦自守,子仙不能克。时京师大饥,百姓皆于江外就食,朗更招致部曲,解衣推食,以相赈赡,众至数万人。侯景败于巴陵,朗出自濡须截景,破其后军。王僧辩东讨,朗遣其将范宝胜及弟晓领兵二千助之。侯景平后,又别破齐将郭元建于踟蹰山。梁承圣二年,率部曲万馀家济江,入宣城郡界立顿。梁元帝授朗持节、通直散骑常侍、安南将军、都督南兗州诸军事、南兗州刺史。未行而荆州陷。

高祖入辅,齐遣萧轨、东方老等来寇,据石头城。朗自宣城来赴,因与侯安都等大破齐军。永定元年,赐爵兴宁县侯,邑二千户,以朗兄昂为左卫将军,弟晷为太子右卫率。寻遣朗随世祖拒王琳于南皖。

高祖崩,宣太后与舍人蔡景历秘不发丧,朗弟晓在都微知之,乃谋率其家兵袭台。事觉,景历杀晓,仍系其兄弟。世祖即位,并释之。因厚抚慰朗,令与侯安都等共拒王琳。琳平,迁使持节、安北将军、散骑常侍、都督霍、晋、合三州诸军事、合州刺史。天嘉六年卒,时年四十八。赠南豫州刺史,谥曰壮。子法尚嗣。

法尚少倜傥,有文武干略,起家江宁令,袭爵兴宁县侯。太建五年,随吴明彻北伐。寻授通直散骑侍郎,除泾令,历梁、安城太守。祯明中,为都督郢、巴、武三州诸军事、郢州刺史。及隋军济江,法尚降于汉东道元帅秦王。入隋,历邵、观、绵、豊四州刺史,巴东、燉煌二郡太守。

周炅,字文昭,汝南安城人也。祖强,齐太子舍人、梁州刺史。父灵起,梁通直散骑常侍、庐、桂二州刺史,保城县侯。炅少豪侠任气,有将帅才。梁大同中为通直散骑侍郎、硃衣直阁。太清元年,出为弋阳太守。侯景之乱,元帝承制改授西阳太守,封西陵县伯。景遣兄子思穆据守齐安,炅率骁勇袭破思穆,擒斩之。以功授持节、高州刺史。是时炅据武昌、西阳二郡,招聚卒徒,甲兵甚盛。景将任约来据樊山,炅与宁州长史徐文盛击约,斩其部将叱罗子通、赵迦娄等。因乘胜追之,频克,约众殆尽。承圣元年,迁使持节、都督江、定二州诸军事、戎昭将军、江州刺史,进爵为侯,邑五百户。

高祖践祚,王琳拥据上流,炅以州从之。及王琳遣其将曹庆等攻周迪,仍使炅将兵掎角而进,为侯安都所败,擒炅送都。世祖释炅,授戎威将军、定州刺史,带西阳、武昌二郡太守。

天嘉二年,留异据东阳反,世祖召炅还都,欲令讨异。未至而异平,炅还本镇。天康元年,预平华皎之功,授员外散骑常侍。太建元年,迁持节、龙骧将军、通直散骑常侍。五年,进授使持节、西道都督安、蕲、江、衡、司、定六州诸军事、安州刺史,改封龙源县侯,增邑并前一千户。其年随都督吴明彻北讨,所向克捷,一月之中,获十二城。齐遣尚书左丞陆骞以众二万出自巴、蕲,与炅相遇。炅留羸弱辎重,设疑兵以当之,身率精锐,由间道邀其后,大败骞军,虏获器械马驴,不可胜数。进攻巴州,克之。于是江北诸城及谷阳士民,并诛渠帅以城降。进号和戎将军、散骑常侍,增邑并前一千五百户。仍敕追炅入朝。

初,萧詧定州刺史田龙升以城降,诏以为振远将军、定州刺史,封赤亭王。及炅入朝,龙升以江北六州七镇叛入于齐,齐遣历阳王高景安帅师应之。于是令炅为江北道大都督,总统众军,以讨龙升。龙升使弋阳太守田龙琰率众二万阵于亭川,高景安于水陵、阴山为其声援,龙升引军别营山谷。炅乃分兵各当其军,身率骁勇先击龙升,龙升大败,龙琰望尘而奔,并追斩之,高景安遁走,尽复江北之地。以功增邑并前二千户,进号平北将军,定州刺史,持节、都督如故,仍赐女妓一部。太建八年卒官,时年六十四。赠司州刺史,封武昌郡公,谥曰壮。子法僧嗣,官至宣城太守。

史臣曰:彼数子者,或驱驰前代,或拥据故乡,并识运知归,因机景附,位升列牧,爵致通侯,美矣。昔张耳、陈馀自同于至戚,周敷、周迪亦誓等昵亲,寻锋刃而诛残,斯甚夫胡越矣。雠隙因于势利,何其鄙欤!

\hypertarget{header-n4477}{%
\subsubsection{卷八}\label{header-n4477}}

衡阳献王昌南康愍王昙朗子方泰方庆

衡阳献王昌,字敬业,高祖第六子也。梁太清末,高祖南征李贲,命昌与宣后随沈恪还吴兴。及高祖东讨侯景,昌与宣后、世祖并为景所囚。景平,拜长城国世子、吴兴太守,时年十六。

昌容貌伟丽,神情秀朗,雅性聪辩,明习政事。高祖遣陈郡谢哲、济阳蔡景历辅昌为郡,又遣吴郡杜之伟授昌以经书。昌读书一览便诵,明于义理,剖析如流。寻与高宗俱往荆州,梁元帝除员外散骑常侍。荆州陷,又与高宗俱迁关右,西魏以高祖故,甚礼之。

高祖即位,频遣使请高宗及昌,周人许之而未遣,及高祖崩,乃遣之。是时王琳梗于中流,昌未得还,居于安陆。王琳平后,天嘉元年二月,昌发自安陆,由鲁山济江,而巴陵王萧沇等率百僚上表曰:

臣闻宗子维城,隆周之懋轨,封建籓屏,有汉之弘规,是以卜世斯永,式资邢、卫,鼎命灵长,实赖河、楚。伏惟陛下神猷光大,圣德钦明,道高日月,德侔造化。往者王业惟始,天步方艰,参奉权谟,匡合义烈,威略外举,神武内定,故以再康禹迹,大庇生民者矣。及圣武升遐,王师远次,皇嗣夐隔,继业靡归,宗祧危殆,缀旒非喻。既而传车言反,公卿定策,纂我洪基,光昭景运,民心有奉,园寝克宁,后来其苏,复在兹日,物情天意,皎然可求。王琳逆命,逋诛岁久,今者连结犬羊,乘流纵衅,舟旗野阵,绵江蔽陆,兵疲民弊,杼轴用空,中外骚然,蕃篱罔固。乃旰食当朝,凭流授律,苍兕既驰,长蛇自翦,廓清四表,澄涤八纮,雄图遐举,仁声远畅,德化所覃,风行草偃,故以功深于微禹,道大于惟尧,岂直社稷用宁,斯乃黔黎是赖。

第六皇弟昌,近以妙年出质,提契寇手,偏隔关徼,旋踵末由。陛下天伦之爱既深,克让之怀常切。伏以大德无私,至公有在,岂得徇匹夫之恒情,忘王业之大计。宪章故实,式遵典礼,钦若姬、汉,建树贤戚。湘中地维形胜,控带川阜,捍城之寄,匪亲勿居,宜启服衡、疑,兼崇徽饰。臣等参议,以昌为使持节、散骑常侍、都督湘州诸军事、骠骑将军、湘州牧,封衡阳郡王,邑五千户,加给皁轮三望车,后部鼓吹一部,班剑二十人。启可奉行。

诏曰``可''。三月入境,诏令主书舍人缘道迎接。丙子,济江,于中流船坏,以溺薨。

四月庚寅,丧柩至京师,上亲出临哭。乃下诏曰:``夫宠章所以嘉德,礼数所以崇亲,乃历代之通规,固前王之令典。新除使持节、散骑常侍、都督湘州诸军事、骠骑将军、湘州牧衡阳王昌,明哲在躬,珪璋早秀,孝敬内湛,聪睿外宣。梁季艰虞,宗社颠坠,西京沦覆,陷身关陇。及鼎业初基,外蕃逆命,聘问斯阻,音介莫通,睠彼机桥,将邻乌白。今者群公戮力,多难廓清,轻传入郛,无劳假道。周朝敦其继好,骖驾归来,欣此朝闻,庶欢昏定。报施徒语,曾莫辅仁,人之云亡,殄悴斯在,奄焉薨殒,倍增伤悼。津门之恸空在,恒岫之切不追,静言念之,心焉如割。宜隆懋典,以协徽猷。可赠侍中、假黄钺、都督中外诸军事、太宰、扬州牧。给东园温明秘器,九旒銮辂,黄屋左纛,武贲班剑百人,辒辌车,前后部羽葆鼓吹。葬送之仪,一依汉东平宪王、齐豫章文献王故事。仍遣大司空持节迎护丧事,大鸿胪副其羽卫,殡送所须,随由备办。''谥曰献。无子,世祖以第七皇子伯信为嗣。

南康愍王昙朗,高祖母弟忠壮王休先之子也。休先少倜傥有大志,梁简文之在东宫,深被知遇。太清中既纳侯景,有事北方,乃使休先召募得千馀人,授文德主帅,顷之卒。高祖之有天下也,每称休先曰:``此弟若存,河、洛不足定也。''梁敬帝即位,追赠侍中、使持节、骠骑将军、南徐州刺史,封武康县公,邑一千户。高祖受禅,追赠侍中、车骑大将军、司徒,封南康郡王,邑二千户,谥曰忠壮。

昙朗少孤,尤为高祖所爱,宠逾诸子。有胆力,善绥御。侯景平后,起家为著作佐郎。高祖北济江,围广陵,宿预人东方光据乡建义,乃遣昙朗与杜僧明自淮入泗应赴之。齐援大至,昙朗与僧明筑垒抗御。寻奉命班师,以宿预义军三万家济江。高祖诛王僧辩,留昙朗镇京口,知留府事。绍泰元年,除中书侍郎,监南徐州。

二年,徐嗣徽、任约引齐寇攻逼京邑,寻而请和,求高祖子侄为质。时四方州郡并多未宾,京都虚弱,粮运不断,在朝文武咸愿与齐和亲,高祖难之,而重违众议,乃言于朝曰:``孤谬辅王室,而使蛮夷猾夏,不能戡殄,何所逃责。今在位诸贤,且欲息肩偃武,与齐和好,以静边疆,若违众议,必谓孤惜子侄,今决遣昙朗,弃之寇庭。且齐人无信,窥窬不已,谓我浸弱,必当背盟。齐寇若来,诸君须为孤力斗也。''高祖虑昙朗惮行,或奔窜东道,乃自率步骑往京口迎之,以昙朗还京师,仍使为质于齐。

齐果背约,复遣萧轨等随嗣徽渡江,高祖与战,大破之,虏萧轨、东方老等。齐人请割地并入马牛以赎之,高祖不许。及轨等诛,齐人亦害昙朗于晋阳,时年二十八。是时既与齐绝,弗之知也。高祖践祚,犹以昙朗袭封南康郡王,奉忠壮王祀,礼秩一同皇子。天嘉二年,齐人结好,方始知之。世祖诏曰:``夫追远慎终,抑闻前诰。南康王昙朗,明哲懋亲,蕃维是属,入质北齐,用纾时难。皇运兆兴,未获旋反,永言跂予,日夜不忘。齐使始至,凶问奄及,追怀痛悼,兼倍常情,宜隆宠数,以光恒序。可赠侍中、安东将军、开府仪同三司、南徐州刺史,谥曰愍。''乃遣兼郎中令随聘使江德藻、刘师知迎昙朗丧柩,以三年春至都。

初,昙朗未质于齐,生子方泰、方庆。及将适齐,以二妾自随,在北又生两子:方华、方旷,亦同得还。

方泰少粗犷,与诸恶少年群聚,游逸无度,世祖以南康王故,特宽贳之。天嘉元年,诏曰:``南康王昙朗,出隔齐庭,反身莫测,国庙方修,奠飨须主,可以长男方泰为南康世子,嗣南康王。''后闻昙朗薨,于是袭爵南康嗣王。寻为仁威将军、丹阳尹,置佐史。太建四年,迁使持节、都督广、衡、交、越、成、定、明、新、合、罗、德、宜、黄、利、安、建、石、崖十九州诸军事、平越中郎将、广州刺史。为政残暴,为有司所奏,免官。寻起为仁威将军,置佐史。六年,授持节、都督豫章郡诸军事、豫章内史。在郡不修民事,秩满之际,屡放部曲为劫,又纵火延烧邑居,因行暴掠,驱录富人,征求财贿。代至,又淹留不还。至都,诏以为宗正卿,将军、佐史如故。未拜,为御史中丞宗元饶所劾,免官,以王还第。

十一年,起为宁远将军,直殿省。寻加散骑常侍,量置佐史。其年八月,高宗幸大壮观,因大阅武,命都督任忠领步骑十万,陈于玄武湖,都督陈景领楼舰五百,出于瓜步江,高宗登玄武门观,宴群臣以观之。因幸乐游苑,设丝竹会。仍重幸大壮观,集众军振旅而还。是时方泰当从,启称所生母疾,不行,因与亡命杨钟期等二十人,微服往民间,淫人妻,为州所录。又率人仗抗拒,伤禁司,为有司所奏。上大怒,下方泰狱。方泰初但承行淫,不承拒格禁司,上曰不承则上刑,方泰乃投列承引。于是兼御史中丞徐君敷奏曰:``臣闻王者之心,匪漏网而私物,至治之本,无屈法而申慈。谨案南康王陈方泰宗属虽远,幸托葭莩,刺举莫成,共治罕绩。圣上弘以悔往,许其录用,宫闱寄切,宿卫是尸。岂有金门旦启,玉舆晓跸,百司驰骛,千队腾骧,惮此翼从之劳,亡兴晨昏之请?翻以危冠淇上,袨服桑中,臣子之諐,莫斯为大,宜从霜简,允置秋官。臣等参议,请依见事,解方泰所居官,下宗正削爵土。谨以白简奏闻。''上可其奏。寻复本官爵。祯明初,迁侍中,将军如故。

三年,隋师济江,方泰与忠武将军南豫州刺史樊猛、左卫将军蒋元逊领水军于白下,往来断遏江路。隋遣行军元帅、长史高颎领船舰溯流当之,猛及元逊并降,方泰所部将士离散,乃弃船走。及台城陷,与后主俱入关。隋大业中为掖令。

方庆少清警,涉猎书传。及长,有干略。天嘉中,封临汝县侯。寻为给事中、太子洗马,权兼宗正卿,直殿省。太建九年,出为轻车将军、假节、都督定州诸军事、定州刺史。秩满,又为散骑常侍,兼宗正卿。至德二年,进号智武将军、武州刺史。初,广州刺史马靖久居岭表,大得人心,士马强盛,朝廷疑之。至是以方庆为仁威将军、广州刺史,以兵袭靖。靖诛,进号宣毅将军。方庆性清谨,甚得民和。四年,进号云麾将军。

祯明三年,隋师济江东衡州刺史王勇遣高州刺史戴智烈将五百骑迎方庆,欲令承制总督征讨诸军事。是时隋行军总管韦洸帅兵度岭,宣隋文帝敕云:``若岭南平定,留勇与豊州刺史郑万顷且依旧职。''方庆闻之,恐勇卖己,乃不从,率兵以拒智烈。智烈与战,败之,斩方庆于广州,虏其妻子。

王勇,太建中为晋陵太守,在职有能名。方庆之袭马靖也,朝廷以勇为超武将军、东衡州刺史,领始兴内史,以为方庆声势。靖诛,以功封龙阳县子。及隋军临江,诏授勇使持节、光胜将军、总督衡、广、交、桂、武等二十四州诸军事、平越中郎将,仍入援。会京城陷、勇因移檄管内,征兵据守,使其同产弟邓暠将兵五千,顿于岭上。又遣使迎方庆,欲假以为名,而自执兵要。及方庆败绩,虏其妻子,收其赀产,分赏将帅。又令其将王仲宣、曾孝武迎西衡州刺史衡阳王伯信,伯信惧,奔于清远郡,孝武追杀之。是时韦洸兵已上岭,豊州刺史郑万顷据州不受勇召,而高梁女子洗氏举兵以应隋军,攻陷傍郡,勇计无所出,乃以其众降。行至荆州,道病卒,隋赠大将军、宋州刺史,归仁县公。

郑万顷,荥阳人,梁司州刺史绍叔之族子也。父旻,梁末入魏。万顷通达有材干,周武帝时为司城大夫,出为温州刺史。至德中,与司马消难来奔。寻拜散骑常侍、昭武将军、豊州刺史。在州甚有惠政,吏民表请立碑,诏许焉。

初,万顷之在周,深被隋文帝知遇,及隋文践祚,常思还北。及王勇之杀方庆,万顷乃率州兵拒勇,遣使由间道降于隋军。拜上仪同,寻卒。

史臣曰:献、愍二王,联华霄汉,或壤子之昵,或犹子之宠,而机桥为阻,骖驾无由,有隔于休辰,终之以早世。悲夫!

\hypertarget{header-n4502}{%
\subsubsection{卷九}\label{header-n4502}}

陈拟陈详陈慧纪

陈拟,字公正,高祖疏属也。少孤贫,性质直强记,高祖南征交趾,拟从焉。又进讨侯景,至豫章,以拟为罗州刺史,与胡颖共知后事,并应接军粮。高祖作镇硃方,拟除步兵校尉、曲阿令。绍泰元年,授贞威将军、义兴太守。二年,入知卫尉事,除员外散骑常侍、明威将军、雍州刺史资,监南徐州。

高祖践祚,诏曰:``维城宗子,实固有周,盘石懿亲,用隆大汉。故会盟则异姓为后,启土则非刘勿王,所以纠合枝干,广树蕃屏,前王懋典,列代恒规。从子持节、员外散骑常侍、明威将军、雍州刺史、监南徐州拟,持节、通直散骑侍郎、贞威将军、北徐州刺史褒,从子晃、炅,从孙假节、员外散骑常侍、明威将军訬,假节、信威将军、北徐州刺史吉阳县开国侯諠,假节、通直散骑侍郎、信武将军祏,假节、散骑侍郎、雄信将军、青州刺史、广梁太守详,贞戚将军、通直散骑侍郎慧纪,从孙敬雅、敬泰,并枝戚密近,劬劳王室,宜列河山,以光利建。拟可永修县开国侯,褒钟陵县开国侯,晃建城县开国侯,炅上饶县开国侯,訬虔化县开国侯,諠仍前封,祏豫章县开国侯,详遂兴县开国侯,慧纪宜黄县开国侯,敬雅宁都县开国侯,敬泰平固县开国侯,各邑五百户。''拟寻除轻车将军,兼南徐州刺史,常侍如故。其年,授通直散骑常侍、中领军。三年,复以本官监南徐州。世祖嗣位,除丹阳尹,常侍如故。坐事,又以白衣知郡,寻复本职。天嘉元年卒,时年五十八。赠领军将军,凶事所须,并官资给。谥曰定。二年,配享高祖庙廷。子党嗣。

陈详,字文几,少出家为桑门。善书记,谈论清雅。高祖讨侯景,召详,令反初服,配以兵马,从定京邑。高祖东征杜龛,详别下安吉、原乡、故鄣三县。龛平,以功授散骑侍郎、假节、雄信将军、青州刺史资,割故鄣、广德置广梁郡,以详为太守。高祖践祚,改广梁为陈留,又以为陈留太守。永定二年,封遂兴县侯,食邑五百户。其年除明威将军、通直散骑常侍。三年,随侯安都破王琳将常众爱于宫亭湖。世祖嗣位,除宣城太守,将军如故。王琳下据栅口,详随吴明彻袭湓城,取琳家口,不克,因入南湖,自鄱阳步道而归。琳平,详与明彻并无功。天嘉元年,随例增邑并前一千五百户。仍除通直散骑常侍,兼右卫将军。三年,出为假节、都督吴州诸军事、仁威将军、吴州刺史。

周迪据临川举兵,详自州从他道袭迪于濡城别营,获其妻子。迪败走,详还复本镇。五年,周迪复出临川,乃以详为都督,率水步讨迪。军至南城,与贼相遇,战败,死之,时年四十二。以所统失律,无赠谥。子正理嗣。

陈慧纪,字元方,高祖之从孙也。涉猎书史,负才任气。高祖平侯景,慧纪从焉。寻配以兵马。景平,从征杜龛。除贞威将军、通直散骑常侍。高祖践祚,封宜黄县侯,邑五百户,除黄门侍郎。世祖即位,出为安吉县令。迁明威将军军副。司空章昭达征安蜀城,慧纪为水军都督,于荆州烧青泥船舻。光大元年,以功除持节、通直散骑常侍、宣远将军、豊州刺史,增邑并前一千户。太建十年,吴明彻北讨败绩,以慧纪为持节、智武将军、缘江都督、兗州刺史,增邑并前二千户,馀如故。周军乘胜据有淮南,江外骚扰,慧纪收集士卒,自海道还都。寻除使持节、散骑常侍、宣毅将军、都督郢、巴二州诸军事、郢州刺史,增邑并前二千五百户。至德二年,迁使持节、散骑常侍、云麾将军、都督荆、信二州诸军事、荆州刺史,赐女伎一部,增邑并前三千户。祯明元年,萧琮尚书左仆射安平王萧岩、晋熙王萧献等,率其部众男女二万馀口,诣慧纪请降,慧纪以兵迎之。其年,以应接之功,加侍中、金紫光禄大夫、开府仪同三司、征西将军、增邑并前六千户,馀如故。

及隋师济江,元帅清河公杨素下自巴硖,慧纪遣其将吕忠肃、陆伦等拒之,战败,素进据马头。是时,隋将韩擒虎及贺若弼等已济江据蒋山,慧纪闻之,留其长史陈文盛等居守,身率将士三万人,楼船千馀乘,沿江而下,欲趣台城。至汉口,为秦王军所拒,不得进,因与湘州刺史晋熙王叔文、巴州刺史毕宝等请降。入隋,依例授仪同三司。顷之卒。子正平,颇有文学。

史臣曰:《诗》云:``宗子维城,无俾城坏。''又曰:``绵绵瓜瓞,葛藟累之。''西京皆豊、沛故人,东都亦南阳多显,有以哉!

\hypertarget{header-n4513}{%
\subsubsection{卷十}\label{header-n4513}}

赵知礼蔡景历刘师知谢岐

赵知礼,字齐旦,天水陇西人也。父孝穆,梁候官令。知礼涉猎文史,善隶书。高祖之讨元景仲也,或荐之,引为记室参军。知礼为文赡速,每占授军书,下笔便就,率皆称旨。由是恒侍左右,深被委任,当时计划,莫不预焉。知礼亦多所献替。高祖平侯景,军至白茅湾,上表于梁元帝及与王僧辩论述军事,其文并知礼所制。

侯景平,授中书侍郎,封始平县子,邑三百户。高祖为司空,以为从事中郎。高祖入辅,迁给事黄门侍郎,兼卫尉卿。高祖受命,迁通直散骑常侍,直殿省。寻迁散骑常侍,守太府卿,权知领军事。天嘉元年,进爵为伯,增邑通前七百户。王琳平,授持节、督吴州诸军事、明威将军、吴州刺史。

知礼沈静有谋谟,每军国大事,世祖辄令玺书问之。秩满,为明威将军、太子右卫率。迁右卫将军,领前军将军。六年卒,时年四十七。诏赠侍中,谥曰忠。子允恭嗣。

蔡景历,字茂世,济阳考城人也。祖点,梁尚书左民侍郎。父大同,轻车岳阳王记室参军,掌京邑行选。景历少俊爽,有孝行。家贫好学,善尺牍,工草隶。解褐诸王府佐,出为海阳令,为政有能名。侯景乱,梁简文帝为景所幽,景历与南康嗣王萧会理谋,欲挟简文出奔,事泄见执,贼党王伟保护之,获免。因客游京口。侯景平,高祖镇硃方,素闻其名,以书要之。景历对使人答书,笔不停缀,文不重改。曰:

蒙降札书,曲垂引逮,伏览循回,载深欣畅。窃以世求名骏,行地能致千里,时爱奇宝,照车遂有径寸。但《云》、《咸》斯奏,自辍《巴渝》,杞梓方雕,岂盼樗枥。仰惟明将军使君侯节下,英才挺茂,雄姿秀拔,运属时艰,志匡多难,振衡岳而绥五岭,涤赣源而澄九派,带甲十万,强弩数千,誓勤王之师,总义夫之力,鲸鲵式剪,役不逾时,氛雾廓清,士无血刃。虽汉诛禄、产,举朝实赖绛侯,晋讨约、峻,中外一资陶牧,比事论功,彼奚足算。加以抗威兗服,冠盖通于北门,整旆徐方,咏歌溢于东道,能使边亭卧鼓,行旅露宿,巷不拾遗,市无异价,洋洋乎功德政化,旷古未俦,谅非肤浅所能殚述。是以天下之人,向风慕义,接踵披衿,杂遝而至矣。或帝室英贤,贵游令望,齐、楚秀异,荆、吴岐嶷。武夫则猛气纷纭,雄心四据,陆拔山岳,水断虬龙,六钧之弓,左右驰射,万人之剑,短兵交接,攻垒若文鸯,焚舰如黄盖,百战百胜,貔貅为群。文人则通儒博识,英才伟器,雕丽晖焕,摛掞绚藻,子云不能抗其笔,元瑜无以高其记,尺翰驰而聊城下,清谈奋而嬴军却。复有三河辩客,改哀乐于须臾,六奇谋士,断变反于倏忽。治民如子贱,践境有成,折狱如仲由,片辞从理。直言如毛遂,能厉主威,衔使若相如,不辱君命。怀忠抱义,感恩徇己,诚断黄金,精贯白日,海内雄贤,牢笼斯备。明将军彻鞍下马,推案止食,申爵以荣之,筑馆以安之,轻财重气,卑躬厚士,盛矣哉!盛矣哉!

抑又闻之,战国将相,咸推引宾游,中代岳牧,并盛延僚友,济济多士,所以成将军之贵。但量能校实,称才任使,员行方止,各尽其宜,受委责成,谁不毕力。至如走贱,亡庸人耳。秋冬读书,终惭专学,刀笔为吏,竟阙异等。衡门衰素,无所闻达,薄宦轻资,焉能远大。自阳九遘屯,天步艰阻,同彼贵仕,溺于巨寇,亟邻危殆,备践薄冰。今王道中兴,殷忧启运,获存微命,足为幸甚,方欢饮啄,是谓来苏。然皇銮未反,宛、洛芜旷,四壁固三军之馀,长夏无半菽之产,遨游故人,聊为借贷,属此乐土,洵美忘归。窃服高义,暂谒门下,明将军降以颜色,二三士友假其馀论,菅蒯不弃,折简赐留,欲以鸡鹜厕鸳鸿于池沼,将移瓦砾参金碧之声价。昔折胁游秦,忽逢盼采,檐簦入赵,便致留连,今虽羁旅,方之非匹,樊林之贲,何用克堪。但眇眇纤萝,凭乔松以自耸,蠢蠢轻蚋,托骖尾而远骛。窃不自涯,愿备下走,且为腹背之毛,脱充鸣吠之数,增荣改观,为幸已多。海不厌深,山不让高,敢布心腹,惟将军览焉。

高祖得书,甚加钦赏。仍更赐书报答,即日板征北府中记室参军,仍领记室。

衡阳献王昌时为吴兴郡,昌年尚少,吴兴王之乡里,父老故人,尊卑有数,高祖恐昌年少,接对乖礼,乃遣景历辅之。承圣中,授通直散骑侍郎,还掌府记室。高祖将讨王僧辩,独与侯安都等数人谋之,景历弗之知也。部分既毕,召令草檄,景历援笔立成,辞义感激,事皆称旨。僧辩诛,高祖辅政,除从事中郎,掌记室如故。绍泰元年,迁给事黄门侍郎,兼掌相府记室。高祖受禅,迁秘书监,中书通事舍人,掌诏诰。永定二年,坐妻弟刘淹诈受周宝安饷马,为御史中丞沈炯所劾,降为中书侍郎,舍人如故。

三年,高祖崩,时外有强寇,世祖镇于南皖,朝无重臣,宣后呼景历及江大权、杜棱定议,乃秘不发丧,疾召世祖。景历躬共宦者及内人,密营敛服。时既暑热,须治梓宫,恐斤斧之声或闻于外,仍以蜡为秘器。文书诏诰,依旧宣行。世祖即位,复为秘书监,舍人如故。以定策功,封新豊县子,邑四百户。累迁散骑常侍。世祖诛侯安都,景历劝成其事。天嘉三年,以功迁太子左卫率,进爵为侯,增邑百户,常侍、舍人如故。六年,坐妻兄刘洽依倚景历权势,前后奸讹,并受欧阳武威饷绢百匹,免官。

废帝即位,起为镇东鄱阳王谘议参军,兼太府卿。华皎反,以景历为武胜将军、吴明彻军司。皎平,明彻于军中辄戮安成内史杨文通,又受降人马仗有不分明,景历又坐不能匡正,被收付治。久之,获宥,起为镇东鄱阳王谘议参军。

高宗即位,迁宣惠豫章王长史,带会稽郡守,行东扬州府事。秩满,迁戎昭将军、宣毅长沙王长史、寻阳太守,行江州府事,以疾辞,遂不行。入为通直散骑常侍、中书通事舍人,掌诏诰,仍复封邑。迁太子左卫率,常侍、舍人如故。

太建五年,都督吴明彻北伐,所向克捷,与周将梁士彦战于吕梁,大破之,斩获万计,方欲进图彭城。是时高宗锐意河南,以为指麾可定,景历谏称师老将骄,不宜过穷远略。高宗恶其沮众,大怒,犹以朝廷旧臣,不深罪责,出为宣远将军、豫章内史。未行,为飞章所劾,以在省之日,赃污狼藉,帝令有司按问,景历但承其半。于是御史中丞宗元饶奏曰:``臣闻士之行己,忠以事上,廉以持身,苟违斯道,刑兹罔赦。谨按宣远将军、豫章内史新豊县开国侯景历,因藉多幸,豫奉兴王,皇运权舆,颇参缔构。天嘉之世,赃贿狼藉,圣恩录用,许以更鸣,裂壤崇阶,不远斯复。不能改节自励,以报曲成,遂乃专擅贪污,彰于远近,一则已甚,其可再乎?宜置刑书,以明秋宪。臣等参议,以见事免景历所居官,下鸿胪削爵土。谨奉白简以闻。''诏曰``可。''于是徙居会稽。及吴明彻败,帝思景历前言,即日追还,复以为征南鄱阳王谘议参军。数日,迁员外散骑常侍,兼御史中丞,复本封爵,入守度支尚书。旧式拜官在午后,景历拜日,适值舆驾幸玄武观,在位皆侍宴,帝恐景历不豫,特令早拜,其见重如此。

是岁,以疾卒官,时年六十。赠太常卿,谥曰敬。十三年,改葬,重赠中领军。祯明元年,配享高祖庙庭。二年,舆驾亲幸其宅,重赠景历侍中、中抚将军,谥曰忠敬,给鼓吹一部,并于墓所立碑。

景历属文,不尚雕靡,而长于叙事,应机敏速,为当世所称。有文集三十卷。

刘师知,沛国相人也。家世素族。祖奚之,齐晋安王谘议参军,淮南太守,有能政,齐武帝手诏频褒赏。父景彦,梁尚书左丞、司农卿。师知好学,有当世才。博涉书史,工文笔,善仪体,台阁故事,多所详悉。梁世历王府参军。绍泰初,高祖入辅,以师知为中书舍人,掌诏诰。是时兵乱之后,礼仪多阙,高祖为丞相及加九锡并受禅,其仪注并师知所定焉。高祖受命,仍为舍人。性疏简,与物多忤,虽位宦不迁,而委任甚重,其所献替,皆有弘益。

及高祖崩,六日成服,朝臣共议大行皇帝灵座侠御人所服衣服吉凶之制,博士沈文阿议,宜服吉服。师知议云:``既称成服,本备丧礼,灵筵服物,皆悉缟素。今虽无大行侠御官事,按梁昭明太子薨,成服侠侍之官,悉著縗斩,唯著铠不异,此即可拟。愚谓六日成服,侠灵座须服縗绖。''中书舍人蔡景历亦云:``虽不悉准,按山陵有凶吉羽仪,成服唯凶无吉,文武侠御,不容独鸣玉珥貂,情礼二三,理宜縗斩''。中书舍人江德藻、谢岐等并同师知议。文阿重议云``检晋、宋《山陵仪》:`灵舆梓宫降殿,各侍中奏。'又《成服仪》称:`灵舆梓宫容侠御官及香橙。'又检《灵舆梓宫进止仪》称:`直灵侠御吉服,在吉卤簿中。'又云:`梓宫侠御衰服,在凶卤簿中。'是则在殿吉凶两侠御也。''时以二议不同,乃启取左丞徐陵决断。陵云:``梓宫祔山陵,灵筵祔宗庙,有此分判,便验吉凶。按山陵卤簿吉部伍中,公卿以下导引者,爰及武贲、鼓吹、执盖、奉车,并是吉服,岂容侠御独为縗鸑邪?断可知矣。若言公卿胥吏并服縗苴,此与梓宫部伍有何差别?若言文物并吉,司事者凶,岂容衽绖而奉华盖,縗衣而升玉辂邪?同博士议。''师知又议曰:``左丞引梓宫祔山陵,灵延祔宗庙,必有吉凶二部,成服不容上凶,博士犹执前断,终是山陵之礼。若龙驾启殡,銮舆兼设,吉凶之仪,由来本备,准之成服,愚有未安。夫丧礼之制,自天子达。按王文宪《丧服明记》云:`官品第三,侍灵人二十。官品第四,下达士礼,侍灵之数,并有十人。皆白布袴褶,著白绢帽。内丧女侍数如外,而著齐縗。或问内外侍灵是同,何忽縗服有异?答云,若依君臣之礼,则外侍斩,内侍齐。顷世多故,礼随事省。诸侯以下,臣吏盖微,至于侍奉,多出义附,君臣之节不全,縗冠之费实阙,所以因其常服,止变帽而已。妇人侍者,皆是卑隶,君妾之道既纯,服章所以备矣。'皇朝之典,犹自不然,以此而推,是知服斩。彼有侍灵,则犹侠御,既著白帽,理无彤服。且梁昭明《仪注》,今则见存,二文显证,差为成准。且礼出人情,可得消息。凡人有丧,既陈延几,繐帷灵屏,变其常仪,芦箔草庐,即其凶礼。堂室之内,亲宾具来,齐斩麻缌,差池哭次,玄冠不吊,莫非素服。岂见门生故吏,绡縠间趋,左姬右姜,红紫相糅?况四海遏密,率土之情是同,三军缟素,为服之制斯一。逐使千门旦启,非涂垩于彤闱,百僚戾止,变服粗于硃AX,而耀金在列,鸣玉节行,求之怀抱,固为未惬,准以礼经,弥无前事。岂可成服之仪,譬以山陵之礼?葬既始终已毕,故有吉凶之仪,所谓成服,本成丧礼,百司外内,皆变吉容,侠御独不,何谓成服?若灵无侠御则已,有则必应縗服。''谢岐议曰:``灵延祔宗庙,梓宫祔山陵,实如左丞议。但山陵卤簿,备有吉凶,从灵舆者仪服无变,从梓宫者皆服苴縗。爰至士礼,悉同此制,此自是山陵之仪,非关成服。今谓梓宫灵扆,共在西阶,称为成服,亦无卤簿,直是爰自胥吏,上至王公,四海之内,必备縗绖,案梁昭明太子薨,略是成例,岂容凡百士庶,悉皆服重,而侍中至于武卫,最是近官,反鸣玉纡青,与平吉不异?左丞既推以山陵事,愚意或谓与成服有殊。若尔日侠御,文武不异,维侍灵之人,主书、宣传、齐干、应敕,悉应不改。''蔡景历又议云:``侠御之官,本出五百,尔日备服居庐,仍于本省,引上登殿,岂应变服貂玉、若别摄馀官,以充簪珥,则尔日便有不成服者。山陵自有吉凶二议,成服凶而不吉,犹依前议,同刘舍人。''德藻又议云:``愚谓祖葬之辰,始终永毕,达官有追赠,须表恩荣,有吉卤簿,恐由此义,私家放斅,因以成俗。上服本变吉为凶,理不应犹袭纨绮。刘舍人引王卫军《丧仪》及检梁昭明故事,此明据已审,博士、左丞乃各尽事衷,既未取证,须更询详,宜谘八座、詹事、太常、中丞及中庶诸通袁枢、张种、周弘正、弘让、沈炯、孔奂。''时八座以下,并请:``案群议,斟酌旧仪,梁昭明太子《丧成服仪注》,明文见存,足为准的。成服日,侍官理不容犹从吉礼。其葬礼分吉,自是山陵之时,非关成服之日。愚谓刘舍人议,于事为允。''陵重答云:``老病属纩,不能多说,古人争议,多成怨府,傅玄见尤于晋代,王商取陷于汉朝,谨自三缄,敬同高命。若万一不死,犹得展言,庶与朝贤更申扬搉。''文阿犹执所见,众议不能决,乃具录二议奏闻,从师知议。

寻迁鸿胪卿,舍人如故。天嘉元年,坐事免。初,世祖敕师知撰《起居注》,自永定二年秋至天嘉元年冬,为十卷。起为中书舍人,复掌诏诰。天康元年,世祖不豫,师知与尚书仆射到仲举等入侍医药。世祖崩,预受顾命。及高宗为尚书令,入辅,光大元年,师知与仲举等遣舍人殷不佞矫诏令高宗还东府,事觉,于北狱赐死。

谢岐,会稽山阴人也。父达,梁太学博士。岐少机警,好学,见称于梁世。为尚书金部郎,山阴令。侯景乱,岐流寓东阳。景平,依于张彪。彪在吴郡及会稽,庶事一以委之。彪每征讨,恒留岐监郡,知后事。彪败,高祖引岐参预机密,以为兼尚书右丞。时军旅屡兴,粮储多阙,岐所在干理,深被知遇。永定元年,为给事黄门侍郎、中书舍人,兼右丞如故。天嘉二年卒,赠通直散骑常侍。

岐弟峤,笃学,为世通儒。

史臣曰:高祖开基创业,克定祸乱,武猛固其立功,文翰亦乃展力。赵知礼、蔡景历早识攀附,预缔构之臣焉。刘师知博涉多通,而暗于机变,虽欲存乎节义,终陷极刑,斯不智矣。

\hypertarget{header-n4537}{%
\subsubsection{卷十一}\label{header-n4537}}

王冲王通弟劢袁敬兄子枢

王冲,字长深,琅邪临沂人也。祖僧衍,齐侍中。父茂璋,梁给事黄门侍郎。冲母,梁武帝妹新安穆公主,卒于齐世,武帝以冲偏孤,深所钟爱。年十八,起家梁秘书郎。寻为永嘉太守。入为太子舍人,以父忧去职。服阕,除太尉临川王府外兵参军、东宫领直。累迁太子洗马、中舍人。出为招远将军、衡阳内史。迁武威将军、安成嗣王长史、长沙内史,将军如故。王薨于湘州,仍以冲监湘州事。入为太子庶子。迁给事黄门侍郎。大同三年,以帝甥赐爵安东亭侯,邑一百五十户。历明威将军、南郡太守、太子中庶子、侍中。出监吴郡,满岁即真。征为通直散骑常侍,兼左民尚书。出为明威将军、轻车当阳公府长史、江夏太守,行郢州事。迁平西邵陵王长史。转骠骑庐陵王长史、南郡太守。王薨,行州府事。梁元帝镇荆州,为镇西长史,将军、太守如故。冲性和顺,事上谨肃,习于法令,政在平理,佐籓莅人,鲜有失德,虽无赫赫之誉,久而见思,由是推重,累居二千石。又晓音乐,习歌舞,善与人交,贵游之中,声名藉甚。

侯景之乱,梁元帝于荆州承制,冲求解南郡,以让王僧辩,并献女妓十人,以助军赏。元帝授持节、督衡、桂、成、合四州诸军事、云麾将军、衡州刺史。元帝第四子元良为湘州刺史,仍以冲行州事,领长沙内史。侯景平,授翊左将军、丹阳尹。

武陵王举兵至峡口,王琳偏将陆纳等据湘州应之,冲为纳所拘。纳降,重授侍中、中权将军,量置佐史,尹如故。江陵陷,敬帝为太宰,承制以冲为左长史。绍泰中,累迁左光禄大夫、尚书右仆射。迁左仆射、开府仪同三司,侍中、将军如故。寻复领丹阳尹、南徐州大中正,给扶。

高祖受禅,解尹,以本官领左光禄大夫。未拜,改领太子少傅。文帝嗣位,解少傅,加特进、左光禄大夫。寻又以本官领丹阳尹,参撰律令。废帝即位,给亲信十人。

初,高祖以冲前代旧臣,特申长幼之敬。文帝即位,益加尊重,尝从文帝幸司空徐度宅,宴筵之上,赐以几。其见重如此。光大元年薨,时年七十六。赠侍中、司空,谥曰元简。

冲有子三十人,并致通官。第十二子瑒,别有传。

王通,字公达,琅邪临沂人也。祖份,梁左光禄大夫。父琳,司徒左长史。琳齐代娶梁武帝妹义兴长公主,有子九人,并知名。

通,梁世起家国子生,举明经,为秘书郎、太子舍人。以帝甥封武阳亭侯。累迁王府主簿、限外记室参军、司徒主簿、太子中庶子、骠骑庐陵王府给事中郎、中权何敬容府长史、给事黄门侍郎,坐事免。侯景之乱,奔于江陵,元帝以为散骑常侍,迁守太常卿。自侯景乱后,台内宫室,并皆焚烬,以通兼起部尚书,归于京师,专掌缮造。

江陵陷,敬帝承制以通为吏部尚书。绍泰元年,加侍中,尚书如故。寻为尚书右仆射,吏部如故。高祖受禅,迁左仆射,侍中如故。文帝嗣位,领太子少傅。天康元年,为翊右将军、右光禄大夫,量置佐史。废帝即位,号安右将军,又领南徐州大中正。太建元年,迁左光禄大夫。六年,加特进,侍中、将军、光禄、佐史并如故。未拜卒,时年七十二。诏赠本官,谥曰成,葬日给鼓吹一部,弟质、弟固各有传。

劢字公济,通之弟也。美风仪,博涉书史,恬然清简,未尝以利欲干怀。梁世为国子《周易》生,射策举高第,除秘书郎、太子舍人、宣惠武陵王主簿、轻车河东王功曹史。王出镇京口,劢将随之籓,范阳张缵时典选举,劢造缵言别,缵嘉其风采,乃曰:``王生才地,岂可游外府乎?''奏为太子洗马。迁中舍人,司徒左西属。出为南徐州别驾从事史。

大同末,梁武帝谒园陵,道出硃方,劢随例迎候,敕劢令从辇侧,所经山川,莫不顾问,劢随事应对,咸有故实。又从登北顾楼,赋诗,辞义清典,帝甚嘉之。时河东王为广州刺史,乃以劢为冠军河东王长史、南海太守。王至岭南,多所侵掠,因惧罪称疾,委州还朝,劢行广州府事。越中饶沃,前后守宰例多贪纵,劢独以清白著闻。入为给事黄门侍郎。侯景之乱,西奔江陵,元帝承制以为太子中庶子,掌相府管记。出为宁远将军、晋陵太守。时兵饥之后,郡中凋弊,劢为政清简,吏民便安之。征为侍中,迁五兵尚书。

及西魏寇江陵,元帝征湘州刺史宜豊侯萧循入援,以劢监湘州。江陵陷,敬帝承制以为中书令。绍泰元年加侍中。高祖为司空,以劢兼司空长史。高祖为丞相,劢兼丞相长史,侍中、中书令并如故。时吴中遭乱,民多乏绝,乃以劢监吴兴郡。及萧勃平后,又以劢旧在岭表,早有政绩,乃授使持节、都督广州等二十州诸军事、平南将军、平越中郎将、广州刺史。未行,改为衡州刺史,持节、都督并如故。王琳据有上流,衡、广携贰,劢不得之镇,留于大庾岭。天嘉元年,征为侍中、都官尚书,未拜,复为中书令。迁太子詹事,行东宫事,侍中并如故。加金紫光禄大夫,领度支尚书。废帝即位,加散骑常侍。太建元年,迁尚书右仆射。时东境大水,百姓饥馑,以劢为仁武将军、晋陵太守。在郡甚有威惠,郡人表请立碑,颂劢政绩,诏许之。征为中书监,重授尚书右仆射,领右军将军。四年五月卒,时年六十七。赠侍中、中书监,谥曰温。

袁敬,字子恭,陈郡阳夏人也。祖顗,宋侍中、吏部尚书、雍州刺史。父昂,梁侍中、司空,谥穆公。敬纯孝有风格,幼便笃学,老而无倦。释褐秘书郎,累迁太子舍人、洗马、中舍人。江陵沦覆,流寓岭表。高祖受禅,敬在广州,依欧阳頠。及頠卒,其子纥据州,将有异志,敬累谏纥,为陈逆顺之理,言甚切至,纥终不从。高宗即位,遣章昭达率众讨纥,纥将败之时,恨不纳敬言。朝廷义之,其年徵为太子中庶子、通直散骑常侍。俄转司徒左长史。寻迁左民尚书,转都官尚书,领豫州大中正。累迁太常卿、散骑常侍、金紫光禄大夫,加特进。至德三年卒,时年七十九,赠左光禄大夫,谥曰靖德。子元友嗣。弟泌自有传。兄子枢。

枢字践言,梁吴郡太守君正之子也。美容仪,性沈静,好读书,手不释卷。家世显贵,赀产充积,而枢独居处率素,傍无交往,端坐一室,非公事未尝出游,荣利之怀淡如也。起家梁秘书郎,历太子舍人,轻车河东王主簿,安前邵陵王、中军宣城王二府功曹史。侯景之乱,枢往吴郡省父,因丁父忧。时四方扰乱,人求苟免,枢居丧以至孝闻。王僧辩平侯景,镇京城,衣冠争往造请,枢独杜门静居,不求闻达。绍泰元年,征为给事黄门侍郎。未拜,除员外散骑常侍,兼侍中。二年,兼吏部尚书。其年出为吴兴太守。永定二年,征为左民尚书。未至,改侍中,掌大选事。三年,迁都官尚书,掌选如故。

枢博闻强识,明悉旧章。初,高祖长女永世公主先适陈留太守钱蕆,生子岊,主及岊并卒于梁世。高祖受命,唯公主追封。至是将葬,尚书主客请详议,欲加蕆驸马都尉,并赠岊官。枢议曰:``昔王姬下嫁,必适诸侯,同姓为主,闻于《公羊》之说,车服不系,显于诗人之篇。汉氏初兴,列侯尚主,自斯以后,降嫔素族。驸马都尉置由汉武,或以假诸功臣,或以加于戚属,是以魏曹植表驸马、奉车趣为一号。《齐职仪》曰,凡尚公主必拜驸马都尉,魏、晋以来,因为瞻准。盖以王姬之重,庶姓之轻,若不加其等级,宁可合卺而酳,所以假驸马之位,乃崇于皇女也。今公主早薨,伉俪已绝,既无礼数致疑,何须驸马之授?案杜预尚晋宣帝第二女高陵宣公主,晋武践祚,而主已亡,泰始中追赠公主,元凯无复驸马之号。梁文帝女新安穆公主早薨,天监初王氏无追拜之事。远近二例,足以据明。公主所生,既未及成人之礼,无劳此授,今宜追赠亭侯。''时以枢议为长。

天嘉元年,守吏部尚书。三年,即真。寻领右军将军,又领丹阳尹,本官如故。五年,以葬父,拜表自解,诏赐绢布五十匹,钱十万,令葬讫停宅视郡事,服阕,还复本职。其年秩满,解尹,加散骑常侍,将军、尚书并如故。是时,仆射到仲举虽参掌选事,铨衡汲引,并出于枢,其所举荐,多会上旨。谨慎周密,清白自居,文武职司,鲜有游其门者。废帝即位,迁尚书左仆射。光大元年卒,时年五十一。赠侍中、左光禄大夫,谥曰简懿。有集十卷行于世。弟宪,自有传。

史臣曰:王冲、王通并以贵游,早升清贯,而允蹈礼节,笃诚奉上,斯为美焉。王劢之襟神夷淡,袁枢之端操沉冥,虽拘放为异,而胜概一揆,古所谓名士者,盖在其人乎!

\hypertarget{header-n4558}{%
\subsubsection{卷十二}\label{header-n4558}}

沈众袁泌刘仲威陆山才王质韦载族弟翙

沈众,字仲师,吴兴武康人也。祖约,梁特进。父旋,梁给事黄门侍郎。众好学,颇有文词,起家梁镇卫南平王法曹参军、太子舍人。是时,梁武帝制《千字诗》,众为之注解。与陈郡谢景同时召见于文德殿,帝令众为《竹赋》,赋成,奏,帝善之,手敕答曰:``卿文体翩翩,可谓无忝尔祖。''当阳公萧大心为郢州刺史,以众为限内记室参军。寻除镇南湘东王记室参军。迁太子中舍人,兼散骑常侍。聘魏,还,迁骠骑庐陵王谘议参军,舍人如故。

侯景之乱,众表于梁武,称家代所隶故义部曲,并在吴兴,求还召募以讨贼,梁武许之。及景围台城,众率宗族及义附五千馀人,入援京邑,顿于小航,对贼东府置阵,军容甚整,景深惮之。梁武于城内遥授众为太子右卫率。京城陷,众降于景。景平,西上荆州,元帝以为太子中庶子、本州大中正。寻迁司徒左长史。江陵陷,为西魏所虏,寻而逃还,敬帝承制授御史中丞。绍泰元年,除侍中,迁左民尚书。高祖受命,迁中书令,中正如故。高祖以众州里知名,甚敬重之,赏赐优渥,超于时辈。

众性吝啬,内治产业,财帛以亿计,无所分遗。其自奉养甚薄,每于朝会之中,衣裳破裂,或躬提冠屦。永定二年,兼起部尚书,监起太极殿。恒服布袍芒屩,以麻绳为带,又携干鱼蔬菜饭独啖之,朝士共诮其所为。众性狷急,于是忿恨,遂历诋公卿,非毁朝廷。高祖大怒,以众素有令望,不欲显诛之,后因其休假还武康,遂于吴中赐死,时年五十六。

袁泌,字文洋,左光禄大夫敬之弟也。清正有干局,容体魁岸,志行修谨。释褐员外散骑侍郎,历诸王府佐。

侯景之乱,泌欲求为将。是时泌兄君正为吴郡太守,梁简文板泌为东宫领直,令往吴中召募士卒。及景围台城,泌率所领赴援。京城陷,退保东阳,景使兵追之,乃自会稽东岭出湓城,依于鄱阳嗣王萧范。范卒,泌乃降景。

景平,王僧辩表泌为富春太守,兼丹阳尹。贞阳侯僭位,以泌为侍中,奉使于齐。高祖受禅,王琳据有上流,泌自齐从梁永嘉王萧庄达琳所。及庄僭立,以泌为侍中、丞相长史。天嘉二年,泌与琳辅庄至于栅口,琳军败,众皆奔散,唯泌独乘轻舟送庄达于北境,属庄于御史中丞刘仲威,令共入齐,然后拜辞而归,诣阙请罪,文帝深义之。

寻授宁远始兴王府法曹参军,转谘议参军,除通直散骑常侍,兼侍中,领豫州大中正。聘于周,使还,授散骑常侍,御史中丞,其中正如故。高宗入辅,以泌为云旗将军、司徒左长史。光大元年卒,年五十八。临终戒其子蔓华曰:``吾于朝廷素无功绩,瞑目之后,敛手足旋葬,无得辄受赠谥。''其子述泌遗意,表请之,朝廷不许,赠金紫光禄大夫,谥曰质。

刘仲威,南阳涅阳人也。祖虬,齐世以国子博士征,不就。父之迟,荆州治中从事史。仲威少有志气,颇涉文史。梁丞圣中为中书侍郎。萧庄伪署御史中丞,随庄入齐,终于鄴中。

仲威从弟广德,亦好学,负才任气。父之亨,梁安西湘东王长史、南郡太守。广德承圣中以军功官至给事黄门侍郎、湘东太守。荆州陷后,依于王琳。琳平,文帝以广德为宁远始兴王府限外记室参军,仍领其旧兵。寻为太尉侯瑱湘州府司马,历乐山、豫章二郡太守,新安内史。光大中,假节、员外散骑常侍、云旗将军、河东太守。太建元年卒于郡,时年四十三,赠左卫将军。

陆山才,字孔章,吴郡吴人也。祖翁宝,梁尚书水部郎。父泛,散骑常侍。山才少倜傥,好尚文史,范阳张缵,缵弟绾,并钦重之。起家王国常侍,迁外兵参军。寻以父疾,东归侍养。承圣元年,王僧辩授山才仪同府西曹掾。高祖诛僧辩,山才奔会稽依张彪。彪败,乃归高祖。

绍泰中,都督周文育出镇南豫州,不知书疏,乃以山才为长史,政事悉以委之。文育南讨,克萧勃,擒欧阳頠,计画多出山才。及文育西征王琳,留山才监江州事,仍镇豫章。文育与侯安都于沌口败绩,余孝顷自新林来寇豫章,山才收合馀众,依于周迪。擒余孝顷、李孝钦等,遣山才自都阳之乐安岭东道送于京师。除中书侍郎。复由乐安岭绥抚南川诸郡。

文育重镇豫章金口,山才复为贞威将军、镇南长史、豫章太守。文育为熊昙朗所害,昙朗囚山才等,送于王琳。未至,而侯安都败琳将常众爱于宫亭湖,由是山才获反,除贞威将军、新安太守。为王琳未平,留镇富阳,以捍东道。入为员外散骑常侍,迁宣惠始兴王长史,行东扬州事。

侯安都讨留异,山才率王府之众从焉。异平,除明威将军、东阳太守。入为镇东始兴王长史,带会稽郡丞,行东扬州事。未拜,改授散骑常侍,兼度支尚书,满岁为真。

高宗南征周迪,以山才为军司。迪平,复职。余孝顷自海道袭晋安,山才又以本官之会稽,指授方略。还朝,坐侍宴与蔡景历言语过差,为有司所奏,免官。寻授散骑常侍,迁云旗将军、西阳武昌二郡太守。天康元年卒,时年五十八。赠右卫将军,谥曰简子。

王质,字子贞,右光禄大夫通之弟也。少慷慨,涉猎书史。梁世以武帝甥封甲口亭侯,补国子《周易》生,射策高第。起家秘书郎、太子舍人、尚书殿中郎。遭母忧,居丧以孝闻。服阕,除太子洗马、东宫领直。累迁中舍人、庶子。

太清元年,除假节、宁远将军,领东宫兵,从贞阳侯北伐。及贞阳败绩,质脱身逃还。侯景于寿阳构逆,质又领舟师随众军拒之。景军济江,质便退走。寻领步骑顿于宣阳门外。景军至京师,质不战而溃,乃翦发为桑门,潜匿人间。及柳仲礼等会援京邑,军据南岸,质又收合馀众从之。京城陷后,西奔荆州,元帝承制,以质为右长史,带河东太守。俄迁侍中。寻出为持节、都督吴州诸军事、宁远将军、吴州刺史,领鄱阳内史。荆州陷,侯瑱镇于湓城,与质不协,遣偏将羊亮代质,且以兵临之,质率所部度信安岭,依于留异。文帝镇会稽,以兵助质,令镇信安县。

永定二年,高祖命质率所部逾岭出豫章,随都督周文育以讨王琳。质与琳素善,或谮云于军中潜信交通,高祖命周文育杀质,文育启请救之,获免。寻授散骑常侍、晋陵太守。

文帝嗣位,征守五兵尚书。高宗为扬州刺史,以质为仁威将军、骠骑府长史。天嘉二年,除晋安太守。高宗辅政,以为司徒左长史,将军如故。坐公事免官。寻为通直散骑常侍,迁太府卿、都官尚书。太建二年卒,时年六十。赠本官,谥曰安子。

韦载,字德基,京兆杜陵人也。祖叡,梁开府仪同三司,永昌严公。父政,梁黄门侍郎。载少聪惠,笃志好学。年十二,随叔父棱见沛国刘显,显问《汉书》十事,载随问应答,曾无疑滞。及长,博涉文史,沉敏有器局。起家梁邵陵王法曹参军,迁太子舍人、尚书三公郎。

侯景之乱,元帝承制以为中书侍郎。寻为建威将军、寻阳太守,随都督王僧辩东讨侯景。是时僧辩军于湓城,而鲁悉达、樊俊等各拥兵保境,观望成败。元帝以载为假节、都督太原、高唐、新蔡三郡诸军事、高唐太守。仍衔命喻悉达等令出军讨景。及大军东下,载率三郡兵自焦湖出栅口,与僧辩会于梁山。景平,除冠军将军、琅邪太守。寻奉使往东阳、晋安,招抚留异、陈宝应等。仍授信武将军、义兴太守。

高祖诛王僧辨,乃遣周文育轻兵袭载,未至而载先觉,乃婴城自守。文育攻之甚急,载所属县卒并高祖旧兵,多善用弩,载收得数十人,系以长锁,命所亲监之,使射文育军,约曰十发不两中者则死,每发辄中,所中皆毙。文育军稍却,因于城外据水立栅,相持数旬。高祖闻文育军不利,乃自将征之,克其水栅。仍遣载族弟翙赍书喻载以诛王僧辩意,并奉梁敬帝敕,敕载解兵。载得书,乃以其众降于高祖。高祖厚加抚慰,即以其族弟翙监义兴郡,所部将帅,并随才任使,引载恒置左右,与之谋议。

徐嗣徽、任约等引齐军济江,据石头城,高祖问计于载,载曰:``齐军若分兵先据三吴之路,略地东境,则时事去矣。今可急于淮南即侯景故垒筑城,以通东道转输,别命轻兵绝其粮运,使进无所虏,退无所资,则齐将之首,旬日可致。''高祖从其计。

永定元年,除和戎将军、通直散骑常侍。二年,进号轻车将军。寻加散骑常侍、太子右卫率,将军如故。天嘉元年,以疾去官。载有田十馀顷,在江乘县之白山,至是遂筑室而居,屏绝人事,吉凶庆吊,无所往来,不入篱门者几十载。太建中卒于家,时年五十八。

载族弟翙。翙字子羽,少有志操。祖爱,梁辅国将军。父乾向,汝阴太守。翙弱冠丧父,哀毁甚至,养母、抚孤兄弟子,以仁孝著称。高祖为南徐州刺史,召为征北参军,寻监义兴郡。永定元年,授贞毅将军、步兵校尉。迁骁骑将军,领硃衣直阁。骁骑之职,旧领营兵,兼统宿卫。自梁代已来,其任逾重,出则羽仪清道,入则与二卫通直,临轩则升殿侠侍。翙素有名望,每大事恒令侠侍左右,时人荣之,号曰``侠御将军''。寻出为宣城太守。天嘉二年,预平王琳之功,封清源县侯,邑二百户。太建中卒官,赠明、霍、罗三州刺史。子宏,字德礼,有文学,历官至永嘉王府谘议参军。陈亡入隋。

史臣曰:昔邓禹基于文学,杜预出自儒雅,卒致军功,名著前代。晋氏丧乱,播迁江左,顾荣、郗鉴之辈,温峤、谢玄之伦,莫非巾褐书生,晋绅素誉,抗敌以卫社稷,立勋而升台鼎。自斯以降,代有其人。但梁室沸腾,懦夫立志,既身逢际会,见仗于时主,美矣!

\hypertarget{header-n4587}{%
\subsubsection{卷十三}\label{header-n4587}}

沈炯虞荔弟寄马枢

沈炯,字礼明,吴兴武康人也。祖瑀,梁寻阳太守。父续,王府记室参军。炯少有隽才,为当时所重。释褐王国常侍,迁为尚书左民侍郎,出为吴令。侯景之难,吴郡太守袁君正入援京师,以炯监郡。京城陷,景将宋子仙据吴兴,遣使召炯,委以书记之任。炯固辞以疾,子仙怒,命斩之。炯解衣将就戮,碍于路间桑树,乃更牵往他所,或遽救之,仅而获免。子仙爱其才,终逼之令掌书记。及子仙为王僧辩所败,僧辩素闻其名,于军中购得之,酬所获者铁钱十万,自是羽檄军书皆出于炯。及简文遇害,四方岳牧皆上表于江陵劝进,僧辩令炯制表,其文甚工,当时莫有逮者。

高祖南下,与僧辩会于白茅湾,登坛设盟,炯为其文。及侯景东奔至吴郡,获炯妻虞氏,子行简,并杀之,炯弟携其母逃而获免。侯景平,梁元帝愍其妻子婴戮,特封原乡县侯,邑五百户。僧辩为司徒,以炯为从事中郎。梁元帝征为给事黄门侍郎,领尚书左丞。

荆州陷,为西魏所虏,魏人甚礼之,授炯仪同三司。炯以母老在东,恒思归国,恐魏人爱其文才而留之,恒闭门却扫,无所交游。时有文章,随即弃毁,不令流布。尝独行经汉武通天台,为表奏之,陈己思归之意。其辞曰:``臣闻乔山虽掩,鼎湖之灵可祠,有鲁既荒,大庭之迹无泯。伏惟陛下降德猗兰,纂灵豊谷。汉道既登,神仙可望,射之罘于海浦,礼日观而称功,横中流于汾河,指柏梁而高宴,何其乐也,岂不然欤!既而运属上仙,道穷晏驾,甲帐珠帘,一朝零落,茂陵玉碗,宛出人间,陵云故基,共原田而膴々,别风馀址,对陵阜而茫茫,羁旅缧臣,能不落泪!昔承明既厌,严助东归,驷马可乘,长卿西返,恭闻故实,窃有愚心。黍稷非馨,敢忘徼福。''奏讫,其夜炯梦见有宫禁之所,兵卫甚严,炯便以情事陈诉,闻有人言:``甚不惜放卿还,几时可至。''少日,便与王克等并获东归。绍泰二年至都,除司农卿,迁御史中丞。

高祖受禅,加通直散骑常侍,中丞如故。以母老表请归养,诏不许。文帝嗣位,又表曰:``臣婴生不幸,弱冠而孤,母子零丁,兄弟相长。谨身为养,仕不择官,宦成梁朝,命存乱世,冒危履险,百死轻生,妻息诛夷,昆季冥灭,馀臣母子,得逢兴运。臣母妾刘,今年八十有一,臣叔母妾丘,七十有五,臣门弟侄故自无人,妾丘儿孙又久亡泯,两家侍养,馀臣一人。前帝知臣之孤茕,养臣以州里,不欲使顿居草莱,又复矜臣温清,所以一年之内,再三休沐。臣之屡披丹款,频冒宸鉴,非欲苟违朝廷,远离畿辇。一者以年将六十,汤火居心,每跪读家书,前惧后喜,温枕扇席,无复成童。二者职居彝宪,邦之司直,若自亏身体,何问国章?前德绸缪,始许哀放,内侍近臣,多悉此旨。正以选贤与能,广求明哲,趑趄荏苒,未始取才。而上玄降戾,奄至今日,德音在耳,坟土遽乾,悠悠昊天,哀此罔极。兼臣私心煎切,弥迫近时,缕缕之祈,转忘尘触。伏惟陛下睿哲聪明,嗣兴下武,刑于四海,弘此孝治。寸管求天,仰归帷扆,有感必应,实望圣明。特乞霈然申其私礼,则王者之德,覃及无方,矧彼翔沈,孰非涵养。''诏答曰:``省表具怀。卿誉驰咸、雒,情深宛、沛。日者理切倚闾,言归异域,复牵时役,遂乖侍养。虽周生之思,每欲弃官,《戴礼》垂文,得遗从政,前朝光宅四海,劬劳万机,以卿才为独步,职居专席,方深委任,屡屈情礼。朕嗣奉洪基,思弘景业,顾兹寡薄,兼缠哀疚,实赖贤哲,同致雍熙,岂便释简南闱,解绂东路。当令冯亲入舍,荀母从官,用睹朝荣,不亏家礼。寻敕所由,相迎尊累,使卿公私得所,并无废也。''

初,高祖尝称炯宜居王佐,军国大政,多预谋谟,文帝又重其才用,欲宠贵之。会王琳入寇大雷,留异拥据东境,帝欲使炯因是立功,乃解中丞,加明威将军,遣还乡里,收合徒众。以疾卒于吴中,时年五十九。文帝闻之,即日举哀,并遣吊祭,赠侍中,谥曰恭子。有集二十卷行于世。

虞荔,字山披,会稽馀姚人也。祖权,梁廷尉卿、永嘉太守。父检,平北始兴王谘议参军。荔幼聪敏,有志操。年九岁,随从伯阐候太常陆倕,倕问《五经》凡有十事,荔随问辄应,无有遗失,倕甚异之。又尝诣徵士何胤,时太守衡阳王亦造焉,胤言之于王,王欲见荔,荔辞曰:``未有板刺,无容拜谒。''王以荔有高尚之志,雅相钦重,还郡,即辟为主簿,荔又辞以年小不就。及长,美风仪,博览坟籍,善属文。释褐梁西中郎行参军,寻署法曹外兵参军,兼丹阳诏狱正。梁武帝于城西置士林馆,荔乃制碑,奏上,帝命勒之于馆,仍用荔为士林学士。寻为司文郎,迁通直散骑侍郎,兼中书舍人。时左右之任,多参权轴,内外机务,互有带掌,唯荔与顾协淡然靖退,居于西省,但以文史见知,当时号为清白。寻领大著作。

及侯景之乱,荔率亲属入台,除镇西谘议参军,舍人如故。台城陷,逃归乡里。侯景平,元帝征为中书侍郎,贞阳侯,授扬州别驾,并不就。

张彪之据会稽也,荔时在焉。及文帝平彪,高祖遗荔书曰:``丧乱已来,贤哲凋散,君才用有美,声闻许、洛,当今朝廷惟新,广求英隽,岂可栖迟东土,独善其身?今令兄子将接出都,想必副朝廷虚迟也。''文帝又与书曰:``君东南有美,声誉洽闻,自应翰飞京许,共康时弊,而削迹丘园,保兹独善,岂使称空谷之望邪?必愿便尔俶装,且为出都之计。唯迟披觏,在于兹日。''迫切之不得已,乃应命至都。高祖崩,文帝嗣位,除太子中庶子,仍侍太子读书。寻领大著作、东扬扬州二州大中正,庶子如故。

初,荔母随荔入台,卒于台内,寻而城陷,情礼不申,由是终身蔬食布衣,不听音乐,虽任遇隆重,而居止俭素,淡然无营。文帝深器之,常引在左右,朝夕顾访。荔性沉密,少言论,凡所献替,莫有见其际者,故不列于后焉。

时荔第二弟寄寓于闽中,依陈宝应,荔每言之辄流涕。文帝哀而谓曰:``我亦有弟在远,此情甚切,他人岂知。''乃敕宝应求寄,宝应终不遣。荔因以感疾,帝数往临视。令荔将家口人省,荔以禁中非私居之所,乞停城外,文帝不许,乃令住于兰台,乘舆再三临问,手敕中使,相望于道。又以荔蔬食积久,非羸疾所堪,乃敕曰:``能敦布素,乃当为高,卿年事已多,气力稍减,方欲仗委,良须克壮,今给卿鱼肉,不得固从所执也。''荔终不从。天嘉二年卒,时年五十九。文帝甚伤惜之,赠侍中,谥曰德子。及丧柩还乡里,上亲出临送,当时荣之。子世基、世南,并少知名。

寄字次安,少聪敏。年数岁,客有造其父者,遇寄于门,因嘲之曰:``郎君姓虞,必当无智。''寄应声答曰:``文字不辨,岂得非愚?''客大惭。入谓其父曰:``此子非常人,文举之对不是过也。''及长,好学,善属文。性冲静,有栖遁之志。弱冠举秀才,对策高第。起家梁宣城王国左常侍。大同中,尝骤雨,殿前往往有杂色宝珠,梁武观之甚有喜色,寄因上《瑞雨颂》。帝谓寄兄荔曰:``此颂典裁清拔,卿家之士龙也。将如何擢用?''寄闻之,叹曰:``美盛德之形容,以申击壤之情耳。吾岂买名求仕者乎?''乃闭门称疾,唯以书籍自娱。岳阳王为会稽太守,引寄为行参军,迁记室参军,领郡五官掾。又转中记室,掾如故。在职简略烦苛,务存大体,曹局之内,终日寂然。

侯景之乱,寄随兄荔入台,除镇南湘东王谘议参军,加贞威将军。京城陷,遁还乡里。及张彪往临川,强寄俱行,寄与彪将郑玮同舟而载,玮尝忤彪意,乃劫寄奔于晋安。时陈宝应据有闽中,得寄甚喜。高祖平侯景,寄劝令自结,宝应从之,乃遣使归诚。承圣元年,除和戎将军、中书侍郎,宝应爱其才,托以道阻不遣。每欲引寄为僚属,委以文翰,寄固辞,获免。

及宝应结婚留异,潜有逆谋,寄微知其意,言说之际,每陈逆顺之理,微以讽谏,宝应辄引说他事以拒之。又尝令左右诵《汉书》,卧而听之,至蒯通说韩信曰``相君之背,贵不可言'',宝应蹶然起曰``可谓智士''。寄正色曰:``覆郦骄韩,未足称智;岂若班彪《王命》,识所归乎?''寄知宝应不可谏,虑祸及己,乃为居士服以拒绝之。常居东山寺,伪称脚疾,不复起,宝应以为假托,使烧寄所卧屋,寄安卧不动。亲近将扶寄出,寄曰:``吾命有所悬,避欲安往?''所纵火者,旋自救之。宝应自此方信。

及留异称兵,宝应资其部曲,寄乃因书极谏曰:

东山虞寄致书于明将军使君节下:寄流离世故,飘寓贵乡,将军待以上宾之礼,申以国士之眷,意气所感,何日忘之。而寄沈痼弥留,忄妻阴将尽,常恐卒填沟壑,涓尘莫报,是以敢布腹心,冒陈丹款,愿将军留须臾之虑,少思察之,则瞑目之日,所怀毕矣。

夫安危之兆,祸福之机,匪独天时,亦由人事。失之毫厘,差以千里。是以明智之士,据重位而不倾,执大节而不失,岂惑于浮辞哉?将军文武兼资,英威不世,往因多难,仗剑兴师,援旗誓众,抗威千里,岂不以四郊多垒,共谋王室,匡时报主,宁国庇民乎?此所以五尺童子,皆愿荷戟而随将军者也。及高祖武皇肇基草昧,初济艰难。于时天下沸腾,民无定主,豺狼当道,鲸鲵横击,海内业业,未知所从。将军运动微之鉴,折从衡之辩,策名委质,自托宗盟,此将军妙算远图,发于衷诚者也。及主上继业,钦明睿圣,选贤与能,群臣辑睦,结将军以维城之重,崇将军以裂土之封。岂非宏谟庙略,推赤心于物也?屡申明诏,款笃殷勤,君臣之分定矣,骨肉之恩深矣。不意将军惑于邪说,遽生异计,寄所以疾首痛心,泣尽继之以血。万全之策,窃为将军惜之。寄虽疾侵耄及,言无足采,千虑一得,请陈愚算。愿将军少戢雷霆,赊其晷刻,使得尽狂瞽之说,披肝胆之诚,则虽死之日,由生之年也。

自天厌梁德,多难荐臻,寰宇分崩,英雄互起,不可胜纪,人人自以为得之。然夷凶翦乱,拯溺扶危,四海乐推,三灵眷命,揖让而居南面者,陈氏也。岂非历数有在,惟天所授,当璧应运?其事甚明一也。主上承基,明德远被,天纲再张,地维重纽。夫以王琳之强,侯瑱之力,进足以摇荡中原,争衡天下,退足以屈强江外,雄长偏隅。然或命一旅之师,或资一士之说,琳则瓦解冰泮,投身异域,瑱则厥角稽颡,委命阙廷。斯又天假之威,而除其患。其事甚明二也。今将军以籓戚之重,东南之众,尽忠奉上,戮力勤王,岂不勋高窦融,宠过吴芮,析圭判野,南面称孤?其事甚明三也。且圣朝弃瑕忘过,宽厚得人,改过自新,咸加叙擢。至于余孝顷、潘纯陀、李孝钦、欧阳頠等,悉委以心腹,任以爪牙,胸中豁然,曾无纤芥。况将军衅非张绣,罪异毕谌,当何虑于危亡,何失于富贵?此又其事甚明四也。方今周、齐邻睦,境外无虞,并兵一向,匪朝伊夕,非刘、项竞逐之机,楚、赵连从之势,何得雍容高拱,坐论西伯?其事甚明五也。且留将军狼顾一隅,亟经摧衄,声实亏丧,胆气衰沮。高瓖、向文政、留瑜、黄子玉,此数人者,将军所知,首鼠两端,唯利是视;其馀将帅,亦可见矣。孰能被坚执锐,长驱深入,系马埋轮,奋不顾命,以先士卒者乎?此又其事甚明六也。且将军之强,孰如侯景?将军之众,孰如王琳?武皇灭侯景于前,今上摧王琳于后,此乃天时,非复人力。且兵革已后,民皆厌乱,其孰能弃坟墓,捐妻子,出万死不顾之计,从将军于白刃之间乎?此又其事甚明七也。历观前古,鉴之往事,子阳、季孟,倾覆相寻,馀善、右渠,危亡继及,天命可畏,山川难恃。况将军欲以数郡之地,当天下之兵,以诸侯之资,拒天子之命,强弱逆顺,可得侔乎?此又其事甚明八也。且非我族类,其心必异。不爱其亲,岂能及物?留将军身縻国爵,子尚王姬,犹且弃天属而弗顾,背明君而孤立,危急之日,岂能同忧共患,不背将军者乎?至于师老力屈,惧诛利赏,必有韩、智晋阳之谋,张、陈井陉之势。此又其事甚明九也。且北军万里远斗,锋不可当,将军自战其地,人多顾后。梁安背向为心,修旿匹夫之力,众寡不敌,将帅不侔,师以无名而出,事以无机而动,以此称兵,示知其利。夫以汉朝吴、楚,晋室颖、颙,连城数十,长戟百万,拔本塞源,自图家国,其有成功者乎?此又其事甚明十也。

为将军计者,莫若不远而复,绝亲留氏,秦郎、快郎,随遣入质,释甲偃兵,一遵诏旨。且朝廷许以铁券之要,申以白马之盟,朕弗食言,誓之宗社。寄闻明者鉴未形,智者不再计,此成败之效,将军勿疑。吉凶之几,间不容发。方今籓维尚少,皇子幼冲,凡预宗枝,皆蒙宠树。况以将军之地,将军之才,将军之名,将军之势,而能克修籓服,北面称臣,宁与刘泽同年而语其功业哉?岂不身与山河等安,名与金石相敝?愿加三思,虑之无忽。

寄气力绵微,馀阴无几,感恩怀德,不觉狂言,鈇钺之诛,甘之如荠。

宝应览书大怒。或谓宝应曰:``虞公病势渐笃,言多错谬。''宝应意乃小释。亦为寄有民望,且优容之。及宝应败走,夜至蒲田,顾谓其子扞秦曰:``早从虞公计,不至今日。''扞秦但泣而已。宝应既擒,凡诸宾客微有交涉者,皆伏诛,唯寄以先识免祸。

初,沙门慧摽涉猎有才思,及宝应起兵,作五言诗以送之,曰:``送马犹临水,离旗稍引风。好看今夜月,当入紫微宫。''宝应得之甚悦。慧摽赍以示寄,寄一览便止,正色无言。摽退,寄谓所亲曰:``摽既以此始,必以此终。''后竟坐是诛。

文帝寻敕都督章昭达以理发遣,令寄还朝。及至,即日引见,谓寄曰:``管宁无恙?''其慰劳之怀若此。顷之,文帝谓到仲举曰:``衡阳王既出阁,虽未置府僚,然须得一人旦夕游处,兼掌书记,宜求宿士有行业者。''仲举未知所对,文帝曰:``吾自得之。''乃手敕用寄,寄入谢,文帝曰:``所以暂屈卿游籓者,非止以文翰相烦,乃令以师表相事也。''寻兼散骑常侍,聘齐,寄辞老疾,不行,除国子博士。顷之,又表求解职归乡里,文帝优旨报答,许其东还。仍除东扬州别驾,寄又以疾辞。高宗即位,征授扬州治中及尚书左丞,并不就。乃除东中郎建安王谘议,加戎昭将军,又辞以疾,不任旦夕陪列。王于是特令停王府公事,其有疑议,就以决之,但朔望笺修而已。太建八年,加太中大夫,将军如故。十一年卒,时年七十。

寄少笃行,造次必于仁厚,虽僮竖未尝加以声色,至于临危执节,则辞气凛然,白刃不惮也。自流寓南土,与兄荔隔绝,因感气病,每得荔书,气辄奔剧,危殆者数矣。前后所居官,未尝至秩满,才期年数月,便自求解退。常曰:``知足不辱,吾知足矣。''及谢病私庭,每诸王为州将,下车必造门致礼,命释鞭板,以几杖侍坐。常出游近寺,闾里传相告语,老幼罗列,望拜道左。或言誓为约者,但指寄便不欺,其至行所感如此。所制文笔,遭乱多不存。

马枢,字要理,扶风郿人也。祖灵庆,齐竟陵王录事参军。枢数岁而父母俱丧,为其姑所养。六岁,能诵《孝经》、《论语》、《老子》。及长,博极经史,尤善佛经及《周易》、《老子》义。

梁邵陵王纶为南徐州刺史,素闻其名,引为学士。纶时自讲《大品经》,令枢讲《维摩》、《老子》、《周易》,同日发题,道俗听者二千人。王欲极观优劣,乃谓众曰:``与马学士论义,必使屈伏,不得空立主客。''于是数家学者各起问端,枢乃依次剖判,开其宗旨,然后枝分流别,转变无穷,论者拱默听受而已。纶甚嘉之,将引荐于朝廷。寻遇侯景之乱,纶举兵援台,乃留书二万卷以付枢。枢肆志寻览,殆将周遍,乃喟然叹曰:``吾闻贵爵位者以巢、由为桎梏,爱山林者以伊、吕为管库,束名实则刍芥柱下之言,玩清虚则糠秕席上之说,稽之笃论,亦各从其好也。然支父有让王之介,严子有傲帝之规,千载美谈,所不废也。比求志之士,望途而息。岂天之不惠高尚,何山林之无闻甚乎?''乃隐于茅山,有终焉之志。

天嘉元年,文帝征为度支尚书,辞不应命。时枢亲故并居京口,每秋冬之际,时往游焉。及鄱阳王为南徐州刺史,钦其高尚,鄙不能致,乃卑辞厚意,令使者邀之,前后数反,枢固辞以疾。门人或进曰:``鄱阳王待以师友,非关爵位,市朝之间,何妨静默。''枢不得已,乃行。王别筑室以处之,枢恶其崇丽,乃于竹林间自营茅茨而居焉。每王公馈饷,辞不获已者,率十分受一。

枢少属乱离,每所居之处,盗贼不入,依托者常数百家。目精洞黄,能视暗中物。常有白燕一双,巢其庭树,驯狎纮庑,时集几案,春来秋去,几三十年。太建十三年卒,时年六十。撰《道觉论》二十卷行于世。

史臣曰:沈炯仕于梁室,年在知命,冀郎署之薄官,止邑宰之卑职,及下笔盟坛,属辞劝表,激扬旨趣,信文人之伟者欤!虞荔之献筹沈密,尽其诚款,可谓有益明时矣。

\hypertarget{header-n4619}{%
\subsubsection{卷十四}\label{header-n4619}}

到仲举韩子高华皎

到仲举,字德言,彭城武原人也。祖坦,齐中书侍郎。父洽,梁侍中。仲举无他艺业,而立身耿正。释褐著作佐郎、太子舍人、王府主簿。出为长城令,政号廉平。文帝居乡里,尝诣仲举,时天阴雨,仲举独坐斋内,闻城外有箫鼓之声,俄而文帝至,仲举异之,乃深自结托。文帝又尝因饮,夜宿仲举帐中,忽有神光五采照于室内,由是祗承益恭。侯景之乱,仲举依文帝。及景平,文帝为吴兴郡守,以仲举为郡丞,与颍川庾持俱为文帝宾客。文帝为宣毅将军,以仲举为长史,寻带山阴令。文帝嗣位,授侍中,参掌选事。天嘉元年,守都官尚书,封宝安县侯,邑五百户。三年,除都官尚书。其年,迁尚书右仆射、丹阳尹,参掌并如故。寻改封建昌县侯。仲举既无学术,朝章非所长,选举引用,皆出自袁枢。性疏简,不干涉世务,与朝士无所亲狎,但聚财酣饮而已。六年,秩满,解尹。

是时,文帝积年寝疾,不亲御万机,尚书中事,皆使仲举断决。天康元年,迁侍中、尚书仆射,参掌如故。文帝疾甚,入侍医药。及文帝崩,高宗受遗诏为尚书令入辅,仲举与左丞王暹、中书舍人刘师知、殷不佞等,以朝望有归,乃遣不佞矫宣旨遣高宗还东府。事发,师知下北狱赐死,暹、不佞并付治,乃以仲举为贞毅将军、金紫光禄大夫。

初,仲举子郁尚文帝妹信义长公主,官至中书侍郎,出为宣城太守,文帝配以士马,是年迁为南康内史,以国哀未之任。仲举既废居私宅,与郁皆不自安。时韩子高在都,人马素盛,郁每乘小舆蒙妇人衣与子高谋。子高军主告言其事,高宗收子高、仲举及郁并付廷尉。诏曰:``到仲举庸劣小才,坐叨显贵,受任前朝,荣宠隆赫,父参王政,子据大邦,礼盛外姻,势均戚里。而肆此骄暗,凌傲百司,遏密之初,擅行国政,排黜懿亲,欺蔑台衮。韩子高蕞尔细微,擢自卑末,入参禁卫,委以腹心,蜂虿有毒,敢行反噬。仲举、子高,共为表里,阴构奸谋,密为异计。安成王朕之叔父,亲莫重焉。受命导扬,禀承顾托,以朕冲弱,属当保祐。家国安危,事归宰辅,伊、周之重,物无异议,将相旧臣,咸知宗仰。而率聚凶徒,欲相掩袭,屯据东城,进逼崇礼,规树仲举,以执国权,陵斥司徒,意在专政,潜结党附,方危社稷。赖祖宗之灵,奸谋显露。前上虞令陆昉等具告其事,并有据验,并克今月七日,纵其凶谋。领军将军明彻,左卫将军、卫尉卿宝安及诸公等,又并知其事。二三飐迹,彰于朝野,反道背德,事骇闻见。今大憝克歼,罪人斯得,并可收付廷尉,肃正刑书。罪止仲举父子及子高三人而已,其馀一从旷荡,并所不问。''仲举及郁并于狱赐死,时年五十一。郁诸男女,以帝甥获免。

韩子高,会稽山阴人也。家本微贱。侯景之乱,寓在京都。景平,文帝出守吴兴,子高年十六,为总角,容貌美丽,状似妇人,于淮渚附部伍寄载欲还乡。文帝见而问之,曰``能事我乎?''子高许诺。子高本名蛮子,文帝改名之。性恭谨,勤于侍奉,恒执备身刀及传酒炙。文帝性急,子高恒会意旨。及长,稍习骑射,颇有胆决,愿为将帅,及平杜龛,配以士卒。文帝甚宠爱之,未尝离于左右。文帝尝梦见骑马登山,路危欲堕,子高推捧而升。

文帝之讨张彪也,沈泰等先降,文帝据有州城,周文育镇北郭香岩寺。张彪自剡县夜还袭城,文帝自北门出,仓卒暗夕,军人扰乱,文育亦未测文帝所在,唯子高在侧,文帝乃遣子高自乱兵中往见文育,反命,酬答于暗中,又往慰劳众军。文帝散兵稍集,子高引导入文育营,因共立栅。明日,与彪战,彪将申缙复降,彪奔松山,浙东平。文帝乃分麾下多配子高,子高亦轻财礼士,归之者甚众。

文帝嗣位,除右军将军。天嘉元年,封文招县子,邑三百户。王琳至于栅口,子高宿卫台内。及琳平,子高所统益多,将士依附之者,子高尽力论进,文帝皆任使焉。二年,迁员外散骑常侍、壮武将军、成州刺史。及征留异,随侯安都顿桃支岭岩下。时子高兵甲精锐,别御一营,单马入陈,伤项之左,一髻半落。异平,除假节、贞毅将军、东阳太守。五年,章昭达等自临川征晋安,子高自安泉岭会于建安,诸将中人马最为强盛。晋安平,以功迁通直散骑常侍,进爵为伯,增邑并前四百户。六年,征为右卫将军,至都,镇领军府。文帝不豫,入侍医药。废帝即位,迁散骑常侍,右卫如故,移顿于新安寺。

高宗入辅,子高兵权过重,深不自安,好参访台阁,又求出为衡、广诸镇。光大元年八月,前上虞县令陆昉及子高军主告其谋反,高宗在尚书省,因召文武在位议立皇太子,子高预焉,平旦入省,执之,送廷尉,其夕与到仲举同赐死,时年三十。父延庆及子弟并原宥。延庆因子高之宠,官至给事中、山阴令。

华皎,晋陵暨阳人。世为小吏。皎梁代为尚书比部令史。侯景之乱,事景党王伟。高祖南下,文帝为景所囚,皎遇文帝甚厚。景平,文帝为吴兴太守,以皎为都录事,军府谷帛,多以委之。皎聪慧,勤于簿领。及文帝平杜龛,仍配以人马甲仗,犹为都录事。御下分明,善于抚养。时兵荒之后,百姓饥馑,皎解衣推食,多少必均,因稍擢为暨阳、山阴二县令。文帝即位,除开远将军,左军将军。天嘉元年,封怀仁县伯,邑四百户。

王琳东下,皎随侯瑱拒之。琳平,镇湓城,知江州事。时南州守宰多乡里酋豪,不遵朝宪,文帝令皎以法驭之。王琳奔散,将卒多附于皎。三年,除假节、通直散骑常侍、仁武将军、新州刺史资,监江州。寻诏督寻阳、太原、高唐、南北新蔡五郡诸军事、寻阳太守,假节、将军、州资、监如故。周迪谋反,遣其兄子伏甲于船中,伪称贾人,欲于湓城袭皎。未发,事觉,皎遣人逆击之,尽获其船仗。其年,皎随都督吴明彻征迪,迪平,以功授散骑常侍、平南将军、临川太守,进爵为侯,增封并前五百户。未拜,入朝,仍授使持节、都督湘、巴等四州诸军事、湘州刺史,常侍、将军如故。

皎起自下吏,善营产业,湘川地多所出,所得并入朝廷,粮运竹木,委输甚众;至于油蜜脯菜之属,莫不营办。又征伐川洞,多致铜鼓、生口,并送于京师。废帝即位,进号安南将军,改封重安县侯,食邑一千五百户。文帝以湘州出杉木舟,使皎营造大舰金翅等二百馀艘,并诸水战之具,欲以入汉及峡。

韩子高诛后,皎内不自安,缮甲聚徒,厚礼所部守宰。高宗频命皎送大舰金翅等,推迁不至。光大元年,密启求广州,以观时主意。高宗伪许之,而诏书未出。皎亦遣使句引周兵,又崇奉萧岿为主,士马甚盛。诏乃以吴明彻为湘州刺史,实欲以轻兵袭之。是时虑皎先发,乃前遣明彻率众三万,乘金翅直趋郢州,又遣抚军大将军淳于量率众五万,乘大舰以继之,又令假节、冠武将军杨文通别从安成步道出茶陵,又令巴山太守黄法慧别从宜阳出澧陵,往掩袭,出其不意,并与江州刺史章昭达、郢州刺史程灵洗等参谋讨贼。

是时萧岿遣水军为皎声援。周武又遣其弟卫国公宇文直率众屯鲁山,又遣其柱国长胡公拓跋定人马三万,攻围郢州。萧岿授皎司空,巴州刺史戴僧朔,衡阳内史任蛮奴,巴陵内史潘智虔,岳阳太守章昭裕,桂阳太守曹宣,湘东太守钱明,并隶于皎。又长沙太守曹庆等本隶皎下,因为之用。帝恐上流宰守并为皎扇惑,乃下诏曰:``贼皎舆皁微贱,特逢奖擢,任据籓牧,属当宠寄,背斯造育,兴构奸谋,树立萧氏,盟约彰露,鸩毒存心,志危宗社,扇结边境,驱逼士庶,蚁聚巴、湘,豕突鄢、郢,逆天反地,人神忿嫉。征南将军量、安南将军明彻、郢州刺史灵洗,受律专征,备尽心力,抚劳骁雄,舟师俱进,义烈争奋,凶恶奔殄,献捷相望,重氛载廓,言念泣罪,思与惟新。可曲赦湘、巴二州:凡厥为贼所逼制,预在凶党,悉皆不问;其贼主帅节将,并许开恩出首,一同旷荡。''

先是,诏又遣司空徐度与杨文通等自安成步出湘东,以袭皎后。时皎阵于巴州之白螺,列舟舰与王师相持未决。及闻徐度趋湘州,乃率兵自巴、郢因便风下战。淳于量、吴明彻等募军中小舰,多赏金银,令先出当贼大舰,受其拍。贼舰发拍皆尽,然后官军以大舰拍之,贼舰皆碎,没于中流。贼又以大舰载薪,因风放火,俄而风转自焚,贼军大败。皎乃与戴僧朔单舸走,过巴陵,不敢登城,径奔江陵。拓跋定等无复船渡,步趋巴陵,巴陵城邑为官军所据,乃向湘州。至水口,不得济,食且尽,诣军请降。俘获万馀人,马四千馀匹,送于京师。皎党曹庆、钱明、潘智虔、鲁闲、席慧略等四十馀人并诛,唯任蛮奴、章昭裕、曹宣、刘广业获免。

戴僧朔,吴郡钱塘人也。有膂力,勇健善战,族兄右将军僧锡甚爱之。僧锡年老,征讨恒使僧朔领众。平王琳有功,僧锡卒,仍代为南丹阳太守。镇采石。从征留异,侯安都于岩下出战,为贼斫伤,僧朔单刀步援。以功除壮武将军、北江州刺史,领南陵太守。又从征周迪有功,迁巴州刺史,假节、将军如故。至是同皎为逆,伏诛于江陵。

曹庆,本王琳将,萧庄伪署左卫将军、吴州刺史,部领亚于潘纯陀。琳败,文帝以配皎,官至长沙太守。钱明,本高祖主帅,后历湘州诸郡守。潘智虔,纯陀之子,少有志气,年二十为巴陵内史。鲁闲,吴郡钱塘人。席慧略,安定人。闲本张彪主帅,慧略王琳部下,文帝皆配于皎,官至郡守。并伏诛。

章昭裕,昭达之弟;刘广业,广德之弟;曹宣,高祖旧臣;任蛮奴尝有密启于朝廷;由是并获宥。

史臣曰:韩子高、华皎虽复瓶筲小器,舆台末品,文帝鉴往古之得人,救当今之急弊,达聪明目之术,安黎和众之宜,寄以腹心,不论胄阀。皎早参近昵,尝预艰虞,知其无隐,赏以悉力,有见信之诚,非可疑之地。皎据有上游,忠于文帝。仲举、子高亦无爽于臣节者矣。

\hypertarget{header-n4640}{%
\subsubsection{卷十五}\label{header-n4640}}

谢哲萧乾谢嘏张种王固孔奂萧允弟引

谢哲,字颖豫,陈郡阳夏人也。祖朏,梁司徒。父譓,梁右光禄大夫。哲美风仪,举止酝藉,而襟情豁然,为士君子所重。起家梁秘书郎,累迁广陵太守。侯景之乱,以母老因寓居广陵,高祖自京口渡江应接郭元建,哲乃委质,深被敬重。高祖为南徐州刺史,表哲为长史。荆州陷,高祖使哲奉表于晋安王劝进。敬帝承制征为给事黄门侍郎,领步兵校尉。贞阳侯僭位,以哲为通直散骑常侍,侍东宫。敬帝即位,迁长兼侍中。高祖受命,迁都官尚书、豫州大中正、吏部尚书。出为明威将军、晋陵太守,入为中书令。世祖嗣位,为太子詹事。出为明威将军、衡阳内史,秩中二千石。迁长沙太守,将军、加秩如故。还除散骑常侍、中书令。废帝即位,以本官领前将军。高宗为录尚书,引为侍中、仁威将军、司徒左长史。未拜,光大元年卒,时年五十九。赠侍中、中书监,谥康子。

萧乾,字思惕,兰陵人也。祖嶷,齐丞相豫章文献王。父子范,梁秘书监。乾容止雅正,性恬简,善隶书,得叔父子云之法。年九岁,召补国子《周易》生,梁司空袁昂时为祭酒,深敬重之。十五,举明经。释褐东中郎湘东王法曹参军,迁太子舍人。建安侯萧正立出镇南豫州,又板录事参军。累迁中军宣城王中录事谘议参军。侯景平,高祖镇南徐州,引乾为贞威将军、司空从事中郎。迁中书侍郎、太子家令。

永定元年,除给事黄门侍郎。是时熊昙朗在豫章,周迪在临川,留异在东阳,陈宝应在建、晋,共相连结,闽中豪帅,往往立砦以自保,高祖甚患之,乃令乾往使,谕以逆顺,并观虚实。将发,高祖谓乾曰:``建、晋恃险,好为奸宄,方今天下初定,难便出兵。昔陆贾南征,赵佗归顺,随何奉使,黥布来臣,追想清风,仿佛在目。况卿坐镇雅俗,才高昔贤,宜勉建功名,不烦更劳师旅。''乾既至,晓以逆顺,所在渠帅并率部众开壁款附。其年,就除贞威将军、建安太守。

天嘉二年,留异反,陈宝应将兵助之,又资周迪兵粮,出寇临川,因逼建安。乾单使临郡,素无士卒,力不能守,乃弃郡以避宝应。时闽中守宰,并为宝应迫胁,受其署置,乾独不为屈,徙居郊野,屏绝人事。及宝应平,乃出诣都督章昭达,昭达以状表闻,世祖甚嘉之,超授五兵尚书。光大元年卒,谥曰静子。

谢嘏,字含茂,陈郡阳夏人也。祖,齐金紫光禄大夫。父举,梁中卫将军、开府仪同三司。嘏风神清雅,颇善属文。起家梁秘书郎,稍迁太子中庶子,掌东宫管记,出为建安太守。侯景之乱,嘏之广州依萧勃,承圣中,元帝征为五兵尚书,辞以道阻,转授智武将军。萧勃以为镇南长史、南海太守。勃败,还至临川,为周迪所留。久之,又度岭之晋安依陈宝应,世祖前后频召之,嘏崎岖寇虏,不能自拔。及宝应平,嘏方诣阙,为御史中丞江德藻所举劾,世祖不加罪责,以为给事黄门侍郎。寻转侍中,天康元年,以公事免,寻复本职。光大元年,为信威将军、中卫始兴王长史。迁中书令、豫州大中正、都官尚书,领羽林监,中正如故。太建元年卒,赠侍中、中书令,谥曰光子。有文集行于世。

二子俨、伷。俨官至散骑常侍、侍中、御史中丞、太常卿,出监东扬州。祯明二年卒于会稽,赠中护军。

张种,字士苗,吴郡人也。祖辩,宋司空右长史、广州刺史。父略,梁太子中庶子、临海太守。种少恬静,居处雅正,不妄交游,傍无造请,时人为之语曰:``宋称敷、演,梁则卷、充。清虚学尚,种有其风。''仕梁王府法曹,迁外兵参军,以父忧去职。服阕,为中军宣城王府主簿。种时年四十馀,家贫,求为始豊令,入除中卫西昌侯府西曹掾。时武陵王为益州刺史,重选府僚,以种为征西东曹掾,种辞以母老,抗表陈请,为有司所奏,坐黜免。

侯景之乱,种奉其母东奔,久之得达乡里。俄而母卒,种时年五十,而毁瘠过甚,又迫以凶荒,未获时葬,服制虽毕,而居处饮食,恒若在丧。及景平,司徒王僧辩以状奏闻,起为贞威将军、治中从事史,并为具葬礼,葬讫,种方即吉。僧辩又以种年老,傍无胤嗣,赐之以妾,及居处之具。

贞阳侯僭位,除廷尉卿、太子中庶子。敬帝即位,为散骑常侍,迁御史中丞,领前军将军。高祖受禅,为太府卿。天嘉元年,除左民尚书。二年,权监吴郡,寻征复本职。迁侍中,领步兵校尉,以公事免,白衣兼太常卿,俄而即真。废帝即位,加领右军将军,未拜,改领弘善宫卫尉,又领扬、东扬二州大中正。高宗即位,重为都官尚书,领左骁骑将军,迁中书令,骁骑、中正并如故。以疾授金紫光禄大夫。

种沈深虚静,而识量宏博,时人皆以为宰相之器。仆射徐陵尝抗表让位于种曰:``臣种器怀沈密,文史优裕,东南贵秀,朝庭亲贤,克壮其猷,宜居左执。''其为时所推重如此。太建五年卒,时年七十,赠特进,谥曰元子。

种仁恕寡欲,虽历居显位,而家产屡空,终日晏然,不以为病。太建初,女为始兴王妃,以居处僻陋,特赐宅一区,又累赐无锡、嘉兴县侯秩。尝于无锡见有重囚在狱,天寒,呼出曝日,遂失之,世祖大笑,而不深责。有集十四卷。

种弟棱,亦清静有识度,官至司徒左长史,太建十一年卒,时年七十,赠光禄大夫。

种族子稚才,齐护军冲之孙。少孤介特立,仕为尚书金部郎中。迁右丞,建康令、太府卿、扬州别驾从事史,兼散骑常侍。使于周,还为司农、廷尉卿。所历并以清白称。

王固,字子坚,左光禄大夫通之弟也。少清正,颇涉文史,以梁武帝甥封莫口亭侯。举秀才。起家梁秘书郎,迁太子洗马,掌东宫管记,丁所生母忧去职。服阕,除丹阳尹丞。侯景之乱,奔于荆州,梁元帝承制以为相国户曹属,掌管记。寻聘于西魏,魏人以其梁氏外戚,待之甚厚。承圣元年,迁太子中庶子,寻为贞威将军、安南长史、寻阳太守。荆州陷,固之鄱阳,随兄质度东岭,居信安县。绍泰元年,征为侍中,不就。永定中,移居吴郡。世祖以固清静,且欲申以婚姻。天嘉二年,至都,拜国子祭酒。三年,迁中书令。四年,又为散骑常侍、国子祭酒。其年,以固女为皇太子妃,礼遇甚重。

废帝即位,授侍中、金紫光禄大夫。时高宗辅政,固以废帝外戚,妳媪恒往来禁中,颇宣密旨,事泄,比将伏诛,高宗以固本无兵权,且居处清洁,止免所居官,禁锢。

太建二年,随例为招远将军、宣惠豫章王谘议参军。迁太中大夫、太常卿、南徐州大中正。七年,卒官,时年六十三。赠金紫光禄大夫。丧事所须,随由资给。至德二年改葬,谥曰恭子。

固清虚寡欲,居丧以孝闻。又崇信佛法,及丁所生母忧,遂终身蔬食,夜则坐禅,昼诵佛经,兼习《成实论》义,而于玄言非所长。尝聘于西魏,因宴飨之际,请停杀一羊,羊于固前跪拜。又宴于昆明池,魏人以南人嗜鱼,大设罟网,固以佛法咒之,遂一鳞不获。

子宽,官至司徒左史、侍中。

孔奂,字休文,会稽山阴人也。曾祖琇之,齐左民尚书、吴兴太守。祖臶,太子舍人、尚书三公郎。父稚孙,梁宁远枝江公主簿、无锡令。奂数岁而孤,为叔父虔孙所养。好学,善属文,经史百家,莫不通涉。沛国刘显时称学府,每共奂讨论,深相叹服,乃执奂手曰:``昔伯喈坟素悉与仲宣,吾当希彼蔡君,足下无愧王氏。''所保书籍,寻以相付。

州举秀才,射策高第。起家扬州主簿、宣惠湘东王行参军,并不就。又除镇西湘东王外兵参军,入为尚书仓部郎中,迁仪曹侍郎。时左民郎沈炯为飞书所谤,将陷重辟,事连台阁,人怀忧惧,奂廷议理之,竟得明白。丹阳尹何敬容以奂刚正,请补功曹史。出为南昌侯相,值侯景乱,不之官。

京城陷,朝士并被拘絷,或荐奂于贼帅侯子鉴,子鉴命脱桎梏,厚遇之,令掌书记。时景军士悉恣其凶威,子鉴景之腹心,委任又重,朝士见者,莫不卑俯屈折,奂独敖然自若,无所下。或谏奂曰:``当今乱世,人思苟免,獯羯无知,岂可抗之以义?''奂曰:``吾性命有在,虽未能死,岂可取媚凶丑,以求全乎?''时贼徒剥掠子女,拘逼士庶,奂每保持之,得全济者甚众。

寻遭母忧,哀毁过礼。时天下丧乱,皆不能终三年之丧,唯奂及吴国张种,在寇乱中守持法度,并以孝闻。

及景平,司徒王僧辩先下辟书,引奂为左西曹掾,又除丹阳尹丞。梁元帝于荆州即位,征奂及沈炯并令西上,僧辩累表请留之。帝手敕报僧辩曰:``孔、沈二士,今且借公。''其为朝廷所重如此。仍除太尉从事中郎。僧辩为扬州刺史,又补扬州治中从事史。时侯景新平,每事草创,宪章故事,无复存者,奂博物强识,甄明故实,问无不知,仪注体式,笺表书翰,皆出于奂。

高祖作相,除司徒右长史,迁给事黄门侍郎。齐遣东方老、萧轨等来寇,军至后湖,都邑搔扰,又四方壅隔,粮运不继,三军取给,唯在京师,乃除奂为贞威将军、建康令。时累岁兵荒,户口流散,勍敌忽至,征求无所,高祖克日决战,乃令奂多营麦饭,以荷叶裹之,一宿之间,得数万裹,军人旦食讫,弃其馀,因而决战,遂大破贼。

高祖受禅,迁太子中庶子。永定二年,除晋陵太守。晋陵自宋、齐以来,旧为大郡,虽经寇扰,犹为全实,前后二千石多行侵暴,奂清白自守,妻子并不之官,唯以单船监郡,所得秩俸,随即分赡孤寡,郡中大悦,号曰``神君''。曲阿富人殷绮,见奂居处素俭,乃饷衣一袭,氈被一具。奂曰:``太守身居美禄,何为不能办此,但民有未周,不容独享温饱耳。劳卿厚意,幸勿为烦。''

初,世祖在吴中,闻奂善政,及践祚,征为御史中丞,领扬州大中正。奂性刚直,善持理,多所纠劾,朝廷甚敬惮之。深达治体,每所敷奏,上未尝不称善,百司滞事,皆付奂决之。迁散骑常侍,领步兵校尉,中书舍人,掌诏诰,扬、东扬二州大中正。天嘉四年,重除御史中丞,寻为五兵尚书,常侍、中正如故。时世祖不豫,台阁众事,并令仆射到仲举共奂决之。及世祖疾笃,奂与高宗及仲举并吏部尚书袁枢、中书舍人刘师知等入侍医药。世祖尝谓奂等曰:``今三方鼎峙,生民未乂,四海事重,宜须长君。朕欲近则晋成,远隆殷法,卿等须遵此意。''奂乃流涕歔欷而对曰:``陛下御膳违和,痊复非久,皇太子春秋鼎盛,圣德日跻,安成王介弟之尊,足为周旦,阿衡宰辅,若有废立之心,臣等愚诚,不敢闻诏。''世祖曰:``古之遗直,复见于卿。''天康元年,乃用奂为太子詹事,二州中正如故。

世祖崩,废帝即位,除散骑常侍、国子祭酒。光大二年,出为信武将军、南中郎康乐侯长史、寻阳太守,行江州事。高宗即位,进号仁威将军、云麾始兴王长史,馀并如故。奂在职清俭,多所规正,高宗嘉之,赐米五百斛,并累降敕书殷勤劳问。太建三年,征为度支尚书,领右军将军。五年,改领太子中庶子,与左仆射徐陵参掌尚书五条事。六年,迁吏部尚书。七年,加散骑常侍。八年,改加侍中。时有事北讨,克复淮、泗,徐、豫酋长,降附相继,封赏选叙,纷纭重叠,奂应接引进,门无停宾。加以鉴识人物,详练百氏,凡所甄拔,衣冠缙绅,莫不悦伏。

性耿介,绝请托,虽储副之尊,公侯之重,溺情相及,终不为屈。始兴王叔陵之在湘州,累讽有司,固求台铉。奂曰:``衮章之职,本以德举,未必皇枝。''因抗言于高宗。高宗曰:``始兴那忽望公,且朕儿为公,须在鄱阳王后。''奂曰:``臣之所见,亦如圣旨。''后主时在东宫,欲以江总为太子詹事,令管记陆瑜言之于奂。奂谓瑜曰:``江有潘、陆之华,而无园、绮之实,辅弼储宫,窃有所难。''瑜具以白后主,后主深以为恨,乃自言于高宗。高宗将许之,奂乃奏曰:``江总文华之人,今皇太子文华不少,岂藉于总!如臣愚见,愿选敦重之才,以居辅导。''帝曰:``即如卿言,谁当居此?''奂曰:``都官尚书王廓,世有懿德,识性敦敏,可以居之。''后主时亦在侧,乃曰:``廓王泰之子,不可居太子詹事。''奂又奏曰:``宋朝范晔即范泰之子,亦为太子詹事,前代不疑。''后主固争之,帝卒以总为詹事,由是忤旨。其梗正如此。

初,后主欲官其私宠,以属奂,奂不从。及右仆射陆缮迁职,高宗欲用奂,已草诏讫,为后主所抑,遂不行。九年,迁侍中、中书令、领左骁骑将军、扬、东扬、豊三州大中正。十一年,转太常卿,侍中、中正并如故。十四年,迁散骑常侍、金紫光禄大夫,领前军将军,未拜,改领弘范宫卫尉。至德元年卒,时年七十。赠散骑常侍,本官如故。有集十五卷,弹文四卷。

子绍薪、绍忠。绍忠字孝扬,亦有才学,官至太子洗马、仪同鄱阳王东曹掾。

萧允,字叔佐,兰陵人也。曾祖思话,宋征西将军、开府仪同三司、尚书右仆射,封阳穆公。祖惠蒨,散骑常侍、太府卿、左民尚书。父介,梁侍中、都官尚书。允少知名,风神凝远,通达有识鉴,容止酝藉,动合规矩。起家邵陵王法曹参军,转湘东王主簿,迁太子洗马。侯景攻陷台城,百僚奔散,允独整衣冠坐于宫坊,景军人敬而弗之逼也。寻出居京口。时寇贼纵横,百姓波骇,衣冠士族,四出奔散,允独不行。人问其故,允答曰:``夫性命之道,自有常分,岂可逃而获免乎?但患难之生,皆生于利,苟不求利,祸从何生?方今百姓争欲奋臂而论大功,一言而取卿相,亦何事于一书生哉?庄周所谓畏影避迹,吾弗为也。''乃闭门静处,并日而食,卒免于患。

侯景平后,高祖镇南徐州,以书召之,允又辞疾。永定中,侯安都为南徐州刺史,躬造其庐,以申长幼之敬,天嘉三年,征为太子庶子。三年,除棱威将军、丹阳尹丞。五年,兼侍中,聘于周,还拜中书侍郎、大匠卿。高宗即位,迁黄门侍郎。五年,出为安前晋安王长史。六年,晋安王为南豫州,允复为王长史。时王尚少,未亲民务,故委允行府州事。入为光禄卿。允性敦重,未尝以荣利干怀。及晋安出镇湘州,又苦携允,允少与蔡景历善,景历子徵修父党之敬,闻允将行,乃诣允曰:``公年德并高。国之元老,从容坐镇,旦夕自为列曹,何为方复辛苦在外!''允答曰:``已许晋安,岂可忘信。''其恬于荣势如此。

至德三年,除中卫豫章王长史,累迁通直散骑常侍、光胜将军、司徒左长史、安德宫少府。镇卫鄱阳王出镇会稽,允又为长史,带会稽郡丞。行经延陵季子庙,设萍藻之荐,托为异代之交,为诗以叙意,辞理清典。后主尝问蔡徵曰:``卿世与萧允相知,此公志操何如?''徵曰:``其清虚玄远,殆不可测,至于文章,可得而言。''因诵允诗以对,后主嗟赏久之。其年拜光禄大夫。

及隋师济江,允迁于关右。是时朝士至长安者,例并授官,唯允与尚书仆射谢伷辞以老疾,隋文帝义之,并厚赐钱帛。寻以疾卒于长安,时年八十四。弟引。

引字叔休。方正有器局,望之俨然,虽造次之间,必由法度。性聪敏,博学,善属文。释褐著作佐郎,转西昌侯仪同府主簿。侯景之乱,梁元帝为荆州刺史,朝士多往归之。引曰:``诸王力争,祸患方始,今日逃难,未是择君之秋。吾家再世为始兴郡,遗爱在民,正可南行以存家门耳。''于是与弟彤及宗亲等百馀人奔岭表。时始兴人欧阳頠为衡州刺史,引往依焉。頠后迁为广州,病死,子纥领其众。引每疑纥有异,因事规正,由是情礼渐疏。及纥举兵反,时京都士人岑之敬、公孙挺等并皆惶骇,唯引恬然,谓之敬等曰:``管幼安、袁曜卿亦但安坐耳。君子正身以明道,直己以行义,亦复何忧惧乎?''及章昭达平番禺,引始北还。高宗召引问岭表事,引具陈始末,帝甚悦,即日拜金部侍郎。

引善隶书,为当时所重。高宗尝披奏事,指引署名曰:``此字笔势翩翩,似鸟之欲飞。''引谢曰:``此乃陛下假其羽毛耳。''又谓引曰:``我每有所忿,见卿辄意解,何也?''引曰:``此自陛下不迁怒,臣何预此恩。''太建七年,加戎昭将军。九年,除中卫始兴王咨议参军,兼金部侍郎。

引性抗直,不事权贵,左右近臣无所造请,高宗每欲迁用,辄为用事者所裁。及吕梁覆师,戎储空匮,乃转引为库部侍郎,掌知营造弓弩槊箭等事。引在职一年,而器械充牣。频加中书侍郎、贞威将军、黄门郎。十二年,吏部侍郎缺,所司屡举王宽、谢燮等,帝并不用,乃中诏用引。

时广州刺史马靖甚得岭表人心,而兵甲精练,每年深入俚洞,又数有战功,朝野颇生异议。高宗以引悉岭外物情,且遣引观靖,审其举措,讽令送质。引奉密旨南行,外托收督赕物。既至番禺,靖即悟旨,尽遣儿弟下都为质。还至赣水,而高宗崩,后主即位,转引为中庶子,以疾去官。明年,京师多盗,乃复起为贞威将军、建康令。

时殿内队主吴璡,及宦官李善度、蔡脱儿等多所请属,引一皆不许。引族子密时为黄门郎,谏引曰:``李、蔡之势,在位皆畏惮之,亦宜小为身计。''引曰:``吾之立身,自有本末,亦安能为李、蔡改行。就令不平,不过解职耳。''吴璡竟作飞书,李、蔡证之,坐免官,卒于家,时年五十八。子德言,最知名。

引宗族子弟,多以行义知名。弟彤,以恬静好学,官至太子中庶子、南康王长史。密字士机,幼而聪敏,博学有文词。祖琛,梁特进。父游,少府卿。密太建八年,兼散骑常侍,聘于齐。历位黄门侍郎、太子中庶子、散骑常侍。

史臣曰:谢、王、张、萧,咸以清净为风,文雅流誉,虽更多难,终克成名。奂謇谔在公,英飙振俗,详其行事,抑古之遗爱矣。固之蔬菲禅悦,斯乃出俗者焉,犹且致絓于黜免,有惧于倾覆。是知上官、博陆之权势,阎、邓、梁、窦之震动,吁可畏哉!

\hypertarget{header-n4685}{%
\subsubsection{卷十六}\label{header-n4685}}

陆子隆钱道戢骆牙

陆子隆,字兴世,吴郡吴人也。祖敞之,梁嘉兴令。父悛,封氏令。子隆少慷慨,有志功名。起家东宫直后。侯景之乱,于乡里聚徒。是时张彪为吴郡太守,引为将帅。彪徙镇会稽,子隆随之。及世祖讨彪,彪将沈泰、吴宝真、申缙等皆降,而子隆力战败绩,世祖义之,复使领其部曲,板为中兵参军。历始豊、永兴二县令。

世祖嗣位,子隆领甲仗宿卫。寻随侯安都拒王琳于栅口。王琳平,授左中郎将。天嘉元年,封益阳县子,邑三百户。出为高唐郡太守。二年,除明威将军、庐陵太守。时周迪据临川反,东昌县人修行师应之,率兵以攻子隆,其锋甚盛。子隆设伏于外,仍闭门偃甲,示之以弱。及行师至,腹背击之,行师大败,因乞降,子隆许之,送于京师。

四年,周迪引陈宝应复出临川,子隆随都督章昭达讨迪。迪退走,因随昭达逾东兴岭,讨陈宝应。军至建安,以子隆监郡。宝应据建安之湖际以拒官军,子隆与昭达各据一营,昭达先与贼战,不利,亡其鼓角,子隆闻之,率兵来救,大破贼徒,尽获昭达所亡羽仪甲仗。晋安平,子隆功最,迁假节、都督武州诸军事,将军如故。寻改封朝阳县伯,邑五百户。废帝即位,进号智武将军,加员外散骑常侍,馀如故。

华皎据湘州反,以子隆居其心腹,皎深患之,频遣使招诱,子隆不从,皎因遣兵攻之,又不能克。及皎败于郢州,子隆出兵以袭其后,因与王师相会。授持节、通直散骑常侍、都督武州诸军事,进爵为侯,增邑并前七百户。寻迁都督荆、信、祐三州诸军事、宣毅将军、荆州刺史,持节、常侍如故。是时荆州新置,治于公安,城池未固,子隆修建城郭,绥集夷夏,甚得民和,当时号为称职。三年,吏民诣都上表,请立碑颂美功绩,诏许之。太建元年,进号云麾将军。二年卒,时年四十七。赠散骑常侍,谥曰威。子之武嗣。

之武年十六,领其旧军,随吴明彻北伐有功,官至王府主簿、弘农太守,仍隶明彻。明彻于吕梁败绩,之武逃归,为人所害,时年二十二。

子隆弟子才,亦有干略,从子隆征讨有功,除南平太守,封始兴县子,邑三百户。从吴明彻北伐,监安州,镇于宿预。除中卫始兴王咨议参军,迁飙猛将军、信州刺史。太建十三年卒,时年四十二,赠员外散骑常侍。

钱道戢,字子韬,吴兴长城人也。父景深,梁汉寿令。道戢少以孝行著闻,及长,颇有干略,高祖微时,以从妹妻焉。从平卢子略于广州,除滨江令。高祖辅政,遣道戢随世祖平张彪于会稽,以功拜直阁将军,除员外散骑常侍、假节、东徐州刺史,封永安县侯,邑五百户。仍领甲卒三千,随侯安都镇防梁山,寻领钱塘、馀杭二县令。永定三年,随世祖镇于南皖口。天嘉元年,又领剡令,镇于县之南岩,寻为临海太守,镇岩如故。

侯安都之讨留异也,道戢帅军出松阳以断其后。异平,以功拜持节、通直散骑常侍、轻车将军、都督东西二衡州诸军事、衡州刺史,领始兴内史。光大元年,增邑并前七百户。

高宗即位,征欧阳纥入朝,纥疑惧,乃举兵来攻衡州,道戢与战,却之。及都督章昭达率兵讨纥,以道戢为步军都督,由间道断纥之后。纥平,除左卫将军。

太建二年,又随昭达征萧岿于江陵,道戢别督众军与陆子隆焚青泥舟舰,仍为昭达前军,攻安蜀城,降之。以功加散骑常侍、仁武将军,增邑并前九百户。其年,迁仁威将军、吴兴太守。未行,改授使持节、都督郢、巴、武三州诸军事、郢州刺史。王师北讨,道戢与仪同黄法抃围历阳。历阳城平,因以道戢镇之。以功加云麾将军,增邑并前一千五百户。其年十一月遘疾卒,时年六十三。赠本官,谥曰肃。子邈嗣。

骆牙,字旗门,吴兴临安人也。祖秘道,梁安成王田曹参军。父裕,鄱阳嗣王中兵参军事。牙年十二,宗人有善相者,云:``此郎容貌非常,必将远致。''梁太清末,世祖尝避地临安,牙母陵,睹世祖仪表,知非常人,宾待甚厚。及世祖为吴兴太守,引牙为将帅,因从平杜龛、张彪等,每战辄先锋陷阵,勇冠众军,以功授真阁将军。太平二年,以母忧去职。世祖镇会稽,起为山阴令。永定三年,除安东府中兵参军,出镇冶城。寻从世祖拒王琳于南皖。世祖即位,授假节、威虏将军、员外散骑常侍,封常安县侯,邑五百户。寻为临安令,迁越州刺史,馀并如故。

初,牙母之卒也,于时饥馑兵荒,至是始葬,诏赠牙母常安国太夫人,谥曰恭。迁牙为贞威将军、晋陵太守。

三年,以平周迪之功,迁冠军将军、临川内史。太建三年,授安远将军、衡阳内史,未拜,徙为桂阳太守。八年,还朝,迁散骑常侍,入直殿省。十年,授豊州刺史,馀并如故。至德二年卒,时年五十七。赠安远将军、广州刺史。子义嗣。

史臣曰:陆子隆、钱道戢,或举门愿从,或旧齿树勋,有统领之才,充师旅之寄。至于受任籓屏,功绩并著,美矣!骆牙识真有奉,知世祖天授之德,盖张良之亚欤?牙母智深先觉,符柏谷之礼,君子知鉴识弘远,其在兹乎!

\hypertarget{header-n4703}{%
\subsubsection{卷十七}\label{header-n4703}}

沈君理王瑒陆缮

沈君理,字仲伦,吴兴人也。祖僧畟,梁左民尚书。父巡,素与高祖相善,梁太清中为东阳太守。侯景平后,元帝征为少府卿。荆州陷,萧詧署金紫光禄大夫。

君理美风仪,博涉经史,有识鉴。起家湘东王法曹参军。高祖镇南徐州,巡遣君理自东阳谒于高祖,高祖器之,命尚会稽长公主,辟为府西曹掾,稍迁中卫豫章王从事中郎,寻加明威将军,兼尚书吏部侍郎。迁给事黄门侍郎,监吴郡。高祖受禅,拜驸马都尉,封永安亭侯。出为吴郡太守。是时兵革未宁,百姓荒弊,军国之用,咸资东境,君理招集士卒,修治器械,民下悦附,深以干理见称。

世祖嗣位,征为侍中,迁守左民尚书,未拜,为明威将军、丹阳尹。天嘉三年,重授左民尚书,领步兵校尉,寻改前军将军。四年,侯安都徙镇江州,以本官监南徐州。六年,出为仁威将军、东阳太守。天康元年,以父忧去职。君理因自请往荆州迎丧柩,朝议以在位重臣,难令出境,乃遣令长兄君严往焉。及还,将葬,诏赠巡侍中、领军将军,谥曰敬子。其年起君理为信威将军、左卫将军。又起为持节、都督东衡、衡二州诸军事、仁威将军、东衡州刺史,领始兴内史。又起为明威将军、中书令。前后夺情者三,并不就。

太建元年,服阕,除太子詹事,行东宫事,迁吏部尚书。二年,高宗以君理女为皇太子妃,赐爵望蔡县侯,邑五百户。四年,加侍中。五年,迁尚书右仆射,领吏部,侍中如故。其年有疾,舆驾亲临视,九月卒,时年四十九。诏赠侍中、太子少傅。丧事所须,随由资给。重赠翊左将军、开府仪同三司,侍中如故。谥曰贞宪。君理子遵俭早卒,以弟君高子遵礼为嗣。

君理第五叔迈,亦方正有干局,仕梁为尚书金部郎。永定中,累迁中书侍郎。天嘉中,历太仆、廷尉,出为镇东始兴王长史、会稽郡丞,行东扬州事。光大元年,除尚书吏部郎。太建元年,迁为通直散骑常侍,侍东宫。二年卒,时年五十二,赠散骑常侍。

君理第六弟君高,字季高,少知名,性刚直,有吏能。以家门外戚,早居清显,历太子舍人、洗马、中舍人、高宗司空府从事中郎、廷尉卿。太建元年,东境大水,百姓饥弊,乃以君高为贞威将军、吴令。寻除太子中庶子、尚书吏部郎、卫尉卿。出为宣远将军、平南长沙王长史、南海太守,行广州事。以女为王妃,固辞不行,复为卫尉卿。八年,诏授持节、都督广等十八州诸军事、宁远将军、平越中郎将、广州刺史。岭南俚、獠世相攻伐,君高本文吏,无武干,推心抚御,甚得民和。十年,卒于官,时年四十七。赠散骑常侍,谥曰祁子。

王瑒,字子玙,司空冲之第十二子也。沈静有器局,美风仪,举止酝藉。梁大同中,起家秘书郎,迁太子洗马。元帝承制,征为中书侍郎,直殿省,仍掌相府管记。出为东宫内史,迁太子中庶子。丁所生母忧,归于丹阳。江陵陷,梁敬帝承制,除仁威将军、尚书吏部郎中。贞阳侯僭位,以敬帝为太子,授瑒散骑常侍,侍东宫。寻迁长史兼侍中。

高祖入辅,以为司徒左长史。永定元年,迁守五兵尚书。世祖嗣位,授散骑常侍,领太子庶子,侍东宫。迁领左骁骑将军、太子中庶子,常侍、侍中如故。瑒为侍中六载,父冲尝为瑒辞领中庶子,世祖顾谓冲曰:``所以久留瑒于承华,政欲使太子微有瑒风法耳。''废帝嗣位,以侍中领左骁骑将军。光大元年,以父忧去职。

高宗即位,太建元年,复除侍中,领左骁骑将军。迁度支尚书,领羽林监。出为信威将军、云麾始兴王长史,行州府事。未行,迁中书令,寻加散骑常侍,除吏部尚书,常侍如故。瑒性宽和,及居选职,务在清静,谨守文案,无所抑扬。寻授尚书右仆射,未拜,加侍中,迁左仆射,参掌选事,侍中如故。瑒兄弟三十馀人,居家笃睦,每岁时馈遗,遍及近亲,敦诱诸弟,并禀其规训。太建八年卒,时年五十四。赠侍中、特进、护军将军。丧事随所资给。谥曰光子。

瑒第十三弟瑜,字子珪,亦知名,美容仪,早历清显,年三十,官至侍中。永定元年,使于齐,以陈郡袁宪为副,齐以王琳之故,执而囚之。齐文宣帝每行,载死囚以从,齐人呼曰``供御囚'',每有他怒,则召杀之,以快其意。瑜及宪并危殆者数矣,齐仆射杨遵彦悯其无辜,每救护之。天嘉二年还朝,诏复侍中,顷之卒,时年四十。赠本官,谥曰贞子。

陆缮,字士繻,吴郡吴人也。祖惠晓,齐太常卿。父任,梁御史中丞。缮幼有志尚,以雅正知名。起家梁宣惠武陵王法曹参军。承圣中,授中书侍郎,掌东宫管记。江陵陷,缮微服遁还京师。绍泰元年,除司徒右长史,御史中丞,以父任所终,固辞不就。高祖引缮为司徒司马,迁给事黄门侍郎、领步兵校尉、通直散骑常侍,兼侍中。永定元年,迁侍中。时留异拥割东阳,新安人向文政与异连结,因据本郡,朝廷以缮为贞威将军、新安太守。

世祖嗣位,征为太子中庶子,领步兵校尉,掌东宫管记。缮仪表端丽,进退闲雅,世祖使太子诸王咸取则焉。其趋步蹑履,皆令习缮规矩。除尚书吏部郎中,步兵如故,仍侍东宫。陈宝应平后,出为贞毅将军、建安太守。秩满,为散骑常侍、御史中丞,犹以父之所终,固辞,不许,乃权换廨宇徙居之。

太建初,迁度支尚书、侍中、太子詹事,行东宫事,领扬州大中正。及太子亲莅庶政,解行事,加散骑常侍,改加侍中。迁尚书右仆射,寻迁左仆射,参掌选事,侍中如故。更为尚书仆射,领前将军。重授左仆射,领扬州大中正,别敕令与徐陵等七人参议政事。十二年卒,时年六十三。赠侍中、特进、金紫光禄大夫,谥曰安子。太子以缮东宫旧臣,特赐祖奠。

缮子辩惠,年数岁,诏引入殿内,辩惠应对进止有父风,高宗因赐名辩惠,字敬仁云。

缮兄子见贤,亦方雅,高宗为扬州牧,乃以为治中从事史,深被知遇。历给事黄门侍郎,长沙、鄱阳二王长史,带寻阳太守,少府卿。太建十年卒,时年五十。赠廷尉卿,谥曰平子。

史臣曰:夫衣冠雅道,廊庙嘉猷,谅以操履敦修,局宇详正。经曰``容止可观'',《诗》言``其仪罔忒'',彼三子者,其有斯风焉。

\hypertarget{header-n4723}{%
\subsubsection{卷十八}\label{header-n4723}}

周弘正弟弘直弘直子确袁宪

周弘正,字思行,汝南安城人,晋光禄大夫顗之九世孙也。祖颙,齐中书侍郎,领著作。父宝始,梁司徒祭酒。弘正幼孤,及弟弘让、弘直,俱为伯父侍中护军舍所养。年十岁,通《老子》、《周易》,舍每与谈论,辄异之,曰:``观汝神情颖晤,清理警发,后世知名,当出吾右。''河东裴子野深相赏纳,请以女妻之。十五,召补国子生,仍于国学讲《周易》,诸生传习其义。以季春入学,孟冬应举,学司以其日浅,弗之许焉。博士到洽议曰:``周郎年未弱冠,便自讲一经,虽曰诸生,实堪师表,无俟策试。''起家梁太学博士。晋安王为丹阳尹,引为主簿。出为鄴令,丁母忧去职。服阕,历曲阿、安吉令。普通中,初置司文义郎,直寿光省,以弘正为司义侍郎。

中大通三年,梁昭明太子薨,其嗣华容公不得立,乃以晋安王为皇太子,弘正乃奏记曰:

窃闻捴谦之象,起于羲、轩爻画,揖让之源,生于尧、舜禅受,其来尚矣,可得而详焉。夫以庙堂、汾水,殊途而同归,稷、契、巢、许,异名而一贯,出者称为元首,处者谓之外臣,莫不内外相资,表里成治,斯盖万代同规,百王不易者也。暨于三王之世,浸以陵夷,各亲其亲,各子其子。乃至七国争雄,刘项竞逐,皇汉扇其俗,有晋扬其波,谦让之道废,多历年所矣。夫文质递变,浇淳相革,还朴反古,今也其时。

伏惟明大王殿下,天挺将圣,聪明神武,百辟冠冕,四海归仁。是以皇上发德音,下明诏,以大王为国之储副,乃天下之本焉。虽复夏启、周诵,汉储、魏两,此数君者,安足为大王道哉。意者愿闻殿下抗目夷上仁之义,执子臧大贤之节,逃玉舆而弗乘,弃万乘如脱屣,庶改浇竞之俗,以大吴国之风。古有其人,今闻其语,能行之者,非殿下而谁?能使无为之化,复兴于邃古,让王之道,不坠于来叶,岂不盛欤!岂不盛欤!

弘正陋学书生,义惭稽古,家自汝、颍,世传忠烈,先人决曹掾燕抗辞九谏,高节万乘,正色三府,虽盛德之业将绝,而狂直之风未坠。是以敢布腹心,肆其愚瞽。如使刍言野说,少陈于听览,纵复委身烹鼎之下,绝命肺石之上,虽死之日,犹生之年。

其抗直守正,皆此类也。

累迁国子博士。时于城西立士林馆,弘正居以讲授,听者倾朝野焉。弘正启梁武帝《周易》疑义五十条,又请释《乾》、《坤》、《二系》曰:``臣闻《易》称立以尽意,系辞以尽言,然后知圣人之情,几可见矣。自非含微体极,尽化穷神,岂能通志成务,探赜致远。而宣尼比之桎梏,绝韦编于漆字,轩辕之所听莹,遗玄珠于赤水。伏惟陛下一日万机,匪劳神于瞬息,凝心妙本,常自得于天真,圣智无以隐其几深,明神无以沦其不测。至若爻画之苞于《六经》,文辞之穷于《两系》,名儒剧谈以历载,鸿生抵掌以终年,莫有试游其籓,未尝一见其涘。自制旨降谈,裁成《易》道,析至微于秋毫,涣曾冰于幽谷。臣亲承音旨,职司宣授,后进诜诜,不无传业。但《乾》、《坤》之蕴未剖,《系》表之妙莫诠,使一经深致,尚多所惑。臣不涯庸浅,轻率短陋,谨与受业诸生清河张讥等三百一十二人,于《乾》、《坤》、《二系》、《象》、《爻》未启,伏愿听览之闲,曲垂提训,得使微臣钻仰,成其笃习,后昆好事,专门有奉。自惟多幸,欢沐道于尧年,肄业终身,不知老之将至。天尊不闻,而冒陈请,冰谷置怀,罔识攸厝。''诏答曰:``设《卦》观象,事远文高,作《系》表言,辞深理奥,东鲁绝编之思,西伯幽忧之作,事逾三古,人更七圣,自商瞿禀承,子庸传授,篇简湮没,岁月辽远。田生表菑川之誉,梁丘擅琅邪之学,代郡范生,山阳王氏,人藏荆山之宝,各尽玄言之趣,说或去取,意有详略。近搢绅之学,咸有稽疑,随答所问,已具别解。知与张讥等三百一十二人须释《乾》、《坤》、《文言》及《二系》,万机小暇,试当讨论。''

弘正博物知玄象,善占候。大同末,尝谓弟弘让曰:``国家厄运,数年当有兵起,吾与汝不知何所逃之。''及梁武帝纳侯景,弘正谓弘让曰:``乱阶此矣。''京城陷,弘直为衡阳内史,元帝在江陵,遗弘直书曰:``适有都信,贤兄博士平安。但京师搢绅,无不附逆,王克已为家臣,陆缅身充卒伍,唯有周生,确乎不拔。言及西军,潺湲掩泪,恒思吾至,如望岁焉,松柏后凋,一人而已。''王僧辩之讨侯景也,弘正与弘让自拔迎军,僧辩得之甚喜,即日启元帝,元帝手书与弘正曰:``獯丑逆乱,寒暑亟离,海内相识,零落略尽。韩非之智,不免秦狱,刘歆之学,犹弊亡新,音尘不嗣,每以耿灼。常欲访山东而寻子云,问关西而求伯起,遇有今信,力附相闻,迟比来邮,慰其延伫。''仍遣使迎之,谓朝士曰:``晋氏平吴,喜获二陆,今我破贼,亦得两周,今古一时,足为连类。''及弘正至,礼数甚优,朝臣无与比者。授黄门侍郎,直侍中省。俄迁左民尚书,寻加散骑常侍。

元帝尝著《金楼子》,曰:``余于诸僧重招提琰法师,隐士重华阳陶贞白,士大夫重汝南周弘正,其于义理,清转无穷,亦一时之名士也。''及侯景平,僧辩启送秘书图籍,敕弘正雠校。

时朝议迁都,朝士家在荆州者,皆不欲迁,唯弘正与仆射王裒言于元帝曰:``若束脩以上诸士大夫微见古今者,知帝王所都本无定处,无所与疑。至如黔首万姓,若未见舆驾入建鄴,谓是列国诸王,未名天子。今宜赴百姓之心,从四海之望。''时荆陕人士咸云王、周皆是东人,志愿东下,恐非良计。弘正面折之曰:``若东人劝东,谓为非计,君等西人欲西,岂成良策?''元帝乃大笑之,竟不还都。

及江陵陷,弘正遁围而出,归于京师,敬帝以为大司马王僧辩长史,行扬州事。太平元年,授侍中,领国子祭酒,迁太常卿、都官尚书。高祖受禅,授太子詹事。天嘉元年,迁侍中、国子祭酒,往长安迎高宗。三年,自周还,诏授金紫光禄大夫,加金章紫绶,领慈训太仆。废帝嗣位,领都官尚书,总知五礼事。仍授太傅长史,加明威将军。高宗即位,迁特进,重领国子祭酒,豫州大中正,加扶。太建五年,授尚书右仆射,祭酒、中正如故。寻敕侍东宫讲《论语》、《孝经》。太子以弘正朝廷旧臣,德望素重,于是降情屈礼,横经请益,有师资之敬焉。

弘正特善玄言,兼明释典,虽硕学名僧,莫不请质疑滞。六年,卒于官,时年七十九。诏曰:``追远褒德,抑有恒规。故尚书右仆射、领国子祭酒、豫州大中正弘正,识宇凝深,艺业通备,辞林义府,国老民宗,道映庠门,望高礼阁,卒然殂殒,朕用恻然。可赠侍中、中书监,丧事所须,量加资给。''便出临哭。谥曰简子。所著《周易讲疏》十六卷,《论语疏》十一卷,《庄子疏》八卷,《老子疏》五卷,《孝经疏》两卷,《集》二十卷,行于世。子坟,官至吏部郎。

弘正二弟:弘让,弘直。弘让性简素,博学多通,天嘉初,以白衣领太常卿、光禄大夫,加金章紫绶。

弘直字思方,幼而聪敏。解褐梁太学博士,稍迁西中郎湘东王外兵记室参军,与东海鲍泉、南阳宗懔、平原刘缓、沛郡刘同掌书记。入为尚书仪曹郎。湘东王出镇江、荆二州,累除录事咨议参军,带柴桑、当阳二县令。及梁元帝承制,授假节、英果将军、世子长史。寻除智武将军、衡阳内史。迁贞毅将军、平南长史、长沙内史,行湘州府州事,湘滨县侯,邑六百户。历邵陵、零陵太守、云麾将军、昌州刺史。王琳之举兵也,弘直在湘州,琳败,乃还朝。天嘉中,历国子博士、庐陵王长史、尚书左丞、领羽林监、中散大夫、秘书监,掌国史。迁太常卿、光禄大夫,加金章紫绶。

太建七年,遇疾且卒,乃遗疏敕其家曰:``吾今年已来,筋力减耗,可谓衰矣,而好生之情,曾不自觉,唯务行乐,不知老之将至。今时制云及,将同朝露,七十馀年,颇经称足,启手告全,差无遗恨。气绝已后,便买市中见材,材必须小形者,使易提挈。敛以时服,古人通制,但下见先人,必须备礼,可著单衣裙衫故履。既应侍养,宜备纷兑,或逢善友,又须香烟,棺内唯安白布手巾、粗香炉而已,其外一无所用。''卒于家,时年七十六。有集二十卷。子确。

确字士潜,美容仪,宽大有行检,博涉经史,笃好玄言,世父弘正特所钟爱。解褐梁太学博士、司徒祭酒、晋安王主簿。高祖受禅,除尚书殿中郎,累迁安成王限内记室。高宗即位,授东宫通事舍人,丁母忧,去职。及欧阳纥平,起为中书舍人,命于广州慰劳,服阕,为太常卿。历太子中庶子、尚书左丞、太子家令,以父忧去职。寻起为贞威将军、吴令,确固辞不之官。至德元年,授太子左卫率、中书舍人,迁散骑常侍,加贞威将军、信州南平王府长史,行扬州事,为政平允,称为良吏。迁都官尚书。祯明初,遘疾、卒于官,时年五十九。诏赠散骑常侍、太常卿,官给丧事。

袁宪,字德章,尚书左仆射枢之弟也。幼聪敏,好学,有雅量。梁武帝修建庠序,别开五馆,其一馆在宪宅西,宪常招引诸生,与之谈论,每有新议,出人意表,同辈咸嗟服焉。

大同八年,武帝撰《孔子正言章句》,诏下国学,宣制旨义。宪时年十四,被召为国子《正言》生,谒祭酒到溉,溉目而送之,爱其神彩。在学一岁,国子博士周弘正谓宪父君正曰:``贤子今兹欲策试不?''君正曰:``经义犹浅,未敢令试。''居数日,君正遣门下客岑文豪与宪候弘正,会弘正将登讲坐,弟子毕集,乃延宪入室,授以麈尾,令宪树义。时谢岐、何妥在坐,弘正谓曰:``二贤虽穷奥赜,得无惮此后生耶!''何、谢于是递起义端,深极理致,宪与往复数番,酬对闲敏。弘正谓妥曰:``恣卿所问,勿以童稚相期。''时学众满堂,观者重沓,而宪神色自若,辩论有馀。弘正请起数难,终不能屈,因告文豪曰:``卿还咨袁吴郡,此郎已堪见代为博士矣。''时生徒对策,多行贿赂,文豪请具束脩,君正曰:``我岂能用钱为儿买第耶?''学司衔之。及宪试,争起剧难,宪随问抗答,剖析如流,到溉顾宪曰:``袁君正其有后矣。''及君正将之吴郡,溉祖道于征虏亭,谓君正曰:``昨策生萧敏孙、徐孝克,非不解义,至于风神器局,去贤子远矣。''寻举高第。以贵公子选尚南沙公主,即梁简文之女也。

大同元年,释褐秘书郎。太清二年,迁太子舍人。侯景寇逆,宪东之吴郡,寻丁父忧,哀毁过礼。敬帝承制,征授尚书殿中郎。高祖作相,除司徒户曹。永定元年,授中书侍郎,兼散骑常侍。与黄门侍郎王瑜使齐,数年不遣,天嘉初乃还。四年,诏复中书侍郎,直侍中省。太建元年,除给事黄门侍郎,仍知太常事。二年,转尚书吏部侍郎,寻除散骑常侍,侍东宫。三年,迁御史中丞,领羽林监。时豫章王叔英不奉法度,逼取人马,宪依事劾奏,叔英由是坐免黜,自是朝野皆严惮焉。

宪详练朝章,尤明听断,至有狱情未尽而有司具法者,即伺闲暇,常为上言之,其所申理者甚众。尝陪宴承香阁,宾退之后,高宗留宪与卫尉樊俊徙席山亭,谈宴终日。高宗目宪而谓俊曰``袁家故为有人'',其见重如此。

五年,入为侍中。六年,除吴郡太守,以父任固辞不拜,改授明威将军、南康内史。九年,秩满,除散骑常侍,兼吏部尚书,寻而为真。宪以久居清显,累表自求解任。高宗曰:``诸人在职,屡有谤书。卿处事已多,可谓清白,别相甄录,且勿致辞。''十三年,迁右仆射,参掌选事。先是宪长兄简懿子为左仆射,至是宪为右仆射,台省内目简懿为大仆射,宪为小仆射,朝廷荣之。

及高宗不豫,宪与吏部尚书毛喜俱受顾命。始兴王叔陵之肆逆也,宪指麾部分,预有力焉。后主被疮病笃,执宪手曰:``我儿尚幼,后事委卿。''宪曰:``群情喁喁,冀圣躬康复,后事之旨,未敢奉诏。''以功封建安县伯,邑四百户,领太子中庶子,馀并如故。寻除侍中、信威将军、太子詹事。

至德元年,太子加元服,二年,行释奠之礼,宪于是表请解职,后主不许,给扶二人,进号云麾将军,置佐史。皇太子颇不率典训,宪手表陈谏凡十条,皆援引古今,言辞切直,太子虽外示容纳,而心无悛改。后主欲立宠姬张贵妃子始安王为嗣,尝从容言之,吏部尚书蔡徵顺旨称赏,宪厉色折之曰:``皇太子国家储嗣,亿兆宅心。卿是何人,轻言废立!''夏,竟废太子为吴兴王。后主知宪有规谏之事,叹曰``袁德章实骨鲠之臣'',即日诏为尚书仆射。

祯明三年,隋军来伐,隋将贺若弼进烧宫城北掖门,宫卫皆散走,朝士稍各引去,惟宪卫侍左右。后主谓宪曰:``我从来待卿不先馀人,今日见卿,可谓岁寒知松柏后凋也。''后主遑遽将避匿,宪正色曰:``北兵之入,必无所犯,大事如此,陛下安之。臣愿陛下正衣冠,御前殿,依梁武见侯景故事。''后主不从,因下榻驰去。宪从后堂景阳殿入,后主投下井中,宪拜哭而出。

京城陷,入于隋,隋授使持节、昌州诸军事、开府仪同三司、昌州刺史。开皇十四年,诏授晋王府长史。十八年卒,时年七十。赠大将军,安城郡公,谥曰简。长子承家,仕隋至秘书丞、国子司业。

史臣曰:梁元帝称士大夫中重汝南周弘正,信哉斯言也!观其雅量标举,尤善玄言,亦一代之国师矣。袁宪风格整峻,徇义履道。韩子称为人臣委质,心无有二。宪弗渝终始,良可嘉焉。

\hypertarget{header-n4753}{%
\subsubsection{卷十九}\label{header-n4753}}

裴忌孙瑒

裴忌,字无畏,河东闻喜人也。祖髦,梁中散大夫。父之平,倜傥有志略,召补文德主帅。梁普通中众军北伐,之平随都督夏侯亶克定涡、潼,以功封费县侯。会衡州部民相聚寇抄,诏以之平为假节、超武将军、都督衡州五郡征讨诸军事。及之平至,即皆平殄,梁武帝甚嘉赏之。元帝承圣中,累迁散骑常侍、右卫将军、晋陵太守。世祖即位,除光禄大夫,慈训宫卫尉,并不就,乃筑山穿池,植以卉木,居处其中,有终焉之志。天康元年卒,赠仁威将军、光禄大夫,谥曰僖子。

忌少聪敏,有识量,颇涉史传,为当时所称。解褐梁豫章王法曹参军。侯景之乱,忌招集勇力,随高祖征讨,累功为宁远将军。及高祖诛王僧辩,僧辩弟僧智举兵据吴郡,高祖遣黄他率众攻之,僧智出兵于西昌门拒战,他与相持,不能克。高祖谓忌曰:``三吴奥壤,旧称饶沃,虽凶荒之馀,犹为殷盛,而今贼徒扇聚,天下摇心,非公无以定之,宜善思其策。''忌乃勒部下精兵,轻行倍道,自钱塘直趣吴郡,夜至城下,鼓噪薄之。僧智疑大军至,轻舟奔杜龛,忌入据其郡。高祖嘉之,表授吴郡太守。

高祖受禅,征为左卫将军。天嘉初,出为持节、南康内史。时义安太守张绍宾据郡反,世祖以忌为持节、都督岭北诸军事,率众讨平之。还除散骑常侍、司徒左长史。五年,授云麾将军、卫尉卿,封东兴县侯,邑六百户。及华皎称兵上流,高宗时为录尚书辅政,尽命众军出讨,委忌总知中外城防诸军事。及皎平,高宗即位,太建无年,授东阳太守,改封乐安县侯,邑一千户。四年,入为太府卿。五年,转都官尚书。

吴明彻督众军北伐,诏忌以本官监明彻军。淮南平,授军师将军、豫州刺史。忌善于绥抚,甚得民和。改授使持节、都督谯州诸军事、谯州刺史。未及之官,会明彻受诏进讨彭、汴,以忌为都督,与明彻掎角俱进。吕梁军败,陷于周,周授上开府。隋开皇十四年,卒于长安,时年七十三。

孙瑒,字德琏,吴郡吴人也。祖文惠,齐越骑校尉、清远太守。父循道,梁中散大夫,以雅素知名。瑒少倜傥,好谋略,博涉经史,尤便书翰。起家梁轻车临川嗣王行参军,累迁为安西邵陵王水曹中兵参军事。王出镇郢州,瑒尽室随府,甚被赏遇。太清之难,授假节、宣猛将军、军主。王僧辩之讨侯景也,王琳为前军,琳与瑒同门,乃表荐为戎昭将军、宜都太守,仍从僧辩救徐文盛于武昌。会郢州陷,乃留军镇巴陵,修战守之备。俄而侯景兵至,日夜攻围,瑒督所部兵悉力拒战,贼众奔退。瑒从大军沿流而下,及克姑熟,瑒力战有功,除员外散骑常侍,封富阳县侯,邑一千户。寻受假节、雄信将军、衡阳内史,未及之官,仍迁衡州平南府司马。破黄洞蛮贼有功,除东莞太守,行广州刺史。寻除智武将军,监湘州事。敬帝嗣位,授持节、仁威将军、巴州刺史。

高祖受禅,王琳立梁永嘉王萧庄于郢州,征瑒为太府卿,加通直散骑常侍。及王琳入寇,以瑒为使持节、散骑常侍、都督郢、荆、巴、武、湘五州诸军事、安西将军、郢州刺史,总留府之任。周遣大将史宁率众四万,乘虚奄至,瑒助防张世贵举外城以应之,所失军民男女三千馀口。周军又起土山高梯,日夜攻逼,因风纵火,烧其内城南面五十馀楼。时瑒兵不满千人,乘城拒守,瑒亲自抚巡,行酒赋食,士卒皆为之用命。周人苦攻不能克,乃矫授瑒柱国、郢州刺史,封万户郡公。瑒伪许以缓之,而潜修战具,楼雉器械,一朝严设,周人甚惮焉。及闻大军败王琳,乘胜而进,周兵乃解。瑒于是尽有中流之地,集其将士而谓之曰:``吾与王公陈力协义,同奖梁室,亦已勤矣。今时事如此,天可违乎!''遂遣使奉表诣阙。

天嘉元年,授使持节、散骑常侍、安南将军、湘州刺史,封定襄县侯,邑一千户。瑒怀不自安,乃固请入朝,征为散骑常侍、中领军。未拜,而世祖从容谓瑒曰:``昔硃买臣愿为本郡,卿岂有意乎?''仍改授持节、安东将军、吴郡太守,给鼓吹一部。及将之镇,乘舆幸近畿饯送,乡里荣之。秩满,征拜散骑常侍、中护军,鼓吹如故。留异之反东阳,诏瑒督舟师进讨。异平,迁镇右将军,常侍、鼓吹并如故。顷之,出为使持节、安东将军、建安太守。光大中,以公事免,寻起为通直散骑常侍。

高宗即位,以瑒功名素著,深委任焉。太建四年,授都督荆、信二州诸军事、安西将军、荆州刺史,出镇公安。瑒增修城池,怀服边远,为邻境所惮。居职六年,又以事免,更为通直散骑常侍。及吴明彻军败吕梁,授使持节、督缘江水陆诸军事、镇西将军,给鼓吹一部。寻授散骑常侍、都督荆、郢、巴、武、湘五州诸军事、郢州刺史,持节、将军、鼓吹并如故。十二年,坐疆埸交通抵罪。

后主嗣位,复除通直散骑常侍,兼起部尚书。寻除中护军,复爵邑,入为度支尚书,领步兵校尉。俄加散骑常侍,迁侍中、祠部尚书。后主频幸其第,及著诗赋述勋德之美,展君臣之意焉。又为五兵尚书,领右军将军,侍中如故。以年老累乞骸骨,优诏不许。祯明元年卒官,时年七十二。后主临哭尽哀,赠护军将军,侍中如故,给鼓吹一部,朝服一具,衣一袭,丧事量加资给,谥曰桓子。

瑒事亲以孝闻,于诸弟甚笃睦。性通泰,有财物散之亲友。其自居处,颇失于奢豪,庭院穿筑,极林泉之致,歌钟舞女,当世罕俦,宾客填门,轩盖不绝。及出镇郢州,乃合十馀船为大舫,于中立亭池,植荷芰,每良辰美景,宾僚并集,泛长江而置酒,亦一时之胜赏焉。常于山斋设讲肆,集玄儒之士,冬夏资奉,为学者所称。而处己率易,不以名位骄物。时兴皇寺朗法师该通释典,瑒每造讲筵,时有抗论,法侣莫不倾心。又巧思过人,为起部尚书,军国器械,多所创立。有鉴识,男女婚姻,皆择素贵。及卒,尚书令江总为其志铭,后主又题铭后四十字,遣左民尚书蔡徵宣敕就宅镌之。其词曰:``秋风动竹,烟水惊波。几人樵径,何处山阿?今时日月,宿昔绮罗。天长路远,地久云多。功臣未勒,此意如何。''时论以为荣。瑒二十一子,咸有父风。世子让,早卒。第二子训,颇知名,历临湘令,直阁将军、高唐太守。陈亡入隋。

史臣曰:在梁之季,寇贼实繁,高祖建义杖旗,将宁区夏,裴忌早识攀附,每预戎麾,摧锋却敌,立功者数矣。孙瑒有文武干略。见知时主,及行军用兵,师司马之法,至于战胜攻取,屡著勋庸,加以好施接物,士咸慕向。然性不循恒,频以罪免,盖亦陈汤之徒焉。

\hypertarget{header-n4768}{%
\subsubsection{卷二十}\label{header-n4768}}

徐陵子俭份仪弟孝克

徐陵,字孝穆,东海郯人也。祖超之,齐郁林太守,梁员外散骑常侍。父摛,梁戎昭将军、太子左卫率,赠侍中、太子詹事,谥贞子。母臧氏,尝梦五色云化而为凤,集左肩上,已而诞陵焉。时宝志上人者,世称其有道,陵年数岁,家人携以候之,宝志手摩其顶,曰:``天上石麒麟也。''光宅惠云法师每嗟陵早成就,谓之颜回。八岁能属文,十二通《庄》、《老》义。既长,博涉史籍,纵横有口辩。

梁普通二年,晋安王为平西将军、宁蛮校尉,父摛为王咨议,王又引陵参宁蛮府军事。中大通三年,王立为皇太子,东宫置学士,陵充其选。稍迁尚书度支郎。出为上虞令,御史中丞刘孝仪与陵先有隙,风闻劾陵在县赃污,因坐免。久之,起为南平王府行参军,迁通直散骑侍郎。梁简文在东宫撰《长春殿义记》,使陵为序。又令于少傅府述所制《庄子义》。寻迁镇西湘东王中记室参军。

太清二年,兼通直散骑常侍。使魏,魏人授馆宴宾。是日甚热,其主客魏收嘲陵曰:``今日之热,当由徐常侍来。''陵即答曰:``昔王肃至此,为魏始制礼仪;今我来聘,使卿复知寒暑。''收大惭。

及侯景寇京师,陵父摛先在围城之内,陵不奉家信,便蔬食布衣,若居忧恤。会齐受魏禅,梁元帝承制于江陵,复通使于齐。陵累求复命,终拘留不遣,陵乃致书于仆射杨遵彦曰:

夫一言所感,凝晖照于鲁阳,一志冥通,飞泉涌于疏勒,况复元首康哉,股肱良哉,邻国相闻,风教相期者也!天道穷剥,钟乱本朝,情计驰惶,公私哽惧,而骸骨之请,徒淹岁寒,颠沛之祈,空盈卷轴,是所不图也,非所仰望也。

执事不闻之乎:昔分鰲命勣之世,观河拜洛之年,则有日乌流灾,风禽骋暴,天倾西北,地缺东南,盛旱坼三川,长波含五岳。我大梁应金图而有亢,纂玉镜而犹屯。何则?圣人不能为时,斯固穷通之恒理也。至如荆州刺史湘东王,机神之本,无寄名言,陶铸之馀,犹为尧、舜,虽复六代之舞,陈于总章,九州之歌,登于司乐,虞夔拊石,晋旷调钟,未足颂此英声,无以宣其盛德者也。若使郊禋楚翼,宁非祀夏之君,戡定艰难,便是匡周之霸,岂徒豳王徙雍,期月为都,姚帝迁河,周年成邑。方今越常藐藐,驯雉北飞,肃纻茫茫,风牛南偃,吾君之子,含识知归,而答旨云何所投身,斯其未喻一也。

又晋熙等郡,皆入贵朝,去我寻阳,经途何几。至于铛铛晓漏,的的宵烽,隔溆浦而相闻,临高台而可望。泉流宝碗,遥忆湓城,峰号香炉,依然庐岳。日者鄱阳嗣王治兵汇派,屯戍沦波,朝夕笺书,春秋方物,吾无从以蹑屩,彼何路而齐镳。岂其然乎?斯不然矣。又近者邵陵王通和此国,郢中上客,云聚魏都,鄴下名卿,风驰江浦,岂卢龙之径于彼新开,铜驼之街于我长闭?何彼途甚易,非劳于五丁,我路为难,如登于九折?地不私载,何其爽欤?而答旨云还路无从,斯所未喻二也。

晋熙、庐江,义阳、安陆,皆云款附,非复危邦,计彼中途,便当静晏,自斯以北,桴鼓不鸣,自此以南,封疆未壹。如其境外,脱殒轻躯,幸非边吏之羞,何在匹夫之命。又此宾游,通无货殖,忝非韩起聘郑,私买玉环,吴札过徐,躬要宝剑。由来宴锡,凡厥囊装,行役淹留,皆已虚罄,散有限之微财,供无期之久客,斯可知矣。且据图刎首,愚者不为,运斧全身,庸流所鉴。何则?生轻一发,自重千钧,不以贾盗明矣。骨肉不任充鼎俎,皮毛不足入货财,盗有道焉,吾无忧矣。又公家遣使,脱有资须,本朝非隆平之时,游客岂皇华之势。轻装独宿,非劳聚橐之仪,微骑闲行,宁望輶轩之礼。归人将从,私具驴骡,缘道亭邮,唯希蔬粟。若曰留之无烦于执事,遣之有费于官司,或以颠沛为言,或云资装可惧,固非通论,皆是外篇。斯所未喻三也。

又若以吾徒应还侯景,侯景凶逆,歼我国家,天下含灵,人怀愤厉,既不获投身社稷,卫难乘舆,四冢磔蚩尤,千刀剸王莽,安所谓俯首顿膝,归奉寇仇,珮弭腰鞬,为其皁隶?日者通和,方敦曩睦,凶人狙诈,遂骇狼心,颇疑宋万之诛,弥惧荀幹之请,所以奔蹄劲角,专恣凭陵,凡我行人,偏膺仇憾。政复俎筋醢骨,抽舌探肝,于彼凶情,犹当未雪,海内之所知也,君侯之所具焉。又闻本朝公主,都人士女,风行雨散,东播西流,京邑丘墟,奸蓬萧瑟,偃师还望,咸为草莱,霸陵回首,俱沾霜露,此又君之所知也。彼以何义,争免寇仇?我以何亲,争归委质?昔钜平贵将,悬重于陆公,叔向名流,深知于鬷篾。吾虽不敏,常慕前修,不图明庶有怀,翻其以此量物。昔魏氏将亡,群凶挺争,诸贤戮力,想得其朋。为葛荣之党邪?为邢杲之徒邪?如曰不然,斯所未喻四也。

假使吾徒还为凶党,侯景生于赵代,家自幽恒,居则台司,行为连率,山川形势,军国彝章,不劳请箸为筹,便当屈指能算。景以逋逃小丑,羊豕同群,身寓江皋,家留河朔,舂舂井井,如鬼如神。其不然乎?抑又君之所知也。且夫宫闱秘事,并若云霄,英俊訏谟,宁非帷幄,或阳惊以定策,或焚藁而奏书,朝廷之士,犹难参预,羁旅之人,何阶耳目。至于礼乐沿革,刑政宽猛,则讴歌已远,万舞成风,不知手之舞之足之蹈之也。安在摇其牙齿,为间谍者哉?若谓复命西朝,终奔东虏,虽齐、梁有隔,尉候奚殊?岂以河曲之难浮,而曰江关之可济?河桥马度,宁非宋典之奸?关路鸡鸣,皆曰田文之客。何其通蔽,乃尔相妨?斯所未喻五也。

又兵交使在,虽著前经,傥同徇仆之尤,追肆寒山之怒,则凡诸元帅,并释缧囚,爰及偏裨,同无翦馘。乃至钟仪见赦,朋笑遵途,襄老蒙归,《虞歌》引路。吾等张拭玉,修好寻盟,涉泗之与浮河,郊劳至于赠贿,公恩既被,宾敬无违,今者何愆,翻蒙贬责?若以此为言,斯所未喻六也。

若曰妖氛永久,丧乱悠然,哀我奔波,存其形魄,固已铭兹厚德,戴此洪恩,譬渤澥而俱深,方嵩、华而犹重。但山梁饮啄,非有意于笼樊,江海飞浮,本无情于钟鼓。况吾等营魂已谢,馀息空留,悲默为生,何能支久,是则虽蒙养护,更夭天年。若以此为言,斯所未喻七也。

若云逆竖歼夷,当听反命,高轩继路,飞盖相随,未解其言,何能善谑?夫屯亨治乱,岂有意于前期。谢常侍今年五十有一,吾今年四十有四,介已知命,宾又杖乡,计彼侯生,肩随而已。岂银台之要,彼未从师,金灶之方,吾知其决,政恐南阳菊水,竟不延龄,东海桑田,无由可望。若以此为言,斯所未喻八也。

足下清襟胜托,书囿文林,凡自洪荒,终乎幽、厉,如吾今日,宁有其人,爰至《春秋》,微宜商略。夫宗姬殄坠,霸道昏凶,或执政之多门,或陪臣之凉德,故臧孙有礼,翻囚与国之宾,周伯无愆,空怒天王之使,迁箕卿于两馆,絷骥子于三年。斯匪贪乱之风邪?宁当今之高例也?至于双崤且帝,四海争雄,或构赵而侵燕,或连韩而谋魏,身求盟于楚殿,躬夺璧于秦庭,输宝鼎以托齐王,驰安车而诱梁客。其外膏脣贩舌,分路扬镳,无罪无辜,如兄如弟。逮乎中阳受命,天下同规,巡省诸华,无闻幽辱。及三方之霸也,孙甘言以娬媚,曹屈诈以羁縻,旍轸岁到于句吴,冠盖年驰于庸蜀,则客嘲殊险,宾戏已深,共尽游谈,谁云猜忤,若使搜求故实,脱有前踪,恐是叔世之奸谋,而非为邦之胜略也。

抑又闻之,云师火帝,浇淳乃异其风,龙跃麟惊,王霸虽殊其道,莫不崇君亲以铭物,敦敬养以治民,预有邦司,曾无隆替。吾奉违温清,仍属乱离,寇虏猖狂,公私播越。萧轩靡御,王舫谁持?瞻望乡关,何心天地?自非生凭廪竹,源出空桑,行路含情,犹其相愍。常谓择官而仕,非曰孝家,择事而趋,非云忠国。况乎钦承有道,骖驾前王,郎吏明经,鸱鸢知礼,巡省方化,咸问高年,东序西胶,皆尊耆耋。吾以圭璋玉帛,通聘来朝,属世道之屯期,钟生民之否运,兼年累载,无申元直之祈,衔泣吞声,长对公闾之怒,情礼之诉,将同逆鳞,忠孝之言,皆应齚舌,是所不图也,非所仰望也。

且天伦之爱,何得忘怀?妻子之情,谁能无累?夫以清河公主之贵,馀姚书佐之家,莫限高卑,皆被驱略。自东南丑虏,抄贩饥民,台署郎官,俱馁墙壁,况吾生离死别,多历暄寒,孀室婴儿,何可言念。如得身还乡土,躬自推求,犹冀提携,俱免凶虐。

夫四聪不达,华阳君所谓乱臣,百姓无冤,孙叔敖称为良相。足下高才重誉,参赞经纶,非豹非貔,闻《诗》闻《礼》,而中朝大议,曾未矜论,清禁嘉谋,安能相及,谔谔非周舍,容容类胡广,何其无诤臣哉?岁月如流,平生何几,晨看旅雁,心赴江、淮,昏望牵牛,情驰扬、越,朝千悲而掩泣,夜万绪而回肠,不自知其为生,不自知其为死也。足下素挺词锋,兼长理窟,匡丞相解颐之说,乐令君清耳之谈,向所咨疑,谁能晓喻。若鄙言为谬,来旨必通,分请灰钉,甘从斧镬,何但规规默默,齰舌低头而已哉。若一理存焉,犹希矜眷,何必期令我等必死齐都,足赵、魏之黄尘,加幽、并之片骨,遂使东平拱树,长怀向汉之悲,西洛孤坟,恒表思乡之梦。干祈以屡,哽恸增深。

遵彦竟不报书。及江陵陷,齐送贞阳侯萧渊明为梁嗣,乃遣陵随还。太尉王僧辩初拒境不纳,渊明往复致书,皆陵词也。及渊明之入,僧辩得陵大喜,接待馈遗,其礼甚优。以陵为尚书吏部郎,掌诏诰。其年高祖率兵诛僧辩,仍进讨韦载。时任约、徐嗣徽乘虚袭石头,陵感僧辩旧恩,乃往赴约。及约等平,高祖释陵不问。寻以为贞威将军、尚书左丞。

绍泰二年,又使于齐,还除给事黄门侍郎、秘书监。高祖受禅,加散骑常侍,左丞如故。天嘉初,除太府卿。四年,迁五兵尚书,领大著作。六年,除散骑常侍、御史中丞。时安成王顼为司空,以帝弟之尊,势倾朝野。直兵鲍僧叡假王威权,抑塞辞讼,大臣莫敢言者。陵闻之,乃为奏弹,导从南台官属,引奏案而入。世祖见陵服章严肃,若不可犯,为敛容正坐。陵进读奏版时,安成王殿上侍立,仰视世祖,流汗失色。陵遣殿中御史引王下殿,遂劾免侍中、中书监。自此朝廷肃然。

天康元年,迁吏部尚书,领大著作。陵以梁末以来,选授多失其所,于是提举纲维,综核名实。时有冒进求官,喧竞不已者,陵乃为书宣示曰:``自古吏部尚书者,品藻人伦,简其才能,寻其门胄,逐其大小,量其官爵。梁元帝承侯景之凶荒,王太尉接荆州之祸败,尔时丧乱,无复典章,故使官方,穷此纷杂。永定之时,圣朝草创,干戈未息,亦无条序。府库空虚,赏赐悬乏,白银难得,黄札易营,权以官阶,代于钱绢,义存抚接,无计多少,致令员外、常侍,路上比肩,咨议、参军,市中无数,岂是朝章,应其如此?今衣冠礼乐,日富年华,何可犹作旧意,非理望也。所见诸君,多逾本分,犹言大屈,未喻高怀。若问梁朝硃领军异亦为卿相,此不逾其本分邪?此是天子所拔,非关选序。梁武帝云`世间人言有目色,我特不目色范悌'。宋文帝亦云`人世岂无运命,每有好官缺,辄忆羊玄保。'此则清阶显职,不由选也。秦有车府令赵高直至丞相,汉有高庙令田千秋亦为丞相,此复可为例邪?既忝衡流,应须粉墨。所望诸贤,深明鄙意。''自是众咸服焉。时论比之毛玠。

废帝即位,高宗入辅,谋黜异志者,引陵预其议。高宗纂历,封建昌县侯,邑五百户。太建元年,除尚书右仆射。三年,迁尚书左仆射,陵抗表推周弘正、王劢等,高宗召陵入内殿,曰:``卿何为固辞此职而举人乎?''陵曰:``周弘正从陛下西还,旧籓长史,王劢太平相府长史,张种帝乡贤戚,若选贤与旧,臣宜居后。''固辞累日,高宗苦属之,陵乃奉诏。

及朝议北伐,高宗曰:``朕意已决,卿可举元帅。''众议咸以中权将军淳于量位重,共署推之。陵独曰:``不然。吴明彻家在淮左,悉彼风俗,将略人才,当今亦无过者。''于是争论累日不能决。都官尚书裴忌曰:``臣同徐仆射。''陵应声曰:``非但明彻良将,裴忌即良副也。''是日,诏明彻为大都督,令忌监军事,遂克淮南数十州之地。高宗因置酒,举杯属陵曰:``赏卿知人。''陵避席对曰:``定策出自圣衷,非臣之力也。''其年加侍中,馀并如故。七年,领国子祭酒、南徐州大中正。以公事免侍中、仆射。寻加侍中,给扶,又除领军将军。八年,加翊右将军、太子詹事,置佐史。俄迁右光禄大夫,馀并如故。十年,重为领军将军。寻迁安右将军、丹阳尹。十三年,为中书监,领太子詹事,给鼓吹一部,侍中、将军、右光禄、中正如故。陵以年老累表求致仕,高宗亦优礼之,乃诏将作为造大斋,令陵就第摄事。

后主即位,迁左光禄大夫、太子少傅,馀如故。至德元年卒,时年七十七。诏曰:``慎终有典,抑乃旧章,令德可甄,谅宜追远。侍中、安右将军、左光禄大夫、太子少傅、南徐州大中正建昌县开国侯陵,弱龄学尚,登朝秀颖,业高名辈,文曰词宗。朕近岁承华,特相引狎,虽多卧疾,方期克壮,奄然殒逝,震悼于怀。可赠镇右将军、特进,其侍中、左光禄、鼓吹、侯如故,并出举哀,丧事所须,量加资给。谥曰章。''

陵器局深远,容止可观,性又清简,无所营树,禄俸与亲族共之。太建中,食建昌邑,邑户送米至于水次,陵亲戚有贫匮者,皆令取之,数日便尽,陵家寻致乏绝。府僚怪而问其故,陵云:``我有车牛衣裳可卖,馀家有可卖不?''其周给如此。少而崇信释教,经论多所精解。后主在东宫,令陵讲大品经,义学名僧,自远云集,每讲筵商较,四座莫能与抗。目有青睛,时人以为聪惠之相也。自有陈创业,文檄军书及禅授诏策,皆陵所制,而《九锡》尤美。为一代文宗,亦不以此矜物,未尝诋诃作者。其于后进之徒,接引无倦。世祖、高宗之世,国家有大手笔,皆陵草之。其文颇变旧体,缉裁巧密,多有新意。每一文出手,好事者已传写成诵,遂被之华夷,家藏其本。后逢丧乱,多散失,存者三十卷。有四子:俭,份,仪,僔。

俭一名众。幼而修立,勤学有志操,汝南周弘正重其为人,妻以女。梁太清初,起家豫章王府行参军。侯景乱,陵使魏未反,俭时年二十一,携老幼避于江陵,梁元帝闻其名,召为尚书金部郎中。尝侍宴赋诗,元帝叹赏曰:``徐氏之子,复有文矣。''江陵陷,复还于京师。永定初,为太子洗马,迁镇东从事中郎。天嘉三年,迁中书侍郎。

太建初,广州刺史欧阳纥举兵反,高宗令俭持节喻旨。纥初见俭,盛列仗卫,言辞不恭,俭曰:``吕嘉之事,诚当已远,将军独不见周迪、陈宝应乎?转祸为福,未为晚也。''纥默然不答,惧俭沮其众,不许入城,置俭于孤园寺,遣人守卫,累旬不得还。纥尝出见俭,俭谓之曰:``将军业已举事,俭须还报天子,俭之性命虽在将军,将军成败不在于俭,幸不见留。''纥于是乃遣俭从间道驰还。高宗乃命章昭达率众讨纥,仍以俭悉其形势,敕俭监昭达军。纥平,高宗嘉之,赐奴婢十人,米五百斛,除镇北鄱阳王咨议参军,兼中书舍人。累迁国子博士、大匠卿,馀并如故。寻迁黄门侍郎,转太子中庶子,加通直散骑常侍,兼尚书左丞,以公事免。寻起为中卫始兴王限外咨议参军,兼中书舍人。又为太子中庶子,迁贞威将军、太子左卫率,舍人如故。

后主立,授和戎将军、宣惠晋熙王长史,行丹阳郡国事。俄以父忧去职。寻起为和戎将军,累迁寻阳内史,为政严明,盗贼静息。迁散骑常侍,袭封建昌侯,入为御史中丞。俭性公平,无所阿附,尚书令江总望重一时,亦为俭所纠劾,后主深委任焉。又领右军。祯明二年卒。

份少有父风,年九岁,为《梦赋》,陵见之,谓所亲曰:``吾幼属文,亦不加此。''解褐为秘书郎。转太子舍人。累迁豫章王主簿、太子洗马。出为海盐令,甚有治绩。秩满,入为太子洗马。份性孝悌,陵尝遇疾,甚笃,份烧香泣涕,跪诵《孝经》,昼夜不息,如此者三日,陵疾豁然而愈,亲戚皆谓份孝感所致。太建二年卒,时年二十二。

仪少聪警,以《周易》生举高第为秘书郎,出为乌伤令。祯明初,迁尚书殿中郎,寻兼东宫学士。陈亡入隋。开皇九年,隐于钱塘之赭山,炀帝召为学士,寻除著作郎。大业四年卒。

孝克,陵之第三弟也。少为《周易》生,有口辩,能谈玄理。既长,遍通《五经》,博览史籍,亦善属文,而文不逮义。梁太清初,起家为太学博士。

性至孝,遭父忧,殆不胜丧,事所生母陈氏,尽就养之道。梁末,侯景寇乱,京邑大饥,饿死者十八九。孝克养母,饘粥不能给,妻东莞臧氏,领军将军臧盾之女也,甚有容色,孝克乃谓之曰:``今饥荒如此,供养交阙,欲嫁卿与富人,望彼此俱济,于卿意如何?''臧氏弗之许也。时有孔景行者,为侯景将,富于财,孝克密因媒者陈意,景行多从左右,逼而迎之,臧涕泣而去,所得谷帛,悉以供养。孝克又剃发为沙门,改名法整,兼乞食以充给焉。臧氏亦深念旧恩,数私自馈饷,故不乏绝。后景行战死,臧伺孝克于途中,累日乃见,谓孝克曰:``往日之事,非为相负,今既得脱,当归供养。''孝克默然无答。于是归俗,更为夫妻。

后东游,居于钱塘之佳义里,与诸僧讨论释典,遂通《三论》。每日二时讲,旦讲佛经,晚讲《礼传》,道俗受业者数百人。天嘉中,除剡令,非其好也,寻复去职。太建四年,征为秘书丞,不就,乃蔬食长斋,持菩萨戒,昼夜讲诵《法华经》,高宗甚嘉其操行。

六年,除国子博士,迁通直散骑常侍,兼国子祭酒,寻为真。孝克每侍宴,无所食啖,至席散,当其前膳羞损减,高宗密记以问中书舍人管斌,斌不能对。自是斌以意伺之,见孝克取珍果内绅带中,斌当时莫识其意,后更寻访,方知还以遗母。斌以实启,高宗嗟叹良久,乃敕所司,自今宴享,孝克前馔,并遣将还,以饷其母,时论美之。

至德中,皇太子入之释奠,百司陪列,孝克发《孝经》题,后主诏皇太子北面致敬。祯明元年,入为都官尚书。自晋以来,尚书官僚皆携家属居省。省在台城内下舍门,中有阁道,东西跨路,通于朝堂。其第一即都官之省,西抵阁道,年化久远,多有鬼怪,每昏夜之际,无故有声光,或见人著衣冠从井中出,须臾复没,或门阁自然开闭。居省者多死亡,尚书周确卒于此省。孝克代确,便即居之,经涉两载,妖变皆息,时人咸以为贞正所致。

孝克性清素而好施惠,故不免饥寒,后主敕以石头津税给之,孝克悉用设斋写经,随得随尽。二年,为散骑常侍,侍东宫。陈亡,随例入关。家道壁立,所生母患,欲粳米为粥,不能常办。母亡之后,孝克遂常啖麦,有遗粳米者,孝克对而悲泣,终身不复食之焉。

开皇十年,长安疾疫,隋文帝闻其名行,召令于尚书都堂讲《金刚般若经》。寻授国子博士。后侍东宫讲《礼传》。十九年,以疾卒,时年七十三。临终,正坐念佛,室内有非常异香气,邻里皆惊异之。子万载,仕至晋安王功曹史、太子洗马。

史臣曰:徐孝穆挺五行之秀,禀天地之灵,聪明特达,笼罩今古。及缔构兴王,遭逢泰运,位隆朝宰,献替谋猷,盖亮直存矣。孝克砥身厉行,养亲逾礼,亦参、闵之志欤!

\hypertarget{header-n4809}{%
\subsubsection{卷二十一}\label{header-n4809}}

江总姚察

江总,字总持,济阳考城人也,晋散骑常侍统之十世孙。五世祖湛,宋左光禄大夫、开府仪同三司,忠简公。祖蒨,梁光禄大夫,有名当代。父紑,本州迎主簿,少居父忧,以毁卒,在《梁书孝行传》。

总七岁而孤,依于外氏。幼聪敏,有至性。舅吴平光侯萧劢,名重当时,特所钟爱,尝谓总曰:``尔操行殊异,神采英拔,后之知名,当出吾右。''及长,笃学有辞采,家传赐书数千卷,总昼夜寻读,未尝辍手。年十八,解褐宣惠武陵王府法曹参军。中权将军、丹阳尹何敬容开府,置佐史,并以贵胄充之,仍除敬容府主簿。迁尚书殿中郎。梁武帝撰《正言》始毕,制《述怀诗》,总预同此作,帝览总诗,深降嗟赏。仍转侍郎。尚书仆射范阳张缵,度支尚书琅邪王筠,都官尚书南阳刘之遴,并高才硕学,总时年少有名,缵等雅相推重,为忘年友会。之遴尝酬总诗,其略曰:``上位居崇礼,寺署邻栖息。忌闻晓驺唱,每畏晨光赩。高谈意未穷,晤对赏无极。探急共遨游,休沐忘退食。曷用销鄙吝,枉趾觏颜色。下上数千载,扬搉吐胸臆。''其为通人所钦挹如此。迁太子洗马,又出为临安令,还为中军宣城王府限内录事参军,转太子中舍人。

及魏国通好,敕以总及徐陵摄官报聘,总以疾不行。侯景寇京都,诏以总权兼太常卿,守小庙。台城陷,总避难崎岖,累年,至会稽郡,憩于龙华寺,乃制《修心赋》,略序时事。其辞曰:

太清四年秋七月,避地于会稽龙华寺。此伽蓝者,余六世祖宋尚书右仆射州陵侯元嘉二十四年之所构也。侯之王父晋护军将军AN,昔莅此邦,卜居山阴都阳里,贻厥子孙,有终焉之志。寺域则宅之旧基,左江右湖,面山背豁,东西连跨,南北纡萦,聊与苦节名僧,同销日用,晓修经戒,夕览图书,寝处风云,凭栖水月。不意华戎莫辨,朝市倾沦,以此伤情,情可知矣。啜泣濡翰,岂摅郁结,庶后生君子,悯余此概焉。

嘉南斗之分次,肇东越之灵秘。表《桧风》于韩什,著镇山于周记。蕴大禹之金书,镌暴秦之石字。太史来而探穴,钟离去而开笥。信竹箭之为珍,何珷玞之罕值。奉盛德之鸿祀,寓安禅之古寺。实豫章之旧圃,成黄金之胜地。遂寂默之幽心,若镜中而远寻。面曾阜之超忽,迩平湖之迥深。山条偃蹇,水叶侵淫。挂猿朝落,饥鼯夜吟。果丛药苑,桃蹊橘林。梢云拂日,结暗生阴。保自然之雅趣,鄙人间之荒杂。望岛屿之邅回,面江源之重沓。泛流月之夜迥,曳光烟之晓匝。风引蜩而嘶噪,雨鸣林而修飒,鸟稍狎而知来,云无情而自合。尔乃野开灵塔,地筑禅居,喜园迢遰,乐树扶疏。经行籍草,宴坐临渠,持戒振锡,度影甘蔬。坚固之林可喻,寂灭之场蹔如。异曲终而悲起,非木落而悲始。岂降志而辱身,不露才而扬己。钟风雨之如晦,倦鸡鸣之聒耳。幸避地而高栖,凭调御之遗旨。折四辩之微言,悟三乘之妙理。遣十缠之系缚,祛五惑之尘滓。久遗荣于势利,庶忘累于妻子。感意气于畴日,寄知音于来祀。何远客之可悲,知自怜其何已。

总第九舅萧勃先据广州,总又自会稽往依焉。梁元帝平侯景,征总为明威将军、始兴内史,以郡秩米八百斛给总行装。会江陵陷,遂不行,总自此流寓岭南积岁。天嘉四年,以中书侍郎征还朝,直侍中省。累迁司徒右长史,掌东宫管记,给事黄门侍郎,领南徐州大中正。授太子中庶子、通直散骑常侍,东宫、中正如故。迁左民尚书,转太子詹事,中正如故。以与太子为长夜之饮,养良娣陈氏为女,太子微行总舍,上怒免之。寻为侍中,领左骁骑将军。复为左民尚书领左军将军,未拜,又以公事免。寻起为散骑常侍、明烈将军、司徒左长史,迁太常卿。

后主即位,除祠部尚书,又领左骁骑将军,参掌选事。转散骑常侍、吏部尚书。寻迁尚书仆射,参掌如故。至德四年,加宣惠将军,量置佐史。寻授尚书令,给鼓吹一部,加扶,馀并如故。策曰:``於戏,夫文昌政本,司会治经,韦彪谓之枢机,李固方之斗极。况其五曹斯综,百揆是谐,同冢宰之司,专台阁之任。惟尔道业标峻,寓量弘深,胜范清规,风流以为准的,辞宗学府,衣冠以为领袖。故能师长六官,具瞻允塞,明府八座,仪形载远,其端朝握揆,朕所望焉。往钦哉,懋建尔徽猷,亮采我邦国,可不慎欤!''祯明二年,进号中权将军。京城陷,入隋,为上开府。开皇十四年,卒于江都,时年七十六。

总尝自叙其略曰:

历升清显,备位朝列,不邀世利,不涉权幸。尝抚躬仰天太息曰:庄青翟位至丞相,无迹可纪;赵元叔为上计吏,光乎列传。官陈以来,未尝逢迎一物,干预一事。悠悠风尘,流俗之士,颇致怨憎,荣枯宠辱,不以介意。太建之世,权移群小,谄嫉作威,屡被摧黜,奈何命也。后主昔在东朝,留意文艺,夙荷昭晋,恩纪契阔。嗣位之日,时寄谬隆,仪形天府,厘正庶绩,八法六典,无所不统。昔晋武帝策荀公曾曰``周之冢宰,今之尚书令也''。况复才未半古,尸素若兹。晋太尉陆玩云``以我为三公,知天下无人矣''。轩冕傥来之一物,岂是预要乎?弱岁归心释教,年二十馀,入钟山就灵曜寺则法师受菩萨戒。暮齿官陈,与摄山布上人游款,深悟苦空,更复练戒,运善于心,行慈于物,颇知自励,而不能蔬菲,尚染尘劳,以此负愧平生耳。

总之自叙,时人谓之实录。

总笃行义,宽和温裕。好学,能属文,于五言七言尤善;然伤于浮艳,故为后主所爱幸。多有侧篇,好事者相传讽玩,于今不绝。后主之世,总当权宰,不持政务,但日与后主游宴后庭,共陈暄、孔范、王瑳等十馀人,当时谓之狎客。由是国政日颓,纲纪不立,有言之者,辄以罪斥之,君臣昏乱,以至于灭。有文集三十卷,并行于世焉。

长子溢,字深源,颇有文辞。性傲诞,恃势骄物,虽近属故友,不免诋欺。历官著作佐郎、太子舍人、洗马、中书黄门侍郎、太子中庶子。入隋,为秦王文学。

第七子漼,驸马都尉、秘书郎、隋给事郎,直秘书省学士。

姚察,字伯审,吴兴武康人也。九世祖信,吴太常卿,有名江左。察幼有至性,事亲以孝闻。六岁,诵书万馀言。弱不好弄,博弈杂戏,初不经心。勤苦厉精,以夜继日。年十二,便能属文。父上开府僧垣,知名梁武代,二宫礼遇优厚,每得供赐,皆回给察兄弟,为游学之资,察并用聚蓄图书,由是闻见日博。年十三,梁简文帝时在东宫,盛修文义,即引于宣猷堂听讲论难,为儒者所称。及简文嗣位,尤加礼接。起家南海王国左常侍,兼司文侍郎。除南郡王行参军,兼尚书驾部郎。

值梁室丧乱,于金陵随二亲还乡里。时东土兵荒,人饥相食,告籴无处,察家口既多,并采野蔬自给。察每崎岖艰阻,求请供养之资,粮粒恒得相继。又常以己分减推诸弟妹,乃至故旧乏绝者皆相分恤,自甘唯藜藿而已。在乱离之间,笃学不废。

元帝于荆州即位,父随朝士例往赴西台,元帝授察原乡令。时邑境萧条,流亡不反,察轻其赋役,劝以耕种,于是户口殷盛,民至今称焉。

中书侍郎领著作杜之伟与察深相眷遇,表用察佐著作,仍撰史。永定初,拜始兴王府功曹参军,寻补嘉德殿学士,转中卫、仪同始兴王府记室参军。吏部尚书徐陵时领著作,复引为史佐,及陵让官致仕等表,并请察制焉,陵见叹曰:``吾弗逮也。''太建初,补宣明殿学士,除散骑侍郎、左通直。寻兼通直散骑常侍,报聘于周。江左耆旧先在关右者,咸相倾慕。沛国刘臻窃于公馆访《汉书》疑事十馀条,并为剖析,皆有经据。臻谓所亲曰:``名下定无虚士。''著《西聘道里记》,所叙事甚详。

使还,补东宫学士。于时济阳江总、吴国顾野王、陆琼、从弟瑜、河南褚玠、北地傅縡等,皆以才学之美,晨夕娱侍。察每言论制述,咸为诸人宗重。储君深加礼异,情越群僚,宫内所须方幅手笔,皆付察立草。又数令共野王递相策问,恒蒙赏激。

迁尚书祠部侍郎。此曹职司郊庙,昔魏王肃奏祀天地,设宫县之乐,八佾之舞,尔后因循不革。梁武帝以为事人礼缛,事神礼简,古无宫县之文。陈初承用,莫有损益。高宗欲设备乐,付有司立议,以梁武帝为非。时硕学名儒、朝端在位者,咸希上旨,并即注同。察乃博引经籍,独违群议,据梁乐为是,当时惊骇,莫不惭服,仆射徐陵因改同察议。其不顺时随俗,皆此类也。

拜宣惠宜都王中录事参军,带东宫学士。历仁威淮南王、平南建安王二府咨议参军,丁内忧去职。俄起为戎昭将军,知撰梁史事,固辞不免。后主纂业,敕兼东宫通事舍人,将军、知撰史如故。又敕专知优册谥议等文笔。至德元年,除中书侍郎,转太子仆,馀并如故。

初,梁季沦没,父僧垣入于长安,察蔬食布衣,不听音乐,至是凶问因聘使到江南。时察母韦氏丧制适除,后主以察羸瘠,虑加毁顿,乃密遣中书舍人司马申就宅发哀,仍敕申专加譬抑。尔后又遣申宣旨诫喻曰:``知比哀毁过礼,甚用为忧。卿迥然一身,宗奠是寄,毁而灭性,圣教所不许。宜微自遣割,以存礼制。忧怀既深,故有此及。''

寻以忠毅将军起兼东宫通事舍人。察志在终丧,频有陈让,并抑而不许。又推表其略曰:``臣私门祸,并罹殃罚,偷生晷漏,冀申情礼,而尪疹相仍,苴緌秽质,非复人流,将毕苫壤。岂期朝恩曲覃,被之缨绂,寻斯宠服,弥见惭靦。且宫闼秘奥,趋奏便繁,宁可以兹荒毁所宜叨预。伏愿至德孝治,矜其理夺,使残魂喘息,以遂馀生。''诏答曰:``省表具怀。卿行业淳深,声誉素显,理徇情礼,未膺刀笔。但参务承华,良所期寄,允兹抑夺,不得致辞也。''俄敕知著作郎事,服阕,除给事黄门侍郎,领著作。

察既累居忧服,兼斋素日久,自免忧后,因加气疾。后主尝别召见,见察柴瘠过甚,为之动容,乃谓察曰:``朝廷惜卿,卿宜自惜,即蔬菲岁久,可停持长斋。''又遣度支尚书王瑗宣旨,重加慰喻,令从晚食。手敕曰:``卿羸瘠如此,斋菲累年,不宜一饭,有乖将摄,若从所示,甚为佳也。''察虽奉此敕,而犹敦宿誓。

又诏授秘书监,领著作如故,乃累进让,并优荅不许。察其秘书省大加删正,又奏撰中书表集。拜散骑常侍,寻授度支尚书,旬月迁吏部尚书,领著作并如故。察既博极坟素,尤善人物,至于姓氏所起,枝叶所分,官职姻娶,兴衰高下,举而论之,无所遗失。且澄鉴之职,时人久以梓匠相许,及迁选部,雅允朝望。初,吏部尚书蔡徵移中书令,后主方择其人,尚书令江总等咸共荐察,敕答曰:``姚察非唯学艺优博,亦是操行清修,典选难才,今得之矣。''乃神笔草诏,读以示察,察辞让甚切。

别日召入论选事,察垂涕拜请曰:``臣东皋贱族,身才庸近,情忘远致,念绝修途。顷来忝窃,久知逾分,特以东朝攀奉,恩纪谬加。今日叨滥,非由才举,纵陛下特升庸薄,其如朝序何?臣九世祖信,名高往代,当时才居选部,自后罕有继踪。臣遭逢成擢,沐浴恩造,累致非据,每切妨贤。臣虽无识,颇知审己,言行所践,无期荣贵,岂意铨衡之重,妄委非才。且皇明御历,事高昔代,羽仪世胄,帷幄名臣,若授受得宜,方为称职。臣夙陶教义,必知不可。''后主曰:``选众之举,佥议所归,昔毛玠雅量清恪,卢毓心平体正,王蕴铨量得地,山涛举不失才,就卿而求,必兼此矣。且我与卿虽君臣礼隔,情分殊常,藻镜人伦,良所期寄,亦以无惭则悊也。''

察自居显要,甚励清洁,且廪锡以外,一不交通。尝有私门生不敢厚饷,止送南布一端,花綀一匹。察谓之曰:``吾所衣著,止是麻布蒲綀,此物于吾无用。既欲相款接,幸不烦尔。''此人逊请,犹冀受纳,察厉色驱出,因此伏事者莫敢馈遗。

陈灭,入隋,开皇九年,诏授秘书丞,别敕成梁、陈二代史。又敕于硃华阁长参。文帝知察蔬菲,别日乃独召入内殿,赐果菜,乃指察谓朝臣曰:``闻姚察学行当今无比,我平陈唯得此一人。''十三年,袭封北绛郡公。察往岁之聘周也,因得与父僧垣相见,将别之际,绝而复苏,至是承袭,愈更悲感,见者莫不为之歔欷。

察幼年尝就钟山明庆寺尚禅师受菩萨戒,及官陈,禄俸皆舍寺起造,并追为禅师树碑,文甚遒丽。及是,遇见梁国子祭酒萧子云书此寺禅斋诗,览之怆然,乃用萧韵述怀为咏,词又哀切,法俗益以此称之。丁后母杜氏丧,解职。在服制之中,有白鸠巢于户上。

仁寿二年,诏曰:``前秘书丞北绛郡开国公姚察,强学待问,博极群典,脩身立德,白首不渝,虽在哀疚,宜夺情礼,可员外散骑常侍,封如故。''又敕侍晋王昭读。炀帝初在东宫,数被召见,访以文籍。即位之始,诏授太子内舍人,馀并如故。车驾巡幸,恒侍从焉。及改易衣冠,删正朝式,切问近对,察一人而已。

年七十四,大业二年,终于东都,遗命薄葬,务从率俭。其略曰:``吾家世素士,自有常法。吾意敛以法服,并宜用布,土周于身。又恐汝等不忍行此,必不尔,须松板薄棺,才可周身,土周于棺而已。葬日,止粗车,即送厝旧茔北。吾在梁世,当时年十四,就钟山明庆寺尚禅师受菩萨戒,自尔深悟苦空,颇知回向矣。尝得留连山寺,一去忘归。及仕陈代,诸名流遂许与声价,兼时主恩遇,宦途遂至通显。自入朝来,又蒙恩渥。既牵缠人世,素志弗从。且吾习蔬菲五十馀年,既历岁时,循而不失。瞑目之后,不须立灵,置一小床,每日设清水,六斋日设斋食果菜,任家有无,不须别经营也。''初,察愿读一藏经,并已究竟,将终,曾无痛恼,但西向坐,正念,云``一切空寂''。其后身体柔软,颜色如恒。两宫悼惜,赗赙甚厚。

察性至孝,有人伦鉴识。冲虚谦逊,不以所长矜人。终日恬静,唯以书记为乐,于坟籍无所不睹。每有制述,多用新奇,人所未见,咸重富博。且专志著书,白首不倦,手自抄撰,无时蹔辍。尤好研核古今,諟正文字,精采流赡,虽老不衰。兼谙识内典,所撰寺塔及众僧文章,特为绮密,在位多所称引,一善可录,无不赏荐。若非分相干,咸以理遣。尽心事上,知无不为。侍奉机密,未尝淹漏。且任遇已隆,衣冠攸属,深怀退静,避于声势。清洁自处,赀产每虚,或有劝营生计,笑而不答。穆于亲属,笃于旧故,所得禄赐,咸充周恤。

后主所制文笔,卷轴甚多,乃别写一本付察,有疑悉令刊定,察亦推心奉上,事在无隐。后主尝从容谓朝士曰:``姚察达学洽闻,手笔典裁,求之于古,犹难辈匹,在于今世,足为师范。且访对甚详明,听之使人忘倦。''察每制文笔,敕便索本,上曰:``我于姚察文章,非唯玩味无已,故是一宗匠。''

徐陵名高一代,每见察制述,尤所推重。尝谓子俭曰:``姚学士德学无前,汝可师之也。''尚书令江总与察尤笃厚善,每有制作,必先以简察,然后施用。总为詹事时,尝制登宫城五百字诗,当时副君及徐陵以下诸名贤并同此作。徐公后谓江曰:``我所和弟五十韵,寄弟集内。''及江编次文章,无复察所和本,述徐此意,谓察曰:``高才硕学,庶光拙文,今须公所和五百字,用偶徐侯章也。''察谦逊未付,江曰:``若不得公此制,仆诗亦须弃本,复乖徐公所寄,岂得见令两失。''察不获已,乃写本付之。为通人推挹,例皆如此。

所著《汉书训纂》三十卷,《说林》十卷,《四聘》、《玉玺》、《建康三钟》等记各一卷,悉穷该博,并《文集》二十卷,并行于世。察所撰梁、陈史虽未毕功,隋文帝开皇之时,遣内史舍人虞世基索本,且进上,今在内殿。梁、陈二史本多是察之所撰,其中序论及纪、传有所阙者,临亡之时,仍以体例诫约子思廉,博访撰续,思廉泣涕奉行。思廉在陈为衡阳王府法曹参军,转会稽王主簿。入隋,补驻王府行参军,掌记室,寻除河间郡司法。大业初,内史侍郎虞世基奏思廉踵成梁、陈二代史,自尔以来,稍就补续。

史臣曰:江总持清标简贵,加润以辞采,及师长六官,雅允朝望。史臣先臣禀兹令德,光斯百行,可以厉风俗,可以厚人伦。至于九流、《七略》之书,名山石室之记,汲郡、孔堂之书,玉箱金板之文,莫不穷研旨奥,遍探坎井,故道冠人师,晋绅以为准的。既历职贵显,国典朝章,古今疑议,后主皆取先臣断决焉。

\hypertarget{header-n4848}{%
\subsubsection{卷二十二}\label{header-n4848}}

世祖九王高宗二十九王后主十一子

世祖十三男:沈皇后生废帝、始兴王伯茂,严淑媛生鄱阳王伯山、晋安王伯恭,潘容华生新安王伯固,刘昭华生衡阳王伯信,王充华生庐陵王伯仁,张修容生江夏王伯义,韩修华生武陵王伯礼,江贵妃生永阳王伯智,孔贵妃生桂阳王伯谋。其伯固犯逆别有传。二男早卒,本书无名。

始兴王伯茂,字郁之,世祖第二子也。初,高祖兄始兴昭烈王道谈仕于梁世,为东宫直阁将军,侯景之乱,领弩手二千援台,于城中中流矢卒。太平二年,追赠侍中、使持节、都督南兗州诸军事、南兗州刺史,封长城县公,谥曰昭烈。高祖受禅,重赠骠骑大将军、太傅、扬州牧,改封始兴郡王,邑二千户。王生世祖及高宗。高宗以梁承圣末迁于关右,至是高祖遥以高宗袭封始兴嗣王,以奉昭烈王祀。永定三年六月,高祖崩,是月世祖入纂帝位。时高宗在周未还,世祖以本宗乏飨,其年十月下诏曰:``日者皇基肇建,封树枝戚,朕亲地攸在,特启大邦。弟顼嗣承门祀,虽土宇开建,荐飨莫由。重以遭家不造,闵凶夙遘,储贰遐隔,轊车未返。猥以眇身,膺兹景命,式循龟鼎,冰谷载怀。今既入奉大宗,事绝籓裸,始兴国庙蒸尝无主,瞻言霜露,感寻恸绝。其徙封嗣王顼为安成王,封第二子伯茂为始兴王,以奉昭烈王祀。赐天下为父后者爵一级。庶申罔极之情,永保山河之祚。''

旧制诸王受封,未加戎号者,不置佐史,于是尚书八座奏曰:``夫增崇徽号,饰表车服,所以阐彰厥德,下变民望。第二皇子新除始兴王伯茂,体自尊极,神姿明颖,玉暎觿辰,兰芬绮岁,清晖美誉,日茂月升,道郁平、河,声超衮、植。皇情追感,圣性天深,以本宗阙绪,纂承籓嗣,虽圭社是膺,而戎章未袭,岂所以光崇睿哲,宠树皇枝。臣等参议,宜加宁远将军,置佐史。''诏曰``可''。寻除使持节、都督南琅邪彭城二郡诸军事、彭城太守。天嘉二年,进号宣惠将军、扬州刺史。

伯茂性聪敏,好学,谦恭下士,又以太子母弟,世祖深爱重之。是时征北军人于丹徒盗发晋郗昙墓,大获晋右将军王羲之书及诸名贤遗迹。事觉,其书并没县官,藏于秘府,世祖以伯茂好古,多以赐之,由是伯茂大工草隶,甚得右军之法。三年,除镇东将军、开府仪同三司、东扬州刺史。

废帝即位,时伯茂在都,刘师知等矫诏出高宗也,伯茂劝成之。师知等诛后,高宗恐伯茂扇动朝廷,光大元年,乃进号中卫将军,令入居禁中,专与废帝游处。是时四海之望,咸归高宗,伯茂深不平,日夕愤怨,数肆恶言,高宗以其无能,不以为意。及建安人蒋裕与韩子高等谋反,伯茂并阴豫其事。二年十一月,皇太后令黜废帝为临海王,其日又下令曰:``伯茂轻薄,爰自弱龄,辜负严训,弥肆凶狡。常以次居介弟,宜秉国权,不涯年德,逾逞狂躁,图为祸乱,扇动宫闱,要招粗险,觖望台阁,嗣君丧道,由此乱阶,是诸凶德,咸作谋主。允宜罄彼司甸,刑斯蠙人。言念皇支,尚怀悲懑,可特降为温麻侯,宜加禁止,别遣就第。不意如此,言增泫叹。''时六门之外有别馆,以为诸王冠婚之所,名为婚第,至是命伯茂出居之。于路遇盗,殒于车中,时年十八。

鄱阳王伯山,字静之,世祖第三子也。伟容仪,举止闲雅,喜愠不形于色,世祖深器之。初高祖时,天下草创,诸王受封仪注多阙,及伯山受封,世祖欲重其事,天嘉元年七月丙辰,尚书八座奏曰:``臣闻本枝惟允,宗周之业以弘,盘石既建,皇汉之基斯远,故能协宣五运,规范百王,式固灵根,克隆卜世。第三皇子伯山,发睿德于龆年,表歧姿于丱日,光昭丹掖,晖暎青闱,而玉圭未秉,金锡靡驾,岂所以敦序维翰,建树籓戚。臣等参议,宜封鄱阳郡王。''诏曰``可''。乃遣散骑常侍、度支尚书萧睿持节兼太宰告于太庙;又遣五兵尚书王质持节兼太宰告于太社。其年十月,上临轩策命之曰:``於戏!夫建树籓屏,翼奖王室,钦若前典,咸必由之。惟尔夙挺圭璋,生知孝敬,令德茂亲,佥誉所集,启建大邦,实惟伦序,是用敬遵民瞻,锡此圭瑞。往钦哉!其勉树声业,永保宗社,可不慎欤!''策讫,敕令王公已下并宴于王第。仍授东中郎将、吴郡太守。六年,为缘江都督、平北将军、南徐州刺史。天康元年,进号镇北将军。

高宗辅政,不欲令伯山处边,光大元年,徙为镇东将军、东扬州刺史。太建元年,征为中卫将军、中领军。六年,又为征北将军、南徐州刺史。寻为征南将军、江州刺史。十一年,入为护军将军,加开府仪同三司,仍给鼓吹并扶。后主即位,进号中权大将军。至德四年,出为持节、都督东扬、豊二州诸军事、东扬州刺史,加侍中,馀并如故。祯明元年,丁所生母忧,去职。明年,起为镇卫大将军、开府仪同三司,给班剑十人。三年正月薨,时年四十。

伯山性宽厚,美风仪,又于诸王最长,后主深敬重之,每朝廷有冠婚飨宴之事,恒使伯山为主。及丁所生母忧,居丧以孝闻。后主尝幸吏部尚书蔡徵宅,因往吊之,伯山号恸殆绝,因起为镇卫将军,仍谓群臣曰:``鄱阳王至性可嘉,又是西第之长,豫章已兼司空,其亦须迁太尉。''未及发诏而伯山薨,寻值陈亡,遂无赠谥。

长子君范,太建中拜鄱阳国世子,寻为贞威将军、晋陵太守,未袭爵而隋师至。是时宗室王侯在都者百馀人,后主恐其为变,乃并召入,令屯朝堂,使豫章王叔英总督之,而又阴为之备。及六军败绩,相率出降,因从后主入关。至长安,隋文帝并配于陇右及河西诸州,各给田业以处之。初,君范与尚书仆射江总友善,至是总赠君范书五言诗,以叙他乡离别之意,辞甚酸切,当世文士咸讽诵之。大业二年,隋炀帝以后主第六女女婤为贵人,绝爱幸,因召陈氏子弟尽还京师,随才叙用,由是并为守宰,遍于天下。其年君范为温令。

晋安王伯恭字肃之,世祖第六子也。天嘉六年,立为晋安王。寻为平东将军、吴郡太守,置佐史。时伯恭年十馀岁,便留心政事,官曹治理。太建元年,入为安前将军、中护军,迁中领军。寻为中卫将军、扬州刺史,以公事免。四年,起为安左将军,寻为镇右将军、特进,给扶。六年,出为安南将军、南豫州刺史。九年,入为安前将军、祠部尚书。十一年,进号军师将军、尚书右仆射。十二年,迁仆射。十三年,迁左仆射。十四年,出为安南将军、湘州剌史,未拜。至德元年,为侍中、中卫将军、光禄大夫,丁所生母忧,去职。祯明元年,起为中卫将军、右光禄大夫,置佐史、扶并如故。三年入关。隋大业初,为成州刺史、太常卿。

衡阳王伯信,字孚之,世祖第七子也。天嘉元年,衡阳献王昌自周还朝,于道薨,其年世祖立伯信为衡阳王,奉献王祀。寻为宣惠将军、丹阳尹,置佐史。太建四年,为中护军。六年,为宣毅将军、扬州刺史。寻加侍中、散骑常侍。十一年,进号镇前将军,太子詹事,馀并如故。祯明元年,出为镇南将军、西衡州刺史。三年,隋军济江,与临汝侯方庆并为东衡州刺史王勇所害,事在方庆传。

庐陵王伯仁,字寿之,世祖第八子也。天嘉六年,立为庐陵王。太建初,为轻车将军,置佐史。七年,迁冠军将军、中领军。寻为平北将军、南徐州刺史。十二年,为翊左将军、中领军。贞明元年,加侍中、国子祭酒,领太子中庶子。三年入关,卒于长安。

长子番,先封湘滨侯,隋大业中,不资阳令。

江夏王伯义,字坚之,世祖第九子也。天嘉六年,立为江夏王。太建初,为宣惠将军、东扬州刺史,置佐史。寻为宣毅将军、持节、散骑常侍、都督合、霍二州诸军事、合州刺史。十四年,征为侍中、忠武将军、金紫光禄大夫。祯明三年入关,迁于瓜州,于道卒。

长子元基,先封湘潭侯,隋大业中为谷熟县令。

武陵王伯礼,字用之,世祖第十子也。天嘉六年,立为武陵王。太建初,为云旗将军、持节、都督吴兴诸军事、吴兴太守。在郡恣行暴掠,驱录民下,逼夺财货,前后委积,百姓患之。太建九年,为有司所劾,上曰:``王年少,未达治道,皆由佐史不能匡弼所致,特降军号,后若更犯,必致之以法,有司不言与同罪。''十一年春,被代征还,伯礼遂迁延不发。其年十月,散骑常侍、御史中丞徐君敷奏曰:``臣闻车屦不俟,君命之通规,夙夜匪懈,臣子之恒节。谨案云旗将军、持节、都督吴兴诸军事、吴兴太守武陵王伯礼,早擅英猷,久驰令问,惟良寄重,枌乡是属。圣上爱育黔黎,留情政本,共化求瘼,早赴皇心,遂复稽缓归骖,取移凉燠,迟回去鹢,空淹载路,淑慎未彰,违惰斯在,绳愆检迹,以为惩诫。臣等参议以见事免伯礼所居官,以王还第,谨以白简奏闻。''诏曰:``可''。祯明三年入关,隋大业中为散骑侍郎、临洮太守。

永阳王伯智,字策之,世祖第十二子也。少敦厚,有器局,博涉经史。太建中,立为永阳王。寻为侍中,加明威将军,置佐史。寻加散骑常侍,累迁尚书左仆射,出为使持节、都督东扬、豊二州诸军事、平东将军,领会稽内史。至德二年,入为侍中、翊左将军,加特进。祯明三年入关。隋大业中,为岐州司马,迁国子司业。

桂阳王伯谋,字深之,世祖第十三子也。太建中,立为桂阳王。七年,为明威将军,置佐史。寻为信威将军、丹阳尹。十年,加侍中。出为持节、都督吴兴诸军事、东中郎将、吴兴太守。十一年,加散骑常侍。至德元年薨。

子豊嗣,大业中,为番禾令。

高宗四十二男:柳皇后生后主,彭贵人生始兴王叔陵,曹淑华生豫章王叔英,何淑仪生长沙王叔坚、宜都王叔明,魏昭容生建安王叔卿,钱贵妃生河东王叔献,刘昭仪生新蔡王叔齐,袁昭容生晋熙王叔文、义阳王叔达、新会王叔坦,王姬生淮南王叔彪、巴山王叔雄,吴姬生始兴王叔重,徐姬生寻阳王叔俨,淳于姬生岳阳王叔慎,王修华生武昌王叔虞,韦修容生湘东王叔平,施姬生临贺王叔敖、沅陵王叔兴,曾姬生阳山王叔宣,杨姬生西阳王叔穆,申婕妤生南安王叔俭、南郡王叔澄、岳山王叔韶、太原王叔匡,袁姬生新兴王叔纯,吴姬生巴东王叔谟,刘姬生临江王叔显,秦姬生新宁王叔隆、新昌王叔荣。其皇子叔叡、叔忠、叔弘、叔毅、叔训、叔武、叔处、叔封等八人,并未及封。叔陵犯逆,别有传。三子早卒,本书无名。

豫章王叔英,字子烈,高宗第三子也。少宽厚仁爱。天嘉元年,封建安侯。太建元年,改封豫章王,仍为宣惠将军、都督东扬州诸军事、东扬州刺史。五年,进号平北将军、南豫州刺史。十一年,为镇前将军、江州刺史。后主即位,进号征南将军,寻加开府仪同三司、中卫大将军,馀并如故。四年,进号骠骑大将军。祯明元年,给鼓吹一部,班剑十人。其年,迁司空。三年,隋师济江,叔英知石头军戍事。寻令入屯朝堂。及六军败绩,降于隋将韩擒虎。其年入关。隋大业中为涪陵太守。

长子弘,至德元年,拜豫章国世子。

长沙王叔坚,字子成,高宗第四子也。母本吴中酒家隶,高宗微时,尝往饮,遂与通,及贵,召拜淑仪。叔坚少杰黠,凶虐使酒,尤好数术、卜筮、祝禁,钅容金琢玉,并究其妙。天嘉中,封豊城侯。太建元年,立为长沙王,仍为东中郎将、吴郡太守。四年,为宣毅将军、江州刺史,置佐史。七年,进号云麾将军、郢州刺史,未拜,转为平越中郎将、广州刺史。寻为平北将军、合州刺史。八年,复为平西将军、郢州刺史。十一年,入为翊左将军、丹阳尹。

初,叔坚与始兴王叔陵并招聚宾客,各争权宠,甚不平。每朝会卤簿,不肯为先后,必分道而趋,左右或争道而斗,至有死者。及高宗弗豫,叔坚、叔陵等并从后主侍疾。叔陵阴有异志,乃命典药吏曰:``切药刀甚钝,可砺之。''及高宗崩,仓卒之际,又命其左右于外取剑,左右弗悟,乃取朝服所佩木剑以进,叔陵怒。叔坚在侧闻之,疑有变,伺其所为。及翌日小敛,叔陵袖锉药刀趋进,斫后主,中项,后主闷绝于地,皇太后与后主乳母乐安君吴氏俱以身捍之,获免。叔坚自后扼叔陵,擒之,并夺其刀,将杀之,问后主曰:``即尽之,为待也?''后主不能应。叔陵旧多力,须臾,自奋得脱,出云龙门,入于东府城,召左右断青溪桥道,放东城囚以充战士。又遣人往新林,追其所部兵马,仍自被甲,著白布帽,登城西门,招募百姓。是时众军并缘江防守,台内空虚,叔坚乃白太后使太子舍人司马申以后主命召萧摩诃,令讨之。即日擒其将戴温、谭骐驎等,送台,斩于尚书阁下,持其首徇于东城。叔陵恇扰不知所为,乃尽杀其妻妾,率左右数百人走趋新林。摩诃追之,斩于丹阳郡,馀党悉擒。其年,以功进号骠骑将军、开府仪同三司、扬州刺史。寻迁司空,将军、刺史如故。

是时后主患创,不能视事,政无小大,悉委叔坚决之,于是势倾朝廷。叔坚因肆骄纵,事多不法,后主由是疏而忌之。孔范、管斌、施文庆之徒,并东宫旧臣,日夜阴持其短。至德元年,乃诏令即本号用三司之仪,出为江州刺史。未发,寻有诏又以为骠骑将军,重为司空,实欲去其权势。叔坚不自安,稍怨望,乃为左道厌魅以求福助,刻木为偶人,衣以道士之服,施机关,能拜跪,昼夜于日月下醮之,祝诅于上。其年冬,有人上书告其事,案验并实,后主召叔坚囚于西省,将杀之。其夜,令近侍宣敕,数之以罪,叔坚对曰:``臣之本心,非有他故,但欲求亲媚耳。臣既犯天宪,罪当万死,臣死之日,必见叔陵,愿宣明诏,责于九泉之下。''后主感其前功,乃赦之,特免所居官,以王还第。寻起为侍中、镇左将军。二年,又给鼓吹,油幢车。三年,出为征西将军、荆州刺史。四年,进号中军大将军、开府仪同三司。祯明二年,秩满还都。

三年入关,迁于瓜州,更名叔贤。叔贤素贵,不知家人生产,至是与妃沈氏酤酒,以佣保为事。隋大业中,为遂宁郡太守。

建安王叔卿,字子弼,高宗第五子也。性质直有材器,容貌甚伟。太建四年,立为建安王,授东中郎将、东扬州刺史。七年,为云麾将军、郢州刺史,置佐史。九年,进号平南将军、湘州刺史。后主即位,进号安南将军。又为侍中、镇右将军、中书令。迁中书监。祯明三年入关,隋大业中,为都官郎、上党通守。

宜都王叔明,字子昭,高宗第六子也。仪容美丽,举止和弱,状似妇人。太建五年,立为宜都王,寻授宣惠将军,置佐史。七年,授东中郎将、东扬州刺史,寻为轻车将军、卫尉卿。十三年,出为使持节、云麾将军、南徐州刺史。又为侍中、翊右将军。至德四年,进号安右将军。祯明三年入关,隋大业中为鸿胪少卿。

河东王叔献,字子恭,高宗第九子也。性恭谨,聪敏好学。太建五年,立为河东王。七年,授宣毅将军,置佐史。寻为散骑常侍、军师将军、都督南徐州诸军事、南徐州刺史。十二年薨,年十三。赠侍中、中抚将军、司空,谥曰康简。子孝宽嗣。孝宽以至德元年,袭爵河东王。祯明三年入关,隋大业中为汶城令。

新蔡王叔齐,字子肃,高宗第十一子也。风彩明赡,博涉经史,善属文。太建七年,立为新蔡王,寻为智武将军,置佐史。出为东中郎将、东扬州刺史。至德二年,入为侍中,将军、佐史如故。祯明元年,除国子祭酒,侍中、将军、佐史如故。三年入关。隋大业中为尚书主客郎。

晋熙王叔文,字子才,高宗第十二子也。性轻险,好虚誉,颇涉书史。太建七年,立为晋熙王。寻为侍中、散骑常侍、宣惠将军,置佐史。进号轻车将军、扬州刺史。至德元年,授持节、都督江州诸军事、江州刺史。二年,迁信威将军、督湘、衡、武、桂四州诸军事、湘州刺史。祯明二年,秩满,征为侍中、宣毅将军,佐史如故。未还,而隋军济江,破台城,隋汉东道行军元帅秦王至于汉口。时叔文自湘州还朝,至巴州,乃率巴州刺史毕宝等请降,致书于秦王曰:``窃以天无二日,晦明之序不差,土无二王,尊卑之位乃别。今车书混壹,文轨大同,敢披丹款,申其屈膝。''秦王得书,因遣行军吏部柳庄与元帅府僚属等往巴州迎劳叔文。叔文于是与毕宝、荆州刺史陈纪及文武将吏赴于汉口,秦王并厚待之,置于宾馆。隋开皇九年三月,众军凯旋,文帝亲幸温汤劳之,叔文与陈纪、周罗睺、荀法尚等并诸降人,见于路次。数日,叔文从后主及诸王侯将相并乘舆、服御、天文图籍等,并以次行列,仍以铁骑围之,随晋王、秦王等献凯而入,列于庙庭。明日,隋文帝坐于广阳门观,叔文又从后主至朝堂南。文帝使内史令李德林宣旨,责其君臣不能相弼,以致丧亡。后主与其群臣并惭惧拜伏,莫能仰视,叔文独欣然而有自得之志。旬有六日,乃上表曰:``昔在巴州,已先送款,乞知此情,望异常例。''文帝虽嫌其不忠,而方欲怀柔江表,乃授开府,拜宜州刺史。

淮南王叔彪,字子华,高宗第十三子也。少聪惠,善属文。太建八年,立为淮南王。寻位侍中、仁威将军,置佐史。祯明三年入关,卒于长安。

始兴王叔重,字子厚,高宗第十四子也。性质朴,无伎艺。高宗崩,始兴王叔陵为逆。诛死,其年立叔重为始兴王,以奉昭烈王后。至德元年,为仁威将军、扬州刺史,置佐史。二年,加使持节、都督江州诸军事、江州刺史。祯明三年入关。隋大业中为太府少卿,卒。

寻阳王叔俨,字子思,高宗第十五子也。性凝重,举止方正。后主即位,立为寻阳王。至德元年,为侍中、仁武将军,置佐史。祯明三年入关,寻卒。

岳阳王叔慎,字子敬,高宗第十六子也。少聪敏,十岁能属文。太建十四年,立为岳阳王,时年十一。至德四年,拜侍中、智武将军、丹阳尹。是时,后主尤爱文章,叔慎与衡阳王伯信、新蔡王叔齐等日夕陪侍,每应诏赋诗,恒被嗟赏。祯明元年,出为使持节、都督湘、衡、桂、武四州诸军事、智武将军、湘州刺史。三年,隋师济江,破台城,前刺史晋熙王叔文还至巴州,与巴州刺史毕宝、荆州刺史陈纪并降。隋行军元帅清河公杨素兵下荆门,别遣其将庞晖将兵略地,南至湘州,城内将士,莫有固志,克日请降。叔慎乃置酒会文武僚吏,酒酣,叔慎叹曰``君臣之义,尽于此乎!''长史谢基伏而流涕,湘州助防遂兴侯正理在坐,乃起曰:``主辱臣死,诸君独非陈国之臣乎?今天下有难,实是致命之秋也。纵其无成,犹见臣节,青门之外,有死不能。今日之机,不可犹豫,后应者斩。''众咸许诺,乃刑牲结盟。仍遣人诈奉降书于庞晖,晖信之,克期而入,叔慎伏甲待之。晖令数百人屯于城门,自将左右数十人入于厅事,俄而伏兵发,缚晖以徇,尽擒其党,皆斩之。叔慎坐于射堂,招合士众,数日之中,兵至五千人。衡阳太守樊通、武州刺史邬居业,皆请赴难。未至,隋遣中牟公薛胄为湘州刺史,闻庞晖死,乃益请兵,隋又遣行军总管刘仁恩救之。未至,薛胄兵次鹅羊山,叔慎遣正理及樊通等拒之,因大合战,自旦至于日昃,隋军迭息迭战,而正理兵少不敌,于是大败。胄乘胜入城,生擒叔慎。是时,邬居业率其众自武州来赴,出横桥江,闻叔慎败绩,乃顿于新康口。隋总管刘仁恩兵亦至横桥,据水置营,相持信宿,因合战,居业又败。仁恩虏叔慎、正理、居业及其党与十馀人,秦王斩之于汉口。叔慎时年十八。

义阳王叔达,字子聪,高宗第十七子也。太建十四年,立为义阳王,寻拜仁武将军,置佐史。祯明元年,除丹阳尹。三年入关。隋大业中为内史,至绛郡通守。

巴山王叔雄,字子猛,高宗第十八子也。太建十四年,立为巴山王。祯明三年入关,卒于长安。

武昌王叔虞,字子安,高宗第十九子也。太建十四年,立为武昌王,寻为壮武将军,置佐史。祯明三年入关。隋大业中为高苑令。

湘东王叔平,字子康,高宗第二十子也。至德元年,立为湘东王。祯明三年入关。隋大业中为胡苏令。

临贺王叔敖,字子仁,高宗第二十一子也。至德元年,立为临贺王,寻为仁武将军,置佐史。祯明三年入关。隋大业初拜仪同三司。

阳山王叔宣,字子通,高宗第二十二子也。至德元年,立为阳山王。祯明三年入关。隋大业中为泾城令。

西阳王叔穆,字子和,高宗第二十三子也。至德元年,立为西阳王。祯明三年入关,卒于长安。

南安王叔俭,字子约,高宗第二十四子也。至德元年,立为南安王。祯明三年入关,卒于长安。

南郡王叔澄,字子泉,高宗第二十五子也。至德元年,立为南郡王。祯明三年入关。隋大业中为灵武令。

沅陵王叔兴,字子推,高宗第二十六子也。至德元年,立为沅陵王。祯明三年入关。隋大业中为给事郎。

岳山王叔韶,字子钦,高宗第二十七子也。至德元年,立为岳山王,寻为智武将军,置佐史。四年,除丹阳尹。祯明三年入关,卒于长安。

新兴王叔纯,字子共,高宗第二十八子也。至德元年,立为新兴王。祯明三年入关。隋大业中为河北令。

巴东王叔谟,字子轨,高宗第二十九子也。至德四年,立为巴东王。祯明三年入关。隋大业中为岍阳令。

临江王叔显,字子明,高宗第三十子也。至德四年,立为临江王。祯明三年入关。隋大业中为鹑觚令。

新会王叔坦,字子开,高宗第三十一子也。至德四年,立为新会王。祯明三年入关。隋大业中为涉令。

新宁王叔隆,字子远,高宗第三十二子也。至德四年,立为新宁王。祯明三年入关。卒于长安。

新昌王叔荣,字子彻,高宗第三十三子也。祯明二年,立为新昌王。三年入关。隋大业中为内黄令。

太原王叔匡,字子佐,高宗第三十四子也。祯明二年,立为太原王。三年入关。隋大业中为寿光令。

后主二十二男:张贵妃生皇太子深、会稽王庄,孙姬生吴兴王胤,高昭仪生南平王嶷,吕淑媛生永嘉王彦、邵陵王兢,龚贵嫔生南海王虔、钱塘王恬,张叔华生信义王祗,徐淑仪生东阳王恮,孔贵人生吴郡王蕃。其皇子总、观、明、纲、统、冲、洽、縚、绰、威、辩十一人,并未及封。

皇太子深,字承源,后主第四子也。少聪惠,有志操,容止俨然,虽左右近侍,未尝见其喜愠。以母张贵妃故,特为后主所爱。至德元年,封始安王,邑二千户。寻为军师将军、扬州刺史,置佐史。祯明二年,皇太子胤废,后主乃立深为皇太子。三年,隋师济江,六军败绩,隋将韩擒虎自南掖门入,百僚逃散。深时年十馀岁,闭阁而坐,舍人孔伯鱼侍焉。隋军排阁而入,深使宣令劳之曰:``军旅在途,不乃劳也?''军人咸敬焉。其年入关。隋大业中为枹罕太守。

吴兴王胤,字承业,后主长子也。太建五年二月乙丑生于东宫,母孙姬因产卒,沈皇后哀而养之,以为己子。时后主年长,未有胤嗣,高宗因命以为嫡孙,其日下诏曰:``皇孙初诞,国祚方熙,思与群臣,共同斯庆,内外文武赐帛各有差,为父后者赐爵一级。''十年,封为永康公。后主即位,立为皇太子。胤性聪敏,好学,执经肄业,终日不倦,博通大义,兼善属文。至德三年,躬出太学讲《孝经》,讲毕,又释奠于先圣先师。其日设金石之乐于太学,王公卿士及太学生并预宴。是时张贵妃、孔贵嫔并爱幸,沈皇后无宠,而近侍左右数于东宫往来,太子亦数使人至后所,后主疑其怨望,甚恶之。而张、孔二贵妃又日夜构成后及太子之短,孔范之徒又于外合成其事,祯明二年,废为吴兴王,仍加侍中、中卫将军。三年入关,卒于长安。

南平王嶷,字承岳,后主第二子也。方正有器局,年数岁,风采举动,有若成人。至德元年,立为南平王。寻除信武将军、南琅邪、彭城二郡太守,置佐史。迁扬州刺史,进号镇南将军。寻为使持节、都督郢、荆、湘三州诸军事、征西将军、郢州刺史。未行而隋军济江。祯明三年入关,卒于长安。

永嘉王彦,字承懿,后主第三子也。至德元年,立为永嘉王。寻为忠武将军、南徐州刺史,进号安南将军。授散骑常侍、使持节、都督江、巴、东衡三州诸军事、平南将军、江州刺史。未行,隋师济江。祯明三年入关。隋大业中为襄武令。

南海王虔,字承恪,后主第五子也。至德元年,立为南海王。寻为武毅将军,置佐史,进号军师将军。祯明二年,出为平北将军、南徐州刺史。三年入关。隋大业中为涿令。

信义王祗,字承敬,后主第六子也。至德元年,立为信义王。寻为壮武将军,置佐史。授使持节、都督、智武将军、琅邪、彭城二郡太守。祯明三年入关。隋大业中为通议郎。

邵陵王兢,字承检,后主第七子也。祯明元年,立为邵陵王,邑一千户。寻为仁武将军,置佐史。三年入关。隋大业中为国子监丞。

会稽王庄,字承肃,后主第八子也。容貌蕞陋,性严酷,数岁,左右有不如意,辄剟刺其面,或加烧爇。以母张贵妃有宠,后主甚爱之。至德四年,立为会稽王。寻为翊前将军,置佐史。除使持节、都督扬州诸军事、扬州刺史。祯明三年入关。隋大业中为昌隆令。

东阳王恮,字承厚,后主第九子也。祯明二年,立为东阳王,邑一千户。未拜,三年入关。隋大业中为通议郎。

吴郡王蕃,字承广,后主第十子也。祯明二年,封吴郡王。三年入关。隋大业中为涪城令。

钱塘王恬,字承惔,后主第十一子也。祯明二年,立为钱塘王,邑一千户。三年入关,卒于长安。

江左自西晋相承,诸王开国,并以户数相差为大小三品。大国置上、中、下三将军,又置司马一人;次国置中、下二将军;小国置将军一人。馀官亦准此为差。高祖受命,自永定讫于祯明,唯衡阳王昌特加殊宠,至五千户。自馀大国不过二千户,小国即千户。而旧史残缺,不能别知其国户数,故缀其遗事附于此。

史臣曰:世祖、高宗、后主并建籓屏,以树懿亲,固乃本根,隆斯盘石。鄱阳王伯山有风采德器,亦一代令籓矣。岳阳王叔慎属社稷倾危,情哀家国,竭诚赴敌,志不图生。呜呼!古之忠烈致命,斯之谓也。

\hypertarget{header-n4919}{%
\subsubsection{卷二十三}\label{header-n4919}}

宗元饶司马申毛喜蔡徵

宗元饶,南郡江陵人也。少好学,以孝敬闻。仕梁世,解褐本州主簿,迁征南府行参军,仍转外兵参军。及司徒王僧辩幕府初建,元饶与沛国刘师知同为主簿。高祖受禅,除晋陵令。入为尚书功论郎。使齐还,为廷尉正。迁太仆卿,领本邑大中正,中书通事舍人。寻转廷尉卿,加通直散骑常侍,兼尚书左丞。时高宗初即位,军国务广,事无巨细,一以咨之,台省号为称职。

迁御史中丞,知五礼事。时合州刺史陈裦赃污狼藉,遣使就渚敛鱼,又于六郡乞米,百姓甚苦之。元饶劾奏曰:``臣闻建旟求瘼,实寄廉平,褰帷恤隐,本资仁恕。如或贪污是肆,征赋无厌,天网虽疏,兹焉弗漏。谨案钟陵县开国侯、合州刺史臣褵,因藉多幸,预逢抽擢,爵由恩被,官以私加,无德无功,坐尸荣贵。谯、肥之地,久沦非所,皇威克复,物仰仁风。新邦用轻,弥俟宽惠,应斯作牧,其寄尤重。爰降曲恩,祖行宣室,亲承规诲,事等言提。虽廉洁之怀,诚无素蓄,而禀兹严训,可以厉精。遂乃擅行赋敛,专肆贪取,求粟不厌,愧王沉之出赈,征鱼无限,异羊续之悬枯,置以严科,实惟明宪。臣等参议,请依旨免褵所应复除官,其应禁锢及后选左降本资,悉依免官之法。''遂可其奏。吴兴太守武陵王伯礼,豫章内史南康嗣王方泰,并骄蹇放横,元饶案奏之,皆见削黜。

元饶性公平,善持法,谙晓故事,明练治体,吏有犯法、政不便民及于名教不足者,随事纠正,多所裨益。迁贞威将军、南康内史,以秩米三千馀斛助民租课,存问高年,拯救乏绝,百姓甚赖焉。以课最入朝,诏加散骑常侍、荆、雍、湘、巴、武五州大中正。寻以本官重领尚书左丞。又为御史中丞。历左民尚书、右卫将军、领前将军,迁吏部尚书。太建十三年卒,时年六十四。诏赠侍中、金紫光禄大夫,官给丧事。

司马申,字季和,河内温人也。祖慧远,梁都水使者。父玄通,梁尚书左民郎。申早有风概,十四便善弈棋,尝随父候吏尚书到溉,时梁州刺史阴子春、领军硃异在焉。子春素知申,即于坐所呼与为对,申每有妙思,异观而奇之,因引申游处。梁邵陵王为丹阳尹,以申为主簿。属太清之难,父母俱没,因此自誓,菜食终身。

梁元帝承制,起为开远将军,迁镇西外兵记室参军。及侯景寇郢州,申随都督王僧辩据巴陵,每进筹策,皆见行用。僧辩叹曰:``此生要鞬汗马,或非所长,若使抚众守城,必有奇绩。''僧辩之讨陆纳也,申在军中,于时贼众奄至,左右披靡,申躬蔽僧辩,蒙楯而前,会裴之横救至,贼乃退,僧辩顾申而笑曰:``仁者必有勇,岂虚言哉!''除散骑侍郎。绍泰初,迁仪同侯安都从事中郎。

高祖受禅,除安东临川王谘议参军。天嘉三年,迁征北谘议参军,兼廷尉监。五年,除镇东谘议参军,兼起部郎。出为戎昭将军、江乘令,甚有治绩。入为尚书金部郎。迁左民郎,以公事免。太建初,起为贞威将军、征南鄱阳谘议参军。九年,除秣陵令,在职以清能见纪,有白雀巢于县庭。秩满,顷之,预东宫宾客,寻兼东宫通事舍人。迁员外散骑常侍,舍人如故。及叔陵之肆逆也,事既不捷,出据东府,申驰召右卫萧摩诃帅兵先至,追斩之,因入城中,收其府库,后主深嘉之。以功除太子左卫率,封文招县伯,邑四百户,兼中书通事舍人。寻迁右卫将军,加通直散骑常侍。以疾还第,就加散骑常侍,右卫、舍人如故。

至德四年卒,后主嗟悼久之,下诏曰:``慎终追远,钦若旧则,阖棺定谥,抑乃前典。故散骑常侍、右卫将军、文招县开国伯申,忠肃在公,清正立己,治繁处约,投躯殉义。朕任寄情深,方康庶绩,奄然化往,伤恻于怀。可赠侍中、护军将军,进爵为侯,增邑为五百户,谥曰忠。给朝服一具,衣一袭,克日举哀,丧事所须,随由资给。''及葬,后主自制志铭,辞情伤切。卒章曰:``嗟乎!天不与善,歼我良臣。''其见幸如此。

申历事三帝,内掌机密,至于仓卒之间,军国大事,指麾断决,无有滞留。子琇嗣,官至太子舍人。

毛喜,字伯武,荥阳阳武人也。祖称,梁散骑侍郎。父栖忠,梁尚书比部侍郎、中权司马。喜少好学,善草隶。起家梁中卫西昌侯行参军,寻迁记室参军。高祖素知于喜,及镇京口,命喜与高宗俱往江陵,仍敕高宗曰:``汝至西朝,可谘禀毛喜。''喜与高宗同谒梁元帝,即以高宗为领直,喜为尚书功论侍郎。及江陵陷,喜及高宗俱迁关右。世祖即位,喜自周还,进和好之策,朝廷乃遣周弘正等通聘。及高宗反国,喜于郢州奉迎。又遣喜入关,以家属为请。周冢宰宇文护执喜手曰:``能结二国之好者,卿也。''仍迎柳皇后及后主还。天嘉三年至京师,高宗时为骠骑将军,仍以喜为府谘议参军,领中记室。府朝文翰,皆喜词也。

世祖尝谓高宗曰:``我诸子皆以`伯'为名,汝诸儿宜用`叔'为称。''高宗以访于喜,喜即条牒自古名贤杜叔英、虞叔卿等二十馀人以启世祖,世祖称善。

世祖崩,废帝冲昧,高宗录尚书辅政,仆射到仲举等知朝望有归,乃矫太后令遣高宗还东府,当时疑惧,无敢措言。喜即驰入,谓高宗曰:``陈有天下日浅,海内未夷,兼国祸并钟,万邦危惧。皇太后深惟社稷至计,令王入省,方当共康庶绩,比德伊、周。今日之言,必非太后之意。宗社之重,愿加三思。以喜之愚,须更闻奏,无使奸贼得肆其谋。''竟如其策。

右卫将军韩子高始与仲举通谋,其事未发,喜请高宗曰:``宜简选人马,配与子高,并赐铁炭,使修器甲。''高宗惊曰:``子高谋反,即欲收执,何为更如是邪?''喜答曰:``山陵始毕,边寇尚多,而子高受委前朝,名为杖顺,然甚轻狷,恐不时授首,脱其稽诛,或愆王度。宜推心安诱,使不自疑,图之一壮士之力耳。''高宗深然之,卒行其计。

高宗即位,除给事黄门侍郎,兼中书舍人,典军国机密。高宗将议北伐,敕喜撰军制,凡十三条,诏颁天下,文多不载。寻迁太子右卫率、右卫将军。以定策功,封东昌县侯,邑五百户。又以本官行江夏、武陵、桂阳三王府国事。太建三年,丁母忧去职,诏追赠喜母庾氏东昌国太夫人,赐布五百匹,钱三十万,官给丧事。又遣员外散骑常侍杜缅图其墓田,高宗亲与缅案图指画,其见重如此。寻起为明威将军,右卫、舍人如故。改授宣远将军、义兴太守。寻以本号入为御史中丞。服阕,加散骑常侍、五兵尚书,参掌选事。

及众军北伐,得淮南地,喜陈安边之术,高宗纳之,即日施行。又问喜曰:``我欲进兵彭、汴,于卿意如何?''喜对曰:``臣实才非智者,安敢预兆未然。窃以淮左新平,边氓未乂,周氏始吞齐国,难与争锋,岂以弊卒疲兵,复加深入。且弃舟楫之工,践车骑之地,去长就短,非吴人所便。臣愚以为不若安民保境,寝兵复约,然后广募英奇,顺时而动,斯久长之术也。''高宗不从。后吴明彻陷周,高宗谓喜曰:``卿之所言,验于今矣。''

十二年,加侍中。十三年,授散骑常侍、丹阳尹。迁吏部尚书,常侍如故。及高宗崩,叔陵构逆,敕中庶子陆琼宣旨,令南北诸军,皆取喜处分。贼平,又加侍中,增封并前九百户。至德元年,授信威将军、永嘉内史,加秩中二千石。

初,高宗委政于喜,喜亦勤心纳忠,多所匡益,数有谏诤,事并见从,由是十馀年间,江东狭小,遂称全盛。唯略地淮北,不纳喜谋,而吴明彻竟败,高宗深悔之,谓袁宪曰:``不用毛喜计,遂令至此,朕之过也。''喜既益亲,乃言无回避,而皇太子好酒德,每共幸人为长夜之宴,喜尝为言,高宗以诫太子,太子阴患之,至是稍见疏远。

初,后主为始兴王所伤,及疮愈而自庆,置酒于后殿,引江总以下,展乐赋诗,醉而命喜。于时山陵初毕,未及逾年,喜见之不怿,欲谏而后主已醉,喜升阶,佯为心疾,仆于阶下,移出省中。后主醒,乃疑之,谓江总曰:``我悔召毛喜,知其无疾,但欲阻我欢宴,非我所为,故奸诈耳。''乃与司马申谋曰:``此人负气,吾欲将乞鄱阳兄弟听其报仇,可乎?''对曰:``终不为官用,愿如圣旨。''傅縡争之曰:``不然。若许报仇,欲置先皇何地?''后主曰:``当乞一小郡,勿令见人事也。''乃以喜为永嘉内史。

喜至郡,不受俸秩,政弘清静,民吏便之。遇豊州刺史章大宝举兵反,郡与豊州相接,而素无备御,喜乃修治城隍,严饰器械。又遣所部松阳令周磻领千兵援建安。贼平,授南安内史。祯明元年,征为光禄大夫,领左骁骑将军。喜在郡有惠政,乃征入朝,道路追送者数百里。其年道病卒,时年七十二。有集十卷。子处冲嗣,官至仪同从事中郎、中书侍郎。

蔡徵,字希祥,侍中、中抚军将军景历子也。幼聪敏,精识强记。年六岁,诣梁吏部尚书河南褚翔,翔字仲举,嗟其颖悟。七岁,丁母忧,居丧如成人礼。继母刘氏性悍忌,视之不以道,徵供侍益谨,初无怨色。徵本名览,景历以为有王祥之性,更名徵,字希祥。

梁承圣初,高祖为南徐州刺史,召补迎主簿。寻授太学博士。天嘉初,迁始兴王府法曹行参军,历外兵参军事、尚书主客郎,所居以干理称。太建初,迁太子少傅丞、新安王主簿、通直散骑侍郎、晋安王功曹史、太子中舍人,兼东宫领直,中舍人如故。丁父忧去职,服阕,袭封新豊县侯,授戎昭将军、镇右新安王谘议参军。

至德二年,迁廷尉卿,寻为吏部郎。迁太子中庶子、中书舍人,掌诏诰。寻授左民尚书,与仆射江总知撰五礼事。寻加宁远将军。后主器其材干,任寄日重,迁吏部尚书、安右将军,每十日一往东宫,于太子前论述古今得丧及当时政务。又敕以廷尉寺狱,事无大小,取徵议决。俄有敕遣徵收募兵士,自为部曲,徵善抚恤,得物情,旬月之间,众近一万。徵位望既重,兼声势熏灼,物议咸忌惮之。寻徙为中书令,将军如故。中令清简无事,或云徵有怨言,事闻后主,后主大怒,收夺人马,将诛之,有固谏者获免。

祯明三年,隋军济江,后主以徵有干用,权知中领军。徵日夜勤苦,备尽心力,后主嘉焉,谓曰``事宁有以相报''。及决战于钟山南岗,敕徵守宫城西北大营,寻令督众军战事。城陷,随例入关。

徵美容仪,有口辩,多所详究。至于士流官宦,皇家戚属,及当朝制度,宪章仪轨,户口风俗,山川土地,问无不对。然性颇便佞进取,不能以退素自业。初拜吏部尚书,启后主借鼓吹,后主谓所司曰:``鼓吹军乐,有功乃授,蔡徵不自量揆,紊我朝章。然其父景历既有缔构之功,宜且如所启,拜讫即追还。''徵不修廉隅,皆此类也。隋文帝闻其敏赡,召见顾问,言辄会旨,然累年不调,久之,除太常丞。历尚书民部仪曹郎,转给事郎,卒,时年六十七。子翼,治《尚书》,官至司徒属、德教学士。入隋,为东宫学士。

史臣曰:宗元饶夙夜匪懈,济务益时。司马申清恪在朝,攻苦立行,加之以忠节,美矣。毛喜深达事机,匡赞时主。蔡徵聪敏才赡,而擅权自踬,惜哉!

\hypertarget{header-n4947}{%
\subsubsection{卷二十四}\label{header-n4947}}

萧济陆琼子从典顾野王傅縡章华

萧济,字孝康,东海兰陵人也。少好学,博通经史,谘梁武帝《左氏》疑义三十馀条,尚书仆射范阳张缵、太常卿南阳刘之遴并与济讨论,缵等莫能抗对。解褐梁秘书郎,迁太子舍人。预平侯景之功,封松阳县侯,邑五百户。

及高祖作镇徐方,以济为明威将军、征北长史。承圣二年,征为中书侍郎,转通直散骑常侍。世祖为会稽太守,又以济为宣毅府长史,迁司徒左长史。世祖即位,授侍中。寻迁太府卿,丁所生母忧,不拜。济毘佐二主,恩遇甚笃,赏赐加于凡等。历守兰陵、阳羡、临津、临安等郡,所在皆著声绩。太建初,入为五兵尚书,与左仆射徐陵、特进周弘正、度支尚书王瑒、散骑常侍袁宪俱侍东宫。复为司徒长史。寻授度支尚书,领羽林监。迁国子祭酒,领羽林如故。加金紫光禄大夫,兼安德宫卫尉。寻迁仁威将军、扬州长史。高宗尝敕取扬州曹事,躬自省览,见济条理详悉,文无滞害,乃顾谓左右曰:``我本期萧长史长于经传,不言精练繁剧,乃至于此。''迁祠部尚书,加给事中,复为金紫光禄大夫。未拜而卒,时年六十六。诏赠本官,官给丧事。

陆琼,字伯玉,吴郡吴人也。祖完,梁琅邪、彭城二郡丞。父云公,梁给事黄门侍郎,掌著作。琼幼聪惠有思理,六岁为五言诗,颇有词采。大同末,云公受梁武帝诏校定《棋品》,到溉、硃异以下并集。琼时年八岁,于客前覆局,由是京师号曰神童。异言之武帝,有敕召见,琼风神警亮,进退详审,帝甚异之。十一,丁父忧,毁瘠有至性,从祖襄叹曰:``此儿必荷门基,所谓一不为少。''及侯景作逆,携母避地于县之西乡,勤苦读书,昼夜无怠,遂博学,善属文。

永定中,州举秀才。天嘉元年,为宁远始兴王府法曹行参军。寻以本官兼尚书外兵郎,以文学转兼殿中郎,满岁为真。琼素有令名,深为世祖所赏。及讨周迪、陈宝应等,都官符及诸大手笔,并中敕付琼。迁新安王文学,掌东宫管记。及高宗为司徒,妙简僚佐,吏部尚书徐陵荐琼于高宗曰:``新安王文学陆琼,见识优敏,文史足用,进居郎署,岁月过淹,左西掾缺,允膺兹选,阶次小逾,其屈滞已积。''乃除司徒左西掾。寻兼通直散骑常侍,聘齐。

太建元年,重以本官掌东宫管记。除太子庶子,兼通事舍人。转中书侍郎、太子家令。长沙王为江州刺史,不循法度,高宗以王年少,授琼长史,行江州府国事,带寻阳太守。琼以母老,不欲远出,太子亦固请留之,遂不行。累迁给事黄门侍郎,领羽林监。转太子中庶子,领步兵校尉。又领大著作,撰国史。

后主即位。直中书省,掌诏诰。俄授散骑常侍,兼度支尚书,领扬州大中正。至德元年,除度支尚书,参掌诏诰,并判廷尉、建康二狱事。初,琼父云公奉梁武帝敕撰《嘉瑞记》,琼述其旨而续焉,自永定讫于至德,勒成一家之言。迁吏部尚书,著作如故。琼详练谱谍,雅鉴人伦,先是,吏部尚书宗元饶卒,右仆射袁宪举琼,高宗未之用也,至是居之,号为称职,后主甚委任焉。

琼性谦俭,不自封植,虽位望日隆,而执志愈下。园池室宇,无所改作,车马衣服,不尚鲜华,四时禄俸,皆散之宗族,家无馀财。暮年深怀止足,思避权要,恒谢病不视事。俄丁母忧,去职。初,琼之侍东宫也,母随在官舍,后主赏赐优厚。及丧柩还乡,诏加赙赠,并遣谒者黄长贵持册奠祭,后主又自制志铭,朝野荣之。琼哀慕过毁,以至德四年卒,时年五十,诏赠领军将军,官给丧事。有集二十卷行于世。长子从宜,仕至武昌王文学。

第三子从典,字由仪。幼而聪敏。八岁,读沈约集,见回文研铭,从典援笔拟之,便有佳致。年十三,作《柳赋》,其词其美。琼时为东宫管记,宫僚并一时俊伟,琼示以此赋,咸奇其异才。从父瑜特所赏爱,及瑜将终,家中坟籍皆付从典,从典乃集瑜文为十卷,仍制集序,其文甚工。

从典笃好学业,博涉群书,于《班史》尤所属意。年十五,本州举秀才。解褐著作佐郎,转太子舍人。时后主赐仆射江总并其父琼诗,总命从典为谢启,俄顷便就,文华理畅,总甚异焉。寻授信义王文学,转太子洗马。又迁司徒左西掾,兼东宫学士。丁父忧去职。寻起为德教学士,固辞不就,后主敕留一员,以待从典。俄属金陵沦没,随例迁关右。仕隋为给事郎,兼东宫学士。又除著作佐郎。右仆射杨素奏从典续司马迁《史记》迄于隋,其书未就。值隋末丧乱,寓居南阳郡,以疾卒,时年五十七。

顾野王,字希冯,吴郡吴人也。祖子乔,梁东中郎武陵王府参军事。父亘,信威临贺王记室,兼本郡五官掾,以儒术知名。野王幼好学。七岁,读《五经》,略知大旨。九岁能属文,尝制《日赋》,领军硃异见而奇之。年十二,随父之建安,撰《建安地记》二篇。长而遍观经史,精记嘿识,天文地理、蓍龟占候、虫篆奇字,无所不通。梁大同四年,除太学博士。迁中领军临贺王府记室参军。宣城王为扬州刺史,野王及琅邪王褒并为宾客,王甚爱其才。野王又好丹青,善图写,王于东府起斋,乃令野王画古贤,命王褒书赞,时人称为二绝。

及侯景之乱,野王丁父忧,归本郡,乃召募乡党数百人,随义军援京邑。野王体素清羸,裁长六尺,又居丧过毁,殆不胜衣,及杖戈被甲,陈君臣之义,逆顺之理,抗辞作色,见者莫不壮之。京城陷,野王逃会稽,寻往东阳,与刘归义合军据城拒贼。侯景平,太尉王僧辩深嘉之,使监海盐县。

高祖作宰,为金威将军、安东临川王府记室参军,寻转府谘议参军。天嘉元年,敕补撰史学士,寻加招远将军。光大元年,除镇东鄱阳王谘议参军。太建二年,迁国子博士。后主在东宫,野王兼东宫管记,本官如故。六年,除太子率更令,寻领大著作,掌国史,知梁史事,兼东宫通事舍人。时宫僚有济阳江总,吴国陆琼,北地傅縡,吴兴姚察,并以才学显著,论者推重焉。迁黄门侍郎,光禄卿,知五礼事,馀官并如故。十三年卒,时年六十三。诏赠秘书监。至德二年,又赠右卫将军。

野王少以笃学至性知名,在物无过辞失色,观其容貌,似不能言,及其励精力行,皆人所莫及。第三弟充国早卒,野王抚养孤幼,恩义甚厚。其所撰著《玉篇》三十卷,《舆地志》三十卷,《符瑞图》十卷,《顾氏谱传》十卷,《分野枢要》一卷,《续洞冥纪》一卷,《玄象表》一卷,并行于世。又撰《通史要略》一百卷,《国史纪传》二百卷,未就而卒。有文集二十卷。

傅縡,字宜事,北地灵州人也。父彝,梁临沂令。縡幼聪敏,七岁诵古诗赋至十馀万言。长好学,能属文。梁太清末,携母南奔避难,俄丁母忧,在兵乱之中,居丧尽礼,哀毁骨立,士友以此称之。后依湘州刺史萧循,循颇好士,广集坟籍,縡肆志寻阅,因博通群书。王琳闻其名,引为府记室。琳败,随琳将孙瑒还都。时世祖使颜晃赐瑒杂物,瑒托縡启谢,词理优洽,文无加点,晃还言之世祖,寻召为撰史学士。除司空府记室参军,迁骠骑安成王中记室,撰史如故。

縡笃信佛教,从兴皇惠朗法师受《三论》,尽通其学。时有大心暠法师著《无诤论》以诋之,縡乃为《明道论》,用释其难。其略曰:

《无诤论》言:比有弘《三论》者,雷同诃诋,恣言罪状,历毁诸师,非斥众学,论中道而执偏心,语忘怀而竞独胜,方学数论,更为仇敌,仇敌既构,诤斗大生,以此之心,而成罪业,罪业不止,岂不重增生死,大苦聚集?答曰:《三论》之兴,为日久矣。龙树创其源,除内学之偏见,提婆扬其旨,荡外道之邪执。欲使大化流而不拥,玄风阐而无坠。其言旷,其意远,其道博,其流深。斯固龙象之腾骧,鲲鹏之抟运。蹇乘决羽,岂能觖望其间哉?顷代浇薄,时无旷士,苟习小学,以化蒙心,渐染成俗,遂迷正路,唯竞穿凿,各肆营造,枝叶徒繁,本源日翳,一师解释,复异一师,更改旧宗,各立新意,同学之中,取寤复别,如是展转,添糅倍多。总而用之,心无的准;择而行之,何者为正?岂不浑沌伤窍,嘉树弊牙?虽复人说非马,家握灵蛇,以无当之卮,同画地之饼矣。其于失道,不亦宜乎?摄山之学,则不如是。守一遵本,无改作之过;约文申意,杜臆断之情。言无预说,理非宿构。睹缘尔乃应,见敌然后动。纵横络驿,忽恍杳冥。或弥纶而不穷。或消散而无所。焕乎有文章,踪朕不可得;深乎不可量,即事而非远。凡相酬对,随理详核。有何嫉诈,干犯诸师?且诸师所说,为是可毁?为不可毁?若可毁者,毁故为衰;若不可毁,毁自不及。法师何独蔽护不听毁乎?且教有大小,备在圣诰,大乘之文,则指斥小道。今弘大法,宁得不言大乘之意耶?斯则褒贬之事,从弘放学;与夺之辞,依经议论。何得见佛说而信顺,在我语而忤逆?无诤平等心如是耶?且忿恚烦恼,凡夫恒性,失理之徒,率皆有此。岂可以三修未惬,六师怀恨,而蕴涅槃妙法,永不宣扬?但冀其忿愤之心既极,恬淡之寤自成耳。人面不同,其心亦异,或有辞意相反,或有心口相符。岂得必谓他人说中道而心偏执,己行无诤,外不违而内平等?仇敌斗讼,岂我事焉;罪业聚集,斗诤者所畏耳。

《无诤论》言:摄山大师诱进化导,则不如此,即习行于无诤者也。导悟之德既往,淳一之风已浇,竞胜之心,阿毁之曲,盛于兹矣。吾愿息诤以通道,让胜以忘德。何必排拂异家,生其恚怒者乎?若以中道之心行于《成实》,亦能不诤;若以偏著之心说于《中论》,亦得有诤。固知诤与不诤,偏在一法。答曰:摄山大师实无诤矣,但法师所赏,未衷其节。彼静守幽谷,寂尔无为,凡有训勉,莫匪同志,从容语嘿,物无间然,故其意虽深,其言甚约。今之敷畅,地势不然。处王城之隅,居聚落之内,呼吸顾望之客,脣吻纵横之士,奋锋颖,励羽翼,明目张胆,被坚执锐,聘异家,衒别解,窥伺间隙,邀冀长短,与相酬对,捔其轻重,岂得默默无言,唯唯应命?必须掎摭同异,发擿玼瑕,忘身而弘道,忤俗而通教,以此为病,益知未达。若令大师当此之地,亦何必默己,而为法师所贵耶?法师又言:``吾愿息诤以通道,让胜以忘德。''道德之事,不止在诤与不诤,让与不让也。此语直是人间所重,法师慕而言之,竟未知胜若为可让也。若他人道高,则自胜不劳让矣;他人道劣,则虽让而无益矣。欲让之辞,将非虚设?中道之心,无处不可。《成实三论》,何事致乖?但须息守株之解,除胶柱之意,是事皆中也。来旨言``诤与不诤,偏在一法''。何为独褒无诤耶?讵非矛盾?

《无诤论》言:邪正得失,胜负是非,必生于心矣,非谓所说之法,而有定相论胜劣也。若异论是非,以偏著为失言,无是无非,消彼得失,以此论为胜妙者,他论所不及,此亦为失也。何者?凡心所破,岂无心于能破,则胜负之心不忘,宁不存胜者乎?斯则矜我为得,弃他之失,即有取舍,大生是非,便是增诤。答曰:言为心使,心受言诠;和合根尘,鼓动风气,故成语也。事必由心,实如来说。至于心造伪以使口,口行诈以应心,外和而内险,言随而意逆,求利养,引声名,入道之人,在家之士,斯辈非一。圣人所以曲陈教诫,深致防杜,说见在之殃咎,叙将来之患害,此文明著,甚于日月,犹有忘爱躯,冒峻制,蹈汤炭,甘齑粉,必行而不顾也。岂能悦无诤之作,而回首革音耶?若弘道之人,宣化之士,心知胜也,口言胜也,心知劣也,口言劣也,亦无所苞藏,亦无所忌禅,但直心而行之耳。他道虽劣,圣人之教也;己德虽优,亦圣人之教也。我胜则圣人胜,他劣则圣人劣。圣人之优劣,盖根缘所宜尔。于彼于此,何所厚薄哉?虽复终日按剑,极夜击柝,瞋目以争得失,作气以求胜负,在谁处乎?有心之与无心,徒欲分别虚空耳。何意不许我论说,而使我谦退?此谓鹪褷已翔于寥廓,而虞者犹窥薮泽而求之。嗟乎!丈夫当弘斯道矣。

《无诤论》言:无诤之道,通于内外。子所言须诤者,此用末而救本,失本而营末者也。今为子言之。何则?若依外典,寻书契之前,至淳之世,朴质其心,行不言之教,当于此时,民至老死不相往来,而各得其所,复有何诤乎?固知本末不诤,是物之真矣。答曰:诤与无诤,不可偏执。本之与末,又安可知?由来不诤,宁知非末?于今而诤,何验非本?夫居后而望前,则为前;居前而望后,则为后。而前后之事犹如彼此,彼呼此为彼,此呼彼为彼,彼此之名,的居谁处?以此言之,万事可知矣。本末前后,是非善恶,可恒守邪?何得自信聪明,废他耳目?夫水泡生灭,火轮旋转,入牢阱,受羁绁,生忧畏,起烦恼,其失何哉?不与道相应,而起诸见故也。相应者则不然,无为也,无不为也。善恶不能偕,而未曾离善恶,生死不能至,亦终然在生死,故得永离而任放焉。是以圣人念绕桎之不脱,愍黏胶之难离,故殷勤教示,备诸便巧。希向之徒,涉求有类,虽驎角难成,象形易失,宁得不仿佛遐路,勉励短晨?且当念己身之善恶,莫揣他物,而欲分别,而言我聪明,我知见,我计校,我思惟,以此而言,亦为疏矣。他人者实难测,或可是凡夫真尔,亦可是圣人俯同,时俗所宜见,果报所应睹。安得肆胸衿,尽情性,而生讥诮乎?正应虚己而游乎世,俯仰于电露之间耳。明月在天,众水咸见,清风至林,群籁毕响。吾岂逆物哉?不入鲍鱼,不甘腐鼠。吾岂同物哉?谁能知我,共行斯路,浩浩乎!堂堂乎!岂复见有诤为非,无诤为是?此则诤者自诤,无诤者自无诤,吾俱取而用之。宁劳法师费功夫,点笔纸,但申于无诤;弟子疲脣舌,消晷漏,唯对于明道?戏论哉!糟粕哉!必欲且考真伪,蹔观得失,无过依贤圣之言,检行藏之理,始终研究,表里综核,使浮辞无所用,诈道自然消。请待后筵,以观其妙矣。

寻以本官兼通直散骑侍郎使齐,还除散骑侍郎、镇南始兴王谘议参军,兼东宫管记。历太子庶子、仆,兼管记如故。后主即位,迁秘书监、右卫将军,兼中书通事舍人,掌诏诰。

縡为文典丽,性又敏速,虽军国大事,下笔辄成,未尝起草,沉思者亦无以加焉,甚为后主所重。然性木强,不持检操,负才使气,陵侮人物,朝士多衔之。会施文庆、沈客卿以便佞亲幸,专制衡轴,而縡益疏。文庆等因共谮縡受高丽使金,后主收縡下狱。縡素刚,因愤恚,乃于狱中上书曰:``夫君人者,恭事上帝,子爱下民,省嗜欲,远谄佞,未明求衣,日旰忘食,是以泽被区宇,庆流子孙。陛下顷来酒色过度,不虔郊庙之神,专媚淫昏之鬼;小人在侧,宦竖弄权,恶忠直若仇雠,视生民如草芥;后宫曳绮绣,厩马馀菽粟,百姓流离,僵尸蔽野;货贿公行,帑藏损耗,神怒民怨,众叛亲离。恐东南王气,自斯而尽。''书奏,后主大怒。顷之,意稍解,遣使谓縡曰:``我欲赦卿,卿能改过不?''縡对曰:``臣心如面,臣面可改,则臣心可改。''后主于是益怒,令宦者李善庆穷治其事,遂赐死狱中,时年五十五。有集十卷行于世。

时有吴兴章华,字仲宗,家世农夫,至华独好学,与士君子游处,颇览经史,善属文。侯景之乱,乃游岭南,居罗浮山寺,专精习业。欧阳頠为广州刺史,署为南海太守。及欧阳纥败,乃还京师。太建中,高宗使吏部侍郎萧引喻广州刺史马靖,令入子为质,引奏华与俱行。使还,而高宗崩。后主即位,朝臣以华素无伐阅,竞排诋之,乃除大市令,既雅非所好,乃辞以疾,郁郁不得志。祯明初,上书极谏,其大略曰:``昔高祖南平百越,北诛逆虏;世祖东定吴会,西破王琳;高宗克复淮南,辟地千里:三祖之功,亦至勤矣。陛下即位,于今五年,不思先帝之艰难,不知天命之可畏,溺于嬖宠,惑于酒色,祠七庙而不出,拜妃嫔而临轩,老臣宿将,弃之草莽,谄佞谗邪,升之朝廷。今疆埸日蹙,隋军压境,陛下如不改弦易张,臣见麋鹿复游于姑苏台矣。''书奏,后主大怒,即日命斩之。

史臣曰:萧济、陆琼,俱以才学显著,顾野王博极群典,傅縡聪警特达,并一代之英灵矣。然縡不能循道进退,遂置极网,悲夫!

\hypertarget{header-n4974}{%
\subsubsection{卷二十五}\label{header-n4974}}

萧摩诃子世廉任忠樊毅弟猛鲁广达

萧摩诃,字元胤,兰陵人也。祖靓,梁右将军。父谅,梁始兴郡丞。摩诃随父之郡,年数岁而父卒,其姑夫蔡路养时在南康,乃收养之。稍长,果毅有勇力。侯景之乱,高祖赴援京师,路养起兵拒高祖,摩诃时年十三,单骑出战,军中莫有当者。及路养败,摩诃归于侯安都,安都遇之甚厚,自此常隶安都征讨。及任约、徐嗣徽引齐兵为寇,高祖遣安都北拒齐军于钟山龙尾及北郊坛。安都谓摩诃曰:``卿骁勇有名,千闻不如一见。''摩诃对曰:``今日令公见矣。''及战,安都坠马被围,摩诃独骑大呼,直冲齐军,齐军披靡,因稍解去,安都乃免。天嘉初,除本县令,以平留异、欧阳纥之功,累迁巴山太守。

太建五年,众军北伐,摩诃随都督吴明彻济江攻秦郡。时齐遣大将尉破胡等率众十万来援,其前队有``苍头''、``犀角''、``大力''之号,皆身长八尺,膂力绝伦,其锋甚锐。又有西域胡,妙于弓矢,弦无虚发,众军尤惮之。及将战,明彻谓摩诃曰:``若殪此胡,则彼军夺气,君有关、张之名,可斩颜良矣。''摩诃曰:``愿示其形状,当为公取之。''明彻乃召降人有识胡者,云胡著绛衣,桦皮装弓,两端骨弭。明彻遣人觇伺,知胡在阵,乃自酌酒以饮摩诃。摩诃饮讫,驰马冲齐军,胡挺身出阵前十馀步,彀弓未发,摩诃遥掷铣鋧,正中其额,应手而仆。齐军``大力''十馀人出战,摩诃又斩之,于是齐军退走。以功授明毅将军、员外散骑常侍,封廉平县伯,邑五百户。寻进爵为侯,转太仆卿,馀如故。七年,又随明彻进围宿预,击走齐将王康德,以功除晋熙太守。九年,明彻进军吕梁,与齐人大战,摩诃率七骑先入,手夺齐军大旗,齐众大溃。以功授持节、武毅将军、谯州刺史。

及周武帝灭齐,遣其将宇文忻率众争吕梁,战于龙晦。时忻有精骑数千,摩诃领十二骑深入周军,纵横奋击,斩馘甚众。及周遣大将军王轨来赴,结长围连锁于吕梁下流,断大军还路。摩诃谓明彻曰:``闻王轨始锁下流,其两头筑城,今尚未立,公若见遣击之,彼必不敢相拒。水路未断,贼势不坚,彼城若立,则吾属且为虏矣。''明彻乃奋髯曰:``搴旗陷阵,将军事也;长算远略,老夫事也。''摩诃失色而退。一旬之间,周兵益至,摩诃又请于明彻曰:``今求战不得,进退无路,若潜军突围,未足为耻。愿公率步卒,乘马舆徐行,摩诃领铁骑数千,驱驰前后,必当使公安达京邑。''明彻曰:``弟之此计,乃良图也。然老夫受脤专征,不能战胜攻取,今被围逼蹙,惭置无地。且步军既多,吾为总督,必须身居其后,相率兼行。弟马军宜须在前,不可迟缓。''摩诃因率马军夜发。先是,周军长围既合,又于要路下伏数重,摩诃选精骑八十,率先冲突,自后众骑继焉,比旦达淮南。高宗诏征还,授右卫将军。十一年,周兵寇寿阳,摩诃与樊毅等众军赴援,无功而还。

十四年,高宗崩,始兴王叔陵于殿内手刃后主,伤而不死,叔陵奔东府城。时众心犹预,莫有讨贼者,东宫舍人司马申启后主,驰召摩诃,入见受敕,乃率马步数百,先趣东府城西门屯军。叔陵惶遽,自城南门而出,摩诃勒兵追斩之。以功授散骑常侍、车骑大将军,封绥建郡公,邑三千户,叔陵素所蓄聚金帛累巨万,后主悉以赐之。寻改授侍中、骠骑大将军,加左光禄大夫。旧制三公黄阁听事置鸱尾,后主特赐摩诃开黄阁,门施行马,听事寝堂并置鸱尾。仍以其女为皇太子妃。

会隋总管贺若弼镇广陵,窥觎江左,后主委摩诃备御之任,授南徐州刺史,馀并如故。祯明三年正月元会,征摩诃还朝,贺若弼乘虚济江,袭京口,摩诃请兵逆战,后主不许。及弼进军钟山,摩诃又请曰:``贺若弼悬军深入,声援犹远,且其垒堑未坚,人情惶惧,出兵掩袭,必大克之。''后主又不许。及隋军大至,将出战,后主谓摩诃曰:``公可为我一决。''摩诃曰:``从来行阵,为国为身,今日之事,兼为妻子。''后主多出金帛,颁赏诸军,令中领军鲁广达陈兵白土岗,居众军之南偏,镇东大将军任忠次之,护军将军樊毅、都官尚书孔范次之,摩诃军最居北,众军南北亘二十里,首尾进退,各不相知。贺若弼初谓未战,将轻骑,登山观望形势,及见众军,因驰下置阵。广达首率所部进薄,弼军屡却,俄而复振,更分军趣北突诸将,孔范出战,兵交而走,诸将支离,阵犹未合,骑卒溃散,驻之弗止,摩诃无所用力焉,为隋军所执。

及京城陷,贺若弼置后主于德教殿,令兵卫守,摩诃请弼曰:``今为囚虏,命在斯须,愿得一见旧主,死无所恨。''弼哀而许之。摩诃入见后主,俯伏号泣,仍于旧厨取食而进之,辞诀而出,守卫者皆不能仰视。其年入隋,授开府仪同三司。寻从汉王谅诣并州,同谅作逆,伏诛,时年七十三。

摩诃讷于语言,恂恂长者,至于临戎对寇,志气奋发,所向无前。年未弱冠,随侯安都在京口,性好射猎,无日不畋游。及安都东征西伐,战胜攻取,摩诃功实居多。

子世廉,少警俊,敢勇有父风。性至孝,及摩诃凶终,服阕后,追慕弥切。其父时宾故脱有所言及,世廉对之,哀恸不自胜,言者为之歔欷。终身不执刀斧,时人嘉焉。

摩诃有骑士陈智深者,勇力过人,以平叔陵之功,为巴陵内史。摩诃之戮也,其妻子先已籍没,智深收摩诃尸,手自殡敛,哀感行路,君子义之。

颍川陈禹,亦随摩诃征讨,聪敏有识量,涉猎经史,解风角、兵书,颇能属文,便骑射,官至王府谘议。

任忠,字奉诚,小名蛮奴,汝阴人也。少孤微,不为乡党所齿。及长,谲诡多计略,膂力过人,尤善骑射,州里少年皆附之。梁鄱阳王萧范为合州刺史,闻其名,引置左右。侯景之乱,忠率乡党数百人,随晋熙太守梅伯龙讨景将王贵显于寿春,每战却敌。会土人胡通聚众寇抄,范命忠与主帅梅思立并军讨平之。仍随范世子嗣率众入援,会京城陷,旋戍晋熙。侯景平,授荡寇将军。

王琳立萧庄,署忠为巴陵太守。琳败还朝,迁明毅将军、安湘太守,仍随侯瑱真进讨巴、湘。累迁豫宁太守、衡阳内史。华皎之举兵也,忠预其谋。及皎平,高宗以忠先有密启于朝廷,释而不问。太建初,随章昭达讨欧阳纥于广州,以功授直阁将军。迁武毅将军、庐陵内史,秩满,入为右军将军。

五年,众军北伐,忠将兵出西道,击走齐历阳王高景安于大岘,逐北至东关,仍克其东西二城。进军蕲、谯,并拔之。径袭合肥,入其郛。进克霍州。以功授员外散骑常侍,封安复县侯,邑五百户。吕梁之丧师也,忠全军而还。寻诏忠都督寿阳、新蔡、霍州缘淮众军,进号宁远将军、霍州刺史。入为左卫将军。十一年,加北讨前军事,进号平北将军,率众步骑趣秦郡。十二年,迁使持节、散骑常侍、都督南豫州诸军事、平南将军、南豫州刺史,增邑并前一千五百户。仍率步骑趣历阳。周遣王延贵率众为援,忠大破之,生擒延贵。后主嗣位,进号镇南将军,给鼓吹一部。入为领军将军,加侍中,改封梁信郡公,邑三千户。出为吴兴内史,加秩中二千石。

及隋兵济江,忠自吴兴入赴,屯军硃雀门。后主召萧摩诃以下于内殿定议,忠执议曰:``兵家称客主异势,客贵速战,主贵持重。宜且益兵坚守宫城,遣水军分向南豫州及京口道,断寇粮运。待春水长,上江周罗珣等众军,必沿流赴援,此良计矣。''众议不同,因遂出战。及败,忠驰入台见后主,言败状,启云:``陛下唯当具舟楫,就上流众军,臣以死奉卫。''后主信之,敕忠出部分,忠辞云:``臣处分讫,即当奉迎。''后主令宫人装束以待忠,久望不至。隋将韩擒虎自新林进军,忠乃率数骑往石子岗降之,仍引擒虎军共入南掖门。台城陷,其年入长安,隋授开府仪同三司。卒,时年七十七。子幼武,官至仪同三司。

时有沈客卿者,吴兴武康人,性便佞忍酷,为中书舍人,每立异端,唯以刻削百姓为事,由是自进。有施文庆者,吴兴乌程人,起自微贱,有吏用,后主拔为主书,迁中书舍人,俄擢为湘州刺史。未及之官,会隋军来伐,四方州镇,相继以闻。文庆、客卿俱掌机密,外有表启,皆由其呈奏。文庆心悦湘州重镇,冀欲早行,遂与客卿共为表里,抑而不言,后主弗之知也,遂以无备,至乎败国,实二人之罪。隋军既入,并戮之于前阙。

樊毅,字智烈,南阳湖阳人也。祖方兴,梁散骑常侍、仁威将军、司州刺史,鱼复县侯。父文炽,梁散骑常侍、信武将军、益州刺史,新蔡县侯。毅累叶将门,少习武善射。侯景之乱,毅率部曲随叔父文皎援台。文皎于青溪战殁,毅将宗族子弟赴江陵,仍隶王僧辩,讨河东王萧誉,以功除假节、威戎将军、右中郎将。代兄俊为梁兴太守,领三州游军,随宜豊侯萧循讨陆纳于湘州。军次巴陵,营顿未立,纳潜军夜至,薄营大噪,营中将士皆惊扰,毅独与左右数十人,当营门力战,斩十馀级,击鼓申命,众乃定焉。以功授持节、通直散骑常侍、贞威将军,封夷道县伯,食邑三百户。寻除天门太守,进爵为侯,增邑并前一千户。及西魏围江陵,毅率兵赴援,会江陵陷,为岳阳王所执,久之遁归。

高祖受禅,毅与弟猛举兵应王琳,琳败奔齐,太尉侯瑱遣使招毅,毅率子弟部曲还朝。天嘉二年,授通直散骑常侍,仍随侯瑱进讨巴、湘。累迁武州刺史。太建初,转豊州刺史,封高昌县侯,邑一千户。入为左卫将军。五年,众军北伐,毅率众攻广陵楚子城,拔之,击走齐军于颍口,齐援沧陵,又破之。七年,进克潼州、下邳、高栅等六城。及吕梁丧师,诏以毅为大都督,进号平北将军,率众渡淮,对清口筑城,与周人相抗,霖雨城坏,毅全军自拔。寻迁中领军。十一年,周将梁士彦将兵围寿阳,诏以毅为都督北讨前军事,率水军入焦湖。寻授镇西将军、都督荆、郢、巴、武四州水陆诸军事。十二年,进督沔、汉诸军事,以公事免。十三年,征授中护军。寻迁护军将军、荆州刺史。

后主即位,进号征西将军,改封逍遥郡公,邑三千户,馀并如故。入为侍中、护军将军。及隋兵济江,毅谓仆射袁宪曰:``京口、采石,俱是要所,各须锐卒数千,金翅二百,都下江中,上下防捍。如其不然,大事去矣。''诸将咸从其议。会施文庆等寝隋兵消息,毅计不行。京城陷,随例入关,顷之卒。

猛字智武,毅之弟也。幼倜傥,有干略。既壮,便弓马,胆气过人。青溪之战,猛自旦讫暮,与虏短兵接,杀伤甚众。台城陷,随兄毅西上京,累战功为威戎将军。梁南安侯萧方矩为湘州刺史,以猛为司马。会武陵王萧纪举兵自汉江东下,方矩遣猛率湘、郢之卒,随都督陆法和进军以拒之。时纪已下,楼船战舰据巴江,争峡口,相持久之,不能决。法和揣纪师老卒堕,因令猛率骁勇三千,轻舸百馀乘,冲流直上,出其不意,鼓噪薄之。纪众仓卒惊骇,不及整列,皆弃舰登岸,赴水死者以千数。时纪心膂数百人,犹在左右,猛将部曲三十馀人,蒙楯横戈,直登纪舟,瞋目大呼,纪侍卫皆披靡,相枕藉不敢动。猛手擒纪父子三人,斩于絺中,尽收其船舰器械。以功授游骑将军,封安山县伯,邑一千户。仍进军抚定梁、益,蜀境悉平。军还,迁持节、散骑常侍、轻车将军、司州刺史,进爵为侯,增邑并前二千户。

永定元年,周文育等败于沌口,为王琳所获。琳乘胜将略南中诸郡,遣猛与李孝钦等将兵攻豫章,进逼周迪,军败,为迪斩执。寻遁归王琳。王琳败,还朝。天嘉二年,授通直散骑常侍、永阳太守。迁安成王府司马。光大元年,授壮武将军、庐陵内史。太建初,迁武毅将军、始兴平南府长史,领长沙内史。寻隶章昭达西讨江陵,潜军入峡,焚周军船舰,以功封富川县侯,邑五百户。历散骑常侍,迁使持节、都督荆信二州诸军事、宣远将军、荆州刺史。入为左卫将军。

后主即位,增邑并前一千户,馀并如故。至德四年,授使持节、都督南豫州诸军事、忠武将军、南豫州刺史。隋将韩擒虎之济江也,猛在京师,第六子巡摄行州事,擒虎进军攻陷之,巡及家口并见执。时猛与左卫将军蒋元逊领青龙八十艘为水军,于白下游弈,以御隋六合兵,后主知猛妻子在隋军,惧其有异志,欲使任忠代之,又重伤其意,乃止。祯明三年入于隋。

鲁广达,字遍览,吴州刺史悉达之弟也。少慷慨,志立功名,虚心爱士,宾客或自远而至。时江表将帅,各领部曲,动以千数,而鲁氏尤多。释褐梁邵陵王国右常侍,迁平南当阳公府中兵参军。侯景之乱,与兄悉达聚众保新蔡。梁元帝承制,授假节、壮武将军、晋州刺史。王僧辩之讨侯景也,广达出境候接,资奉军储,僧辩谓沈炯曰:``鲁晋州亦是王师东道主人。''仍率众随僧辩。景平,加员外散骑常侍,馀如故。

高祖受禅,授征远将军、东海太守。寻徙为桂阳太守,固辞不拜,入为员外散骑常侍。除假节、信武将军、北新蔡太守。随吴明彻讨周迪于临川,每战功居最。仍代兄悉达为吴州刺史,封中宿县侯,邑五百户。

光大元年,授通直散骑常侍、都督南豫州诸军事、南豫州刺史。华皎称兵上流,诏司空淳于量率众军进讨。军至夏口,皎舟师强盛,莫敢进者,广达首率骁勇,直冲贼军。战舰既交,广达愤怒大呼,登舰楼,奖励士卒,风急舰转,楼摇动,广达足跌堕水,沈溺久之,因救获免。皎平,授持节、智武将军、都督巴州诸军事、巴州刺史。

太建初,与仪同章昭达入峡口,拓定安蜀等诸州镇。时周氏将图江左,大造舟舰于蜀,并运粮青泥,广达与钱道戢等将兵掩袭,纵火焚之。以功增封并前二千户,仍还本镇。广达为政简要,推诚任下,吏民便之。及秩满。皆诣阙表请,于是诏留二年。五年,众军北伐,略淮南旧地,广达与齐军会于大岘,大破之,斩其敷城王张元范,虏获不可胜数。进克北徐州,乃授都督北徐州诸军事、北徐州刺史。寻加散骑常侍,入为右卫将军。八年,出为北兗州刺史,迁晋州刺史。十年,授使持节、都督合霍二州诸军事,进号仁威将军、合州刺史。十一年,周将梁士彦将兵围寿春,诏遣中领军樊毅、左卫将军任忠等分部趣阳平、秦郡,广达率众入淮,为掎角以击之。周军攻陷豫、霍二州,南、北兗、晋等各自拔,诸将并无功,尽失淮南之地,广达因免官,以侯还第。十二年,与豫州刺史樊毅率众北讨,克郭默城。寻授使持节、平西将军、都督郢州以上十州诸军事,率舟师四万,治江夏。周安州总管元景将兵寇江外,广达命偏师击走之。

后主即位,入为安左将军。寻受平南将军、南豫州刺史。至德二年,授安南将军,征拜侍中,又为安左将军,改封绥越郡公,封邑如前。寻为中领军。及贺若弼进军钟山,广达率众于白土岗南置阵,与弼旗鼓相对。广达躬擐甲胄,手执桴鼓,率励敢死,冒刃而前,隋军退走,广达逐北至营,杀伤甚众,如是者数四焉。及弼攻败诸将,乘胜至宫城,烧北掖门,广达犹督馀兵,苦战不息,斩获数十百人。会日暮,乃解甲,面台再拜恸哭,谓众曰:``我身不能救国,负罪深矣。''士卒皆涕泣歔欷,于是乃就执。祯明三年,依例入隋。

广达怆本朝沦覆,遘疾不治,寻以愤慨卒,时年五十九。尚书令江总抚柩恸哭,乃命笔题其棺头,为诗曰:``黄泉虽抱恨,白日自流名。悲君感义死,不作负恩生。''总又制广达墓铭,其略曰:``灾流淮海,险失金汤,时屯运极,代革天亡。爪牙背义,介胄无良,独摽忠勇,率御有方。诚贯皎日,气励严霜,怀恩感报,抚事何忘。''

初,隋将韩擒虎之济江也,广达长子世真在新蔡,乃与其弟世雄及所部奔擒虎,擒虎遣使致书,以招广达。广达时屯兵京师,乃自劾廷尉请罪。后主谓之曰:``世真虽异路中大夫,公国之重臣,吾所恃赖,岂得自同嫌疑之间乎?''加赐黄金,即日还营。

广达有队主杨孝辩,时从广达在军中,力战陷阵,其子亦随孝辩,挥刃杀隋兵十馀人,力穷,父子俱死。

史臣曰:萧摩诃气冠三军,当时良将,虽无智略,亦一代匹夫之勇矣;然口讷心劲,恂恂李广之徒欤!任忠虽勇决强断,而心怀反覆,诬绐君上,自踬其恶,鄙矣!至于鲁广达全忠守道,殉义忘身,盖亦陈代之良臣也。

\hypertarget{header-n5008}{%
\subsubsection{卷二十六}\label{header-n5008}}

孝行

◎殷不害弟不佞谢贞司马皓张昭

孔子曰:``夫圣人之德,何以加于孝乎!''孝者百行之本,人伦之至极也。凡在性灵,孰不由此。若乃奉生尽养,送终尽哀,或泣血三年,绝浆七日,思《蓼莪》之慕切,追顾复之恩深,或德感乾坤,诚贯幽显,在于历代,盖有人矣。陈承梁室丧乱,风漓化薄,及迹隐阎闾,无闻视听,今之采缀,以备阙云。

殷不害,字长卿,陈郡长平人也。祖任,齐豫章王行参军。父高明,梁尚书中兵郎。不害性至孝,居父忧过礼,由是少知名。家世俭约,居甚贫窭,有弟五人,皆幼弱,不害事老母,养小弟,勤剧无所不至,士大夫以笃行称之。

年十七,仕梁廷尉平。不害长于政事,兼饰以儒术,名法有轻重不便者,辄上书言之,多见纳用。大同五年,迁镇西府记室参军,寻以本官兼东宫通事舍人。是时朝廷政事多委东宫,不害与舍人庾肩吾直日奏事,梁武帝尝谓肩吾曰:``卿是文学之士,吏事非卿所长,何不使殷不害来邪?''其见知如此。简文又以不害善事亲,赐其母蔡氏锦裙襦、氈席被褥,单复毕备。七年,除东宫步兵校尉。太清初,迁平北府谘议参军,舍人如故。

侯景之乱,不害从简文入台。及台城陷,简文在中书省,景带甲将兵入朝陛见,过谒简文。景兵士皆羌、胡杂种,冲突左右,甚不逊,侍卫者莫不惊恐辟易,唯不害与中庶子徐摛侍侧不动。及简文为景所幽,遣人请不害与居处,景许之,不害供侍益谨。简文夜梦吞一塊土,意甚不悦,以告不害,不害曰:``昔晋文公出奔,野人遗之塊,卒反晋国,陛下此梦,事符是乎?''简文曰:``若天有徵,冀斯言不妄。''

梁元帝立,以不害为中书郎,兼廷尉卿,因将家属西上。江陵之陷也,不害先于别所督战,失母所在。于时甚寒,冰雪交下,老弱冻死者填满沟堑。不害行哭道路,远近寻求,无所不至,遇见死人沟水中,即投身而下,扶捧阅视,举体冻湿,水浆不入口,号泣不辍声,如是者七日,始得母尸。不害凭尸而哭,每举音辄气绝,行路无不为之流涕。即于江陵权殡,与王裒、庾信俱入长安,自是蔬食布衣,枯槁骨立,见者莫不哀之。

太建七年,自周还朝,其年诏除司农卿,寻迁光禄大夫。八年,加明威将军、晋陵太守。在郡感疾,诏以光禄大夫征还养疾。后主即位,加给事中。初,不害之还也,周留其长子僧首,因居关中。祯明三年,京城陷,僧首来迎,不害道病卒,时年八十五。

不佞字季卿,不害弟也。少立名节,居父丧以至孝称。好读书,尤长吏术,仕梁,起家为尚书中兵郎,甚有能称。梁元帝承制,授戎昭将军、武陵王谘议参军。承圣初,迁武康令。时兵荒饥馑,百姓流移,不佞巡抚招集,繈负而至者以千数。会江陵陷,而母卒,道路隔绝,久不得奔赴,四载之中,昼夜号泣,居处饮食,常为居丧之礼。高祖受禅,起为戎昭将军,除娄令。至是,第四兄不齐始之江陵,迎母丧柩归葬。不佞居处之节,如始闻问,若此者又三年。身自负土,手植松柏,每岁时伏腊,必三日不食。

世祖即位,除尚书左民郎,不就。后为始兴王谘议参军,兼尚书右丞,迁东宫通事舍人。及世祖崩,废帝嗣立,高宗为太傅,录尚书辅政,甚为朝望所归。不佞素以名节自立,又受委东宫,乃与仆射到仲举、中书舍人刘师知、尚书右丞王暹等,谋矫诏出高宗。众人犹豫,未敢先发,不佞乃驰诣相府,面宣敕,令相王还第。及事发,仲举等皆伏诛,高宗雅重不佞,特赦之,免其官而已。

高宗即位,以为军师始兴王谘议参军,加招远将军。寻除大匠卿,未拜,加员外散骑常侍,又兼尚书右丞。俄迁通直散骑常侍,右丞如故。太建五年卒,时年五十六。诏赠秘书监。

第三兄不疑,次不占,次不齐,并早亡。不佞最小,事第二寡嫂张氏甚谨,所得禄俸,不入私室。长子梵童,官至尚书金部郎。

谢贞,字元正,陈郡阳夏人,晋太傅安九世孙也。祖绥,梁著作佐郎、太子舍人。父蔺,正员外郎,兼散骑常侍。贞幼聪敏,有至性。祖母阮氏先苦风眩,每发便一二日不能饮食,贞时年七岁,祖母不食,贞亦不食,往往如是,亲族莫不奇之。母王氏,授贞《论语》、《孝经》,读讫便诵。八岁,尝为《春日闲居》五言诗,从舅尚书王筠奇其有佳致,谓所亲曰:``此儿方可大成,至如`风定花犹落',乃追步惠连矣。''由是名辈知之。年十三,略通《五经》大旨。尤善《左氏传》,工草隶虫篆。十四,丁父艰,号顿于地,绝而复苏者数矣。初,父蔺居母阮氏忧,不食泣血而卒,家人宾客惧贞复然,从父洽、族兄皓乃共往华严寺,请长爪禅师为贞说法,仍谓贞曰:``孝子既无兄弟,极须自爱,若忧毁灭性,谁养母邪?''自后少进饘粥。

太清之乱,亲属散亡,贞于江陵陷没,皓逃难番禺,贞母出家于宣明寺。及高祖受禅,皓还乡里,供养贞母,将二十年。太建五年,贞乃还朝,除智武府外兵参军事。俄迁尚书驾部郎中,寻迁侍郎。及始兴王叔陵为扬州刺史,引祠部侍郎阮卓为记室,辟贞为主簿,贞不得已乃行。寻迁府录事参军,领丹阳丞。贞度叔陵将有异志,因与卓自疏于王,每有宴游,辄辞以疾,未尝参预,叔陵雅钦重之,弗之罪也。俄而高宗崩,叔陵肆逆,府僚多相连逮,唯贞与卓独不坐。

后主仍诏贞入掌中宫管记,迁南平王友,加招远将军,掌记室事。府长史汝南周确新除都官尚书,请贞为让表,后主览而奇之。尝因宴席问确曰:``卿表自制邪?''确对曰:``臣表谢贞所作。''后主因敕舍人施文庆曰:``谢贞在王处,未有禄秩,可赐米百石。''至德三年,以母忧去职。顷之,敕起还府,仍加招远将军,掌记室。贞累启固辞,敕报曰:``省启具怀,虽知哀茕在疚,而官俟得才,礼有权夺,可便力疾还府也。''贞哀毁羸瘠,终不能之官舍。时尚书右丞徐祚、尚书左丞沈客卿俱来候贞,见其形体骨立,祚等怆然叹息,徐喻之曰:``弟年事已衰,礼有恒制,小宜引割自全。''贞因更感恸,气绝良久,二人涕泣,不能自胜,悯默而出。祚谓客卿曰:``信哉,孝门有孝子。''客卿曰:``谢公家传至孝,士大夫谁不仰止,此恐不能起,如何?''吏部尚书吴兴姚察与贞友善,及贞病笃,察往省之,问以后事,贞曰:``孤子飐祸所集,将随灰壤。族子凯等粗自成立,已有疏付之,此固不足仰尘厚德。即日迷喘,时不可移,便为永诀。弱儿年甫六岁,名靖,字依仁,情累所不能忘,敢以为托耳。''是夜卒,敕赙米一百斛,布三十匹。后主问察曰:``谢贞有何亲属?''察因启曰:``贞有一子年六岁。''即有敕长给衣粮。

初,贞之病亟也,遗疏告族子凯曰:``吾少罹酷罚,十四倾外廕,十六钟太清之祸,流离绝国,二十馀载。号天蹐地,遂同有感,得还侍奉,守先人坟墓,于吾之分足矣。不悟朝廷采拾空薄,累致清阶,纵其殒绝,无所酬报。今在忧棘,晷漏将尽,敛手而归,何所多念。气绝之后,若直弃之草野,依僧家尸陀林法,是吾所愿,正恐过为独异耳。可用薄板周身,载以灵车,覆以苇席,坎山而埋之。又吾终鲜兄弟,无他子孙,靖年幼少,未闲人事,但可三月施小床,设香水,尽卿兄弟相厚之情,即除之,无益之事,勿为也。''

初,贞在周尝侍赵王读,王即周武帝之爱弟也,厚相礼遇。王尝闻左右说贞每独处必昼夜涕泣,因私使访问,知贞母年老,远在江南,乃谓贞曰:``寡人若出居籓,当遣侍读还家供养。''后数年,王果出,因辞见,面奏曰:``谢贞至孝而母老,臣愿放还。''帝奇王仁爱而遣之,因随聘使杜子晖还国。所有文集,值兵乱多不存。

司马皓,字文升,河内温人也。高祖晋侍中、光禄勋柔之,以南顿王孙绍齐文献王攸之后。父子产,梁尚书水部侍郎、后阳太守,即梁武帝之外兄也。

皓幼聪警,有至性。年十二,丁内艰,孺慕过礼,水浆不入口,殆经一旬。每至号恸,必致闷绝,内外亲戚,皆惧其不胜丧。父子产每晓喻之,逼进饘粥,然毁瘠骨立。服阕,以姻戚子弟,预入问讯,梁武帝见皓羸瘦,叹息良久,谓其父子产曰:``昨见罗儿面颜憔悴,使人恻然,便是不坠家风,为有子矣。''罗儿,即皓小字也。释褐太学博士,累迁正员郎。丁父艰,哀毁逾甚,庐于墓侧,一日之内,唯进薄麦粥一升。墓在新林,连接山阜,旧多猛兽,皓结庐数载,豺狼绝迹。常有两鸠栖宿庐所,驯狎异常,新林至今犹传之。

承圣中,除太子庶子。江陵陷,随例入关,而梁室屠戮,太子瘗殡失所,皓以宫臣,乃抗表周朝,求还江陵改葬,辞甚酸切。周朝优诏答曰:``昔主父从戮,孔车有长者之风,彭越就诛,栾布得陪臣之礼。庶子乡国已改,犹怀送往之情,始验忠贞,方知臣道,即敕荆州,以礼安厝。''

太建八年,自周还朝,高宗特降殊礼,赏锡有加。除宜都王谘议参军事,徙安德宫长秋卿、通直散骑常侍、太中大夫、司州大中正,卒于官。有集十卷。

子延义,字希忠,少沈敏好学。江陵之陷,随父入关。丁母忧,丧过于礼。及皓还都,延义乃躬负灵榇,昼伏宵行,冒履冰霜,手足皆皲瘃。及至都,以中风冷,遂致挛废,数年方愈。稍迁鄱阳王录事参军、沅陵王友、司徒从事中郎。

张昭,字德明,吴郡吴人也。幼有孝性,色养甚谨,礼无违者。父,常患消渴,嗜鲜鱼,昭乃身自结网捕鱼,以供朝夕。弟乾,字玄明,聪敏博学,亦有至性。及父卒,兄弟并不衣绵帛,不食盐醋,日唯食一升麦屑粥而已。每一感恸,必致呕血,邻里闻其哭声,皆为之涕泣。父服未终,母陆氏又亡,兄弟遂六年哀毁,形容骨立,亲友见者莫识焉。家贫,未得大葬,遂布衣蔬食,十有馀年,杜门不出,屏绝人事。时衡阳王伯信临郡,举乾孝廉,固辞不就。兄弟并因毁成疾,昭失一眼,乾亦中冷苦癖,年并未五十终于家,子胤俱绝。

高宗世有太原王知玄者,侨居于会稽剡县,居家以孝闻。及丁父忧,哀毁而卒,高宗嘉之,诏改其所居清苦里为孝家里云。

史臣曰:人伦之德,莫大于孝,是以报本反始,尽性穷神,孝乎惟孝,不可不勖矣。故《记》云``塞乎天地'',盛哉!

\hypertarget{header-n5036}{%
\subsubsection{卷二十七}\label{header-n5036}}

儒林

◎沈文阿沈洙戚衮郑灼张崖陆诩沈德威贺德基全缓

张讥顾越沈不害王元规

盖今儒者,本因古之六学,斯则王教之典籍,先圣所以明天道,正人伦,致治之成法也。秦始皇焚书坑儒,六学自此缺矣。汉武帝立《五经》博士,置弟子员,设科射策,劝以官禄,其传业者甚众焉。自两汉登贤,咸资经术。魏、晋浮荡,儒教沦歇,公卿士庶,罕通经业矣。宋、齐之间,国学时复开置。梁武帝开五馆,建国学,总以《五经》教授,经各置助教云。武帝或纡銮驾,临幸庠序,释奠先师,躬亲试胄,申之宴语,劳之束帛,济济焉斯盖一代之盛矣。高祖创业开基,承前代离乱,衣冠殄尽,寇贼未宁,既日不暇给,弗遑劝课。世祖以降,稍置学官,虽博延生徒,成业盖寡。今之采缀,盖亦梁之遗儒云。

沈文阿,字国卫,吴兴武康人也。父峻,以儒学闻于梁世,授桂州刺史,不行。文阿性刚强,有膂力,少习父业,研精章句。祖舅太史叔明、舅王慧兴并通经术,而文阿颇传之。又博采先儒异同,自为义疏。治《三礼》、《三传》。察孝廉,为梁临川王国侍郎,累迁兼国子助教、《五经》博士。

梁简文在东宫,引为学士,深相礼遇,及撰《长春义记》,多使文阿撮异闻以广之。及侯景寇逆,简文别遣文阿招募士卒,入援京师。城陷,与张乘共保吴兴,乘败,文阿窜于山野。景素闻其名。求之甚急,文阿穷迫不知所出,登树自缢,遇有所亲救之,便自投而下,折其左臂。及景平,高祖以文阿州里,表为原乡令,监江阴郡。

绍泰元年,入为国子博士,寻领步兵校尉,兼掌仪礼。自太清之乱,台阁故事,无有存者,文阿父峻,梁武世尝掌朝仪,颇有遗稿,于是斟酌裁撰,礼度皆自之出。及高祖受禅,文阿辄弃官还武康,高祖大怒,发使往诛之。时文阿宗人沈恪为郡,请使者宽其死,即面缚锁颈致于高祖,高祖视而笑曰:``腐儒复何为者?''遂赦之。

高祖崩,文阿与尚书左丞徐陵、中书舍人刘师知等议大行皇帝灵座侠御衣服之制,语在师知传。及世祖即皇帝位,克日谒庙,尚书右丞庾持奉诏遣博士议其礼。文阿议曰:

民物推移,质文殊轨,圣贤因机而立教,王公随时以适宜。夫千人无君,不散则乱,万乘无主,不危则亡。当隆周之日,公旦叔父,吕、召爪牙,成王在丧,祸几覆国。是以既葬便有公冠之仪,始殡受麻冕之策。斯盖示天下以有主,虑社稷之艰难。逮乎末叶纵横,汉承其弊,虽文、景刑厝,而七国连兵。或逾月即尊,或崩日称诏,此皆有为而为之,非无心于礼制也。今国讳之日,虽抑哀于玺绂之重,犹未序于君臣之仪。古礼,朝庙退坐正寝,听群臣之政,今皇帝拜庙还,宜御太极殿,以正南面之尊,此即周康在朝一二臣卫者也。其壤奠之节,周礼以玉作贽,公侯以圭,子男执璧,此瑞玉也。奠贽既竟,又复致享,天子以璧,王后用琮。秦烧经典,威仪散灭,叔孙通定礼,尤失前宪,奠贽不圭,致享无帛,公王同璧,鸿胪奏贺。若此数事,未闻于古,后相沿袭,至梁行之。夫称觞奉寿,家国大庆,四厢雅乐,歌奏欢欣。今君臣吞哀,万民抑割,岂同于惟新之礼乎?且周康宾称奉圭,无万寿之献,此则前准明矣。三宿三咤,上宗曰飨,斯盖祭傧受福,宁谓贺酒邪!愚以今坐正殿,止行荐璧之仪,无贺酒之礼。谨撰谒庙还升正寝、群臣陪荐仪注如别。

诏可施行。寻迁通直散骑常侍,兼国子博士,领羽林监,仍令于东宫讲《孝经》、《论语》。天嘉四年卒,时年六十一。诏赠廷尉卿。

文阿所撰《仪礼》八十馀卷,《经典大义》十八卷,并行于世,诸儒多传其学。

沈洙,字弘道,吴兴武康人也。祖休稚,梁馀杭令。父山卿,梁国子博士、中散大夫。洙少方雅好学,不妄交游。治《三礼》、《春秋左氏传》。精识强记,《五经》章句,诸子史书,问无不答。解巾梁湘东王国左常侍,转中军宣城王限内参军,板仁威临贺王记室参军,迁尚书祠部郎中,时年盖二十馀。大同中,学者多涉猎文史,不为章句,而洙独积思经术,吴郡硃异、会稽贺琛甚嘉之。及异、琛于士林馆讲制旨义,常使洙为都讲。侯景之乱,洙窜于临安,时世祖在焉,亲就习业。及高祖入辅,除国子博士,与沈文阿同掌仪礼。

高祖受禅,加员外散骑常侍,历扬州别驾从事史、大匠卿。有司奏前宁远将军、建康令沈孝轨门生陈三儿牒称主人翁灵柩在周,主人奉使关内,因欲迎丧,久而未返。此月晦即是再周,主人弟息见在此者,为至月末除灵,内外即吉?为待主人还情礼申竟?以事谘左丞江德藻,德藻议:``王卫军云:`久丧不葬,唯主人不变,其馀亲各终月数而除。'此盖引《礼》文论在家内有事故未得葬者耳。孝轨既在异域,虽已迎丧,还期无指,诸弟若遂不除,永绝婚嫁,此于人情,或为未允。中原沦陷已后,理有事例,宜谘沈常侍详议。''洙议曰:``礼有变正,又有从宜。《礼小记》云:`久而不葬者,唯主丧者不除,其馀以麻终月数者除丧则已。'《注》云:`其馀谓傍亲。'如郑所解,众子皆应不除,王卫军所引,此盖礼之正也。但魏氏东关之役,既失亡尸柩,葬礼无期,议以为礼无终身之丧,故制使除服。晋氏丧乱,或死于虏庭,无由迎殡,江左故复申明其制。李胤之祖,王华之父,并存亡不测,其子制服依时释縗,此并变礼之宜也。孝轨虽因奉使便欲迎丧,而戎狄难亲,还期未克。愚谓宜依东关故事,在此国内者,并应释除縗麻,毁灵附祭,若丧柩得还,别行改葬之礼。自天下寇乱,西朝倾覆,流播绝域,情礼莫申,若此之徒,谅非一二,宁可丧期无数,而弗除衰服,朝庭自应为之限制,以义断恩,通访博识,折之礼衷。''德藻依洙议,奏可。

世祖即位,迁通直散骑常侍,侍东宫读。寻兼尚书左丞,领扬州大中正,迁光禄卿,侍读如故。废帝嗣位,重为通直散骑常侍,兼尚书左丞。迁戎昭将军、轻车衡阳王长史,行府国事,带琅邪、彭城二郡丞。梁代旧律,测囚之法,日一上,起自晡鼓,尽于二更。及比部郎范泉删定律令,以旧法测立时久,非人所堪,分其刻数,日再上。廷尉以为新制过轻,请集八座丞郎并祭酒孔奂、行事沈洙五舍人会尚书省详议。时高宗录尚书,集众议之,都官尚书周弘正曰:``未知狱所测人,有几人款?几人不款?须前责取人名及数并其罪目,然后更集。''得廷尉监沈仲由列称,别制已后,有寿羽儿一人坐杀寿慧,刘磊渴等八人坐偷马仗家口渡北,依法测之,限讫不款。刘道朔坐犯七改偷,依法测立,首尾二日而款。陈法满坐被使封藏、阿法受钱,未及上而款。弘正议曰:``凡小大之狱,必应以情,正言依准五听,验其虚实,岂可全恣考掠,以判刑罪。且测人时节,本非古制,近代已来,方有此法。起自晡鼓,迄于二更,岂是常人所能堪忍?所以重械之下,危堕之上,无人不服,诬枉者多。朝晚二时,同等刻数,进退而求,于事为衷。若谓小促前期,致实罪不伏,如复时节延长,则无愆妄款。且人之所堪,既有强弱,人之立意,固亦多途。至如贯高榜笞刺爇,身无完者,戴就熏针并极,困笃不移,岂关时刻长短,掠测优劣?夫与杀不辜,宁失不经,罪疑惟轻,功疑惟重,斯则古之圣王,垂此明法。愚谓依范泉著制,于事为允。''舍人盛权议曰:``比部范泉新制,尚书周弘正明议,咸允《虞书》惟轻之旨,《殷颂》敷正之言。窃寻廷尉监沈仲由等列新制以后,凡有狱十一人,其所测者十人,款者唯一。愚谓染罪之囚,狱官宜明加辩析,穷考事理。若罪有可疑,自宜启审分判,幸无滥测;若罪有实验,乃可启审测立;此则枉直有分,刑宥斯理。范泉今牒述《汉律》,云`死罪及除名,罪证明白,考掠已至,而抵隐不服者,处当列上'。杜预注云`处当,证验明白之状,列其抵隐之意'。窃寻旧制深峻,百中不款者一,新制宽优,十中不款者九,参会两文,宽猛实异,处当列上,未见厘革。愚谓宜付典法,更详`处当列上'之文。''洙议曰:``夜中测立,缓急易欺,兼用昼漏,于事为允。但漏刻赊促,今古不同,《汉书·律历》,何承天、祖冲之、釭之父子《漏经》,并自关鼓至下鼓,自晡鼓至关鼓,皆十三刻,冬夏四时不异。若其日有长短,分在中时前后。今用梁末改漏,下鼓之后,分其短长,夏至之日,各十七刻,冬至之日,各十二刻。伏承命旨,刻同勒令,检一日之刻乃同,而四时之用不等,廷尉今牒,以时刻短促,致罪人不款。愚意愿去夜测之昧,从昼漏之明,斟酌今古之间,参会二漏之义,舍秋冬之少刻,从夏日之长晷,不问寒暑,并依今之夏至,朝夕上测,各十七刻。比之古漏,则一上多昔四刻,即用今漏,则冬至多五刻。虽冬至之时,数刻侵夜,正是少日,于事非疑。庶罪人不以漏短而为捍,狱囚无以在夜而致诬,求之鄙意,窃谓允合。''众议以为宜依范泉前制,高宗曰:``沈长史议得中,宜更博议。''左丞宗元饶议曰:``窃寻沈议非顿异范,正是欲使四时均其刻数,兼斟酌其佳,以会优剧。即同牒请写还删定曹详改前制。''高宗依事施行。

洙以太建元年卒,时年五十二。

戚衮,字公文,吴郡盐官人也。祖显,齐给事中。父霸,梁临贺王府中兵参军。衮少聪慧,游学京都,受《三礼》于国子助教刘文绍,一二年中,大义略备。年十九,梁武帝敕策《孔子正言》并《周礼》、《礼记》义,衮对高第。仍除扬州祭酒从事史。

就国子博士宋怀方质《仪礼》义,怀方北人,自魏携《仪礼》、《礼记》疏,秘惜不传,及将亡,谓家人曰:``吾死后,戚生若赴,便以《仪礼》、《礼记》义本付之,若其不来,即宜随尸而殡。''其为儒者推许如此。寻兼太学博士。

梁简文在东宫,召衮讲论。又尝置宴集玄儒之士,先命道学互相质难,次令中庶子徐摛驰骋大义,间以剧谈。摛辞辩纵横,难以答抗,诸人慑气,皆失次序。衮时骋义,摛与往复,衮精采自若,对答如流,简文深加叹赏。寻除员外散骑侍郎,又迁员外散骑常侍。敬帝承制,出为江州长史,仍随沈泰镇南豫州。泰之奔齐也,逼衮俱行,后自鄴下遁还。又随程文季北伐,吕梁军败,衮没于周,久之得归。仍兼国子助教,除中卫始兴王府录事参军。太建十三年卒,时年六十三。

衮于梁代撰《三礼义记》,值乱亡失,《礼记义》四十卷行于世。

郑灼,字茂昭,东阳信安人也。祖惠,梁衡阳太守。父季徽,通直散骑侍郎、建安令。灼幼而聪敏,励志儒学,少受业于皇侃。梁中大通五年,释褐奉朝请。累迁员外散骑侍郎、给事中、安东临川王府记室参军,转平西邵陵王府记室。简文在东宫,雅爱经术,引灼为西省义学士。承圣中,除通直散骑侍郎,兼国子博士。寻为威戎将军,兼中书通事舍人。高祖、世祖之世,历安东临川、镇北鄱阳二王府谘议参军,累迁中散大夫,以本职兼国子博士。未拜,太建十三年卒,时年六十八。

灼性精勤,尤明《三礼》。少时尝梦与皇侃遇于途,侃谓灼曰``郑郎开口'',侃因唾灼口中,自后义理逾进。灼家贫,抄义疏以日继夜,笔毫尽,每削用之。灼常蔬食,讲授多苦心热,若瓜时,辄偃卧以瓜镇心,起便诵读,其笃志如此。

时有晋陵张崖、吴郡陆诩、吴兴沈德威、会稽贺德基,俱以礼学自命。

张崖传《三礼》于同郡刘文绍,仕梁历王府中记室。天嘉元年,为尚书仪曹郎,广沈文阿《仪注》,撰五礼。出为丹阳令、王府谘议参军。御史中丞宗元饶表荐为国子博士。

陆诩少习崔灵恩《三礼义宗》,梁世百济国表求讲礼博士,诏令诩行。还除给事中、定阳令。天嘉初,侍始兴王伯茂读,迁尚书祠部郎中。

沈德威字怀远,少有操行。梁太清末,遁于天目山,筑室以居,虽处乱离,而笃学无倦,遂治经业。天嘉元年,征出都,侍太子讲《礼传》。寻授太学博士,转国子助教。每自学还私室以讲授,道俗受业者数十百人,率常如此。迁太常丞,兼五礼学士,寻为尚书仪曹郎,后为祠部郎。俄丁母忧去职。祯明三年入隋,官至秦王府主簿。年五十五卒。

贺德基字承业,世传《礼》学。祖文发,父淹,仕梁俱为祠部郎,并有名当世。德基少游学于京邑,积年不归,衣资罄乏,又耻服故弊,盛冬止衣裌襦袴。尝于白马寺前逢一妇人,容服甚盛,呼德基入寺门,脱白纶巾以赠之。仍谓德基曰:``君方为重器,不久贫寒,故以此相遗耳。''德基问妪姓名,不答而去。德基于《礼记》称为精明,居以传授,累迁尚书祠部郎。德基虽不至大官,而三世儒学,俱为祠部,时论美其不坠焉。

全缓,字弘立,吴郡钱塘人也。幼受《易》于博士褚仲都,笃志研玩,得其精微。梁太清初,历王国侍郎、奉朝请,俄转国子助教,兼司义郎,专讲《诗》、《易》。绍泰元年,除尚书水部郎。太建中,累迁镇南始兴王府谘议参军,随府诣湘州,以疾卒,时年七十四。缓治《周易》、《老庄》,时人言玄者咸推之。

张讥,字直言,清河武城人也。祖僧宝,梁散骑侍郎、太子洗马。父仲悦,梁庐陵王府录事参军、尚书祠部郎中。讥幼聪俊,有思理,年十四,通《孝经》、《论语》。笃好玄言,受学于汝南周弘正,每有新意,为先辈推伏。梁大同中,召补国子《正言》生。梁武帝尝于文德殿释《乾》、《坤》文言,讥与陈郡袁宪等预焉,敕令论议,诸儒莫敢先出,讥乃整容而进,谘审循环,辞令温雅。梁武帝甚异之,赐裙襦绢等,仍云``表卿稽古之力''。

讥幼丧母,有错彩经帕,即母之遗制,及有所识,家人具以告之,每岁时辄对帕哽噎,不能自胜。及丁父忧,居丧过礼。服阕,召补湘东王国左常侍,转田曹参军,迁士林馆学士。

简文在东宫,出士林馆发《孝经》题,讥论议往复,甚见嗟赏,自是每有讲集,必遣使召讥。及侯景寇逆,于围城之中,犹侍哀太子于武德后殿讲《老》、《庄》。梁台陷,讥崎岖避难,卒不事景,景平,历临安令。

高祖受禅,除太常丞,转始兴王府刑狱参军。天嘉中,迁国子助教。是时周弘正在国学,发《周易》题,弘正第四弟弘直亦在讲席。讥与弘正论议,弘正乃屈,弘直危坐厉声,助其申理。讥乃正色谓弘直曰:``今日义集,辩正名理,虽知兄弟急难,四公不得有助。''弘直曰:``仆助君师,何为不可?''举座以为笑乐。弘正尝谓人曰:``吾每登座,见张讥在席,使入懔然。''高宗世,历建安王府记室参军,兼东宫学士,转武陵王限内记室,学士如故。

后主在东宫,集宫僚置宴,时造玉柄麈尾新成,后主亲执之,曰:``当今虽复多士如林,至于堪捉此者,独张讥耳。''即手授讥。仍令于温文殿讲《庄》、《老》,高宗幸宫临听,赐御所服衣一袭。后主嗣位,领南平王府谘议参军、东宫学士。寻迁国子博士,学士如故。后主尝幸钟山开善寺,召从臣坐于寺西南松林下,敕召讥竖义。时索麈尾未至,后主敕取松枝,手以属讥,曰``可代麈尾''。顾谓群臣曰``此即是张讥后事''。祯明三年入隋,终于长安,时年七十六。

讥性恬静,不求荣利,常慕闲逸,所居宅营山池,植花果,讲《周易》、《老》、《庄》而教授焉。吴郡陆元朗、硃孟博、一乘寺沙门法才、法云寺沙门慧休、至真观道士姚绥,皆传其业。讥所撰《周易义》三十卷,《尚书义》十五卷,《毛诗义》二十卷,《孝经义》八卷,《论语义》二十卷,《老子义》十一卷,《庄子内篇义》十二卷,《外篇义》二十卷,《杂篇义》十卷,《玄部通义》十二卷,又撰《游玄桂林》二十四卷,后主尝敕人就其家写入秘阁。

子孝则,官至始安王记室参军。

顾越,字思南,吴郡盐官人也。所居新坡黄冈,世有乡校,由是顾氏多儒学焉。越少孤,以勤苦自立,聪慧有口辩,说《毛氏诗》,傍通异义,梁太子詹事周舍甚赏之。解褐扬州议曹史,兼太子左率丞。越于义理精明,尤善持论,与会稽贺文发俱为梁南平王伟所重,引为宾客。寻补《五经》博士。绍泰元年,迁国子博士。世祖即位,除始兴王谘议参军,侍东宫读。世祖以越笃老,厚遇之,除给事黄门侍郎,又领国子博士,侍读如故。废帝嗣立,除通直散骑常侍、中书舍人。华皎之构逆也,越在东阳,或谮之于高宗,言其有异志,诏下狱,因坐免。太建元年卒于家,时年七十八。

时有东阳龚孟舒者,亦治《毛氏诗》,善谈名理。梁武世,仕至寻阳郡丞,元帝在江州,遇之甚重,躬师事焉。承圣中,兼中书舍人。天嘉初,除员外散骑常侍,兼国子助教、太中大夫。太建中卒。

沈不害,字孝和,吴兴武康人也。祖总,齐尚书祠部郎。父懿,梁邵陵王参军。不害幼孤,而修立好学。十四召补国子生,举明经。累迁梁太学博士。转庐陵王府刑狱参军,长沙王府谘议,带汝南令。天嘉初,除衡阳王府中记室参军,兼嘉德殿学士。自梁季丧乱,至是国学未立,不害上书曰:

臣闻立人建国,莫尚于尊儒,成俗化民,必崇于教学。故东胶西序,事隆乎三代,环林璧水,业盛于两京。自淳源既远,浇波已扇,物之感人无穷,人之逐欲无节,是以设训垂范,启导心灵,譬彼染蓝,类诸琢玉,然后人伦以睦,卑高有序,忠孝之理既明,君臣之道攸固。执礼自基,鲁公所以难侮,歌乐已细,郑伯于是前亡,干戚舞而有苗至,泮宫成而淮夷服,长想洙、泗之风,载怀淹、稷之盛,有国有家,莫不尚已。

梁太清季年,数钟否剥,戎狄外侵,奸回内{[}B192{]},朝闻鼓鼙,夕炤烽火。洪儒硕学,解散甚于坑夷,《五典》、《九丘》,湮灭逾乎帷盖。成均自斯坠业,瞽宗于是不修,裒成之祠弗陈稞享,释菜之礼无称俎豆,颂声寂寞,遂逾一纪。后生敦悦,不见函杖之仪,晚学钻仰,徒深倚席之叹。

陛下继历升统,握镜临宇,道洽寰中,威加无外,浊流已清,重氛载廓,含生熙阜,品庶咸亨。宜其弘振礼乐,建立庠序,式稽古典,纡迹儒宫,选公卿门子,皆入于学,助教博士,朝夕讲肄,使担簦负笈,锵锵接衽,方领矩步,济济成林。如切如磋,闻《诗》闻《礼》,一年可以功倍,三冬于是足用。故能擢秀雄州,扬庭观国,入仕登朝,资优学以自辅,莅官从政,有经业以治身,轖驾列庭,青紫拾地。

古者王世子之贵,犹与国子齿,降及汉储,兹礼不坠,暨乎两晋,斯事弥隆,所以见师严而道尊者也。皇太子天纵生知,无待审喻,犹宜晦迹俯同,专经请业,奠爵前师,肃若旧典。昔阙里之堂,草莱自辟,旧宅之内,丝竹流音,前圣遗烈,深以炯戒。况复江表无虞,海外有截,岂得不开阐大猷,恢弘至道?宁可使玄教儒风,弗兴圣世,盛德大业,遂蕴尧年?臣末学小生,词无足算,轻献瞽言,伏增悚惕。

诏答曰:``省表闻之。自旧章弛废,微言将绝,朕嗣膺宝业,念在缉熙,而兵革未息,军国草创,常恐前王令典,一朝泯灭。卿才思优洽,文理可求,弘惜大体,殷勤名教,付外详议,依事施行。''又表改定乐章,诏使制三朝乐歌八首,合二十八曲,行之乐府。

五年,除赣令。入为尚书仪曹郎,迁国子博士,领羽林监,敕治五礼,掌策文谥议。太建中,除仁武南康嗣王府长史,行丹阳郡事。转员外散骑常侍、光禄卿。寻为戎昭将军、明威武陵王长史,行吴兴郡事。俄入为通直散骑常侍,兼尚书左丞。十二年卒,时年六十三。

不害治经术,善属文,虽博综坟典,而家无卷轴。每制文,操笔立成,曾无寻检。仆射汝南周弘正常称之曰:``沈生可谓意圣人乎!''著治《五礼仪》一百卷,《文集》十四卷。

子志道,字崇基,少知名。解褐扬州主簿,寻兼文林著士,历安东新蔡王记室参军。祯明三年入隋。

王元规,字正范,太原晋阳人也。祖道宝,齐员外散骑常侍、晋安郡守。父玮,梁武陵王府中记室参军。元规八岁而孤,兄弟三人,随母依舅氏往临海郡,时年十二。郡土豪刘瑱者,资财巨万,以女妻之。元规母以其兄弟幼弱,欲结强援,元规泣请曰:``姻不失亲,古人所重。岂得苟安异壤,辄婚非类!''母感其言而止。

元规性孝,事母甚谨,晨昏未尝离左右。梁时山阴县有暴水,流漂居宅,元规唯有一小船,仓卒引其母妹并孤侄入船,元规自执楫棹而去,留其男女三人,阁于树杪,及水退获全,时人皆称其至行。

元规少好学,从吴兴沈文阿受业,十八,通《春秋左氏》、《孝经》、《论语》、《丧服》。梁中大通元年,诏策《春秋》,举高第,时名儒咸称赏之。起家湘东王国左常侍,转员外散骑侍郎。简文之在东宫,引为宾客,每令讲论,甚见优礼。除中军宣城王府记室参军。及侯景寇乱,携家属还会稽。天嘉中,除始兴王府功曹参军,领国子助教,转镇东鄱阳王府记室参军,领助教如故。

后主在东宫,引为学士,亲受《礼记》、《左传》、《丧服》等义,赏赐优厚。迁国子祭酒。新安王伯固尝因入宫,适会元规将讲,乃启请执经,时论以为荣。俄除尚书祠部郎。自梁代诸儒相传为《左氏》学者,皆以贾逵、服虔之义难驳杜预,凡一百八十条,元规引证通析,无复疑滞。每国家议吉凶大礼,常参预焉。丁母忧去职,服阕,除鄱阳王府中录事参军,俄转散骑侍郎,迁南平王府限内参军。王为江州,元规随府之镇,四方学徒,不远千里来请道者,常数十百人。祯明三年入隋,为秦王府东阁祭酒。年七十四,卒于广陵。

元规著《春秋发题辞》及《义记》十一卷,《续经典大义》十四卷,《孝经义记》两卷,《左传音》三卷,《礼记音》两卷。

子大业,聪敏知名。

时有吴郡陆庆,少好学,遍知《五经》,尤明《春秋左氏传》,节操甚高。释褐梁武陵王国右常侍,历征西府墨曹行参军,除娄令。值梁季丧乱,乃覃心释典,经论靡不该究。天嘉初,征为通直散骑侍郎,不就。永阳王为吴郡太守,闻其名,欲与相见,庆固辞以疾。时宗人陆荣为郡五官掾,庆尝诣焉,王乃微服往荣第,穿壁以观之。王谓荣曰:``观陆庆风神凝峻,殆不可测,严君平、郑子真何以尚兹。''鄱阳、晋安王俱以记室征,并不就。乃筑室屏居,以禅诵为事,由是传经受业者盖鲜焉。

史臣曰:夫砥身励行,必先经术,树国崇家,率由兹道,故王政因之而至治,人伦得之而攸序。若沈文阿之徒,各专经授业,亦一代之鸿儒焉。文阿加复草创礼仪,盖叔孙通之流亚矣。

\hypertarget{header-n5092}{%
\subsubsection{卷二十八}\label{header-n5092}}

文学

◎杜之伟颜晃江德藻庾持许亨褚玠岑之敬陆琰弟瑜

何之元徐伯阳张正见蔡凝阮卓

《易》曰``观乎人文以化成天下'',孔子曰``焕乎其有文章''也。自楚、汉以降,辞人世出,洛汭、江左,其流弥畅。莫不思侔造化,明并日月,大则宪章典谟,裨赞王道,小则文理清正,申纾性灵。至于经礼乐,综人伦,通古今,述美恶,莫尚乎此。后主嗣业,雅尚文词,傍求学艺,焕乎俱集。每臣下表疏及献上赋颂者,躬自省览,其有辞工,则神笔赏激,加其爵位,是以搢绅之徒,咸知自励矣。若名位文学晃著者,别以功迹论。今缀杜之伟等学既兼文,备于此篇云尔。

杜之伟,字子大,吴郡钱塘人也。家世儒学,以《三礼》专门。父规,梁奉朝请,与光禄大夫济阳江革、都官尚书会稽孔休源友善。

之伟幼精敏,有逸才。七岁,受《尚书》,稍习《诗》、《礼》,略通其学。十五,遍观文史及仪礼故事,时辈称其早成。仆射徐勉尝见其文,重其有笔力。中大通元年,梁武帝幸同泰寺舍身,敕勉撰定仪注,勉以台阁先无此礼,召之伟草具其仪。乃启补东宫学士,与学士刘陟等钞撰群书,各为题目。所撰《富教》、《政道》二篇,皆之伟为序。及湘阴侯萧昂为江州刺史,以之伟掌记室。昂卒,庐陵王续代之,又手教招引,之伟固辞不应命,乃送昂丧柩还京。仍侍临城公读。寻除扬州议曹从事、南康嗣王墨曹参军,兼太学限内博士。大同七年,梁皇太子释奠于国学,时乐府无孔子、颜子登歌词,尚书参议令之伟制其文,伶人传习,以为故事。转补安前邵陵王田曹参军,又转刑狱参军。之伟年位甚卑,特以强识俊才,颇有名当世,吏部尚书张缵深知之,以为廊庙器也。

侯景反,之伟逃窜山泽。及高祖为丞相,素闻其名,召补记室参军。迁中书侍郎,领大著作。高祖受禅,除鸿胪卿,馀并如故。之伟启求解著作,曰:``臣以绍泰元年,忝中书侍郎,掌国史,于今四载。臣本庸贱,谬蒙盼识,思报恩奖,不敢废官。皇历惟新,驱驭轩、昊,记言记事,未易其人,著作之材,更宜选众。御史中丞沈炯、尚书左丞徐陵、梁前兼大著作虞荔、梁前黄门侍郎孔奂,或清文赡笔,或强识稽古,迁、董之任,允属群才,臣无容遽变市朝,再妨贤路。尧朝皆让,诚不可追,陈力就列,庶几知免。''优敕不许。寻转大匠卿,迁太中大夫,仍敕撰梁史。永定三年卒,时年五十二。高祖甚悼惜之,诏赠通直散骑常侍,赙钱五万,布五十匹,棺一具,克日举哀。

之伟为文,不尚浮华,而温雅博赡。所制多遗失,存者十七卷。

颜晃,字元明,琅邪临沂人也。少孤贫,好学,有辞采。解褐梁邵陵王兼记室参军。时东宫学士庾信尝使于府中,王使晃接对,信轻其尚少,曰:``此府兼记室几人?''晃答曰:``犹当少于宫中学士。''当时以为善对。

侯景之乱,西奔荆州。承圣初,除中书侍郎。时杜龛为吴兴太守,专好勇力,其所部多轻险少年,元帝患之,乃使晃管其书翰。仍敕龛曰:``卿年时尚少,习读未晚,颜晃文学之士,使相毘佐,造次之间,必宜谘禀。''及龛诛,晃归世祖,世祖委以书记,亲遇甚笃。除宣毅府中录事,兼记室参军。

永定二年,高祖幸大庄严寺,其夜甘露降,晃献《甘露颂》,词义该典,高祖甚奇之。天嘉初,迁员外散骑常侍,兼中书舍人,掌诏诰。三年卒,时年五十三。诏赠司农卿,谥曰贞子,并赐墓地。

晃家世单门,傍无戚援,而介然修立,为当世所知。其表奏诏诰,下笔立成,便得事理,而雅有气质。有集二十卷。

江德藻,字德藻,济阳考城人也。祖柔之,齐尚书仓部郎中。父革,梁度支尚书、光禄大夫。德藻好学,善属文。美风仪,身长七尺四寸。性至孝,事亲尽礼。与异产昆弟居,恩惠甚笃。起家梁南中郎武陵王行参军。大司马南平王萧伟闻其才,召为东阁祭酒。迁安西湘东王府外兵参军,寻除尚书比部郎,以父忧去职。服阕之后,容貌毁瘠,如居丧时。除安西武陵王记室,不就。久之,授庐陵王记室参军。除廷尉正,寻出为南兗州治中。及高祖为司空、征北将军,引德藻为府谘议。转中书侍郎,迁云麾临海王长史。陈台建,拜尚书吏部侍郎。

高祖受禅,授秘书监,兼尚书左丞。寻以本官兼中书舍人。天嘉四年,兼散骑常侍,与中书郎刘师知使齐,著《北征道理记》三卷。还拜太子中庶子,领步兵校尉。顷之迁御史中丞,坐公事免。寻拜振远将军、以通直散骑常侍。自求宰县,出补新喻令,政尚恩惠,颇有异绩。六年,卒于官,时年五十七。世祖甚悼惜之,诏赠散骑常侍。所著文笔十五卷。

子椿,亦善属文,历太子庶子、尚书左丞。

庾持,字允德,颍川鄢陵人也。祖佩玉,宋长沙内史。父沙弥,梁长城令。持少孤,性至孝,居父忧过礼。笃志好学,尤善书记,以才艺闻。解褐梁南平王国左常侍、轻车河东王府行参军,兼尚书郎,寻而为真。出为安吉令,迁镇东邵陵王府限外记室,兼建康令。天监初,世祖与持有旧,及世祖为吴兴太守,以持为郡丞,兼掌书翰,自是常依文帝。文帝克张彪,镇会稽,又令持监临海郡。以贪纵失民和,为山盗所劫,幽执十旬,世祖遣刘澄讨平之,持乃获免。高祖受禅,授安东临川王府谘议参军。天嘉初,迁尚书左丞。以预长城之功,封崇德县子,邑三百户。拜封之日,请令史为客,受其饷遗,世祖怒之,因坐免。寻为宣惠始兴王府谘议参军。除临安令,坐杖杀县民免封。迁为给事黄门侍郎。除棱威将军、盐官令。光大元年,迁秘书监,知国史事。又为少府卿,领羽林监。迁太中大夫,领步兵校尉。太建元年卒,时年六十二。诏赠光禄大夫。

持善字书,每属辞,好为奇字,文士亦以此讥之。有集十卷。

许亨,字亨道,高阳新城人,晋徵士询之六世孙也。曾祖珪,历给事中,委桂阳太守,高尚其志,居永兴之究山,即询之所隐也。祖勇慧,齐太子家令、冗从仆射。父懋,梁始平天门二郡守、太子中庶子、散骑常侍,以学艺闻,撰《毛诗风雅比兴义类》十五卷,《述行记》四卷。亨少传家业,孤介有节行。博通群书,多识前代旧事,名辈皆推许之,甚为南阳刘之遴所重,每相称述。解褐梁安东王行参军,兼太学博士,寻除平西府记室参军。太清初,为征西中记室,兼太常丞。

侯景之乱,避地郢州,会梁邵陵王自东道至,引为谘议参军。王僧辩之袭郢州也,素闻其名,召为仪同从事中郎。迁太尉从事中郎,与吴兴沈炯对掌书记,府朝政务,一以委焉。晋安王承制,授给事黄门侍郎,亨奉笺辞府,僧辩答曰:``省告,承有朝授,良为德举。卿操尚惇深,文艺该洽,学优而官,自致青紫。况久羁骏足,将成顿辔,匡辅虚暗,期寄实深。既欣游处,用忘劳屈,而枳棘栖鹓,常以增叹。夕郎之选,虽为清显,位以才升,差自无愧。且卿始云知命,方骋康衢,未有执戟之疲,便深夜行之慨,循复来翰,殊用怃然。古人相思,千里命驾,素心不昧,宁限城闉,存顾之深,荒惭无已。''

高祖受禅,授中散大夫,领羽林监。迁太中大夫,领大著作,知梁史事。初,僧辩之诛也。所司收僧辩及其子頠尸,于方山同坎埋瘗,至是无敢言者。亨以故吏,抗表请葬之,乃与故义徐陵、张种、孔奂等,相率以家财营葬,凡七柩皆改窆焉。

光大初,高宗入辅,以亨贞正有古人之风,甚相钦重,常以师礼事之。及到仲举之谋出高宗也,毛喜知其诈,高宗问亨,亨劝勿奉诏。高宗即位,拜卫尉卿。太建二年卒,时年五十四。

初撰《齐书》并《志》五十卷,遇乱失亡。后撰《梁史》,成者五十八卷。梁太清之后所制文笔六卷。

子善心,早知名,官至尚书度支侍郎。

褚玠,字温理,河南阳翟人也。曾祖炫,宋升明初与谢朏、江斅、刘俣入侍殿中,谓之四友。官至侍中、吏部尚书,谥贞子。祖涷,梁御史中丞。父蒙,太子舍人。玠九岁而孤,为叔父骠骑从事中郎随所养。早有令誉,先达多以才器许之。及长,美风仪,善占对,博学能属文,词义典实,不好艳靡。起家王府法曹,历转外兵记室。天嘉中,兼通直散骑常侍,聘齐,还为桂阳王友。迁太子庶子、中书侍郎。

太建中,山阴县多豪猾,前后令皆以赃污免,高宗患之,谓中书舍人蔡景历曰:``稽阴大邑,久无良宰,卿文士之内,试思其人。''景历进曰:``褚玠廉俭有干用,未审堪其选不?''高宗曰:``甚善,卿言与朕意同。''乃除戎昭将军、山阴令。县民张次的、王休达等与诸猾吏贿赂通奸,全丁大户,类多隐没。玠乃锁次的等,具状启台,高宗手敕慰劳,并遣使助玠搜括,所出军民八百馀户。时舍人曹义达为高宗所宠,县民陈信家富于财,谄事义达,信父显文恃势横暴。玠乃遣使执显文,鞭之一百,于是吏民股栗,莫敢犯者。信后因义达谮玠,竟坐免官。玠在任岁馀,守禄俸而已,去官之日,不堪自致,因留县境,种蔬菜以自给。或嗤玠以非百里之才,玠答曰:``吾委输课最,不后列城,除残去暴,奸吏局蹐。若谓其不能自润脂膏,则如来命。以为不达从政,吾未服也。''时人以为信然。皇太子知玠无还装,手书赐粟米二百斛,于是还都。太子爱玠文辞,令入直殿省。十年,除电威将军、仁威淮南王长史,顷之,以本官掌东宫管记。十二年,迁御史中丞,卒于官,时年五十二。

玠刚毅有胆决,兼善骑射。尝从司空侯安都于徐州出猎,遇有猛虎,玠引弓射之,再发皆中口入腹,俄而虎毙。及为御史中丞,甚有直绳之称。自梁末丧乱,朝章废弛,司宪因循,守而勿革,玠方欲改张,大为条例,纲维略举,而编次未讫,故不列于后焉。及卒,太子亲制志铭,以表惟旧。至德二年,追赠秘书监。所制章奏杂文二百馀篇,皆切事理,由是见重于时。

子亮,有才学,官至尚书殿中侍郎。

岑之敬,字思礼,南阳棘阳人也。父善纡,梁世以经学闻,官至吴宁令、司义郎。之敬年五岁,读《孝经》,每烧香正坐,亲戚咸加叹异。年十六,策《春秋左氏》、制旨《孝经》义,擢为高第。御史奏曰:``皇朝多士,例止明经,若颜、闵之流,乃应高第。''梁武帝省其策曰:``何妨我复有颜、闵邪?''因召入面试,令之敬升讲座,敕中书舍人硃异执《孝经》,唱《士孝章》,武帝亲自论难。之敬剖释纵横,应对如响,左右莫不嗟服。乃除童子奉车郎,赏赐优厚。十八,预重云殿法会,时武帝亲行香,熟视之敬曰:``未几见兮,突而弁兮!''即日除太学限内博士。寻为寿光学士、司义郎,又除武陵王安西府刑狱参军事。太清元年,表请试吏,除南沙令。

侯景之乱,之敬率所部赴援京师。至郡境,闻台城陷,乃与众辞诀,归乡里。承圣二年,除晋安王宣惠府中记室参军。是时萧勃据岭表,敕之敬宣旨慰喻,会江陵陷,仍留广州。太建初,还朝,授东宫义省学士,太子素闻其名,尤降赏接。累迁鄱阳王中卫府记室、镇北府中录事参军、南台治书侍御史、征南府谘议参军。

之敬始以经业进,而博涉文史,雅有词笔,不为醇儒。性廉谨,未尝以才学矜物,接引后进,恂恂如也。每忌日营斋,必躬自洒扫,涕泣终日,士君子以笃行称之。十一年卒,时年六十一。太子嗟惜,赙赠甚厚。有集十卷行于世。

子德润,有父风,官至中军吴兴王记室。

陆琰,字温玉,吏部尚书琼之从父弟也。父令公,梁中军宣城王记室参军。琰幼孤、好学,有志操。州举秀才。解褐宣惠始兴王行参军,累迁法曹外兵参军,直嘉德殿学士。世祖听览馀暇,颇留心史籍,以琰博学,善占诵,引置左右。尝使制《刀铭》,琰援笔即成,无所点窜,世祖嗟赏久之,赐衣一袭。俄兼通直散骑常侍,副琅邪王厚聘齐,及至鄴下而厚病卒,琰自为使主。时年二十馀,风神韶亮,占对闲敏,齐士大夫甚倾心焉。还为云麾新安王主簿,迁安成王长史,宁远府记室参军。太建初,为武陵王明威府功曹史,兼东宫管记。丁母忧去官。五年卒,时年三十四。太子甚伤悼之,手令举哀,加其赙赠,又自制志铭。至德二年,追赠司农卿。

琰寡嗜欲,鲜矜竞,游心经籍,晏如也。其所制文笔多不存本,后主求其遗文,撰成二卷。有弟瑜。

瑜字干玉。少笃学,美词藻。州举秀才。解褐骠骑安成王行参军,转军师晋安王外兵参军、东宫学士。兄琰时为管记,并以才学娱侍左右,时人比之二应。太建二年,太子释奠于太学,宫臣并赋诗,命瑜为序,文甚赡丽。迁尚书祠部郎中,丁母忧去职。服阕,为桂阳王明威将军功曹史,兼东宫管记。累迁永阳王文学、太子洗马、中舍人。

瑜幼长读书,昼夜不废,聪敏强记,一览无复遗失。尝受《庄》、《老》于汝南周弘正,学《成实论》于僧滔法师,并通大旨。时皇太子好学,欲博览群书,以子集繁多,命瑜钞撰,未就而卒,时年四十四。太子为之流涕,手令举哀,官给丧事,并亲制祭文,遣使者吊祭。仍与詹事江总书曰:``管记陆瑜,奄然殂化,悲伤悼惜,此情何已。吾生平爱好,卿等所悉,自以学涉儒雅,不逮古人,钦贤慕士,是情尤笃。梁室乱离,天下糜沸,书史残缺,礼乐崩沦,晚生后学,匪无墙面,卓尔出群,斯人而已。吾识览虽局,未曾以言议假人,至于片善小才,特用嗟赏。况复洪识奇士,此故忘言之地。论其博综子史,谙究儒墨,经耳无遗,触目成诵,一褒一贬,一激一扬,语玄析理,披文摘句,未尝不闻者心伏,听者解颐,会意相得,自以为布衣之赏。吾监抚之暇,事隙之辰,颇用谭笑娱情,琴樽间作,雅篇艳什,迭互锋起。每清风朗月,美景良辰,对群山之参差,望巨波之滉漾,或玩新花,时观落叶,即听春鸟,又聆秋雁,未尝不促膝举觞,连情发藻,且代琢磨,间以嘲谑,俱怡耳目,并留情致。自谓百年为速,朝露可伤,岂谓玉折兰摧,遽从短运,为悲为恨,当复何言。遗迹馀文,触目增泫,绝弦投笔,恒有酸恨。以卿同志,聊复叙怀,涕之无从,言不写意。''其见重如此。至德二年,追赠光禄卿。有集十卷。瑜有从父兄玠,从父弟琛。

玠字润玉,梁大匠卿晏子之子。弘雅有识度,好学,能属文。举秀才,对策高第。吏部尚书袁枢荐之于世祖,超授衡阳王文学,直天保殿学士。太建初,迁长沙王友,领记室。后主在东宫,闻其名,徵为管记。仍除中舍人,管记如故,甚见亲待。寻以疾失明,将还乡里,太子解衣赠玠,为之流涕。八年卒,时年三十七。有令举哀,并加赗赠。至德二年,追赠少府卿。有集十卷。

琛字洁玉,宣毅临川王长史丘公之子。少警俊,事后母以孝闻。世祖为会稽太守,琛年十八,上《善政颂》,甚有词采,由此知名,举秀才。起家为衡阳王主簿,兼东宫管记。历豫章王文学,领记室,司徒主簿,直宣明殿学士。寻迁尚书三公侍郎,兼通直散骑常侍,聘齐,还为司徒左西掾。又掌东宫管记,太子爱琛才辩,深礼遇之。后主嗣位,迁给事黄门侍郎、中书舍人,参掌机密。琛性颇疏,坐漏泄禁中语,诏赐死,时年四十二。

何之元,庐江灊人也。祖僧达,齐南台治书侍御史。父法胜,以行业闻。之元幼好学,有才思,居丧过礼,为梁司空袁昂所重。天监末,昂表荐之,因得召见。解褐梁太尉临川王扬州议曹从事史,寻转主簿。及昂为丹阳尹,辟为丹阳五官掾,总户曹事。寻除信义令。之元宗人敬容者,势位隆重,频相顾访,之元终不造焉。或问其故,之元曰:``昔楚人得宠于观起,有马者皆亡。夫德薄任隆,必近覆败,吾恐不获其利而招其祸。''识者以是称之。

会安西武陵王为益州刺史,以之元为安西刑狱参军。侯景之乱,武陵王以太尉承制,授南梁州刺史、北巴西太守。武陵王自成都举兵东下,之元与蜀中民庶抗表请无行,王以为沮众,囚之元于舰中。及武陵兵败,之元从邵陵太守刘恭之郡。俄而江陵陷,刘恭卒,王琳召为记室参军。梁敬帝册琳为司空,之元除司空府谘议参军,领记室。

王琳之立萧庄也,署为中书侍郎。会齐文宣帝薨,令之元赴吊,还至寿春,而王琳败,齐主以为扬州别驾,所治即寿春也。及在军北伐,得淮南地,湘州刺史始兴王叔陵遣功曹史柳咸赍书召之元。之元始与朝庭有隙,及书至,大惶恐,读书至``孔璋无罪,左车见用'',之元仰而叹曰:``辞约若此,岂欺我哉!''遂随咸至湘州。太建八年,除中卫府功曹参军事,寻迁谘议参军。

及叔陵诛,之元乃屏绝人事,锐精著述。以为梁氏肇自武皇,终于敬帝,其兴亡之运,盛衰之迹,足以垂鉴戒,定褒贬。究其始终,起齐永元元年,迄于王琳遇获,七十五年行事,草创为三十卷,号曰《梁典》。其序曰:

记事之史,其流不一,编年之作,无若《春秋》,则鲁史之书,非帝皇之籍也。案三皇之简为《三坟》,五帝之策为《五典》,此典义所由生也。至乃《尚书》述唐帝为《尧典》,虞帝为《舜典》,斯又经文明据。是以典之为义久矣哉。若夫《马史》、《班汉》,述帝称纪,自兹厥后,因相祖习。及陈寿所撰,名之曰志,总其三国,分路扬镳。唯何法盛《晋书》变帝纪为帝典,既云师古,在理为优。故今之所作,称为《梁典》。

梁有天下,自中大同以前,区宇宁晏,太清以后,寇盗交侵,首尾而言,未为尽美,故开此一书,分为六意。以高祖创基,因乎齐末,寻宗讨本,起自永元,今以前如干卷为《追述》。高祖生自布衣,长于弊俗,知风教之臧否,识民黎之情伪。爰逮君临,弘斯政术,四纪之内,实云殷阜。今以如干卷为《太平》。世不常夷,时无恒治,非自我后,仍属横流,今以如干卷为《叙乱》。洎高祖晏驾之年,太宗幽辱之岁,讴歌狱讼,向西陕不向东都;不庭之民,流逸之士,征伐礼乐,归世祖不归太宗。拨乱反正,厥庸斯在,治定功成,其勋有属。今以如干卷为《世祖》。至于四海困穷,五德升替,则敬皇绍立,仍以禅陈,今以如干卷为《敬帝》。骠骑王琳,崇立后嗣,虽不达天命,然是其忠节,今以如干卷为《后嗣主》。至在太宗,虽加美谥,而大宝之号,世所不遵,盖以拘于贼景故也。承圣纪历,自接太清,神笔诏书,非宜辄改,详之后论,盖有理焉。

夫事有始终,人有业行,本末之间,颇宜诠叙。案臧荣绪称史无裁断,犹起居注耳,由此而言,实资详悉。

又编年而举其岁次者,盖取分明而易寻也。若夫猃狁孔炽,鲠我中原,始自一君,终为二主,事有相涉,言成混漫。今以未分之前为北魏,既分之后高氏所辅为东魏,宇文所挟为西魏,所以相分别也。重以盖彰殊体,繁省异文,其间损益,颇有凡例。

祯明三年,京城陷,乃移居常州之晋陵县。隋开皇十三年,卒于家。

徐伯阳,字隐忍,东海人也。祖度之,齐南徐州议曹从事史。父僧权,梁东宫通事舍人,领秘书,以善书知名。伯阳敏而好学,善色养,进止有节。年十五,以文笔称。学《春秋左氏》。家有史书,所读者近三千馀卷。试策高第,尚书板补梁河东王国右常侍、东宫学士、临川嗣王府墨曹参军。大同中,出为候官令,甚得民和。侯景之乱,伯阳浮海南至广州,依于萧勃,勃平还朝,仍将家属之吴郡。

天嘉二年,诏侍晋安王读。寻除司空侯安都府记室参军事,安都素闻其名,见之,降席为礼。甘露降乐游苑,诏赐安都,令伯阳为谢表,世祖览而奇之。太建初,中记室李爽、记室张正见、左民郎贺彻、学士阮卓、黄门郎萧诠、三公郎王由礼、处士马枢、记室祖孙登、比部贺循、长史刘删等为文会之友,后有蔡凝、刘助、陈暄、孔范亦预焉。皆一时之士也。游宴赋诗,勒成卷轴,伯阳为其集序,盛传于世。

及新安王为南徐州刺史,除镇北新安王府中记室参军,兼南徐州别驾,带东海郡丞。鄱阳王为江州刺史,伯阳尝奉使造焉,王率府僚与伯阳登匡岭,置宴,酒酣,命笔赋剧韵二十,伯阳与祖孙登前成,王赐以奴婢杂物。及新安王还京,除临海嗣王府限外谘议参军。十一年春,皇太子幸太学,诏新安王于辟雍发《论语》题,仍命伯阳为《辟雍颂》,甚见嘉赏。除镇右新安王府谘议参军事。十三年,闻姊丧,发疾而卒,时年六十六。

张正见,字见赜,清河东武城人也。祖盖之,魏散骑常侍、勃海长乐二郡太守。父修礼,魏散骑侍郎,归梁,仍拜本职,迁怀方太守。正见幼好学,有清才。梁简文在东宫,正见年十三,献颂,简文深赞赏之。简文雅尚学业,每自升座说经,正见尝预讲筵,请决疑义,吐纳和顺,进退详雅,四座咸属目焉。太清初,射策高第,除邵陵王国左常侍。梁元帝立,拜通直散骑侍郎,迁彭泽令。属梁季丧乱,避地于匡俗山,时焦僧度拥众自保,遣使请交,正见惧之,逊辞延纳,然以礼法自持,僧度亦雅相敬惮。

高祖受禅,诏正见还都,除镇东鄱阳王府墨曹行参军,兼衡阳王府长史。历宜都王限外记室、撰史著士,带寻阳郡丞。累迁尚书度支郎、通直散骑侍郎,著士如故。太建中卒,时年四十九。有集十四卷,其五言诗尤善,大行于世。

蔡凝,字子居,济阳考城人也。祖撙,梁吏部尚书、金紫光禄大夫。父彦高,梁给事黄门侍郎。凝幼聪晤,美容止。既长,博涉经传,有文辞,尤工草隶。天嘉四年,释褐授秘书郎,转庐陵王文学。光大元年,除太子洗马、司徒主簿。太建元年,迁太子中舍人。以名公子选尚信义公主,拜驸马都尉、中书侍郎。迁晋陵太守。及将之郡,更令左右缉治中书廨宇,谓宾友曰:``庶来者无劳,不亦可乎?''寻授宁远将军、尚书吏部侍郎。

凝年位未高,而才地为时所重,常端坐西斋,自非素贵名流,罕所交接,趣时者多讥焉。高宗常谓凝曰:``我欲用义兴主婿钱肃为黄门郎,卿意何如?''凝正色对曰:``帝乡旧戚,恩由圣旨,则无所复问。若格以佥议,黄散之职,故须人门兼美,惟陛下裁之。''高宗默然而止。肃闻而有憾,令义兴主日谮之于高宗,寻免官,迁交止。顷之,追还。

后主嗣位,受晋安王谘议参军,转给事黄门侍郎。后主尝置酒会,群臣欢甚,将移宴于弘范宫,众人咸从,唯凝与袁宪不行。后主曰:``卿何为者?''凝对曰:``长乐尊严,非酒后所过,臣不敢奉诏。''众人失色。后主曰:``卿醉矣。''即令引出。他日,后主谓吏部尚书蔡徵曰:``蔡凝负地矜才,无所用也。''寻迁信威晋熙王府长史,郁郁不得志,乃喟然叹曰:``天道有废兴,夫子云`乐天知命',斯理庶几可达。''因制《小室赋》以见志,甚有辞理。陈亡入隋,道病卒,时年四十七。

子君知,颇知名。

阮卓,陈留尉氏人。祖诠,梁散骑侍郎。父问道,梁宁远岳阳王府记室参军。卓幼而聪敏,笃志经籍,善谈论,尤工五言诗。性至孝,其父随岳阳王出镇江州,遇疾而卒,卓时年十五,自都奔赴,水浆不入口者累日。属侯景之乱,道路阻绝,卓冒履险艰,载丧柩还都。在路遇贼,卓形容毁瘁,号哭自陈,贼哀而不杀之,仍护送出境。及渡彭蠡湖,中流忽遇疾风,船几没者数四,卓仰天悲号,俄而风息,人皆以为孝感之至焉。

世祖即位,除轻车鄱阳王府外兵参军。天康元年,转云麾新安王府记室参军,仍隋府转翊右记室,带撰史著士。迁鄱阳王中卫府录事,转晋安王府记室,著士如故。及平欧阳纥,交阯夷獠往往相聚为寇抄,卓奉使招慰。交阯通日南、象郡,多金翠珠贝珍怪之产,前后使者皆致之,唯卓挺身而还,衣装无他,时论咸伏其廉。迁衡阳王府中录事参军。入为尚书祠部郎。迁始兴王中卫府记室参军。

叔陵之诛也,后主谓朝臣曰:``阮卓素不同逆,宜加旌异。''至德元年,入为德教殿学士。寻兼通直散骑常侍,副王话聘隋。隋主夙闻卓名,乃遣河东薛道衡、琅邪颜之推等,与卓谈宴赋诗,赐遗加礼。还除招远将军、南海王府谘议参军。以目疾不之官,退居里舍,改构亭宇,修山池卉木,招致宾友,以文酒自娱。祯明三年入于隋,行至江州,追感其父所终,因遘疾而卒,时年五十九。

时有武威阴铿,字子坚,梁左卫将军子春之子。幼聪慧,五岁能诵诗赋,日千言。及长,博涉史传,尤善五言诗,为当时所重。释褐梁湘东王法曹参军。天寒,铿尝与宾友宴饮,见行觞者,因回酒炙以授之,众坐皆笑,铿曰:``吾侪终日酣饮,而执爵者不知其味,非人情也。''及侯景之乱,铿尝为贼所擒,或救之获免,铿问其故,乃前所行觞者。天嘉中,为始兴王府中录事参军。世祖尝宴群臣赋诗,徐陵言之于世祖,即日召铿预宴,使赋新成安乐宫,铿授笔便就,世祖甚叹赏之。累迁招远将军、晋陵太守、员外散骑常侍,顷之卒。有集三卷行于世。

史臣曰:夫文学者,盖人伦之所基欤?是以君子异乎众庶。昔仲尼之论四科,始乎德行,终于文学,斯则圣人亦所贵也。至如杜之伟之徒,值于休运,各展才用,之伟尤著美焉。

\hypertarget{header-n5155}{%
\subsubsection{卷二十九}\label{header-n5155}}

熊昙朗周迪留异陈宝应

熊昙朗,豫章南昌人也。世为郡著姓。昙朗跅弛不羁,有膂力,容貌甚伟。侯景之乱,稍聚少年,据豊城县为栅,桀黠劫盗多附之。梁元帝以为巴山太守。荆州陷,昙朗兵力稍强,劫掠邻县,缚卖居民,山谷之中,最为巨患。

及侯瑱镇豫章,昙朗外示服从,阴欲图瑱。侯方儿之反瑱也,昙朗为之谋主。瑱败,昙朗获瑱马仗子女甚多。及萧勃逾领,欧阳頠为前军,昙朗绐頠共往巴山袭黄法抃,又报法抃期共破頠,约曰``事捷与我马仗''。及出军,与頠掎角而进,又绐頠曰``余孝顷欲相掩袭,须分留奇兵,甲仗既少,恐不能济''。頠乃送甲三百领助之。及至城下,将战,昙朗伪北,法抃乘之,頠失援,狼狈退衄,昙朗取其马仗而归。时巴山陈定亦拥兵立寨,昙朗伪以女妻定子。又谓定曰``周迪、余孝顷并不愿此婚,必须以强兵来迎''。定乃遣精甲三百并土豪二十人往迎,既至,昙朗执之,收其马仗,并论价责赎。

绍泰二年,昙朗以南川豪帅,随例除游骑将军。寻为持节、飙猛将军、桂州刺史资,领豊城令,历宜新、豫章二郡太守。王琳遣李孝钦等随余孝顷于临川攻周迪,昙朗率所领赴援。其年,以功除持节、通直散骑常侍、宁远将军,封永化县侯,邑一千户,给鼓吹一部。又以抗御王琳之功,授平西将军、开府仪同三司,馀并如故。及周文育攻余孝劢于豫章,昙朗出军会之,文育失利,昙朗乃害文育,以应王琳,事见文育传。于是尽执文育所部诸将,据新淦县,带江为城。

王琳东下,世祖征南川兵,江州刺史周迪、高州刺史黄法抃欲沿流应赴,昙朗乃据城列舰断遏,迪等与法抃因帅南中兵筑城围之,绝其与琳信使。及王琳败走,昙朗党援离心,迪攻陷其城,虏其男女万馀口。昙朗走入村中,村民斩之,传首京师,悬于硃雀观。于是尽收其宗族,无少长皆弃市。

周迪,临川南城人也。少居山谷,有膂力,能挽强弩,以弋猎为事。侯景之乱,迪宗人周续起兵于临川,梁始兴王萧毅以郡让续,迪召募乡人从之,每战必勇冠众军。续所部渠帅,皆郡中豪族,稍骄横,续颇禁之,渠帅等并怨望,乃相率杀续,推迪为主,迪乃据有临川之地,筑城于工塘。梁元帝授迪持节、通直散骑常侍、壮武将军、高州刺史,封临汝县侯,邑五百户。

绍泰二年,除临川内史。寻授使持节、散骑常侍、信威将军、衡州刺史,领临川内史。周文育之讨萧勃也,迪按甲保境,以观成败。文育使长史陆山才说迪,迪乃大出粮饷,以资文育。勃平,以功加振远将军,迁江州刺史。

高祖受禅,王琳东下,迪欲自据南川,乃总召所部八郡守宰结盟,声言入赴,朝廷恐其为变,因厚慰抚之。琳至湓城,新吴洞主余孝顷举兵应琳。琳以为南川诸郡可传檄而定,乃遣其将李孝钦、樊猛等南征粮饷。猛等与余孝顷相合,众且二万,来趋工塘,连八城以逼迪。迪使周敷率众顿临川故郡,截断江口,因出与战,大败之,屠其八城,生擒李孝钦、樊猛、余孝顷送于京师,收其军实,器械山积,并虏其人马,迪并自纳之。永定二年,以功加平南将军、开府仪同三司,增邑一千五百户,给鼓吹一部。

世祖嗣位,进号安南将军。熊昙朗之反也,迪与周敷、黄法抃等率兵共围昙朗,屠之,尽有其众。王琳败后,世祖徵迪出镇湓城,又徵其子入朝,迪趑趄顾望,并不至。豫章太守周敷本属于迪。至是与黄法抃率其所部诣阙,世祖录其破熊昙朗之功,并加官赏,迪闻之,甚不平,乃阴与留异相结。及王师讨异,迪疑惧不自安,乃使其弟方兴率兵袭周敷,敷与战,破之。又别使兵袭华皎于湓城,事觉,尽为皎所擒。天嘉三年春,世祖乃下诏赦南川士民为迪所诖误者,使江州刺史吴明彻都督众军,与高州刺史黄法抃、豫章太守周敷讨迪。于是尚书下符曰:

告临川郡士庶:昔西京为盛,信、越背诞;东都中兴,萌、宠违戾。是以鹰鹯竞逐,菹醢极诛,自古有之,其来尚矣。逆贼周迪,本出舆台,有梁丧乱,暴掠山谷。我高祖躬率百越,师次九川,濯其泥沙,假以毛羽,裁解豚佩,仍剖兽符,卵翼之恩,方斯莫喻。皇运肇基,颇布诚款,国步艰阻,竟微效力。龙节绣衣,藉王爵而御下,熊旗组甲,因地险而陵上。日者王琳始贰,萧勃未夷,西结三湘,南通五岭,衡、广戡定,既安反侧,江、郢纷梗,复生携背,拥据一郡,苟且百心,志貌常违,言迹不副。特以新吴未静,地远兵强,互相兼并,成其形势。收获器械,俘虏士民,并曰私财,曾无献捷。时遣一介,终持两端。朝廷光大含弘,引纳崇遇,遂乃位等三槐,任均四岳,富贵隆赫,超绝功臣。加以出师逾岭,远相响援,按甲断江,翻然猜拒。故司空愍公,敦以宗盟,情同骨肉,城池连接,势犹脣齿,彭亡之祸,坐观难作,阶此飐故,结其党与。于时北寇侵轶,西贼凭陵,扉屦糇粮,悉以资寇,爵号军容,一遵伪党。及王师凯振,大定区中,天网恢弘,弃之度外,玺书纶诰,抚慰绸缪,冠盖缙绅,敦授重叠。至于熊昙朗剿灭,豊城克定,盖由仪同法抃之元功,安西周敷之效力,司勋有典,懋赏斯旧,恶直丑正,自为仇仇,悖礼奸谋,因此滋甚。征出湓城,历年不就,求遣侍子,累载未朝。外诱逋亡,招集不逞,中调京辇,规冀非常。擅敛征赋,罕归九府,拥遏二贾,害及四民,潜结贼异,共为表里,同恶相求,密加应援。谓我六军薄伐,三越未宁,屠破述城,虏缚妻息,分袭湓镇,称兵蠡邦,拘逼酋豪,攻围城邑,幸国有备,应时衄殄。

假节、通直散骑常侍、仁武将军、寻阳太守怀仁县伯华皎,明威将军、庐陵太守益阳县子陆子隆,并破贼徒,克全郡境。持节、散骑常侍、安西将军、定州刺史、领豫章太守西豊县侯周敷,躬扞沟垒,身当矢石,率兹义勇,以寡摧众,斩馘万计,俘虏千群。迪方收馀烬,还固墉堞。使持节、安南将军、开府仪同三司、高州刺史新建县侯法抃,雄绩早宣,忠诚夙著,未奉王命,前率义旅,既援敷等,又全子隆,裹粮擐甲,仍蹑飞走,批熊之旅,驱驰越电,振武之众,叱咤移山,以此追奔,理无遗类。虽复朽株将拔,非待寻斧,落叶就殒,无劳烈风;但去草绝根,在于未蔓,扑火止燎,贵乎速灭,分命将帅,实资英果。今遣镇南仪同司马、湘东公相刘广德,兼平西司马孙晓,北新蔡太守鲁广达,持节、安南将军、吴州刺史彭泽县侯鲁悉达,甲士万人,步出兴口。又遣前吴兴太守胡铄,树功将军、前宣城太守钱法成,天门、义阳二郡太守樊毅,云麾将军、合州刺史南固县侯焦僧度,严武将军、建州刺史辰县子张智达,持节、都督江吴二州诸军事、安南将军、江州刺史安吴县侯吴明彻,楼舰马步,直指临川。前安成内史刘士京,巴山太守蔡僧贵,南康内史刘峰,庐陵太守陆子隆,安成内史阙慎,并受仪同法抃节度,同会故郡。又命寻阳太守华皎,光烈将军、巴州刺史潘纯陀,平西将军、郢州刺史欣乐县侯章昭达,并率貔豹,迳造贼城。使持节、散骑常侍、镇南将军、开府仪同三司、湘州刺史湘东郡公度,分遣偏裨,相继上道,戈船蔽水,彀骑弥山。又诏镇南将军、开府仪同三司欧阳頠,率其子弟交州刺史盛、新除太子右率邃、衡州刺史侯晓等,以劲越之兵,逾岭北迈。千里同期,百道俱集,如脱稽诛,更淹旬晦,司空、大都督安都已平贼异,凯归非久,饮至礼毕,乘胜长驱,剿扑凶丑,如燎毛发。已有明诏,罪唯迪身,黎民何辜,一皆原宥。其有因机立功,赏如别格;执迷不改,刑兹罔赦。

吴明彻至临川,令众军作连城攻迪,相拒不能克,世祖乃遣高宗总督讨之,迪众溃,妻子悉擒,乃脱身逾岭之晋安,依于陈宝应。宝应以兵资迪,留异又遣第二子忠臣随之。

明年秋,复越东兴岭,东兴、南城、永成县民,皆迪故人,复共应之。世祖遣都督章昭达征迪,迪又散于山谷。初,侯景之乱也,百姓皆弃本业,群聚为盗,唯迪所部,独不侵扰,并分给田畴,督其耕作,民下肆业,各有赢储,政教严明,征敛必至,馀郡乏绝者,皆仰以取给。迪性质朴,不事威仪,冬则短身布袍,夏则紫纱袜腹,居常徒跣,虽外列兵卫,内有女伎,挼绳破篾,傍若无人。然轻财好施,凡所周赡,毫厘必钧,讷于言语,而襟怀信实,临川人皆德之。至是并共藏匿,虽加诛戮,无肯言者。昭达仍度岭,顿于建安,与陈宝应相抗,迪复收合出东兴。时宣城太守钱肃镇东兴,以城降迪。吴州刺史陈详,率师攻迪,详兵大败,虔化侯陈訬、陈留太守张遂并战死,于是迪众复振。世祖遣都督程灵洗击破之,迪又与十馀人窜于山穴中。日月转久,相随者亦稍苦之。后遣人潜出临川郡市鱼鲑,足痛,舍于邑子,邑子告临川太守骆牙,牙执之,令取迪自效。因使腹心勇士随入山中,诱迪出猎,伏兵于道傍,斩之,传首京都,枭于硃雀观三日。

留异,东阳长山人也。世为郡著姓。异善自居处,言语酝藉,为乡里雄豪。多聚恶少,陵侮贫贱,守宰皆患之。梁代为蟹浦戍主,历晋安、安固二县令。侯景之乱,还乡里,召募士卒,东阳郡丞与异有隙,引兵诛之,及其妻子。太守沈巡援台,让郡于异,异使兄子超监知郡事,率兵随巡出都。

及京城陷,异随临城公萧大连,大连板为司马,委以军事。异性残暴,无远略,督责大连军主及以左右私树威福,众并患之。会景将军宋子仙济浙江,异奔还乡里,寻以其众降于子仙。是时大连亦趣东阳之信安岭,欲之鄱阳,异乃为子仙乡导,令执大连。侯景署异为东阳太守,收其妻子为质。景行台刘神茂建义拒景,异外同神茂,而密契于景。及神茂败绩,为景所诛,异独获免。

侯景平后,王僧辩使异慰劳东阳,仍纠合乡闾,保据岩阻,其徒甚盛,州郡惮焉。元帝以为信安令。荆州陷,王僧辩以异为东阳太守。世祖平定会稽,异虽转输粮馈,而拥擅一郡,威福在己。绍泰二年,以应接之功,除持节、通直散骑常侍、信武将军、缙州刺史,领东阳太守,封永兴县侯,邑五百户。其年迁散骑常侍、信威将军,增邑三百户,馀并如故。又以世祖长女豊安公主配异第三子贞臣。永定二年,征异为使持节、散骑常侍、都督南徐州诸军事、平北将军、南徐州刺史,异迁延不就。

世祖即位,改授都督缙州诸军事、安南将军、缙州刺史,领东阳太守。异频遣其长史王澌为使入朝,澌每言朝廷虚弱,异信之,虽外示臣节,恒怀两端,与王琳自鄱阳信安岭潜通信使。王琳又遣使往东阳,署守宰。及琳败,世祖遣左卫将军沈恪代异为郡,实以兵袭之。异出下淮抗御,恪与战,败绩,退还钱塘,异乃表启逊谢。是时众军方事湘、郢,乃降诏书慰喻,且羁縻之,异亦知朝廷终讨于己,乃使兵戍下淮及建德,以备江路。湘州平,世祖乃下诏曰:

昔四罪难弘,大妫之所无赦,九黎乱德,少昊之所必诛。自古皇王,不贪征伐,苟为时蠹,事非获已。逆贼留异,数应亡灭,缮甲完聚,由来积年。进谢群龙,自跃于千里,退怀首鼠,恒持于百心。中岁密契番禺,既弘天网,赐以名爵,敦以国姻,傥望怀音,犹能革面。王琳窃据中流,翻相应接,别引南川之岭路,专为东道之主人,结附凶渠,唯欣祸乱。既妖氛荡定,气沮心孤,类伤鸟之惊弦,等穷兽之谋触。虽复遣家入质,子阳之态转遒;侍子还朝,隗嚣之心方炽。

朕志相成养,不计疵慝,披襟解带,敦喻殷勤。蜂目弥彰,枭声无改,遂置军江口,严戍下淮,显然反叛,非可容匿。且缙邦膏腴,稽南殷旷,永割王赋,长壅国民,竹箭良材,绝望京辇,萑蒲小盗,共肆贪残,念彼馀,兼其慨息。西戎屈膝,自款重关,秦国依风,并输侵地,三边已乂,四表咸宁,唯此微妖,所宜清殄。可遣使持节、都督南徐州诸军事、征北将军、司空、南徐州刺史桂阳郡开国公安都指往擒戮,罪止异身,馀无所问。

异本谓官军自钱塘江而上,安都乃由会稽、诸暨步道袭之。异闻兵至,大恐,弃郡奔于桃支岭,于岭口立栅自固。明年春,安都大破其栅,异与第二子忠臣奔于陈宝应,于是虏其馀党男女数千人。天嘉五年,陈宝应平,并擒异送都,斩于建康市,子侄及同党无少长皆伏诛,唯第三子贞臣以尚主获免。

陈宝应,晋安候官人也。世为闽中四姓。父羽,有材干,为郡雄豪。宝应性反覆,多变诈。梁代晋安数反,累杀郡将,羽初并扇惑合成其事,后复为官军乡导破之,由是一郡兵权皆自己出。

侯景之乱,晋安太守、宾化侯萧云以郡让羽,羽年老,但治郡事,令宝应典兵。是时东境饥馑,会稽尤甚,死者十七八,平民男女,并皆自卖,而晋安独豊沃。宝应自海道寇临安、永嘉及会稽、馀姚、诸暨,又载米粟与之贸易,多致玉帛子女,其有能致舟乘者,亦并奔归之,由是大致赀产,士众强盛。侯景平,元帝因以羽为晋安太守。

高祖辅政,羽请归老,求传郡于宝应,高祖许之。绍泰元年,授壮武将军、晋安太守,寻加员外散骑常侍。二年,封候官县侯,邑五百户。时东西岭路,寇贼拥隔,宝应自海道趋于会稽贡献。高祖受禅,授持节、散骑常侍、信武将军、闽州刺史,领会稽太守。世祖嗣位,进号宣毅将军,又加其父光禄大夫,仍命宗正录其本系,编为宗室,并遣使条其子女,无大小并加封爵。

宝应娶留异女为妻,侯安都之讨异也,宝应遣兵助之,又资周迪兵粮,出寇临川。及都督章昭达于东兴、南城破迪,世祖因命昭达都督众军,由建安南道渡岭,又命益州刺史领信义太守余孝顷都督会稽、东阳、临海、永嘉诸军自东道会之,以讨宝应,并诏宗正绝其属籍。于是尚书下符曰:

告晋安士庶:昔陇西旅拒,汉不稽诛,辽东叛换,魏申宏略。若夫无诸汉之策勋,有扈夏之同姓,至于纳吴濞之子,致横海之师,违姒启之命,有《甘誓》之讨。况乃族不系于宗盟,名无纪于庸器,而显成三叛,飐深四罪者乎?

案闽寇陈宝应父子,卉服支孽,本迷爱敬。梁季丧乱,闽隅阻绝,父既豪侠,扇动蛮陬,椎髻箕坐,自为渠帅,无闻训义,所资奸谄,爰肆蜂豺,俄而解印。炎行方谢,网漏吞舟,日月居诸,弃之度外。自东南王气,实表圣基,斗牛聚星,允符王迹,梯山航海,虽若款诚,擅割瑰珍,竟微职贡。朝廷遵养含弘,宠灵隆赫,起家临郡,兼昼绣之荣,裂地置州,假籓麾之盛。即封户牖,仍邑栎阳,乘华毂者十人,保弊庐而万石。又以盛汉君临,推恩娄敬,隆周朝会,乃长滕侯,由是紫泥青纸,远贲恩泽,乡亭龟组,颁及婴孩。自谷迁乔,孰复为拟,而苞藏鸩毒,敢行狼戾。连结留异,表里周迪,盟歃婚姻,自为脣齿,屈强山谷,推移岁时。及我彀骑防山,定秦望之西部,戈船下濑,克汇泽之南川,遂敢举斧,并助凶孽,莫不应弦摧衄,尽殪丑徒。每以罪在酋渠,悯兹驱逼,所收俘馘,并勒矜放。仍遣中使,爰降诏书,天网恢弘,犹许改思。异既走险,迪又逃刑,诳侮王人,为之川薮,遂使袁熙请席,远叹头行,马援观蛙,犹安井底。至如遏绝九赋,剽掠四民,阖境资财,尽室封夺,凡厥苍头,皆略黔首。蝥贼相扇,叶契连踪,乃复逾超瀛冥,寇扰浃口,侵轶岭峤,掩袭述城,缚掠吏民,焚烧官寺,此而可纵,孰不可容?

今遣沙州刺史俞文冏,明威将军程文季,假节、宣猛将军、成州刺史甘他,假节、云旗将军谭瑱,假节、宣猛将军、前监临海郡陈思庆,前军将军徐智远,明毅将军宜黄县开国侯慧纪,开远将军、新除晋安太守赵彖,持节、通直散骑常侍、壮武将军、定州刺史康乐县开国侯林冯,假节、信威将军、都督东讨诸军事、益州刺史余孝顷,率羽林二万,蒙冲盖海,乘跨沧波,扫荡巢窟。此皆明耻教战,濡须鞠旅,累从杨仆,亟走孙恩,斩蛟中流,命冯夷而鸣鼓,鼋鼍为驾,阑方壶而建旗。

义安太守张绍宾,忠诚款到,累使求军,南康内史裴忌,新除轻车将军刘峰,东衡州刺史钱道戢,并即遣人仗,与绍宾同行。

故司空欧阳公,昔有表奏,请宣薄伐,遥途意合,若伏波之论兵,长逝遗诚,同子颜之勿赦。征南薨谢,上策无忘,周南馀恨,嗣子弗忝。广州刺史欧阳纥,克符家声,聿遵广略,舟师步卒,二万分趋,水扼长鲸,陆制封犭希,董率衡、广之师,会我六军。

潼州刺史李者,明州刺史戴晃,新州刺史区白兽,壮武将军修行师,陈留太守张遂,前安成内史阙慎,前庐陵太守陆子隆,前豫宁太守任蛮奴,巴山太守黄法慈,戎昭将军、湘东公世子徐敬成,吴州刺史鲁广达,前吴州刺史遂兴县开国侯详,使持节、都督征讨诸军事、散骑常侍、护军将军昭达,率缇骑五千,组甲二万,直渡邵武,仍顿晋安。按辔扬旌,夷山堙谷,指期掎角,以制飞走。

前宣城太守钱肃,临川太守骆牙,太子左卫率孙诩,寻阳太守莫景隆,豫章太守刘广德,并随机镇遏,络驿在路。

使持节、散骑常侍、镇南将军、开府仪同三司、江州刺史新建县开国侯法抃,戒严中流,以为后殿。

斧钺所临,罪唯元恶及留异父子。其党主帅,虽有请泥函谷,相背淮阴,若能翻然改图,因机立效,非止肆眚,仍加赏擢。其建、晋士民,久被驱迫者,大军明加抚慰,各安乐业,流寓失乡,既还本土。其馀立功立事,已具赏格。若执迷不改,同恶趑趄,斧钺一临,罔知所赦。

昭达既克周迪,逾东兴岭,顿于建安,余孝顷又自临海道袭于晋安,宝应据建安之湖际,逆拒王师,水陆为栅。昭达深沟高垒,不与战,但命军士伐木为簰。俄而水盛,乘流放之,突其水栅,仍水步薄之,宝应众溃,身奔山草间,窘而就执,并其子弟二十人送都,斩于建康市。

史臣曰:梁末之灾沴,群凶竞起,郡邑岩穴之长,村屯邬壁之豪,资剽掠以致强,恣陵侮而为大。高祖应期拨乱,戡定安辑,熊昙朗、周迪、留异、陈宝应虽身逢兴运,犹志在乱常。昙朗奸慝翻覆,夷灭斯为幸矣。宝应及异,世祖或敦以婚姻,或处其类族,岂有不能威制,盖以德怀也。遂乃背恩负义,各立异图,地匪淮南,有为帝之志,势非庸、蜀,启自王之心。呜呼,既其迷暗所致,五宗屠剿,宜哉!

\hypertarget{header-n5193}{%
\subsubsection{卷三十}\label{header-n5193}}

始兴王叔陵新安王伯固

始兴王叔陵,字子嵩,高宗之第二子也。梁承圣中,高宗在江陵为直阁将军,而叔陵生焉。江陵陷,高宗迁关右,叔陵留于穰城。高宗之还也,以后主及叔陵为质。天嘉三年,随后主还朝,封康乐侯,邑五百户。

叔陵少机辩,徇声名,强梁无所推屈。光大元年,除中书侍郎。二年,出为持节、都督江州诸军事、南中郎将、江州刺史。太建元年,封始兴郡王,奉昭烈王祀。进授使持节、都督江、郢、晋三州诸军事、军师将军,刺史如故。叔陵时年十六,政自己出,僚佐莫预焉。性严刻,部下慑惮。诸公子侄及罢县令长,皆逼令事己。豫章内史钱法成诣府进谒,即配其子季卿将领马仗,季卿惭耻,不时至,叔陵大怒,侵辱法成,法成愤怨自缢而死。州县非其部内,亦征摄案治之,朝贵及下吏有乖忤者,辄诬奏其罪,陷以重辟。寻进号云麾将军,加散骑常侍。三年,加侍中。四年,迁都督湘、衡、桂、武四州诸军事、平南将军、湘州刺史,侍中、使持节如故。诸州镇闻其至,皆震恐股栗。叔陵日益暴横,征伐夷獠,所得皆入己,丝毫不以赏赐。征求役使,无有纪极。夜常不卧,烧烛达晓,呼召宾客,说民间细事,戏谑无所不为。性不饮酒,唯多置肴脔,昼夜食啖而已。自旦至中,方始寝寐。其曹局文案,非呼不得辄自呈。笞罪者皆系狱,动数年不省视。潇湘以南,皆逼为左右,廛里殆无遗者。其中脱有逃窜,辄杀其妻子。州县无敢上言,高宗弗之知也。寻进号镇南将军,给鼓吹一部,迁中卫将军。九年,除使持节、都督扬、徐、东扬、南豫四州诸军事、扬州刺史,侍中、将军、鼓吹如故。

十年,至都,加扶,给油幢车。叔陵治在东府,事务多关涉省阁,执事之司,承意顺旨,即讽上进用之,微致违忤,必抵以大罪,重者至殊死,道路籍籍,皆言其有非常志。叔陵修饰虚名,每入朝,常于车中马上执卷读书,高声长诵,阳阳自若。归坐斋中,或自执斧斤为沐猴百戏。又好游冢墓间,遇有茔表主名可知者,辄令左右发掘,取其石志古器,并骸骨肘胫,持为玩弄,藏之库中。府内民间少妻处女,微有色貌者,并即逼纳。

十一年,丁所生母彭氏忧去职。顷之,起为中卫将军,使持节、都督、刺史如故。晋世王公贵人,多葬梅岭,及彭卒,叔陵启求于梅岭葬之,乃发故太傅谢安旧墓,弃去安柩,以葬其母。初丧之日,伪为哀毁,自称刺血写《涅槃经》,未及十日,乃令庖厨击鲜,日进甘膳。又私召左右妻女,与之奸合,所作尤不轨,侵淫上闻。高宗谴责御史中丞王政,以不举奏免政官,又黜其典签亲事,仍加鞭捶。高宗素爱叔陵,不绳之以法,但责让而已。服阕,又为侍中、中军大将军。

及高宗不豫,太子诸王并入侍疾。高宗崩于宣福殿,翌日旦,后主哀顿俯伏,叔陵以剉药刀斫后主中项。太后驰来救焉,叔陵又斫太后数下。后主乳媪吴氏,时在太后侧,自后掣其肘,后主因得起。叔陵仍持后主衣,后主自奋得免。长沙王叔坚手搤叔陵,夺去其刀,仍牵就柱,以其褶袖缚之。时吴媪已扶后主避贼,叔坚求后主所在,将受命焉。叔陵因奋袖得脱,突走出云龙门,驰车还东府,呼其甲士,散金银以赏赐,外召诸王将帅,莫有应者,唯新安王伯固闻而赴之。

叔陵聚兵仅千人,初欲据城保守,俄而右卫将军萧摩诃将兵至府西门,叔陵事急惶恐,乃遣记室韦谅送其鼓吹与摩诃,仍谓之曰:``如其事捷,必以公为台鼎。''摩诃绐报之,曰``须王心膂节将自来,方敢从命''。叔陵即遣戴温、谭骐驎二人诣摩诃所,摩诃执以送台,斩于阁道下。叔陵自知不济,遂入内沈其妃张氏及宠妾七人于井中。叔陵有部下兵先在新林,于是率人马数百,自小航渡,欲趋新林,以舟舰入北。行至白杨路,为台军所邀,伯固见兵至,旋避入巷,叔陵驰骑拔刃追之,伯固复还。叔陵部下,多弃甲溃散,摩诃马容陈智深迎刺叔陵,僵毙于地,阉竖王飞禽抽刀斫之十数下,马容陈仲华就斩其首,送于台。自寅至巳乃定。

尚书八座奏曰:``逆贼故侍中、中军大将军、始兴王叔陵,幼而很戾,长肆贪虐。出抚湘南,及镇九水,两籓庶,扫地无遗。蜂目豺声,狎近轻薄,不孝不仁,阻兵安忍,无礼无义,唯戮是闻。及居偏忧,淫乐自恣,产子就馆,日月相接。昼伏夜游,恒习奸诡,抄掠居民,历发丘墓。谢太傅晋朝佐命,草创江左,斫棺露骸,事惊听视。自大行皇帝寝疾,翌日未瘳,叔陵以贵介之地,参侍医药,外无戚容,内怀逆弑。大渐之后,圣躬号擗,遂因匍匐,手犯乘舆。皇太后奉临,又加锋刃,穷凶极逆,旷古未俦。赖长沙王叔坚诚孝恳至,英果奋发,手加挫拉,身蔽圣躬。叔陵仍奔东城,招集凶党,馀毒方炽,自害妻孥。虽应时枭悬,犹未摅愤怨,臣等参议,请依宋代故事,流尸中江,污潴其室,并毁其所生彭氏坟庙,还谢氏之茔。''制曰:``凶逆枭獍,反噬宫闱,赖宗庙之灵,时从殄灭。抚情语事,酸愤兼怀,朝议有章,宜从所奏也。''

叔陵诸子,即日并赐死。前衡阳内史彭暠谘议参军兼记室郑信、中录事参军兼记室韦谅、典签俞公喜,并伏诛。暠,叔陵舅也,初随高宗在关中,颇有勤效,因藉叔陵将领历阳、衡阳二郡。信以便书记,有宠,谋谟皆预焉。谅,京兆人,梁侍中、护军将军粲之子也,以学业为叔陵所引。

陈智深以诛叔陵之功为巴陵内史,封游安县子。陈仲华为下巂太守,封新夷县子。王飞禽除伏波将军。赐金各有差。

新安王伯固,字牢之,世祖之第五子也。生而龟胸,目通精扬白,形状眇小,而俊辩善言论。天嘉六年,立为新安郡王,邑二千户。废帝嗣立,为使持节、都督南琅邪、彭城、东海三郡诸军事、云麾将军、彭城、琅邪二郡太守。寻入为丹阳尹,将军如故。

太建元年,进号智武将军,尹如故。秩满,进号翊右将军。寻授使持节、都督吴兴诸军事、平东将军、吴兴太守。四年,入为侍中、翊前将军,迁安前将军、中领军。七年,出为使持节、散骑常侍、都督南徐、南豫、南、北兗四州诸军事、镇北将军、南徐州刺史。伯固性嗜酒,而不好积聚,所得禄俸,用度无节。酣醉以后,多所乞丐,于诸王之中,最为贫窭。高宗每矜之,特加赏赐。伯固雅性轻率,好行鞭捶,在州不知政事,日出田猎,或乘眠轝至于草间,辄呼民下从游,动至旬日,所捕麞鹿,多使生致。高宗颇知之,遣使责让者数矣。

十年,入朝,又为侍中、镇右将军,寻除护军将军。其年,为国子祭酒,领左骁骑将军,侍中、镇右并如故。伯固颇知玄理,而堕业无所通,至于擿句问难,往往有奇意。为政严苛,国学有堕游不修习者,重加槚楚,生徒惧焉,由是学业颇进。

十二年,领宗正卿。十三年,为使持节、都督扬、南徐、东扬、南豫四州诸军事、扬州刺史,侍中、将军如故。斋

后主初在东宫,与伯固甚相亲狎,伯固又善嘲谑,高宗每宴集,多引之。叔陵在江州,心害其宠,阴求疵瑕,将中之以法。及叔陵入朝,伯固惧罪,谄求其意,乃共讪毁朝贤,历诋文武,虽耆年高位,皆面折之,无所畏忌。伯因性好射雉,叔陵又好开发冢墓,出游野外,必与偕行,于是情好大叶,遂谋不轨。伯固侍禁中,每有密语,必报叔陵。及叔陵出奔东府,遣使告之,伯固单马驰赴,助叔陵指挥。知事不捷,便欲遁走,会四门已闭不得出,因同趣白扬道。台马容至,为乱兵所杀,尸于东昌馆门,时年二十八。诏曰:``伯固同兹悖逆,殒身途路。今依外议,意犹弗忍,可特许以庶人礼葬。''又诏曰:``伯固随同巨逆,自绝于天,俾无遗育,抑有恒典。但童孺靡识,兼预葭莩,置之甸人,良以恻悯,及伯固所生王氏,可并特宥为庶人。''国除。

史臣曰:孔子称``富与贵,是人之所欲,非其道得之,不处也''。上自帝王,至于黎献,莫不嫡庶有差,长幼攸序。叔陵险躁奔竞,遂行悖逆,辕袴形骸,未臻其罪,污潴居处,不足彰过,悲哉!

\end{document}
