\PassOptionsToPackage{unicode=true}{hyperref} % options for packages loaded elsewhere
\PassOptionsToPackage{hyphens}{url}
%
\documentclass[]{article}
\usepackage{lmodern}
\usepackage{amssymb,amsmath}
\usepackage{ifxetex,ifluatex}
\usepackage{fixltx2e} % provides \textsubscript
\ifnum 0\ifxetex 1\fi\ifluatex 1\fi=0 % if pdftex
  \usepackage[T1]{fontenc}
  \usepackage[utf8]{inputenc}
  \usepackage{textcomp} % provides euro and other symbols
\else % if luatex or xelatex
  \usepackage{unicode-math}
  \defaultfontfeatures{Ligatures=TeX,Scale=MatchLowercase}
\fi
% use upquote if available, for straight quotes in verbatim environments
\IfFileExists{upquote.sty}{\usepackage{upquote}}{}
% use microtype if available
\IfFileExists{microtype.sty}{%
\usepackage[]{microtype}
\UseMicrotypeSet[protrusion]{basicmath} % disable protrusion for tt fonts
}{}
\IfFileExists{parskip.sty}{%
\usepackage{parskip}
}{% else
\setlength{\parindent}{0pt}
\setlength{\parskip}{6pt plus 2pt minus 1pt}
}
\usepackage{hyperref}
\hypersetup{
            pdfborder={0 0 0},
            breaklinks=true}
\urlstyle{same}  % don't use monospace font for urls
\setlength{\emergencystretch}{3em}  % prevent overfull lines
\providecommand{\tightlist}{%
  \setlength{\itemsep}{0pt}\setlength{\parskip}{0pt}}
\setcounter{secnumdepth}{0}
% Redefines (sub)paragraphs to behave more like sections
\ifx\paragraph\undefined\else
\let\oldparagraph\paragraph
\renewcommand{\paragraph}[1]{\oldparagraph{#1}\mbox{}}
\fi
\ifx\subparagraph\undefined\else
\let\oldsubparagraph\subparagraph
\renewcommand{\subparagraph}[1]{\oldsubparagraph{#1}\mbox{}}
\fi

% set default figure placement to htbp
\makeatletter
\def\fps@figure{htbp}
\makeatother


\date{}

\begin{document}

\hypertarget{header-n0}{%
\section{陶庵梦忆}\label{header-n0}}

\begin{center}\rule{0.5\linewidth}{\linethickness}\end{center}

\tableofcontents

\begin{center}\rule{0.5\linewidth}{\linethickness}\end{center}

\hypertarget{header-n6}{%
\subsection{序}\label{header-n6}}

\hypertarget{header-n8}{%
\subsubsection{序}\label{header-n8}}

陶庵国破家亡,无所归止,披发入山,駴駴为野人。故旧见之,如毒药猛兽,愕窒不敢与接。作自挽诗,每欲引决。因《石匮书》未成,尚视息人世。然瓶粟屡罄,不能举火,始知首阳二老直头饿死,不食周粟,还是后人妆点语也。饥饿之余,好弄笔墨,因思昔人生长王、谢,颇事豪华,今日罹此果报。以笠报颅,以篑报踵,仇簪履也;以衲报裘,以苎报絺,仇轻暖也;以藿报肉,以粝报粻,仇甘旨也;以荐报床,以石报枕,仇温柔也;以绳报枢,以瓮报牖,仇爽垲也;以烟报目,以粪报鼻,仇香艳也;以途报足,以囊报肩,仇舆从也。种种罪案,从种种果报中见之。鸡鸣枕上,夜气方回,因想余生平,繁华靡丽,过眼皆空,五十年来,总成一梦。今当黍熟黄粱,车旅蚁穴,当作如何消受?遥思往事,忆即书之,持向佛前,一一忏悔。不次岁月,异年谱也;不分门类,别志林也。偶拈一则,如游旧径,如见故人,城郭人民,翻用自喜,真所谓痴人前不得说梦矣。昔有西陵脚夫为人担酒,失足破其瓮,念无所偿,痴坐伫想曰:``得是梦便好!''一寒士乡试中式,方赴鹿鸣宴,恍然犹意非真,自啮其臂曰:``莫是梦否?''一梦耳,惟恐其非梦,又惟恐其是梦,其为痴人则一也。余今大梦将寤,犹事雕虫,又是一番梦呓。因叹慧业文人,名心难化,正如邯郸梦断,漏尽钟鸣,卢生遗表,犹思摹拓二王,以流传后世。则其名根一点,坚固如佛家舍利,劫火猛烈,犹烧之不失也。

\hypertarget{header-n12}{%
\subsection{卷一}\label{header-n12}}

\hypertarget{header-n14}{%
\subsubsection{钟山}\label{header-n14}}

钟山上有云气,浮浮冉冉,红紫间之,人言王气,龙蜕藏焉。高皇帝与刘诚意、徐中山、汤东瓯定寝穴,各志其处,藏袖中。三人合,穴遂定。门左有孙权墓,请徙。太祖曰:``孙权亦是好汉子,留他守门。''及开藏,下为梁志公和尚塔。

真身不坏,指爪绕身数匝。军士辇之,不起。太祖亲礼之,许以金棺银椁,庄田三百六十,奉香火,舁灵谷寺塔之。今寺僧数千人,日食一庄田焉。陵寝定,闭外羡,人不及知。所见者,门三、飨殿一、寝殿一,后山苍莽而已。壬午七月,朱兆宣簿太常,中元祭期,岱观之。飨殿深穆,暖阁去殿三尺,黄龙幔幔之。列二交椅,褥以黄锦,孔雀翎织正面龙,甚华重。席地以毡,走其上必去舄轻趾。稍咳,内侍辄叱曰:``莫惊驾!''

近阁下一座,稍前,为碽妃,是成祖生母。成祖生,孝慈皇后妊为己子,事甚秘。再下,东西列四十六席,或坐或否。祭品极简陋。朱红木簋、木壶、木酒樽,甚粗朴。簋中肉止三片,粉一铗,黍数粒,东瓜汤一瓯而已。暖阁上一几,陈铜炉一、小筯瓶二、杯棬二;下一大几,陈太牢一、少牢一而已。他祭或不同,岱所见如是。先祭一日,太常官属开牺牲所中门,导以鼓乐旗帜,牛羊自出,龙袱盖之。至宰割所,以四索缚牛蹄。太常官属至,牛正面立,太常官属朝牲揖,揖未起,而牛头已入燖所。燖已,舁至飨殿。次日五鼓,魏国至,主祀,太常官属不随班,侍立飨殿上。祀毕,牛羊已臭腐不堪闻矣。平常日进二膳,亦魏国陪祀,日必至云。

戊寅,岱寓鹫峰寺。有言孝陵上黑气一股,冲入牛斗,百有余日矣。岱夜起视,见之。自是流贼猖獗,处处告警。壬午,朱成国与王应华奉敕修陵,木枯三百年者尽出为薪,发根,隧其下数丈,识者为伤地脉、泄王气,今果有甲申之变,则寸斩应华亦不足赎也。孝陵玉石二百八十二年,今岁清明,乃遂不得一盂麦饭,思之猿咽。

\hypertarget{header-n22}{%
\subsubsection{报恩塔}\label{header-n22}}

中国之大古董,永乐之大窑器,则报恩塔是也。报恩塔成于永乐初年,非成祖开国之精神、开国之物力、开国之功令,其胆智才略足以吞吐此塔者,不能成焉。塔上下金刚佛像千百亿金身。一金身,琉璃砖十数块凑砌成之,其衣折不爽分,其面目不爽毫,其须眉不爽忽,斗笋合缝,信属鬼工。

闻烧成时,具三塔相,成其一,埋其二,编号识之。今塔上损砖一块,以字号报工部,发一砖补之,如生成焉。夜必灯,岁费油若干斛。天日高霁,霏霏霭霭,摇摇曳曳,有光怪出其上,如香烟燎绕,半日方散。永乐时,海外夷蛮重译至者百有余国,见报恩塔必顶礼赞叹而去,谓四大部洲所无也。

\hypertarget{header-n28}{%
\subsubsection{天台牡丹}\label{header-n28}}

天台多牡丹,大如拱把,其常也。某村中有鹅黄牡丹,一株三干,其大如小斗,植五圣祠前。枝叶离披,错出檐甃之上,三间满焉。花时数十朵,鹅子、黄鹂、松花、蒸栗,萼楼穰吐,淋漓簇沓。土人于其外搭棚演戏四五台,婆娑乐神。

有侵花至漂发者,立致奇祟。土人戒勿犯,故花得蔽芾而寿。

\hypertarget{header-n34}{%
\subsubsection{金乳生草花}\label{header-n34}}

金乳生喜莳草花。住宅前有空地,小河界之。乳牛濒河构小轩三间,纵其趾于北,不方而长,设竹篱经其左。北临街,筑土墙,墙内砌花栏护其趾。再前,又砌石花栏,长丈余而稍狭。栏前以螺山石垒山披数折,有画意。草木百余本,错杂莳之,浓淡疏密,俱有情致。春以罂粟、虞美人为主,而山兰、素馨、决明佐之。春老以芍药为主,而西番莲、土萱、紫兰、山矾佐之。夏以洛阳花、建兰为主,而蜀葵、乌斯菊、望江南、茉莉、杜若、珍珠兰佐之。秋以菊为主,而剪秋纱、秋葵、僧鞋菊、万寿芙蓉、老少年、秋海棠、雁来红、矮鸡冠佐之。冬以水仙为主,而长春佐之。其木本如紫白丁香、绿萼、玉碟、蜡梅、西府、滇茶、日丹、白梨花,种之墙头屋角,以遮烈日。乳生弱质多病,早起,不盥不栉,蒲伏阶下,捕菊虎,芟地蚕,花根叶底,虽千百本,一日必一周之。

癃头者火蚁,瘠枝者黑蚰,伤根者蚯蚓、蜒蝣,贼叶者象干、毛猬。火蚁,以鲞骨、鳖甲置旁引出弃之。黑蚰,以麻裹筯头捋出之。蜒蝣,以夜静持灯灭杀之。蚯蚓,以石灰水灌河水解之。毛猬,以马粪水杀之。象干虫,磨铁钱穴搜之。事必亲历,虽冰龟其手,日焦其额,不顾也。青帝喜其勤,近产芝三本,以祥瑞之。

\hypertarget{header-n40}{%
\subsubsection{日月湖}\label{header-n40}}

宁波府城内,近南门,有日月湖。日湖圆,略小,故日之;月湖长,方广,故月之。二湖连络如环,中亘一堤,小桥纽之。日湖有贺少监祠。季真朝服拖绅,绝无黄冠气象。祠中勒唐玄宗《饯行》诗以荣之。季真乞鉴湖归老,年八十余矣。其《回乡》诗曰:``幼小离家老大回,乡音无改鬓毛衰。儿孙相见不相识,笑问客从何处来?''
八十归老,不为早矣,乃时人称为急流勇退,今古传之。季真曾谒一卖药王老,求冲举之术,持一珠贻之。王老见卖饼者过,取珠易饼。季真口不敢言,甚懊惜之。王老曰:``悭吝未除,术何由得!''乃还其珠而去。则季真直一富贵利禄中人耳。《唐书》入之《隐逸传》,亦不伦甚矣。月湖一泓汪洋,明瑟可爱,直抵南城。

城下密密植桃柳,四围湖岸,亦间植名花果木以萦带之。湖中栉比者皆士夫园亭,台榭倾圮,而松石苍老。石上凌霄藤有斗大者,率百年以上物也。四明缙绅,田宅及其子,园亭及其身。平泉木石,多暮楚朝秦,故园亭亦聊且为之,如传舍衙署焉。屠赤水娑罗馆亦仅存娑罗而已。所称``雪浪''等石,在某氏园久矣。清明日,二湖游船甚盛,但桥小船不能大。城墙下趾稍广,桃柳烂漫,游人席地坐,亦饮亦歌,声存西湖一曲。

\hypertarget{header-n46}{%
\subsubsection{金山夜戏}\label{header-n46}}

崇祯二年中秋后一日,余道镇江往兖。日晡,至北固,舣舟江口。月光倒囊入水,江涛吞吐,露气吸之,噀天为白。余大惊喜。移舟过金山寺,已二鼓矣。经龙王堂,入大殿,皆漆静。林下漏月光,疏疏如残雪。余呼小傒携戏具,盛张灯火大殿中,唱韩蕲王金山及长江大战诸剧。锣鼓喧阗,一寺人皆起看。有老僧以手背採眼翳,翕然张口,呵欠与笑嚏俱至。徐定睛,视为何许人,以何事何时至,皆不敢问。剧完,将曙,解缆过江。山僧至山脚,目送久之,不知是人、是怪、是鬼。

\hypertarget{header-n51}{%
\subsubsection{筠芝亭}\label{header-n51}}

筠芝亭,浑朴一亭耳。然而亭之事尽,筠芝亭一山之事亦尽。吾家后此亭而亭者,不及筠芝亭;后此亭而楼者、阁者、斋者,亦不及。总之,多一楼,亭中多一楼之碍;多一墙,亭中多一墙之碍。太仆公造此亭成,亭之外更不增一椽一瓦,亭之内亦不设一槛一扉,此其意有在也。亭前后,太仆公手植树皆合抱,清樾轻岚,滃滃翳翳,如在秋水。亭前石台,躐取亭中之景物而先得之,升高眺远,眼界光明。敬亭诸山,箕踞麓下;溪壑萦回,水出松叶之上。台下右旋,曲磴三折,老松偻背而立,顶垂一干,倒下如小幢,小枝盘郁,曲出辅之,旋盖如曲柄葆羽。癸丑以前,不垣不台,松意尤畅。

\hypertarget{header-n56}{%
\subsubsection{硚园}\label{header-n56}}

硚园,水盘据之,而得水之用,又安顿之若无水者。寿花堂,界以堤,以小眉山,以天问台,以竹径,则曲而长,则水之。内宅,隔以霞爽轩,以酣漱,以长廊,以小曲桥,以东篱,则深而邃,则水之。临池,截以鲈香亭、梅花禅,则静而远,则水之。缘城,护以贞六居,以无漏庵,以菜园,以邻居小户,则閟而安,则水之用尽。而水之意色,指归乎庞公池之水。庞公池,入弃我取,一意向园,目不他瞩,肠不他回,口不他诺,龙山夔蚭,三折就之,而水不之顾。人称硚园能用水,而卒得水力焉。大父在曰,园极华缛。有二老盘旋其中,一老曰:``竟是蓬莱阆苑了也!''一老咈之曰:``个边那有这样!''

葑门荷宕天启壬戌六月二十四日,偶至苏州,见士女倾城而出,毕集于葑门外之荷花宕。楼船画舫至鱼艖小艇,雇觅一空。远方游客,有持数万钱无所得舟,蚁旋岸上者。

余移舟往观,一无所见。宕中以大船为经,小船为纬,游冶子弟,轻舟鼓吹,往来如梭。舟中丽人皆倩妆淡服,摩肩簇舄,汗透重纱。舟楫之胜以挤,鼓吹之胜以集,男女之胜以溷,歊暑燂烁,靡沸终日而已。荷花宕经岁无人迹,是日,士女以鞋靸不至为耻。袁石公曰:``其男女之杂,灿烂之景,不可名状。大约露帏则千花竞笑,举袂则乱云出峡,挥扇则星流月映,闻歌则雷辊涛趋。''盖恨虎丘中秋夜之模糊躲闪,特至是日而明白昭著之也。

\hypertarget{header-n63}{%
\subsubsection{越俗扫墓}\label{header-n63}}

越俗扫墓,男女袨服靓妆,画船箫鼓,如杭州人游湖,厚人薄鬼,率以为常。二十年前,中人之家尚用平水屋帻船,男女分两截坐,不坐船,不鼓吹。先辈谑之曰:``以结上文两节之意。''后渐华靡,虽监门小户,男女必用两坐船,必巾,必鼓吹,必欢呼畅饮。下午必就其路之所近,游庵堂寺院及士夫家花园。鼓吹近城,必吹《海东青》、《独行千里》,锣鼓错杂。酒徒沾醉,必岸帻嚣嚎,唱无字曲,或舟中攘臂,与侪列厮打。自二月朔至夏至,填城溢国,日日如之。乙酉方兵,划江而守,虽鱼艖菱舠,收拾略尽。坟垅数十里而遥,子孙数人挑鱼肉楮钱,徒步往返之,妇女不得出城者三岁矣。

萧索凄凉,亦物极必反之一。

\hypertarget{header-n69}{%
\subsubsection{奔云石}\label{header-n69}}

南屏石,无出奔云右者。奔云得其情,未得其理。石如滇茶一朵,风雨落之,半入泥土,花瓣棱棱,三四层折。人走其中,如蝶入花心,无须不缀也。黄寓庸先生读书其中,四方弟子千余人,门如市。余幼从大父访先生。先生面黧黑,多髭须,毛颊,河目海口,眉棱鼻梁,张口多笑。交际酬酢,八面应之。耳聆客言,目睹来牍,手书回札,口嘱傒奴,杂沓于前,未尝少错。客至,无贵贱,便肉、便饭食之,夜即与同榻。余一书记往,颇秽恶,先生寝食之不异也,余深服之。

丙寅至武林,亭榭倾圮,堂中窀先生遗蜕,不胜人琴之感。余见奔云黝润,色泽不减,谓客曰:``愿假此一室,以石磥门,坐卧其下,可十年不出也。''客曰:``有盗。''余曰:``布衣褐被,身外长物则瓶粟与残书数本而已。王弇州不曰:`盗亦有道也'哉?''

\hypertarget{header-n75}{%
\subsubsection{木犹龙}\label{header-n75}}

木龙出辽海,为风涛漱击,形如巨浪跳蹴,遍体多著波纹,常开平王得之辽东,辇至京。开平第毁,谓木龙炭矣。及发瓦砾,见木龙埋入地数尺,火不及,惊异之,遂呼为龙。不知何缘出易于市,先君子以犀觥十七只售之,进鲁献王,误书``木龙''犯讳,峻辞之,遂留长史署中。先君子弃世,余载归,传为世宝。丁丑诗社,恳名公人赐之名,并赋小言咏之。周墨农字以``木犹龙'',倪鸿宝字以``木寓龙'',祁世培字以``海槎'',王士美字以``槎浪'',张毅儒字以``陆槎'',诗遂盈帙。木龙体肥痴,重千余斤,自辽之京、之兖、之济,由陆。济之杭,由水。杭之江、之萧山、之山阴、之余舍,水陆错。前后费至百金,所易价不与焉。呜呼,木龙可谓遇矣!

余磨其龙脑尺木,勒铭志之,曰:``夜壑风雷,骞槎化石;海立山崩,烟云灭没;谓有龙焉,呼之或出。''又曰:``扰龙张子,尺木书铭;何以似之?秋涛夏云。''

\hypertarget{header-n81}{%
\subsubsection{天砚}\label{header-n81}}

少年视砚,不得砚丑。徽州汪砚伯至,以古款废砚,立得重价,越中藏石俱尽。阅砚多,砚理出。曾托友人秦一生为余觅石,遍城中无有。山阴狱中大盗出一石,璞耳,索银二斤。余适往武林,一生造次不能辨,持示燕客。燕客指石中白眼曰:``黄牙臭口,堪留支桌。''赚一生还盗。燕客夜以三十金攫去。命砚伯制一天砚,上五小星一大星,谱曰``五星拱月''。燕客恐一生见,铲去大小二星,止留三小星。一生知之,大懊恨,向余言。余笑曰:``犹子比儿。''亟往索看。燕客捧出,赤比马肝,酥润如玉,背隐白丝类玛瑙,指螺细篆,面三星坟起如弩眼,着墨无声而墨沉烟起,一生痴瘛,口张而不能翕。燕客属余铭,铭曰:``女娲炼天,不分玉石;鳌血芦灰,烹霞铸日;星河溷扰,参横箕翕。''

\hypertarget{header-n86}{%
\subsubsection{吴中绝技}\label{header-n86}}

吴中绝技:陆子冈之治玉,鲍天成之治犀,周柱之治嵌镶,赵良璧之治梳,朱碧山之治金银,马勋、荷叶李之治扇,张寄修之治琴,范昆白之治三弦子,俱可上下百年保无敌手。

但其良工苦心,亦技艺之能事。至其厚薄深浅,浓淡疏密,适与后世赏鉴家之心力、目力针芥相投,是岂工匠之所能办乎?

盖技也而进乎道矣。

\hypertarget{header-n93}{%
\subsubsection{濮仲谦雕刻}\label{header-n93}}

南京濮仲谦,古貌古心,粥粥若无能者,然其技艺之巧,夺天工焉。其竹器,一帚、一刷,竹寸耳,勾勒数刀,价以两计。然其所以自喜者,又必用竹之盘根错节,以不事刀斧为奇,则是经其手略刮磨之,而遂得重价,真不可解也。仲谦名噪甚,得其一款,物辄腾贵。三山街润泽于仲谦之手者数十人焉,而仲谦赤贫自如也。于友人座间见有佳竹、佳犀,辄自为之。意偶不属,虽势劫之、利啖之,终不可得。

\hypertarget{header-n97}{%
\subsection{卷二}\label{header-n97}}

\hypertarget{header-n99}{%
\subsubsection{孔庙桧}\label{header-n99}}

己巳,至曲阜谒孔庙,买门者门以入。宫墙上有楼耸出,匾曰``梁山伯祝英台读书处'',骇异之。进仪门,看孔子手植桧。桧历周、秦、汉晋几千年,至晋怀帝永嘉三年而枯。枯三百有九年,子孙守之不毁,至隋恭帝义宁元年复生。生五十一年,至唐高宗乾封三年再枯。枯三百七十有四年,至宋仁宗康定元年再荣。至金宣宗贞祐三年罹于兵火,枝叶俱焚,仅存其干,高二丈有奇。后八十一年,元世祖三十一年再发。

至洪武二十二年己巳,发数枝,蓊郁;后十余年又落。摩其干,滑泽坚润,纹皆左纽,扣之作金石声。孔氏子孙恒视其荣枯,以占世运焉。再进一大亭,卧一碑,书``杏坛''二字,党英笔也。亭界一桥,洙、泗水汇此。过桥,入大殿,殿壮丽,宣圣及四配、十哲俱塑像冕旒。案上列铜鼎三、一牺、一象、一辟邪,款制遒古,浑身翡翠,以钉钉案上。阶下竖历代帝王碑记,独元碑高大,用风磨铜赑屭,高丈余。左殿三楹,规模略小,为孔氏家庙。东西两壁,用小木匾书历代帝王祭文。西壁之隅,高皇帝殿焉。庙中凡明朝封号,俱置不用,总以见其大也。孔家人曰:``天下只三家人家:我家与江西张、凤阳朱而已。江西张,道士气;凤阳朱,暴发人家,小家气。''

\hypertarget{header-n105}{%
\subsubsection{孔林}\label{header-n105}}

曲阜出北门五里许,为孔林。紫金城,城之门以楼,楼上见小山一点,正对东南者,峄山也。折而西,有石虎、石羊三四,在榛莽中。过一桥,二水汇,泗水也。享殿后有子贡手植楷。楷大小千余本,鲁人取为材、为棋枰。享殿正对伯鱼墓,圣人葬其子得中气。由伯鱼墓折而右,为宣圣墓。去数丈,案一小山,小山之南为子思墓。数百武之内,父、子、孙三墓在焉。谯周云:``孔子死后,鲁人就冢次而居者百有余家,曰`孔里'。''

《孔丛子》曰:``夫子墓茔方一里,在鲁城北六里泗水上''。诸孔氏封五十余所,人名昭穆,不可复识。

有碑铭三,兽碣俱在。《皇览》曰:``弟子各以四方奇木来植,故多异树不能名。一里之中未尝产棘木、荆草。''紫金城外,环而墓者数千家,三千二百余年,子孙列葬不他徙,从古帝王所不能比隆也。宣圣墓右有小屋三间,匾曰``子贡庐墓处''。盖自兖州至曲阜道上,时官以木坊表识,有曰``齐人归讙处'',有曰``子在川上处'',尚有义理;至泰山顶上,乃勒石曰``孔子小天下处'',则不觉失笑矣。

\hypertarget{header-n112}{%
\subsubsection{燕子矶}\label{header-n112}}

燕子矶,余三过之。水势湁潗,舟人至此,捷捽抒取,钩挽铁缆,蚁附而上。篷窗中见石骨棱层,撑拒水际,不喜而怖,不识岸上有如许境界。戊寅到京后,同吕吉士出观音门,游燕子矶。方晓佛地仙都,当面蹉过之矣。登关王殿,吴头楚尾,是侯用武之地,灵爽赫赫,须眉戟起。缘山走矶上,坐亭子,看江水潎洌,舟下如箭。折而南,走观音阁,度索上之。阁旁僧院,有峭壁千寻,碚礌如铁;大枫数株,蓊以他树,森森冷绿;小楼痴对,便可十年面壁。今僧寮佛阁,故故背之,其心何忍?是年,余归浙,闵老子、王月生送至矶,饮石壁下。

\hypertarget{header-n117}{%
\subsubsection{鲁藩烟火}\label{header-n117}}

兖州鲁藩烟火妙天下。烟火必张灯,鲁藩之灯,灯其殿、灯其壁、灯其楹柱、灯其屏、灯其座、灯其宫扇伞盖。诸王公子、宫娥僚属、队舞乐工,尽收为灯中景物。及放烟火,灯中景物又收为烟火中景物。天下之看灯者,看灯灯外;看烟火者,看烟火烟火外。未有身入灯中、光中、影中、烟中、火中,闪烁变幻,不知其为王宫内之烟火,亦不知其为烟火内之王宫也。殿前搭木架数层,上放``黄蜂出窠''、``撒花盖顶''、``天花喷礴''。四旁珍珠帘八架,架高二丈许,每一帘嵌孝、悌、忠、信、礼、义、廉、耻一大字。每字高丈许,晶映高明。下以五色火漆塑狮、象、橐驼之属百余头,上骑百蛮,手中持象牙、犀角、珊瑚、玉斗诸器,器中实``千丈菊''、``千丈梨''诸火器,兽足蹑以车轮,腹内藏人。旋转其下,百蛮手中瓶花徐发,雁雁行行,且阵且走。移时,百兽口出火,尻亦出火,纵横践踏。端门内外,烟焰蔽天,月不得明,露不得下。看者耳目攫夺,屡欲狂易,恒内手持之。

昔者有一苏州人,自夸其州中灯事之盛,曰:``苏州此时有烟火,亦无处放,放亦不得上。''众曰:``何也?''曰:``此时天上被烟火挤住,无空隙处耳!''人笑其诞。于鲁府观之,殆不诬也。

\hypertarget{header-n123}{%
\subsubsection{朱云崃女戏}\label{header-n123}}

朱云崃教女戏,非教戏也。未教戏先教琴,先教琵琶,先教提琴、弦子、萧、管,鼓吹歌舞,借戏为之,其实不专为戏也。郭汾阳、杨越公、王司徒女乐,当日未必有此。

丝竹错杂,檀板清讴,入妙腠理,唱完以曲白终之,反觉多事矣。西施歌舞,对舞者五人,长袖缓带,绕身若环,曾挠摩地,扶旋猗那,弱如秋药。女官内侍,执扇葆璇盖、金莲宝炬、纨扇宫灯二十余人,光焰荧煌,锦绣纷叠,见者错愕。云老好胜,遇得意处,辄盱目视客;得一赞语,辄走戏房,与诸姬道之,佹出佹入,颇极劳顿。且闻云老多疑忌,诸姬曲房密户,重重封锁,夜犹躬自巡历,诸姬心憎之。

有当御者,辄遁去,互相藏闪,只在曲房,无可觅处,必叱咤而罢。殷殷防护,日夜为劳,是无知老贱自讨苦吃者也,堪为老年好色之戒。

\hypertarget{header-n130}{%
\subsubsection{绍兴琴派}\label{header-n130}}

丙辰,学琴于王侣鹅。绍兴存王明泉派者推侣鹅,学《渔樵回答》、《列子御风》
《碧玉调》、《水龙吟》、《捣衣环珮声》等曲。戊午,学琴于王本吾,半年得二十余曲:《雁落平沙》、《山居吟》、《静观吟》、《清夜坐钟》、《乌夜咏》、《汉宫秋》
《高山流水》、《梅花弄》、《淳化引》、《沧江夜雨》、《庄周梦》,又《胡笳十八拍》、《普庵咒》等小曲十余种。王本吾指法圆静,微带油腔。余得其法,练熟还生,以涩勒出之,遂称合作。同学者,范与兰、尹尔韬、何紫翔、王士美、燕客、平子。与兰、士美、燕客、平子俱不成,紫翔得本吾之八九而微嫩,尔韬得本吾之八九而微迂。余曾与本吾、紫翔、尔韬取琴四张弹之,如出一手,听者駴服。后本吾而来越者,有张慎行、何明台,结实有余而萧散不足,无出本吾上者。

\hypertarget{header-n135}{%
\subsubsection{花石纲遗石}\label{header-n135}}

越中无佳石。董文简斋中一石,磊块正骨,窋咤数孔,疏爽明易,不作灵谲波诡,朱勔花石纲所遗,陆放翁家物也。文简竖之庭除,石后种剔牙松一株,辟咡负剑,与石意相得。文简轩其北,名``独石轩'',石之轩独之无异也。石篑先生读书其中,勒铭志之。大江以南花石纲遗石,以吴门徐清之家一石为石祖。石高丈五,朱勔移舟中,石盘沉太湖底,觅不得,遂不果行。后归乌程董氏,载至中流,船复覆。董氏破资募善入水者取之。先得其盘,诧异之,又溺水取石,石亦旋起。

时人比之延津剑焉。后数十年,遂为徐氏有。再传至清之,以三百金竖之。石连底高二丈许,变幻百出,无可名状。大约如吴无奇游黄山,见一怪石,辄瞋目叫曰:``岂有此理!岂有此理!''

\hypertarget{header-n141}{%
\subsubsection{焦山}\label{header-n141}}

仲叔守瓜州,余借住于园,无事辄登金山寺。风月清爽,二鼓,犹上妙高台,长江之险,遂同沟浍。一日,放舟焦山,山更纡谲可喜。江曲涡山下,水望澄明,渊无潜甲。海猪、海马,投饭起食,驯扰若豢鱼。看水晶殿,寻瘗鹤铭,山无人杂,静若太古。回首瓜州烟火城中,真如隔世。饭饱睡足,新浴而出,走拜焦处士祠。见其轩冕黼黻,夫人列坐,陪臣四,女官四,羽葆云罕,俨然王者。盖土人奉为土谷,以王礼祀之。是犹以杜十姨配伍髭须,千古不能正其非也。处士有灵,不知走向何所?

\hypertarget{header-n146}{%
\subsubsection{表胜庵}\label{header-n146}}

炉峰石屋,为一金和尚结茅守土之地,后住锡柯桥融光寺。大父造表胜庵成,迎和尚还山住持。命余作启,启曰:

``伏以丛林表胜,惭给孤之大地布金;天瓦安禅,冀宝掌自五天飞锡。重来石塔,戒长老特为东坡;悬契松枝,万回师却逢西向。去无作相,住亦随缘。伏惟九里山之精蓝,实是一金师之初地。偶听柯亭之竹笛,留滞人间;久虚石屋之烟霞,应超尘外。譬之孤天之鹤,尚眷旧枝;想彼弥空之云,亦归故岫。况兹胜域,宜兆异人,了住山之夙因,立开堂之新范。

护门容虎,洗钵归龙。茗得先春,仍是寒泉风味;香来破腊,依然茅屋梅花。半月岩似与人猜,请大师试为标指;一片石正堪对语,听生公说到点头。敬藉山灵,愿同石隐。倘静念结远公之社,定不攒眉;若居心如康乐之流,自难开口。立返山中之驾,看回湖上之船,仰望慈悲,俯从大众。''

\hypertarget{header-n153}{%
\subsubsection{梅花书屋}\label{header-n153}}

陔萼楼后老屋倾圮,余筑基四尺,造书屋一大间。旁广耳室如纱幮,设卧榻。前后空地,后墙坛其趾,西瓜瓤大牡丹三株,花出墙上,岁满三百余朵。坛前西府二树,花时积三尺香雪。前四壁稍高,对面砌石台,插太湖石数峰。西溪梅骨古劲,滇茶数茎,妩媚其旁。梅根种西番莲,缠绕如缨络。窗外竹棚,密宝襄盖之。阶下翠草深三尺,秋海棠疏疏杂入。前后明窗,宝襄西府,渐作绿暗。余坐卧其中,非高流佳客,不得辄入。慕倪迂``清閟'',又以``云林秘阁''名之。

\hypertarget{header-n158}{%
\subsubsection{不二斋}\label{header-n158}}

不二斋,高梧三丈,翠樾千重,墙西稍空,蜡梅补之,但有绿天,暑气不到。后窗墙高于槛,方竹数竿,潇潇洒洒,郑子昭``满耳秋声''横披一幅。天光下射,望空视之,晶沁如玻璃、云母,坐者恒在清凉世界。图书四壁,充栋连床;鼎彝尊罍,不移而具。余于左设石床竹几,帷之纱幕,以障蚊虻;绿暗侵纱,照面成碧。夏日,建兰、茉莉,芗泽浸人,沁入衣裾。重阳前后,移菊北窗下,菊盆五层,高下列之,颜色空明,天光晶映,如沉秋水。冬则梧叶落,蜡梅开,暖日晒窗,红炉毾氍。以昆山石种水仙,列阶趾。春时,四壁下皆山兰,槛前芍药半亩,多有异本。余解衣盘礴,寒暑未尝轻出,思之如在隔世。

\hypertarget{header-n163}{%
\subsubsection{砂罐锡注}\label{header-n163}}

宜兴罐,以龚春为上,时大彬次之,陈用卿又次之。锡注,以王元吉为上,归懋德次之。夫砂罐,砂也;锡注,锡也。器方脱手,而一罐一注价五六金,则是砂与锡与价,其轻重正相等焉,岂非怪事!一砂罐、一锡注,直跻之商彝、周鼎之列而毫无惭色,则是其品地也。

\hypertarget{header-n168}{%
\subsubsection{沈梅冈}\label{header-n168}}

沈梅冈先生许相嵩,在狱十八年。读书之暇,旁攻匠艺,无斧锯,以片铁日夕磨之,遂铦利。得香楠尺许,琢为文具一,大匣三、小匣七、壁锁二;棕竹数片,为箑一,为骨十八,以笋、以缝、以键,坚密肉好,巧匠谢不能事。夫人丐先文恭志公墓,持以为贽,文恭拜受之。铭其匣曰:``十九年,中郎节,十八年,给谏匣;节邪匣邪同一辙。''铭其箑曰:``塞外毡,饥可餐;狱中箑,尘莫干;前苏后沈名班班。''梅冈制,文恭铭,徐文长书,张应尧镌,人称四绝,余珍藏之。

又闻其以粥炼土,凡数年,范为铜鼓者二,声闻里许,胜暹罗铜。

\hypertarget{header-n174}{%
\subsubsection{岣嵝山房}\label{header-n174}}

岣嵝山房,逼山、逼溪、逼韬光路,故无径不梁,无屋不阁。门外苍松傲睨,蓊以杂木,冷绿万顷,人面俱失。石桥低磴,可坐十人。寺僧刳竹引泉,桥下交交牙牙,皆为竹节。天启甲子,余键户其中者七阅月,耳饱溪声,目饱清樾。

山上下多西栗、边笋,甘芳无比。邻人以山房为市,蓏果、羽族日致之,而独无鱼。乃潴溪为壑,系巨鱼数十头。有客至,辄取鱼给鲜。日晡,必步冷泉亭、包园、飞来峰。一日,缘溪走看佛像,口口骂杨髡。见一波斯坐龙象,蛮女四五献花果,皆裸形,勒石志之,乃真伽像也。余椎落其首,并碎诸蛮女,置溺溲处以报之。寺僧以余为椎佛也,咄咄作怪事,及知为杨髡,皆欢喜赞叹。

\hypertarget{header-n180}{%
\subsubsection{三世藏书}\label{header-n180}}

余家三世积书三万余卷。大父诏余曰:``诸孙中惟尔好书,尔要看者,随意携去''余简太仆、文恭大父丹铅所及有手泽者存焉,汇以请,大父喜,命舁去,约二千余卷。天启乙丑,大父去世,余适往武林,父叔及诸弟、门客、匠指、臧获、巢婢辈乱取之,三代遗书一日尽失。余自垂髫聚书四十年,不下三万卷。乙酉避兵入剡,略携数簏随行,而所存者,为方兵所据,日裂以吹烟,并舁至江干,籍甲内,挡箭弹,四十年所积,亦一日尽失。此吾家书运,亦复谁尤!余因叹古今藏书之富,无过隋、唐。隋嘉则殿分三品,有红琉璃、绀琉璃、漆轴之异。殿垂锦幔,绕刻飞仙。帝幸书室,践暗机,则飞仙收幔而上,橱扉自启;帝出,闭如初。隋之书计三十七万卷。唐迁内库书于东宫丽正殿,置修文、著作两院学士,得通籍出入。太府月给蜀都麻纸五千番,季给上谷墨三百三十六丸,岁给河间、景城、清河、博平四郡兔千五百皮为笔,以甲、乙、丙、丁为次。唐之书计二十万八千卷。我明中秘书不可胜计,即《永乐大典》一书,亦堆积数库焉。余书直九牛一毛耳,何足数哉!

\hypertarget{header-n184}{%
\subsection{卷三}\label{header-n184}}

\hypertarget{header-n186}{%
\subsubsection{丝社}\label{header-n186}}

越中琴客不满五六人,经年不事操缦,琴安得佳?余结丝社,月必三会之。有小檄曰:``中郎音癖,《清溪弄》三载乃成;贺令神交,《广陵散》千年不绝。器由神以合道,人易学而难精。幸生岩壑之乡,共志丝桐之雅。清泉磐石,援琴歌《水仙》之操,便足怡情;涧响松风,三者皆自然之声,正须类聚。偕我同志,爱立琴盟,约有常期,宁虚芳日。杂丝和竹,用以鼓吹清音;动操鸣弦,自令众山皆响。非关匣里,不在指头,东坡老方是解人;但识琴中,无劳弦上,元亮辈正堪佳侣。既调商角,翻信肉不如丝;谐畅风神,雅羡心生于手。从容秘玩,莫令解秽于花奴;抑按盘桓,敢谓倦生于古乐。共怜同调之友声,用振丝坛之盛举。''

\hypertarget{header-n191}{%
\subsubsection{南镇祈梦}\label{header-n191}}

万历壬子,余年十六,祈梦于南镇梦神之前,因作疏曰:

``爰自混沌谱中,别开天地;华胥国里,早见春秋。梦两楹,梦赤舄,至人不无;梦蕉鹿,梦轩冕,痴人敢说。惟其无想无因,未尝梦乘车入鼠穴,捣齑啖铁杵;非其先知先觉,何以将得位梦棺器,得财梦秽矢,正在恍惚之交,俨若神明之赐?某也躨跜偃潴,轩翥樊笼,顾影自怜,将谁以告?为人所玩,吾何以堪!一鸣惊人,赤壁鹤耶?局促辕下,南柯蚁耶?得时则驾,渭水熊耶?半榻蘧除,漆园蝶耶?神其诏我,或寝或吪;我得先知,何从何去。择此一阳之始,以祈六梦之正。功名志急,欲搔首而问天;祈祷心坚,故举头以抢地。

轩辕氏圆梦鼎湖,已知一字而有一验;李卫公上书西岳,可云三问而三不灵。肃此以闻,惟神垂鉴。''

\hypertarget{header-n198}{%
\subsubsection{禊泉}\label{header-n198}}

惠山泉不渡钱塘,西兴脚子挑水过江,喃喃作怪事。有缙绅先生造大父,饮茗大佳,问曰:``何地水?''大父曰:``惠泉水。''缙绅先生顾其价曰:``我家逼近卫前,而不知打水吃,切记之。''董日铸先生常曰:``浓、热、满三字尽茶理,陆羽《经》可烧也''两先生之言,足见绍兴人之村之朴。余不能饮潟卤,又无力递惠山水。甲寅夏,过斑竹庵,取水啜之,磷磷有圭角,异之。走看其色,如秋月霜空,噀天为白;又如轻岚出岫,缭松迷石,淡淡欲散。余仓卒见井口有字划,用帚刷之,``禊泉''字出,书法大似右军,益异之。试茶,茶香发。新汲少有石腥,宿三日气方尽。辨禊泉者无他法,取水入口,第桥舌舐腭,过颊即空,若无水可咽者,是为禊泉。好事者信之。汲日至,或取以酿酒,或开禊泉茶馆,或瓮而卖,及馈送有司。董方伯守越,饮其水,甘之,恐不给,封锁禊泉,禊泉名日益重。会稽陶溪、萧山北干、杭州虎跑,皆非其伍,惠山差堪伯仲。在蠡城,惠泉亦劳而微热,此方鲜磊,亦胜一筹矣。长年卤莽,水递不至其地,易他水,余笞之,詈同伴,谓发其私。及余辨是某地某井水,方信服。昔人水辨淄、渑,侈为异事。诸水到口,实实易辨,何待易牙?余友赵介臣亦不余信,同事久,别余去,曰:``家下水实行口不得,须还我口去。''

\hypertarget{header-n203}{%
\subsubsection{兰雪茶}\label{header-n203}}

日铸者,越王铸剑地也。茶味棱棱,有金石之气。欧阳永叔曰:``两浙之茶,日铸第一。''王龟龄曰:``龙山瑞草,日铸雪芽。''日铸名起此。京师茶客,有茶则至,意不在雪芽也。

而雪芽利之,一如京茶式,不敢独异。三峨叔知松萝焙法,取瑞草试之,香扑冽。余曰:``瑞草固佳,汉武帝食露盘,无补多欲;日铸茶薮,`牛虽瘠愤于豚上'也。''遂募歙人入日铸。

扚法、掐法、挪法、撒法、扇法、炒法、焙法、藏法,一如松萝。他泉瀹之,香气不出,煮禊泉,投以小罐,则香太浓郁。杂入茉莉,再三较量,用敞口瓷瓯淡放之,候其冷;以旋滚汤冲泻之,色如竹箨方解,绿粉初匀;又如山窗初曙,透纸黎光。取清妃白,倾向素瓷,真如百茎素兰同雪涛并泻也。

雪芽得其色矣,未得其气,余戏呼之``兰雪''。四五年后,``兰雪茶''一哄如市焉。越之好事者不食松萝,止食兰雪。兰雪则食,以松萝而纂兰雪者亦食,盖松萝贬声价俯就兰雪,从俗也。乃近日徽歙间松萝亦名兰雪,向以松萝名者,封面系换,则又奇矣。

\hypertarget{header-n211}{%
\subsubsection{白洋湖}\label{header-n211}}

故事三江看潮,实无潮看。午后喧传曰:``今年暗涨潮。''

岁岁如之。
庚辰八月,吊朱恒岳少师,至白洋,陈章侯、祁世培同席。海塘上呼看潮,余遄往,章侯、世培踵至。立塘上,见潮头一线,从海宁而来,直奔塘上。稍近,则隐隐露白,如驱千百群小鹅,擘翼惊飞。渐近喷沫,冰花蹴起,如百万雪狮蔽江而下,怒雷鞭之,万首镞镞,无敢后先。再近,则飓风逼之,势欲拍岸而上。看者辟易,走避塘下。潮到塘,尽力一礴,水击射,溅起数丈,著面皆湿。旋卷而右,龟山一挡,轰怒非常,炮碎龙湫,半空雪舞。看之惊眩,坐半日,颜始定。先辈言:浙江潮头自龛、赭两山漱激而起。白洋在两山外,潮头更大,何耶?

\hypertarget{header-n217}{%
\subsubsection{阳和泉}\label{header-n217}}

禊泉出城中,水递者日至。臧获到庵借炊,索薪、索菜、索米,后索酒、索肉;无酒肉,辄挥老拳。僧苦之。无计脱此苦,乃罪泉,投之刍秽。不已,乃决沟水败泉,泉大坏。张子知之,至禊井,命长年浚之。及半,见竹管积其下,皆黧胀作气;竹尽,见刍秽,又作奇臭。张子淘洗数次,俟泉至,泉实不坏,又甘冽。张子去,僧又坏之。不旋踵,至再、至三,卒不能救,禊泉竟坏矣。是时,食之而知其坏者半,食之不知其坏、而仍食之者半,食之知其坏而无泉可食、不得已而仍食之者半。壬申,有称阳和岭玉带泉者,张子试之,空灵不及禊而清冽过之。特以玉带名不雅驯。张子谓:阳和岭实为余家祖墓,诞生我文恭,遗风余烈,与山水俱长。昔孤山泉出,东坡名之``六一'',今此泉名之``阳和'',至当不易。

盖生岭、生泉,俱在生文恭之前,不待文恭而天固已阳和之矣,夫复何疑!土人有好事者,恐玉带失其姓,遂勒石署之。

且曰:``自张志`禊泉'而`禊泉'为张氏有,今琶山是其祖垄,擅之益易。立石署之,惧其夺也。''时有传其语者,阳和泉之名益著。铭曰:``有山如砺,有泉如砥;太史遗烈,落落磊磊。孤屿溢流,`六一'擅之。千年巴蜀,实繁其齿;但言眉山,自属苏氏。''

\hypertarget{header-n224}{%
\subsubsection{闵老子茶}\label{header-n224}}

周墨农向余道闵汶水茶不置口。戊寅九月至留都,抵岸,即访闵汶水于桃叶渡。日晡,汶水他出,迟其归,乃婆娑一老。方叙话,遽起曰:``杖忘某所。''又去。余曰:``今日岂可空去?''迟之又久,汶水返,更定矣。睨余曰:``客尚在耶!客在奚为者?''余曰:``慕汶老久,今日不畅饮汶老茶,决不去。''汶水喜,自起当炉。茶旋煮,速如风雨。导至一室,明窗净儿,荆溪壶、成宣窑磁瓯十余种,皆精绝。灯下视茶色,与磁瓯无别,而香气逼人,余叫绝。余问汶水曰:``此茶何产?''汶水曰:``阆苑茶也。''余再啜之,曰:``莫绐余!是阆苑制法,而味不似。''汶水匿笑曰:``客知是何产?''余再啜之,曰:``何其似罗岕甚也?''汶水吐舌曰:``奇,奇!''余问:``水何水?''曰:``惠泉。''余又曰:``莫绐余!惠泉走千里,水劳而圭角不动,何也?''汶水曰:``不复敢隐。其取惠水,必淘井,静夜候新泉至,旋汲之。山石磊磊藉瓮底,舟非风则勿行,放水之生磊。即寻常惠水犹逊一头地,况他水耶!''又吐舌曰:``奇,奇!''言未毕,汶水去。少顷,持一壶满斟余曰:``客啜此。''余曰:``香扑烈,味甚浑厚,此春茶耶?向瀹者的是秋采。''汶水大笑曰:``予年七十,精赏鉴者,无客比。''遂定交。

\hypertarget{header-n229}{%
\subsubsection{龙喷池}\label{header-n229}}

卧龙骧首于耶溪,大池百仞出其颔下。六十年内,陵谷迁徙,水道分裂。崇祯己卯,余请太守檄,捐金紏众,参锸千人,毁屋三十余间,开土壤二十余亩,辟除瓦砾刍秽千有余艘,伏道蜿蜓,偃潴澄靛,克还旧观。昔之日不通线道者,今可肆行舟楫矣。喜而铭之,铭曰:``蹴醒骊龙,如寐斯揭;不避逆鳞,扶其鲠噎。潴蓄澄泓,煦湿濡沫。夜静水寒,颔珠如月。风雷逼之,扬鬐鼓鬣。''

\hypertarget{header-n234}{%
\subsubsection{朱文懿家桂}\label{header-n234}}

桂以香山名,然覆墓木耳,北邙萧然,不堪久立。单醪河钱氏二桂,老而秃;独朱文懿公宅后一桂,干大如斗,枝叶溟蒙,樾荫亩许,下可坐客三四十席。不亭、不屋、不台、不栏、不砌,弃之篱落间。花时不许人入看,而主人亦禁足勿之往,听其自开自谢已耳。樗栎以不材终其天年,其得力全在弃也。百岁老人多出蓬户,子孙第厌其癃瘇耳,何足称瑞!

\hypertarget{header-n239}{%
\subsubsection{逍遥楼}\label{header-n239}}

滇茶故不易得,亦未有老其材八十余年者。朱文懿公逍遥楼滇茶,为陈海樵先生手植,扶疏蓊翳,老而愈茂。诸文孙恐其力不胜葩,岁删其萼盈斛,然所遗落枝头,犹自燔山熠谷焉。文懿公,张无垢后身。无垢降乩与文懿,谈宿世因甚悉,约公某日面晤于逍遥楼。公伫立久之,有老人至,剧谈良久,公殊不为意。但与公言:``柯亭绿竹庵梁上,有残经一卷,可了之。''寻别去,公始悟老人为无垢。次日,走绿竹庵,简梁上,有《维摩经》一部,缮写精良,后二卷未竟,盖无垢笔也。公取而续书之,如出一手。先君言,乩仙供余家寿芝楼,悬笔挂壁间,有事辄自动,扶下书之,有奇验。娠祈子,病祈药,赐丹,诏取某处,立应。先君祈嗣,诏取丹于某簏临川笔内,簏失钥闭久,先君简视之,横自出觚管中,有金丹一粒,先宜人吞之,即娠余。朱文懿公有姬媵,陈夫人狮子吼,公苦之。祷于仙,求化妒丹。乩书曰:``难,难!丹在公枕内。''取以进夫人,夫人服之,语人曰:``老头子有仙丹,不饷诸婢,而余是饷,尚昵余。''与公相好如初。

\hypertarget{header-n244}{%
\subsubsection{天镜园}\label{header-n244}}

天镜园浴凫堂,高槐深竹,樾暗千层,坐对兰荡,一泓漾之,水木明瑟,鱼鸟藻荇,类若乘空。余读书其中,扑面临头,受用一绿,幽窗开卷,字俱碧鲜。每岁春老,破塘笋必道此。轻舠飞出,牙人择顶大笋一株掷水面,呼园中人曰:``捞笋!''鼓枻飞去。园丁划小舟拾之,形如象牙,白如雪,嫩如花藕,甜如蔗霜。煮食之,无可名言,但有惭愧。

\hypertarget{header-n249}{%
\subsubsection{包涵所}\label{header-n249}}

西湖之船有楼,实包副使涵所创为之。大小三号:头号置歌筵,储歌童;次载书画;再次偫美人。涵老以声伎非侍妾比,仿石季伦、宋子京家法,都令见客。常靓妆走马,媻姗勃窣,穿柳过之,以为笑乐。明槛绮疏,曼讴其下,擫籥弹筝,声如莺试。客至,则歌童演剧,队舞鼓吹,无不绝伦。

乘兴一出,住必浃旬,观者相逐,问其所止。南园在雷峰塔下,北园在飞来峰下。两地皆石薮,积牒磊砢,无非奇峭。但亦借作溪涧桥梁,不于山上叠山,大有文理。大厅以拱斗抬梁,偷其中间四柱,队舞狮子甚畅。北园作八卦房,园亭如规,分作八格,形如扇面。当其狭处,横亘一床,帐前后开合,下里帐则床向外,下外帐则床向内。涵老据其中,扃上开明窗,焚香倚枕,则八床面面皆出。穷奢极欲,老于西湖者二十年。金谷、郿坞,着一毫寒俭不得,索性繁华到底,亦杭州人所谓``左右是左右''也。西湖大家何所不有,西子有时亦贮金星。咄咄书空,则穷措大耳。

\hypertarget{header-n255}{%
\subsubsection{斗鸡社}\label{header-n255}}

天启壬戌间好斗鸡,设斗鸡社于龙山下,仿王勃《斗鸡檄》,檄同社。仲叔秦一生日携古董、书画、文锦、川扇等物与余博,余鸡屡胜之。仲叔忿懑,金其距,介其羽,凡足以助其腷膊敪咮者,无遗策。又不胜。人有言徐州武阳侯樊哙子孙,斗鸡雄天下,长颈乌喙,能于高桌上啄粟。仲叔心动,密遣使访之,又不得,益忿懑。一日,余阅稗史,有言唐玄宗以酉年酉月生,好斗鸡而亡其国。余亦酉年酉月生,遂止。

\hypertarget{header-n260}{%
\subsubsection{栖霞}\label{header-n260}}

戊寅冬,余携竹兜一、苍头一,游栖霞,三宿之。山上下左右鳞次而栉比之,岩石颇佳,尽刻佛像,与杭州飞来峰同受黥劓,是大可恨事。山顶怪石巉岏,灌木苍郁,有颠僧住之。与余谈,荒诞有奇理,惜不得穷诘之。日晡,上摄山顶观霞,非复霞理,余坐石上痴对。复走庵后,看长江帆影,老鹳河、黄天荡,条条出麓下,悄然有山河辽廓之感。一客盘礴余前,熟视余,余晋与揖,问之,为萧伯玉先生,因坐与剧谈,庵僧设茶供。伯玉问及补陀,余适以是年朝海归,谈之甚悉。《补陀志》方成,在箧底,出示伯玉,伯玉大喜,为余作叙。取火下山,拉与同寓宿,夜长,无不谈之,伯玉强余再留一宿。

\hypertarget{header-n265}{%
\subsubsection{湖心亭看雪}\label{header-n265}}

崇祯五年十二月,余住西湖。大雪三日,湖中人鸟声俱绝。是日更定矣,余拿一小舟,拥毳衣炉火,独往湖心亭看雪。雾淞沆砀,天与云、与山、与水,上下一白。湖上影子,惟长堤一痕,湖心亭一点,与余舟一芥,舟中人两三粒而已。

到亭上,有两人铺毡对坐,一童子烧酒,炉正沸。见余大惊喜,曰:``湖中焉得更有此人!''拉余同饮。余强饮三大白而别。问其姓氏,是金陵人,客此。及下船,舟子喃喃曰:``莫说相公痴,更有痴似相公者。''

\hypertarget{header-n271}{%
\subsubsection{陈章侯}\label{header-n271}}

崇祯己卯八月十三,侍南华老人饮湖舫,先月早归。章侯怅怅向余曰:``如此好月,拥被卧耶?''余敦苍头携家酿斗许,呼一小划船再到断桥,章侯独饮,不觉沾醉。过玉莲亭,丁叔潜呼舟北岸,出塘栖蜜桔相饷,畅啖之。章侯方卧船上嚎嚣。岸上有女郎,命童子致意云:``相公船肯载我女郎至一桥否?''余许之。女郎欣然下,轻绔淡弱,婉嫕可人。章侯被酒挑之曰:``女郎侠如张一妹,能同虬髯客饮否?''女郎欣然就饮。移舟至一桥,漏二下矣,竟倾家酿而去。问其住处,笑而不答。章侯欲蹑之,见其过岳王坟,不能追也。

\hypertarget{header-n275}{%
\subsection{卷四}\label{header-n275}}

\hypertarget{header-n277}{%
\subsubsection{不系园}\label{header-n277}}

甲戌十月,携楚生住不系园看红叶。至定香桥,客不期而至者八人:南京曾波臣,东阳赵纯卿,金坛彭天锡,诸暨陈章侯,杭州杨与民、陆九、罗三,女伶陈素芝。余留饮。章侯携缣素为纯卿画古佛,波臣为纯卿写照,杨与民弹三弦子,罗三唱曲,陆九吹箫。与民复出寸许紫檀界尺,据小梧,用北调说《金瓶梅》一剧,使人绝倒。是夜,彭天锡与罗三、与民串本腔戏,妙绝;与楚生、素芝串调腔戏,又复妙绝。章侯唱村落小歌,余取琴和之,牙牙如语。纯卿笑曰:``恨弟无一长,以侑兄辈酒。''余曰:``唐裴将军旻居丧,请吴道子画天宫壁度亡母。道子曰:`将军为我舞剑一回,庶因猛厉以通幽冥。'旻脱缞衣,缠结,上马驰骤,挥剑入云,高十数丈,若电光下射,执鞘承之,剑透室而入,观者惊栗。道子奋袂如风,画壁立就。章侯为纯卿画佛,而纯卿舞剑,正今日事也。''纯卿跳身起,取其竹节鞭,重三十斤,作胡旋舞数缠,大噱而罢。

\hypertarget{header-n282}{%
\subsubsection{秦淮河房}\label{header-n282}}

秦淮河河房,便寓、便交际、便淫冶,房值甚贵,而寓之者无虚日。画船萧鼓,去去来来,周折其间。河房之外,家有露台,朱栏绮疏,竹帘纱幔。夏月浴罢,露台杂坐。两岸水楼中,茉莉风起动儿女香甚。女各团扇轻绔,缓鬓倾髻,软媚着人。年年端午,京城士女填溢,竞看灯船。好事者集小篷船百什艇,篷上挂羊角灯如联珠,船首尾相衔,有连至十余艇者。船如烛龙火蜃,屈曲连蜷,蟠委旋折,水火激射。舟中鏾钹星铙,宴歌弦管,腾腾如沸。士女凭栏轰笑,声光凌乱,耳目不能自主。午夜,曲倦灯残,星星自散。钟伯敬有《秦淮河灯船赋》,备极形致。

\hypertarget{header-n287}{%
\subsubsection{兖州阅武}\label{header-n287}}

辛未三月,余至兖州,见直指阅武。马骑三千,步兵七千,军容甚壮。马蹄卒步,滔滔旷旷,眼与俱驶,猛掣始回。

其阵法奇在变换,旍动而鼓,左抽右旋,疾若风雨。阵既成列,则进图直指前,立一牌曰:``某阵变某阵''。连变十余阵,奇不在整齐而在便捷。扮敌人百余骑,数里外烟尘坌起。迾卒五骑,小如黑子,顷刻驰至,入辕门报警。建大将旗鼓,出奇设伏。敌骑突至,一鼓成擒,俘献中军。内以姣童扮女三四十骑,荷旃被毳,绣袪魋结,马上走解,颠倒横竖,借骑翻腾,柔如无骨。乐奏马上,三弦、胡拨、琥珀词、四上儿、密失叉儿机、僸佅兜离,罔不毕集,在直指筵前供唱,北调淫俚,曲尽其妙。是年,参将罗某,北人,所扮者皆其歌童外宅,故极姣丽,恐易人为之,未必能尔也。

\hypertarget{header-n292}{%
\subsubsection{牛首山打猎}\label{header-n292}}

戊寅冬,余在留都,同族人隆平侯与其弟勋卫、甥赵忻城,贵州杨爱生,扬州顾不盈,余友吕吉士、姚简叔,姬侍王月生、顾眉、董白、李十、杨能,取戎衣衣客,并衣姬侍。

姬侍服大红锦狐嵌箭衣、昭君套,乘款段马,鞲青骹,绁韩卢,统箭手百余人,旗帜棍棒称是,出南门,校猎于牛首山前后,极驰骤纵送之乐。得鹿一、麂三、兔四、雉三、猫狸七。看剧于献花岩,宿于祖茔。次日午后猎归,出鹿麂以飨士,复纵饮于隆平家。江南不晓猎较为何事,余见之图画戏剧,今身亲为之,果称雄快。然自须勋戚豪右为之,寒酸不办也。

\hypertarget{header-n297}{%
\subsubsection{杨神庙台阁}\label{header-n297}}

枫桥杨神庙,九月迎台阁。十年前迎台图,台阁而已;自骆氏兄弟主之,一以思致文理为之。扮马上故事二三十骑,扮传奇一本,年年换,三日亦三换之。其人与传奇中人必酷肖方用,全在未扮时一指点为某似某,非人人绝倒者不之用。迎后,如扮胡梿者,直呼为胡梿,遂无不胡梿之,而此人反失其姓。人定,然后议扮法。必裂缯为之。果其人其袍铠须某色、某缎、某花样,虽匹锦数十金不惜也。一冠一履,主人全副精神在焉。诸友中有能生造刻画者,一月前礼聘至,匠意为之,唯其使。装束备,先期扮演,非百口叫绝又不用。故一人一骑,其中思致文理,如玩古董名画,勾一勒不得放过焉。土人有小小灾祲,辄以小白旗一面到庙禳之,所积盈库。

是日以一竿穿旗三四,一人持竿三四走神前,长可七八里,如几百万白蝴蝶回翔盘礴在山坳树隙。四方来观者数十万人。市枫桥下,亦摊亦篷。台阁上马上,有金珠宝石堕地,拾者,如有物凭焉不能去,必送还神前;其在树丛田坎间者,问神,辄示其处,不或爽。

\hypertarget{header-n302}{%
\subsubsection{雪精}\label{header-n302}}

外祖陶兰风先生,倅寿州,得白骡,蹄跲都白,日行二百里,畜署中。寿州人病噎嗝,辄取其尿疗之。凡告期,乞骡尿状,常十数纸。外祖以木香沁其尿,诏百姓来取。后致仕归,捐馆,舅氏啬轩解骖赠余。余豢之十年许,实未尝具一日草料。日夜听其自出觅食,视其腹未尝不饱,然亦不晓其何从得饱也。天曙,必至门祗候,进厩候驱策,至午勿御,仍出觅食如故。后渐跋扈难御,见余则驯服不动,跨鞍去如箭,易人则咆哮蹄啮,百计鞭策之不应也。一日,与风马争道城上,失足堕濠堑死,余命葬之,谥之曰``雪精''。

\hypertarget{header-n307}{%
\subsubsection{严助庙}\label{header-n307}}

陶堰司徒庙,汉会稽太守严助庙也。岁上元设供,任事者,聚族谋之终岁。凡山物粗粗(虎、豹、麋鹿、獾猪之类),海物噩噩(江豚、海马、鲟黄、鲨鱼之类),陆物痴痴(猪必三百斤,羊必二百斤,一日一换。鸡、鹅、凫、鸭之属,不极肥不上贡),水物哈哈(凡虾、鱼、蟹、蚌之类,无不鲜活),羽物毨毨(孔雀、白鹇、锦鸡、白鹦鹉之属,即生供之),毛物毧毧(白鹿、白兔、活貂鼠之属,亦生供之),洎非地(闽鲜荔枝、圆眼、北苹婆果、沙果、文官果之类)、非天(桃、梅、李、杏、杨梅、枇杷、樱桃之属,收藏如新撷)、非制(熊掌、猩唇、豹胎之属)、非性(酒醉、蜜饯之类)、非理(云南蜜唧、峨眉雪蛆之类)、非想(天花龙蜓、雕镂瓜枣、捻塑米面之类)之物,无不集。庭实之盛,自帝王宗庙社稷坛亹所不能比隆者。十三日,以大船二十艘载盘軨,以童崽扮故事,无甚文理,以多为胜。城中及村落人,水逐陆奔,随路兜截,转折看之,谓之``看灯头''。五夜,夜在庙演剧,梨园必倩越中上三班,或雇自武林者,缠头日数万钱。

唱《伯喈》、《荆钗》,一老者坐台下,对院本,一字脱落,群起噪之,又开场重做。越中有``全伯喈''、``全荆钗''之名起此。天启三年,余兄弟携南院王岑、老串杨四、徐孟雅、圆社河南张大来辈往观之。到庙蹴术,张大来以``一丁泥''``一串珠''名世。球着足,浑身旋滚,一似粘麰有胶、提掇有线、穿插有孔者,人人叫绝。剧至半,王岑汾李三娘,杨四扮火工窦老,徐孟雅扮洪一嫂,马小卿十二岁,扮咬脐,串《磨房》、《撇池》、《送子》、《出猎》四出。科诨曲白,妙入筋髓,又复叫绝。遂解维归。戏场气夺,锣不得响,灯不得亮。

\hypertarget{header-n312}{%
\subsubsection{乳酪}\label{header-n312}}

乳酪自驵侩为之,气味已失,再无佳理。余自豢一牛,夜取乳置盆盎,比晓,乳花簇起尺许,用铜铛煮之,瀹兰雪汁,乳斤和汁四瓯,百沸之。玉液珠胶,雪腴霜腻,吹气胜兰,沁入肺腑,自是天供。或用鹤觞花露入甑蒸之,以热妙;或用豆粉搀和,漉之成腐,以冷妙;或煎酥,或作皮,或缚饼,或酒凝,或盐腌,或醋捉,无不佳妙。而苏州过小拙和以蔗浆霜,熬之、滤之、钻之、掇之、印之,为带骨鲍螺,天下称至味。其制法秘甚,锁密房,以纸封固,虽父子不轻传之。

\hypertarget{header-n317}{%
\subsubsection{二十四桥风月}\label{header-n317}}

广陵二十四桥风月,邗沟尚存其意。渡钞关,横亘半里许,为巷者九条。巷故九,凡周旋折旋于巷之左右前后者,什百之。巷口狭而肠曲,寸寸节节,有精房密户,名妓、歪妓杂处之。名妓匿不见人,非向导莫得入。歪妓多可五六百人,每日傍晚,膏沐熏烧,出巷口,倚徙盘礴于茶馆酒肆之前,谓之``站关''。茶馆酒肆岸上纱灯百盏,诸妓掩映闪灭于其间,疤戾者帘,雄趾者阈。灯前月下,人无正色,所谓``一白能遮百丑''者,粉之力也。游子过客,往来如梭,摩睛相觑,有当意者,逼前牵之去;而是妓忽出身分,肃客先行,自缓步尾之。至巷口,有侦伺者,向巷门呼曰:``某姐有客了!''内应声如雷。火燎即出,一俱去,剩者不过二三十人。沉沉二漏,灯烛将烬,茶馆黑魆无人声。茶博士不好请出,惟作呵欠,而诸妓醵钱向茶博士买烛寸许,以待迟客。或发娇声,唱《擘破玉》等小词,或自相谑浪嘻笑,故作热闹,以乱时候;然笑言哑哑声中,渐带凄楚。夜分不得不去,悄然暗摸如鬼。

见老鸨,受饿、受笞俱不可知矣。余族弟卓如,美须髯,有情痴,善笑,到钞关必狎妓,向余噱曰:``弟今日之乐,不减王公。''余曰:``何谓也?''曰:``王公大人侍妾数百,到晚耽耽望幸,当御者不过一人。弟过钞关,美人数百人,目挑心招,视我如潘安,弟颐指气使,任意拣择,亦必得一当意者呼而侍我。王公大人岂过我哉!''复大噱,余亦大噱。

\hypertarget{header-n322}{%
\subsubsection{世美堂灯}\label{header-n322}}

儿时跨苍头颈,犹及见王新建灯。灯皆贵重华美,珠灯料丝无论,即羊角灯亦描金细画,缨络罩之。悬灯百盏尚须秉烛而行,大是闷人。余见《水浒传》``灯景诗''有云:``楼台上下火照火,车马往来人看人。''已尽灯理。余谓灯不在多,总求一亮。余每放灯,必用如椽大烛,专令数人剪卸烬煤,故光迸重垣,无微不见。十年前,里人有李某者,为闽中二尹,抚台委其造灯,选雕佛匠,穷工极巧,造灯十架,凡两年。灯成而抚台已物故,携归藏椟中。又十年许,知余好灯,举以相赠,余酬之五十金,十不当一,是为主灯。遂以烧珠、料丝、羊角、剔纱诸灯辅之。而友人有夏耳金者,剪采为花,巧夺天工,罩以冰纱,有烟笼芍药之致。更用粗铁线界划规矩,匠意出样,剔纱为蜀锦,墁其界地,鲜艳出人。耳金岁供镇神,必造灯一些,灯后,余每以善价购之。余一小傒善收藏,虽纸灯亦十年不得坏,故灯日富。又从南京得赵士元夹纱屏及灯带数副,皆属鬼工,决非人力。灯宵,出其所有,便称胜事。鼓吹弦索,厮养臧获,皆能为之。有苍头善制盆花,夏间以羊毛炼泥墩,高二尺许,筑``地涌金莲'',声同雷炮,花盖亩余。不用煞拍鼓饶,清吹唢呐应之,望花缓急为唢呐缓急,望花高下为唢呐高下。灯不演剧,则灯意不酣;然无队舞鼓吹,则灯焰不发。余敕小傒串元剧四五十本。演元剧四出,则队舞一回,鼓吹一回,弦索一回。其间浓淡繁简松实之妙,全在主人位置。使易人易地为之,自不能尔尔。故越中夸灯事之盛,必曰``世美堂灯''。

\hypertarget{header-n327}{%
\subsubsection{宁了}\label{header-n327}}

大父母喜豢珍禽:舞鹤三对、白鹇一对,孔雀二对,吐绶鸡一只,白鹦鹉、鹩哥、绿鹦鹉十数架。一异鸟名``宁了'',身小如鸽,黑翎如八哥,能作人语,绝不含糊。大母呼媵婢,辄应声曰:``某丫头,太太叫!''有客至,叫曰:``太太,客来了,看茶!''有一新娘子善睡,黎明辄呼曰:``新娘子,天明了,起来吧!太太叫,快起来!''不起,辄骂曰:``新娘子,臭淫妇,浪蹄子!''新娘子恨甚,置毒药杀之。``宁了''疑即``秦吉了'',蜀叙州出,能人言。一日夷人买去,惊死,其灵异酷似之。

\hypertarget{header-n332}{%
\subsubsection{张氏声伎}\label{header-n332}}

谢太傅不畜声伎,曰:``畏解,故不畜。''王右军曰:``老年赖丝竹陶写,恒恐儿辈觉。''曰``解'',曰``觉'',古人用字深确。盖声音之道入人最微,一解则自不能已,一觉则自不能禁也。我家声伎,前世无之,自大父于万历年间与范长白、邹愚公、黄贞父、包涵所诸先生讲究此道,遂破天荒为之。有``可餐班'',以张彩、王可餐、何闰、张福寿名;次则``武陵班'',以何韵士、傅吉甫、夏清之名;再次则``梯仙班'',以高眉生、李岕生、马蓝生名;再次则``吴郡班'',以王畹生、夏汝开、杨啸生名;再次则``苏小小班'',以马小卿、潘小妃名;再次则平子``茂苑班'',以李含香、顾岕竹、应楚烟、杨騄駬名。主人解事日精一日,而傒童技艺亦愈出愈奇。余历年半百,小傒自小而老、老而复小、小而复老者,凡五易之。

无论``可餐''、``武陵''诸人,如三代法物,不可复见;``梯仙''、``吴郡''间有存者,皆为佝偻老人;而``苏小小班''亦强半化为异物矣;``茂苑班''则吾弟先去,而诸人再易其主。

余则婆娑一老,以碧眼波斯,尚能别其妍丑。山中人至海上归,种种海错皆在其眼,请共舐之。

\hypertarget{header-n337}{%
\subsubsection{方物}\label{header-n337}}

越中清馋,无过余者,喜啖方物。北京则苹婆果、黄巤、马牙松;山东则羊肚菜、秋白梨、文官果、甜子;福建则福桔、福桔饼、牛皮糖、红腐乳;江西则青根、丰城脯;山西则天花菜;苏州则带骨鲍螺、山查丁、山查糕、松子糖、白圆、橄榄脯;嘉兴则马交鱼脯、陶庄黄雀;南京则套樱桃、桃门枣、地栗团、窝笋团、山查糖;杭州则西瓜、鸡豆子、花下藕、韭芽、玄笋、塘栖蜜桔;萧山则杨梅、莼菜、鸠鸟、青鲫、方柿;诸暨则香狸、樱桃、虎栗;嵊则蕨粉、细榧、龙游糖;临海则枕头瓜;台州则瓦楞蚶、江瑶柱;浦江则火肉;

东阳则南枣;山阴则破塘笋、谢桔、独山菱、河蟹、三江屯坚、白蛤、江鱼、鲥鱼、里河鰦。远则岁致之,近则月致之、日致之。耽耽逐逐,日为口腹谋,罪孽固重。但由今思之,四方兵燹,寸寸割裂,钱塘衣带水,犹不敢轻渡,则向之传食四方,不可不谓之福德也。

\hypertarget{header-n342}{%
\subsubsection{祁止祥癖}\label{header-n342}}

人无癖不可与交,以其无深情也;人无疵不可与交,以其无真气也。余友祁止祥有书画癖,有蹴鞠癖,有鼓钹癖,有鬼戏癖,有梨园癖。壬午,至南都,止祥出阿宝示余,余谓:``此西方迦陵鸟,何处得来?''阿宝妖冶如蕊女,而娇痴无赖,故作涩勒,不肯着人。如食橄榄,咽涩无味,而韵在回甘;如吃烟酒,鲠詰无奈,而软同沾醉。初如可厌,而过即思之。止祥精音律,咬钉嚼铁,一字百磨,口口亲授,阿宝辈皆能曲通主意。乙酉,南都失守,止祥奔归,遇土贼,刀剑加颈,性命可倾,阿宝是宝。丙戌,以监军驻台州,乱民卤掠,止祥囊箧都尽,阿宝沿途唱曲,以膳主人。及归,刚半月,又挟之远去。止祥去妻子如脱屣耳,独以娈童崽子为性命,其癖如此。

\hypertarget{header-n347}{%
\subsubsection{泰安州客店}\label{header-n347}}

客店至泰安州,不复敢以客店目之。余进香泰山,未至店里许,见驴马槽房二三十间;再近,有戏子寓二十余处;再近,则密户曲房,皆妓女妖冶其中。余谓是一州之事,不知其为一店之事也。投店者,先至一厅事,上簿挂号,人纳店例银三钱八分,又人纳税山银一钱八分。店房三等:下客夜素早亦素,午在山上用素酒果核劳之,谓之``接顶''。夜至店,设席贺,谓烧香后求官得官,求子得子,求利得利,故曰贺也。贺亦三等:上者专席,糖饼、五果、十肴、果核、演戏;次者二人一席,亦糖饼,亦肴核,亦演戏;下者三四人一席,亦糖饼、骨核,不演戏,用弹唱。计其店中,演戏者二十余处,弹唱者不胜计。庖厨炊灶亦二十余所,奔走服役者一二百人。下山后,荤酒狎妓惟所欲,此皆一日事也。若上山落山,客日日至,而新旧客房不相袭,荤素庖厨不相混,迎送厮役不相兼,是则不可测识之矣。泰安一州与此店比者五六所,又更奇。

\hypertarget{header-n351}{%
\subsection{卷五}\label{header-n351}}

\hypertarget{header-n353}{%
\subsubsection{范长白}\label{header-n353}}

范长白园在天平山下,万石都焉。龙性难驯,石皆笏起,旁为范文正墓。园外有长堤,桃柳曲桥,蟠屈湖面,桥尽抵园,园门故作低小,进门则长廊复壁,直达山麓。其绘楼幔阁、秘室曲房,故故匿之,不使人见也。山之左为桃源,峭壁回湍,桃花片片流出。右孤山,种梅千树。渡涧为小兰亭,茂林修竹,曲水流觞,件件有之。竹大如椽,明静娟洁,打磨滑泽如扇骨,是则兰亭所无也。地必古迹,名必古人,此是主人学问。但桃则溪之,梅则屿之,竹则林之,尽可自名其家,不必寄人篱下也。余至,主人出见。主人与大父同籍,以奇丑著。是日释褐,大父嬲之曰:``丑不冠带,范年兄亦冠带了也。''人传以笑。余亟欲一见。及出,状貌果奇,似羊肚石雕一小猱,其鼻垩,颧颐犹残缺失次也。冠履精洁,若谐谑谈笑面目中不应有此。开山堂小饮,绮疏藻幕,备极华褥,秘阁请讴,丝竹摇飏,忽出层垣,知为女乐。饮罢,又移席小兰亭,比晚辞去。主人曰:``宽坐,请看`少焉'。''金不解,主人曰:``吾乡有缙绅先生,喜调文袋,以《赤壁赋》有`少焉月出于东山之上'句,遂字月为`少焉'。顷言`少焉'者,月也。''固留看月,晚景果妙。主人曰:``四方客来,都不及见小园雪,山石崡岈,银涛蹴起,掀翻五泄,捣碎龙湫,世上伟观,惜不令宗子见也。''步月而出,至玄墓,宿葆生叔书画舫中。

\hypertarget{header-n358}{%
\subsubsection{于园}\label{header-n358}}

于园在瓜州步五里铺,富人于五所园也。非显者刺,则门钥不得出。葆生叔同知瓜州,携余往,主人处处款之。园中无他奇,奇在磥石。前堂石坡高二丈,上植果子松数棵,缘坡植牡月、芍药,人不得上,以实奇。后厅临大池,池中奇峰绝壑,陡上陡下,人走池底,仰视莲花,反在天上,以空奇。卧房槛外,一壑旋下如螺蛳缠,以幽阴深邃奇。再后一水阁,长如艇子,跨小河,四围灌木蒙丛,禽鸟啾唧,如深山茂林,坐其中,颓然碧窈。瓜州诸园亭,俱以假山显,胎于石,娠于磥石之手,男女于琢磨搜剔之主人,至于园可无憾矣。仪真汪园,葢石费至四五万,其所最加意者,为``飞来''一峰,阴翳泥泞,供人唾骂。余见其弃地下一白石,高一丈、阔二丈而痴,痴妙;一黑石,阔八尺、高丈五而瘦,瘦妙。得此二石足矣,省下二三万收其子母,以世守此二石何如?

\hypertarget{header-n363}{%
\subsubsection{诸工}\label{header-n363}}

竹与漆与铜与窑,贱工也。嘉兴之腊竹,王二之漆竹,苏州姜华雨之籋箓竹,嘉兴洪漆之漆,张铜之铜,徽州吴明官之窑,皆以竹与漆与铜与窑名家起家,而其人且与缙绅先生列坐抗礼焉。则天下何物不足以贵人,特人自贱之耳。

\hypertarget{header-n368}{%
\subsubsection{姚简叔画}\label{header-n368}}

姚简叔画千古,人亦千古。戊寅,简叔客魏为上宾。余寓桃叶渡,往来者闵汶水、曾波臣一二人而已。简叔无半面交,访余,一见如平生欢,遂榻余寓。与余料理米盐之事,不使余知。有空,则拉余饮淮上馆,潦倒而归。京中诸勋戚大老、朋侪缁衲、高人名妓与简叔交者,必使交余,无或遗者。

与余同起居者十日,有苍头至,方知其有妾在寓也。简叔塞渊不露聪明,为人落落难合,孤意一往,使人不可亲疏。与余交不知何缘,反而求之不得也。访友报恩寺,出册叶百方,宋元名笔。简叔眼光透入重纸,据梧精思,面无人色。及归,为余仿苏汉臣一图:小儿方据澡盆浴,一脚入水,一脚退缩欲出;宫人蹲盆侧,一手掖儿,一手为儿擤鼻涕;旁坐宫娥,一儿浴起伏其膝,为结绣裾。一图,宫娥盛装端立有所俟,双鬟尾之;一侍儿捧盘,盘列二瓯,意色向客;一宫娥持其盘,为整茶锹,详视端谨。复视原本,一笔不失。

\hypertarget{header-n373}{%
\subsubsection{炉峰月}\label{header-n373}}

炉峰绝顶,复岫回峦,斗耸相乱,千丈岩陬牙横梧,两石不相接者丈许,俯身下视,足震慑不得前。王文成少年曾趵而过,人服其胆。余叔尔蕴以毡裹体,缒而下,余挟二樵子,从壑底摉而上,可谓痴绝。丁卯四月,余读书天瓦庵,午后同二三友人绝顶,看落照。一友曰:``少需之,俟月出去。胜期难再得,纵遇虎,亦命也。且虎亦有道,夜则下山觅豚犬食耳,渠上山亦看月耶?''语亦有理。四人踞坐金简石上。

是日,月正望,日没月出,山中草木都发光怪,悄然生恐。月白路明,相与策杖而下。行未数武,半山嘄呼,乃余苍头同山僧七八人,持火燎、靿刀、木棍,疑余辈遇虎失路,缘山叫喊耳。余接声应,奔而上,扶掖下之。次日,山背有人言:``昨晚更定,有火燎数十把,大盗百余人,过张公岭,不知出何地?''吾辈匿笑不之语。谢灵运开山临澥,从者数百人,太守王琇惊駴,谓是山贼,及知为灵运,乃安。吾辈是夜不以山贼缚献太守,亦幸矣。

\hypertarget{header-n378}{%
\subsubsection{湘湖}\label{header-n378}}

西湖,田也而湖之,成湖焉;湘湖,亦田也而湖之,不成湖焉。湖西湖者,坡公也,有意于湖而湖之者也;湖湘湖者,任长者也,不愿湖而湖之者也。任长者有湘湖田数百顷,称巨富。有术者相其一夜而贫,不信。县官请湖湘湖,灌萧山田,诏湖之,而长者之田一夜失,遂赤贫如术者言。今虽湖,尚田也,不下插板,不筑堰,则水立涸;是以湖中水道,非熟于湖者不能行咫尺。游湖者坚欲去,必寻湖中小船与湖中识水道之人,溯十阏三,鲠咽不之畅焉。湖里外锁以桥,里湖愈佳。盖西湖止一湖心亭为眼中黑子,湘湖皆小阜、小墩、小山乱插水面,四围山趾,棱棱砺砺,濡足入水,尤为奇峭。

余谓西湖如名妓,人人得而媟亵之;鉴湖如闺秀,可钦而不可狎;湘湖如处子,目氐娗羞涩,犹及见其未嫁时也。此是定评,确不可易。

\hypertarget{header-n383}{%
\subsubsection{柳敬亭说书}\label{header-n383}}

南京柳麻子,黧黑,满面疤槃,悠悠忽忽,土木形骸,善说书。一日说书一回,定价一两。十日前先送书帕下定,常不得空。南京一时有两行情人:王月生、柳麻子是也。余听其说《景阳冈武松打虎》白文,与本传大异。其描写刻画,微入毫发,然又找截干净,并不唠叨。勃夬声如巨钟,说至筋节处,叱咤叫喊,汹汹崩屋。武松到店沽酒,店内无人,謈地一吼,店中空缸空甓皆瓮瓮有声。闲中着色,细微至此。主人必屏息静坐,倾耳听之,彼方掉舌。稍见下人呫哔耳语,听者欠伸有倦色,辄不言,故不得强。每至丙夜,拭桌剪灯,素瓷静递,款款言之,其疾徐轻重,吞吐抑扬,入情入理,入筋入骨,摘世上说书之耳而使之谛听,不怕其不齚舌死也。柳麻子貌奇丑,然其口角波俏,眼目流利,衣服恬静,直与王月生同其婉娈,故其行情正等。

\hypertarget{header-n388}{%
\subsubsection{樊江陈氏桔}\label{header-n388}}

樊江陈氏,辟地为果园,枸菊围之。自麦为蒟酱,自称酿酒,酒香洌,色如淡金蜜珀,酒人称之。自果自蓏,以螫乳醴之为冥果。树谢桔百株,青不撷,酸不撷,不树上红不撷,不霜不撷,不连蒂剪不撷。故其所撷,桔皮宽而绽,色黄而深,瓤坚而脆,筋解而脱,味甜而鲜。第四门、陶堰、道墟以至塘栖,皆无其比。余岁必亲至其园买桔,宁迟、宁贵、宁少。购得之,用黄砂缸,藉以金城稻草或燥松毛收之。阅十日,草有润气,又更换之。可藏至三月尽,甘脆如新撷者。枸菊城主人桔百树,岁获绢百匹,不愧木奴。

\hypertarget{header-n393}{%
\subsubsection{治沅堂}\label{header-n393}}

占有拆字法。宣和间,成都谢石拆字,言祸福如响。钦宗闻之,书一``朝''字,令中贵人持试之。石见字,端视中贵人曰:``此非观察书也。''中贵人愕然。石曰:```朝'字离之为`十月十日',乃此月此日所生之天人,得非上位耶?''一国骇异。吾越谢文正厅事名``保锡堂'',后易之他姓,主人至,亟去其匾,人问之,曰:``分明写`呆人易金堂'。''朱石门为文选署中额``典劇''二字,继之者顾诸吏曰:``尔知朱公意乎?此二字离合言之,曰:`曲處曲處,八刀八刀'耳。''歙许相国孙志吉为大理评事,受魏珰指,案卖黄山,势张甚,当道媚之,送一匾曰``大卜于门''。里人夜至,增减其笔划凡三:一曰``天下未闻'';一倒读之曰``阉手下犬'';一曰``太平拿问''。后直指提问,械至太平,果如其言。凡此数者皆有义味。而吾乡缙绅有名``治沅堂''者,人不解其义,问之,笑不答,力究之,缮绅曰:``无他意,亦止取`三台三元'之义云耳!''闻者喷饭。

\hypertarget{header-n398}{%
\subsubsection{虎丘中秋夜}\label{header-n398}}

虎丘八月半,土著流寓、士夫眷属、女乐声伎、曲中名妓戏婆、民间少妇好女、崽子娈童及游冶恶少、清客帮闲、傒僮走空之辈,无不鳞集。自生公台、千人石、鹅涧、剑池、申文定祠下,至试剑石、一二山门,皆铺毡席地坐,登高望之,如雁落平沙,霞铺江上。天暝月上,鼓吹百十处,大吹大擂,十番铙钹,渔阳掺挝,动地翻天,雷轰鼎沸,呼叫不闻。更定,鼓铙渐歇,丝管繁兴,杂以歌唱,皆``锦帆开,澄湖万顷''同场大曲,蹲踏和锣丝竹肉声,不辨拍煞。更深,人渐散去,士夫眷属皆下船水嬉,席席征歌,人人献技,南北杂之,管弦迭奏,听者方辨句字,藻鉴随之。二鼓人静,悉屏管弦,洞萧一缕,哀涩清绵,与肉相引,尚存三四,迭更为之。三鼓,月孤气肃,人皆寂阒,不杂蚊虻。一夫登场,高坐石上,不箫不拍,声出如丝,裂石穿云,串度抑扬,一字一刻。听者寻入针芥,心血为枯,不敢击节,惟有点头。然此时雁比而坐者,犹存百十人焉。使非苏州,焉讨识者!

\hypertarget{header-n403}{%
\subsubsection{麋公}\label{header-n403}}

万历甲辰,有老医驯一大角鹿,以铁钳其趾,设鞼韅其上,用笼头衔勒,骑而走,角上挂葫芦药瓮,随所病出药,服之辄愈。家大人见之喜,欲售其鹿,老人欣然,肯解以赠,大人以三十金售之。五月朔日,为大父寿,大父伟硕,跨之走数百步,辄立而喘,常命小裾笼之,从游山泽。次年,至云间,解赠陈眉公。眉公羸瘦,行可连二三里,大喜。后携至西湖六桥、三竺间,竹冠羽衣,往来于长堤深柳之下,见者啧啧,称为``谪仙''。后眉公复号``麋公''者,以此。

\hypertarget{header-n408}{%
\subsubsection{扬州清明}\label{header-n408}}

扬州清明日,城中男女毕出,家家展墓。虽家有数墓,日必展之。故轻车骏马,箫鼓画船,转折再三,不辞往复。监门小户亦携肴核纸钱,走至墓所、祭毕,则席地饮胙。自钞关南门、古渡桥、天宁寺、平山堂一带,靓妆藻野,袨服缛川。随有货郎,路旁摆设古董古玩并小儿器具。博徒持小杌坐空地,左右铺衵衫半臂,纱裙汗帨,铜炉锡注,瓷瓯漆奁,及肩彘鲜鱼、秋梨福桔之属,呼朋引类,以钱掷地,谓之``跌成'';或六或八或十,谓之``六成''``八成''``十成''焉。百十其处,人环观之。是日,四方流离及徽商西贾、曲中名妓,一切好事之徒,无不咸集。长塘丰草,走马放鹰;高阜平冈,斗鸡蹴踘;茂林清樾,劈阮弹筝。浪子相扑,童稚纸鸢,老僧因果,瞽者说书,立者林林,蹲者蛰蛰。日暮霞生,车马纷沓。宦门淑秀,车幕尽开,婢媵倦归,山花斜插,臻臻簇簇,夺门而入。余所见者,惟西湖春、秦淮夏、虎丘秋,差足比拟。然彼皆团簇一块,如画家横披;此独鱼贯雁比,舒长且三十里焉,则画家之手卷矣。南宋张择端作《清明上河图》,追摹汴京景物,有方美人之思,而余目盱盱,能无梦想!

\hypertarget{header-n413}{%
\subsubsection{金山竞渡}\label{header-n413}}

看西湖竞渡十二三次,己巳竞渡于秦淮,辛未竞渡于无锡,壬午竞渡于瓜州,于金山寺。西湖竞渡,以看竞渡之人胜,无锡亦如之。秦淮有灯船无龙船,龙船无瓜州比,而看龙船亦无金山寺比。瓜州龙船一二十只,刻画龙头尾,取其怒;旁坐二十人持大楫,取其悍;中用彩篷,前后旌幢绣伞,取其绚;撞钲挝鼓,取其节;艄后列军器一架,取其锷;龙头上一人足倒竖,敁敠其上,取其危;龙尾挂一小儿,取其险。自五月初一至十五,日日画地而出。五日出金山,镇江亦出。惊湍跳沫,群龙格斗,偶堕洄涡,则蜐捷捽,蟠委出之。金山上人团簇,隔江望之,蚁附蜂屯,蠢蠢欲动。晚则万艓齐开,两岸沓沓然而沸。

\hypertarget{header-n418}{%
\subsubsection{刘晖吉女戏}\label{header-n418}}

女戏以妖冶恕,以啴缓恕,以态度恕,故女戏者全乎其为恕也。若刘晖吉则异是。刘晖吉奇情幻想,欲补从来梨园之缺陷。如《唐明皇游月宫》,叶法善作,场上一时黑魆地暗,手起剑落,霹雳一声,黑幔忽收,露出一月,其圆如规,四下以羊角染五色云气,中坐常仪,桂树吴刚,白兔捣药。轻纱幔之,内燃``赛月明''数株,光焰青黎,色如初曙,撒布成梁,遂蹑月窟,境界神奇,忘其为戏也。其他如舞灯,十数人手携一灯,忽隐忽现,怪幻百出,匪夷所思,令唐明皇见之,亦必目睁口开,谓氍毹场中那得如许光怪耶!彭天锡向余道:``女戏至刘晖吉,何必男子!何必彭大!''天锡曲中南、董,绝少许可,而独心折晖吉家姬,其所鉴赏,定不草草。

\hypertarget{header-n423}{%
\subsubsection{朱楚生}\label{header-n423}}

朱楚生,女戏耳,调腔戏耳。其科白之妙,有本腔不能得十分之一者。盖四明姚益城先生精音律,尝与楚生辈讲究关节,妙入情理,如《江天暮雪》、《霄光剑》、《画中人》等戏,虽昆山老教师细细摹拟,断不能加其毫末也。班中脚色,足以鼓吹楚生者方留之,故班次愈妙。楚生色不甚美,虽绝世佳人,无其风韵。楚楚谡谡,其孤意在眉,其深情在睫,其解意在烟视媚行。性命于戏,下全力为之。曲白有误,稍为订正之,虽后数月,其误处必改削如所语。楚生多坐驰,一往深情,摇飏无主。一日,同余在定香桥,日晡烟生,林木窅冥,楚生低头不语,泣如雨下,余问之,作饰语以对。劳心忡忡,终以情死。

\hypertarget{header-n428}{%
\subsubsection{扬州瘦马}\label{header-n428}}

扬州人日饮食于瘦马之身者数十百人。娶妾者切勿露意,稍透消息,牙婆驵侩,咸集其门,如蝇附膻,撩扑不去。黎明,即促之出门,媒人先到者先挟之去,其余尾其后,接踵伺之。至瘦马家,坐定,进茶,牙婆扶瘦马出,曰:``姑娘拜客。''下拜。曰:``姑娘往上走。''走。曰:``姑娘转身。''转身向明立,面出。曰:``姑娘借手睄睄。''尽褫其袂,手出、臂出、肤亦出。曰:``姑娘睄相公。''转眼偷觑,眼出。曰:``姑娘几岁?''曰几岁,声出。曰:``姑娘再走走。''以手拉其裙,趾出。然看趾有法,凡出门裙幅先响者,必大;高系其裙,人未出而趾先出者,必小。曰:``姑娘请回。''一人进,一人又出。看一家必五六人,咸如之。看中者,用金簪或钗一股插其鬓,曰``插带''。看不中,出钱数百文,赏牙婆或赏其家侍婢,又去看。牙婆倦,又有数牙婆踵伺之。一日、二日至四五日,不倦亦不尽,然看至五六十人,白面红衫,千篇一律,如学字者,一字写至百至千,连此字亦不认得矣。心与目谋,毫无把柄,不得不聊且迁就,定其一人。``插带''后,本家出一红单,上写彩缎若干,金花若干,财礼若干,布匹若干,用笔蘸墨,送客点阅。客批财礼及缎匹如其意,则肃客归。归未抵寓,而鼓乐盘担、红绿羊酒在其门久矣。不一刻,而礼币、糕果俱齐,鼓乐导之去。去未半里,而花轿花灯、擎燎火把、山人傧相、纸烛供果牲醴之属,门前环侍。厨子挑一担至,则蔬果、肴馔汤点、花棚糖饼、桌围坐褥、酒壶杯箸、龙虎寿星、撒帐牵红、小唱弦索之类,又毕备矣。不待复命,亦不待主人命,而花轿及亲送小轿一齐往迎,鼓乐灯燎,新人轿与亲送轿一时俱到矣。新人拜堂,亲送上席,小唱鼓吹,喧阗热闹。日未午而讨赏遽去,急往他家,又复如是。

\hypertarget{header-n432}{%
\subsection{卷六}\label{header-n432}}

\hypertarget{header-n434}{%
\subsubsection{彭天锡串戏}\label{header-n434}}

彭天锡串戏妙天下,然出出皆有传头,未尝一字杜撰。曾以一出戏,延其人至家,费数十金者,家业十万缘手而尽。三春多在西湖,曾五至绍兴,到余家串戏五六十场,而穷其技不尽。天锡多扮丑净,千古之奸雄佞幸,经天锡之心肝而愈狠,借天锡之面目而愈刁,出天锡之口角而愈险。设身处地,恐纣之恶不如是之甚也。皱眉视眼,实实腹中有剑,笑里有刀,鬼气杀机,阴森可畏。盖天锡一肚皮书史,一肚皮山川,一肚皮机械,一肚皮磊砢不平之气,无地发泄,特于是发泄之耳。余尝见一出好戏,恨不得法锦包裹,传之不朽;尝比之天上一夜好月,与得火候一杯好茶,只可供一刻受用,其实珍惜之不尽也。桓子野见山水佳处,辄呼``余何!奈何!''真有无可奈何者,口说不出。

\hypertarget{header-n439}{%
\subsubsection{目莲戏}\label{header-n439}}

余蕴叔演武场搭一大台,选徽州旌阳戏子剽轻精悍、能相扑跌打者三四十人,搬演目莲,凡三日三夜。四围女台百什座,戏子献技台上,如度索舞絙、翻桌翻梯、觔斗蜻蜓、蹬坛蹬臼、跳索跳圈,窜火窜剑之类,大非情理。凡天神地祇、牛头马面、鬼母丧门、夜叉罗刹、锯磨鼎镬、刀山寒冰、剑树森罗、铁城血澥,一似吴道子《地狱变相》,为之费纸札者万钱,人心惴惴,灯下面皆鬼色。戏中套数,如《招五方恶鬼》、《刘氏逃棚》等剧,万余人齐声呐喊。熊太守谓是海寇卒至,惊起,差衙官侦问,余叔自往复之,乃安。台成,叔走笔书二对。一曰:``果证幽明,看善善恶恶随形答响,到底来个能逃?道通昼夜,任生生死死换姓移名,下场去此人还在。''一曰:``装神扮鬼,愚蠢的心下惊慌,怕当真也是如此。成佛作祖,聪明人眼底忽略,临了时还待怎生?''真是以戏说法。

\hypertarget{header-n444}{%
\subsubsection{甘文台炉}\label{header-n444}}

香炉贵适用,尤贵耐火。三代青绿,见火即败坏,哥、汝窑亦如之。便用便火,莫如宣炉。然近日宣铜一炉价百四五十金,焉能办之?北铸如施银匠亦佳,但粗夯可厌。苏州甘回子文台,其拨蜡范沙,深心有法,而烧铜色等分两,与宣铜款致分毫无二,俱可乱真;然其与人不同者,尤在铜料。甘文台以回回教门不崇佛法,乌斯藏渗金佛,见即锤碎之,不介意,故其铜质不特与宣铜等,而有时实胜之。甘文台自言佛像遭劫已七百尊有奇矣。余曰:``使回回国别有地狱,则可。''

\hypertarget{header-n449}{%
\subsubsection{绍兴灯景}\label{header-n449}}

绍兴灯景为海内所夸者无他,竹贱、灯贱、烛贱。贱,故家家可为之;贱,故家家以不能灯为耻。故自庄逵以至穷檐曲巷,无不灯、无不棚者。棚以二竿竹搭过桥,中横一竹,挂雪灯一,灯球六。大街以百计,小巷以十计。从巷口回视巷内,复迭堆垛,鲜妍飘洒,亦足动人。十字街搭木棚,挂大灯一,俗曰``呆灯'',画《四书》、《千家诗》故事,或写灯谜,环立猜射之。庵堂寺观以木架作柱灯及门额,写``庆赏元宵''、``与民同乐''等字。佛前红纸荷花琉璃百盏,以佛图灯带间之,熊熊煜煜。庙门前高台,鼓吹五夜。市廛如横街轩亭、会稽县西桥,闾里相约,故盛其灯,更于其地斗狮子灯,鼓吹弹唱,施放烟火,挤挤杂杂。小街曲巷有空地,则跳大头和尚,锣鼓声错,处处有人团簇看之。城中妇女多相率步行,往闹处看灯;否则,大家小户杂坐门前,吃瓜子、糖豆,看往来士女,午夜方散。乡村夫妇多在白日进城,乔乔画画,东穿西走,曰``钻灯棚'',曰``走灯桥'',天晴无日无之。万历间,父叔辈于龙山放灯,称盛事,而年来有效之者。次年,朱相国家放灯塔山。再次年,放灯蕺山。蕺山以小户效颦,用竹棚,多挂纸魁星灯。有轻薄子作口号嘲之曰:``蕺山灯景实堪夸,葫筿芋头挂夜叉。若问搭彩是何物,手巾脚布神袍纱。''由今思之,亦是不恶。

\hypertarget{header-n454}{%
\subsubsection{韵山}\label{header-n454}}

大父至老,手不释卷,斋头亦喜书画、瓶几布设。不数日,翻阅搜讨,尘堆砚表,卷帙正倒参差。常从尘砚中磨墨一方,头眼入于纸笔,潦草作书牛家蝇头细字。日晡向晦,则携卷出帘外,就天光爇烛,檠高光不到纸,辄倚几携书就灯,与光俱俯,每至夜分,不以为疲。常恨《韵府群玉》、《五车韵瑞》寒俭可笑,意欲广之。乃博采群书,用淮南``大小山''义,摘其事曰《大山》,摘其语曰《小山》,事语已详本韵而偶寄他韵下曰《他山》,脍炙人口者曰《残山》,总名之曰《韵山》。小字襞积,烟煤残楮,厚如砖块者三百余本。一韵积至十余本,《韵府》、《五车》不啻千倍之矣。正欲成帙,胡仪部青莲携其尊人所出中秘书,名《永乐大典》者,与《韵山》正相类,大帙三十余本,一韵中之一字犹不尽焉。大父见而太息曰:``书囊无尽,精卫衔石填海,所得几何!''遂辍笔而止。以三十年之精神,使为别书,其博洽应不在王弇州、杨升庵下。今此书再加三十年,亦不能成,纵成亦力不能刻。笔冢如山,只堪覆瓿,余深惜之。丙戌兵乱,余载往九里山,藏之藏经阁,以待后人。

\hypertarget{header-n459}{%
\subsubsection{天童寺僧}\label{header-n459}}

戊寅,同秦一生诣天童访金粟和尚。到山门,见万工池绿净,可鉴须眉,旁有大锅覆地,问僧,僧曰:``天童山有龙藏,龙常下饮池水,故此水刍秽不入。正德间,二龙斗,寺僧五六百人撞钟鼓撼之,龙怒,扫寺成白地,锅其遗也。''入大殿,宏丽庄严。折入方丈,通名刺。老和尚见人便打,曰``棒喝''。余坐方丈,老和尚迟迟出,二侍者执杖、执如意先导之,南向立,曰:``老和尚出。''又曰:``怎么行礼?''盖官长见者皆下拜,无抗礼,余屹立不动,老和尚下行宾主礼。侍者又曰:``老和尚怎么坐?''余又屹立不动,老和尚肃余坐。坐定,余曰:``二生门外汉,不知佛理,亦不知佛法,望老和尚慈悲,明白开示。勿劳棒喝,勿落机锋,只求如家常白话,老实商量,求个下落。''老和尚首肯余言,导余随喜。早晚斋方丈,敬礼特甚。余遍观寺中僧匠千五百人,俱春者、碓者、磨者、甑者、汲者、爨者、锯者、劈者、菜者、饭者,狰狞急遽,大似吴道子一幅《地狱变相》。老和尚规矩严肃,常自起撞人,不止``棒喝''。

\hypertarget{header-n464}{%
\subsubsection{水浒牌}\label{header-n464}}

古貌古服、古兜鍪、古铠胄、古器械,章侯自写其所学所问已耳。而辄呼之曰``宋江'',曰``吴用'',而``宋江''、``吴用''亦无不应者,以英雄忠义之气,郁郁芋芋,积于笔墨间也。周孔嘉丐余促章侯,孔嘉丐之,余促之,凡四阅月而成。余为作缘起曰:``余友章侯,才足掞天,笔能泣鬼,昌谷道上,婢囊呕血之诗;兰清寺中,僧秘开花之字。兼之力开画苑,遂能目无古人,有索必酬,无求不与。既蠲郭恕先之癖,喜周贾耘老之贫,画《水浒》四十人,为孔嘉八口计,遂使宋江兄弟,复睹汉官威仪。伯益考著《山海》遗经,兽毨鸟氄皆拾为千古奇文;吴道子画《地狱变相》,青面獠牙尽化作一团清气。收掌付双荷叶,能月继三石米,致二斗酒,不妨持赠;珍重如柳河东,必日灌蔷薇露,薰玉蕤香,方许解观。非敢阿私,愿公同好。''

\hypertarget{header-n469}{%
\subsubsection{烟雨楼}\label{header-n469}}

嘉兴人开口烟雨楼,天下笑之。然烟雨楼故自佳。楼襟对莺泽湖,涳涳蒙蒙,时带雨意,长芦高柳,能与湖为浅深。

湖多精舫,美人航之,载书画茶酒,与客期于烟雨楼。客至,则载之去,舣舟干烟波缥缈。态度幽闲,茗炉相对,意之所安,经旬不返。舟中有所需,则逸出宣公桥、角里街,果蓏蔬鲜,法膳琼苏,咄嗟立办,旋即归航。柳湾桃坞,痴迷伫想,若遇仙缘,洒然言别,不落姓氏。间有倩女离魂,文君新寡,亦效颦为之。淫靡之事,出以风韵,习俗之恶,愈出愈奇。

\hypertarget{header-n474}{%
\subsubsection{朱氏收藏}\label{header-n474}}

朱氏家藏,如``龙尾觥''、``合卺杯'',雕镂锲刻,真属鬼工,世不再见。余如秦铜汉玉、周鼎商彝、哥窑倭漆、厂盒宣炉、法书名画、晋帖唐琴,所畜之多,与分宜埒富,时人讥之。余谓博洽好古,犹是文人韵事,风雅之列,不黜曹瞒,鉴赏之家,尚存秋壑。诗文书画未尝不抬举古人,恒恐子孙效尤,以袖攫石、攫金银以赚田宅,豪夺巧取,未免有累盛德。闻昔年朱氏子孙,有欲卖尽``坐朝问道''四号田者,余外祖兰风先生谑之曰:``你只管坐朝问道,怎不管垂拱平章?''

一时传为佳话。

\hypertarget{header-n479}{%
\subsubsection{仲叔古董}\label{header-n479}}

葆生叔少从渭阳游,遂精赏鉴。得白定炉、哥窑瓶、官窑酒匜,项墨林以五百金售之,辞曰:``留以殉葬。''癸卯,道淮上,有铁梨木天然几,长丈六、阔三尺,滑泽坚润,非常理。淮抚李三才百五十金不能得,仲叔以二百金得之,解维遽去。淮抚大恚怒,差兵蹑之,不及而返。庚戌,得石璞三十斤,取日下水涤之,石罅中光射如鹦哥祖母,知是水碧,仲叔大喜。募玉工仿朱氏``龙尾觥''一,``合卺杯''一,享价三千,其余片屑寸皮,皆成异宝。仲叔赢资巨万,收藏日富。戊辰后,倅姑熟,倅姑苏,寻令盟津。河南为铜薮,所得铜器盈数车,``美人觚''一种,大小十五六枚,青绿彻骨,如翡翠,如鬼眼青,有不可正视之者,归之燕客,一日失之。或是龙藏收去。

\hypertarget{header-n484}{%
\subsubsection{噱社}\label{header-n484}}

仲叔善诙谐,在京师与漏仲容、沈虎臣、韩求仲辈结``噱社'',唼喋数言,必绝缨喷饭。漏仲容为贴括名士,常曰:``吾辈老年读书做文字,与少年不同。少年读书,如快刀切物,眼光逼注,皆在行墨空处,一过辄了。老年如以指头掐字,掐得一个,只是一个,掐得不着时,只是白地。少年做文字,白眼看天,一篇现成文字挂在天上,顷刻下来,刷入纸上,一刷便完。老年如恶心呕吐,以手扼入齿哕出之,出亦无多,总是渣秽。''此是格言,非止谐语。一日,韩求仲与仲叔同宴一客,欲连名速之,仲叔曰:``我长求仲,则我名应在求仲前,但缀绳头于如拳之上,则是细注在前,白文在后,那有此理!''人皆失笑。沈虎臣出语尤尖巧。仲叔候座师收一帽套,此日严寒,沈虎臣嘲之曰:``座主已收帽套去,此地空余帽套头;帽套一去不复返,此头千载冷悠悠。''其滑稽多类此。

\hypertarget{header-n489}{%
\subsubsection{鲁府松棚}\label{header-n489}}

报国寺松,蔓引亸委,已入藤理。入其下者,蹒跚局蹐,气不得舒。鲁府旧邸二松,高丈五,上及檐甃,劲竿如蛇脊,屈曲撑距,意色酣怒,鳞爪拿攫,义不受制,鬣起针针,怒张如戟。旧府呼``松棚'',故松之意态情理无不棚之。便殿三楹盘郁殆遍,暗不通天,密不通雨。鲁宪王晚年好道,尝取松肘一节,抱与同卧,久则滑泽酣酡,似有血气。

\hypertarget{header-n494}{%
\subsubsection{一尺雪}\label{header-n494}}

``一尺雪''为芍药异种,余于兖州见之。花瓣纯白,无须萼,无檀心,无星星红紫,洁如羊脂,细如鹤翮,结楼吐舌,粉艳雪腴。上下四旁方三尺,干小而弱,力不能支,蕊大如芙蓉,辄缚一小架扶之。大江以南,有其名无其种,有其种无其土,盖非兖勿易见之也。兖州种芍药者如种麦,以邻以亩。花时宴客,棚于路、彩于门、衣于壁、障于屏、缀于帘、簪于席、茵于阶者,毕用之,日费数千勿惜。余昔在兖,友人日剪数百朵送寓所,堆垛狼藉,真无法处之。

\hypertarget{header-n499}{%
\subsubsection{菊海}\label{header-n499}}

兖州张氏期余看菊,去城五里。余至其园,尽其所为园者而折旋之,又尽其所不尽为园者而周旋之,绝不见一菊,异之。移时,主人导至一苍莽空地,有苇厂三间,肃余入,遍观之,不敢以菊言,真菊海也。厂三面,砌坛三层,以菊之高下高下之。花大如瓷瓯,无不球,无不甲,无不金银荷花瓣,色鲜艳,异凡本,而翠叶层层,无一早脱者。此是天道,是土力,是人工,缺一不可焉。兖州缙绅家风气袭王府,赏菊之日,其桌,其炕、其灯、其炉、其盘、其盒、其盆盎、其肴器、其杯盘大觥、其壶、其帏、其褥、其酒、其面食、其衣服花样,无不菊者。夜烧烛照之,蒸蒸烘染,较日色更浮出数层。席散,撤苇帘以受繁露。

\hypertarget{header-n504}{%
\subsubsection{曹山}\label{header-n504}}

万历甲辰,大父游曹山,大张乐于狮子岩下。石梁先生戏作山君檄讨大父,祖昭明太子语,谓若以管弦污我岩壑。大父作檄骂之,有曰:``谁云鬼刻神镂,竟是残山剩水!''石篑先生嗤石梁曰:``文人也,那得犯其锋!不若自认,以`残山剩水'四字摩崖勒之。''先辈之引重如此。曹石宕为外祖放生池,积三十余年,放生几百千万,有见池中放光如万炬烛天,鱼虾荇藻附之而起,直达天河者。余少时从先宜人至曹山庵作佛事,以大竹篰贮西瓜四,浸宕内。须臾,大声起岩下,水喷起十余丈,三小舟缆断,颠翻波中,冲击几碎。舟人急起视,见大鱼如舟,口欱四瓜,掉尾而下。

\hypertarget{header-n509}{%
\subsubsection{齐景公墓花樽}\label{header-n509}}

霞头沈佥事宦游时,有发掘齐景公墓者,迹之,得铜豆三,大花樽二。豆朴素无奇。花樽高三尺,束腰拱起,口方而敞,四面戟楞,花纹兽面,粗细得款,自是三代法物。归乾刘阳太公,余见赏识之,太公取与严,一介不敢请。及宦粤西,外母归余斋头,余拂拭之,为发异光。取浸梅花,贮水,汗下如雨,逾刻始收,花谢结子,大如雀卵。余藏之两年,太公归自粤西,稽复之,余恐伤外母意,亟归之。后为驵侩所啖,竟以百金售去,可惜!今闻在歙县某氏家庙。

\hypertarget{header-n513}{%
\subsection{卷七}\label{header-n513}}

\hypertarget{header-n515}{%
\subsubsection{西湖香市}\label{header-n515}}

西湖香市,起于花朝,尽于端午。山东进香普陀者日至,嘉湖进香天竺者日至,至则与湖之人市焉,故曰香市。然进香之人市于三天竺,市于岳王坟,市于湖心亭,市于陆宣公祠,无不市,而独凑集于昭庆寺。昭庆寺两廊故无日不市者,三代八朝之古董,蛮夷闽貊之珍异,皆集焉。至香市,则殿中边甬道上下、池左右、山门内外,有屋则摊,无屋则厂,厂外又棚,棚外又摊,节节寸寸。凡胭脂簪珥、牙尺剪刀,以至经典木鱼、伢儿嬉具之类,无不集。此时春暖,桃柳明媚,鼓吹清和,岸无留船,寓无留客,肆无留酿。袁石公所谓``山色如娥,花光如颊,温风如酒,波纹如绫'',已画出西湖三月。而此以香客杂来,光景又别。士女闲都,不胜其村妆野妇之乔画;芳兰芗泽,不胜其合香芫荽之薰蒸;丝竹管弦,不胜其摇鼓欱笙之聒帐;鼎彝光怪,不胜其泥人竹马之行情;

宋元名画,不胜其湖景佛图之纸贵。如逃如逐,如奔如追,撩扑不开,牵挽不住。数百十万男男女女、老老少少,日簇拥于寺之前后左右者,凡四阅月方罢。恐大江以东,断无此二地矣。崇祯庚辰三月,昭庆寺火。是岁及辛巳、壬午洊饥,民强半饿死。壬午虏鲠山东,香客断绝,无有至者,市遂废。辛巳夏,余在西湖,但见城中饿殍舁出,扛挽相属。时杭州刘太守梦谦,汴梁人,乡里抽丰者多寓西湖,日以民词馈送。有轻薄子改古诗诮之曰:``山不青山楼不楼,西湖歌舞一时休。暖风吹得死人臭,还把杭州送汴州。''可作西湖实录。

\hypertarget{header-n520}{%
\subsubsection{鹿苑寺方柿}\label{header-n520}}

萧山方柿,皮绿者不佳,皮红而肉糜烂者不佳,必树头红而坚脆如藕者,方称绝品。然间遇之,不多得。余向言西瓜生于六月,享尽天福;秋白梨生于秋,方柿、绿柿生于冬,未免失候。丙戌,余避兵西白山,鹿苑寺前后有夏方柿十数株。六月歊暑,柿大如瓜,生脆如咀冰嚼雪,目为之明,但无法制之,则涩勒不可入口。土人以桑叶煎汤,候冷,加盐少许,入瓮内,浸柿没其颈,隔二宿取食,鲜磊异常。余食萧山柿多涩,请赠以此法。

\hypertarget{header-n525}{%
\subsubsection{西湖七月半}\label{header-n525}}

西湖七月半,一无可看,止可看看七月半之人。看七月半之人,以五类看之。其一,楼船萧鼓,峨冠盛筵,灯火优傒,声光相乱,名为看月而实不见月者,看之。其一,亦船亦楼,名娃闺秀,携及童变,笑啼杂之,环坐露台,左右盼望,身在月下而实不看月者,看之。其一,亦船亦声歌,名妓闲僧,浅斟低唱,弱管轻丝,竹肉相发,亦在月下,亦看月,而欲人看其看月者,看之。其一,不舟不车,不衫不帻,酒醉饭饱,呼群三五,跻入人丛,昭庆、断桥,嘄呼嘈杂,装假醉,唱无腔曲,月亦看,看月者亦看,不看月者亦看,而实无一看者,看之。其一,小船轻幌,净几暖炉,茶铛旋煮,素瓷静递,好友佳人,邀月同坐,或匿影树下,或逃嚣里湖,看月而人不见其看月之态,亦不作意看月者,看之。杭人游湖,巳出酉归,避月如仇,是夕好名,逐队争出,多犒门军酒钱,轿夫擎燎,列俟岸上。一入舟,速舟子急放断桥,赶入胜会。以故二鼓以前,人声鼓吹,如沸如撼,如魇如呓,如聋如哑,大船小船一齐凑岸,一无所见,止见篙击篙,舟触舟,肩摩肩,面看面而已。少刻兴尽,官府席散,皂隶喝道去,轿夫叫船上人,怖以关门,灯笼火把如列星,------簇拥而去。岸上人亦逐队赶门,渐稀渐薄,顷刻散尽矣。吾辈始舣舟近岸,断桥石磴始凉,席其上,呼客纵饮。此时,月如镜新磨,山复整妆,湖复颒面。向之浅斟低唱者出,匿影树下者亦出,吾辈往通声气,拉与同坐。韵友来,名妓至,杯箸安,竹肉发。月色苍凉,东方将白,客方散去。吾辈纵舟,酣睡于十里荷花之中,香气拍人,清梦甚惬。

\hypertarget{header-n530}{%
\subsubsection{及时雨}\label{header-n530}}

壬申七月,村村祷雨,日日扮潮神海鬼,争唾之。余里中扮《水浒》,且曰:画《水浒》者,龙眠、松雪近章侯,总不如施耐庵,但如其面勿黛,如其髭勿鬣,如其兜鍪勿纸,如其刀杖勿树,如其传勿杜撰,勿戈阳腔,则十得八九矣。于是分头四出,寻黑矮汉,寻梢长大汉,寻头陀,寻胖大和尚,寻茁壮妇人,寻姣长妇人,寻青面,寻歪头,寻赤须,寻美髯,寻黑大汉,寻赤脸长须,大索城中。无则之郭、之村、之山僻、之邻府州县,用重价聘之,得三十六人。梁山泊好汉,个个呵活,臻臻至至,人马称娖而行,观者兜截遮拦,直欲看杀玠。五雪叔归自广陵,多购法锦宫缎,从以台阁者八:雷部六,大士一,龙宫一,华重美都,见者目夺气亦夺。盖自有台阁,有其华无其重,有其美无其都,有其华重美都,无其思致,无其文理。轻薄子有言:``不替他谦了,也事事精办。''

季祖南华老人喃喃怪问余曰:``《水浒》与祷雨有何义味?近余山盗起,迎盗何为耶?''余俯首思之,果诞而无谓,徐应之曰:``有之。天罡尽,以宿太尉殿焉。用大牌六,书`奉旨招安'者二,书`风调雨顺'者一,`盗息民安'者一,更大书`及时雨'者二,前导之。''观者欢喜赞叹,老人亦匿笑而去。

\hypertarget{header-n535}{%
\subsubsection{山艇子}\label{header-n535}}

龙山自巘花阁而西皆骨立,得其一节,亦尽名家。山艇子石,意尤孤孑,壁立霞剥,义不受土。大樟徙其上,石不容也,然不恨石,屈而下,与石相亲疏。石方广三丈,右坳而凹,非竹则尽矣,何以浅深乎石。然竹怪甚,能孤行,实不藉石。竹节促而虬叶毨毨,如猬毛、如松狗尾,离离矗矗,捎捩攒挤,若有所惊者。竹不可一世,不敢以竹二之。或曰:古今错刀也。或曰:竹生石上,土肤浅,蚀其根,故轮囷盘郁,如黄山上松。山艇子樟,始之石,中之竹,终之楼,意长楼不得竟其长,故艇之。然伤于贪,特特向石,石意反不之属,使去丈而楼壁出,樟出,竹亦尽出。竹石间意,在以淡远取之。

\hypertarget{header-n540}{%
\subsubsection{悬杪亭}\label{header-n540}}

余六岁随先君子读书于悬抄亭,记在一峭壁之下,木石撑距,不藉尺土,飞阁虚堂,延骈如栉。缘崖而上,皆灌木高柯,与檐甃相错。取杜审言``树杪玉堂悬''句,名之``悬杪'',度索寻樟,大有奇致。后仲叔庐其崖下,信堪舆家言,谓碍其龙脉,百计购之,一夜徒去,鞠为茂草。儿时怡寄,常梦寐寻往。

\hypertarget{header-n545}{%
\subsubsection{雷殿}\label{header-n545}}

雷殿在龙山磨盘冈下,钱武肃王于此建蓬莱阁,有断碣在焉。殿前石台高爽,乔木萧疏。六月,月从南来,树不蔽月。余每浴后拉秦一生、石田上人、平子辈坐台上,乘凉风,携肴核,饮香雪酒,剥鸡豆,啜乌龙井水,水凉冽激齿。下午着人投西瓜浸之,夜剖食,寒栗逼人,可雠三伏。林中多鹘,闻人声辄惊起,磔磔云霄间,半日不得下。

\hypertarget{header-n550}{%
\subsubsection{龙山雪}\label{header-n550}}

天启六年十二月,大雪深三尺许。晚霁,余登龙山,坐上城隍庙山门,李岕生、高眉生、王畹生、马小卿、潘小妃侍。万山载雪,明月薄之,月不能光,雪皆呆白。坐久清冽,苍头送酒至,余勉强举大觥敌寒,酒气冉冉,积雪欱之,竟不得醉。马小卿唱曲,李岕生吹洞箫和之,声为寒威所慑,咽涩不得出。三鼓归寝。马小卿、潘小妃相抱从百步街旋滚而下,直至山趾,浴雪而立。余坐一小羊头车,拖冰凌而归。

\hypertarget{header-n555}{%
\subsubsection{庞公池}\label{header-n555}}

庞公池岁不得船,况夜船,况看月而船。自余读书山艇子,辄留小舟于池中,月夜,夜夜出,缘城至北海坂,往返可五里,盘旋其中。山后人家,闭门高卧,不见灯火,悄悄冥冥,意颇凄恻。余设凉簟,卧舟中看月,小傒船头唱曲,醉梦相杂,声声渐远,月亦渐淡,嗒然睡去。歌终忽寤,含糊赞之,寻复鼾齁。小傒亦呵欠歪斜,互相枕藉。舟子回船到岸,篙啄丁丁,促起就寝。此时胸中浩浩落落,并无芥蒂,一枕黑甜,高舂始起,不晓世间何物谓之忧愁。

\hypertarget{header-n560}{%
\subsubsection{品山堂鱼宕}\label{header-n560}}

二十年前强半住众香国,日进城市,夜必出之。品山堂孤松箕踞,岸帻入水。池广三亩,莲花起岸,莲房以百以千,鲜磊可喜。新雨过,收叶上荷珠煮酒,香扑烈。门外鱼宕,横亘三百余亩,多种菱芡。小菱如姜芽,辄采食之,嫩如莲实,香似建兰,无味可匹。深秋,橘奴饱霜,非个个红绽不轻下剪。季冬观鱼,鱼艓千余艘,鳞次栉比,罱者夹之,罛者扣之,簎者罨之,罥者撒之,罩者抑之,罣者举之,水皆泥泛,浊如土浆。鱼入网者圉圉,漏网者圉圉,寸鲵纤鳞,无不毕出。集舟分鱼,鱼税三百余斤,赤魚白肚,满载而归。约吾昆弟,烹鲜剧饮,竟日方散。

\hypertarget{header-n565}{%
\subsubsection{松花石}\label{header-n565}}

松花石,大父舁自潇江署中。石在江口神祠,土人割牲飨神,以毛血洒石上为恭敬,血渍毛毵,几不见石。大父舁入署,亲自祓濯,呼为``石丈'',有《松花石纪》。今弃阶下,载花缸,不称使。余嫌其轮囷臃肿,失松理,不若董文简家茁错二松橛,节理槎枒,皮断犹附,视此更胜。大父石上磨崖铭之曰:``尔昔鬣而鼓兮,松也;尔今脱而骨兮,石也;尔形可使代兮,贞勿易也;尔视余笑兮,莫余逆也。''其见宝如此。

\hypertarget{header-n570}{%
\subsubsection{闰中秋}\label{header-n570}}

崇祯七年闰中秋,仿虎丘故事,会各友于蕺山亭。每友携斗酒、五簋、十蔬果、红毡一床,席地鳞次坐。缘山七十余床,衰童塌妓,无席无之。在席者七百余人,能歌者百余人,同声唱``澄湖万顷'',声如潮涌,山为雷动。诸酒徒轰饮,酒行如泉。夜深客饥,借戒珠寺斋僧大锅煮饭饭客,长年以大桶担饭不继。命小傒岕竹、楚烟于山亭演剧十余出,妙入情理,拥观者千人,无蚊虻声,四鼓方散。月光泼地如水,人在月中,濯濯如新出浴。夜半,白云冉冉起脚下,前山俱失,香炉、鹅鼻、天柱诸峰,仅露髻尖而已,米家山雪景仿佛见之。

\hypertarget{header-n575}{%
\subsubsection{愚公谷}\label{header-n575}}

无锡去县北五里为铭山。进桥,店在左岸,店精雅,卖泉酒水坛、花缸、宜兴罐、风炉、盆盎、泥人等货。愚公谷在惠山右,屋半倾圮,惟存木石。惠水涓涓,由井之涧,由涧之溪,由溪之池、之厨、之湢,以涤、以濯、以灌园、以沐浴、以净溺器,无不惠山泉者,故居园者福德与罪孽正等。

愚公先生交游遍天下,名公巨卿多就之,歌儿舞女、绮席华筵、诗文字画,无不虚往实归。名士清客至则留,留则款,款则饯,饯则赆。以故愚公之用钱如水,天下人至今称之不少衰。愚公文人,其园亭实有思致文理者为之,磥石为垣,编柴为户,堂不层不庑,树不配不行。堂之南,高槐古朴,树皆合抱,茂叶繁柯,阴森满院。藕花一塘,隔岸数石,治而卧。土墙生苔,如山脚到涧边,不记在人间。园东逼墙一台,外瞰寺,老柳卧墙角而不让台,台遂不尽瞰,与他园花树故故为亭、台意特特为园者不同。

\hypertarget{header-n580}{%
\subsubsection{定海水操}\label{header-n580}}

定海演武场在招宝山海岸。水操用大战船、唬船、蒙冲、斗舰数千余艘,杂以鱼艓轻艖,来往如织。舳舻相隔,呼吸难通,以表语目,以鼓语耳,截击要遮,尺寸不爽。健儿瞭望,猿蹲桅斗,哨见敌船,从斗上掷身腾空溺水,破浪冲涛,顷刻到岸,走报中军,又趵跃入水,轻如鱼凫。水操尤奇在夜战,旌旗干橹皆挂一小镫,青布幕之,画角一声,万蜡齐举,火光映射,影又倍之。招宝山凭槛俯视,如烹斗煮星,釜汤正沸。火炮轰裂,如风雨晦冥中电光翕焱,使人不敢正视;又如雷斧断崖石,下坠不测之渊,观者褫魄。

\hypertarget{header-n585}{%
\subsubsection{阿育王寺舍利}\label{header-n585}}

阿育王寺,梵宇深静,阶前老松八九棵,森罗有古色。殿隔山门远,烟光树樾,摄入山门,望空视明,冰凉晶沁。右旋至方丈门外,有娑罗二株,高插霄汉。便殿供旃檀佛,中储一铜塔,铜色甚古,万历间慈圣皇太后所赐,藏舍利子塔也。舍利子常放光,琉璃五彩,百道迸裂,出塔缝中,岁三四见。凡人瞻礼舍利,随人因缘现诸色相。如墨墨无所见者,是人必死。昔湛和尚至寺,亦不见舍利,而是年死。屡有验。

次早,日光初曙,僧导余礼佛,开铜塔,一紫檀佛龛供一小塔,如笔筒,六角,非木非楮,非皮非漆,上下皲定,四围镂刻花楞梵字。舍利子悬塔顶,下垂摇摇不定,人透眼光入楞内,复目氐眼上视舍利,辨其形状。余初见三珠连络如牟尼串,煜煜有光。余复下顶礼,求见形相,再视之,见一白衣观音小像,眉目分明,鬋鬘皆见。秦一生反复视之,讫无所见,一生遑邃,面发赤,出涕而去。一生果以是年八月死,奇验若此。

\hypertarget{header-n590}{%
\subsubsection{过剑门}\label{header-n590}}

南曲中妓,以串戏为韵事,性命以之。杨元、杨能、顾眉生、李十、董白以戏名,属姚简叔期余观剧。傒僮下午唱《西楼》,夜则自串。傒僮为兴化大班,余旧伶马小卿、陆子云在焉,加意唱七出,戏至更定,曲中大咤异。杨元走鬼房问小卿曰:``今日戏,气色大异,何也?''小卿曰:``坐上坐者余主人。主人精赏鉴,延师课戏,童手指千,傒僮到其家谓`过剑门',焉敢草草!''杨元始来物色余。《西楼》不及完,串《教子》。顾眉生:周羽,杨元:周娘子,杨能:周瑞隆。杨元胆怯肤栗,不能出声,眼眼相觑,渠欲讨好不能,余欲献媚不得,持久之,伺便喝采一二,杨元始放胆,戏亦遂发。嗣后曲中戏,必以余为导师,余不至,虽夜分不开台也。以余而长声价,以余长声价之人、而后长余声价者,多有之。

\hypertarget{header-n595}{%
\subsubsection{冰山记}\label{header-n595}}

魏珰败,好事者作传奇十数本,多失实,余为删改之,仍名《冰山》。城隍庙扬台,观者数万人,台址鳞比,挤至大门外。一人上,白曰:``某杨涟。''口口谇(言察)曰:``杨涟!杨涟!''

声达外,如潮涌,人人皆如之。杖范元白,逼死裕妃,怒气忿涌,噤断嚄唶。至颜佩韦击杀缇骑,嘄呼跳蹴,汹汹崩屋。

沈青霞缚橐人射相嵩,以为笑乐,不是过也。是秋,携之至兖,为大人寿。一日,宴守道刘半舫,半舫曰:``此剧已十得八九,惜不及内操菊宴、及逼灵犀与囊收数事耳。''余闻之,是夜席散,余填词,督小傒强记之。次日,至道署搬演,已增入七出,如半舫言。半舫大骇异,知余所构,遂诣大人,与余定交。

\hypertarget{header-n599}{%
\subsection{卷八}\label{header-n599}}

\hypertarget{header-n601}{%
\subsubsection{龙山放灯}\label{header-n601}}

万历辛丑年,父叔辈张灯龙山,剡木为架者百,涂以丹雘,悦以文锦,一灯三之。灯不专在架,亦不专在磴道,沿山袭谷,枝头树杪无不灯者,自城隍庙门至蓬莱岗上下,亦无不灯者。山下望如星河倒注,浴浴熊熊,又如隋炀帝夜游,倾数斛萤火于山谷间,团结方开,倚草附木,迷迷不去者。好事者卖酒,缘出席地坐。山无不灯,灯无不席,席无不人,人无不歌唱鼓吹。男女看灯者,一入庙门,头不得顾,踵不得旋,只可随势潮上潮下,不知去落何所,有听之而已。庙门悬禁条:禁车马,禁烟火,禁喧哗,禁豪家奴不得行辟人。父叔辈台于大松树下,亦席,亦声歌,每夜鼓吹笙簧与宴歌弦管,沉沉昧旦。十六夜,张分守宴织造太监于山巅星宿阁,傍晚至山下,见禁条,太监忙出舆笑曰:``遵他,遵他,自咱们遵他起!''却随役,用二丱角扶掖上山。夜半,星宿阁火罢,宴亦遂罢。灯凡四夜,山上下糟丘肉林,日扫果核蔗滓及鱼肉骨蠡蜕,堆砌成高阜,拾妇女鞋挂树上,如秋叶。相传十五夜,灯残人静,当垆者正收盘核,有美妇六七人买酒,酒尽,有未开瓮者。买大罍一,可四斗许,出袖中瓜果,顷刻罄罍而去。疑是女人星,或曰酒星。又一事:有无赖子于城隍庙左借空楼数楹,以姣童实之,为``帘子胡同''。是夜,有美少年来狎某童,剪烛殢酒,媟亵非理,解襦,乃女子也,未曙即去,不知其地、其人,或是妖狐所化。

\hypertarget{header-n606}{%
\subsubsection{王月生}\label{header-n606}}

南京朱市妓,曲中羞与为伍;王月生出朱市,曲中上下三十年决无其比也。面色如建兰初开,楚楚文弱,纤趾一牙,如出水红菱,矜贵寡言笑,女兄弟闲客多方狡狯嘲弄咍侮,不能勾其一粲。善楷书,画兰竹水仙,亦解吴歌,不易出口。南京勋戚大老力致之,亦不能竟一席。富商权胥得其主席半晌,先一日送书帕,非十金则五金,不敢亵订。与合卺,非下聘一二月前,则终岁不得也。好茶,善闵老子,虽大风雨、大宴会,必至老子家啜茶数壶始去。所交有当意者,亦期与老子家会。一日,老子邻居有大贾,集曲中妓十数人,群谇嘻笑,环坐纵饮。月生立露台上,倚徙栏楯,目氐娗羞涩,群婢见之皆气夺,徙他室避之。月生寒淡如孤梅冷月,含冰傲霜,不喜与俗子交接;或时对面同坐起,若无睹者。有公子狎之,同寝食者半月,不得其一言。一日口嗫嚅动,闲客惊喜,走报公子曰:``月生开言矣!''哄然以为祥瑞,急走伺之,面赪,寻又止,公子力请再三,蹇涩出二字曰:``家去。''

\hypertarget{header-n611}{%
\subsubsection{张东谷好酒}\label{header-n611}}

余家自太仆公称豪饮,后竟失传,余父余叔不能饮一蠡壳,食糟茄,面即发赪,家常宴会,但留心烹饪,庖厨之精,遂甲江左。一簋进,兄弟争啖之立尽,饱即自去,终席未尝举杯。有客在,不待客辞,亦即自去。山人张东谷,酒徒也,每悒悒不自得。一日起谓家君曰:``尔兄弟奇矣!肉只是吃,不管好吃不好吃;酒只是不吃,不知会吃不会吃。''二语颇韵,有晋人风味。而近有伧父载之《舌华录》,曰:``张氏兄弟赋性奇哉!肉不论美恶,只是吃;酒不论美恶,只是不吃。''字字板实,一去千里,世上真不少点金成铁手也。东谷善滑稽,贫无立锥,与恶少讼,指东谷为万金豪富,东谷忙忙走诉大父曰:``绍兴人可恶,对半说谎,便说我是万金豪富!''大父常举以为笑。

\hypertarget{header-n616}{%
\subsubsection{楼船}\label{header-n616}}

家大人造楼,船之;造船,楼之。故里中人谓船楼,谓楼船,颠倒之不置。是日落成,为七月十五,自大父以下,男女老稚靡不集焉。以木排数重搭台演戏,城中村落来观者,大小千余艘。午后飓风起,巨浪磅礴,大雨如注,楼船孤危,风逼之几覆,以木排为戙索缆数千条,网网如织,风不能撼。少顷风定,完剧而散。越中舟如蠡壳,局蹐篷底看山,如矮人观场,仅见鞋靸而已,升高视明,颇为山水吐气。

\hypertarget{header-n621}{%
\subsubsection{阮圆海戏}\label{header-n621}}

阮圆海家优,讲关目,讲情理,讲筋节,与他班孟浪不同。然其所打院本,又皆主人自制,笔笔勾勒,苦心尽出,与他班卤莽者又不同。故所搬演,本本出色,脚脚出色,出出出色,句句出色,字字出色。余在其家看《十错认》、《摩尼珠》、《燕子笺》三剧,其串架斗笋、插科打诨、意色眼目,主人细细与之讲明。知其义味,知其指归,故咬嚼吞吐,寻味不尽。至于《十错认》之龙灯、之紫姑,《摩尼珠》之走解、之猴戏,《燕子笺》之飞燕、之舞象、之波斯进宝,纸札装束,无不尽情刻画,故其出色也愈甚。阮圆海大有才华,恨居心勿静,其所编诸剧,骂世十七,解嘲十三,多诋毁东林,辩宥魏党,为士君子所唾弃,故其传奇不之著焉。如就戏论,则亦镞镞能新,不落窠臼者也。

\hypertarget{header-n626}{%
\subsubsection{巘花阁}\label{header-n626}}

巘花阁在筠芝亭松峡下,层崖古木,高出林皋,秋有红叶。坡下支壑回涡,石拇棱棱,与水相距。阁不槛、不牖,地不楼、不台,意正不尽也。五雪叔归自广陵,一肚皮园亭,于此小试。台之、亭之、廊之、栈道之,照面楼之侧,又堂之、阁之、梅花缠折旋之,未免伤板、伤实、伤排挤,意反局蹐,若石窟书砚。隔水看山、看阁、看石麓、看松峡上松,庐山面目反于山外得之。五雪叔属余作对,余曰:``身在襄阳袖石里,家来辋口扇图中。''言其小处。

\hypertarget{header-n631}{%
\subsubsection{范与兰}\label{header-n631}}

范与兰七十有三,好琴,喜种兰及盆池小景。建兰三十余缸,大如簸箕。早舁而入,夜异而出者,夏也;早舁而出,夜舁而入者,冬也;长年辛苦,不减农事。花时,香出里外,客至坐一时,香袭衣裾,三五日不散。余至花期至其家,坐卧不去,香气酷烈,逆鼻不敢嗅,第开口吞欱之,如流瀣焉。

花谢,粪之满箕,余不忍弃,与与兰谋曰:``有面可煎,有蜜可浸,有火可焙,奈何不食之也?''与兰首肯余言。与兰少年学琴于王明泉,能弹《汉宫秋》、《山居吟》、《水龙吟》三曲。

后见王本吾琴,大称善,尽弃所学而学焉,半年学《石上流泉》一曲,生涩犹棘手。王本吾去,旋亦忘之,旧所学又锐意去之,不复能记忆,究竟终无一字,终日抚琴,但和弦而已。所畜小景,有豆板黄杨,枝干苍古奇妙,盆石称之。朱樵峰以二十金售之,不肯易,与兰珍爱,``小妾''呼之。余强借斋头三月,枯其垂一干,余懊惜,急舁归与兰。与兰惊惶无措,煮参汁浇灌,日夜摩之不置,一月后枯干复活。

\hypertarget{header-n636}{%
\subsubsection{蟹会}\label{header-n636}}

食品不加盐醋而五味全者,为蚶、为河蟹。河蟹至十月与稻梁俱肥,壳如盘大,坟起,而紫螯巨如拳,小脚肉出,油油如螾愆。掀其壳,膏腻堆积,如玉脂珀屑,团结不散,甘腴虽八珍不及。一到十月,余与友人兄弟辈立蟹会,期于午后至,煮蟹食之,人六只,恐冷腥,迭番煮之。从以肥腊鸭、牛乳酪。醉蚶如琥珀,以鸭汁煮白菜如玉版。果瓜以谢橘、以风栗、以风菱。饮以玉壶冰,蔬以兵坑笋,饭以新余杭白,漱以兰雪茶。由今思之,真如天厨仙供,酒醉饭饱,惭愧惭愧。

\hypertarget{header-n641}{%
\subsubsection{露兄}\label{header-n641}}

崇祯癸酉,有好事者开茶馆,泉实玉带,茶实兰雪,汤以旋煮,无老汤,器以时涤,无秽器,其火候、汤候,亦时有天合之者。余喜之,名其馆曰``露兄'',取米颠``茶甘露有兄''句也。为之作《斗茶檄》,曰:``水淫茶癖,爰有古风;瑞草雪芽,素称越绝。特以烹煮非法,向来葛灶生尘;更兼赏鉴无人,致使羽《经》积蠹。迩者择有胜地,复举汤盟,水符递自玉泉,茗战争来兰雪。瓜子炒豆,何须瑞草桥边;橘柚查梨,出自仲山圃内。八功德水,无过甘滑香洁清凉;七家常事,不管柴米油盐酱醋。一日何可少此,子猷竹庶可齐名;七碗吃不得了,卢仝茶不算知味。一壶挥塵,用畅清谈;半榻焚香,共期白醉。''

\hypertarget{header-n646}{%
\subsubsection{闰元宵}\label{header-n646}}

崇祯庚辰闰正月,与越中父老约重张五夜灯,余作张灯致语曰:``两逢元正,岁成闰于摄提之辰;再值孟陬,天假人以闲暇之月。《春秋传》详记二百四十二年事,春王正月,孔子未得重书;开封府更放十七、十八两夜灯,乾德五年,宋祖犹烦钦赐。兹闰正月者,三生奇遇,何幸今日而当场;百岁难逢,须效古人而秉烛。况吾大越,蓬莱福地,宛委洞天。

大江以东,民皆安堵;遵海而北,水不扬波。含哺嬉兮,共乐太平之世界;重译至者,皆言中国有圣人。千百国来朝,白雉之陈无算;十三年于兹,黄耇之说有征。乐圣衔杯,宜纵饮屠苏之酒;较书分火,应暂辍太乙之藜。前此元宵,竟因雪妒,天亦知点缀丰年;后来灯夕,欲与月期,人不可蹉跎胜事。六警山立,只说飞来东武,使鸡犬不惊;百兽室悬,毋曰下守海澨,唯鱼鳖是见。笙箫聒地,竹椽出自柯亭;花草盈街,禊帖携来兰渚。士女潮涌,撼动蠡城;车马雷殷,唤醒龙屿。况时逢丰穰,呼庚呼癸,一岁自兆重登;且科际辰年,为龙为光,两榜必征双首。莫轻此五夜之乐,眼望何时?试问那百年之人,躬逢几次?敢祈同志,勿负良宵。敬藉赫蹄,喧传口号。''

\hypertarget{header-n651}{%
\subsubsection{合采牌}\label{header-n651}}

余作文武牌,以纸易骨,便于角斗,而燕客复刻一牌,集天下之斗虎、斗鹰、斗豹者,而多其色目、多其采,曰``合采牌''。余为之作叙曰:``太史公曰:`凡编户之民,富相什则卑下之,伯则畏惮之,千则役,万则仆,物之理也。'古人以钱之名不雅驯,缙绅先生难道之,故易其名曰赋、曰禄、曰饷,天子千里外曰采。采者,采其美物以为贡,犹赋也。诸侯在天子之县内曰采,有地以处其子孙亦曰采,名不一,其实皆谷也,饭食之谓也。周封建多采则胜,秦无采则亡。采在下无以合之,则齐桓、晋文起矣。列国有采而分析之,则主父偃之谋也。由是而亮采服采,好官不过多得采耳。充类至义之尽,窃亦采也,盗亦采也,鹰虎豹由此其选也。然则奚为而不禁?曰:小役大,弱役强,斯二者天也。《皋陶谟》曰:`载采采',微哉、之哉、庶哉!''

\hypertarget{header-n656}{%
\subsubsection{瑞草溪亭}\label{header-n656}}

瑞草溪亭为龙山支麓,高与屋等。燕客相其下有奇石,身执蔓臿,为匠石先,发掘之。见土葢土,见石甃石,去三丈许,始与基平,乃就其上建屋。屋今日成,明日拆,后日又成,再后日又拆,凡十七变而溪亭始出。盖此地无溪也,而溪之,溪之不足,又潴之、壑之,一日鸠工数千指,索性池之,索性阔一亩,索性深八尺。无水,挑水贮之,中留一石如案,回潴浮峦,颇亦有致。燕客以山石新开,意不苍古,乃用马粪涂之,使长苔藓,苔藓不得即出,又呼画工以石青石绿皴之。一日左右视,谓此石案焉可无天目松数棵盘郁其上,遂以重价购天目松五六棵,凿石种之。石不受锸,石崩裂,不石不树,亦不复案,燕客怒,连夜凿成砚山形,缺一角,又葢一岩石补之。燕客性卞急,种树不得大,移大树种之,移种而死,又寻大树补之。种不死不已,死亦种不已,以故树不得不死,然亦不得即死。溪亭比旧址低四丈,运土至东多成高山,一亩之室,沧桑忽变。见其一室成,必多坐看之,至隔宿或即无有矣。故溪亭虽渺小,所费至巨万焉。燕客看小说:``姚崇梦游地狱,至一大厂,炉鞴千副,恶鬼数千,铸泻甚急,问之,曰:`为燕国公铸横财。'后至一处,炉灶冷落,疲鬼一二人鼓橐,奄奄无力,崇问之,曰:`此相公财库也。'崇寤而叹曰:`燕公豪奢,殆天纵也。'''燕客喜其事,遂号``燕客''。二叔业四五万,燕客缘手立尽。甲申,二叔客死淮安,燕客奔丧,所积薪俸及玩好币帛之类又二万许,燕客携归,甫三月又辄尽,时人比之鱼宏四尽焉。溪亭住宅,一头造,一头改,一头卖,翻山倒水无虚日。有夏耳金者,制灯剪彩为花,亦无虚日。人称耳金为``败落隋炀帝'',称燕客为``穷极秦始皇'',可发一粲。

\hypertarget{header-n661}{%
\subsubsection{琅嬛福地}\label{header-n661}}

陶庵梦有夙因,常梦至一石厂,峥窅岩岪,前有急湍洄溪,水落如雪,松石奇古,杂以名花。梦坐其中,童子进茗果,积书满架,开卷视之,多蝌蚪、鸟迹、霹雳篆文,梦中读之,似能通其棘涩。闲居无事,夜辄梦之,醒后伫思,欲得一胜地仿佛为之。郊外有一小山,石骨棱砺,上多筠篁,偃伏园内。余欲造厂,堂东西向,前后轩之,后磥一石坪,植黄山松数棵,奇石峡之。堂前树娑罗二,资其清樾。左附虚室,坐对山麓,磴磴齿齿,划裂如试剑,匾曰``一丘''。右踞厂阁三间,前临大沼,秋水明瑟,深柳读书,匾曰``一壑''。

缘山以北,精舍小房,绌屈蜿蜒,有古木,有层崖,有小涧,有幽篁,节节有致。山尽有佳穴,造生圹,俟陶庵蜕焉,碑曰``呜呼有明陶庵张长公之圹''。圹左有空地亩许,架一草庵,供佛,供陶庵像,迎僧住之奉香火。大沼阔十亩许,沼外小河三四折,可纳舟入沼。河两崖皆高阜,可植果木,以橘、以梅、以梨、以枣,枸菊围之。山顶可亭。山之西鄙,有腴田二十亩,可秫可秔。门临大河,小楼翼之,可看炉峰、敬亭诸山。楼下门之,匾曰``琅嬛福地''。缘河北走,有石桥极古朴,上有灌木,可坐、可风、可月。

\end{document}
