\PassOptionsToPackage{unicode=true}{hyperref} % options for packages loaded elsewhere
\PassOptionsToPackage{hyphens}{url}
%
\documentclass[]{article}
\usepackage{lmodern}
\usepackage{amssymb,amsmath}
\usepackage{ifxetex,ifluatex}
\usepackage{fixltx2e} % provides \textsubscript
\ifnum 0\ifxetex 1\fi\ifluatex 1\fi=0 % if pdftex
  \usepackage[T1]{fontenc}
  \usepackage[utf8]{inputenc}
  \usepackage{textcomp} % provides euro and other symbols
\else % if luatex or xelatex
  \usepackage{unicode-math}
  \defaultfontfeatures{Ligatures=TeX,Scale=MatchLowercase}
\fi
% use upquote if available, for straight quotes in verbatim environments
\IfFileExists{upquote.sty}{\usepackage{upquote}}{}
% use microtype if available
\IfFileExists{microtype.sty}{%
\usepackage[]{microtype}
\UseMicrotypeSet[protrusion]{basicmath} % disable protrusion for tt fonts
}{}
\IfFileExists{parskip.sty}{%
\usepackage{parskip}
}{% else
\setlength{\parindent}{0pt}
\setlength{\parskip}{6pt plus 2pt minus 1pt}
}
\usepackage{hyperref}
\hypersetup{
            pdfborder={0 0 0},
            breaklinks=true}
\urlstyle{same}  % don't use monospace font for urls
\setlength{\emergencystretch}{3em}  % prevent overfull lines
\providecommand{\tightlist}{%
  \setlength{\itemsep}{0pt}\setlength{\parskip}{0pt}}
\setcounter{secnumdepth}{0}
% Redefines (sub)paragraphs to behave more like sections
\ifx\paragraph\undefined\else
\let\oldparagraph\paragraph
\renewcommand{\paragraph}[1]{\oldparagraph{#1}\mbox{}}
\fi
\ifx\subparagraph\undefined\else
\let\oldsubparagraph\subparagraph
\renewcommand{\subparagraph}[1]{\oldsubparagraph{#1}\mbox{}}
\fi

% set default figure placement to htbp
\makeatletter
\def\fps@figure{htbp}
\makeatother


\date{}

\begin{document}

\hypertarget{header-n5}{%
\section{随园食单}\label{header-n5}}

\begin{center}\rule{0.5\linewidth}{\linethickness}\end{center}

\hypertarget{header-n8}{%
\subsection{序}\label{header-n8}}

诗人美周公而日``笾豆有践'',恶凡伯而曰``彼疏斯稗''。古之于饮食也若是重乎?他若《易》称``鼎烹'',《书》称``盐梅'',《乡党》、《内则》琐琐言之。孟子虽贱``饮食之人'',而又言饥渴未能得饮食之正。可见凡事须求一是处,都非易言。《中庸》曰:``人莫不饮食也,鲜能知味也。''《典论》日:``一世长者知居处,三世长者知服食。''古人进离肺,皆有法焉,未尝苟且。``子与人歌而善,必使反之,而后和之。''圣人于一艺之微,其善取于人也如是。

余雅慕此旨,每食于某氏而饱,必使家厨往彼灶觚,执弟子之礼。四十年来,颇集众美。有学就者,有十分中得六七者,有仅得二三者,亦有竟失传者。余都问其方略,集而存之。虽不甚省记,亦载某家某味,以志景行。自觉好学之心,理宜如是。虽死法不足以限生厨,名手作书,亦多出入,未可专求之于故纸;然能率由;日章,终元大谬,临时治具,亦易指名。

或曰:``人心不同,各如其面。子能必天下之口,皆子之口乎?''曰:``执柯以伐柯,其则不远。吾虽不能强天下之口与吾同嗜,而姑且推己及物;则食饮虽微,而吾于忠恕之道,则已尽矣。吾何憾哉!''若夫《说郛》所载饮食之书三十余种,眉公。笠翁,亦有陈言。曾亲试之,皆阔于鼻而蜇于口,大半陋儒附会,吾无取焉。

\hypertarget{header-n9}{%
\subsection{须知单}\label{header-n9}}

学问之道,先知而后行,饮食亦然。作《须知单》。

\textbf{先天须知}
凡物各有先天,如人各有资禀。人性下愚,虽孔、孟教之,元益也;物性不良,虽易牙烹之,亦元味也。指其大略:猪宜皮薄,不可腥臊;鸡宜骗嫩,不可老稚;鲫鱼以扁身白肚为佳,乌背者,必崛强于盘中;鳗鱼以湖溪游泳为贵,江生者,必搓讶其骨节;谷喂之鸭,其膘肥而白色;奎土之笋,其节少而甘鲜;同一火腿也,而好丑判若天渊;同一台鳖也,而美恶分为冰炭;其他杂物,可以类推。大抵一席佳肴,司厨之功居其六,买办之功居其四。

\textbf{作料须知}
厨者之作料,如妇人之衣服首饰也。虽有大姿,虽善涂抹,而敝衣蓝缕,西子亦难以为容。善烹调者,酱用伏酱,先尝甘否;油用香油,须审生熟;酒用酒酿,应去糟粕;醋用米醋,须求清例。且酱有清浓之分,油有荤素之别,酒有酸甜之异,醋有陈新之殊,不可丝毫错误。其他葱、椒、姜、桂、糖、盐,虽用之不多,而俱宜选择上品。苏州店卖秋油,有上、中。下三等。镇江醋颜色虽佳,味不甚酸,失醋之本旨矣。以板浦醋为第一,浦口醋次之。

\textbf{洗刷须知}
洗刷之法,燕窝去毛,海参去泥,鱼翅去沙,鹿筋去臊。肉有筋瓣,剔之则酥;鸭有肾臊,削之则净;鱼胆破,而全盘皆苦;鳗涎存,而满碗多腥;韭删叶而白存,菜弃边而心出。《内则》曰:``鱼去乙,鳖去丑。''此之谓也。谚云:``若要鱼好吃,洗得白筋出。''亦此之谓也。

\textbf{调剂须知}
调剂之法,相物而施。有酒水兼用者,有专用酒不用水者,有专用水不用酒者;有盐酱井用者,有专用清酱不用盐者,有用盐不用酱者;有物太腻,要用油先炙者;有气大腥,要用醋先喷者;有取鲜必用冰糖者;有以干燥为贵者,使其味人于内,煎炒之物是也;有以汤多为贵者,使其味溢于外,清浮之物是也。

\textbf{配搭须知}
谚曰:``相女配夫。''《记》曰:``{[}亻疑{]}人必于其伦。''烹调之法,何以异焉?凡一物烹成,必需辅佐。要使清者配清,浓者配浓,柔者配柔,刚者配刚,方有和合之妙。其中可荤可素者,蘑菇、鲜笋、冬瓜是也。可荤不可素者,葱韭、茴香、新蒜是也。可素不可荤者,芹菜、百合、刀豆是也。常见人置蟹粉于燕窝之中,放百合于鸡、猪之肉,毋乃唐尧与苏峻对坐,不太悻乎?亦有交互见功者,炒荤莱,用素油,炒素菜,用荤油是也。

\textbf{独用须知}
味太浓重者,只宜独用,不可搭配。如李赞皇、张江陵一流,须专用之,方尽其才。食物中,鳗也,鳖也,蟹也,鲥鱼也,牛羊也,皆宜独食,不可加搭配。何也?此数物者味甚厚,力量甚大,而流弊亦甚多,用五味调和,全力治之,方能取其长而去其弊。何暇舍其本题,别生枝节哉?金陵人好以海参配甲鱼,鱼翅配蟹粉,我见辄攒眉。觉甲鱼、蟹粉之味,海参、鱼翅分之而不足;海参、鱼翅之弊,甲鱼、蟹粉染之而有余。

\textbf{火候须知}
熟物之法,最重火候。有须武火者,煎炒是也,火弱则物疲矣。有须文火者,煨煮是也,火猛则物枯矣。有先用武火而后用文火者,收汤之物是也;性急则皮焦而里不熟矣。有愈煮愈嫩者,腰子、鸡蛋之类是也。有略煮即不嫩者,鲜鱼、酣蛤之类是也。肉起迟则红色变黑,鱼起迟则活肉变死。屡开锅盖,则多沫而少香。火熄再烧,则无油而味失。道人以丹成九转为仙,儒家以无过、不及为中。司厨者,能知火候而谨伺之,则几于道矣。鱼临食时,色白如玉,凝而不散者,活肉也;色白如粉,不相胶粘者,死肉也。明明鲜鱼,而使之不鲜,可恨己极。

\textbf{色臭须知}
目与鼻,口之邻也,亦口之媒介也。嘉肴到目。到鼻,色臭便有不同。或净若秋云,或艳如琥珀,其芬芳之气亦扑鼻而来,不必齿决之,舌尝之,而后知其妙也。然求色艳不可用糖炒,求香不可用香料。一涉粉饰便伤至味。

\textbf{迟速须知}
凡人请客,相约于三日之前,自有工夫平章百味。若斗然客至,急需便餐;作客在外,行船落店,此何能取东海之水,救南池之焚乎?必须预备一种急就章之菜,如炒鸡片,炒肉丝,炒虾米豆腐及糟鱼、茶腿之类,反能因速而见巧者,不可不知。

\textbf{变换须知}
一物有一物之味,不可混而同之。犹如圣人设教,因才乐育,不拘一律。所谓君子成人之美也。今见俗厨,动以鸡,鸭、猪、鹅一汤同滚,遂令千手雷同,味同嚼蜡。吾恐鸡、猪。鹅、鸭有灵,必到在死城中告状矣。善治菜者,须多设锅、灶、盂、钵之类,使一物各献一性,一碗各成一味。嗜者舌本应接不暇,自觉心花顿开。

\textbf{器具须知}
古语云:美食不如美器。斯语是也。然宣、成、嘉、万窑器太贵,颇愁损伤,不如竟用御窑:已觉雅丽。惟是宜碗者碗,宜盘者盘,宜大者大,宜小者小,参错其间,方觉生色。若板板于十碗八盘之说,便嫌笨俗。大抵物贵者器宜大,物贱者器宜小。煎炒宜盘,汤羹宜碗,煎炒宜铁锅,煨煮宜砂罐。

\textbf{上菜须知}
上莱之法,盐者宜先,淡者宜后;浓者宜先,薄者宜后;无汤者宜先,有汤者宜后。且天下原有五味,不可以咸之一味概之。度客食饱,则脾困矣,须用辛辣以振动之;虑客酒多,则胃疲矣,须用酸甘以提醒之。

\textbf{时节须知}
夏日长而热,宰杀太早,则肉败矣。冬日短而寒,烹饪稍迟,则物生矣。冬宜食牛羊,移之于夏,非其时也。夏宜食干腊,移之于冬,非其时也。辅佐之物,夏宜用芥末,冬宜用胡椒。当三伏大而得冬腌菜,贱物也,而竟成至宾矣。当秋凉时而得行鞭笋,亦贱物也,而视若珍馐矣。有先时而见好者,三月食鲥鱼是也。有后时而见好者,四月食芋芳是也。其他亦可类推。有过时而不可吃者,萝卜过时则心空,山笋过时则味苦,刀鲚过时则骨硬。所谓四时之序,成功者退,精华已竭,褰裳去之也。

\textbf{多寡须知}
用贵物宜多,用贱物宜少。煎炒之物多,则火力不透,肉亦不松。故用肉不得过半斤,用鸡、鱼不得过六两。或问:食之不足如何?曰:俟食毕后另炒可也。以多为贵者,白煮肉,非二十斤以外,则淡而无味。粥亦然,非斗米则汁浆不厚,且须扣水,水多物少,则味亦薄矣。

\textbf{洁净须知}
切葱之刀,不可以切笋;捣椒之臼,不可以捣粉。闻菜有抹布气者,由其布之不洁也;闻菜有砧板气者,由其板之不净也。``工欲善其事,必先利其器。''良厨先多磨刀,多换布,多刮板,多洗手,然后治莱。至于口吸之烟灰,头上之汗汁,灶上之蝇蚁,锅上之烟煤,一玷入菜中,虽绝好烹庖,如西子蒙不洁,人皆掩鼻而过之矣。

\textbf{用纤须知}
俗名豆粉为纤者,即拉船用纤也,须顾名思义。因治肉者要作团而不能合,要作羹而不能腻,故用粉以牵合之。煎炒之时,虑肉贴锅,必至焦老,故用粉以护持之。此纤义也。能解此义用纤,纤必恰当,否则乱用可笑,但觉一片糊涂。汉制考齐呼曲麸为媒,媒即纤矣。

\textbf{选用须知}
选用之法,小炒肉用后臀,做肉圆用前夹心,煨肉用硬短勒。炒鱼片用青鱼、季鱼,做鱼松用{[}鱼军{]}鱼,鲤鱼。蒸鸡用雏鸡,煨鸡用骟鸡,取鸡汁用老鸡;鸡用雌才嫩,鸭用雄才肥;药菜用头,芹韭用根;皆一定之理。余可类推。

\textbf{疑似须知}
味要浓厚,不可油腻;味要清鲜,不可淡薄。此疑似之间,差之毫厘,失以千里。浓厚者,取精多而糟粕去之谓也;若徒贪肥腻,不如专食猪油矣。清鲜者,真味出而俗尘无之谓也;若徒贪淡薄,则不如饮水矣。

\textbf{补救须知}
名手调羹,咸淡合宜,老嫩如式,原元需补救。不得已为中人说法,则调味者,宁淡毋咸;淡可加盐以救之,咸则不能使之再淡矣。烹鱼者火候以补之,老则不能强之再嫩矣。比中消息,于一切下作料时,静观火色便可参详。

\textbf{本分须知}
满洲菜多烧煮,汉人菜多羹汤,童而习之,故擅长也。汉请满人,满请汉人,各用所长之菜,转觉人口新鲜,不失邯郸故步。今人忘其本分,而要格外讨好。汉请满人用满菜,满请汉人用汉菜,反致依样葫芦,有名无实,画虎不成反类犬矣。秀才下场,专作自己文字,务极其工,自有遇合。若逢一宗师而摹仿之,逢一主考而摹仿之,则掇皮元真,终身不中矣。

\hypertarget{header-n11}{%
\subsection{戒单}\label{header-n11}}

为政者兴一利,不如除一弊,能除饮食之弊则思过半矣。作《戒单》。

\textbf{戒外加油}
俗厨制菜,动熬猪油一锅,临上莱时,勺取而分浇之,以为肥腻。甚至燕窝至清之物,亦复受此玷污。而俗人不知,长吞大嚼,以为得油水入腹。故知前生是饿鬼投来。

\textbf{戒同锅熟} 同锅熟之弊,已载前``变换须知''一条中。

\textbf{戒耳餐}
何谓耳餐?耳餐者,务名之谓也。贪贵物之名,夸敬客之意,是以耳餐,非口餐也。不知豆腐得味,远胜燕窝;海菜不佳,不如蔬笋。余尝谓鸡、猪、鱼、鸭豪杰之士也,各有本味,自成一家;海参、燕窝庸陋之人也,全无性情,寄人篱下。尝见某太守宴客,大碗如缸,白煮燕窝四两,丝毫无味,人争夸之。余笑曰,``我辈来吃燕窝,非来贩燕窝也。''可贩不可吃,虽多奚为?若徒夸体面,不如碗中竟放明珠百粒,则价值万金矣。其如吃不得何?

\textbf{戒目食}
何谓目食?目食者,贪多之谓也。今人慕``食前方丈''之名,多盘叠碗,是以目食,非口食也。不知名手写字,多则必有败笔;名人作诗,烦则必有累句。极名厨之心力,一日之中,所作好菜不过四五味耳,尚难拿准,况拉杂横陈乎?就使帮助多人,亦各有意见,全无纪律,愈多愈坏。余尝过一商家,上菜三撤席,点心十六道,共算食品将至四十余种。主人自觉欣欣得意,而我散席还家,仍煮粥充饥。可想见其席之丰而不洁矣。南朝孔琳之曰:``今人好用多品,适口之外,皆为悦目之资。''余以为肴馔横陈,熏蒸腥秽,口亦无可悦也。

\textbf{戒穿凿}
物有本性,不可穿凿为之。自成小巧,即如燕窝佳矣,何必捶以为团?海参可矣,何必熬之为酱?西瓜被切,略迟不鲜,竟有制以为糕者。苹果太熟,上口不脆,竟有蒸之以为脯者。他如《尊生八笺》之秋藤饼,李笠翁之玉兰糕,都是矫揉造作,以杞柳为杯{[}木卷{]},全失大方。譬如庸德庸行,做到家便是圣人,何必索隐行怪乎?

\textbf{戒停顿}
物味取鲜,全在起锅时极锋而试,略为停顿,便如霉过衣裳,虽锦绣绮罗,亦晦闷而旧气可憎矣。尝见性急主人,每摆菜必一齐搬出。于是厨人将一席之莱,都放蒸笼中,候主人催取,通行齐上。此中尚得有佳味哉?在善烹任者,一盘一碗,费尽心思;在吃者,卤莽暴戾,囫囵吞下,真所谓得哀家梨,仍复蒸食者矣。余到粤东,食杨兰坡明府鳝羹而美,访其故,曰:``不过现杀现烹、现熟现吃,不停顿而已。''他物皆可类推。

哀家梨:传说汉朝秣陵人哀仲所之梨,实大而味美,入口消释,当时人称为``哀家梨''。这里是比喻愚人不辩滋味,得好梨仍蒸食之。

\textbf{戒暴珍}
暴者不恤人功,殄者不惜物力。鸡、鱼、鹅、鸭自首至尾,俱有味存,不必少取多弃也。尝见烹甲鱼者,专取其裙而不知味在肉中;蒸鲥鱼者,专取其肚而不知鲜在背上。至贱莫如腌蛋,其佳处虽在黄不在白,然全去其白而专取其黄,则食者亦觉索然矣。且予为此言,并非俗人惜福之谓,假使暴殄而有益于饮食,犹之可也;暴殄而反累于饮食,又何苦为之?至于烈炭以炙活鹅之掌,刺刀以取生鸡之肝,皆君子所不为也。何也、物为人用,使之死可也,使之求死不得不可也。

\textbf{戒纵酒}
事之是非,惟醒人能知之;味之美恶,亦惟醒人能知之。伊尹曰:``味之精微,口不能言也。''口且不能言,岂有呼呶酗酒之人,能知味者乎?往往见拇战之徒,啖佳菜如啖木屑,心不存焉。所谓惟酒是务,焉知其余,而治味之道扫地矣。万不得已,先于正席尝菜之味,后于撤席逞酒之能,庶乎其两可也。

\textbf{戒火锅}
冬日宴客,惯用火锅,对客喧腾,已属可厌;且各菜之味,有一定火候,宜文宜武,宜撤宜添,瞬息难差。今一例以火逼之,其味尚可问哉?近人用烧酒代炭,以为得计,而不知物经多滚总能变味。或问:菜冷奈何?曰:以起锅滚热之菜,不使客登时食尽,而尚能留之以至于冷,则其味之恶劣可知矣。

\textbf{戒强让}
治具宴客,礼也。然一肴既上,理直凭客举箸,精肥整碎,各有所好,听从客便,方是道理,何必强让之?常见主人以箸夹取,堆置客前,污盘没碗,令人生厌。须知客非无手无目之人,又非儿童、新妇,怕羞忍饿,何必以村妪小家子之见解待之?其慢客也至矣!近日倡家,尤多此种恶习,以箸取菜,硬入人口,有类强奸,殊为可恶。长安有甚好请客,而菜不佳者,一客问曰:``我与君算相好乎?''主人曰:``相好!''客跽而请曰:``果然相好,我有所求,必允许而后起。''主人惊问````何求?''曰:``此后君家宴客,求免见招。''合坐为之大笑。

\textbf{戒走油}
凡鱼、肉、鸡、鸭虽极肥之物,总要使其油在肉中,不落汤中,其味方存而不散。若肉中之油,半落汤中,则汤中之味反在肉外矣。推原其病有三:一误于火大猛,滚急水干。重番加水;一误于火势忽停,既断复续;一病在于太要相度,屡起锅盖,则油必走。

\textbf{戒落套}
唐诗最佳,而五言八韵之试帖,名家不选,何也?以其落套故也。诗尚如此,食亦宜然。今官场之菜,名号有十六碟、八簋、四点心之称,有满汉席之称,有八小吃之称,有十大菜之称,种种俗名皆恶厨陋习。只可用之于新亲上门,上司入境,以此敷衍;配上椅披桌裙,插屏香案,三揖百拜方称。若家居欢宴,文酒开筵,安可用此恶套哉?必须盘碗参差,整散杂进,方有名贵之气象。余家寿筵婚席,动至五六桌者,传唤外厨,亦不免落套,然训练之卒,范我驰驱者,其味亦终竟不同。

\textbf{戒混浊}
混浊者,并非浓厚之谓。同一汤也,望去非黑非白,如缸中搅浑之水。同一卤也,食之不清不腻,如染缸倒出之浆。此种色味令人难耐。救之之法,总在洗净本身,善加作料,伺察水火,体验酸咸,不使食者舌上有隔皮隔膜之嫌。庾子山论文云:``索索元真气,昏昏有俗心。''是即混浊之谓也。

\textbf{戒苟且}
凡事不宜苟且,而于饮食尤甚。厨者,皆小人下村,一日不加赏罚,则一日必生怠玩。火齐未到而姑且下咽,则明日之菜必更加生。真味已失而含忍不言,则下次之羹必加草率。且又不止空赏空罚而已也。其佳者,必指示其所以能佳之由;其劣者,必寻求其所以致劣之故。咸淡必适其中,不可丝毫加减,久暂必得其当,不可任意登盘。厨者偷安,吃者随便,皆饮食之大弊。审问慎思明辨,为学之方也;随时指点,教学相长,作师之道也。于是味何独不然?

\hypertarget{header-n13}{%
\subsection{海鲜单}\label{header-n13}}

古人珍并无海鲜之说,今世俗尚之,不得不吾从众。作《海鲜单》。

\textbf{燕窝}
燕窝贵物,原不轻用。如用之,每碗必须二两,先用天泉滚水泡之,将银针挑去黑丝。用嫩鸡汤、好火腿汤、新蘑菇三样汤滚之,看燕窝变成玉色为度。此物至清,不可以油腻杂之;此物至文,不可以武物串之。今人用肉丝、鸡丝杂之,是吃鸡丝、肉丝,非吃燕窝也。且徒务其名,往往以三钱生燕窝盖碗面,如白发数茎,使客一撩不见,空剩粗物满碗。真乞儿卖富,反露贫相。不得已则蘑菇丝,笋尖丝、鲫鱼肚、野鸡嫩片尚可用也。余到粤东,阳明府冬瓜燕窝甚佳,以柔配柔,以清入清,重用鸡汁、蘑菇汁而已,燕窝皆作玉色,不纯白也。或打作团,或敲成面,俱属穿凿。

\textbf{海参三法}
海参无味之物,沙多气腥,最难讨好。然天性浓重,断不可以清汤煨也。须检小刺参,先泡去沙泥,用肉汤滚泡三次,然后以鸡、肉两汁红煨极烂。辅佐则用香蕈、木耳,以其色黑相似也。大抵明日请客,则先一日要煨,海参才烂。尝见钱观察家,夏日用芥未、鸡汁拌冷海参丝甚佳。或切小碎丁,用笋丁、香蕈丁入鸡汤煨作羹。蒋侍郎家用豆腐皮、鸡腿、蘑菇煨海参亦佳。

\textbf{鱼翅二法}
鱼翅难烂,须煮两日,才能摧刚为柔。用有二法:一用好火腿、好鸡汤,如鲜笋、冰糖钱许煨烂,此一法也;一纯用鸡汤串细萝卜丝,拆碎鳞翅搀和其中,飘浮碗面。令食者不能辨其为萝卜丝、为鱼翅,此又一法也。用火腿者,汤宜少;用萝卜丝者,汤宜多。总以融洽柔腻为佳,若海参触鼻,鱼翅跳盘,便成笑话。吴道士家做鱼翅,不用下鳞,单用上半原根,亦有风味。萝卜丝须出水二次,其臭才去。尝在郭耕礼家吃鱼翅炒菜,妙绝!未传其方法。

\textbf{鳆鱼}
鳆鱼炒薄片甚佳,杨中丞家削片人鸡汤豆腐中。号称``鳆鱼豆腐'';上加陈糟油浇之。庄大守用大块鳆鱼煨整鸭,亦别有风趣。但其性坚,终不能齿决。火偎三日。才拆得碎。

\textbf{淡菜} 淡菜煨肉加汤,颇鲜,取肉去心,酒炒亦可。

\textbf{海蝘} 海蝘,宁波小鱼也,味同虾米,以之蒸蛋甚佳。作小菜亦可。

\textbf{乌鱼蛋}
乌鱼蛋最鲜,最难服事。须河水滚透,撤沙去臊,再加鸡汤、蘑菇爆烂。龚云若司马家制之最精。

\textbf{江瑶柱}
江瑶柱出宁波,治法与蚶、蛏同。其鲜脆在柱,故剖壳时多弃少取。

\textbf{蛎黄}
蛎黄生石子上。壳与石子胶粘不分。剥肉作羹,与蚶、蛤相似。一名鬼眼,乐清、奉化两县上产,别地所无。

\hypertarget{header-n15}{%
\subsection{江鲜单}\label{header-n15}}

郭璞《江赋》鱼族甚繁。今择其常有者治之。作《江鲜单》。

\textbf{刀鱼二法}
刀鱼用蜜酒酿、清酱放盘中,如鲥鱼法蒸之最佳。不必加水。如嫌刺多,则将极快刀刮取鱼片,用钳抽去其刺。用火腿汤、鸡汤、笋汤{[}火畏{]}之,鲜妙绝伦。金陵人畏其多刺,竟油炙极枯,然后煎之。谚曰:``驼背夹直,其人不活。''此之谓也。或用快刀将鱼背斜切之,使碎骨尽断,再下锅煎黄,加作料,临食时竟不知有骨:芜湖陶大太法也。

\textbf{鲥鱼}
鲥鱼用蜜酒蒸食,如治刀鱼之法便佳。或竟用油煎,加清酱、酒酿亦佳。万不可切成碎块加鸡汤煮,或去其背,专取肚皮,则真味全失矣。

\textbf{鲟鱼}
尹文端公,自夸治鲟鳇最佳,然煨之太熟,颇嫌重浊。惟在苏州唐氏,吃炒蝗鱼片甚佳。其法切片油炮,加酒、秋油滚三十次,下寸再滚起锅,加作料,重用瓜、姜、葱花。又一法,将鱼白水煮十滚,去大骨,肉切小方块,取明骨切小方块;鸡汤去沫,先煨明骨八分熟,下酒、秋油,再下鱼肉,煨二分烂起锅,加葱、椒、韭,重用姜汁一大杯。

\textbf{黄鱼}
黄鱼切小块,酱酒郁一个时辰。沥干。入锅爆炒两面黄,加金华豆鼓一茶杯,甜酒一碗,秋油一小杯,同滚。候卤干色红,加糖,加瓜、姜收起,有沉浸浓郁之妙。又一法,将黄鱼拆碎人鸡汤作羹,微用甜酱水、纤粉收起之,亦佳。大抵黄鱼亦系浓厚之物,不可以清治之也。

\textbf{班鱼}
班鱼最嫩,剥皮去秽,分肝肉二种,以鸡汤煨之,下酒三分、水二分、秋油一分;起锅时加姜汁一大碗,葱数茎,杀去腥气。

\textbf{假蟹}
煮黄鱼二条,取肉去骨,加生盐蛋四个,调碎,不拌入鱼肉;起油锅炮,下鸡汤滚,将盐蛋搅匀,加香蕈、葱、姜汁、酒,吃时酌用醋。

\hypertarget{header-n17}{%
\subsection{特牲单}\label{header-n17}}

猪用最多,可称``广大教主''。宜古人有特豚馈食之礼。作《特牲单》。

\textbf{猪头二法}
洗净五斤重者,用甜酒三斤;七八斤者,用甜酒五斤。先将猪头下锅同酒煮,下葱三十根、八角三钱,煮二百余滚;下秋油一大杯、糖一两,候熟后尝咸淡,再将秋油加减;添开水要漫过猪头一寸,上压重物,大火烧一炷香;退出大火,用文火细煨,收干以腻为度;烂后即开锅盖,迟则走油。一法打木桶一个,中用铜簾隔开,将猪头洗净,加作料闷入桶中,用文火隔汤蒸之,猪头熟烂,而其腻垢悉从桶外流出亦妙。

\textbf{猪蹄四法}
蹄膀一只,不用爪,白水煮烂,去汤,好酒一斤,清酱油杯半,陈皮一钱,红枣四五个,煨烂。起锅时,用葱、椒、酒泼入,去陈皮、红枣,此一法也。又一法:先用虾米煎汤代水,加酒、秋油煨之。又一法:用蹄膀一只,先煮熟,用素油灼皱其皮,再加作料红煨。有土人好先掇食其皮,号称``揭单被''。又一法:用蹄膀一个,两钵合之,加酒,加秋油,隔水蒸之,以二枝香为度,号``神仙肉''。钱观察家制最精。

\textbf{猪爪猪筋}
专取猪爪,剔去大骨,用鸡肉汤清煨之。筋味与爪相同,可以搭配;有好腿爪,亦可搀入。

\textbf{猪肚二法}
将肚洗精,取极厚处,去上下皮,单用中心,切骰子块,滚油炮炒,加作料起锅,以极脆为佳。此北人法也。南人白水加酒,煨两枝香,以极烂为度,蘸清盐食之,亦可;或加鸡汤作料,煨烂熏切,亦佳。

\textbf{猪肺二法}
洗肺最难,以冽尽肺管血水,剔去包衣为第一着。敲之仆之,挂之倒之,抽管割膜,工夫最细。用酒水滚一日一夜。肺缩小如一片白芙蓉,浮于水面,再加上作料。上口如泥。汤西厓少宰宴客,每碗四片,已用四肺矣。近人无此工夫,只得将肺拆碎,入鸡汤煨烂亦佳。得野鸡汤更妙,以清配清故也。用好火腿煨亦可。

\textbf{猪腰}
腰片炒枯则木,炒嫩则令人生疑;不如煨烂,蘸椒盐食之为佳。或加作料亦可。只宜手摘,不宜刀切。但须一日工夫,才得如泥耳。此物只宜独用,断不可搀入别菜中,最能夺味而惹腥。煨三刻则老,煨一日则嫩。

\textbf{猪里肉}
猪里肉精而且嫩。人多不食。尝在扬州谢蕴山太守席上,食而甘之。云以里肉切片,用纤粉团成小把,入虾汤中,加香簟、紫菜清煨,一熟便起。

\textbf{白肉片}
须自养之猪,宰后入锅,煮到八分熟,泡在汤中,一个时辰取起。将猪身上行动之处,薄片上桌。不冷不热,以温为度。此是北人擅长之菜。南人效之,终不能佳。且零星市脯,亦难用也。寒士请客,宁用燕窝,不用白片肉,以非多不可故也。割法须用小快刀片之,以肥瘦相参,横斜碎杂为佳,与圣人``割不正不食''一语截然相反。其猪身,肉之名目甚多。满洲``跳神肉''最妙。

\textbf{红煨肉三法}
或用甜酱,或用秋油,或竟不用秋油、甜酱。每肉一斤,用盐三钱,纯酒煨之;亦有用水者,但须熬干水气。三种治法皆红如琥珀,不可加糖炒色。早起锅则黄,当可则红,过迟则红色变紫,而精肉转硬。常起锅盖,则油走而味都在油中矣。大抵割肉虽方,以烂到不见锋棱,上口而精肉俱化为妙。全以火候为主。谚云:``紧火粥,慢火肉。''至哉言乎!

\textbf{白煨肉}
每肉一斤,用白水煮八分好,起出去汤;用酒半斤,盐二钱半,煨一个时辰。用原汤一半加入,滚干汤腻为度,再加葱、椒、木耳、韭菜之类。火先武后文。又一法:每肉一斤,用糖一钱,酒半斤,水一斤,清酱半茶杯;先放酒滚肉一、二十次,加茴香一钱,加水闷烂,亦佳。

\textbf{油灼肉}
用硬短勒切方块,去筋襻,酒酱郁过,入滚油中炮炙之,使肥者不腻,精者肉松。将起锅时,加葱、蒜,微加醋喷之。

\textbf{干锅蒸肉}
用小磁钵,将肉切方块,加甜酒、秋油,装大钵内封口,放锅内,下用文火干蒸之。以两枝香为度,不用水。秋油与酒之多寡,相肉而行,以盖满肉面为度。

\textbf{盖碗装肉} 放手炉上,法与前同。

\textbf{磁坛装肉} 放砻糠中慢煨。法与前同。总须封口。

\textbf{脱沙肉}
去皮切碎,每一斤用鸡子三个,青黄俱用,调和拌肉;再斩碎,入秋油半酒杯,葱末拌匀,用网油一张裹之;外再用菜油四两,煎两面,起出去油;用好酒一茶杯,清酱半酒杯,闷透,提起切片;肉之面上,加韭菜、香蕈、笋丁。

\textbf{晒干肉} 切薄片精肉,晒烈日中,以干为度。用陈大头菜,夹片干炒。

\textbf{火腿煨肉}
火腿切方块,冷水滚三次,去汤沥干;将肉切方块,冷水滚二次,去汤沥干;放清水煨,加酒四两,葱、椒、损、香蕈。

\textbf{台鳖煨肉}
法与火腿煨肉同。鳖易烂,须先煨肉至八分,再加鳖;凉之则号``鳖冻''。绍兴人菜也。鳖不佳者,不必用。

\textbf{粉蒸肉}
用精肥参半之肉,炒米粉黄色,拌面酱蒸之,下用白菜作垫,熟时不但肉美,菜亦美。以不见水,故味独全。江西人菜也。

\textbf{熏煨肉}
先用秋油、酒将肉煨好,带汁上不屑,略熏之,不可太久,使干湿参半,香嫩异常。吴小谷广文家制之精极。

\textbf{芙蓉肉}
精肉一斤,切片,清酱拖过,风干一个时辰。用大虾肉四十个,猪油二两,切骰子大,将虾肉放在猪肉上,一只虾,一块肉,敲扁,将滚水煮熟撩起。熬菜油半斤,将肉片放在眼铜勺内,将滚油灌熟。再用秋油半酒杯,酒一杯,鸡汤一茶杯,熬滚,浇肉片上,加蒸粉、葱、椒,糁上起锅。

\textbf{荔枝肉}
用肉切大骨牌片,放白水煮二、三十滚,撩起;熬菜油半斤,将肉放入炮透,撩起,用冷水一激,肉皱,撩起;放入锅内,用酒半斤,清酱一小杯,水半斤,煮烂。

\textbf{八宝肉}
用肉一斤,精肥各半,白煮、二十滚,切柳叶片。小淡菜二两,鹰爪二两,香蕈一两,花海蜇二两,胡桃肉四个去皮,笋片四两,好火腿二两,麻油一两。将肉入锅,秋油、洒煨至五分熟,再加余物,海蜇下在最后。

\textbf{菜花头煨肉} 用台心菜嫩蕊微腌,晒干用之。

\textbf{炒肉丝}
切细丝,去筋襻、皮、骨,用清酱、酒郁片时,用菜油熬起白烟变青烟后,下肉炒匀,不停手,加蒸粉,醋一滴,糖一撮,葱的、韭蒜之类;只炒半斤,大火葬,不用水。又一法:用油泡后,用酱水,加酒略煨,起锅红色,加韭菜尤香。

\textbf{炒肉片}
将肉精肥各半切成薄片,清酱拌之。入锅油炒,闻响即加酱、水、葱、瓜、冬笋、韭芽,起锅火要猛烈。

\textbf{八宝肉圆}
猪肉精、肥各半,斩成细酱,用松仁、得香蕈、笋尖、荸荠、瓜姜之类斩成细酱,加纤粉和捏成团,放入盘中,加甜洒、秋油、蒸之。入口松脆。家致华云:``肉圆家切不宜斩。''必别有所见。

\textbf{空心肉圆}
将肉捶碎郁过,用冻猪油一小团作馅子,放在团内蒸之,则油流去而团子空矣。此法镇江人最善。

\textbf{锅烧肉} 煮熟不去皮,放麻油灼过,切块加盐,或蘸清酱亦可。

\textbf{酱肉} 先微腌,用面酱酱之,或单用秋油拌郁,风干。

\textbf{糟肉} 先微腌,再加米糟。

\textbf{暴腌肉} 微盐擦揉,三日内即用。以上三味,皆冬月菜也。春夏不宜。

\textbf{尹文端公家风肉}
杀猪一口,斩成八块,每块炒盐四钱,细细揉擦,使之无微不到。然后高挂有风无日处。偶有虫蚀,以香油涂之。夏日取用,先放水中泡一宵,再煮,水亦不可太少,以盖肉面为度。削片时,用快刀横切,不可顺肉丝而斩也。此物惟尹府至精,常以进贡。今徐州风肉不及,亦不知何故。

\textbf{家乡肉}
杭州家乡肉,好丑不同。有上、中、下三等。大概淡而能鲜,精肉可横咬者为上品。放久即是好火腿。

\textbf{笋煨火肉}
冬笋切方块,火肉切方块,同煨。火腿撤去盐水两遍,再入冰糖煨烂。席武山别驾云:凡火肉煮好后,若留作次日吃者,须留原汤,待次日将火肉投入汤中滚热才好。若干放离汤,则风燥而肉枯;用白水则又味淡。

\textbf{烧小猪}
小猪一个,六七斤重者,钳毛去秽,叉上炭火炙之。要四面齐到时,以深黄色为度。皮上慢慢以奶酥油涂之,屡涂屡炙。食时酥为上,脆次之,硬斯下矣。旗人有单用酒、秋油蒸者,亦惟吾家龙文弟,颇得其法。

\textbf{烧猪肉}
凡烧猪肉,须耐性。先炙里面肉,使油膏走入皮内,则皮松脆而味不走。若先炙皮,则肉中之油尽落火上,皮既焦硬,味亦不佳。烧小猪亦然。

\textbf{排骨}
取勒条排骨精肥各半者,抽去当中直骨,以葱代之,炙用醋、酱频频刷上,不可太枯。

\textbf{罗簑肉}
以作鸡松法作之。存盖面之皮。将皮下精肉斩成碎团,加作料烹熟。聂厨能之。

\textbf{端州三种肉}
一罗簑肉。一锅烧白肉,不加作料,以芝麻、盐拌之;切片煨好,以清酱拌之。三种俱宜于家常。端州聂、李二厨所作。特令杨二学之。

\textbf{杨公圆}
杨明作肉圆,大如茶杯,细腻绝伦。汤尤鲜洁,入口如酥。大概去筋去节,斩之极细,肥瘦各半,用纤合匀。

\textbf{黄芽菜煨火腿}
用好火腿削下外皮,去油存肉。先用鸡汤将皮煨酥,再将肉煨酥,放黄芽菜心,连根切段,约二寸许长;加蜜、酒酿及水,连煨半日。上口甘鲜,肉菜俱化,而菜根及菜心丝毫不散。汤亦美极。朝天宫道士法也。

\textbf{蜜火腿}
取好火腿,连皮切大方块,用蜜酒煨极烂,最佳。但火腿好丑、高低,判若天渊。虽出金华、兰溪、义乌三处,而有名无实者多。其不佳者,反不如腌肉矣。惟杭州忠清里王三房,四钱一斤者佳。余在尹文端公苏州公馆吃过一次,其香隔户便至,甘鲜异常。此后不能再遇此尤物矣。

\hypertarget{header-n19}{%
\subsection{杂牲单}\label{header-n19}}

牛、羊、鹿三牲,非南人家常时有之之物。然制法不可不知。作《杂牲单》。

\textbf{牛肉}
买牛肉法,先下各[食甫]定钱,凑取腿筋夹肉处,不精不肥。然后带回家中,剔去皮膜,用三分酒、二分水清煨,极烂;再加秋油收汤。此太牢独法治孤行者也,不可加加别物配搭。

\textbf{牛舌}
牛舌最佳。去皮、撕膜、切片,入肉中同煨。亦有冬腌风干者,隔年食之,极似好火腿。

\textbf{羊头}
羊头毛是去净,如去不净,用火烧之。洗净切开,煮烂去骨。其口内老皮俱要去净。将眼睛切成二块,去黑皮,眼珠不用,切成碎丁。取老肥母鸡汤煮之,加香蕈、笋丁,甜酒四两,秋油一杯。如吃辣,用小胡椒十二颗、葱花十二段;如吃酸,用好米醋一杯。

\textbf{羊蹄}
煨羊蹄照煨猪蹄法,分红、白二色。大抵用清酱煮红,用盐者白。山药丁同煨。

\textbf{羊羹}
取熟羊肉斩小块,如骰子大。鸡汤煨,加笋丁、香蕈丁、山药丁同煨。

\textbf{羊肚羹}
将羊肚洗净,煮烂切丝,用本汤煨之。加胡椒、醋俱可。北人炒法,南人不能如其脆。钱[王与]沙方伯家锅烧羊肉极佳,将求其法。

\textbf{红煨羊肉} 与红煨猪肉同。加刺眼、核桃,放入去膻。亦古法也。

\textbf{炒羊肉丝} 与炒猪肉丝同。可以用纤,愈细愈佳。葱丝拌之。

\textbf{烧羊肉}
羊肉切大块,重五七斤者,铁叉火上烧之。味果甘脆,宜惹宋仁宗认夜半之思也。

\textbf{全羊}
全羊法有七十二种,可吃者不过十八九种而已。此屠龙之技,家厨难学。一盘一碗全是羊肉,而味各不同才好。

\textbf{鹿肉}
鹿肉不可轻得。得而制之,其嫩鲜的獐肉之上。烧食可,煨食亦可。

\textbf{鹿筋二法}
鹿筋难烂。须三日前先捶煮之,绞出臊水数遍,加肉汁汤煨之,再用鸡汁汤煨;加秋油、酒,微纤收汤;不搀他物,便成白色,用盘盛之。如兼用火腿、冬笋、香蕈同煨,便成红色,不收汤,以碗盛之。白色者加花椒细末。

\textbf{獐肉} 制獐肉与制牛鹿同。可以作脯。不如鹿肉之活,而细腻过之。

\textbf{果子狸}
果子狸,鲜者难得。其腌干者,用蜜酒酿,蒸熟,快刀切片上桌。先用米泔水泡一日,去尽盐秽。较火腿沉嫩而肥。

\textbf{假牛乳}
用鸡蛋清拌蜜酒酿,打掇入化,上锅蒸之。以嫩腻为月。火候迟便老,蛋清太多亦老。

\textbf{鹿尾}
尹文端公品味,以鹿尾为第一。然南方人不能常得从北京来者,又苦不鲜新。余尝得极大者,用菜叶包而蒸之,味果不同。其最佳处的尾上一道浆耳。

\hypertarget{header-n21}{%
\subsection{羽族单}\label{header-n21}}

鸡功最巨,诸菜赖之。如善人积阴德而人不知。故令领羽族之首,而以他禽附之。作《羽族单》

\textbf{白片鸡}
肥鸡白片,自是太羹、玄酒之味。尤宜于下乡村、入旅店,烹饪不及之时,最为省便。煮时不可多。\\
太羹:古代祭祀时所用的肉汁\\
玄酒:指水。上古无酒,祭祀用水,以水代酒。水本无色,古人习以为黑色,故称玄酒。后引申为薄酒。

\textbf{鸡松}
肥鸡一只,用两腿,去筋骨剁碎,不可伤皮。用鸡蛋清、粉纤、松子肉,同剁成块。如腿不敷用,添脯子肉,切成方块,用香油灼黄,起放钵头内,加百花酒半斤、秋油一大杯、鸡油一铁勺,加冬笋、香覃、姜葱等。将所余鸡骨皮盖面,加水一大碗,下蒸笼蒸透,临吃去之。

\textbf{生炮鸡}
小雏鸡斩小方块,秋油、酒拌,临吃时拿起,放滚油内灼之,起锅又灼,连灼三回,盛起,用醋、酒、粉纤、葱花喷之。

\textbf{鸡粥}
肥母鸡一只,用刀将两脯肉去皮细刮,或用刨刀亦可;只可刮刨,不可斩,斩之便不腻矣。再用余鸡熬汤下之。吃时加细米粉、火腿屑、松子肉,共敲碎放汤内。起锅时放葱姜,浇鸡油,或去渣,或存渣滓,俱可。宜于老人。大概斩碎者去渣,刮刨者不去渣。

\textbf{焦鸡}
肥母鸡洗净,整下锅煮。用猪油四两、茴香四个,煮成八分熟,再拿香油灼黄,还下原汤熬浓,用秋油、酒、整葱收起。临上片碎,并将原卤浇之,或拌蘸亦可。此杨中丞家法也。方辅兄家亦好。

\textbf{捶鸡} 将整鸡捶碎,秋油、酒煮之。南京高南昌太守家制之最精。

\textbf{炒鸡片}
用鸡脯肉去皮,斩成薄片。用豆粉、麻油、秋油拌之,纤粉调之,鸡蛋清拌。临下锅加酱、瓜、姜、葱花末。须用极旺之火炒。一盘不过四两,火气才透。

\textbf{蒸小鸡}
用小嫩鸡雏,整放盘中,上加秋油、甜酒、香蕈、笋尖,饭锅上蒸之。

\textbf{酱鸡} 生鸡一只,用清酱浸一昼夜而风干之。此三冬菜也。

\textbf{鸡丁}
取鸡脯子切骰子小块,入滚油炮炒之,用秋油、酒收起;加荸荠丁、笋丁、香蕈丁拌之,汤以黑色为佳。

\textbf{鸡圆}
斩鸡脯子肉为圆,如酒杯大,鲜嫩如虾团。扬州臧八太爷制之最精。法用猪油、萝卜、纤粉揉成,不可放馅。

\textbf{蘑菇煨鸡}
口蘑菇四两,开水泡去砂,用冷水漂,牙刷擦,再用清水漂四次,用菜油二两炮透,加酒喷。将鸡斩块放锅内,滚去沫,下甜酒、清酱,煨八分功程,下蘑菇,再煨二分功程,加笋、葱、椒起锅,不用水,加冰糖三钱。

\textbf{梨炒鸡}
取雏鸡胸肉切片,先用猪油三两熬熟,炒三四次,加麻油一瓢,纤粉、盐花、姜汁、花椒末各一茶匙,再加雪梨薄片,香蕈小块,炒三四次起锅,盛五寸盘。

\textbf{假野鸡卷}
将脯子斩碎,用鸡子一个,调清酱郁之,将网油画碎,分包小包,油里炮透,再加清酱、酒作料,香蕈、木耳起锅,加糖一撮。

\textbf{黄芽菜炒鸡}
将鸡切块,起油锅生炒透,酒滚二三十次,加秋油后滚二三十次,下水滚,将菜切块,俟鸡有七分熟,将菜下锅;再滚三分,加糖、葱、大料。其菜要另滚熟搀用。每一只用油四两。

\textbf{栗子炒鸡}
鸡斩块,用菜油二两炮,加酒一饭碗,秋油一小杯,水一饭碗,煨七分熟;先将栗子煮熟,同笋下之,再煨三分起锅,下糖一撮。

\textbf{灼八块}
嫩鸡一只,斩八块,滚油炮透,去油,加清酱一杯、酒半斤,煨熟便起,不用水,用武火。

\textbf{珍珠团}
熟鸡脯子,切黄豆大块,清酱、酒拌匀,用干面滚满,入锅炒。炒用素油。

\textbf{黄芪蒸鸡治疗}
取童鸡未曾生蛋者杀之,不见水,取出肚脏,塞黄芪一两,架箸放锅内蒸之,四面封口,熟时取出。卤浓而鲜,可疗弱症。

\textbf{卤鸡}
囫囵鸡一只,肚内塞葱三十条,茴香二钱,用酒一斤,秋油一小杯半,先滚一枝香,加水一斤,脂油二两,一齐同煨;待鸡熟,取出脂油。水要用熟水,收浓卤一饭碗,才取起;或拆碎,或薄刀片之,仍以原卤拌食。

\textbf{蒋鸡}
童子鸡一只,用盐四钱、酱油一匙、老酒半茶杯、姜三大片,放砂锅内,隔水蒸烂,去骨,不用水,蒋御史家法也。

\textbf{唐鸡}
鸡一只,或二斤,或三斤,如用二斤者,用酒一饭碗,水三饭碗;用三斤者,酌添。先将鸡切块,用菜油二两,候滚滚以熟,爆鸡要透。先用酒滚一、二十滚,再下水约二、三百滚,用秋油一酒杯,起锅时加白糖一钱,唐静涵家法也。

\textbf{鸡肝} 用酒、醋喷炒,以嫩为贵。

\textbf{鸡血} 取鸡血为条,加鸡汤、酱醋、索粉作羹,宜于老人。

\textbf{鸡丝}
拆鸡为丝,秋油、芥末、醋拌之。此杭菜也。加笋芹俱可。用笋丝、秋油、酒炒之亦可。拌者用熟鸡,炒者用生鸡。

\textbf{糟鸡} 糟鸡与糟肉同。

\textbf{鸡肾} 取鸡肾三十个,煮微熟,去皮,用鸡汤加作炒煨之。鲜嫩绝伦。

\textbf{鸡蛋}
鸡蛋去壳放碗中,将竹箸打一千加蒸之,绝嫩。凡蛋一煮而老,一千煮而反嫩。加茶叶煮者,以两炷香为度。蛋一百,用盐一两;五十,用盐五钱。加酱煨亦可。其他则或煎或炒俱可。斩碎黄雀蒸之,亦佳。

\textbf{野鸡五法}
野鸡披胸肉,清酱郁过,以网油包放铁奁上烧之。作方片可,作卷子亦可。此一法也。切片加作料炒,一法也。取胸肉作丁,一法也。当家鸡整煨,一法也。先用油灼,拆丝加酒、秋油、醋,同芹菜冷拌,一法也。生片其肉,入火锅中,登时便吃,亦一法也。其弊的肉嫩则味不入,味入则肉又老。

\textbf{赤炖肉鸡}
赤炖肉鸡,洗切净,每一斤用好酒十二两、盐二钱五分、冰糖四钱,研酌加桂皮,同入砂锅中,文炭火煨之。倘酒将干,鸡肉沿未烂,每斤酌加清开水一茶杯。

\textbf{蘑菇煨鸡}
鸡肉一斤,甜酒一斤,盐三钱,冰糖四钱,蘑菇用新鲜不霉者,文火煨两枝线香为度。不可用水,先煨鸡八分熟,再下蘑菇。

\textbf{鸽子} 鸽子加好火腿同煨,甚佳。不用火腿亦可。

\textbf{鸽蛋} 煨鸽蛋法与煨鸡肾同。或煎食亦可,加微醋亦可。

\textbf{野鸭}
野鸭切厚片,秋油郁过,用两片雪梨夹住炮炒之。苏州包道台家制法最精,今失传矣。用蒸家鸭法蒸之亦可。

\textbf{蒸鸭}
生肥鸭去骨,内用糯米一酒杯,火腿丁、大头菜丁、香蕈、笋丁、秋油、酒、小蘑麻油、葱花,俱灌鸭肚内,外用鸡汤放盘中,隔水蒸透,此真定魏太守家法也。

\textbf{鸭糊涂}
用肥鸭白煮八分熟,冷定去骨,拆成天然不方不圆之块,下原汤内煨,加盐三钱、酒半斤、捶碎山药同下锅作纤,临煨烂时,再加姜末、香蕈、葱花。如要浓汤,加放粉纤。以芋代山药亦妙。

\textbf{卤鸭} 不用水用酒,煮鸭去骨,加作料食之,高要令杨公家法也。

\textbf{鸭脯}
用肥鸭斩大方块,用酒半斤、秋油一杯、笋、香蕈、葱花闷之,收卤起锅。

\textbf{烧鸭} 用雏鸭上叉烧之。冯观察家厨最精。

\textbf{挂卤鸭}
塞葱鸭腹,盖闷而烧。水西门许店最精。家中不能作。有黄黑二色,黄者更妙。

\textbf{干蒸鸭}
杭州商人何星举家干蒸鸭。将肥鸭一只,洗净斩八块,加甜酒、秋油,淹满鸭面,放磁罐中封好,置干锅中蒸之;用文炭火,不用水,临上时,其精肉皆烂如泥。以线香二枝为度。

\textbf{野鸭团}
细斩野鸭胸前肉,加猪油微纤,调揉成团,入鸡汤滚之。或用本鸭汤亦佳。太兴孔亲家制之甚精。

\textbf{徐鸭}
顶大鲜鸭一只,用百花酒十二两,青盐一两二钱、滚水一汤碗,冲化去渣沫,再兑冷水七饭碗,鲜姜四厚片,约重一两,同入大瓦盖钵内,将皮纸封固口,用大火笼烧透大炭吉三元(约二文一个);外用套包一个,将火笼罩定,不可令其走气。约早点时炖起,至晚方好。速则恐其不透,味便不佳矣。其炭吉烧透后,不宜更换瓦钵,亦不预先开看。鸭破开时,将清水洗后,用洁净无浆布拭干入钵。

\textbf{煨麻雀}
取麻雀五十只,以清酱、甜酒煨之,熟后去爪脚,单取雀胸、头肉,连放盘中,甘鲜异常。其他鸟鹊俱可类推。但鲜者一时难得。薛生白常劝人勿食人间豢养之物,以野禽味鲜,且易消化。

\textbf{煨鹩鹑、黄雀}
鹩鹑用六合来者最佳。有现成制好者。黄雀用苏州糟,加蜜酒煨烂,下作料,与煨麻雀同。苏州沈观察煨黄雀并骨如泥,不知作何制法。炒鱼片亦精。其厨馔之精,合吴门推为第一。

\textbf{云林鹅}
《倪云林集》中载制鹅法。整套鹅一只,洗净后用盐三钱擦其腹内,塞葱一帚填实其中,外将蜜拌酒通身满涂之,锅中一大碗酒、一大碗水蒸之,用竹箸架之,不使鹅身近水。灶内用山茅二束,缓缓烧尽为度。俟锅盖冷后揭开锅盖,将鹅翻身,仍将锅盖封好蒸之,再用茅柴一束烧尽为度。柴俟其自尽,不可挑拨。锅盖用绵纸糊封,逼燥裂缝,以水润之。起锅时,不但鹅烂如泥,汤亦鲜美。以此法制鸭,味美亦同。每茅柴一束,重一斤八两。擦盐时,串入葱、椒末子,以酒和匀。《云林集》中,载食品甚多;只此一法,试之颇效,余俱附会。

\hypertarget{header-n23}{%
\subsection{水族有鳞单}\label{header-n23}}

鱼皆去鳞,惟鲥鱼不去。我道有鳞而鱼形始全。作《水族有鳞单》

\textbf{边鱼}
边鱼活者,加酒、秋油蒸之。玉色为度。一作呆白天色,则肉老而味变矣。并须盖好,不可受锅盖上之水气。临起加香蕈、笋尖。或用酒煎亦佳;用酒不用水,号``假鲥鱼''。

\textbf{鲫鱼}
鲫鱼先要善买。择其扁身而带白色者,其肉嫩而松;熟后一提,肉即卸骨而下。黑脊浑身者,崛强槎枒,鱼中之喇子也,断不可食。照边鱼蒸法,最佳。其次煎吃亦妙。拆肉下可以作羹。通州人能煨之,骨尾俱酥,号``麻鱼'',利小儿食。然总不如蒸食之得真味也。六合龙池出者,愈大愈嫩,亦奇。蒸时用酒不用水,稍稍用糖以起其鲜。以鱼之小大,酌情量秋油、酒之多寡。

\textbf{白鱼}
白鱼肉最细。用糟鲥鱼同蒸之,最佳。或冬日微腌,加酒酿糟二日,亦佳。余在江中得网起活者,用酒蒸食,美不可言。糟之最佳,不可太久,久则肉木矣。

\textbf{季鱼}
季鱼少骨,炒片最佳。炒者以片薄为贵。用秋油细郁后,用纤粉、蛋清搂之,入油锅炒,加作料炒之。油用素油。

\textbf{土步鱼}
杭州以土步鱼为上品。而金陵人贱之,目为虎头蛇,可发一笑。肉最松嫩。煎之,煮之,蒸之俱可。加腌芥作汤,作羹,尤鲜。

\textbf{鱼松}
用青鱼、[鱼军]鱼蒸熟,将肉拆下,放油锅中灼之,黄色,加盐花、葱、椒、瓜、姜。冬日封瓶中,可以一月。

\textbf{鱼圆}
用白鱼、青鱼活者,剖半钉板上,用刀刮下肉,留剌的板上;将肉斩化,用豆粉、猪油拌,将手搅之;放微微盐水,不用清酱,加葱、姜汁作团,成后,放滚水中煮熟撩起,冷水养之,临吃入鸡汤、紫菜滚。

\textbf{鱼片}
取青鱼、季鱼片,秋油郁之,加纤纷、蛋清,起油锅炮炒,用小盘盛起,加葱、椒、瓜、姜,极多不过六两,太多则火气不透。

\textbf{连鱼豆腐}
用大连鱼煎熟,加豆腐,喷酱、水、葱、酒滚之,俟汤也半红起锅,其头味尤美。此杭州菜也。用酱多少,须相鱼而行。

\textbf{醋搂鱼}
用活青鱼切大块,油灼之,加酱、醋、酒喷之,汤多为妙。俟熟即速起锅。此物杭州西湖上五柳居有名。而今则酱臭而鱼败矣。甚矣!宋嫂鱼羹,徒存虚名。《梦梁录》不足信也。鱼不可大,大则味不入;不可小,小则剌多。

\textbf{银鱼}
银鱼起水时,名冰鲜。加鸡汤、火腿汤煨之。或炒食甚嫩。干者泡软,用酱水炒亦妙。

\textbf{台鲞}
台鲞好丑不一。出台州松门者为佳,肉软而鲜肥。生时拆之,便可当作小菜,不必煮食也;用鲜肉同煨,须肉烂时放鲞,否则鲞消化不见矣,冻之即为鲞冻,绍兴人法也。

\textbf{糟鲞}
冬日用大鲤鱼腌而干之,入酒糟,置坛中,封口。夏日食之。不可烧酒作泡。用烧酒者,不无辣味。

\textbf{虾子勒鲞}
夏日选白净带子勒鲞,放水中一日,泡去盐味,太阳晒干,入锅油煎一面黄取起,以一面未黄者铺上虾子,放盘中,加白糖蒸之,以一炷香为度。三伏日食之绝妙。

\textbf{鱼脯}
活青鱼去头尾,斩小方块,盐腌透,风干,入锅油煎;加作料收卤,再炒芝麻滚拌起锅,苏州法也。

\textbf{家常煎鱼}
家常煎鱼,须要耐性。将[鱼军]鱼洗净,切块盐腌,压扁,入油中两面熯黄,多加酒、秋油,文火慢慢滚之,然后收汤作卤,使作料之味全入鱼中。第此法指鱼之不活者而言。如活者,又以速起锅为妙。

\textbf{黄姑鱼}
岳州出小鱼,长二三寸,晒干寄来。加酒剥皮,放饭锅上蒸而食之,味最鲜,号``黄姑鱼''。

\hypertarget{header-n25}{%
\subsection{水族无鳞单}\label{header-n25}}

鱼无鳞者,其腥加倍,须加意烹饪;以姜、桂胜之。作《水族无鳞单》

\textbf{汤鳗}
鳗鱼最忌出骨。因此物性本腥重,不可过于摆布,失其天真,犹鲥鱼之不可去鳞也。清煨者,以河鳗一条,洗去滑涎,斩寸为段,入磁罐中,用酒水煨烂,下秋油起锅,加冬腌新芥菜作汤,重用葱、姜之类,以杀其腥。常熟顾比部家,用纤粉、山药干煨,亦妙。或加作料直置盘中蒸之,不用水。家致华分司蒸鳗最佳。秋油、酒四六兑,务使汤浮于本身。起笼时,尤要恰好,迟则皮皱味失。

\textbf{红煨鳗}
鳗鱼用酒、水煨烂,加甜酱代秋油,入锅收汤煨干,加茴香大料起锅。有三病宜戒者:一皮有皱纹,皮便不酥;一肉散碗中,箸夹不起;一早下盐豉,入口不化。扬州朱分司家制之最精。大抵红煨者以干为贵,使卤味收入鳗肉中。

\textbf{炸鳗}
择鳗鱼大者,去首尾,寸断之。先用麻油炸熟,取起;另将鲜蒿菜嫩尖入锅中,仍用原油炒透,即以鳗鱼平铺菜上,加作料煨一炷香。蒿菜分量,较鱼减半。

\textbf{生炒甲鱼}
将甲鱼去骨,用麻油炮炒之,加秋油一杯、鸡汁一杯。此真定魏太守家法也。

\textbf{酱炒甲鱼}
将甲鱼煮半熟,去骨,起油锅炮炒,加酱水、葱、椒,收汤成卤,然后起锅。此杭州法也。

\textbf{带骨甲鱼}
要一个半斤重者,斩四块,加脂油三两,起油锅煎两面黄,加水、秋油、酒煨;先武火,后文火,至八分熟加蒜,起锅用葱、姜、糖。甲鱼宜小不宜大。俗号``童子脚鱼''才嫩。

\textbf{青盐甲鱼}
斩四块,起油锅炮透。每甲鱼一斤,用酒四两、大茴香三钱、盐一钱半,煨至半好,下脂油二两;切小豆块再煨,加蒜头、笋尖,起时用葱、椒,或用秋油,则不用盐。此苏州唐静涵家法。甲鱼大则老,小则腥,须买其中样者。

\textbf{汤煨甲鱼}
将甲鱼白煮,去骨拆碎,用鸡汤、秋油、酒煨汤二碗,收至一碗,起锅,用葱、椒、姜末糁之。吴竹屿制之最佳。微用纤,才得汤腻。

\textbf{全壳甲鱼}
山东杨参将家,制甲鱼去首尾,取肉及裙,加作料煨好,仍以原壳覆之。每宴客,一客之前以小盘献一甲鱼。见者悚然,犹虑其动。惜未传其法。

\textbf{鳝丝羹}
鳝鱼煮半熟,划丝去骨,加酒、秋油煨之,微用纤粉,用真金菜、冬瓜、长葱为羹。南京厨者辄制鳝为炭,殊不可解。

\textbf{炒鳝} 拆鳝丝炒之,略焦,如炒肉鸡之法,不可用水。

\textbf{段鳝}
切鳝以寸为段,照煨鳗法煨之,或先用油炙,使坚,再以冬瓜、鲜笋、香蕈作配,微用酱水,重用姜汁。

\textbf{虾圆}
虾圆照鱼圆法。鸡汤煨之,干炒亦可。大概捶虾时不宜过细,恐失真味。鱼圆亦然。或竟剥夺虾肉以紫菜拌之,亦佳。

\textbf{虾饼} 以虾捶烂,团而煎之,即为虾饼。

\textbf{醉虾}
带壳用酒炙黄,捞起,加清酱、米醋煨之,用碗闷之。临食放盘中,其壳俱酥。

\textbf{炒虾}
炒虾照炒鱼法,可用韭配。或加冬腌芥菜,则不可用韭矣。有捶扁其尾单炒者,亦觉新异。

\textbf{蟹}
蟹宜独食,不宜搭配他物。最好以淡盐汤煮熟,自剥自食为妙。蒸者味虽全,而失之太淡。

\textbf{蟹羹}
剥蟹为羹,即用原汤煨之,不加鸡汁,独用为妙。见俗厨从中加鸭舌,或鱼翅,或海参者,徒夺其味而惹其腥恶,劣极矣!

\textbf{炒蟹粉} 以现剥现炒之蟹为佳。过两个时辰,则肉干而味失。

\textbf{剥壳蒸蟹}
将蟹剥壳,取肉、取黄,仍置壳中,放五六只在生鸡蛋上蒸之。上桌时完然一蟹,惟去爪脚。比炒蟹粉觉有新色。杨兰坡明府,以南瓜肉拌蟹,颇奇。

\textbf{蛤蜊} 剥蛤蜊肉,加韭菜炒之佳。或为汤亦可。起迟便枯。

\textbf{蚶}
蚶有三吃法。用热水喷之,半熟去盖,加酒、秋油醉之;或用鸡汤滚熟,去盖入汤;或全去其盖,作羹亦可。但宜速起,迟则肉枯。蚶出奉化县,品在蛼螯、蛤蜊之上。

\textbf{蛼螯}
先将五花肉切片,用作料闷烂。将蛼螯洗净,麻油炒仍将肉片连卤烹之。秋油要重些,方得有味。加豆腐亦可。蛼螯从扬州来,虑坏则取壳中肉,置猪油中,可以远行。有晒为干者,亦佳。入鸡汤烹之,味在蛏干之上。捶烂蛼螯作饼,如虾饼样,煎吃加作料亦佳。

\textbf{程泽弓蛏干}
程泽弓商人家制蛏干,用冷水泡一日,滚水煮两日,撤汤五次。一寸之干,发开有二寸,如鲜蛏一般,才人鸡汤煨之。扬州人学之,俱不能及。

\textbf{鲜蛏} 烹蛏法与蛼螯同。单炒亦可。何春巢家蛏汤豆腐之炒,竟成绝品。

\textbf{水鸡}
水鸡去身用腿,先用油灼之,加秋油、甜酒、瓜、姜起锅。或拆肉炒之,味与鸡相似。

\textbf{熏蛋} 将鸡蛋加作料煨好,微微熏干,切片放盘中,可以佐膳。

\textbf{茶叶蛋}
鸡蛋百个,用盐一两,粗茶叶煮两枝线香为度。如蛋五十个,只用五钱盐,照数加减。可作点心。

\hypertarget{header-n27}{%
\subsection{杂素菜单}\label{header-n27}}

菜有荤素,犹衣有表里也。富贵之人嗜素甚于嗜荤。作《素菜单》。

\textbf{蒋侍郎豆腐}
豆腐两面去皮,每块切成十六片,晾干用猪油熬清烟起才下豆腐,略洒盐花一撮,翻身后,用好甜酒一茶杯,大虾米一百二十个;如无大虾米,用小虾米三百个;先将虾米滚泡一个时辰,秋油一小杯,再滚一回,加糖一撮,再滚一回,用细葱半寸许长,一百二十段,缓缓起锅。

\textbf{杨中丞豆腐}
用嫩豆腐煮去豆气,入鸡汤,同鳆鱼片滚数刻,加糟油、香蕈起锅。鸡汁须浓,鱼片要薄。

\textbf{张恺豆腐} 将虾米捣碎,入豆腐中,起油锅,加作料干炒。

\textbf{庆元豆腐} 将豆豉一茶杯,水泡烂,入豆腐同炒起锅。

\textbf{芙蓉豆腐}
用腐脑放井水泡三次,去豆气,入鸡汤中滚,起锅时加紫菜、虾肉。

\textbf{王太守八宝豆腐}
用嫩片切粉碎,加香蕈屑、蘑菇屑、松子仁屑、瓜子仁屑、鸡屑、火腿屑,同入浓鸡汁中,炒滚起锅。用腐脑亦可。用瓢不用箸。孟亭太守云:``此圣祖师赐徐健庵尚书方也。尚书取方时,御膳房费一千两。''太守之祖楼村先生为尚书门生,故得之。

\textbf{程立万豆腐}
乾隆廿三年,同金寿门在扬州程立尤家食煎豆腐,精绝无双。其腐两面黄干,无丝毫卤汁,微有[虫车][虫敖]鲜味,然盘中并无[虫车][虫敖]及他杂物也。次日告查宣门,查曰:``我能之!我当特请。''已而,同杭堇浦同食于查家,则上箸大笑;乃纯是鸡雀脑为之,并非真豆腐,肥腻难耐矣。其费十倍于程,而味远不及也。惜其时余以妹丧急归,不及向程求方。程逾年亡。至今悔之。仍存其名,以俟再访。

\textbf{冻豆腐}
将豆腐冻一夜,切方块,滚去豆味,加鸡汤汁、火腿汁、肉汁煨之。上桌时,撤去鸡火腿之类,单留香蕈、冬笋。豆腐煨久则松,面起蜂窝,如冻腐矣。故炒腐宜嫩,煨者宜老。家致华分司,用蘑菇煮豆腐,虽夏月亦照冻腐之法,甚佳。切不可加荤汤,致失清味。

\textbf{虾油豆腐}
取陈虾油,代清酱炒豆腐。须两面熯黄。油锅要热,用猪油、葱、椒。

\textbf{蓬蒿菜} 取蒿尖用油灼瘪,放鸡汤中滚之,起时加松菌百枚。

\textbf{蕨菜}
用蕨菜不可爱惜,须尽去其枝叶,单取直根,洗净煨烂,再用鸡肉汤煨。必买矮弱者才肥。

\textbf{葛仙米}
将米细检淘净,煮米烂,用鸡汤、火腿汤煨。临上时,要只见米,不见鸡肉、火腿搀和才佳。此物陶方伯家制之最精。

\textbf{石发} 制法与葛仙米同。夏日用麻油、醋、秋油拌之,亦佳。

\textbf{素烧鹅}
煮烂山药,切寸为段,腐皮包,入油煎之,加秋油、酒、糖、瓜、姜,以色红为度。

\textbf{韭}
韭,荤物也。专取韭白,加虾米炒之便佳。或用鲜蚬亦可,蚬亦可,肉亦可。

\textbf{芹}
芹,素物也,愈肥愈妙。取白根炒之,加笋,以熟为度。今人有以炒肉者,清浊不伦。不熟者,虽脆无味。或生拌野鸡,又当别论。

\textbf{豆芽}
豆芽柔脆,余颇爱之。炒须熟烂。作料之味,才能融洽。可配燕窝,以柔配柔,以白配白故也。然以极贱而陪极贵,人多嗤之。不知惟巢、由正可陪尧、舜耳。

\textbf{茭}
茭白炒肉、炒鸡俱可。切整段,酱醋炙之,尤佳。煨肉亦佳。须切片,以寸为度,初出太细者无味。

\textbf{青菜}
青菜择嫩者,笋炒之。夏日芥末拌,加微醋,可以醒胃。加火腿片,可以作汤。亦须现拨者才软。

\textbf{台菜}
炒台菜心最懦,剥去外皮,入蘑菇、新笋作汤。炒食加虾肉,亦佳。

\textbf{白菜} 白菜炒食,或笋煨亦可。火腿片煨、鸡汤煨俱可。

\textbf{黄芽菜}
此菜以北方来者为佳。或用醋搂,或加虾米煨之,一熟便吃,迟则色、味俱变。

\textbf{瓢儿菜}
炒瓢菜心,以干鲜无汤为贵。雪压后更软。王孟亭太守家制之最精。不加别物,宜用荤油。

\textbf{波菜}
波菜肥嫩,加酱水豆腐煮之。杭人名``金镶白玉板''是也。如此种菜虽瘦而肥,可不必再加笋尖、香蕈。

\textbf{蘑菇}
蘑菇不止作汤。炒食域佳。但口蘑最易藏沙,更易受霉,须藏之得法,制之得宜。鸡腿蘑便易收拾,亦复讨好。

\textbf{松菌}
松菌加口蘑炒最佳。或单用秋油泡食,亦妙。惟不便久留耳,置各菜中,俱能助鲜,可入燕窝作底垫,以其嫩也。

\textbf{面筋三法}
一法面筋入油锅炙枯,再用鸡汤、蘑菇清煨。一法不炙,用水泡,切条入浓鸡汁炒之,加冬笋、天花。章淮树观察家制之最精。上盘时宜毛撕,不宜光切。加虾米泡汁,甜酱炒之,甚佳。

\textbf{茄二法}
吴小谷广文家,将整茄子削皮,滚水泡去苦汁,猪油炙之。炙时须待泡水干后,用甜酱水干煨,甚佳。卢八太爷家,切茄作小块,不去皮,入油灼微黄,加秋油炮炒,亦佳。是二法者,俱学之而未尽其妙,惟蒸烂划开,用麻油、米醋拌,则夏间亦颇可食。或煨干作脯,置盘中。

\textbf{苋羹} 苋须细摘嫩尖,干炒,加虾米或虾仁,更佳。不可见汤。

\textbf{芋羹}
芋性柔腻,入荤入素俱可。或切碎作鸭羹,或煨肉,或同豆腐加酱水煨。徐兆璜明府家,选小芋子,入嫩鸡煨汤,炒极!惜其制法未传。大抵只用作料,不用水。

\textbf{豆腐皮}
将腐皮泡软,加秋油中、醋、虾米拌之,宜于夏日。蒋侍郎家入海参用,颇妙。加紫菜、虾肉作汤,亦相宜。或用蘑菇、笋煨清汤,亦佳。以烂为度。芜湖敬和尚,将腐皮卷筒切段,油中微炙,入蘑菇煨烂,极佳。不可加鸡汤。

\textbf{扁豆}
现采扁豆,用肉,汤炒之,去肉存豆。单炒者油重为佳。以肥软为贵。毛糙而瘦薄者,瘠土所生,不可食。

\textbf{瓠子、王瓜} 将[鱼军]鱼切片先炒,加瓠子,同酱汁煨。王瓜亦然。

\textbf{煨木耳、香蕈}
扬州定慧庵僧,能将木耳煨二分厚,香蕈煨三分厚。先取蘑菇熬汁为卤。

\textbf{冬瓜}
冬瓜之用最多。拌燕窝、鱼肉、鳗、鳝、火腿皆可。扬州定慧庵所制尤佳。红如血珀,不用荤汤。

\textbf{煨鲜菱}
煨鲜菱,以鸡汤滚之。上时将汤撤去一半。池中现起者才鲜,浮水面者才嫩。加新栗、白果煨烂,尤佳。或用糖亦可。作点心亦可。

\textbf{缸豆} 缸豆炒肉,临上时,去肉存豆。以极嫩者,抽去其筋。

\textbf{煨三笋} 将天目笋、冬笋、问政笋,煨入鸡汤,号``三笋羹''。

\textbf{芋煨白菜}
芋煨极烂,入白菜心,烹之,加酱水调和,家常菜之最佳者,惟折菜须新摘肥嫩者,色青则老,摘久则枯。

\textbf{香珠豆}
毛豆至八九月间晚收者,最阔大而嫩,号``香珠豆''。煮熟以秋油、酒泡之。出壳可,带壳亦可,香软可爱。寻常之豆,不可食也。

\textbf{马兰} 马兰头菜,摘取嫩者,醋合笋拌食。油腻后食之,可以醒脾。

\textbf{杨花菜} 南京三月有杨花菜,柔脆与波菜相似,名甚雅。

\textbf{问政笋丝}
问政笋,即杭州笋也。徽州人送者,多是淡笋干,只好泡烂切丝,用鸡肉汤煨用。龚司马取秋油煮笋,烘干上桌,徽人食之惊为异味。余笑其如梦之方醒也。

\textbf{炒鸡腿蘑菇}
芜湖大庵和尚,洗净鸡腿,蘑菇去沙,加秋油、酒炒熟,盛盘宴客,甚佳。

\textbf{猪油煮萝卜}
用熟猪油炒萝卜,加虾米煨之,以极熟为度。临起加葱花,色如琥珀。

\hypertarget{header-n29}{%
\subsection{小菜单}\label{header-n29}}

小菜佐食,如府史胥徒佐六官司也。醒脾解浊,全在于斯。作《小菜单》。

\textbf{笋脯}
笋脯出处最多,以家园所烘为第一。取鲜笋加盐煮熟,上篮烘之。须昼夜环看,稍火不旺则溲矣。用清酱者,色微黑。春笋、冬笋皆可为之。

\textbf{天目笋}
天目笋多在苏州发卖。其篓中盖面者最佳,下二寸便搀入老根硬节矣。须出重价,专买其盖面者数十条,如集狐成腋之义。

\textbf{玉兰片}
以冬笋烘片,微加蜜焉。苏州孙春杨家有盐、甜二种,以盐者为佳。

\textbf{素火腿}
处州笋脯,号``素火腿'',即处片也。久之太硬,不如买毛笋自烘之为妙。

\textbf{宣城笋脯} 宣城笋尖,色黑而肥,与天目笋大同小异,极佳。

\textbf{人参笋} 制细笋如人参形,微加蜜水。扬州人重之,故价颇贵。

\textbf{笋油}
笋十斤,蒸一日一夜,穿通其节,铺板上,如作豆腐法,上加一板压而榨之,使汁水流出,加炒盐一两,便是笋油。其笋晒干仍可作脯。天台僧制以送人。

\textbf{虾油}
买虾子数斤,同秋油入锅熬之,起锅用布沥出秋油,乃将布包虾子,同放罐中盛油。

\textbf{喇虎酱} 秦椒捣烂,和甜酱蒸之,可用虾米搀人。

\textbf{熏鱼子}
熏鱼子色如琥珀,以没重为贵。出苏州孙春杨家,愈新愈妙,陈则味变而油枯。

\textbf{腌冬菜、黄芽菜}
腌冬菜、黄芽菜,淡则味鲜,咸则味恶。然欲久放,则非盐不可。常腌一大坛,三伏时开之,上半截虽臭、烂,而下次半截香美异常,色白如玉。甚矣!相士之不可但观皮毛也。

\textbf{莴苣}
食莴苣有二法:新酱者,松脆可爱。或腌之为脯,切片食甚鲜。然必以淡为贵,咸则味恶矣。

\textbf{香干菜}
春芥心风干,取梗淡腌,晒干,加酒、加糖、加秋油,拌后再加蒸之,风干入瓶。

\textbf{冬芥}
冬芥名雪里红。一法整腌,以淡为佳;一法取心风干,斩碎,腌入瓶中,熟后杂鱼羹中,极鲜。或用醋煨,入锅中作辣菜亦可同,煮鳗、煮鲫鱼最佳。

\textbf{春芥} 取芥心风干、斩碎,腌熟入瓶,号称``挪菜''。

\textbf{芥头} 芥根切片,入菜同腌,食之甚脆。或整腌晒干作脯食之尤妙。

\textbf{芝麻菜} 腌芥晒干,斩之碎极,蒸而食之,号``芝麻菜''。老人所宜。

\textbf{腐干丝} 将好腐干切丝极细,以虾子、秋油拌之。

\textbf{风瘪菜}
将冬菜取心风干,腌后榨出卤,小瓶装之,泥封其口,倒放灰上。夏食之,其色黄,其臭香。

\textbf{糟菜}
取腌过风瘪菜,以菜叶包之,每一小包,铺一面香糟,重叠放坛内。取食时,开包食之,糟不沾菜,而菜得糟味。

\textbf{酸菜}
冬菜心风干微腌,加糖、醋、芥末,带卤入罐中,微加秋油亦可。席间醉饱之余,食之醒脾解酒。

\textbf{台菜心}
取春日台菜心腌之,榨出其卤,装小瓶之中,夏日食之。风干其花,即名菜花头,可以烹肉。

\textbf{大头菜} 大头菜出南京承恩寺,愈陈愈佳。入荤菜中,最能发鲜。

\textbf{萝卜}
萝卜取肥大者,酱一二日即吃,甜脆可爱。有侯尼能制为鲞,煎片如蝴蝶,长至丈许,连翩不断,亦一奇也。承恩寺有卖者,用醋为之,以陈为妙。

\textbf{乳腐}
乳腐,以苏州温将军庙前者为佳,黑色而味鲜。有干湿二种,有虾子腐亦鲜,微嫌腥耳。广西白乳腐最佳。王库官司家制亦妙。

\textbf{酱炒三果}
核桃、杏仁去皮,榛子不必去皮。先用油炮脆,再下酱,不可太焦。酱之多少,亦须相物而行。

\textbf{酱石花} 将石花洗净入酱中,临吃时再洗。一名麒麟菜。

\textbf{石花糕} 将石花熬烂作膏,仍用刀划开,色如蜜蜡。

\textbf{小松菌}
将清酱同松菌入锅滚熟,收起,加麻油入罐中,可食二日,入则味变。

\textbf{吐蛈}
吐蛈出兴化、泰兴。有生成极嫩者,用酒酿浸之,加糖则自吐其油,名为泥螺,以无泥为佳。

\textbf{海蛰}
用嫩海蛰,甜酒浸之,颇有风味。其光者名为白皮,作丝,酒醋同拌。

\textbf{虾子鱼} 子鱼出苏州。小鱼生而有子。生时烹食之,较美于鲞。

\textbf{酱姜}
生姜取嫩者微腌,先用粗酱套之,再用细酱套之,凡三套而始成。古法用蝉退一入酱,则姜久而不老。

\textbf{酱瓜}
将瓜腌后,风干入酱,如酱姜之法。不难其甜,而难其脆。杭州放鲁箴家制之最佳。据云:酱后晒干又酱,故皮薄而皱,上口脆。

\textbf{新蚕豆} 新蚕豆之嫩者,以腌芥菜炒之甚妙。随采随食方佳。

\textbf{腌蛋}
腌蛋以高邮为佳,颜色红而油多。高文端公最喜食之。席间先夹取以敬客。放盘中,总宜切开带壳,黄白兼用;不可存黄去白,使味不全,油亦走散。

\textbf{混套}
将鸡蛋外壳微敲一小洞,将清黄倒出,去黄用清,加浓鸡卤煨就者拌入,用箸打良久,使之融化,仍装入蛋壳中,上用纸封好,饭锅蒸熟,剥去外壳,仍浑然一鸡卵,此味极鲜。

\textbf{茭瓜脯} 茭瓜入酱,取起风干,切片成脯,与笋脯相似。

\textbf{牛首腐干}
豆腐干以牛首僧制者为佳。但山下卖此物者有七家惟晓堂和尚家所制方妙。

\textbf{酱王瓜} 王瓜初生时,择者腌之入酱,脆而鲜。

\hypertarget{header-n31}{%
\subsection{点心菜}\label{header-n31}}

梁昭明以点心为小食,郑傪嫂劝叔且点心,由来旧矣。作《点心单》。

\textbf{鳗面}
大鳗一条蒸烂,拆肉去骨,和入面中,入鸡汤清揉之擀成面皮,小刀划成细条,入鸡汁、火腿汁、蘑菇汁滚。

\textbf{温面}
将细面下汤沥干,放碗中,用鸡肉、香蕈浓卤,临吃,各自取瓢加上。

\textbf{鳝面} 熬鳝成卤,加面再滚。此杭州法。

\textbf{裙带面}
以小刀截面成条,微宽,则号``裙带面''。大概作面,总以汤多为佳,在碗中望不见面为妙。宁使食毕再加,以便引人入胜。此法扬州盛行,恰甚有道理。

\textbf{素面}
先一日将蘑菇蓬熬汁,定清;次日将笋熬汁,加面滚上。此法扬州定慧庵僧人制之极精,不肯传人。然其大概亦可仿求。其纯黑色的或云暗用虾汁、蘑菇原汁,只宜澄云泥沙,不重换水,则原味薄矣。

\textbf{蓑衣饼}
干面用冷水调,不可多,揉擀薄后,卷拢再擀薄了,用猪油、白糖铺匀,再卷拢擀成薄饼,用猪油熯黄。如要盐的,用葱椒盐亦可。

\textbf{虾饼} 生虾肉,葱盐、花椒、甜酒脚少许,加水和面,香油灼透。

\textbf{薄饼}
山东孔藩台家制薄饼,薄若蝉翼,大若茶盘,柔腻绝伦。家人如其法为之,卒不能及,不知何故。秦人制小锡罐,装饼三十张。每客一罐。饼小如柑。罐有盖,可以贮馅。用炒肉丝,其细如发。葱亦如之。猪羊并用,号曰``西饼''。

\textbf{面老鼠}
以热水和面,俟鸡汁滚时,以箸夹入,不分大小,加活菜心,别有风味。

\textbf{颠不棱即肉饺也}
糊面摊开,裹肉为馅蒸之。其计好处全在作馅得法,不过肉嫩去筋作料而已。余到广东,吃官司镇台颠不棱,甚佳。中用肉皮煨膏为馅,故觉软美。

\textbf{韭合} 韭菜切末拌肉,加作料,面皮包之,入油灼之。面内加酥更妙。

\textbf{糖饼(又名面衣)}
糖水溲面,起油锅令热,用箸夹入;其作成饼形者,号``软锅饼'',杭州法也。

\textbf{烧饼}
用松子、胡桃仁敲碎,加糖屑、脂油和面炙之,以两面熯黄为度,而加芝麻。叩儿会做,面罗至四五次,则白如雪矣。须用两面锅,上下放火,得奶酥更佳。

\textbf{千层馒头}
杨参戎家制馒头,其白如雪,揭之如有千层。金陵人不能也。其法扬州得半,常州、无锡亦得其半。

\textbf{面茶}
熬粗茶汁,炒面兑入,加芝麻酱亦可,加牛乳亦可,微加一撮盐。无乳则加奶酥、奶皮亦可。

\textbf{杏酪} 捶杏仁作浆,挍去渣,拌米粉,加紧糖熬之。

\textbf{粉衣} 如作面衣之法。加糖、俱可,取其便也。

\textbf{竹叶粽} 取竹叶裹白糯米煮之。尖小如初生菱角。

\textbf{萝卜汤圆}
萝卜刨丝滚熟,去臭气,微干,加葱酱拌之,放粉团中作馅,再用麻油灼之。汤滚亦可。春圃方伯家制萝卜饼,叩儿学会,可照此法作韭菜饼、野鸡饼试之。

\textbf{水粉汤圆}
用水粉和作汤圆,滑腻异常,中用松仁、核桃、猪油、糖作馅,或嫩肉去筋丝捶烂,加葱末、秋油作馅亦可。作水粉法,以糯米浸水中一日夜,带水磨之,用布盛接,布下加灰,以去其渣,取细粉晒干用。

\textbf{脂油糕}
用纯糯粉拌脂油,放盘中蒸熟,加冰糖捶碎,入粉中蒸好,用刀切开。

\textbf{雪花糕} 蒸糯饭捣烂,用芝麻屑加糖为馅,打成一饼,再切方块。

\textbf{软香糕}
软香糕,以苏州都林桥为第一。其次虎丘糕,西施家为第二。南京南门外报恩寺则第三矣。

\textbf{百果糕}
杭州北关外卖者最佳。以粉糯多松仁、胡桃而不放橙丁者为妙。其甜处非蜜非糖,可暂可久。家中不能得其法。

\textbf{栗糕}
煮栗极烂,以纯糯粉加糖为糕蒸之,上加瓜仁、松子。此重阳小食也。

\textbf{青糕、青团} 捣青草为汁,和粉作粉团,色如碧玉。

\textbf{合欢饼}
蒸糕为饭,以木印印之,如小珙璧状,入铁架熯之,微用油,方不粘架。

\textbf{鸡豆糕} 研碎鸡豆,用微粉为糕,放盘中蒸之。临食用小刀片开。

\textbf{鸡豆粥} 磨碎鸡豆为粥,鲜者最佳,陈者亦可。加山药、茯苓尤妙。

\textbf{金团}
杭州金团,凿木为桃、杏、元宝之状,和粉搦成,入木印中便成。其馅不拘荤素。

\textbf{麻团} 蒸糯米捣烂为团,用芝麻屑拌糖作馅。

\textbf{芋粉团}
磨芋粉晒干,和米粉用之。朝天宫道士制芋粉团,野鸡馅,极佳。

\textbf{熟藕}
藕须贯米加糖自煮,并汤极佳。外卖者多用灰水,味变,不可食也。余性爱食嫩藕,虽软熟而以齿决,故味在也。如老藕一煮成泥,便无味矣。

\textbf{新栗、新菱}
新出之栗,烂煮之,有松子仁香。厨人不肯煨烂,故金陵人有终身不知其味者。新菱亦然。金陵人待其老方食故也。

\textbf{莲子}
建莲虽贵,不如湖莲之易煮也。大概小熟抽心去皮,后下汤,用文火煨之,闷住合盖,不可开视,河停火。如此两炷香,则莲子熟时,不生骨矣。

\textbf{芋}
十月天晴时,取芋子、芋头,晒之极干,放草中,勿使冻伤。春间煮食,有自然之甘。俗人不知。

\textbf{萧美人点心}
仪真南门外,萧美人善制点心,凡馒头、糕、饺之类,小巧可爱,洁白如雪。

\textbf{刘方伯月饼}
用山东飞面,作酥为皮,中用松仁、核桃仁、瓜子仁为细末,微加冰糖和猪油儿馅,食之不觉甚甜,而香松柔腻,迥异寻常。

\textbf{陶方伯十景点心}
每至年节,陶方伯夫人手制点心十种,皆山东飞面所为。奇形诡状,五色纷披。食之皆甘,令人应接不暇。萨制军云:``吃孔方伯薄饼,而天下之薄饼可废;吃陶方伯十景点心,而天下之点心可废。''自陶方伯亡,而此点心亦成《广陵散》矣。呜呼!

\textbf{杨中丞西洋饼}
用鸡蛋清和飞面作稠水,放碗中。打铜夹剪一把,头上作饼形,如蝶大,上下两面,铜合缝处不到一分。生烈火烘铜夹,撩稠水,一糊一夹一熯,顷刻成饼。白如雪,明如绵纸,微加冰糖、松仁屑子。

\textbf{白云片}
南殊锅巴,薄如绵纸,以油炙之,微加白糖,上口极脆。金陵人制之最精,号``白云片''。

\textbf{风枵}
以白粉浸透,制小片入猪油灼之,起锅时加糖糁之,色白如霜,上口而化。杭人号曰``风枵''。

\textbf{三层玉带糕}
以纯糯粉作糕,分作三层;一层粉,一层猪油白糖,夹好蒸之,蒸熟切开。苏州人法也。

\textbf{运司糕}
卢雅雨作运司,年已老矣。扬州店中作糕献之,大加称赏。从此遂有``运司糕''之名。色白如雪,点胭脂,红如桃花。微糖作馅,淡而弥旨。以运司衙门前店作为佳。他店粉粗色劣。

\textbf{沙糕} 糯粉蒸糕,中夹芝麻、糖屑。

\textbf{小馒头、小馄饨}
作馒头如胡桃大,就蒸笼食之。每箸可夹一双。扬州物也。扬州发酵最佳。手捺之不盈半寸,放松仍隆然而高。小馄饨小如龙眼,用鸡汤下之。

\textbf{雪蒸糕法}
每磨细粉,用糯米二分,粳米八分为则,一拌粉,将置盘中,用凉水细细洒之,以捏则如团、撒则如砂为度。将粗麻筛筛出,其剩下块搓碎,仍于筛上尽出之,前后和匀,使干湿不偏枯,以巾覆之,勿令风干日燥,听用。(水中酌加上洋糖则更有味,与市中枕儿糕法同。)一锡圈及锡钱,俱宜洗剔极净,临时略将香油和水,布蘸拭之。每一蒸后,必一洗一拭。一锡圈内,将锡钱置妥,先松装粉一小半,将果馅轻置当中,后将粉松装满圈,轻轻攩平,套汤瓶上盖之,视盖口气直冲为度。取出覆之,先去圈,后去钱,饰以胭脂,两圈更递为用。一汤瓶宜洗净,置汤分寸以及肩为度。然多滚则汤易涸,宜留心看视,备热水频添。

\textbf{作酥饼法}
冷定脂油一碗,开水一碗,先将油同水搅匀,入生面,尽揉要软,如擀饼一样,外用蒸熟面入脂油,合作一处,不要硬了。然后将生面做团子,如核桃大,将熟面亦作团子,略小一晕,再将熟面团子包在生面团子中,擀成长饼,长可八寸,宽二三寸许,然后折叠如碗样,包上穰子。

\textbf{天然饼}
泾阳张荷塘明府家制天然饼,用上白飞面,加微糖及脂油为酥,随意搦成饼样,如碗大,不拘方圆,厚二分许。用洁净小鹅子石衬而熯之,随其自为凹凸,色半黄便起,松美异常。或用盐亦可。

\textbf{花边月饼}
明府家制花边月饼,不在山东刘方伯之下。余常以轿迎其女厨来园制造,看用飞面拌生猪油子团百搦,才用枣肉嵌入为馅,裁如碗大,以手搦其四边菱花样。用火盆两个,上下覆而炙之。枣不去皮,取其鲜也;油不先熬,取其生也。含之上口而化,甘而不腻,松而不滞,其工夫全在搦中,愈多愈妙。

\textbf{制馒头法}
偶食新明府馒头,白细如雪,面有银光,以为是北面之故。龙云不然。面不分南北,只要罗得极细。罗筛至五次,则自然白细,不必北面也。惟做酵最难。请其庖人来教,学之卒不能松散。

\textbf{扬州洪府粽子}
洪府制粽,取顶高糯米,捡其完善长白者,去共半颗散碎者,淘之极熟,用大箸叶裹之,中放好火腿一大块,封锅闷煨一日一夜,柴薪不断。食之滑腻温柔,肉与米化。或云:即用火腿肥者斩碎,散置米中。

\hypertarget{header-n33}{%
\subsection{饭粥单}\label{header-n33}}

粥饭本也,余菜末也。本立而道生。作《饭粥单》。

\textbf{饭}
王莽云:``盐者,百肴之将。''余则曰:``饭者,百味之本。''《诗》称:``释之溲溲,蒸之浮浮。''是古人亦吃蒸饭。然终嫌米汁不在饭中。善煮饭者,虽煮如蒸,依旧颗粒分明,入口软糯。其诀有四:一要米好,或``香稻'',或``冬霜'',或``晚米'',或``观音籼'',或``桃花籼'',春之极熟,霉天风摊播之,不使惹霉发疹。一要善淘,淘米时不惜工夫,用手揉擦,使水从箩中淋出,竟成清水,无复米色。一要用火先武后文,闷起得宜。一要相米放水,不多不少,燥湿得宜。往往见富贵人家,讲菜不讲饭,逐末忘本,真为可笑。余不喜汤浇饭,恶失饭之本味故也。汤果佳,宁一口吃汤,一口吃饭,分前后食之,方两全其美。不得已,则用茶、用开水淘之,犹不夺饭之正味。饭之甘,在百味之上,知味者,遇好饭不必用菜。

\textbf{粥}
见水不见米,非粥也;见米不见水,非粥也。必使水米融洽,柔腻如一,而后谓之粥。尹文端公曰:``宁人等粥,毋粥等人。''此真名言,防停顿而味变汤干故也。近有为鸭粥者,入以荤腥;为八宝粥者,入以果品,俱失粥之正味。不得已,则夏用绿豆,冬用黍米,以五谷入五谷,尚属不妨。余常食于某观察家,诸菜尚可,而饭粥粗粝,勉强咽下,归而大病。尝戏语人曰:``此是五脏神暴落难。''是故自禁受不得。

\end{document}
